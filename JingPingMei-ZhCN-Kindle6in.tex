%!TEX program = xelatex
% 完整编译: xelatex -> biber/bibtex -> xelatex -> xelatex
\documentclass[lang=cn,12pt,twoside]{elegantpaper}  % uncomment for kindle

% 本书使用的中文字体
\setCJKmainfont[BoldFont={Source Han Serif SC Bold}, ItalicFont={LiSu}]{SimHei}
%\setCJKmainfont[BoldFont={Source Han Serif SC Bold}, ItalicFont={LiSu}]{Source Han Serif SC SemiBold}
\setCJKsansfont{Microsoft YaHei}
\setCJKmonofont{STFangsong}

%\setCJKfamilyfont{Askaisong}{AaKaisong2wanzijf}
%\newcommand{\kaisong}{\CJKfamily{Askaisong}}
\setCJKfamilyfont{Askaisong}{CloudLiBianGBK}
\newcommand{\kaisong}{\CJKfamily{Askaisong}}
\setCJKfamilyfont{AHeiti}{SimHei}
\newcommand{\simhei}{\CJKfamily{AHeiti}}

\title{金瓶梅【崇祯本】}
\author{{\kaisong 兰陵笑笑生} \\ 作者 \and {\kaisong 南洋一粟} \\  编辑\footnote{勘误、排版\LaTeX{}:https://enoche.github.io/}}
\date{}


\titleformat{\section}{\simhei}{\thesection}{1em}{}
\renewcommand{\thesection}{第\chinese{section}回} 



% 设置章节编号
%\renewcommand\thesection{{第\chinese{section}回、}}
%\renewcommand\thesubsection{\arabic{subsection}.}
%\renewcommand\thesubsubsection{\alph{subsubsection})}
% 设置章节编号与章节标题的距离
%\makeatletter
%\renewcommand\@seccntformat[1]{%
%	{\csname the#1\endcsname}\hspace{0.1em}
%}
%\makeatother


\usepackage{array}

\newcommand{\ccr}[1]{\makecell{{\color{#1}\rule{1cm}{1cm}}}}
\renewcommand{\abstractname}{序}

\newcommand{\KG}{\quad}%
\newcommand{\chapter}{\section}%

\newenvironment{cipaim}[1]{【#1】}
%\setlength{\footskip}{16pt}

\usepackage[
	open,
	openlevel=2,
	atend,
	numbered
]{bookmark}

\usepackage[
paperheight=145mm,   % uncomment for Kindle
paperwidth=115mm,
left=0.25in,
right=0.25in,
top=0.4in,
bottom=0.3in,
headheight=40pt,
headsep=10pt]{geometry}

\usepackage{lastpage}
\usepackage{fancyhdr}
\pagestyle{fancy}
\renewcommand{\sectionmark}[1]{\markright{\thesection ~ \ #1}}
\fancyhead{} % clear all fields
\fancyhead[LE,RO]{\footnotesize $\cdot$\enskip\thepage\enskip$\cdot$}
%\fancyhead[RE]{\yahei{\rightmark}}
%\fancyhead[LO]{\yahei 金瓶梅【崇祯本】}
\fancyhead[RE]{\footnotesize {\kaisong{\rightmark}}}
\fancyhead[LO]{\footnotesize {\kaisong 金瓶梅【崇祯本】}}
\fancyfoot{}
%\rhead{$\cdot$ \thepage $\cdot$}

\renewcommand{\headrulewidth}{1.0pt}

\begin{document}

%\vspace*{\textheight}
%\columnbreak

\maketitle
\thispagestyle{empty}

\newpage

\begin{abstract}
\thispagestyle{empty}
{\kaisong 《金瓶梅》,秽书也。袁石公亟称之,亦自寄其牢骚耳,非有取于《金瓶梅》也。然作者亦自有意,盖为世戒,非为世劝也。如诸妇多矣,而独以潘金莲、李瓶儿、春梅命名者,亦楚《梼杌》之意也。盖金莲以奸死,瓶儿以孽死,春梅以淫死,较诸妇为更惨耳。借西门庆以描画世之大净,应伯爵以描绘世之小丑,诸淫妇以描画世之丑婆、净婆,令人读之汗下。盖为世戒,非为世劝也。

余尝曰:“读《金瓶梅》而生怜悯心者,菩萨也;生畏惧心者,君子也;生欢喜心者,小人也;生效法心者,乃禽兽耳。”余友人褚孝秀偕一少年同赴歌舞之筵,衍至霸王夜宴,少年垂涎曰:“男儿何可不如此!”褚孝秀曰:“也只为这乌江设此一着耳。”同座闻之,叹为有道之言。若有人识得此意,方许他读《金瓶梅》也。不然,石公几为导淫宣欲之尤矣。奉劝世人,勿为西门之后车可也。

\bigskip\mbox{}\large\hfill 东吴弄珠客题}
\end{abstract}
\newpage
\pagestyle{plain}
\tableofcontents
\newpage
\pagestyle{fancy}

\setcounter{page}{1}
%# -*- coding:utf-8 -*-
%%%%%%%%%%%%%%%%%%%%%%%%%%%%%%%%%%%%%%%%%%%%%%%%%%%%%%%%%%%%%%%%%%%%%%%%%%%%%%%%%%%%%


\chapter{西门庆热结十弟兄\KG 武二郎冷遇亲哥嫂}

诗曰:

\[
豪华去后行人绝,箫筝不响歌喉咽。
雄剑无威光彩沉,宝琴零落金星灭。
玉阶寂寞坠秋露,月照当时歌舞处。
当时歌舞人不回,化为今日西陵灰。
\]

又诗曰:

\[
二八佳人体似酥,腰间仗剑斩愚夫。
虽然不见人头落,暗里教君骨髓枯。
\]

这一首诗,是昔年大唐国时,一个修真炼性的英雄,入圣超凡的豪杰,到后来位居紫府,名列仙班,率领上八洞群仙,救拔四部洲沉苦一位仙长,姓吕名岩,道号纯阳子祖师所作。单道世上人,营营逐逐,急急巴巴,跳不出七情六欲关头,打不破酒色财气圈子。到头来同归于尽,着甚要紧!虽是如此说,只这酒色财气四件中,唯有“财色”二者更为利害。怎见得他的利害?假如一个人到了那穷苦的田地,受尽无限凄凉,耐尽无端懊恼,晚来摸一摸米瓮,苦无隔宿之炊,早起看一看厨前,愧无半星烟火,妻子饥寒,一身冻馁,就是那粥饭尚且艰难,那讨馀钱沽酒!更有一种可恨处,亲朋白眼,面目寒酸,便是凌云志气,分外消磨,怎能勾与人争气!正是:

\[
一朝马死黄金尽,亲者如同陌路人。
\]

到得那有钱时节,挥金买笑,一掷巨万。思饮酒真个琼浆玉液,不数那琥珀杯流;要斗气钱可通神,果然是颐指气使。趋炎的压脊挨肩,附势的吮痈舐痔,真所谓得势叠肩而来,失势掉臂而去。古今炎冷恶态,莫有甚于此者。这两等人,岂不是受那财的利害处!如今再说那色的利害。请看如今世界,你说那坐怀不乱的柳下惠,闭门不纳的鲁男子,与那秉烛达旦的关云长,古今能有几人?至如三妻四妾,买笑追欢的,又当别论。还有那一种好色的人,见了个妇女略有几分颜色,便百计千方偷寒送暖,一到了着手时节,只图那一瞬欢娱,也全不顾亲戚的名分,也不想朋友的交情。起初时不知用了多少滥钱,费了几遭酒食。正是:

\[
三杯花作合,两盏色媒人。
\]
到后来情浓事露,甚而斗狠杀伤,性命不保,妻孥难顾,事业成灰。就如那石季伦泼天豪富,为绿珠命丧囹圄;楚霸王气概拔山,因虞姬头悬垓下。真说谓:

\[
“生我之门死我户,看得破时忍不过”。
\]
这样人岂不是受那色的利害处!

说便如此说,这“财色”二字,从来只没有看得破的。若有那看得破的,便见得堆金积玉,是棺材内带不去的瓦砾泥沙;贯朽粟红,是皮囊内装不尽的臭淤粪土。高堂广厦,玉宇琼楼,是坟山上起不得的享堂;锦衣绣袄,狐服貂裘,是骷髅上裹不了的败絮。即如那妖姬艳女,献媚工妍,看得破的,却如交锋阵上将军叱咤献威风;朱唇皓齿,掩袖回眸,懂得来时,便是阎罗殿前鬼判夜叉增恶态。罗袜一弯,金莲三寸,是砌坟时破土的锹锄;枕上绸缪,被中恩爱,是五殿下油锅中生活。只有那《金刚经》上两句说得好,他说道:“如梦幻泡影,如电复如露。”见得人生在世,一件也少不得,到了那结束时,一件也用不着。随着你举鼎荡舟的神力,到头来少不得骨软筋麻;由着你铜山金谷的奢华,正好时却又要冰消雪散。假饶倾闭月羞花的容貌,一到了垂眉落眼,人皆掩鼻而过之;比如你陆贾隋何的机锋,若遇着齿冷唇寒,吾未如之何也已。到不如削去六根清净,披上一领袈裟,参透了空色世界,打磨穿生灭机关,直超无上乘,不落是非窠,倒得个清闭自在,不向火坑中翻筋斗也。正是:

\[
三寸气在千般用,一日无常万事休。
\]

说话的为何说此一段酒色财气的缘故?只为当时有一个人家,先前恁地富贵,到后来煞甚凄凉,权谋术智,一毫也用不着,亲友兄弟,一个也靠不着,享不过几年的荣华,倒做了许多的话靶。内中又有几个斗宠争强,迎奸卖俏的,起先好不妖娆妩媚,到后来也免不得尸横灯影,血染空房。正是:

\[
善有善报,恶有恶报;天网恢恢,疏而不漏。
\]

话说大宋徽宗皇帝政和年间,山东省东平府清河县中,有一个风流子弟,生得状貌魁梧,性情潇洒,饶有几贯家资,年纪二十六七。这人复姓西门,单讳一个庆字。他父亲西门达,原走川广贩药材,就在这清河县前开着一个大大的生药铺。现住着门面五间到底七进的房子。家中呼奴使婢,骡马成群,虽算不得十分富贵,却也是清河县中一个殷实的人家。只为这西门达员外夫妇去世的早,单生这个儿子却又百般爱惜,听其所为,所以这人不甚读书,终日闲游浪荡。一自父母亡后,专一在外眠花宿柳,惹草招风,学得些好拳棒,又会赌博,双陆象棋,抹牌道字,无不通晓。结识的朋友,也都是些帮闲抹嘴,不守本分的人。第一个最相契的,姓应名伯爵,表字光侯,原是开绸缎铺应员外的第二个儿子,落了本钱,跌落下来,专在本司三院帮嫖贴食,因此人都起他一个浑名叫做应花子。又会一腿好气毬,双陆棋子,件件皆通。第二个姓谢名希大,字子纯,乃清河卫千户官儿应袭子孙,自幼父母双亡,游手好闲,把前程丢了,亦是帮闲勤儿,会一手好琵琶。自这两个与西门庆甚合得来。其余还有几个,都是些破落户,没名器的。一个叫做祝实念,表字贡诚。一个叫做孙天化,表字伯修,绰号孙寡嘴。一个叫做吴典恩,乃是本县阴阳生,因事革退,专一在县前与官吏保债,以此与西门庆往来。还有一个云参将的兄弟叫做云理守,字非去。一个叫做常峙节,表字坚初。一个叫做卜志道。一个叫做白赉光,表字光汤。说这白赉光,众人中也有道他名字取的不好听的,他却自己解说道:“不然我也改了,只为当初取名的时节,原是一个门馆先生,说我姓白,当初有一个什么故事,是白鱼跃入武王舟。又说有两句书是‘周有大赉,于汤有光’,取这个意思,所以表字就叫做光汤。我因他有这段故事,也便不改了。”说这一干共十数人,见西门庆手里有钱,又撒漫肯使,所以都乱撮哄着他耍钱饮酒,嫖赌齐行。正是:

\[
把盏衔杯意气深,兄兄弟弟抑何亲。
一朝平地风波起,此际相交才见心。
\]

说话的,这等一个人家,生出这等一个不肖的儿子,又搭了这等一班无益有损的朋友,随你怎的豪富也要穷了,还有甚长进的日子!却有一个缘故,只为这西门庆生来秉性刚强,作事机深诡谲,又放官吏债,就是那朝中高、杨、童、蔡四大奸臣,他也有门路与他浸润。所以专在县里管些公事,与人把搅说事过钱,因此满县人都惧怕他。因他排行第一,人都叫他是西门大官人。这西门大官人先头浑家陈氏早逝,身边只生得一个女儿,叫做西门大姐,就许与东京八十万禁军杨提督的亲家陈洪的儿子陈敬济为室,尚未过门。只为亡了浑家,无人管理家务,新近又娶了本县清河左卫吴千户之女填房为继室。这吴氏年纪二十五六,是八月十五生的,小名叫做月姐,后来嫁到西门庆家,都顺口叫他月娘。却说这月娘秉性贤能,夫主面上百依百随。房中也有三四个丫鬟妇女,都是西门庆收用过的。又尝与勾栏内李娇儿打热,也娶在家里做了第二房娘子。南街又占着窠子卓二姐,名卓丢儿,包了些时,也娶来家做了第三房。只为卓二姐身子瘦怯,时常三病四痛,他却又去飘风戏月,调弄人家妇女。正是:

\[
东家歌笑醉红颜,又向西邻开玳宴。
几日碧桃花下卧,牡丹开处总堪怜。
\]

话说西门庆一日在家闲坐,对吴月娘说道:“如今是九月廿五日了,出月初三日,却是我兄弟们的会期。到那日也少不的要整两席齐整的酒席,叫两个唱的姐儿,自恁在咱家与兄弟们好生玩耍一日。你与我料理料理。”吴月娘便道:“你也便别要说起这干人,那一个是那有良心和行货!无过每日来勾使的游魂撞尸。我看你自搭了这起人,几时曾有个家哩!现今卓二姐自恁不好,我劝你把那酒也少要吃了。”西门庆道:“你别的话倒也中听。今日这些说话,我却有些不耐烦听他。依你说,这些兄弟们没有好人,使着他,没有一个不依顺的,做事又十分停当,就是那谢子纯这个人,也不失为个伶俐能事的好人。咱如今是这等计较罢,只管恁会来会去,终不着个切实。咱不如到了会期,都结拜了兄弟罢,明日也有个靠傍些。”吴月娘接过来道:“结拜兄弟也好。只怕后日还是别个靠你的多哩。若要你去靠人,提傀儡儿上戏场——还少一口气儿哩。”西门庆笑道:“自恁长把人靠得着,却不更好了。咱只等应二哥来,与他说这话罢。”

正说着话,只见一个小厮儿,生得眉清目秀,伶俐乖觉,原是西门庆贴身伏侍的,唤名玳安儿,走到面前来说:“应二叔和谢大叔在外见爹说话哩。”西门庆道:“我正说他,他却两个就来了。”一面走到厅上来,只见应伯爵头上戴一顶新盔的玄罗帽儿,身上穿一件半新不旧的天青夹绉纱褶子,却下丝鞋净袜,坐在上首。下首坐的,便是姓谢的谢希大。见西门庆出来,一齐立起身来,边忙作揖道:“哥在家,连日少看。”西门庆让他坐下,一面唤茶来吃,说道:“你们好人儿,这几日我心里不耐烦,不出来走跳,你们通不来傍个影儿。”伯爵向希大道:“何如?我说哥哥要说哩。”因对西门庆道:“哥,你怪的是。连咱自也不知道成日忙些什么!自咱们这两只脚,还赶不上一张嘴哩。”西门庆因问道:“你这两日在那里来?”伯爵道:“昨日在院中李家瞧了个孩子儿,就是哥这边二嫂子的侄女儿桂卿的妹子,叫做桂姐儿。几时儿不见他,就出落的好不标致了。到明日成人的时候,还不知怎的样好哩!昨日他妈再三向我说:‘二爹,千万寻个好子弟梳笼他。’敢怕明日还是哥的货儿哩。”西门庆道:“有这等事!等咱空闲了去瞧瞧。”谢希大接过来道:“哥不信,委的生得十分颜色。”西门庆道:“昨日便在他家,前几日却在那里去来?”伯爵道:“便是前日卜志道兄弟死了,咱在他家帮着乱了几日,发送他出门。他嫂子再三向我说,叫我拜上哥,承哥这里送了香楮奠礼去,因他没有宽转地方儿,晚夕又没甚好酒席,不好请哥坐的,甚是过不意去。”西门庆道:“便是我闻得他不好得没多日子,就这等死了。我前日承他送我一把真金川扇儿,我正要拿甚答谢答谢,不想他又作了故人!”

谢希大便叹了一口气道:“咱会中兄弟十人,却又少他一个了。”因向伯爵说:“出月初三日,又是会期,咱每少不得又要烦大官人这里破费,兄弟们顽耍一日哩。”西门庆便道:“正是,我刚才正对房下说来,咱兄弟们似这等会来会去,无过只是吃酒顽耍,不着一个切实,倒不如寻一个寺院里,写上一个疏头,结拜做了兄弟,到后日彼此扶持,有个傍靠。到那日,咱少不得要破些银子,买办三牲,众兄弟也便随多少各出些分资。不是我科派你们,这结拜的事,各人出些,也见些情分。”伯爵连忙道:“哥说的是。婆儿烧香当不的老子念佛,各自要尽自的心。只是俺众人们,老鼠尾巴生疮儿——有脓也不多。”西门庆笑道:“怪狗才,谁要你多来!你说这话。”谢希大道:“结拜须得十个方好。如今卜志道兄弟没了,却教谁补?”西门庆沉吟了一回,说道:“咱这间壁花二哥,原是花太监侄儿,手里肯使一股滥钱,常在院中走动。他家后边院子与咱家只隔着一层壁儿,与我甚说得来,咱不如叫小厮邀他邀去。”应伯爵拍着手道:“敢就是在院中包着吴银儿的花子虚么?”西门庆道:“正是他!”伯爵笑道:“哥,快叫那个大官儿邀他去。与他往来了,咱到日后,敢又有一个酒碗儿。”西门庆笑道:“傻花子,你敢害馋痨痞哩,说着的是吃。”大家笑了一回。西门庆旋叫过玳安儿来说:“你到间壁花家去,对你花二爹说,如此这般:‘俺爹到了出月初三日,要结拜十兄弟,敢叫我请二爹上会哩。’看他怎的说,你就来回我话。你二爹若不在家,就对他二娘说罢。”玳安儿应诺去了。伯爵便道:“到那日还在哥这里是,还在寺院里好?”希大道:“咱这里无过只两个寺院,僧家便是永福寺,道家便是玉皇庙。这两个去处,随分那里去罢。”西门庆道:“这结拜的事,不是僧家管的,那寺里和尚,我又不熟,倒不如玉皇庙吴道官与我相熟,他那里又宽展又幽静。”伯爵接过来道:“哥说的是,敢是永福寺和尚倒和谢家嫂子相好,故要荐与他去的。”希大笑骂道:“老花子,一件正事,说说就放出屁来了。”

正说笑间,只见玳安儿转来了,因对西门庆说道:“他二爹不在家,俺对他二娘说来。二娘听了,好不欢喜,说道:‘既是你西门爹携带你二爹做兄弟,那有个不来的。等来家我与他说,至期以定撺掇他来,多拜上爹。’又与了小的两件茶食来了。”西门庆对应、谢二人道:“自这花二哥,倒好个伶俐标致娘子儿。”说毕,又拿一盏茶吃了,二人一齐起身道:“哥,别了罢,咱好去通知众兄弟,纠他分资来。哥这里先去与吴道官说声。”西门庆道:“我知道了,我也不留你罢。”于是一齐送出大门来。应伯爵走了几步,回转来道:“那日可要叫唱的?”西门庆道:“这也罢了,弟兄们说说笑笑,到有趣些。”说毕,伯爵举手,和希大一路去了。

话休饶舌,捻指过了四五日,却是十月初一日。西门庆早起,刚在月娘房里坐的,只见一个才留头的小厮儿,手里拿着个描金退光拜匣,走将进来,向西门庆磕了一个头儿,立起来站在傍边说道:“俺是花家,俺爹多拜上西门爹。那日西门爹这边叫大官儿请俺爹去,俺爹有事出门了,不曾当面领教的。闻得爹这边是初三日上会,俺爹特使小的先送这些分资来,说爹这边胡乱先用着,等明日爹这里用过多少派开,该俺爹多少,再补过来便了。”西门庆拿起封袋一看,签上写着“分资一两”,便道:“多了,不消补的。到后日叫爹莫往那去,起早就要同众爹上庙去。”那小厮儿应道:“小的知道。”刚待转身,被吴月娘唤住,叫大丫头玉箫在食箩里拣了两件蒸酥果馅儿与他。因说道:“这是与你当茶的。你到家拜上你家娘,你说西门大娘说,迟几日还要请娘过去坐半日儿哩。”那小厮接了,又磕了一个头儿,应着去了。

西门庆才打发花家小厮出门,只见应伯爵家应宝夹着个拜匣,玳安儿引他进来见了,磕了头,说道:“俺爹纠了众爹们分资,叫小的送来,爹请收了。”西门庆取出来看,共总八封,也不拆看,都交与月娘,道:“你收了,到明日上庙,好凑着买东西。”说毕,打发应宝去了。立起身到那边看卓二姐。刚走到坐下,只见玉箫走来,说道:“娘请爹说话哩。”西门庆道:“怎的起先不说来?”随即又到上房,看见月娘摊着些纸包在面前,指着笑道:“你看这些分子,止有应二的是一钱二分八成银子,其余也有三分的,也有五分的,都是些红的黄的,倒象金子一般。咱家也曾没见这银子来,收他的也污个名,不如掠还他罢。”西门庆道:“你也耐烦,丢着罢,咱多的也包补,在乎这些!”说着一直往前去了。

到了次日初二日,西门庆称出四两银子,叫家人来兴儿买了一口猪、一口羊、五六坛金华酒和香烛纸札、鸡鸭案酒之物,又封了五钱银子,旋叫了大家人来保和玳安儿、来兴三个:“送到玉皇庙去,对你吴师父说:‘俺爹明日结拜兄弟,要劳师父做纸疏辞,晚夕就在师父这里散福。烦师父与俺爹预备预备,俺爹明早便来。’”只见玳安儿去了一会,来回说:“已送去了,吴师父说知道了。”

须臾,过了初二,次日初三早,西门庆起来梳洗毕,叫玳安儿:“你去请花二爹,到咱这里吃早饭,一同好上庙去。一发到应二叔家,叫他催催众人。”玳安应诺去,刚请花子虚到来,只见应伯爵和一班兄弟也来了,却正是前头所说的这几个人。为头的便是应伯爵,谢希大、孙天化、祝念实、吴典恩、云理守、常峙节、白赉光,连西门庆、花子虚共成十个。进门来一齐箩圈作了一个揖。伯爵道:“咱时候好去了。”西门庆道:“也等吃了早饭着。”便叫:“拿茶来。”一面叫:“看菜儿。”须臾,吃毕早饭,西门庆换了一身衣服,打选衣帽光鲜,一齐径往玉皇庙来。不到数里之遥,早望见那座庙门,造得甚是雄峻。但见:

\[
殿宇嵯峨,宫墙高耸。正面前起着一座墙门八字,一带都粉赭色红泥;进里边列着三条甬道川纹,四方都砌水痕白石。正殿上金碧辉煌,两廊下檐阿峻峭。三清圣祖庄严宝相列中央,太上老君背倚青牛居后殿。进入第二重殿后,转过一重侧门,却是吴道官的道院。进的门来,两下都是些瑶草琪花,苍松翠竹。
\]
西门庆抬头一看,只见两边门楹上贴着一副对联道:
\[
洞府无穷岁月,壶天别有乾坤。
\]

上面三间敞厅,却是吴道官朝夕做作功课的所在。当日铺设甚是齐整,上面挂的是昊天金阙玉皇上帝,两边列着的紫府星官,侧首挂着便是马、赵、温、关四大元帅。当下吴道官却又在经堂外躬身迎接。西门庆一起人进入里边,献茶已罢,众人都起身,四围观看。白赉光携着常峙节手儿,从左边看将过来,一到马元帅面前,见这元帅威风凛凛,相貌堂堂,面上画着三只眼睛,便叫常峙节道:“哥,这却是怎的说?如今世界,开只眼闭只眼儿便好,还经得多出只眼睛看人破绽哩!”应伯爵听见,走过来道:“呆兄弟,他多只眼儿看你倒不好么?”众人笑了。常峙节便指着下首温元帅道:“二哥,这个通身蓝的,却也古怪,敢怕是卢杞的祖宗。”伯爵笑着猛叫道:“吴先生你过来,我与你说个笑话儿。”那吴道官真个走过来听他。伯爵道:“一个道家死去,见了阎王,阎王问道:‘你是什么人?’道者说:‘是道士。’阎王叫判官查他,果系道士,且无罪孽。这等放他还魂。只见道士转来,路上遇着一个染房中的博士,原认得的,那博士问道:‘师父,怎生得转来?’道者说:‘我是道士,所以放我转来。’那博士记了,见阎王时也说是道士。那阎王叫查他身上,只见伸出两只手来是蓝的,问其何故。那博士打着宣科的声音道:‘曾与温元帅搔胞。’”说的众人大笑。一面又转过右首来,见下首供着个红脸的却是关帝。上首又是一个黑面的是赵元坛元帅,身边画着一个大老虎。白赉光指着道:“哥,你看这老虎,难道是吃素的,随着人不妨事么?”伯爵笑道:“你不知,这老虎是他一个亲随的伴当儿哩。”谢希大听得走过来,伸出舌头道:“这等一个伴当随着,我一刻也成不的。我不怕他要吃我么?”伯爵笑着向西门庆道:“这等亏他怎地过来!”西门庆道:“却怎的说?”伯爵道:“子纯一个要吃他的伴当随不的,似我们这等七八个要吃你的随你,却不吓死了你罢了。”说着,一齐正大笑时,吴道官走过来,说道:“官人们讲这老虎,只俺这清河县,这两日好不受这老虎的亏!往来的人也不知吃了多少,就是猎户,也害死了十来人。”西门庆问道:“是怎的来?”吴道官道:“官人们还不知道。不然我也不晓的,只因日前一个小徒,到沧州横海郡柴大官人那里去化些钱粮,整整住了五七日,才得过来。俺这清河县近着沧州路上,有一条景阳冈,冈上新近出了一个吊睛白额老虎,时常出来吃人。客商过往,好生难走,必须要成群结伙而过。如今县里现出着五十两赏钱,要拿他,白拿不得。可怜这些猎户,不知吃了多少限棒哩!”白赉光跳起来道:“咱今日结拜了,明日就去拿他,也得些银子使。”西门庆道:“你性命不值钱么?”白赉光笑道:“有了银子,要性命怎的!”众人齐笑起来。应伯爵道:“我再说个笑话你们听:一个人被虎衔了,他儿子要救他,拿刀去杀那虎。这人在虎口里叫道:‘儿子,你省可而的砍,怕砍坏了虎皮。’”说着众人哈哈大笑。

只见吴道官打点牲礼停当,来说道:“官人们烧纸罢。”一面取出疏纸来,说:“疏已写了,只是那位居长?那位居次?排列了,好等小道书写尊讳。”众人一齐道:“这自然是西门大官人居长。”西门庆道:“这还是叙齿,应二哥大如我,是应二哥居长。”伯爵伸着舌头道:“爷,可不折杀小人罢了!如今年时,只好叙些财势,那里好叙齿!若叙齿,这还有大如我的哩。且是我做大哥,有两件不妥:第一不如大官人有威有德,众兄弟都服你;第二我原叫做应二哥,如今居长,却又要叫应大哥,倘或有两个人来,一个叫‘应二哥’,一个叫‘应大哥’,我还是应‘应二哥’,应‘应大哥’呢?”西门庆笑道:“你这搊断肠子的,单有这些闲说的!”谢希大道:“哥,休推了。”西门庆再三谦让,被花子虚、应伯爵等一干人逼勒不过,只得做了大哥。第二便是应伯爵,第三谢希大,第四让花子虚有钱做了四哥。其余挨次排列。吴道官写完疏纸,于是点起香烛,众人依次排列。吴道官伸开疏纸朗声读道:

\[
维大宋国山东东平府清河县信士西门庆、应伯爵、谢希大、花子虚、孙天化、祝念实、云理守、吴典恩、常峙节、白赉光等,是日沐手焚香请旨。伏为桃园义重,众心仰慕而敢效其风;管鲍情深,各姓追维而欲同其志。况四海皆可兄弟,岂异姓不如骨肉?是以涓今政和年月日,营备猪羊牲礼,鸾驭金资,瑞叩斋坛,虔诚请祷,拜投昊天金阙玉皇上帝,五方值日功曹,本县城隍社令,过往一切神衹,仗此真香,普同鉴察。伏念庆等生虽异日,死冀同时,期盟言之永固;安乐与共,颠沛相扶,思缔结以常新。必富贵常念贫穷,乃始终有所依倚。情共日往以月来,谊若天高而地厚。伏愿自盟以后,相好无尤,更祈人人增有永之年,户户庆无疆之福。凡在时中,全叨覆庇,谨疏。\named{政和年月日文疏}
\]

吴道官读毕,众人拜神已罢,依次又在神前交拜了八拜。然后送神,焚化钱纸,收下福礼去。不一时,吴道官又早叫人把猪羊卸开,鸡鱼果品之类整理停当,俱是大碗大盘摆下两桌,西门庆居于首席,其余依次而坐,吴道官侧席相陪。须臾,酒过数巡,众人猜枚行令,耍笑哄堂,不必细说。正是:

\[
才见扶桑日出,又看曦驭衔山。
醉后倩人扶去,树梢新月弯弯。
\]

饮酒热闹间,只见玳安儿来附西门庆耳边说道:“娘叫小的接爹来了,说三娘今日发昏哩,请爹早些家去。”西门庆随即立起来说道:“不是我摇席破座,委的我第三个小妾十分病重,咱先去休。”只见花子虚道:“咱与哥同路,咱两个一搭儿去罢。”伯爵道:“你两个财主的都去了,丢下俺们怎的!花二哥你再坐回去。”西门庆道:“他家无人,俺两个一搭里去的是,省和他嫂子疑心。”玳安儿道:“小的来时,二娘也叫天福儿备马来了。”只见一个小厮走近前,向子虚道:“马在这里,娘请爹家去哩。”于是二人一齐起身,向吴道官致谢打搅,与伯爵等举手道:“你们自在耍耍,我们去也。”说着出门上马去了。单留下这几个嚼倒泰山不谢土的,在庙流连痛饮不题。

却表西门庆到家,与花子虚别了进来,问吴月娘:“卓二姐怎的发昏来?”月娘道:“我说一个病人在家,恐怕你搭了这起人又缠到那里去了,故此叫玳安儿恁地说。只是一日日觉得重来,你也要在家看他的是。”西门庆听了,往那边去看,连日在家守着不题。

却说光阴过隙,又早是十月初十外了。一日,西门庆正使小厮请太医诊视卓二姐病症,刚走到厅上,只见应伯爵笑嘻嘻走将进来。西门庆与他作了揖,让他坐了。伯爵道:“哥,嫂子病体如何?”西门庆道:“多分有些不起解,不知怎的好。”因问:“你们前日多咱时分才散?”伯爵道:“承吴道官再三苦留,散时也有二更多天气。咱醉的要不的,倒是哥早早来家的便益些。”西门庆因问道:“你吃了饭不曾?”伯爵不好说不曾吃,因说道:“哥,你试猜。”西门庆道:“你敢是吃了?”伯爵掩口道:“这等猜不着。”西门庆笑道:“怪狗才,不吃便说不曾吃,有这等张致的!”一面叫小厮:“看饭来,咱与二叔吃。”伯爵笑道:“不然咱也吃了来了,咱听得一件稀罕的事儿,来与哥说,要同哥去瞧瞧。”西门庆道:“甚么稀罕的?”伯爵道:“就是前日吴道官所说的景阳冈上那只大虫,昨日被一个人一顿拳头打死了。”西门庆道:“你又来胡说了,咱不信。”伯爵道:“哥,说也不信,你听着,等我细说。”于是手舞足蹈说道:“这个人有名有姓,姓武名松,排行第二。”先前怎的避难在柴大官人庄上,后来怎的害起病来,病好了又怎的要去寻他哥哥,过这景阳冈来,怎的遇了这虎,怎的怎的被他一顿拳脚打死了。一五一十说来,就象是亲见的一般,又象这只猛虎是他打的一般。说毕,西门庆摇着头儿道:“既恁的,咱与你吃了饭同去看来。”伯爵道:“哥,不吃罢,怕误过了。咱们倒不如大街上酒楼上去坐罢。”只见来兴儿来放桌儿,西门庆道:“对你娘说,叫别要看饭了,拿衣服来我穿。”

须臾,换了衣服,与伯爵手拉着手儿同步出来。路上撞着谢希大,笑道:“哥们,敢是来看打虎的么?”西门庆道:“正是。”谢希大道:“大街上好挨挤不开哩。”于是一同到临街一个大酒楼上坐下。不一时,只听得锣鸣鼓响,众人都一齐瞧看。只见一对对缨枪的猎户,摆将过来,后面便是那打死的老虎,好象锦布袋一般,四个人还抬不动。末后一匹大白马上,坐着一个壮士,就是那打虎的这个人。西门庆看了,咬着指头道:“你说这等一个人,若没有千百斤水牛般气力,怎能勾动他一动儿。”这里三个儿饮酒评品,按下不题。

单表迎来的这个壮士怎生模样?但见:

\[
雄躯凛凛,七尺以上身材;阔面棱棱,二十四五年纪。双目直竖,远望处犹如两点明星;两手握来,近觑时好似一双铁碓。脚尖飞起,深山虎豹失精魂;拳手落时,穷谷熊罴皆丧魄。头戴着一顶万字头巾,上簪两朵银花;身穿着一领血腥衲袄,披着一方红锦。
\]
这人不是别人,就是应伯爵说所阳谷县的武二郎。只为要来寻他哥子,不意中打死了这个猛虎,被知县迎请将来。众人看着他迎入县里。却说这时正值知县升堂,武松下马进去,扛着大虫在厅前。知县看了武松这般模样,心中自忖道:“不恁地,怎打得这个猛虎!”便唤武松上厅。参见毕,将打虎首尾诉说一遍。两边官吏都吓呆了。知县在厅上赐了三杯酒,将库中众土户出纳的赏钱五十两,赐与武松。武松禀道:“小人托赖相公福荫,偶然侥幸打死了这个大虫,非小人之能,如何敢受这些赏赐!众猎户因这畜生,受了相公许多责罚,何不就把赏给散与众人,也显得相公恩典。”知县道:“既是如此,任从壮士处分。”武松就把这五十两赏钱,在厅上散与众猎户傅去了。知县见他仁德忠厚,又是一条好汉,有心要抬举他,便道:“你虽是阳谷县人氏,与我这清河县只在咫尺。我今日就参你在我县里做个巡捕的都头,专在河东水西擒拿贼盗,你意下如何?”武松跪谢道:“若蒙恩相抬举,小人终身受赐。”知县随即唤押司立了文案,当日便参武松做了巡捕都头。众里长大户都来与武松作贺庆喜,连连吃了数日酒。正要回阳谷县去抓寻哥哥,不料又在清河县做了都头,却也欢喜。那时传得东平一府两县,皆知武松之名。正是:

\[
壮士英雄艺略芳,挺身直上景阳冈。
醉来打死山中虎,自此声名播四方。
\]

却说武松一日在街上闲行,只听背后一个人叫道:“兄弟,知县相公抬举你做了巡捕都头,怎不看顾我!”武松回头见了这人,不觉的:

\[
欣从额角眉边出,喜逐欢容笑口开。
\]

这人不是别人,却是武松日常间要去寻他的嫡亲哥哥武大。却说武大自从兄弟分别之后,因时遭饥馑,搬移在清河县紫石街赁房居住。人见他为人懦弱,模样猥蕤,起了他个浑名叫做三寸丁谷树皮,俗语言其身上粗糙,头脸窄狭故也。只因他这般软弱朴实,多欺侮也。这也不在话下。且说武大无甚生意,终日挑担子出去街上卖炊饼度日,不幸把浑家故了,丢下个女孩儿,年方十二岁,名唤迎儿,爷儿两个过活。那消半年光景,又消折了资本,移在大街坊张大户家临街房居住。张宅家下人见他本分,常看顾他,照顾他依旧卖些炊饼。闲时在铺中坐地,武大无不奉承。因此张宅家下人个个都欢喜,在大户面前一力与他说方便。因此大户连房钱也不问武大要。

却说这张大户有万贯家财,百间房屋,年约六旬之上,身边寸男尺女皆无。妈妈余氏,主家严厉,房中并无清秀使女。只因大户时常拍胸叹气道:“我许大年纪,又无儿女,虽有几贯家财,终何大用。”妈妈道:“既然如此说,我叫媒人替你买两个使女,早晚习学弹唱,服侍你便了。”大户听了大喜,谢了妈妈。过了几时,妈妈果然叫媒人来,与大户买了两个使女,一个叫做潘金莲,一个唤做白玉莲。玉莲年方二八,乐户人家出身,生得白净小巧。这潘金莲却是南门外潘裁的女儿,排行六姐。因他自幼生得有些姿色,缠得一双好小脚儿,所以就叫金莲。他父亲死了,做娘的度日不过,从九岁卖在王招宣府里,习学弹唱,闲常又教他读书写字。他本性机变伶俐,不过十二三,就会描眉画眼,傅粉施朱,品竹弹丝,女工针指,知书识字,梳一个缠髻儿,着一件扣身衫子,做张做致,乔模乔样。到十五岁的时节,王招宣死了,潘妈妈争将出来,三十两银子转卖于张大户家,与玉莲同时进门。大户教他习学弹唱,金莲原自会的,甚是省力。金莲学琵琶,玉莲学筝,这两个同房歇卧。主家婆余氏初时甚是抬举二人,与他金银首饰装束身子。后日不料白玉莲死了,止落下金莲一人,长成一十八岁,出落的脸衬桃花,眉弯新月。张大户每要收他,只碍主家婆厉害,不得到手。一日主家婆邻家赴席不在,大户暗把金莲唤至房中,遂收用了。正是:

\[
莫讶天台相见晚,刘郎还是老刘郎。
\]

大户自从收用金莲之后,不觉身上添了四五件病症。端的悄五件?第一腰便添疼,第二眼便添泪,第三耳便添聋,第四鼻便添涕,第五尿便添滴。自有了这几件病后,主家婆颇知其事,与大户嚷骂了数日,将金莲百般苦打。大户知道不容,却赌气倒赔了房奁,要寻嫁得一个相应的人家。大户家下人都说武大忠厚,见无妻小,又住着宅内房儿,堪可与他。这大户早晚还要看觑此女,因此不要武大一文钱,白白地嫁与他为妻。这武大自从娶了金莲,大户甚是看顾他。若武大没本钱做炊饼,大户私与他银两。武大若挑担儿出去,大户候无人,便踅入房中与金莲厮会。武大虽一时撞见,原是他的行货,不敢声言。朝来暮往,也有多时。忽一日大户得患阴寒病症,呜呼死了。主家婆察知其事,怒令家僮将金莲、武大即时赶出。武大故此遂寻了紫石街西王皇亲房子,赁内外两间居住,依旧卖炊饼。

原来这金莲自嫁武大,见他一味老实,人物猥琐,甚是憎嫌,常与他合气。报怨大户:“普天世界断生了男子,何故将我嫁与这样个货!每日牵着不走,打着倒退的,只是一味吃酒,着紧处却是锥钯也不动。奴端的悄世里悔气,却嫁了他!是好苦也!”常无人处,唱个《山坡羊》为证:

\[
想当初,姻缘错配,奴把你当男儿汉看觑。不是奴自己夸奖,他乌鸦怎配鸾凤对!奴真金子埋在土里,他是块高号铜,怎与俺金色比!他本是块顽石,有甚福抱着我羊脂玉体!好似粪土上长出灵芝。奈何,随他怎样,到底奴心不美。听知:奴是块金砖,怎比泥土基!
\]

看官听说:但凡世上妇女,若自己有几分颜色,所禀伶俐,配个好男子便罢了,若是武大这般,虽好杀也未免有几分憎嫌。自古佳人才子相配着的少,买金偏撞不着卖金的。

武大每日自挑担儿出去卖炊饼,到晚方归。那妇人每日打发武大出门,只在帘子下磕瓜子儿,一径把那一对小金莲故露出来,勾引浮浪子弟,日逐在门前弹胡博词,撒谜语,叫唱:“一块好羊肉,如何落在狗嘴里?”油似滑的言语,无般不说出来。因此武大在紫石街又住不牢,要往别处搬移,与老婆商议。妇人道:“贼馄饨不晓事的,你赁人家房住,浅房浅屋,可知有小人罗唣!不如添几两银子,看相应的,典上他两间住,却也气概些,免受人欺侮。”武大道:“我那里有钱典房?”妇人道:“呸!浊才料,你是个男子汉,倒摆布不开,常交老娘受气。没有银子,把我的钗梳凑办了去,有何难处!过后有了再治不迟。”武大听老婆这般说,当下凑了十数两银子,典得县门前楼上下两层四间房屋居住。第二层是楼,两个小小院落,甚是干净。

武大自从搬到县西街上来,照旧卖炊饼过活,不想这日撞见自己嫡亲兄弟。当日兄弟相见,心中大喜。一面邀请到家中,让至楼上坐,房里唤出金莲来,与武松相见。因说道:“前日景阳冈上打死大虫的,便是你的小叔。今新充了都头,是我一母同胞兄弟。”那妇人叉手向前,便道:“叔叔万福。”武松施礼,倒身下拜。妇人扶住武松道:“叔叔请起,折杀奴家。”武松道:“嫂嫂受礼。”两个相让了一回,都平磕了头起来。少顷,小女迎儿拿茶,二人吃了。武松见妇人十分妖娆,只把头来低着。不多时,武大安排酒饭,款待武松。

说话中间,武大下楼买酒菜去了,丢下妇人,独自在楼上陪武松坐地。看了武松身材凛凛,相貌堂堂,又想他打死了那大虫,毕竟有千百斤气力。口中不说,心下思量道:“一母所生的兄弟,怎生我家那身不满尺的丁树,三分似人七分似鬼,奴那世里遭瘟撞着他来!如今看起武松这般人壮健,何不叫他搬来我家住?想这段姻缘却在这里了。”于是一面堆下笑来,问道:“叔叔你如今在那里居住?每日饭食谁人整理?”武松道:“武二新充了都头,逐日答应上司,别处住不方便,胡乱在县前寻了个下处,每日拨两个土兵伏侍做饭。”妇人道:“叔叔何不搬来家里住?省的在县前土兵服侍做饭腌臜。一家里住,早晚要些汤水吃时,也方便些。就是奴家亲自安排与叔叔吃,也干净。”武松道:“深谢嫂嫂。”妇人又道:“莫不别处有婶婶?可请来厮会。”武松道:“武二并不曾婚娶。”妇人道:“叔叔青春多少?”武松道:“虚度二十八岁。”妇人道:“原来叔叔倒长奴三岁。叔叔今番从那里来?”武松道:“在沧州住了一年有馀,只想哥哥在旧房居住,不道移在这里。”妇人道:“一言难尽。自从嫁得你哥哥,吃他忒善了,被人欺负,才到这里来。若是叔叔这般雄壮,谁敢道个不字!”武松道:“家兄从来本分,不似武松撒泼。”妇人笑道:“怎的颠倒说!常言:人无刚强,安身不长。奴家平生性快,看不上那三打不回头,四打和身转的”武松道:“家兄不惹祸,免得嫂嫂忧心。”二人在楼上一递一句的说。有诗为证:

\[
叔嫂萍踪得偶逢,娇娆偏逞秀仪容。
私心便欲成欢会,暗把邪言钓武松。
\]

话说金莲陪着武松正在楼上说话未了,只见武大买了些肉菜果饼归家。放在厨,走上楼来,叫道:“大嫂,你且下来则个。”那妇人应道:“你看那不晓事的!叔叔在此无人陪侍,却交我撇了下去。”武松道:“嫂嫂请方便。”妇人道:“何不去间壁请王乾娘来安排?只是这般不见便。”武大便自去央了间壁王婆来。安排端正,都拿上楼来,摆在桌子上,无非是些鱼肉果菜点心之类。随即烫酒上来。武大叫妇人坐了主位,武松对席,武大打横。三人坐下,把酒来斟,武大筛酒在各人面前。那妇人拿起酒来道:“叔叔休怪,没甚管待,请杯儿水酒。”武松道:“感谢嫂嫂,休这般说。”武大只顾上下筛酒,那妇人笑容可掬,满口儿叫:“叔叔,怎的肉果儿也不拣一箸儿?”拣好的递将过来。武松是个直性的汉子,只把做亲嫂嫂相待。谁知这妇人是个使女出身,惯会小意儿。亦不想这妇人一片引人心。那妇人陪武松吃了几杯酒,一双眼只看着武松的身上。武松吃他看不过,只得倒低了头。吃了一歇,酒阑了,便起身。武大道:“二哥没事,再吃几杯儿去。”武松道:“生受,我再来望哥哥嫂嫂罢。”都送下楼来。出的门外,妇人便道:“叔叔是必上心搬来家里住,若是不搬来,俺两口儿也吃别人笑话。亲兄弟难比别人,与我们争口气,也是好处。”武松道:“既是嫂嫂厚意,今晚有行李便取来。”妇人道:“奴这里等候哩!”正是:

\[
满前野意无人识,几点碧桃春自开。
\]


\newpage
%# -*- coding:utf-8 -*-
%%%%%%%%%%%%%%%%%%%%%%%%%%%%%%%%%%%%%%%%%%%%%%%%%%%%%%%%%%%%%%%%%%%%%%%%%%%%%%%%%%%%%


\chapter{俏潘娘帘下勾情\KG 老王婆茶坊说技}


词曰:

\[
芙蓉面,冰雪肌,生来娉婷年已笄。袅袅倚门余。梅花半含蕊,似开还闭。初见帘边,羞涩还留住;再过楼头,款接多欢喜。行也宜,立也宜,坐也宜,偎傍更相宜。
\]

话说当日武松来到县前客店内,收拾行李铺盖,交土兵挑了,引到哥家。那妇人见了,强如拾得金宝一般欢喜,旋打扫一间房与武松安顿停当。武松分付土兵回去,当晚就在哥家歇宿。次日早起,妇人也慌忙起来,与他烧汤净面。武松梳洗裹帻,出门去县里画卯。妇人道:“叔叔画了卯,早些来家吃早饭,休去别处吃了。”武松应的去了。到县里画卯已毕,伺候了一早晨,回到家,那妇人又早齐齐整整安排下饭。三口儿同吃了饭,妇人双手便捧一杯茶来,递与武松。武松道:“交嫂嫂生受,武松寝食不安,明日拨个土兵来使唤。”那妇人连声叫道:“叔叔却怎生这般计较!自家骨肉,又不服事了别人。虽然有这小丫头迎儿,奴家见他拿东拿西,蹀里蹀斜,也不靠他。就是拨了土兵来,那厮上锅上灶不乾净,奴眼里也看不上这等人。”武松道:“恁的却生受嫂嫂了。”有诗为证:

\[
武松仪表岂风流,嫂嫂淫心不可收。
笼络归来家里住,相思常自看衾稠。
\]

话休絮烦。自从武松搬来哥家里住,取些银子出来与武大,买饼馓茶果,请那两边邻舍。都斗分子来与武松人情。武大又安排了回席,不在话下。过了数日,武松取出一匹彩色段子与嫂嫂做衣服。那妇人堆下笑来,便道:“叔叔如何使得!既然赐与奴家,不敢推辞。”只得接了,道个万福。自此武松只在哥家宿歇。武大依前上街挑卖炊饼。武松每日自去县里承差应事,不论归迟归早,妇人顿茶顿饭,欢天喜地伏侍武松,武松倒觉过意不去。那妇人时常把些言语来拨他,武松是个硬心的直汉。

有话即长,无话即短,不觉过了一月有余,看看十一月天气,连日朔风紧起,只见四下彤云密布,又早纷纷扬扬飞下一天瑞雪来。好大雪!怎见得?但见:

\[
万里彤雪密布,空中瑞祥飘帘。琼花片片舞前檐。剡溪当此际,濡滞子猷船。顷刻楼台都压倒,江山银色相连。飞盐撒粉漫连天。当时吕蒙正,窑内叹无钱。
\]

当日这雪下到一更时分,却早银妆世界,玉碾乾坤。次日武松去县里画卯,直到日中未归。武大被妇人早赶出去做买卖,央及间壁王婆买了些酒肉,去武松房里簇了一盆炭火。心里自想道:“我今日着实撩斗他他一撩斗,不怕他不动情。”那妇人独自冷冷清清立在帘儿下,望见武松正在雪里,踏着那乱琼碎玉归来。妇人推起帘子,迎着笑道:“叔叔寒冷?”武松道:“感谢嫂嫂挂心。”入得门来,便把毡笠儿除将下来。那妇人将手去接,武松道:“不劳嫂嫂生受。”自把雪来拂了,挂在壁子上。随即解了缠带,脱了身上鹦哥绿纻丝衲袄,入房内。那妇人便道:“奴等了一早晨,叔叔怎的不归来吃早饭?”武松道:“早间有一相识请我吃饭,却才又有作杯,我不耐烦,一直走到家来。”妇人道:“既恁的,请叔叔向火。”武松道:“正好。”便脱了油靴,换了一双袜子,穿了暖鞋,掇条凳子,自近火盆边坐地。那妇人早令迎儿把前门上了闩,后门也关了。却搬些煮熟菜蔬入房里来,摆在桌子上。武松问道:“哥哥那里去了?”妇人道:“你哥哥出去买卖未回,我和叔叔自吃三杯。”武松道:“一发等哥来家吃也不迟。”妇人道:“那里等的他!”说犹未了,只见迎儿小女早暖了一注酒来。武松道:“又教嫂嫂费心。”妇人也掇一条凳子,近火边坐了。桌上摆着杯盘,妇人拿盏酒擎在手里,看着武松道:“叔叔满饮此杯。”武松接过酒去,一饮而尽。那妇人又筛一杯酒来,说道:“天气寒冷,叔叔饮过成双的盏儿。”武松道:“嫂嫂自请。”接来又一饮而尽。武松却筛一杯酒,递与妇人。妇人接过酒来呷了,却拿注子再斟酒放在武松面前。那妇人一径将酥胸微露,云鬟半軃,脸上堆下笑来,说道:“我听得人说,叔叔在县前街上养着个唱的,有这话么?”武松道:“嫂嫂休听别人胡说,我武二从来不是这等人。”妇人道:“我不信!只怕叔叔口头不似心头。”武松道:“嫂嫂不信时,只问哥哥就是了。”妇人道:“啊呀,你休说他,那里晓得甚么?如在醉生梦死一般!他若知道时,不卖炊饼了。叔叔且请杯。”连筛了三四杯饮过。那妇人也有三杯酒落肚,哄动春心,那里按纳得住。欲心如火,只把闲话来说。武松也知了八九分,自己只把头来低了,却不来兜揽。妇人起身去烫酒。武松自在房内却拿火箸簇火。妇人良久暖了一注子酒来,到房里,一只手拿着注子,一只手便去武松肩上只一捏,说道:“叔叔只穿这些衣裳,不寒冷么?”武松已有五七分不自在,也不理他。妇人见他不应,匹手就来夺火箸,口里道:“叔叔你不会簇火,我与你拨火。只要一似火盆来热便好。”武松有八九分焦燥,只不做声。这妇人也不看武松焦燥,便丢下火箸,却筛一杯酒来,自呷了一口,剩下半盏酒,看着武松道:“你若有心,吃我这半盏儿残酒。”武松匹手夺过来,泼在地下说道:“嫂嫂不要恁的不识羞耻!”把手只一推,争些儿把妇人推了一交。武松睁起眼来说道:“武二是个顶天立地噙齿戴发的男子汉,不是那等败坏风俗伤人伦的猪狗!嫂嫂休要这般不识羞耻,为此等的勾当,倘有风吹草动,我武二眼里认的是嫂嫂,拳头却不认的是嫂嫂!”妇人吃他几句抢得通红了面皮,便叫迎儿收拾了碟盏家伙,口里说道:“我自作耍子,不直得便当真起来。好不识人敬!”收了家伙,自往厨下去了。正是:

\[
落花有意随流水,流水无情恋落花。
\]

这妇人见勾搭武松不动,反被他抢白了一场。武松自在房中气忿忿,自己寻思。天色却是申牌时分,武大挑着担儿,大雪里归来。推门进来,放下担儿,进的里间,见妇人一双眼哭的红红的,便问道:“你和谁闹来?”妇人道:“都是你这不不争气的,交外人来欺负我。”武大道:“谁敢来欺负你?”妇人道:“情知是谁?争奈武二那厮。我见他大雪里归来,好意安排些酒饭与他吃,他见前后没人,便把言语来调戏我。便是迎儿眼见,我不赖他。”武大道:“我兄弟不是这等人,从来老实。休要高声,乞邻舍听见笑话。”武大撇了妇人,便来武二房里叫道:“二哥,你不曾吃点心?我和你吃些个。”武松只不做声,寻思了半晌,一面出大门。武大叫道:“二哥,你那里去?”也不答应,一直只顾去了。武大回到房内,问妇人道:“我叫他又不应,只顾望县里那条路去了。正不知怎的了?”妇人骂道:“贼馄饨虫!有甚难见处?那厮羞了,没脸儿见你,走了出去。我猜他一定叫人来搬行李,不要在这里住。却不道你留他?”武大道:“他搬了去,须乞别人笑话。”妇人骂道:“混沌魍魉,他来调戏我,到不乞别人笑话!你要便自和他过去,我却做不的这样人!你与了我一纸休书,你自留他便了。”武大那里敢再开口。被这妇人倒数骂了一顿。正在家两口儿絮聒,只见武松引了个土兵,拿着条扁担,迳来房内收拾行李,便出门。武大走出来,叫道:“二哥,做甚么便搬了去?”武松道:“哥哥不要问,说起来装你的幌子,只由我自去便了。”武大那里再敢问备细,由武松搬了出去。那妇人在里面喃喃呐呐骂道:“却也好,只道是亲难转债,人不知道一个兄弟做了都头,怎的养活了哥嫂,却不知反来咬嚼人!正是花木瓜空好看。搬了去,倒谢天地,且得冤家离眼睛。”武大见老婆这般言语,不知怎的了,心中反是放不下。自从武松搬去县前客店宿歇,武大自依前上街卖炊饼。本待要去县前寻兄弟说话,却被这妇人千叮万嘱,分付交不要去兜揽他,因此武大不敢去寻武松。

说这武松自从搬离哥家,捻指不觉雪晴,过了十数日光景。却说本县知县自从到任以来,却得二年有余,转得许多金银,要使一心腹人送上东京亲眷处收寄,三年任满朝觐,打点上司。一来却怕路上小人,须得一个有力量的人去方好,猛可想起都头武松,须得此人方了得此事。当日就唤武松到衙内商议道:“我有个亲戚在东京城内做官,姓朱名靦,见做殿前太尉之职,要送一担礼物,捎封书去问安。只恐途中不好行,若得你去方可。你休推辞辛苦,回来我自重赏。”武松应道:“小人得蒙恩相抬举,安敢推辞!既蒙差遣,只此便去。”知县大喜,赏了武松三杯酒,十两路费。不在话下。

且说武松领了知县的言语,出的县门来,到下处,叫了土兵,却来街上买了一瓶酒并菜蔬之类,迳到武大家。武大却街上回来,见武松在门前坐地,交土兵去厨下安排。那妇人余情不断,见武松把将酒食来,心中自思:“莫不这厮思想我了?不然却又回来怎的?到日后我且慢慢问他。”妇人便上楼去重匀粉面,再整云鬟,换了些颜色衣服,来到门前迎接武松。妇人拜道:“叔叔,不知怎的错见了,好几日并不上门,叫奴心里没理会处。今日再喜得叔叔来家。没事坏钞做甚么?”武松道:“武二有句话,特来要与哥哥说知。”妇人道:“既如此,请楼上坐。”三个人来到楼上,武松让哥嫂上首坐了,他便掇杌子打横。土兵摆上酒,并嗄饭一齐拿上来。武松劝哥嫂吃。妇人便把眼来睃武松,武松只顾吃酒。酒至数巡,武松问迎儿讨副劝杯,叫土兵筛一杯酒拿在手里,看着武大道:“大哥在上,武二今日蒙知县相公差往东京干事,明日便要起程,多是两三个月,少是一月便回,有句话特来和你说。你从来为人懦弱,我不在家,恐怕外人来欺负。假如你每日卖十扇笼炊饼,你从明日为始,只做五扇笼炊饼出去,每日迟出早归,不要和人吃酒。归家便下了帘子,早闭门,省了多少是非口舌。若是有人欺负你,不要和他争执,待我回来,自和他理论。大哥你依我时,满饮此杯!”武大接了酒道:“兄弟见得是,我都依你说。”吃过了一杯,武松再斟第二盏酒,对那妇人说道:“嫂嫂是个精细的人,不必要武松多说。我的哥哥为人质朴,全靠嫂嫂做主。常言表壮不如里壮,嫂嫂把得家定,我哥哥烦恼做甚么!岂不闻古人云:篱牢犬不入。”那妇人听了这句话,一点红从耳边起,须臾紫涨了面皮,指着武大骂道:“你这个混沌东西。有甚言语在别处说,来欺负老娘!我是个不带头巾的男子汉,叮叮当当响的婆娘!拳头上也立得人,胳膊上走得马,不是那腲脓血搠不出来鳖!老娘自从嫁了武大,真个蚂蚁不敢入屋里来,甚么篱笆不牢犬儿钻得入来?你休胡言乱语,一句句都要下落!丢下一块瓦砖儿,一个个也要着地!”武松笑道:“若得嫂嫂做主,最好。只要心口相应。既然如此,我武松都记得嫂嫂说的话了,请过此杯。”那妇人一手推开酒盏,一直跑下楼来,走到在胡梯上发话道:“既是你聪明伶俐,恰不道长嫂为母。我初嫁武大时,不曾听得有甚小叔,那里走得来?是亲不是亲,便要做乔家公。自是老娘晦气了,偏撞着这许多鸟事!”一面哭下楼去了。正是:

\[
苦口良言谏劝多,金莲怀恨起风波。
自家惶愧难存坐,气杀英雄小二哥。
\]

那妇人做出许多乔张致来。武大、武松吃了几杯酒,坐不住,都下的楼来,弟兄洒泪而别。武大道:“兄弟去了,早早回来,和你相见。”武松道:“哥哥,你便不做买卖也罢,只在家里坐的。盘缠,兄弟自差人送与你。”临行,武松又分付道:“哥哥,我的言语休要忘了,在家仔细门户。”武大道:“理会得了。”武松辞了武大,回到县前下处,收拾行装并防身器械。次日领了知县礼物,金银驼垛,讨了脚程,起身上路,往东京去了,不题。

只说武大自从兄弟武松说了去,整整吃那婆娘骂了三四日。武大忍声吞气,由他自骂,只依兄弟言语,每日只做一半炊饼出去,未晚便回来。歇了担儿,便先去除了帘子,关上大门,却来屋里坐的。那妇人看了这般,心内焦燥,骂道:“不识时浊物!我倒不曾见,日头在半天里便把牢门关了,也吃邻舍家笑话,说我家怎生禁鬼。听信你兄弟说,空生着卵鸟嘴,也不怕别人笑耻!”武大道:“由他笑也罢,我兄弟说的是好话,省了多少是非。”被妇人啐在脸上道:“呸!浊东西!你是个男子汉,自不做主,却听别人调遣!”武大摇手道:“由他,我兄弟说的是金石之语。”原来武松去后,武大每日只是晏出早归,到家便关门。那妇人气生气死,和他合了几场气。落后闹惯了,自此妇人约莫武大归来时分,先自去收帘子,关上大门。武大见了,心里自也暗喜,寻思道:“恁的却不好?”有诗为证:

\[
慎事关门并早归,眼前恩爱隔崔嵬。
春心一点如丝乱,任锁牢笼总是虚。
\]

白驹过隙,日月如梭,才见梅开腊底,又早天气回阳。一日,三月春光明媚时分,金莲打扮光鲜,单等武大出门,就在门前帘下站立。约莫将及他归来时分,便下了帘子,自去房内坐的。一日也是合当有事,却有一个人从帘子下走过来。自古没巧不成话,姻缘合当凑着。妇人正手里拿着叉竿放帘子,忽被一阵风将叉竿刮倒,妇人手擎不牢,不端不正却打在那人头上。妇人便慌忙陪笑,把眼看那人,也有二十五六年纪,生得十分浮浪。头上戴着缨子帽儿,金铃珑簪儿,金井玉栏杆圈儿;长腰才,身穿绿罗褶儿;脚下细结底陈桥鞋儿,清水布袜儿;手里摇着洒金川扇儿,越显出张生般庞儿,潘安的貌儿。可意的人儿,风风流流从帘子下丢与个眼色儿。这个人被叉竿打在头上,便立住了脚,待要发作时,回过脸来看,却不想是个美貌妖娆的妇人。但见他黑鬒鬒赛鸦鸰的鬓儿,翠弯弯的新月的眉儿,香喷喷樱桃口儿,直隆隆琼瑶鼻儿,粉浓浓红艳腮儿,娇滴滴银盆脸儿,轻袅袅花朵身儿,玉纤纤葱枝手儿,一捻捻杨柳腰儿,软浓浓粉白肚儿,窄星星尖翘脚儿,肉奶奶胸儿,白生生腿儿,更有一件紧揪揪、白鲜鲜、黑茵茵,正不知是甚么东西。观不尽这妇人容貌。且看他怎生打扮?但见:

\[
头上戴着黑油油头发髢髻,一迳里踅出香云,周围小簪儿齐插。斜戴一朵并头花,排草梳儿后押。难描画,柳叶眉衬着两朵桃花。玲珑坠儿最堪夸,露来酥玉胸无价。毛青布大袖衫儿,又短衬湘裙碾绢纱。通花汗巾儿袖口儿边搭剌。香袋儿身边低挂。抹胸儿重重纽扣香喉下。往下看尖翘翘金莲小脚,云头巧缉山鸦。鞋儿白绫高底,步香尘偏衬登踏。红纱膝裤扣莺花,行坐处风吹裙裤。口儿里常喷出异香兰麝,樱桃口笑脸生花。人见了魂飞魄丧,卖弄杀俏冤家。
\]
那人一见,先自酥了半边,那怒气早已钻入爪洼国去了,变做笑吟吟脸儿。这妇人情知不是,叉手望他深深拜了一拜,说道:“奴家一时被风失手,误中官人,休怪!”那人一面把手整头巾,一面把腰曲着地还喏道:“不妨,娘子请方便。”却被这间壁住的卖茶王婆子看见。那婆子笑道:“兀的谁家大官人打这屋檐下过?打的正好!”那人笑道:“倒是我的不是,一时冲撞,娘子休怪。”妇人答道:“官人不要见责。”那人又笑着大大地唱个喏,回应道:“小人不敢。”那一双积年招花惹草,惯觑风情的贼眼,不离这妇人身上,临去也回头了七八回,方一直摇摇摆摆遮着扇儿去了。

\[
风日晴和漫出游,偶从帘下识娇羞。只因临去秋波转,惹起春心不自由。
\]
当时妇人见了那人生的风流浮浪,语言甜净,更加几分留恋:“倒不知此人姓甚名谁,何处居住。他若没我情意时,临去也不回头七八遍了。”却在帘子下眼巴巴的看不见那人,方才收了帘子,关上大门,归房去了。

看官听说,这人你道是谁?却原来正是那嘲风弄月的班头,拾翠寻香的元帅,开生药铺复姓西门单讳一个庆字的西门大官人便是。只因他第三房妾卓二姐死了,发送了当,心中不乐,出来街上行走,要寻应伯爵到那里去散心耍子。却从这武大门前经过,不想撞了这一下子在头上。却说这西门大官人自从帘子下见了那妇人一面,到家寻思道:“好一个雌儿,怎能勾得手?”猛然想起那间壁卖茶王婆子来,堪可如此如此,这般这般:“撮合得此事成,我破费几两银子谢他,也不值甚的。”于是连饭也不吃,走出街上闲游,一直迳踅入王婆茶坊里来,便去里边水帘下坐了。王婆笑道:“大官人却才唱得好个大肥喏!”西门庆道:“干娘,你且来,我问你,间壁这个雌儿是谁的娘子?”王婆道:“他是阎罗大王的妹子,五道将军的女儿,问他怎的?”西门庆道:“我和你说正话,休要取笑。”王婆道:“大官人怎的不认得?他老公便是县前卖熟食的。”西门庆道:“莫不是卖枣糕徐三的老婆?”王婆摇手道:“不是,若是他,也是一对儿。大官人再猜。”西门庆道:“敢是卖馉饳的李三娘子儿?”王婆摇手道:“不是,若是他,倒是一双。”西门庆道:“莫不是花胳膊刘小二的婆儿?”王婆大笑道:“不是,若是他时,又是一对儿。大官人再猜。”西门庆道:“干娘,我其实猜不着了。”王婆哈哈笑道:“我好交大官人得知了罢,他的盖老便是街上卖炊饼的武大郎。”西门庆听,跌脚笑道:“莫不是人叫他三寸丁谷树皮的武大么?”王婆道:“正是他。”西门庆听了,叫起苦来,说是:“好一块羊肉,怎生落在狗口里!”王婆道:“便是这般故事,自古骏马却驮痴汉走,美妻常伴拙夫眠。月下老偏这等配合。”西门庆道:“干娘,我少你多少茶果钱?”王婆道:“不多,由他,歇些时却算不妨。”西门庆又道:“你儿子王潮跟谁出去了?”王婆道:“说不的,跟了一个淮上客人,至今不归,又不知死活。”西门庆道:“却不交他跟我,那孩子倒乖觉伶俐。”王婆道:“若得大官人抬举他时,十分之好。”西门庆道:“待他归来,却再计较。”说毕,作谢起身去了。

约莫未及两个时辰,又踅将来王婆门首,帘边坐的,朝着武大门前半歇。王婆出来道:“大官人,吃个梅汤?”西门庆道:“最好多加些酸味儿。”王婆做了个梅汤,双手递与西门庆吃了。将盏子放下,西门庆道:“干娘,你这梅汤做得好,有多少在屋里?”王婆笑道:“老身做了一世媒,那讨不在屋里!”西门庆笑道:“我问你这梅汤,你却说做媒,差了多少!”王婆道:“老身只听得大官人问这媒做得好。”西门庆道:“干娘,你既是撮合山,也与我做头媒,说头好亲事,我自重重谢你。”王婆道:“看这大官人作戏!你宅上大娘子得知,老婆子这脸上怎吃得那耳刮子!”西门庆道:“我家大娘子最好性格。见今也有几个身边人在家,只是没一个中得我意的。你有这般好的,与我主张一个,便来说也不妨。若是回头人儿也好,只是要中得我意。”王婆道:“前日有一个倒好,只怕大官人不要。”西门庆道:“若是好时,与我说成了,我自重谢你。”王婆道:“生的十二分人才,只是年纪大些。”西门庆道:“自古半老佳人可共,便差一两岁也不打紧。真个多少年纪?”王婆道:“那娘子是丁亥生,属猪的,交新年却九十三岁了。”西门庆笑道:“你看这风婆子,只是扯着风脸取笑。”说毕,西门庆笑着起身去。

看看天色晚了,王婆恰才点上灯来,正要关门,只见西门庆又踅将来,迳去帘子底下凳子上坐下,朝着武大门前只顾将眼睃望。王婆道:“大官人吃个和合汤?”西门庆道:“最好!干娘放甜些。”王婆连忙取一钟来与西门庆吃了。坐到晚夕,起身道:“干娘,记了帐目,明日一发还钱。”王婆道:“由他,伏惟安置,来日再请过论。”西门庆笑了去。到家甚是寝食不安,一片心只在妇人身上。就是他大娘子月娘,见他这等失张失致的,只道为死了卓二姐的缘故,倒没做理会处。当晚无话。

次日清晨,王婆恰才开门,把眼看外时,只见西门庆又早在街前来回踅走。王婆道:“这刷子踅得紧!你看我着些甜糖抹在这厮鼻子上,交他抵不着。那厮全讨县里人便宜,且交他来老娘手里纳些贩钞,撰他几个风流钱使。”原来这开茶坊的王婆,也不是守本分的,便是积年通殷勤,做媒婆,做卖婆,做牙婆,又会收小的,也会抱腰,又善放刁,端的看不出这婆子的本事来。但见:

\[
开言欺陆贾,出口胜隋何。只凭说六国唇枪,全仗话三齐舌剑。只鸾孤凤,霎时间交仗成双;寡妇鳏男,一席话搬说摆对。解使三里门内女,遮莫九皈殿中仙。玉皇殿上侍香金童,把臂拖来;王母宫中传言玉女,拦腰抱住。略施奸计,使阿罗汉抱住比丘尼;才用机关,交李天王搂定鬼子母。甜言说诱,男如封陟也生心;软语调合,女似麻姑须乱性。藏头露尾,撺掇淑女害相思;送暖偷寒,调弄嫦娥偷汉子。
\]


这婆子正开门,在茶局子里整理茶锅,张见西门庆踅过几遍,奔入茶局子水帘下,对着武大门首,不住把眼只望帘子里瞧。王婆只推不看见,只顾在茶局子内煽火,不出来问茶。西门庆叫道:“干娘,点两杯茶来我吃。”王婆应道:“大官人来了?连日少见,且请坐。”不多时,便浓浓点两盏稠茶,放在桌子上。西门庆道:“干娘,相陪我吃了茶。”王婆哈哈笑道:“我又不是你影射的,如何陪你吃茶?”西门庆也笑了,一会便问:“干娘,间壁卖的是甚么?”王婆道:“他家卖的拖煎阿满子,干巴子肉翻包着菜肉匾食饺,窝窝蛤蜊面,热烫温和大辣酥。”西门庆笑道:“你看这风婆子,只是风。”王婆笑道:“我不风,他家自有亲老公。”西门庆道:“我和你说正话。他家如法做得好炊饼,我要问他买四五十个拿的家去。”王婆道:“若要买炊饼,少间等他街上回来买,何消上门上户!”西门庆道:“干娘说的是。”吃了茶,坐了一回,起身去了。

良久,王婆在茶局里冷眼张着,他在门前踅过东,看一看,又转西去,又复一复,一连走了七八遍。少顷,迳入茶房里来。王婆道:“大官人侥幸,好几日不见面了。”西门庆便笑将起来,去身边摸出一两一块银子,递与王婆,说道:“干娘,权且收了做茶钱。”王婆笑道:“何消得许多!”西门庆道:“多者干娘只顾收着。”婆子暗道:“来了,这刷子当败。且把银子收了,到明日与老娘做房钱。”便道:“老身看大官人象有些心事的一般。”西门庆道:“如何干娘便猜得着?”婆子道:“有甚难猜处!自古入门休问荣枯事,观着容颜便得知。老身异样跷蹊古怪的事,不知猜勾多少。”西门庆道:“我这一件心上的事,干娘若猜得着时,便输与你五两银子。”王婆笑道:“老身也不消三智五猜,只一智便猜个中节。大官人你将耳朵来:你这两日脚步儿勤,赶趁得频,一定是记挂着间壁那个人。我这猜如何?”西门庆笑将起来道:“干娘端的智赛隋何,机强陆贾。不瞒干娘说,不知怎的,吃他那日叉帘子时见了一面,恰似收了我三魂六魄的一般,日夜只是放他不下。到家茶饭懒吃,做事没入脚处。不知你会弄手段么?”王婆哈哈笑道:“老身不瞒大官人说,我家卖茶叫做鬼打更。三年前六月初三日下大雪,那一日卖了个泡茶,直到如今不发市,只靠些杂趁养口。”西门庆道:“干娘,如何叫做杂趁?”王婆笑道:“老身自从三十六岁没了老公,丢下这个小厮,没得过日子。迎头儿跟着人说媒,次后揽人家些衣服卖,又与人家抱腰收小的,闲常也会作牵头,做马百六,也会针灸看病。”西门庆听了,笑将起来:“我并不知干娘有如此手段!端的与我说这件事,我便送十两银子与你做棺材本。你好交这雌儿会我一面。”王婆便呵呵笑道:“我自说耍,官人怎便认真起来。你也!”且看下回分解。有诗为证:

\[
西门浪子意猖狂,死下功夫戏女娘。
亏杀卖茶王老母,生交巫女会襄王。 
\]
\newpage
%# -*- coding:utf-8 -*-
%%%%%%%%%%%%%%%%%%%%%%%%%%%%%%%%%%%%%%%%%%%%%%%%%%%%%%%%%%%%%%%%%%%%%%%%%%%%%%%%%%%%%


\chapter{定挨光王婆受贿\KG 设圈套浪子私挑}


诗曰:

\[
乍对不相识,徐思似有情。杯前交一面,花底恋双睛。
艖俹惊新态,含胡问旧名。影含今夜烛,心意几交横。
\]

话说西门庆央王婆,一心要会那雌儿一面,便道:“干娘,你端的与我说这件事成,我便送十两银子与你。”王婆道:“大官人,你听我说:但凡‘挨光’的两个字最难。怎的是‘挨光’?比如如今俗呼‘偷情’就是了。要五件事俱全,方才行的。第一要潘安的貌;第二要驴大行货;第三要邓通般有钱;第四要青春少小,就要绵里针一般软款忍耐;第五要闲工夫。此五件,唤做‘潘驴邓小闲’。都全了,此事便获得着。”西门庆道:“实不瞒你说,这这五件事我都有。第一件,我的貌虽比不得潘安,也充得过;第二件,我小时在三街两巷游串,也曾养得好大龟;第三,我家里也有几贯钱财,虽不及邓通,也颇得过日子;第四,我最忍耐;他便打我四百顿,休想我回他一拳;第五,我最有闲工夫,不然如何来得恁勤。干娘,你自作成,完备了时,我自重重谢你。”王婆道:“大官人,你说五件事都全,我知道还有一件事打搅,也多是成不得。”西门庆道:“且说,甚么一件事打搅?”王婆道:“大官人休怪老身直言,但凡挨光最难,十分,有使钱到九分九厘,也有难成处。我知你从来悭吝,不肯胡乱便使钱,只这件打搅。”西门庆道:“这个容易,我只听你言语便了。”王婆道:“若大官人肯使钱时,老身有一条妙计,须交大官人和这雌儿会一面。”西门庆道:“端的有甚妙计?”王婆笑道:“今日晚了,且回去,过半年三个月来商量。”西门庆央及道:“干娘,你休撒科!自作成我则个,恩有重报。”王婆笑哈哈道:“大官人却又慌了。老身这条计,虽然入不得武成王庙,端的强似孙武子教女兵,十捉八九着。今日实对你说了罢:这个雌儿来历,虽然微末出身,却倒百伶百俐,会一手好弹唱,针指女工,百家歌曲,双陆象棋,无所不知。小名叫做金莲,娘家姓潘,原是南门外潘裁的女儿,卖在张大户家学弹唱。后因大户年老,打发出来,不要武大一文钱,白白与了他为妻。这雌儿等闲不出来,老身无事常过去与他闲坐。他有事亦来请我理会,他也叫我做干娘。武大这两日出门早。大官人如干此事,便买一匹蓝绸、一匹白绸、一匹白绢,再用十两好绵,都把来与老身。老身却走过去问他借历日,央及他拣个好日期,叫个裁缝来做。他若见我这般说,拣了日期,不肯与我来做时,此事便休了;他若欢天喜地说:‘我替你做。’不要我叫裁缝,这光便有一分了。我便请得他来做,就替我缝,这光便二分了。他若来做时,午间我却安排些酒食点心请他吃。他若说不便当,定要将去家中做,此事便休了;他不言语吃了时,这光便有三分了。这一日你也莫来,直至第三日,晌午前后,你整整齐齐打扮了来,以咳嗽为号,你在门前叫道:‘怎的连日不见王干娘?我买盏茶吃。’我便出来请你入房里坐吃茶。他若见你便起身来,走了归去,难道我扯住他不成?此事便休了。他若见你入来,不动身时,这光便有四分了。坐下时,我便对雌儿说道:‘这个便是与我衣服施主的官人,亏杀他。’我便夸大官人许多好处,你便卖弄他针指。若是他不来兜揽答应时,此事便休了;他若口中答应与你说话时,这光便有五分了。我便道:‘却难为这位娘子与我作成出手做,亏杀你两施主,一个出钱,一个出力。不是老身路歧相央,难得这位娘子在这里,官人做个主人替娘子浇浇手。’你便取银子出来,央我买。若是他便走时,难道我扯住他?此事便休了。他若是不动身时,事务易成,这光便有六分了。我却拿银子,临出门时对他说:‘有劳娘子相待官人坐一坐。’他若起身走了家去,我终不成阻挡他?此事便休了。若是他不起身,又好了,这光便有七分了。待我买得东西提在桌子上,便说:‘娘子且收拾过生活去,且吃一杯儿酒,难得这官人坏钱。’他不肯和你同桌吃,去了,此事便休了。若是他不起身,此事又好了,这光便有八分了。待他吃得酒浓时,正说得入港,我便推道没了酒,再交你买,你便拿银子,又央我买酒去并果子来配酒。我把门拽上,关你两个在屋里。他若焦燥跑了归去时,此事便休了;他若由我拽上门,不焦躁时,这光便有九分,只欠一分了。只是这一分倒难。大官人你在房里,便着几句甜话儿说入去,却不可燥暴,便去动手动脚打搅了事,那时我不管你。你先把袖子向桌子上拂落一双箸下去,只推拾箸,将手去他脚上捏一捏。他若闹吵起来,我自来搭救。此事便休了,再也难成。若是他不做声时,此事十分光了。这十分光做完备,你怎的谢我?”西门庆听了大喜道:“虽然上不得凌烟阁,干娘你这条计,端的绝品好妙计!”王婆道:却不要忘了许我那十两银子。”西门庆道:“便得一片橘皮吃,切莫忘了洞庭湖。这条计,干娘几时可行?”婆道:“只今晚来有回报。我如今趁武大未归,过去问他借历日,细细说与他。你快使人送将绸绢绵子来,休要迟了。”西门庆道:“干娘,这是我的事,如何敢失信。”于是作别了王婆,离了茶肆,就去街上买了绸绢三匹并十两清水好绵。家里叫了玳安儿用毡包包了,一直送入王婆家来。王婆欢喜收下,打发小厮回去。正是:

\[
巫山云雨几时就,莫负襄王筑楚台。
\]

当下王婆收了绸绢绵子,开了后门,走过武大家来。那妇人接着,走去楼上坐的。王婆道:“娘子怎的这两日不过贫家吃茶?”那妇人道:“便是我这几日身子不快,懒走动的。”王婆道:“娘子家里有历日,借与老身看一看,要个裁衣的日子。”妇人道:“干娘裁甚衣服?”王婆道:“便是因老身十病九痛,怕一时有些山高水低,我儿子又不在家。”妇人道:“大哥怎的一向不见?”王婆道:“那厮跟了个客人在外边,不见个音信回来,老身日逐耽心不下。”妇人道:“大哥今年多少年纪?”王婆道:“那厮十七岁了。”妇人道:“怎的不与他寻个亲事,与干娘也替得手?”王婆道:“因是这等说,家中没人。待老身东楞西补的来,早晚要替他寻下个儿。等那厮来,却再理会。见如今老身白日黑夜只发喘咳嗽,身子打碎般,睡不倒的,只害疼,一时先要预备下送终衣服。难得一个财主官人,常在贫家吃茶,但凡他宅里看病,买使女,说亲,见老身这般本分,大小事儿无不管顾老身。又布施了老身一套送终衣料,绸绢表里俱全,又有若干好绵,放在家里一年有余,不能勾做得。今年觉得好生不济,不想又撞着闰月,趁着两日倒闲,要做又被那裁缝勒掯,只推生活忙,不肯来做。老身说不得这苦也!”那妇人听了笑道:“只怕奴家做得不中意。若是不嫌时,奴这几日倒闲,出手与干娘做如何?”那婆子听了,堆下笑来说道:“若得娘子贵手做时,老身便死也得好处去。久闻娘子好针指,只是不敢来相央。”那妇人道:“这个何妨!既是许了干娘,务要与干娘做了,将历日去交人拣了黄道好日,奴便动手。”王婆道:“娘子休推老身不知,你诗词百家曲儿内字样,你不知识了多少,如何交人看历日?”妇人微笑道:“奴家自幼失学。”婆子道:“好说,好说。”便取历日递与妇人。妇人接在手内,看了一回,道:“明日是破日,后日也不好,直到外后日方是裁衣日期。”王婆一把手取过历头来挂在墙上,便道:“若得娘子肯与老身做时,就是一点福星。何用选日!老身也曾央人看来,说明日是个破日,老身只道裁衣日不用破日,我不忌他。”那妇人道:“归寿衣服,正用破日便好。”王婆道:“既是娘子肯作成,老身胆大,只是明日起动娘子,到寒家则个。”妇人道:“何不将过来做?”王婆道:“便是老身也要看娘子做生活,又怕门首没人。”妇人道:“既是这等说,奴明日饭后过来。”那婆子千恩万谢下楼去了,当晚回覆了西门庆话,约定后日准来。当夜无话。

次日清晨,王婆收拾房内干净,预备下针线,安排了茶水,在家等候。且说武大吃了早饭,挑着担儿自出去了。那妇人把帘儿挂了,分付迎儿看家,从后门走过王婆家来。那婆子欢喜无限,接入房里坐下,便浓浓点一盏胡桃松子泡茶与妇人吃了。抹得桌子干净,便取出那绸绢三匹来。妇人量了长短,裁得完备,缝将起来。婆子看了,口里不住喝采道:“好手段,老身也活了六七十岁,眼里真个不曾见这般好针指!”那妇人缝到日中,王婆安排些酒食请他,又下了一箸面与那妇人吃。再缝一歇,将次晚来,便收拾了生活,自归家去。恰好武大挑担儿进门,妇人拽门下了帘子。武大入屋里,看见老婆面色微红,问道:“你那里来?”妇人应道:“便是间壁干娘央我做送终衣服,日中安排些酒食点心请我吃。”武大道:“你也不要吃他的才是,我们也有央及他处。他便央你做得衣裳,你便自归来吃些点心,不值得甚么,便搅挠他。你明日再去做时,带些钱在身边,也买些酒食与他回礼。常言道:远亲不如近邻,休要失了人情。他若不肯交你还礼时,你便拿了生活来家,做还与他便了。”正是:

\[
阿母牢笼设计深,大郎愚卤不知音。
带钱买酒酬奸诈,却把婆娘自送人。
\]
妇人听了武大言语,当晚无话。

次日饭后,武大挑担儿出去了,王婆便踅过来相请。妇人去到他家屋里,取出生活来,一面缝来。王婆忙点茶来与他吃了茶。看看缝到日中,那妇人向袖中取出三百文钱来,向王婆说道:“干娘,奴和你买盏酒吃。”王婆道:“啊呀,那里有这个道理。老身央及娘子在这里做生活,如何交娘子倒出钱,婆子的酒食,不到吃伤了哩!”那妇人道:“却是拙夫分付奴来,若是干娘见外时,只是将了家去,做还干娘便了。”那婆子听了道:“大郎直恁地晓事!既然娘子这般说时,老身且收下。”这婆子生怕打搅了事,自又添钱去买好酒好食来,殷勤相待。看官听说:但凡世上妇人,由你十分精细,被小意儿纵十个九个着了道儿。这婆子安排了酒食点心,和那妇人吃了。再缝了一歇,看看晚来,千恩万谢归去了。

话休絮烦。第三日早饭后,王婆只张武大出去了,便走过后後门首叫道:“娘子,老身大胆。”那妇人从楼上应道:“奴却待来也。”两个厮见了,来到王婆房里坐下,取过生活来缝。那婆子点茶来吃,自不必说。妇人看看缝到晌午前后。却说西门庆巴不到此日,打选衣帽齐齐整整,身边带着三五两银子,手里拿着洒金川扇儿,摇摇摆摆迳往紫石街来。到王婆门首,便咳嗽道:“王干娘,连日如何不见?”那婆子瞧科,便应道:“兀的谁叫老娘?”西门庆道:“是我。”那婆子赶出来看了,笑道:“我只道是谁,原来是大官人!你来得正好,且请入屋里去看一看。”把西门庆袖子只一拖,拖进房里来,对那妇人道:“这个便是与老身衣料施主官人。”西门庆睁眼看着那妇人:云鬟叠翠,粉面生春,上穿白布衫儿,桃红裙子,蓝比甲,正在房里做衣服。见西门庆过来,便把头低了。这西门庆连忙向前屈身唱喏。那妇人随即放下生活,还了万福。王婆便道:“难得官人与老身段匹绸绢,放在家一年有余,不曾得做,亏杀邻家这位娘子出手与老身做成全了。真个是布机也似好针线,缝的又好又密,真个难得!大官人,你过来且看一看。”西门庆拿起衣服来看了,一面喝采,口里道:“这位娘子,传得这等好针指,神仙一般的手段!”那妇人低头笑道:“官人休笑话。”西门庆故问王婆道:“干娘,不敢动问,这位娘子是谁家宅上的娘子?”王婆道:“你猜。”西门庆道:“小人如何猜得着。”王婆哈哈笑道:“大官人你请坐,我对你说了罢。”那西门庆与妇人对面坐下。那婆子道:“好交大官人得知罢,你那日屋檐下走,打得正好。”西门庆道:“就是那日在门首叉竿打了我的?倒不知是谁家宅上娘子?”妇人分外把头低了一低,笑道:“那日奴误冲撞,官人休怪!”西门庆连忙应道:“小人不敢。”王婆道:“就是这位,却是间壁武大娘子。”西门庆道:“原来如此,小人失瞻了。”王婆因望妇人说道:“娘子你认得这位官人么?”妇人道:“不识得。”婆子道:“这位官人,便是本县里一个财主,知县相公也和他来往,叫做西门大官人。家有万万贯钱财,在县门前开生药铺。家中钱过北斗,米烂成仓,黄的是金,白的是银,圆的是珠,放光的是宝,也有犀牛头上角,大象口中牙。他家大娘子,也是我说的媒,是吴千户家小姐,生得面伶百俐。”因问:“大官人,怎的不过贫家吃茶?”西门庆道:“便是家中连日小女有人家定了,不得闲来。”婆子道:“大姐有谁家定了?怎的不请老身去说媒?”西门庆道:“被东京八十万禁军杨提督亲家陈宅定了。他儿子陈敬济才十七岁,还上学堂。不是也请干娘说媒,他那边有了个文嫂儿来讨帖儿,俺这里又使常在家中走的卖翠花的薛嫂儿,同做保山,说此亲事。干娘若肯去,到明日下小茶,我使人来请你。”婆子哈哈笑道:“老身哄大官人耍子。俺这媒人们都是狗娘养下来的,他们说亲时又没我,做成的熟饭儿怎肯搭上老身一分?常言道:当行压当行。到明日娶过了门时,老身胡乱三朝五日,拿上些人情去走走,讨得一张半张桌面,到是正经。怎的好和人斗气!”两个一递一句说了一回。婆子只顾夸奖西门庆,口里假嘈,那妇人便低了头缝针线。

\[
水性从来是女流,背夫常与外人偷。
金莲心爱西门庆,淫荡春心不自由。
\]

西门庆见金莲有几分情意欢喜,恨不得就要成双。王婆便去点两盏茶来,递一盏西门庆,一盏与妇人,说道:“娘子相待官人吃些茶。”旋又看着西门庆,把手在脸上摸一摸,西门庆已知有五分光了。自古“风流茶说合,酒是色媒人”。王婆便道:“大官人不来,老身也不敢去宅上相请。一者缘法撞遇,二者来得正好。常言道:一客不烦二主。大官人便是出钱的,这位娘子便是出力的,亏杀你这两位施主。不是老身路歧相烦,难得这位娘子在这里,官人好与老身做个主人,拿出些银子买些酒食来,与娘子浇浇手,如何?”西门庆道:“小人也见不到这里,有银子在此。”便向茄袋里取出来,约有一两一块,递与王婆,交备办酒食。那妇人便道“不消生受。”口里说着恰不动身。王婆接了银子,临出门便道:“有劳娘子相陪大官人坐一坐,我去就来。”那妇人道:“干娘免了罢。”却亦不动身。王婆便出门去了,丢下西门庆和那妇人在屋里。

这西门庆一双眼不转睛,只看着那妇人。那婆娘也把眼来偷睃西门庆,又低着头做生活。不多时,王婆买了见成肥鹅烧鸭、熟肉鲜鲊、细巧果子,归来尽把盘碟盛了,摆在房里桌子上。看那妇人道:“娘子且收拾过生活,吃一杯儿酒。”那妇人道:“你自陪大官人吃,奴却不当。”那婆子道:“正是专与娘子浇手,如何却说这话!”一面将盘馔却摆在面前,三人坐下,把酒来斟。西门庆拿起酒盏来道:“干娘相待娘子满饮几杯。”妇人谢道:“奴家量浅,吃不得。”王婆道:“老身得知娘子洪饮,且请开怀吃两盏儿。”那妇人一面接酒在手,向二人各道了万福。西门庆拿起箸来说道:“干娘替我劝娘子些菜儿。”那婆子拣好的递将过来与妇人吃。一连斟了三巡酒,那婆子便去烫酒来。西门庆道:“小人不敢动问,娘子青春多少?”妇人低头应道:“二十五岁。”西门庆道:“娘子到与家下贱内同庚,也是庚辰属龙的。他是八月十五日子时。”妇人又回应道:“将天比地,折杀奴家。”王婆便插口道:“好个精细的娘子,百伶百俐,又不枉做得一手好针线。诸子百家,双陆象棋,折牌道字,皆通。一笔好写。”西门庆道:“却是那里去讨。”王婆道:“不是老身说是非,大官人宅上有许多,那里讨得一个似娘子的!”西门庆道:“便是这等,一言难尽。只是小人命薄,不曾招得一个好的在家里。”王婆道:“大官人先头娘子须也好。”西门庆道:“休说!我先妻若在时,却不恁的家无主,屋到竖。如今身边枉自有三五七口人吃饭,都不管事。”婆子嘈道:“连我也忘了,没有大娘子得几年了?”西门庆道:“说不得,小人先妻陈氏,虽是微末出身,却倒百伶百俐,是件都替的我。如今不幸他没了,已过三年来。今继娶这个贱累,又常有疾病,不管事,家里的勾当都七颠八倒。为何小人只是走了出来?在家里时,便要呕气。”婆子道:“大官人,休怪我直言,你先头娘子并如今娘子,也没这大娘子这手针线,这一表人物。”西门庆道:“便是房下们也没这大娘子一般儿风流。”那婆子笑道:“官人,你养的外宅东街上住的,如何不请老身去吃茶?”西门庆道:“便是唱慢曲儿的张惜春。我见他是路歧人,不喜欢。”婆子又道:“官人你和勾栏中李娇儿却长久。”西门庆道:“这个人见今已娶在家里。若得他会当家时,自册正了他。”王婆道:“与卓二姐却相交得好?”西门庆道:“卓丢儿别要说起,我也娶在家做了第三房。近来得了个细疾,却又没了。”婆子道:“耶乐,耶乐!若有似大娘子这般中官人意的,来宅上说,不妨事么?”西门庆道:“我的爹娘俱已没了,我自主张,谁敢说个不字?”王婆道:“我自说耍,急切便那里有这般中官人意的!”西门庆道:“做甚么便没?只恨我夫妻缘分上薄,自不撞着哩。”西门庆和婆子一递一句说了一回。王婆道:“正好吃酒,却又没了。官人休怪老身差拨,买一瓶儿酒来吃如何?”西门庆便向茄袋内,还有三四两散银子,都与王婆,说道:“干娘,你拿了去,要吃时只顾取来,多的干娘便就收了。”那婆子谢了起身。睃那粉头时,三钟酒下肚,哄动春心,又自两个言来语去,都有意了,只低了头不起身。正是:

\[
眼意眉情卒未休,姻缘相凑遇风流。
王婆贪贿无他技,一味花言巧舌头。
\]

\newpage
%# -*- coding:utf-8 -*-
%%%%%%%%%%%%%%%%%%%%%%%%%%%%%%%%%%%%%%%%%%%%%%%%%%%%%%%%%%%%%%%%%%%%%%%%%%%%%%%%%%%%%


\chapter{赴巫山潘氏幽欢\KG 闹茶坊郓哥义愤}


诗曰:

\[
璇闺绣户斜光入,千金女儿倚门立。
横波美目虽后来,罗袜遥遥不相及。
闻道今年初避人,珊珊镜挂长随身。
愿得侍儿为道意,后堂罗帐一相亲。
\]


话说王婆拿银子出门,便向妇人满面堆下笑来,说道:“老身去那街上取瓶儿来,有劳娘子相待官人坐一坐。壶里有酒,没便再筛两盏儿,且和大官人吃着,老身直去县东街,那里有好酒买一瓶来,有好一歇儿耽搁。”妇人听了说:“干娘休要去,奴酒不多用了。”婆子便道:“阿呀!娘子,大官人又不是别人,没事相陪吃一盏儿,怕怎的!”妇人口里说“不用了”坐着却不动身。婆子一面把门拽上,用索儿拴了,倒关他二人在屋里。当路坐了,一头续着锁。

这妇人见王婆去了,倒把椅儿扯开一边坐着,却只偷眼睃看。西门庆坐在对面,一径把那双涎瞪瞪的眼睛看着他,便又问道:“却才到忘了问娘子尊姓?”妇人便低着头带笑的回道:“姓武。”西门庆故做不听得,说道:“姓堵?”那妇人却把头又别转着,笑着低声说道:“你耳朵又不聋。”西门庆笑道:“呸,忘了!正是姓武。只是俺清河县姓武的却少,只有县前一个卖饮饼的三寸丁姓武,叫做武大郎,敢是娘子一族么?”妇人听得此言,便把脸通红了,一面低着头微笑道:“便是奴的丈夫。”西门庆听了,半日不做声,呆了脸,假意失声道屈。妇人一面笑着,又斜瞅了他一眼,低声说道:“你又没冤枉事,怎的叫屈?”西门庆道:“我替娘子叫屈哩!”却说西门庆口里娘子长娘子短,只顾白嘈。这妇人一面低着头弄裙子儿,又一回咬着衫袖口儿,咬得袖口儿格格驳驳的响,要便斜溜他一眼儿。只见这西门庆推害热,脱了上面绿纱褶子道:“央烦娘子替我搭在干娘护炕上。”这妇人只顾咬着袖儿别转着,不接他的,低声笑道:“自手又不折,怎的支使人!”西门庆笑着道:“娘子不与小人安放,小人偏要自己安放。”一面伸手隔桌子搭到床炕上去,却故意把桌上一拂,拂落一只箸来。却也是姻缘凑着,那只箸儿刚落在金莲裙下。西门庆一面斟酒劝那妇人,妇人笑着不理他。他却又待拿起箸子起来,让他吃菜儿。寻来寻去不见了一只。这金莲一面低着头,把脚尖儿踢着,笑道:“这不是你的箸儿!”西门庆听说,走过金莲这边来道:“原来在此。”蹲下身去,且不拾箸,便去他绣花鞋头上只一捏。那妇人笑将起来,说道:“怎这的罗唣!我要叫了起来哩!”西门庆便双膝跪下说道:“娘子可怜小人则个!”一面说着,一面便摸他裤子。妇人叉开手道:“你这歪厮缠人,我却要大耳刮子打的呢!”西门庆笑道:“娘子打死了小人,也得个好处。”于是不由分说,抱到王婆床炕上,脱衣解带,共枕同欢。却说这妇人自从与张大户勾搭,这老儿是软如鼻涕脓如酱的一件东西,几时得个爽利!就是嫁了武大,看官试想,三寸丁的物事,能有多少力量?今番遇了西门庆,风月久惯,本事高强的,如何不喜?但见:

\[
交颈鸳鸯戏水,并头鸾凤穿花。喜孜孜连理枝生,美甘甘同心带结。一个将朱唇紧贴,一个将粉脸斜偎。罗袜高挑,肩膀上露两弯新月;金钗斜坠,枕头边堆一朵乌云。誓海盟山,搏弄得千般旖妮;羞云怯雨,揉搓的万种妖娆。恰恰莺声,不离耳畔。津津甜唾,笑吐舌尖。杨柳腰脉脉春浓,樱桃口微微气喘。星眼朦胧,细细汗流香玉颗;酥胸荡漾,涓涓露滴牡丹心。直饶匹配眷姻谐,真个偷情滋味美。
\]

当下二人云雨才罢,正欲各整衣襟,只见王婆推开房门入来,大惊小怪,拍手打掌,低低说道:“你两个做得好事!”西门庆和那妇人都吃了一惊。那婆子便向妇人道:“好呀,好呀!我请你来做衣裳,不曾交你偷汉子!你家武大郎知,须连累我。不若我先去对武大说去。”回身便走。那妇人慌的扯住她裙子,红着脸低了头,只得说声:“干娘饶恕!”王婆便道:“你们都要依我一件事,从今日为始,瞒着武大,每日休要失了大官人的意。早叫你早来,晚叫你晚来,我便罢休。若是一日不来,我便就对你武大说。”那妇人羞得要不的,再说不出来。王婆催逼道:“却是怎的?快些回覆我。”妇人藏转着头,低声道:“来便是了。”王婆又道:“西门大官人,你自不用老身说得,这十分好事已都完了,所许之物,不可失信,你若负心,我也要对武大说。”西门庆道:“干娘放心,并不失信。”婆子道:“你每二人出语无凭,要各人留下件表记拿着,才见真情。”西门庆便向头上拔下一根金头簪来,插在妇人云髻上。妇人除下来袖了,恐怕到家武大看见生疑。妇人便不肯拿甚的出来,却被王婆扯着袖子一掏,掏出一条杭州白绉纱汗巾,掠与西门庆收了。三人又吃了几杯酒,已是下午时分。那妇人起身道:“奴回家去罢。”便丢下王婆与西门庆,踅过后门归来。先去下了帘子,武大恰好进门。

且说王婆看着西门庆道:“好手段么?”西门庆道:“端的亏了干娘,真好手段!”王婆又道:“这雌儿风月如何?”西门庆道:“色系子女不可言。”婆子道:“她房里弹唱姐儿出身,甚么事儿不久惯知道!还亏老娘把你两个生扭做夫妻,强撮成配。你所许老身东西,休要忘了。”西门庆道:“我到家便取银子送来。”王婆道:“眼望旌捷旗,耳听好消息。不要交老身棺材出了讨挽歌郎钱。”西门庆一面笑着,看街上无人,带上眼纱去了。不在话下。

次日,又来王婆家讨茶吃。王婆让坐,连忙点茶来吃了。西门庆便向袖中取出一锭十两银子来,递与王婆。但凡世上人,钱财能动人意。那婆子黑眼睛见了雪花银子,一面欢天喜地收了,一连道了两个万福,说道:“多谢大官人布施!”因向西门庆道:“这咱晚武大还未出门,待老身往她家推借瓢,看一看。”一面从后门踅过妇人家来。妇人正在房中打发武大吃饭,听见叫门,问迎儿:“是谁?”迎儿道:“是王奶奶来借瓢。”妇人连忙迎将出来道:“干娘,有瓢,一任拿去。且请家里坐。”婆子道:“老身那边无人。”因向妇人使手势,妇人就知西门庆来了。婆子拿瓢出了门,一力撺掇武大吃了饭,挑担出去了。先到楼上从新妆点,换了一套艳色新衣,分付迎儿:“好生看家,我往你王奶家坐一坐就来。若是你爹来时,就报我知道。若不听我说,打下你个小贱人下截来。”迎儿应诺不题。

妇人一面走过王婆茶坊里来。正是:

\[
合欢桃杏春堪笑,心里原来别有仁。
\]
有词单道这双关二意:

\[
这瓢是瓢,口儿小身子儿大。你幼在春风棚上恁儿高,到大来人难要。他怎肯守定颜回甘贫乐道,专一趁东风,水上漂。也曾在马房里喂料,也曾在茶房里来叫,如今弄得许由也不要。赤道黑洞洞葫芦中卖的甚么药?
\]

那西门庆见妇人来了,如天上落下来一般,两个并肩叠股而坐。王婆一面点茶来吃了,因问:“昨日归家,武大没问甚么?”妇人道:“他问干娘衣服做了不曾,我说道衣服做了,还与干娘做送终鞋袜。”说毕,婆子连忙安排上酒来,摆在房内,二人交杯畅饮。这西门庆仔细端详那妇人,比初见时越发标致。吃了酒,粉面上透出红白来,两道水鬓描画的长长的。端的平欺神仙,赛过嫦娥。

\[
动人心红白肉色,堪人爱可意裙钗。裙拖着翡翠纱衫,袖挽泥金带。喜孜孜宝髻斜歪。恰便似月里嫦娥下世来,不枉了千金也难买。\named{右调《沉醉东风》}
\]

西门庆夸之不足,搂在怀中,掀起他裙来,看见他一对小脚穿着老鸦缎子鞋儿,恰刚半叉,心中甚喜。一递一口与他吃酒,嘲问话儿。妇人因问西门庆贵庚,西门庆告他说:“二十七岁,七月二十八日子时生。”妇人问:“家中有几位娘子?”西门庆道:“除下拙妻,还有三四个身边人,只是没一个中我意的。”妇人又问:“几位哥儿?”西门庆道:“只是一个小女,早晚出嫁,并无娃儿。”西门庆嘲问了一回,向袖中取出银穿心金裹面盛着香茶木樨饼儿来,用舌尖递送与妇人。两个相搂相抱,鸣咂有声。那婆子只管往来拿菜筛酒,那里去管他闲事,由着二人在房内做一处取乐玩耍。少顷吃得酒浓,不觉烘动春心,西门庆色心辄起,露出腰间那话,引妇人纤手扪弄。原来西门庆自幼常在三街四巷养婆娘,根下犹带着银打就,药煮成的托子。那话煞甚长大,红赤赤黑须,直竖竖坚硬,好个东西:

\[
一物从来六寸长,有时柔软有时刚。
软如醉汉东西倒,硬似风僧上下狂。
出牝入阴为本事,腰州脐下作家乡。
天生二子随身便,曾与佳人斗几场。
\]

少顷,妇人脱了衣裳。西门庆摸见牝户上并无毳毛,犹如白馥馥、鼓蓬蓬发酵的馒头,软浓浓、红绉绉出笼的果馅,真个是千人爱万人贪一件美物:

\[
温紧香干口赛莲,能柔能软最堪怜。
喜便吐舌开颜笑,困便随身贴股眠。
内裆县里为家业,薄草涯边是故园。
若遇风流轻俊子,等闲战斗不开言。
\]

话休饶舌。那妇人自当日为始,每日踅过王婆家来,和西门庆做一处,恩情似漆,心意如胶。自古道:好事不出门,恶事传千里。不到半月之间,街坊邻舍都晓的了,只瞒着武大一个不知。正是:

\[
自知本分为活计,那晓防奸革弊心。
\]

话分两头。且说本县有个小的,年方十五六岁,本身姓乔,因为做军在郓州生养的,取名叫做郓哥。家中只有个老爹,年纪高大。那小厮生得乖觉,自来只靠县前这许多酒店里卖些时新果品,时常得西门庆赍发他些盘缠。其日正寻得一篮儿雪梨,提着绕街寻西门庆。又有一等多口人说:“郓哥你要寻他,我教你一个去处。”郓哥道:“起动老叔,教我那去寻他的是?”那多口的道:“我说与你罢。西门庆刮剌上卖炊饼的武大老婆,每日只在紫石街王婆茶坊里坐的。这咱晚多定只在那里。你小孩子家,只故撞进去不妨。”那郓哥得了这话,谢了那人,提了篮儿,一直往紫石街走来,迳奔入王婆茶坊里去。却正见王婆坐在小凳儿上绩线,郓哥把篮儿放下,看着王婆道:“干娘!声喏。”那婆子问道:“郓哥,你来这里做甚么?”郓哥道:“要寻大官人,赚三五十钱养活老爹。”婆子道:“甚么大官人?”郓哥道:“情知是那个,便只是他那个。”婆子道:“便是大官人,也有个姓名。”郓哥道:“便是两个字的。”婆子道:“甚么两个字的?”郓哥道:“干娘只是要作耍。我要和西门大官人说句话儿!”望里便走。那婆子一把揪住道:“这小猴子那里去?人家屋里,各有内外。”郓哥道:“我去房里便寻出来。”王婆骂道:“含乌小囚儿!我屋里那里讨甚么西门大官?”郓哥道:“干娘不要独自吃,也把些汁水与我呷一呷。我有甚么不理会得!”婆子便骂:“你那小囚攮的,理会得甚么?”郓哥道:“你正事马蹄刀木杓里切菜——水泄不漏,直要我说出来,只怕卖炊饼的哥哥发作!”那婆子吃他这两句道着他真病,心中大怒,喝道:“含乌小猢狲,也来老娘屋里放屁!”郓哥道:“我是小猢狲,你是马伯六,做牵头的老狗肉!”那婆子揪住郓哥凿上两个栗暴。郓哥叫道:“你做甚么便打我?”婆子骂道:“贼肏娘的小猢狲!你敢高做声,大耳刮子打出你去。”郓哥道:“贼老咬虫,没事便打我!”这婆子一头叉,一头大栗暴,直打出街上去,把雪梨篮儿也丢出去。那篮雪梨四分五落滚了开去。这小猴子打那虔婆不过,一头骂,一头哭,一头走,一头街上拾梨儿,指着王婆茶坊里骂道:“老咬虫,我交你不要慌!我不与他不做出来不信!定然遭塌了你这场门面,交你赚不成钱!”这小猴子提个篮儿,迳奔街上寻这个人。却正是:

\[
掀翻孤兔窝中草,惊起鸳鸯沙上眠。
\]

\newpage
%# -*- coding:utf-8 -*-
%%%%%%%%%%%%%%%%%%%%%%%%%%%%%%%%%%%%%%%%%%%%%%%%%%%%%%%%%%%%%%%%%%%%%%%%%%%%%%%%%%%%%


\chapter{捉奸情郓哥定计\KG 饮鸩药武大遭殃}


诗曰:

\[
参透风流二字禅,好姻缘是恶姻缘。
痴心做处人人爱,冷眼观时个个嫌。
野草闲花休采折,真姿劲质自安然。
山妻稚子家常饭,不害相思不损钱。
\]


话说当下郓哥被王婆打了,心中正没出气处,提了雪梨篮儿,一迳奔来街上寻武大郎。转了两条街,只见武大挑着炊饼担儿,正从那条街过来。郓哥见了,立住了脚,看着武大道:“这几时不见你,吃得肥了!”武大歇下担儿道:“我只是这等模样,有甚吃得肥处?”郓哥道:“我前日要籴些麦稃,一地里没籴处,人都道你屋里有。”武大道:“我屋里并不养鹅鸭,那里有这麦稃?”郓哥道:“你说没麦稃,怎的赚得你恁肥耷耷的,便颠倒提你起来也不妨,煮你在锅里也没气。”武大道:“小囚儿,倒骂得我好。我的老婆又不偷汉子,我如何是鸭?”郓哥道:“你老婆不偷汉子,只偷子汉。”武大扯住郓哥道:“还我主儿来!”郓哥道:“我笑你只会扯我,却不道咬下他左边的来。”武大道:“好兄弟,你对我说是谁,我把十个炊饼送你。”郓哥道:“炊饼不济事。你只做个东道,我吃三杯,便说与你。”武大道:“你会吃酒?跟我来。”

武大挑了担儿,引着郓哥,到个小酒店里,歇下担儿,拿几个炊饼,买了些肉,讨了一镟酒,请郓哥吃着。武大道:“好兄弟,你说与我则个。”郓哥道:“且不要慌,等我一发吃完了,却说与你。你却不要气苦,我自帮你打捉。”武大看那猴子吃了酒肉:“你如今却说与我。”郓哥道:“你要得知,把手来摸我头上的疙瘩。”武大道:“却怎地来有这疙瘩?”郓哥道:“我对你说,我今日将这篮雪梨去寻西门大官,一地里没寻处。街上有人道:‘他在王婆茶坊里来,和武大娘子勾搭上了,每日只在那里行走。’我指望见了他,撰他三五十文钱使。叵耐王婆那老猪狗,不放我去房里寻他,大栗暴打出我来。我特地来寻你。我方才把两句话来激你,我不激你时,你须不来问我。”武大道:“真个有这等事?”郓哥道:“又来了,我道你这般屁鸟人!那厮两个落得快活,只专等你出来,便在王婆房里做一处。你问道真个也是假,难道我哄你不成?”武大听罢,道:“兄弟,我实不瞒你说,我这婆娘每日去王婆家里做衣服,做鞋脚,归来便脸红。我先妻丢下个女孩儿,朝打暮骂,不与饭吃,这两日有些精神错乱,见了我,不做欢喜。我自也有些疑忌在心里,这话正是了。我如今寄了担儿,便去捉奸如何?”郓哥道:“你老大一条汉,元来没些见识!那王婆老狗,什么利害怕人的人!你如何出得他手?他二人也有个暗号儿,见你入来拿他,把你老婆藏过了。那西门庆须了得!打你这般二十个。若捉他不着,反吃他一顿好拳头。他又有钱有势,反告你一状子,你须吃他一场官司,又没人做主,干结果了你性命!”武大道:“兄弟,你都说得是。我却怎的出得这口气?”郓哥道:“我吃那王婆打了,也没出气处。我教你一着:今日归去,都不要发作,也不要说,只自做每日一般。明朝便少做些炊饼出来卖,我自在巷口等你。若是见西门庆入去时,我便来叫你。你便挑着担儿只在左近等我。我先去惹那老狗,他必然来打我。我先把篮儿丢出街心来,你却抢入。我便一头顶住那婆子,你便奔入房里去,叫起屈来。此计如何?”武大道:“既是如此,却是亏了兄弟。我有两贯钱,我把你去,你到明日早早来紫石街巷口等我。”郓哥得了钱并几个炊饼,自去了。武大还了酒钱,挑了担儿,自去卖了一遭归去。

原来这妇人,往常时只是骂武大,百般的欺负他。近日来也自知无礼,只得窝盘他些个。当晚武大挑了担儿归来,也是和往日一般,并不题起别事。那妇人道:“大哥,买盏酒吃?”武大道:“却才和一般经纪人买了三盏吃了。”那妇人便安排晚饭与他吃了。当夜无话。次日饭后,武大只做三两扇炊饼,安在担儿上。这妇人一心只想着西门庆,那里来理会武大的做多做少。当日武大挑了担儿,自出去做买卖。这妇人巴不的他出去了,便踅过王婆茶坊里来等西门庆。

且说武大挑着担儿,出到紫石街巷口,迎见郓哥提着篮儿在那里张望。武大道:“如何?”郓哥道:“还早些个。你自去卖一遭来,那厮七八也将来也。你只在左近处伺候,不可远去了。”武大云飞也似去卖了一遭回来。郓哥道:“你只看我篮儿抛出来,你便飞奔入去。”武大把担儿寄下,不在话下。

却说郓哥提着篮儿,走入茶坊里来,向王婆骂道:“老猪狗!你昨日为甚么便打我?”那婆子旧性不改,便跳身起来喝道:“你这小猢狲!老娘与你无干,你如何又来骂我?”郓哥道:“便骂你这马伯六,做牵头的老狗肉,直我鸡巴!”那婆子大怒,揪住郓哥便打。郓哥叫一声:“你打我!”把那篮儿丢出当街上来。那婆子却待揪他,被这小猴子叫一声“你打”时,就打王婆腰里带个住,看着婆子小肚上,只一头撞将去,险些儿不跌倒,却得壁子碍住不倒。那猴子死顶在壁上。只见武大从外裸起衣裳,大踏步直抢入茶坊里来。那婆子见是武大,来得甚急,待要走去阻当,却被这小猴子死力顶住,那里肯放!婆子只叫得“武大来也!”那妇人正和西门庆在房里,做手脚不迭,先奔来顶住了门。这西门庆便钻入床下躲了。武大抢到房门首,用手推那房门时,那里推得开!口里只叫“做得好事!”那妇人顶着门,慌做一团,口里便说道:“你闲常时只好鸟嘴,卖弄杀好拳棒,临时便没些用儿!见了纸虎儿也吓一交!”那妇人这几句话,分明叫西门庆来打武大,夺路走。西门庆在床底下听了妇人这些话,提醒他这个念头,便钻出来说道:“不是我没这本事,一时间没这智量。”便来拔开门,叫声“不要来!”武大却待揪他,被西门庆早飞起脚来。武大矮小,正踢中心窝,扑地望后便倒了。西门庆打闹里一直走了。郓哥见势头不好,也撇了王婆,撒开跑了。街坊邻舍,都知道西门了得,谁敢来管事?王婆当时就地下扶起武大来,见他口里吐血,面皮腊渣也似黄了,便叫那妇人出来,舀碗水来救得苏醒,两个上下肩搀着,便从后门归到家中楼上去,安排他床上睡了。当夜无话。次日,西门庆打听得没事,依前自来王婆家,和这妇人顽耍,只指望武大自死。

武大一病五日不起,更兼要汤不见,要水不见,每日叫那妇人又不应。只见他浓妆艳抹了出去,归来便脸红。小女迎儿又吃妇人禁住,不得向前,吓道:“小贱人,你不对我说,与了他水吃,都在你身上!”那迎儿见妇人这等说,怎敢与武大一点汤水吃!武大几遍只是气得发昏,又没人来采问。一日,武大叫老婆过来,分付他道:“你做的勾当,我亲手捉着你奸,你倒挑拨奸夫踢了我心。至今求生不生,求死不死,你们却自去快活。我死自不妨,和你们争执不得了。我兄弟武二,你须知他性格,倘或早晚归来,他肯干休?你若肯可怜我,早早扶得我好了,他归来时,我都不提起。你若不看顾我时,待他归来,却和你们说话。”这妇人听了,也不回言,却踅过王婆家来,一五一十都对王婆和西门庆说了。那西门庆听了这话,似提在冷水盆内一般,说道:“苦也!我须知景阳冈上打死大虫的武都头。我如今却和娘子眷恋日久,情孚意合,拆散不开。据此等说时,正是怎生得好?却是苦也!”王婆冷笑道:“我倒不曾见,你是个把舵的,我是个撑船的,我倒不慌,你倒慌了手脚!”西门庆道:“我枉自做个男子汉,到这般去处,却摆布不开。你有甚么主见,遮藏我们则个。”王婆道:“既然我遮藏你们,我有一条计。你们却要长做夫妻,短做夫妻?”西门庆道:“干娘,你且说如何是长做夫妻、短做夫妻?”王婆道:“若是短做夫妻,你们就今日便分散。等武大将息好了起来,与他陪了话。武二归来都没言语,待他再差使出去,却又来相会。这是短做夫妻。你们若要长做夫妻,每日同在一处,不耽惊受怕,我却有这条妙计,只是难教你们!”西门庆道:“干娘,周旋了我们则个,只要长做夫妻。”王婆道:“这条计用着件东西,别人家里都没,天生天化,大官人家里却有。”西门庆道:“便是要我的眼睛,也剜来与你。却是甚么东西?”王婆道:“如今这捣子病得重,趁他狼狈,好下手。大官人家里取些砒霜,却交大娘子自去赎一帖心疼的药来,却把这砒霜下在里面,把这矮子结果了,一把火烧得干干净净,没了踪迹。便是武二回来,他待怎的?自古道:‘幼嫁从亲,再嫁由身。’小叔如何管得暗地里事!半年一载,等待夫孝满日,大官人娶到家去。这不是长远夫妻,偕老同欢!此计如何?”西门庆道:“干娘此计甚妙。自古道:欲救生快活,须下死功夫。罢罢罢!一不做,二不休。”王婆道:“可知好哩!这是剪草除根,萌芽不发。大官人往家里去快取此物来,我自教娘子下手。事了时,却要重重谢我。”西门庆道:“这个自然,不消你说。”

\[
云情雨意两绸缪,恋色迷花不肯休。
毕竟人生如泡影,何须死下杀人谋?
\]

且说西门庆去不多时,包了一包砒霜,递与王婆收了。这婆子看着那妇人道:“大娘子,我教你下药的法儿。如今武大不对你说教你救活他?你便乘此把些小意儿贴恋他。他若问你讨药吃时,便把这砒霜调在心疼药里。待他一觉身动,你便把药灌将下去。他若毒气发时,必然肠胃迸断,大叫一声。你却把被一盖,不要使人听见,紧紧的按住被角。预先烧下一锅汤,煮着一条抹布。他那药发之时,必然七窍内流血,口唇上有牙齿咬的痕迹。他若放了命,你便揭起被来,却将煮的抹布只一揩,都揩没了血迹,便入在材里,扛出去烧了,有甚么不了事!”那妇人道:“好却是好,只是奴家手软,临时安排不得尸首。”婆子道:“这个易得。你那边只敲壁子,我自过来帮扶你。”西门庆道:“你们用心整理,明日五更,我来讨话。”说罢,自归家去了。王婆把这砒霜用手捻为细末,递与妇人,将去藏了。

那妇人回到楼上,看着武大,一丝没了两气,看看待死。那妇人坐在床边假哭。武大道:“你做甚么来哭?”妇人拭着眼泪道:“我的一时间不是,吃那西门庆局骗了。谁想脚踢中了你心。我问得一处有好药,我要去赎来医你,又怕你疑忌,不敢去取。”武大道:“你救我活,无事了,一笔都勾。武二来家,亦不提起。你快去赎药来救我则个!”那妇人拿了铜钱,迳来王婆家里坐地,却教王婆赎得药来。把到楼上,交武大看了,说道:“这帖心疼药,太医交你半夜里吃了,倒头一睡,盖一两床被,发些汗,明日便起得来。”武大道:“却是好也。生受大嫂,今夜醒睡些,半夜调来我吃。”那妇人道:“你放心睡,我自扶持你。”看看天色黑了,妇人在房里点上灯,下面烧了大锅汤,拿了一方抹布煮在锅里。听那更鼓时,却正好打三更。那妇人先把砒霜倾在盏内,却舀一碗白汤,把到楼上,叫声:“大哥,药在那里?”武大道:““在我席子底下枕头边,你快调来我吃!”那妇人揭起席子,将那药抖在盏子里,将白汤冲在盏内,把头上银簪儿只一搅,调得匀了。左手扶起武大,右手把药便灌。武大呷了一口,说道:“大嫂,这药好难吃!”那妇人道:“只要他医得病好,管甚么难吃!”武大再呷第二口时,被这婆娘就势只一灌,一盏药都灌下喉咙去了。那妇人便放倒武大,慌忙跳下床来。武大哎了一声,说道:“大嫂,吃下这药去,肚里倒疼起来。苦呀,苦呀!倒当不得了。”这妇人便去脚后扯过两床被来,没头没脸只顾盖。武大叫道:“我也气闷!”那妇人道:“太医分付,教我与你发些汗,便好的快。”武大再要说时,这妇人怕他挣扎,便跳上床来,骑在武大身上,把手紧紧的按住被角,那里肯放些松宽!正是:

\[
油煎肺腑,火燎肝肠。心窝里如霜刀相侵,满腹中似钢刀乱搅。浑身冰冷,七窍血流。牙关紧咬,三魂赴在枉死城中;喉管枯干,七魄投望乡台上。地狱新添食毒鬼,阳间没了捉奸人。
\]

那武大当时哎了两声,喘息了一回,肠胃迸断,呜呼哀哉,身体动不得了。那妇人揭起被来,见了武大咬牙切齿,七窍流血,怕将起来,只得跳下床来,敲那壁子。王婆听得,走过后门头咳嗽。那妇人便下楼来,开了后门。王婆问道:“了也未?”那妇人道:“了便了了,只是我手脚软了,安排不得。”王婆道:“有甚么难处,我帮你便了。”那婆子便把衣袖卷起,舀了一桶汤,把抹布撇在里面,掇上楼来。卷过了被,先把武大口边唇上都抹了,却把七窍淤血痕迹拭净,便把衣裳盖在身上。两个从楼上一步一掇扛将下来,就楼下寻扇旧门停了。与他梳了头,戴上巾帻,穿了衣裳,取双鞋袜与他穿了,将片白绢盖了脸,拣床干净被盖在死尸身上。却上楼来,收拾得干净了,王婆自转将归去了。那婆娘却号号地假哭起“养家人”来。看官听说:原来但凡世上妇人哭有三样:有泪有声谓之哭,有泪无声谓之泣,无泪有声谓之号。当下那妇人干号了半夜。

次早五更,天色未晓,西门庆奔来讨信。王婆说了备细。西门庆取银子把与王婆,教买棺材发送,就叫那妇人商议。这婆娘过来和西门庆说道:“我的武大今日已死,我只靠着你做主!不到后来网巾圈儿打靠后。”西门庆道:“这个何须你费心!”妇人道:“你若负了心,怎的说?”西门庆道:“我若负了心,就是武大一般!”王婆道:“大官人,如今只有一件事要紧:天明就要入殓,只怕被仵作看出破绽来怎了?团头何九,他也是个精细的人,只怕他不肯殓。”西门庆笑道:“这个不妨事。何九我自分付他,他不敢违我的言语。”王婆道:“大官人快去分付他,不可迟了。”西门庆自去对何九说去了。正是:

\[
三光有影谁能待,万事无根只自生。
雪隐鹭鸶飞始见,柳藏鹦鹉语方闻。
\]

\newpage
%# -*- coding:utf-8 -*-
%%%%%%%%%%%%%%%%%%%%%%%%%%%%%%%%%%%%%%%%%%%%%%%%%%%%%%%%%%%%%%%%%%%%%%%%%%%%%%%%%%%%%


\chapter{何九受贿瞒天\KG 王婆帮闲遇雨}


词曰:

\[
别后谁知珠分玉剖。忘海誓山盟天共久,偶恋着山鸡,辄弃鸾俦。从此箫郎泪暗流,过秦楼几空回首。纵新人胜旧,也应须一别,洒泪登舟。
\]


却说西门庆去了。到天大明,王婆拿银子买了棺材冥器,又买些香烛纸钱之类,归来就于武大灵前点起一盏随身灯。邻舍街坊都来看望,那妇人虚掩着粉脸假哭。众街坊问道:“大郎得何病患便死了?”那婆娘答道:“因害心疼,不想一日日越重了,看看不能够好。不幸昨夜三更鼓死了,好是苦也!”又哽哽咽咽假哭起来。众邻舍明知道此人死的不明,不好只顾问他。众人尽劝道:“死是死了,活的自要安稳过。娘子省烦恼,天气暄热。”那妇人只得假意儿谢了,众人各自散去。王婆抬了棺材来,去请仵作团头何九。但是入殓用的都买了,并家里一应物件也都买了。就于报恩寺叫了两个禅和子,晚夕伴灵拜忏。不多时,何九先拨了几个火家整顿。

且说何九到巳牌时分,慢慢的走来,到紫石街巷口,迎见西门庆。叫道:“老九何往?”何九答道:“小人只去前面殓这卖炊饼的武大郎尸首。”西门庆道:“且停一步说话。”何九跟着西门庆,来到转角头一个小酒店里,坐下在阁儿内。西门庆道:“老九请上坐。”何九道:“小人是何等人,敢对大官人一处坐的!”西门庆道:“老九何故见外?且请坐。”二人让了一回,坐下。西门庆吩咐酒保:“取瓶好酒来。”酒保一面铺下菜蔬果品按酒之类,一面烫上酒来。何九心中疑忌,想道:“西门庆自来不曾和我吃酒,今日这杯酒必有蹊跷。”两个饮够多时,只见西门庆向袖子里摸出一锭雪花银子,放在面前说道:“老九休嫌轻微,明日另有酬谢。”何九叉手道:“小人无半点效力之处,如何敢受大官人见赐银两!若是大官人有使令,小人也不敢辞。”西门庆道:“老九休要见外,请收过了。”何九道:“大官人便说不妨。”西门庆道:“别无甚事。少刻他家自有些辛苦钱。只是如今殓武大的尸首,凡百事周全,一床锦被遮盖则个。”何九道:“我道何事!这些小事,有甚打紧,如何敢受大官人银两?”西门庆道:“你若不受时,便是推却。”何九自来惧西门庆是个把持官府的人,只得收了银子。又吃了几杯酒,西门庆呼酒保来:“记了帐目,明日来我铺子内支钱。”两个下楼,一面出了店门。临行,西门庆道:“老九是必记心,不可泄漏。改日另有补报。”吩咐罢,一直去了。

何九接了银子,自忖道:“其中缘故那却是不须提起的了。只是这银子,恐怕武二来家有说话,留着倒是个见证。”一面又忖道:“这两日倒要些银子搅缠,且落得用了,到其间再做理会便了。”于是一直到武大门首。只见那几个火家正在门首伺候。王婆也等的心里火发。何九一到,便问火家:“这武大是甚病死了?”火家道:“他家说害心疼病死了。”何九入门,揭起帘子进来。王婆接着道:“久等多时了,阴阳也来了半日,老九如何这咱才来?”何九道:“便是有些小事绊住了脚,来迟了一步。”只见那妇人穿着一件素淡衣裳,白布䯼髻,从里面假哭出来。何九道:“娘子省烦恼,大郎已是归天去了。”那妇人虚掩着泪眼道:“说不得的苦!我夫心疼病症,几个日子便把命丢了。撇得奴好苦!”这何九一面上上下下看了婆娘的模样,心里暗道:“我从来只听得人说武大娘子,不曾认得他。原来武大郎讨得这个老婆在屋里。西门庆这十两银子使着了!”一面走向灵前,看武大尸首。阴阳宣念经毕,揭起千秋幡,扯开白绢,定睛看时,见武大指甲青,唇口紫,面皮黄,眼皆突出,就知是中恶。旁边那两个火家说道:“怎的脸也紫了,口唇上有牙痕,口中出血?”何九道:“休得胡说!两日天气十分炎热,如何不走动些!”一面七手八脚葫芦提殓了,装入棺材内,两下用长命钉钉了。王婆一力撺掇,拿出一吊钱来与何九,打发众火家去了,就问:“几时出去?”王婆道:“大娘子说只三日便出殡,城外烧化。”何九也便起身。那妇人当夜摆着酒请人,第二日请四个僧念经。第三日早五更,众火家都来扛抬棺材,也有几个邻舍街坊,吊孝相送。那妇人带上孝,坐了一乘轿子,一路上口内假哭“养家人”。来到城外化人场上,便教举火烧化棺材。不一时烧得干干净净,把骨殖撒在池子里,原来斋堂管待,一应都是西门庆出钱整顿。

那妇人归到家中,楼上设个灵牌,上写“亡夫武大郎之灵”。灵床子前点一盏琉璃灯,里面贴些经幡钱纸、金银锭之类。那日却和西门庆做一处,打发王婆家去,二人在楼上任意纵横取乐,不比先前在王婆家茶房里,只是偷鸡盗狗之欢。如今武大已死,家中无人,两个肆意停眠整宿。初时西门庆恐邻舍瞧破,先到王婆那边坐一回,落后带着小厮竟从妇人家后门而入。自此和妇人情沾意密,常时三五夜不归去,把家中大小丢得七颠八倒,都不欢喜。正是:

\[
色胆如天不自由,情深意密两绸缪。
贪欢不管生和死,溺爱谁将身体修。
只为恩深情郁郁,多因爱阔恨悠悠。
要将吴越冤仇解,地老天荒难歇休。
\]

光阴迅速,日月如梭,西门庆刮剌那妇人将两月有余。一日,将近端阳佳节,但见:

\[
绿杨袅袅垂丝碧,海榴点点胭脂赤。
微微风动幔,飒飒凉侵扇。
处处过端阳,家家共举觞。
\]

却说西门庆自岳庙上回来,到王婆茶坊里坐下。那婆子连忙点一盏茶来,便问:“大官人往那里来?怎的不过去看看大娘子?”西门庆道:“今日往庙上走走。大节间记挂着,来看看六姐。”婆子道:“今日他娘潘妈妈在这里,怕还未去哩。等我过去看看,回大官人。”这婆子走过妇人后门看时,妇人正陪潘妈妈在房里吃酒,见婆子来,连忙让坐。妇人笑道:“干娘来得正好,请陪俺娘且吃个进门盏儿,到明日养个好娃娃!”婆子笑道:“老身又没有老伴儿,那里得养出来?你年小少壮,正好养哩!”妇人道:“常言小花不结老花儿结。”婆子便看着潘妈妈嘈道:“你看你女儿,这等伤我,说我是老花子。到明日还用着我老花子哩!”说罢,潘妈道:“他从小是这等快嘴,干娘休要和他一般见识。”王婆道:“你家这姐姐,端的百伶百俐,不枉了好个妇女。到明日,不知什么有福的人受的他起。”潘妈妈道:“干娘既是撮合山,全靠干娘作成则个!”一面安下钟箸,妇人斟酒在他面前。婆子一连陪了几杯酒,吃得脸红红的,又怕西门庆在那边等候,连忙丢了个眼色与妇人,告辞归家。妇人知西门庆来了,因一力撺掇他娘起身去了。将房中收拾干净,烧些异香,从新把娘吃的残馔撇去,另安排一席齐整酒肴预备。

西门庆从后门过来,妇人接着到房中,道个万福坐下。原来妇人自从武大死后,怎肯带孝!把武大灵牌丢在一边,用一张白纸蒙着,羹饭也不揪采。每日只是浓妆艳抹,穿颜色衣服,打扮娇样。因见西门庆两日不来,就骂:“负心的贼,如何撇闪了奴,又往那家另续上心甜的了?把奴冷丢,不来揪采。”西门庆道:“这两日有些事,今日往庙上去,替你置了些首饰珠翠衣服之类。”那妇人满心欢喜。西门庆一面唤过小厮玳安来,毡包内取出,一件件把与妇人。妇人方才拜谢收了。小女迎儿,寻常被妇人打怕的,以此不瞒他,令他拿茶与西门庆吃。一面妇人安放桌儿,陪西门庆吃茶。西门庆道:“你不消费心,我已与了干娘银子买东西去了。大节间,正要和你坐一坐。”妇人道:“此是待俺娘的,奴存下这桌整菜儿。等到干娘买来,且有一回耽搁,咱且吃着。”妇人陪西门庆脸儿相贴,腿儿相压,并肩一处饮酒。

且说婆子提着个篮儿,走到街上打酒买肉。那时正值五月初旬天气,大雨时行。只见红日当天,忽被黑云遮掩,俄而大雨倾盆。但见:

\[
乌云生四野,黑雾锁长空。刷剌剌漫空障日飞来,一点点击得芭蕉声碎。狂风相助,侵天老桧掀翻;霹雳交加,泰华嵩乔震动。洗炎驱暑,润泽田苗。
\]
正是:

\[
江淮河济添新水,翠竹红榴洗濯清。
\]
那婆子正打了一瓶酒,买了一篮菜蔬果品之类,在街上遇见这大雨,慌忙躲在人家房檐下,用手帕裹着头,把衣服都淋湿了。等了一歇,那雨脚慢了些,大步云飞来家。进入门来,把酒肉放在厨房下,走进房来,看妇人和西门庆饮酒,笑嘻嘻道:“大官人和大娘子好饮酒!你看把婆子身上衣服都淋湿了,到明日就教大官人赔我!”西门庆道:“你看老婆子,就是个赖精。”婆子道:“也不是赖精,大官人少不得赔我一匹大海青。”妇人道:“干娘,你且饮盏热酒儿。”那婆子陪着饮了三杯,说道:“老身往厨下烘衣裳去也。”一面走到厨下,把衣服烘干,那鸡鹅嗄饭切割安排停当,用盘碟盛了果品之类,都摆在房中,烫上酒来。西门庆与妇人重斟美酒,交杯叠股而饮。西门庆饮酒中间,看见妇人壁上挂着一面琵琶,便道:“久闻你善弹,今日好夕弹个曲儿我下酒。”妇人笑道:“奴自幼粗学一两句,不十分好,你却休要笑耻。”西门庆一面取下琵琶来,搂妇人在怀,看着他放在膝儿上,轻舒玉笋,款弄冰弦,慢慢弹着,低声唱道:

\[
冠儿不带懒梳妆,髻挽青丝云鬓光,金钗斜插在乌云上。唤梅香,开笼箱,穿一套素缟衣裳,打扮的是西施模样。出绣房,梅香,你与我卷起帘儿,烧一炷儿夜香。
\]
西门庆听了,欢喜的没入脚处,一手搂过妇人粉颈来,就亲了个嘴,称夸道:“谁知姐姐有这段儿聪明!就是小人在构栏三街两巷相交唱的,也没你这手好弹唱!”妇人笑道:“蒙官人抬举,奴今日与你百依百顺,是必过后休忘了奴家。”西门庆一面捧着他香腮,说道:“我怎肯忘了姐姐!”两个殢雨尤云,调笑玩耍。少顷,西门庆又脱下他一只绣花鞋儿,擎在手内,放一小杯酒在内,吃鞋杯耍子。妇人道:“奴家好小脚儿,你休要笑话。”不一时,二人吃得酒浓,掩闭了房门,解衣上床玩耍。王婆把大门顶着,和迎儿在厨房中坐地。二人在房内颠鸾倒凤,似水如鱼。那妇人枕边风月,比娼妓尤甚,百般奉承。西门庆亦施逞枪法打动。两个女貌郎才,俱在妙龄之际。

\[
寂静兰房簟枕凉,佳人才子意何长。
方才枕上浇红烛,忽又偷来火隔墙。
粉蝶探香花萼颤,蜻蜓戏水往来狂。
情浓乐极犹余兴,珍重檀郎莫相忘。
\]

当日西门庆在妇人家盘桓至晚,欲回家,留了几两散碎银子与妇人做盘缠。妇人再三挽留不住。西门庆带上眼罩,出门去了。妇人下了帘子,关上大门,又和王婆吃了一回酒,才散。正是:

\[
倚门相送刘郎去,烟水桃花去路迷。
\]

\newpage
%# -*- coding:utf-8 -*-
%%%%%%%%%%%%%%%%%%%%%%%%%%%%%%%%%%%%%%%%%%%%%%%%%%%%%%%%%%%%%%%%%%%%%%%%%%%%%%%%%%%%%


\chapter{薛媒婆说娶孟三儿\KG 杨姑娘气骂张四舅}


诗曰:

\[
我做媒人实自能,全凭两腿走殷勤。
唇枪惯把鳏男配,舌剑能调烈女心。
利市花常头上带,喜筵饼锭袖中撑。
只有一件不堪处,半是成人半败人。
\]

话说西门庆家中一个卖翠花的薛嫂儿,提着花厢儿,一地里寻西门庆不着。因见西门庆贴身使的小厮玳安儿,便问道:“大官人在那里?”玳安道:“俺爹在铺子里和傅二叔算帐。”原来西门庆家开生药铺,主管姓傅名铭,字自新,排行第二,因此呼他做傅二叔。这薛嫂听了,一直走到铺子门首,掀开帘子,见西门庆正与主管算帐,便点点头儿,唤他出来。西门庆见是薛嫂儿,连忙撇了主管出来,两人走在僻静处说话。西门庆问道:“有甚话说?”薛嫂道:“我有一件亲事,来对大官人说,管情中你老人家意,就顶死了的三娘的窝儿,何如?”西门庆道:“你且说这件亲事是那家的?”薛嫂道:“这位娘子,说起来你老人家也知道,就是南门外贩布杨家的正头娘子。手里有一分好钱。南京拔步床也有两张。四季衣服,插不下手去,也有四五只箱子。金镯银钏不消说,手里现银子也有上千两,好三梭布也有三二百筒。不料他男子汉去贩布,死在外边。他守寡了一年多,身边又没子女,止有一个小叔儿,才十岁。青春年少,守他什么!有他家一个嫡亲姑娘,要主张着他嫁人。这娘子今年不上二十五六岁,生的长挑身材,一表人物,打扮起来就是个灯人儿。风流俊俏,百伶百俐,当家立纪、针指女工、双陆棋子不消说。不瞒大官人说,他娘家姓孟,排行三姐,就住在臭水巷。又会弹一手好月琴,大官人若见了,管情一箭就上垛。”西门庆听见妇人会弹月琴,便可在他心上,就问薛嫂儿:“既是这等,几时相会看去?”薛嫂道:“相看到不打紧。我且和你老人家计议:如今他家一家子,只是姑娘大。虽是他娘舅张四,山核桃——差着一槅哩。这婆子原嫁与北边半边街徐公公房子里住的孙歪头。歪头死了,这婆子守寡了三四十年,男花女花都无,只靠侄男侄女养活。大官人只倒在他身上求他。这婆子爱的是钱财,明知侄儿媳妇有东西,随问什么人家他也不管,只指望要几两银子。大官人家里有的是那嚣段子,拿一段,买上一担礼物,明日亲去见他,再许他几两银子,一拳打倒他。随问旁边有人说话,这婆子一力张主,谁敢怎的!”这薛嫂儿一席话,说的西门庆欢从额角眉尖出,喜向腮边笑脸生。正是:

\[
媒妁殷勤说始终,孟姬爱嫁富家翁。
有缘千里能相会,无缘对面不相逢。
\]

西门庆当日与薛嫂相约下了,明日是好日期,就买礼往他姑娘家去。薛嫂说毕话,提着花厢儿去了。西门庆进来和傅伙计算帐。一宿晚景不题。

到次日,西门庆早起,打选衣帽整齐,拿了一段尺头,买了四盘羹果,装做一盒担,叫人抬了。薛嫂领着,西门庆骑着头口,小厮跟随,迳来杨姑娘家门首。薛嫂先入去通报姑娘,说道:“近边一个财主,要和大娘子说亲。我说一家只姑奶奶是大,先来觌面,亲见过你老人家,讲了话,然后才敢去门外相看。今日小媳妇领来,见在门首伺候。”婆子听见,便道:“阿呀,保山,你如何不先来说声!”一面吩咐丫鬟顿下好茶,一面道:“有请。”这薛嫂一力撺掇,先把盒担抬进去摆下,打发空盒担出去,就请西门庆进来相见。这西门庆头戴缠综大帽,一口一声只叫:“姑娘请受礼。”让了半日,婆子受了半礼。分宾主坐下,薛嫂在旁边打横。婆子便道:“大官人贵姓?”薛嫂道:“便是咱清河县数一数二的财主,西门大官人。在县前开个大生药铺,家中钱过北斗,米烂陈仓,没个当家立纪的娘子。闻得咱家门外大娘子要嫁,特来见姑奶奶讲说亲事。”婆子道:“官人傥然要说俺侄儿媳妇,自恁来闲讲罢了,何必费烦又买礼来,使老身却之不恭,受之有愧。”西门庆道:“姑娘在上,没的礼物,惶恐。”那婆子一面拜了两拜谢了,收过礼物去,拿茶上来。吃毕,婆子开口道:“老身当言不言谓之懦。我侄儿在时,挣了一分钱财,不幸先死了,如今都落在他手里,说少也有上千两银子东西。官人做小做大我不管你,只要与我侄儿念上个好经。老身便是他亲姑娘,又不隔从,就与上我一个棺材本,也不曾要了你家的。我破着老脸,和张四那老狗做臭毛鼠,替你两个硬张主。娶过门时,遇生辰时节,官人放他来走走,就认俺这门穷亲戚,也不过上你穷。”西门庆笑道:“你老人家放心,所说的话,我小人都知道了。只要你老人家主张得定,休说一个棺材本,就是十个,小人也来得起。”说着,便叫小厮拿过拜匣来,取出六锭三十两雪花官银,放在面前,说道:“这个不当甚么,先与你老人家买盏茶吃,到明日娶过门时,还你七十两银子、两匹缎子,与你老人家为送终之资。其四时八节,只管上门行走。”这老虔婆黑眼珠见了二三十两白晃晃的官银,满面堆下笑来,说道:“官人在上,不是老身意小,自古先断后不乱。”薛嫂在旁插口说:“你老人家忒多心,那里这等计较!我这大官人不是这等人,只恁还要掇着盒儿认亲。你老人家不知,如今知县知府相公也都来往,好不四海。你老人家能吃他多少?”一席话说的婆子屁滚尿流。吃了两道茶,西门庆便要起身,婆子挽留不住。薛嫂道:“今日既见了姑奶奶,明日便好往门外相看。”婆子道:“我家侄儿媳妇不用大官人相,保山,你就说我说,不嫁这样人家,再嫁甚样人家!”西门庆作辞起身。婆子道:“老身不知大官人下降,匆忙不曾预备,空了官人,休怪。”拄拐送出。送了两步,西门庆让回去了。薛嫂打发西门庆上马,因说道:“我主张的有理么?你老人家先回去罢,我还在这里和他说句话。明日须早些往门外去。”西门庆便拿出一两银子来,与薛嫂做驴子钱。薛嫂接了,西门庆便上马来家。他还在杨姑娘家说话饮酒,到日暮才归家去。

话休饶舌。到次日,西门庆打选衣帽齐整,袖着插戴,骑着匹白马,玳安、平安两个小厮跟随,薛嫂儿骑着驴子,出的南门外来。不多时,到了杨家门首。却是坐南朝北一间门楼,粉青照壁。薛嫂请西门庆下了马,同进去。里面仪门照墙,竹抢篱影壁,院内摆设榴树盆景,台基上靛缸一溜,打布凳两条。薛嫂推开朱红槅扇,三间倒坐客位,上下椅桌光鲜,帘栊潇洒。薛嫂请西门庆坐了,一面走入里边。片晌出来,向西门庆耳边说:“大娘子梳妆未了,你老人家请坐一坐。”只见一个小厮儿拿出一盏福仁泡茶来,西门庆吃了。这薛嫂一面指手画脚与西门庆说:“这家中除了那头姑娘,只这位娘子是大。虽有他小叔,还小哩,不晓得什么。当初有过世的官人在铺子里,一日不算银子,铜钱也卖两大箥箩。毛青鞋面布,俺每问他买,定要三分一尺。一日常有二三十染的吃饭,都是这位娘子主张整理。手下使着两个丫头,一个小厮。大丫头十五岁,吊起头去了,名唤兰香。小丫头名唤小鸾,才十二岁。到明日过门时,都跟他来。我替你老人家说成这亲事,指望典两间房儿住哩。”西门庆道:“这不打紧。”薛嫂道:“你老人家去年买春梅,许我几匹大布,还没与我。到明日不管一总谢罢了。”

正说着,只见使了个丫头来叫薛嫂。不多时,只闻环佩叮咚,兰麝馥郁,薛嫂忙掀开帘子,妇人出来。西门庆睁眼观那妇人,但见:

\[
月画烟描,粉妆玉琢。俊庞儿不肥不瘦,俏身材难减难增。素额逗几点微麻,天然美丽;缃裙露一双小脚,周正堪怜。行过处花香细生,坐下时淹然百媚。
\]
西门庆一见满心欢喜。妇人走到堂下,望上不端不正道了个万福,就在对面椅子上坐下。西门庆眼不转睛看了一回,妇人把头低了。西门庆开言说:“小人妻亡已久,欲娶娘子管理家事,未知尊意如何?”那妇人偷眼看西门庆,见他人物风流,心下已十分中意,遂转过脸来,问薛婆道:“官人贵庚?没了娘子多少时了?”西门庆道:“小人虚度二十八岁,不幸先妻没了一年有余。不敢请问,娘子青春多少?”妇人道:“奴家是三十岁。”西门庆道:“原来长我二岁。”薛嫂在旁插口道:“妻大两,黄金日日长。妻大三,黄金积如山。”说着,只见小丫鬟拿出三盏蜜饯金橙子泡茶来。妇人起身,先取头一盏,用纤手抹去盏边水渍,递与西门庆,道个万福。薛嫂见妇人立起身,就趁空儿轻轻用手掀起妇人裙子来,正露出一对刚三寸、恰半叉、尖尖趫趫金莲脚来,穿着双大红遍地金云头白绫高低鞋儿。西门庆看了,满心欢喜。妇人取第二盏茶来递与薛嫂。他自取一盏陪坐。吃了茶,西门庆便叫玳安用方盒呈上锦帕二方、宝钗一对、金戒指六个,放在托盘内送过去。薛嫂一面叫妇人拜谢了。因问官人行礼日期:“奴这里好做预备。”西门庆道:“既蒙娘子见允,今月二十四日,有些微礼过门来。六月初二准娶。”妇人道:“既然如此,奴明日就使人对姑娘说去。”薛嫂道:“大官人昨日已到姑奶奶府上讲过话了。”妇人道:“姑娘说甚来?”薛嫂道:“姑奶奶听见大官人说此椿事,好不喜欢!说道,不嫁这等人家,再嫁那样人家!我就做硬主媒,保这门亲事。”妇人道:“既是姑娘恁般说,又好了。”薛嫂道:“好大娘子,莫不俺做媒敢这等捣谎。”说毕,西门庆作辞起身。

薛嫂送出巷口,向西门庆说道:“看了这娘子,你老人家心下如何?”西门庆道:“薛嫂,其实累了你。”薛嫂道:“你老人家先行一步,我和大娘子说句话就来。”西门庆骑马进城去了。薛嫂转来向妇人说道:“娘子,你嫁得这位官人也罢了。”妇人道:“但不知房里有人没有人?见作何生理?”薛嫂道:“好奶奶,就有房里人,那个是成头脑的?我说是谎,你过去就看出来。他老人家名目,谁不知道,清河县数一数二的财主,有名卖生药放官吏债西门庆大官人。知县知府都和他来往。近日又与东京杨提督结亲,都是四门亲家,谁人敢惹他!”妇人安排酒饭,与薛嫂儿正吃着,只见他姑娘家使个小厮安童,盒子里盛着四块黄米面枣儿糕、两块糖、几十个艾窝窝,就来问:“曾受了那人家插定不曾?奶奶说来:这人家不嫁,待嫁甚人家。”妇人道:“多谢你奶奶挂心。今已留下插定了。”薛嫂道:“天么,天么!早是俺媒人不说谎,姑奶奶早说将来了。”妇人收了糕,取出盒子,装了满满一盒子点心腊肉,又与了安童五六十文钱,说:“到家多拜上奶奶。那家日子定在二十四日行礼,出月初二日准娶。”小厮去了。薛嫂道:“姑奶奶家送来什么?与我些,包了家去孩子吃。”妇人与了他一块糖、十个艾窝窝,方才出门,不在话下。

且说他母舅张四,倚着他小外甥杨宗保,要图留妇人东西,一心举保大街坊尚推官儿子尚举人为继室。若小可人家,还有话说,不想闻得是西门庆定了,知他是把持官府的人,遂动不得了。寻思千方百计,不如破为上计。即走来对妇人说:“娘子不该接西门庆插定,还依我嫁尚举人的是。他是诗礼人家,又有庄田地土,颇过得日子,强如嫁西门庆。那厮积年把持官府,刁徒泼皮。他家见有正头娘子,乃是吴千户家女儿,你过去做大是,做小是?况他房里又有三四个老婆,除没上头的丫头不算。你到他家,人多口多,还有的惹气哩!”妇人听见话头,明知张四是破亲之意,便佯说道:“自古船多不碍路。若他家有大娘子,我情愿让他做姐姐。虽然房里人多,只要丈夫作主,若是丈夫喜欢,多亦何妨。丈夫若不喜欢,便只奴一个也难过日子。况且富贵人家,那家没有四五个?你老人家不消多虑,奴过去自有道理,料不妨事。”张四道:“不独这一件。他最惯打妇煞妻,又管挑贩人口,稍不中意,就令媒婆卖了。你受得他这气么?”妇人道:“四舅,你老人家差矣。男子汉虽利害,不打那勤谨省事之妻。我到他家,把得家定,里言不出,外言不入,他敢怎的奴?”张四道:“不是我打听的,他家还有一个十四岁未出嫁的闺女,诚恐去到他家,三窝两块惹气怎了?”妇人道:“四舅说那里话,奴到他家,大是大,小是小,待得孩儿们好,不怕男子汉不欢喜,不怕女儿们不孝顺。休说一个,便是十个也不妨事。”张四道:“还有一件最要紧的事,此人行止欠端,专一在外眠花卧柳。又里虚外实,少人家债负。只怕坑陷了你。”妇人道:“四舅,你老人家又差矣。他少年人,就外边做些风流勾当,也是常事。奴妇人家,那里管得许多?惹说虚实,常言道:世上钱财傥来物,那是长贫久富家?况姻缘事皆前生分定,你老人家到不消这样费心。”张四见说不动妇人,到吃他抢白了几句,好无颜色,吃了两盏清茶,起身去了。有诗为证:

\[
张四无端散楚言,姻缘谁想是前缘。
佳人心爱西门庆,说破咽喉总是闲。
\]
张四羞惭归家,与婆子商议,单等妇人起身,指着外甥杨宗保,要拦夺妇人箱笼。

话休饶舌。到二十四日,西门庆行了礼。到二十六日,请十二位素僧念经烧灵,都是他姑娘一力张主。张四到妇人将起身头一日,请了几位街坊众邻,来和妇人说话。此时薛嫂正引着西门庆家小厮伴当,并守备府里讨的一二十名军牢,正进来搬抬妇人床帐、嫁妆箱笼。被张四拦住说道:“保山且休抬!有话讲。”一面同了街坊邻舍进来见妇人。坐下,张四先开言说:“列位高邻听着:大娘子在这里,不该我张龙说,你家男子汉杨宗锡与你这小叔杨宗保,都是我甥。今日不幸大外甥死了,空挣一场钱。有人主张着你,这也罢了。争奈第二个外甥杨宗保年幼,一个业障都在我身上。他是你男子汉一母同胞所生,莫不家当没他的份儿?今日对着列位高邻在这里,只把你箱笼打开,眼同众人看一看,有东西没东西,大家见个明白。”妇人听言,一面哭起来,说道:“众位听着,你老人家差矣!奴不是歹意谋死了男子汉,今日添羞脸又嫁人。他手里有钱没钱,人所共知,就是积攒了几两银子,都使在这房子上。房子我没带去,都留与小叔。家活等件,分毫不动。就是外边有三四百两银子欠帐,文书合同已都交与你老人家,陆续讨来家中盘缠。再有甚么银两来?”张四道:“你没银两也罢。如今只对着众位打开箱笼看一看。就有,你还拿了去,我又不要你的。”妇人道:“莫不奴的鞋脚也要瞧不成?”正乱着,只姑娘拄拐自后而出。众人便道:“姑娘出来。”都齐声唱喏。姑娘还了万福,陪众人坐下。姑娘开口道:“列位高邻在上,我是他是亲姑娘,又不隔从,莫不没我说处?死了的也是侄儿,活着的也是侄儿,十个指头咬着都疼。如今休说他男子汉手里没钱,他就有十万两银子,你只好看他一眼罢了。他身边又无出,少女嫩妇的,你拦着不教他嫁人做什么?”众街邻高声道:“姑娘见得有理!”婆子道:“难道他娘家陪的东西,也留下他的不成?他背地又不曾自与我什么,说我护他,也要公道。不瞒列位说,我这侄儿媳妇平日有仁义,老身舍不得他,好温克性儿。不然,老身管着他。”那张四在旁,把婆子瞅了一眼,说道:“你好公平心儿!凤凰无宝处不落。”只这一句话道着婆子真病,登时怒起,紫涨了面皮,指定张四大骂道:“张四,你休胡言乱语!我虽不能是杨家正头香主,你这老油嘴,是杨家那膫子\textuni{34B2}的?”张四道:“我虽是异姓,两个外甥是我姐姐养的,你这老咬虫,女生外向,怎一头放火,又一头放水?”姑娘道:“贱没廉耻老狗骨头!他少女嫩妇的,你留他在屋里,有何算计?既不是图色欲,便欲起谋心,将钱肥己。”张四道:“我不是图钱,只恐杨宗保后来大了,过不得日子。不似你这老杀才,搬着大引着小,黄猫儿黑尾。”姑娘道:“张四,你这老花根,老奴才,老粉嘴,你恁骗口张舌的好淡扯,到明日死了时,不使了绳子扛子。”张四道:“你这嚼舌头老淫妇,挣将钱来焦尾靶,怪不得你无儿无女。”姑娘急了,骂道:“张四,贼老苍根,老猪狗,我无儿无女,强似你家妈妈子穿寺院,养和尚,\textuni{34B2}道士,你还在睡梦里。”当下两个差些儿不曾打起来,多亏众邻舍劝住,说道:“老舅,你让姑娘一句儿罢。”薛嫂儿见他二人嚷做一团,领西门庆家小厮伴当,并发来众军牢,赶人闹里,七手八脚将妇人床帐、妆奁、箱笼,扛的扛,抬的抬,一阵风都搬去了。那张四气的眼大睁着,半晌说不出话来。众邻舍见不是事,安抚了一回,各人都散了。

到六月初二日,西门庆一顶大轿,四对红纱灯笼,他小叔杨宗保头上扎着髻儿,穿着青纱衣,撒骑在马上,送他嫂子成亲。西门庆答贺了他一匹锦缎、一柄玉绦儿。兰香、小鸾两个丫头,都跟了来铺床叠被。小厮琴童方年十五岁,亦带过来伏侍。到三日,杨姑娘家并妇人两个嫂子孟大嫂、二嫂都来做生日。西门庆与他杨姑娘七十两银子、两匹尺头。自此亲戚来往不绝。西门庆就把西厢房里收拾三间,与他做房。排行第三,号玉楼,令家中大小都随着叫三姨。到晚一连在他房中歇了三夜。正是:销金帐里,依然两个新人;红锦被中,现出两般旧物。有诗为证:

\[
怎睹多情风月标,教人无福也难消。
风吹列子归何处,夜夜婵娟在柳梢。
\]

\newpage
%# -*- coding:utf-8 -*-
%%%%%%%%%%%%%%%%%%%%%%%%%%%%%%%%%%%%%%%%%%%%%%%%%%%%%%%%%%%%%%%%%%%%%%%%%%%%%%%%%%%%%


\chapter{盼情郎佳人占鬼卦\KG 烧夫灵和尚听淫声}


词曰:

\[
红曙卷窗纱,睡起半拖罗袂。何似等闲睡起,到日高还未。催花阵阵玉楼风,楼上人难睡。有了人儿一个,在眼前心里。
\]

话说西门庆自娶了玉楼在家,燕尔新婚,如胶似漆。又遇陈宅使文嫂儿来通信,六月十二日就要娶大姐过门。西门庆促忙促急攒造不出床来,就把孟玉楼陪来的一张南京描金彩漆拔步床陪了大姐。三朝九日,足乱了一个多月,不曾往潘金莲家去。把那妇人每日门儿倚遍,眼儿望穿。使王婆往他门首去寻,门首小厮知道是潘金莲使来的,多不理他。妇人盼的紧,见婆子回了,又叫小女儿街上去寻。那小妮子怎敢入他深宅大院?只在门首踅探,不见西门庆就回来了。来家被妇人哕骂在脸上,怪他没用,便要叫他跪着。饿到晌午,又不与他饭吃。此时正值三伏天道,妇人害热,分付迎儿热下水,伺候要洗澡。又做了一笼裹馅肉角儿,等西门庆来吃。身上只着薄纱短衫,坐在小凳上,盼不见西门庆到来,骂了几句负心贼。无情无绪,用纤手向脚上脱下两只红绣鞋儿来,试打一个相思卦。正是:

\[
逢人不敢高声语,暗卜金钱问远人。
\]
有《山坡羊》为证:

\[
凌波罗袜,天然生下,红云染就相思卦。似藕生芽,如莲卸花,怎生缠得些儿大!柳条儿比来刚半叉。他不念咱,咱何曾不念他!倚着门儿,私下帘儿,悄呀,空叫奴被儿里叫着他那名儿骂。你怎恋烟花,不来我家!奴眉儿淡淡教谁画?何处绿杨拴系马?他辜负咱,咱何曾辜负他!
\]
妇人打了一回相思卦,不觉困倦,就歪在床上盹睡着了。约一个时辰醒来,心中正没好气。迎儿问:“热了水,娘洗澡也不洗?”妇人就问:“角儿蒸熟了?拿来我看。”迎儿连忙拿到房中。妇人用纤手一数,原做下一扇笼三十个角儿,翻来复去只数得二十九个,便问:“那一个往那里去了?”迎儿道:“我并没看见,只怕娘错数了。”妇人道:“我亲数了两遍,三十个角儿,要等你爹来吃。你如何偷吃了一个?好娇态淫妇奴才,你害馋痨馋痞,心里要想这个角儿吃!你大碗小碗吃捣不下饭去,我做下孝顺你来!”便不由分说,把这小妮子跣剥去身上衣服,拿马鞭子打了二三十下,打的妮子杀猪般也似叫。问着他:“你不承认,我定打你百数!”打的妮子急了,说道:“娘休打,是我害饿的慌,偷吃了一个。”妇人道:“你偷了,如何赖我错数?眼看着就是个牢头祸根淫妇!有那亡八在时,轻学重告,今日往那里去了?还在我跟前弄神弄鬼!我只把你这牢头淫妇,打下你下截来!”打了一回,穿上小衣,放他起来,分付在旁打扇。打了一回扇,口中说道:“贼淫妇,你舒过脸来,等我掐你这皮脸两下子。”那妮子真个舒着脸,被妇人尖指甲掐了两道血口子,才饶了他。

良久,走到镜台前,从新妆点出来,门帘下站立。也是天假其便,只见玳安夹着毡包,骑着马,打妇人门首经过。妇人叫住,问他往何处去来。那小厮说话乖觉,常跟西门庆在妇人家行走,妇人常与他些浸润,以此滑熟。一面下马来,说道:“俺爹使我送人情,往守备府里去来。”妇人叫进门来,问道:“你爹家中有甚事,如何一向不来傍个影儿?想必另续上了一个心甜的姊妹了。”玳安道:“俺爹再没续上姊妹,只是这几日家中事忙,不得脱身来看六姨。”妇人道:“就是家中有事,那里丢我恁个半月,音信不送一个儿!只是不放在心儿上。”因问玳安:“有甚么事?你对我说。”那小厮嘻嘻只是笑,不肯说。妇人见玳安笑得有因,愈丁紧问道:“端的有甚事?”玳安笑道:“只说有椿事儿罢了,六姨只顾吹毛求疵问怎的?”妇人道:“好小油嘴儿,你不对我说,我就恼你一生。”小厮道:“我对六姨说,六姨休对爹说是我说的。”妇人道:“我决不对他说。”玳安就如此这般,把家中娶孟玉楼之事,从头至尾告诉了一遍。这妇人不听便罢,听了由不得珠泪儿顺着香腮流将下来。玳安慌了,便道:“六姨,你原来这等量窄,我故此不对你说。”妇人倚定门儿,长叹了一口气,说道:“玳安,你不知道,我与他从前以往那样恩情,今日如何一旦抛闪了。”止不住纷纷落下泪来。玳安道:“六姨,你何苦如此?家中俺娘也不管着他。”妇人便道:“玳安,你听告诉:乔才心邪,不来一月。奴绣鸳衾旷了三十夜。他俏心儿别,俺痴心儿呆,不合将人十分热。常言道容易得来容易舍。兴,过也;缘,分也。”说毕又哭。玳安道:“六姨,你休哭。俺爹怕不也只在这两日,他生日待来也。你写几个字儿,等我替你捎去,与俺爹看了,必然就来。”妇人道:“是必累你,请的他来。到明日,我做双好鞋与你穿。我这里也要等他来,与他上寿哩。他若不来,都在你小油嘴身上。”说毕,令迎儿把桌上蒸下的角儿,装了一碟,打发玳安儿吃茶。一面走入房中,取过一幅花笺,又轻拈玉管,款弄羊毛,须臾,写了一首《寄生草》。词曰:

\[
将奴这知心话,付花笺寄与他。想当初结下青丝发,门儿倚遍帘儿下,受了些没打弄的耽惊怕。你今果是负了奴心,不来还我香罗帕。
\]
写就,叠成一个方胜儿,封停当,付与玳安收了,道:“好歹多上覆他。待他生日,千万来走走。奴这里专望。”那玳安吃了点心,妇人又与数十文钱。临出门上马,妇人道:“你到家见你爹,就说六姨好不骂你。他若不来,你就说六姨到明日坐轿子亲自来哩。”玳安道:“六姨,自吃你卖粉团的撞见了敲板儿蛮子叫冤屈——麻饭胳胆的帐。”说毕,骑马去了。

那妇人每日长等短等,如石沉大海。七月将尽,到了他生辰。这妇人挨一日似三秋,盼一夜如半夏,等得杳无音信。不觉银牙暗咬,星眼流波。至晚,只得又叫王婆来,安排酒肉与他吃了,向头上拔下一根金头银簪子与他,央往西门庆家去请他来。王婆道:“这早晚,茶前酒后,他定也不来。待老身明日侵早请他去罢。”妇人道:“干娘,是必记心,休要忘了!”婆子道:“老身管着那一门儿,肯误了勾当?”这婆子非钱而不行,得了这根簪子,吃得脸红红,归家去了。且说妇人在房中,香薰鸳被,款剔银灯,睡不着,短叹长吁。正是:

\[
得多少琵琶夜久殷勤弄,寂寞空房不忍弹。
\]
于是独自弹着琵琶,唱一个《绵搭絮》:

\[
谁想你另有了裙钗,气的奴似醉如痴,斜倚定帏屏故意儿猜,不明白。怎生丢开?传书寄柬,你又不来。你若负了奴的恩情,人不为仇天降灾。
\]

妇人一夜翻来覆去,不曾睡着。巴到天明,就使迎儿:“过间壁瞧王奶奶请你爹去了不曾?”迎儿去不多时,说:“王奶奶老早就出去了。”

且说那婆子早晨出门,来到西门庆门首探问,都说不知道。在对门墙脚下等勾多时,只见傅伙计来开铺子。婆子走向前,道了万福:“动问一声,大官人在家么?”傅伙计道:“你老人家寻他怎的?早是问着我,第二个也不知他。大官人昨日寿诞,在家请客,吃了一日酒,到晚拉众朋友往院里去了,一夜通没回家。你往那里去寻他!”这婆子拜辞,出县前来到东街口,正往勾栏那条巷去。只见西门庆骑着马远远从东来,两个小厮跟随,此时宿酒未醒,醉眼摩娑,前合后仰。被婆子高声叫道:“大官人,少吃些儿怎的!”向前一把手把马嚼环扯住。西门庆醉中问道:“你是王干娘,你来想是六姐寻我?”那婆子向他耳畔低言。道不数句,西门庆道:“小厮来家对我说来,我知道六姐恼我哩,我如今就去。”那西门庆一面跟着他,两个一递一句,整说了一路话。

比及到妇人门首,婆子先入去,报道:“大娘子恭喜,还亏老身,没半个时辰,把大官人请将来了。”妇人听见他来,就象天上掉下来的一般,连忙出房来迎接。西门庆摇着扇儿进来,带酒半酣,与妇人唱喏。妇人还了万福,说道:“大官人,贵人稀见面!怎的把奴丢了,一向不来傍个影儿?家中新娘子陪伴,如胶似漆,那里想起奴家来!”西门庆道:“你休听人胡说,那讨什么新娘子来!因小女出嫁,忙了几日,不曾得闲工夫来看你。”妇人道:“你还哄我哩!你若不是怜新弃旧,另有别人,你指着旺跳身子说个誓,我方信你。”西门庆道:“我若负了你,生碗来大疔疮,害三五年黄病,匾担大蛆叮口袋。”妇人道:“负心的贼!匾担大蛆叮口袋,管你甚事?”一手向他头上把一顶新缨子瓦楞帽儿撮下来,望地上只一丢。慌的王婆地下拾起来,替他放在桌上,说道:“大娘子,只怪老身不去请大官人,来就是这般的。”妇人又向他头上拔下一根簪儿,拿在手里观看,却是一点油金簪儿,上面笈着两溜字儿:“金勒马嘶芳草地,玉楼人醉杏花天。”却是孟玉楼带来的。妇人猜做那个唱的送他的,夺了放在袖子里,说道:“你还不变心哩!奴与你的簪儿那里去了?”西门庆道:“你那根簪子,前日因酒醉跌下马来,把帽子落了,头发散开,寻时就不见了。”妇人将手在向西门庆脸边弹个响榧子,道:“哥哥儿,你醉的眼恁花了,哄三岁孩儿也不信!”王婆在傍插口道:“大娘子休怪!大官人,他离城四十里见蜜蜂儿刺屎,出门交獭象绊了一交,原来觑远不觑近。”西门庆道:“紧自他麻犯人,你又自作耍。”妇人见他手中拿着一把红骨细洒金、金钉铰川扇儿,取过来迎亮处只一照,原来妇人久惯知风月中事,见扇上多是牙咬的碎眼儿,就疑是那个妙人与他的。不由分说,两把折了。西门庆救时,已是扯的烂了,说道:“这扇子是我一个朋友卜志道送我的,一向藏着不曾用,今日才拿了三日,被你扯烂了。”

那妇人奚落了他一回,只见迎儿拿茶来,便叫迎儿放下茶托,与西门庆磕头。王婆道:“你两口子刮聒了这半日也勾了,休要误了勾当。老身厨下收拾去也。”妇人一边分付迎儿,将预先安排下与西门庆上寿的酒肴,整理停当,拿到房中,摆在桌上。妇人向箱中取出与西门庆上寿的物事,用盘盛着,摆在面前,与西门庆观看。却是一双玄色段子鞋;一双挑线香草边阑、松竹梅花岁寒三友酱色段子护膝;一条纱绿潞绸、水光绢里儿紫线带儿,里面装着排草玫瑰花兜肚;一根并头莲瓣簪儿。簪儿上笈着五言四句诗一首,云:“奴有并头莲,赠与君关髻。凡事同头上,切勿轻相弃。”西门庆一见满心欢喜,把妇人一手搂过,亲了个嘴,说道:“怎知你有如此聪慧!”妇人教迎儿执壶斟一杯与西门庆,花枝招扬,插烛也似磕了四个头。那西门庆连忙拖起来。两个并肩而坐,交杯换盏饮酒。那王婆陪着吃了几杯酒,吃的脸红红的,告辞回家去了。二人自在取乐玩耍。妇人陪伴西门庆饮酒多时,看看天色晚来,但见:

\[
密云迷晚岫,暗雾锁长空。群星与皓月争辉,绿水共青天同碧。僧投古寺,深林中嚷嚷鸦飞;客奔荒村,闾巷内汪汪犬吠。
\]

当下西门庆分付小厮回马家去,就在妇人家歇了。到晚夕,二人尽力盘桓,淫欲无度。

常言道:乐极生悲。光阴迅速,单表武松自领知县书礼驮担,离了清河县,竟到东京朱太尉处,下了书礼,交割了箱驮。等了几日,讨得回书,领一行人取路回山东而来。去时三四月天气,回来却淡暑新秋,路上雨水连绵,迟了日限。前后往回也有三个月光景。在路上行往坐卧,只觉得神思不安,身心恍惚,不免先差了一个土兵,预报与知县相公。又私自寄一封家书与他哥哥武大,说他只在八月内准还。那土兵先下了知县相公禀帖,然后迳来抓寻武大家。可可天假其便,王婆正在门首。那土兵见武大家门关着,才要叫门,婆子便问:“你是寻谁的?”土兵道:“我是武都头差来下书与他哥哥。”婆子道:“武大郎不在家,都上坟去了。你有书信,交与我,等他回来,我递与他,也是一般。”那土兵向前唱了一个喏,便向身边取出家书来交与王婆,忙忙骑上头口去了。

这王婆拿着那封书,从后门走过妇人家来。原来妇人和西门庆狂了半夜,约睡至饭时还不起来。王婆叫道:“大官人、娘子起来,和你们说话。如今武二差土兵寄书来与他哥哥,说他不久就到。我接下,打发他去了。你们不可迟滞,须要早作长便。”那西门庆不听万事皆休,听了此言,正是:分门八块顶梁骨,倾下半桶冰雪来。慌忙与妇人都起来,穿上衣服,请王婆到房内坐下。取出书来与西门庆看。书中写着,不过中秋回家。二人都慌了手脚,说道:“如此怎了?干娘遮藏我每则个,恩有重报,不敢有忘。我如今二人情深似海,不能相舍。武二那厮回来,便要分散,如何是好?”婆子道:“大官人,有什么难处之事!我前日已说过,幼嫁由亲,后嫁由身。古来叔嫂不通门户,如今武大已百日来到,大娘子请上几个和尚,把这灵牌子烧了。趁武二未到家,大官人一顶轿子娶了家去。等武二那厮回来,我自有话说。他敢怎的?自此你二人自在一生,岂不是妙!”西门庆便道:“干娘说的是。”当日西门庆和妇人用毕早饭,约定八月初六日,是武大百日,请僧烧灵。初八日晚,娶妇人家去。三人计议已定。不一时,玳安拿马来接回家,不在话下。

光阴似箭,日月如梭,又早到了八月初六日。西门庆拿了数两碎银钱,来妇人家,教王婆报恩寺请了六个僧,在家做水陆,超度武大,晚夕除灵。道人头五更就挑了经担来,铺陈道场,悬挂佛像。王婆伴厨子在灶上安排斋供。西门庆那日就在妇人家歇了。不一时,和尚来到,摇响灵杵,打动鼓钹,讽诵经忏,宣扬法事,不必细说。

且说潘金莲怎肯斋戒,陪伴西门庆睡到日头半天,还不起来。和尚请斋主拈香佥字,证盟礼佛,妇人方才起来梳洗,乔素打扮,来到佛前参拜。众和尚见了武大这老婆,一个个都迷了佛性禅心,关不住心猿意马,七颠八倒,酥成一块。但见:

\[
班首轻狂,念佛号不知颠倒;维摩昏乱,诵经言岂顾高低。烧香行者,推倒花瓶;秉烛头陀,误拿香盒。宣盟表白,大宋国错称做大唐国;忏罪阇黎,武大郎几念武大娘。长老心忙,打鼓借拿徒弟手;沙弥情荡,罄槌敲破老僧头。从前苦行一时休,万个金刚降不住。
\]
妇人在佛前烧了香,佥了字,拜礼佛毕,回房去依旧陪伴西门庆。摆上酒席荤腥,自去取乐。西门庆分付王婆:“有事你自答应便了,休教他来聒噪六姐。”婆子哈哈笑道:“你两口儿只管受用,由着老娘和那秃厮缠。”

且说从和尚见了武大老婆乔模乔样,多记在心里。到午斋往寺中歇晌回来,妇人正和西门庆在房里饮酒作欢。原来妇人卧房与佛堂止隔一道板壁。有一个僧人先到,走在妇人窗下水盆里洗手,忽听见妇人在房里颤声柔气,呻呻吟吟,哼哼唧唧,恰似有人交媾一般。遂推洗手,立住脚听。只听得妇人口里喘声呼叫:“达达,你只顾搧打到几时?只怕和尚来听见。饶了奴,快些丢了罢!”西门庆道:“你且休慌!我还要在盖子上烧一下儿哩!”不想都被这秃厮听了个不亦乐乎。落后众和尚到齐了,吹打起法事来,一个传一个,都知妇人有汉子在屋里,不觉都手之舞之,足之蹈之。临佛事完满,晚夕送灵化财出去,妇人又早除了孝髻,登时把灵牌并佛烧了。那贼秃冷眼瞧见,帘子里一个汉子和婆娘影影绰绰并肩站着,想起白日里听见那些勾当,只顾乱打鼓搧钹不住。被风把长老的僧伽帽刮在地上,露出青旋旋光头,不去拾,只顾搧钹打鼓,笑成一块。王婆便叫道:“师父,纸马已烧过了,还只顾搧打怎的?”和尚答道:“还有纸炉盖子上没烧过。”西门庆听见,一面令王婆快打发衬钱与他。长老道:“请斋主娘子谢谢。”妇人道:“干娘说免了罢。”众和尚道:“不如饶了罢。”一齐笑的去了。正是:隔墙须有耳,窗外岂无人!有诗为证:

\[
淫妇烧灵志不平,阇黎窃壁听淫声。
果然佛法能消罪,亡者闻之亦惨魂。
\]

\newpage
%# -*- coding:utf-8 -*-
%%%%%%%%%%%%%%%%%%%%%%%%%%%%%%%%%%%%%%%%%%%%%%%%%%%%%%%%%%%%%%%%%%%%%%%%%%%%%%%%%%%%%


\chapter{西门庆偷娶潘金莲\KG 武都头误打李皂隶}


诗曰:

\[
感郎耽夙爱,着意守香奁。岁月多忘远,情综任久淹。
于飞期燕燕,比翼誓鹣鹣。细数从前意,时时屈指尖。
\]

话说西门庆与潘金莲烧了武大灵,到次日,又安排一席酒,请王婆作辞,就把迎儿交付与王婆看养。因商量道:“武二回来,却怎生不与他知道六姐是我娶了才好?”王婆笑道:“有老身在此,任武二那厮怎地兜达,我自有话回他。大官人只管放心!”西门庆听了,满心欢喜,又将三两银子谢他。当晚就将妇人箱笼,都打发了家去,剩下些破桌、坏凳、旧衣裳,都与了王婆。到次日初八,一顶轿子,四个灯笼,妇人换了一身艳色衣服,王婆送亲,玳安跟轿,把妇人抬到家中来。那条街上,远近人家无一不知此事,都惧怕西门庆有钱有势,不敢来多管,只编了四句口号,说得好:

\[
堪笑西门不识羞,先奸后娶丑名留。
轿内坐着浪淫妇,后边跟着老牵头。
\]

西门庆娶妇人到家,收拾花园内楼下三间与他做房。一个独独小角门儿进去,院内设放花草盆景。白日间人迹罕到,极是一个幽僻去处。一边是外房,一边是卧房。西门庆旋用十六两银子买了一张黑漆欢门描金床,大红罗圈金帐幔,宝象花拣妆,桌椅锦杌,摆设齐整。大娘子吴月娘房里使着两个丫头,一名春梅,一名玉箫。西门庆把春梅叫到金莲房内,令他伏侍金莲,赶着叫娘。却用五两银子另买一个小丫头,名叫小玉,伏侍月娘。又替金莲六两银子买了一个上灶丫头,名唤秋菊。排行金莲做第五房。先头陈家娘子陪嫁的,名唤孙雪娥,约二十年纪,生的五短身材,有姿色。西门庆与他戴了鬒髻,排行第四,以此把金莲做个第五房。此事表过不题。

这妇人一娶过门来,西门庆就在妇人房中宿歇,如鱼似水,美爱无加。到第二日,妇人梳妆打扮,穿一套艳色服,春梅捧茶,走来后边大娘子吴月娘房里,拜见大小,递见面鞋脚。月娘在座上仔细观看,这妇人年纪不上二十五六,生的这样标致。但见:

\[
眉似初春柳叶,常含着雨恨云愁;脸如三月桃花,暗带着风情月意。纤腰袅娜,拘束的燕懒莺慵;檀口轻盈,勾引得峰狂蝶乱。玉貌妖娆花解语,芳容窈窕玉生香。吴月娘从头看到脚,风流往下跑;从脚看到头,风流往上流。论风流,如水泥晶盘内走明珠;语态度,似红杏枝头笼晓日。
\]
看了一回,口中不言,心内想道:“小厮每来家,只说武大怎样一个老婆,不曾看见,不想果然生的标致,怪不的俺那强人爱他。”金莲先与月娘磕了头,递了鞋脚。月娘受了他四礼。次后李娇儿、孟玉楼、孙雪娥,都拜见了,平叙了姊妹之礼,立在傍边。月娘叫丫头拿个坐儿教他坐,分付丫头、媳妇赶着他叫五娘。这妇人坐在傍边,不转睛把众人偷看。见吴月娘约三九年纪,生的面如银盆,眼如杏子,举止温柔,持重寡言。第二个李娇儿,乃院中唱的,生的肌肤丰肥,身体沉重,虽数名妓者之称,而风月多不及金莲也。第三个就是新娶的孟玉楼,约三十年纪,生得貌若梨花,腰如杨柳,长挑身材,瓜子脸儿,稀稀多几点微麻,自是天然俏丽,惟裙下双湾与金莲无大小之分。第四个孙雪娥,乃房里出身,五短身材,轻盈体态,能造五鲜汤水,善舞翠盘之妙。这妇人一抹儿都看在心里。过三日之后,每日清晨起来,就来房里与月娘做针指,做鞋脚,凡事不拿强拿,不动强动。指着丫头赶着月娘,一口一声只叫大娘,快把小意儿贴恋几次,把月娘喜欢得没入脚处,称呼他做六姐。衣服首饰拣心爱的与他,吃饭吃茶都和他在一处。因此,李娇儿众人见月娘错敬他,都气不忿,背后常说:“俺们是旧人,到不理论。他来了多少时,便这等惯了他。大姐姐好没分晓!”西门庆自娶潘金莲来家,住着深宅大院,衣服头面又相趁,二人女貌郎才,正在妙年之际,凡事如胶似漆,百依百随,淫欲之事,无日无之。且按下不题。

单表武松,八月初旬到了清河县,先去县里纳了回书。知县见了大喜,已知金宝交得明白,赏了武松十两银子,酒食管待,不必细说。武松回到下处,换了衣服鞋袜,戴了一顶新头巾,锁了房门,一径投紫石街来。两边众邻舍看见武松回来,都吃一惊,捏两把汗,说道:“这番萧墙祸起了!这个太岁归来,怎肯干休!”武松走到哥哥门前,揭起帘子,探身入来,看见小女迎儿在楼穿廊下撵线。叫声哥哥也不应,叫声嫂嫂也不应,道:“我莫不耳聋了,如何不见哥嫂声音?”向前便问迎儿。那迎儿见他叔叔来,吓的不敢言语。武松道:“你爹娘往那里去了?”迎儿只是哭,不做声。正问间,隔壁王婆听得是武二归来,生怕决撒了,慌忙走过来。武二见王婆过来,唱了喏,问道:“我哥哥往那里去了?嫂嫂也怎的不见?”婆子道:“二哥请坐,我告诉你。你哥哥自从你去后,到四月间得个拙病死了。”武二道:“我哥哥四月几时死的?得什么病?吃谁的药来?”王婆道:“你哥哥四月二十头,猛可地害起心疼起来,病了八九日,求神问卜,什么药不吃到?医治不好,死了。”武二道:“我的哥哥从来不曾有这病,如何心疼便死了?”王婆道:“都头却怎的这般说?天有不测风云,人有旦夕祸福。今晚脱了鞋和袜,未审明朝穿不穿。谁人保得常没事?”武二道:“我哥哥如今埋在那里?”王婆道:“你哥哥一倒了头,家中一文钱也没有,大娘子又是没脚蟹,那里去寻坟地?亏左近一个财主旧与大郎有一面之交,舍助一具棺木,没奈何放了三日,抬出去火葬了。”武二道:“如今嫂嫂往那里去了?”婆子道:“他少女嫩妇的,又没的养赡过日子。胡乱守了百日孝,他娘劝他,前月嫁了外京人去了。丢下这个业障丫头子,教我替他养活。专等你回来交付与你,也了我一场事。”武二听言,沉吟了半晌,便撇下王婆出门去,迳投县前下处。开了门进房里,换了一身素衣,便叫土兵街上打了一条麻绦,买了一双绵裤,一顶孝帽戴在头上;又买了些果品点心、香烛冥纸、金银锭之类,归到哥哥家,从新安设武大灵位。安排羹饭,点起香烛,铺设酒肴,挂起经幡纸缯,安排得端正。约一更已后,武二拈了香,扑翻身便拜,道:“哥哥阴魂不远,你在世时,为人软弱,今日死后,不见分明。你若负屈含冤,被人害了,托梦与我,兄弟替你报冤雪恨!”把酒一面浇奠了,烧化冥纸,武二便放声大哭。终是一路上来的人,哭的那两边邻舍无不凄惶。武二哭罢,将这羹饭酒肴和土兵、迎儿吃了。讨两条席子,教土兵房外傍边睡,迎儿房中睡,他便自把条席子,就武大灵桌子前睡。

约莫将半夜时分,武二翻来覆去那里睡得着,口里只是长吁气。那土兵齁齁的却似死人一般,挺在那里。武二爬将起来看时,那灵桌子上琉璃灯半明半灭。武二坐在席子上,自言自语,口里说道:“我哥哥生时懦弱,死后却无分明。”说犹未了,只见那灵桌子下卷起一阵冷风来。但见:

\[
无形无影,非雾非烟。盘旋似怪风侵骨冷,凛冽如杀气透肌寒。昏昏暗暗,灵前灯火失光明;惨惨幽幽,壁上纸钱飞散乱。隐隐遮藏食毒鬼,纷纷飘逐影魂幡。
\]
那阵冷风,逼得武二毛发皆竖起来。定睛看时,见一个人从灵桌底下钻将出来,叫声:“兄弟!我死得好苦也!”武二看不仔细,却待向前再问时,只见冷气散了,不见了人。武二一交跌翻在席子上坐的,寻思道:“怪哉!似梦非梦。刚才我哥哥正要报我知道,又被我的神气冲散了。想来他这一死,必然不明。”听那更鼓,正打三更三点。回头看那土兵,正睡得好。于是咄咄不乐,只等天明,却再理会。

看看五更鸡叫,东方渐明。土兵起来烧汤,武二洗漱了,唤起迎儿看家,带领土兵出了门。在街上访问街坊邻舍:“我哥哥怎的死了?嫂嫂嫁得何人去了?”那街坊邻舍明知此事,都惧怕西门庆,谁肯来管?只说:“都头,不消访问,王婆在紧隔壁住,只问王婆就知了。”有那多口的说:“卖梨的郓哥儿与仵作何九,二人最知详细。”这武二竟走来街坊前去寻郓哥。只见那小猴子手里拿着个柳笼簸罗儿,正籴米回来。武二便叫郓哥道:“兄弟!”唱喏。那小厮见是武二叫他,便道:“武都头,你来迟了一步儿,须动不得手。只是一件,我的老爹六十岁,没人养赡,我却难保你们打官司。”武二道:“好兄弟,跟我来。”引他到一个饭店楼上,武二叫货卖造两分饭来。武二对郓哥道:“兄弟,你虽年幼,倒有养家孝顺之心。我没甚么——”向身边摸出五两碎银子,递与郓哥道:“你且拿去与老爹做盘费。待事务毕了,我再与你十来两银子做本钱。你可备细说与我:哥哥和甚人合气?被甚人谋害了?家中嫂嫂被那一个娶去?你一一说来,休要隐匿。”这郓哥一手接过银子,自心里想道:“这些银子,老爹也勾盘费得三五个月,便陪他打官司也不妨。”一面说道:“武二哥,你听我说,却休气苦。”于是把卖梨儿寻西门庆,后被王婆怎地打他,不放进去,又怎地帮扶武大捉奸,西门庆怎的踢中了武大,心疼了几日,不知怎的死了,从头至尾细说了一遍。武二听了,便道:“你这话却是实么?”又问道:“我的嫂子实嫁与何人去了?”郓哥道:“你嫂子吃西门庆抬到家,待捣吊底子儿,自还问他实也是虚!”武二道:“你休说谎。”郓哥道:“我便官府面前,也只是这般说。”武二道:“兄弟,既然如此,讨饭来吃。”须臾,吃了饭。武二还了饭钱,两个下楼来,分付郓哥:“你回家把盘缠交与老爹,明日早上来县前,与我作证。”又问:“何九在那里居住?”郓哥道:“你这时候还寻何九?他三日前听见你回,便走的不知去向了。”这武二放了郓哥家去。

到第二日,早起,先在陈先生家写了状子,走到县门前。只见郓哥也在那里伺候,一直奔到厅上跪下,声冤起来。知县看见,认的是武松,便问:“你告什么?因何声冤?”武二告道:“小人哥哥武大,被豪恶西门庆与嫂潘氏通奸,踢中心窝,王婆主谋,陷害性命。何九朦胧入殓,烧毁尸伤。见今西门庆霸占嫂子在家为妾。见有这个小厮郓哥是证见。望相公作主则个。”因递上状子。知县接着,便问:“何九怎的不见?”武二道:“何九知情在逃,不知去向。”知县于是摘问了郓哥口词,当下退厅与佐二官吏通同商议。原来知县、县丞、主簿、典史,上下都是与西门庆有首尾的,因此官吏通同计较,这件事难以问理。知县随出来叫武松道:“你也是个本县中都头,怎不省得法度?自古捉奸见双,杀人见伤。你那哥哥尸首又没了,又不曾捉得他奸。你今只凭这小厮口内言语,便问他杀人的公事,莫非公道忒偏向么?你不可造次,须要自己寻思。”武二道:“告禀相公,这都是实情,不是小人捏造出来的。只望相公拿西门庆与嫂潘氏、王婆来,当堂尽法一番,其冤自见。若有虚诬,小人情愿甘罪。”知县道:“你且起来,待我从长计较。可行时,便与你拿人。”武二方才起来,走出外边,把郓哥留在屋里,不放回家。

早有人把这件事报与西门庆得知。西门庆听得慌了,忙叫心腹家人来保、来旺,身边带着银两,连夜将官吏都买嘱了。到次日早晨,武二在厅上指望告禀知县,催逼拿人。谁想这官人受了贿赂,早发下状子来,说道:“武松,你休听外人挑拨,和西门庆做对头。这件事欠明白,难以问理。圣人云:经目之事,犹恐未真;背后之言,岂能全信?你不可一时造次。”当该吏典在傍,便道:“都头,你在衙门里也晓得法律,但凡人命之事,须要尸、伤、病、物、踪,五件事俱完,方可推问。你那哥哥尸首又没了,怎生问理?”武二道:“若恁的说时,小人哥哥的冤仇,难道终不能报便罢了?既然相公不准所告,且却有理。”遂收了状子,下厅来。来到下处,放了郓哥归家,不觉仰天长叹一声,咬牙切齿,口中骂淫妇不绝。

武松是何等汉子,怎消洋得这口恶气!一直走到西门庆生药店前,要寻西门庆厮打。正见他开铺子的傅伙计在柜身里面,见武二狠狠的走来,问道:“你大官人在宅上么?”傅伙计认的是武二,便道:“不在家了。都头有甚话说?”武二道:“且请借一步说句。”傅伙计不敢不出来,被武二引到僻静巷口。武二翻过脸来,用手撮住他衣领,睁圆怪眼说道:“你要死,却是要活?”傅伙计道:“都头在上,小人又不曾触犯了都头,都头何故发怒?”武二道:“你若要死,便不要说;若要活时,对我实说。西门庆那厮如今在那里?我的嫂子被他娶了多少日子?一一说来,我便罢休?”那傅伙计是个小胆的人,见武二发作,慌了手脚,说道:“都头息怒,小人在他家,每月二两银子雇着,小人只开铺子,并不知他们闲帐。大官人本不在家,刚才和一相知,往狮子街大酒楼上吃酒去了。小人并不敢说谎。”武二听了此言,方才放了手,大叉步飞奔到狮子街来。吓的傅伙计半日移脚不动。那武二迳奔到狮子街桥下酒楼前来。

且说西门庆正和县中一个皂隶李外传在楼上吃酒。原来那李外传专一在府县前绰揽些公事,往来听气儿撰些钱使。若有两家告状的,他便卖串儿;或是官吏打点,他便两下里打背。因此县中就起了他这个浑名,叫做李外传。那日见知县回出武松状子,讨得这个消息,便来回报西门庆知道。因此西门庆让他在酒楼上饮酒,把五两银子送他。正吃酒在热闹处,忽然把眼向楼窗下看,只见武松似凶神般从桥下直奔酒楼前来。已知此人来意不善,不觉心惊,欲待走了,却又下楼不及,遂推更衣,走往后楼躲避。武二奔到酒楼前,便问酒保道:“西门庆在此么?”酒保道:“西门大官人和一相识在楼上吃酒哩。”武二拨步撩衣,飞抢上楼去。早不见了西门庆,只见一个人坐在正面,两个唱的粉头坐在两边。认的是本县皂隶李外传,就知是他来报信,不觉怒从心起,便走近前,指定李外传骂道:“你这厮,把西门庆藏在那里去了?快说了,饶你一顿拳头!”李外传看见武二,先吓呆了,又见他恶狠狠逼紧来问,那里还说得出话来!武二见他不则声,越加恼怒,便一脚把桌子踢倒,碟儿盏儿都打得粉碎。两个粉头吓得魂都没了。李外传见势头不好,强挣起身来,就要往楼下跑。武二一把扯回来道:“你这厮,问着不说,待要往那里去?且吃我一拳,看你说也不说!”早飕的一拳,飞到李外传脸上。李外传叫声啊呀,忍痛不过,只得说道:“西门庆才往后楼更衣去了,不干我事,饶我去罢!”武二听了,就趁势儿用双手将他撮起来,隔着楼窗儿往外只一兜,说道:“你既要去,就饶你去罢!”扑通一声,倒撞落在当街心里。武二随即赶到后楼来寻西门庆。此时西门庆听见武松在前楼行凶,吓得心胆都碎,便不顾性命,从后楼窗一跳,顺着房檐,跳下人家后院内去了。武二见西门庆不在后楼,只道是李外传说谎,急转身奔下楼来,见李外传已跌得半死,直挺挺在地下,还把眼动。气不过,兜裆又是两脚,早已哀哉断气身亡。众人道:“这是李皂隶,他怎的得罪都头来?为何打杀他?”武二道:“我自要打西门庆,不料这厮悔气,却和他一路,也撞在我手里。”那地方保甲见人死了,又不敢向前捉武二,只得慢慢挨上来收笼他,那里肯放松!连酒保王鸾并两个粉头包氏、牛氏都拴了,竟投县衙里来。此时哄动了狮子街,闹了清河县,街上议论的人,不计其数。却不知道西门庆不该死,倒都说是西门庆大官人被武松打死了。正是:

\[
李公吃了张公酿,郑六生儿郑九当。
世间几许不平事,都付时人话短长。
\]

\newpage
%# -*- coding:utf-8 -*-
%%%%%%%%%%%%%%%%%%%%%%%%%%%%%%%%%%%%%%%%%%%%%%%%%%%%%%%%%%%%%%%%%%%%%%%%%%%%%%%%%%%%%


\chapter{义士充配孟州道\KG 妻妾玩赏芙蓉亭}


词曰:

\[
八月中秋,凉飙微逗,芙蓉却是花时候。谁家姊妹斗新妆,园林散步携手。折得花枝,宝瓶随后,归来玩赏全凭酒。三杯酩酊破愁城,醒时愁绪应还又。
\]

话说武二被地方保甲拿去县里见知县,不题。且表西门庆跳下楼窗,扒伏在人家院里藏了。原来是行医的胡老人家。只见他家使的一个大胖丫头,走来毛厕里净手,蹶着大屁股,猛可见一个汉子扒伏在院墙下,往前走不迭,大叫:“有贼了!”慌的胡老人急进来。看见,认得是西门庆,便道:“大官人,且喜武二寻你不着,把那人打死了。地方拿他县中见官去了。这一去定是死罪。大官人归家去,料无事矣。”西门庆拜谢了胡老人,摇摆来家,一五一十对潘金莲说,二人拍手喜笑,以为除了患害。妇人叫西门庆上下多使些钱,务要结果了他,休要放他出来。西门庆一面差心腹家人来旺儿,馈送了知县一副金银酒器、五十两银子,上下吏典也使了许多钱,只要休轻勘了武二。

知县受了贿赂,到次日升厅。地方押着武松并酒保、唱的一班人,当厅跪下。县主翻了脸,便叫:“武松!你这厮昨日诬告平人,我已再三宽你,如何不遵法度,今又平白打死人?”武松道:“小人本与西门庆有仇,寻他厮打,不料撞遇此人。他隐匿西门庆不说,小人一时怒起,误将他打死。只望相公与小人做主,拿西门庆正法,与小人哥哥报这一段冤仇。小人情愿偿此人误伤之罪。”知县道:“这厮胡说,你岂不认得他是县中皂隶!今打杀他,定别有缘故,为何又缠到西门庆身上?不打如何肯招!”喝令左右加刑。两边内三四个皂隶,把武松拖翻,雨点般打了二十。打得武二口口声冤道:“小人也有与相公效劳用力之处,相公岂不怜悯?相公休要苦刑小人!”知县听了此言,越发恼了,道:“你这厮亲手打死了人,尚还口强,抵赖那个?”喝令:“好生与我拶起来!”当下又拶了武松一拶,敲了五十杖子,教取面长枷带了,收在监内。一干人寄监在门房里。内中县丞、佐二官也有和武二好的,念他是个义烈汉子,有心要周旋他,争奈都受了西门庆贿赂,粘住了口,做不的主张。又见武松只是声冤,延挨了几日,只得朦胧取了供招,唤当该吏典并仵作、邻里人等,押到狮子街,检验李外传身尸,填写尸单格目。委的被武松寻问他索讨分钱不均,酒醉怒起,一时斗殴,拳打脚踢,撞跌身死。左肋、面门、心坎、肾囊,俱有青赤伤痕不等。检验明白,回到县中。一日,做了文书申详,解送东平府来,详允发落。

这东平府尹,姓陈双名文昭,乃河南人氏,极是个清廉的官,听的报来,随即升厅。但见他:

\[
平生正直,秉性贤明。幼年向雪案攻书,长大在金銮对策。常怀忠孝之心,每发仁慈之政。户口登,钱粮办,黎民称颂满街衢;词颂减,盗贼休,父老赞歌喧市井。正是:名标青史播千年,声振黄堂传万古。贤良方正号青天,正直清廉民父母。
\]
这府尹陈文昭升了厅,便教押过这干犯人,就当厅先把清河县申文看了,又把各人供状招拟看过,端的上面怎生写着?文曰:

\[
东平府清河县,为人命事呈称:犯人武松,年二十八岁,系阳谷县人氏。因有膂力,本县参做都头。因公差回还,祭奠亡兄,见嫂潘氏不守孝满,擅自嫁人。是日,松在巷口缉听,不合在狮子街上王鸾酒楼上撞遇李外传。因酒醉,索讨前借钱三百文,外传不与;又不合因而斗殴,相互不服,揪打踢撞伤重,当时身死。比有唱妇牛氏、包氏见证,致被地方保甲捉获。委官前至尸所,拘集仵作、里甲人等,检验明白,取供具结,填图解缴前来,覆审无异。拟武松合依斗殴杀人,不问手足、他物、金两,律绞。酒保王鸾并牛氏、包氏,俱供明无罪。今合行申到案发落,请允施行。政和三年八月日知县李达天、县丞乐和安、主簿华荷禄、典史夏恭基、司吏钱劳。
\]

府尹看了一遍,将武松叫过面前,问道:“你如何打死这李外传?”那武松只是朝上磕头告道:“青天老爷!小的到案下,得见天日。容小的说,小的敢说。”府尹道:“你只顾说来。”武松遂将西门庆奸娶潘氏,并哥哥捉奸,踢中心窝,后来县中告状不准,前后情节细说一遍,道:“小的本为哥哥报仇,因寻西门庆厮打,不料误打死此人。委是小的负屈含冤,奈西门庆钱大,禁他不得。小人死不足惜,但只是小人哥哥武大含冤地下,枉了性命。”府尹道:“你不消多言,我已尽知了。”因把司吏钱劳叫来,痛责二十板,说道:“你那知县也不待做官,何故这等任情卖法?”于是将一干人众,一一审录过,用笔将武松供招都改了,因向佐二官说道:“此人为兄报仇,误打死这李外传,也是个有义的烈汉,比故杀平人不同。”一面打开他长枷,换了一面轻罪枷枷了,下在牢里。一干人等都发回本县听候。一面行文书着落清河县,添提豪恶西门庆,并嫂潘氏、王婆、小厮郓哥、仵作何九,一同从公根勘明白,奏请施行。武松在东平府监中,人都知道他是条好汉,因此押牢禁子都不要他一文钱,到把酒食与他吃。

早有人把这件事报到清河县。西门庆知道了,慌了手脚。陈文昭是个清廉官,不敢来打点他。只得走去央求亲家陈宅心腹,并使家人来旺星夜往东京下书与杨提督。提督转央内阁蔡太师。太师又恐怕伤了李知县名节,连忙赍了一封密书,特来东平府下与陈文昭,免提西门庆、潘氏。这陈文昭原系大理寺寺正,升东平府府尹,又系蔡太师门生,又见杨提督乃是朝廷面前说得话的官,以此人情两尽,只把武松免死,问了个脊杖四十,刺配二千里充军。况武大已死,尸伤无存,事涉疑似,勿论。其余一干人犯释放宁家。申详过省院,文书到日,即便施行。陈文昭从牢中取出武松来,当堂读了朝廷明降,开了长枷,免不得脊杖四十,取一具七斤半铁叶团头枷钉了,脸上刺了两行金字,迭配孟州牢城。其余发落已完,当堂府尹押行公文,差两个防送公人,领了武松解赴孟州交割。

当日武松与两个公人出离东平府,来到本县家中,将家活多变卖了,打发那两个公人路上盘费,央托左邻姚二郎看管迎儿:“倘遇朝廷恩典,赦放还家,恩有重报,不敢有忘。”街坊邻舍,上户人家,见武二是个有义的汉子,不幸遭此,都资助他银两,也有送酒食钱米的。武二到下处,问土兵要出行李包裹来,即日离了清河县上路,迤逦往孟州大道而行。有诗为证:

\[
府尹推详秉至公,武松垂死又疏通。
今朝刺配牢城去,病草萋萋遇暖风。
\]

这里武二往孟州充配去了,不题。且说西门庆打听他上路去了,一块石头方落地,心中如去了痞一般,十分自在。于是家中分付家人来旺、来保、来兴儿,收拾打扫后花园芙蓉亭干净,铺设围屏,挂起锦障,安排酒席齐整,叫了一起乐人,吹弹歌舞。请大娘子吴月娘、第二李娇儿、第三孟玉楼、第四孙雪娥、第五潘金莲,合家欢喜饮酒。家人媳妇、丫鬟使女两边侍奉。但见:

\[
香焚宝鼎,花插金瓶。器列象州之古玩,帘开合浦之明珠。水晶盘内,高堆火枣交梨;碧玉杯中,满泛琼浆玉液。烹龙肝,炮凤腑,果然下箸了万钱;黑熊掌,紫驼蹄,酒后献来香满座。碾破凤团,白玉瓯中分白浪;斟来琼液,紫金壶内喷清香。毕竟压赛孟尝君,只此敢欺石崇富。
\]

当下西门庆与吴月娘居上,其余多两傍列坐,传杯弄盏,花簇锦攒。饮酒间,只见小厮玳安领下一个小厮、一个小女儿,才头发齐眉,生得乖觉,拿着两个盒儿,说道:“隔壁花家,送花儿来与娘们戴。”走到西门庆、月娘众人跟前,都磕了头,立在傍边,说:“俺娘使我送这盒儿点心并花儿与西门大娘戴。”揭开盒儿看,一盒是朝廷上用的果馅椒盐金饼,一盒是新摘下来鲜玉簪花。月娘满心欢喜,说道:“又叫你娘费心。”一面看菜儿,打发两个吃了点心。月娘与了那小丫头一方汗巾儿,与了小厮一百文钱,说道:“多上覆你娘,多谢了。”因问小丫头儿:“你叫什么名字?”他回言道:“我叫绣春。小厮便是天福儿。”打发去了。月娘便向西门庆道:“咱这花家娘子儿,倒且是好,常时使小厮丫头送东西与我们。我并不曾回些礼儿与他。”西门庆道:“花二哥娶了这娘子儿,今不上二年光景。他自说娘子好个性儿。不然房里怎生得这两个好丫头。”月娘道:“前者他家老公公死了出殡时,我在山头会他一面。生得五短身材,团面皮,细湾湾两道眉儿,且是白净,好个温克性儿。年纪还小哩,不上二十四五。”西门庆道:“你不知,他原是大名府梁中书妾,晚嫁花家子虚,带一分好钱来。”月娘道:“他送盒儿来,咱休差了礼数,到明日也送些礼物回答他。”

看官听说:原来花子虚浑家姓李,因正月十五所生,那日人家送了一对鱼瓶儿来,就小字唤做瓶姐。先与大名府梁中书为妾。梁中书乃东京蔡太师女婿,夫人性甚嫉妒,婢妾打死者多埋在后花园中。这李氏只在外边书房内住,有养娘伏侍。只因政和三年正月上元之夜,梁中书同夫人在翠云楼上,李逵杀了全家老小,梁中书与夫人各自逃生。这李氏带了一百颗西洋大珠,二两重一对鸦青宝石,与养娘走上东京投亲。那时花太监由御前班直升广南镇守,因侄男花子虚没妻室,就使媒婆说亲,娶为正室。太监到广南去,也带他到广南,住了半年有余。不幸花太监有病,告老在家,因是清河县人,在本县住了。如今花太监死了,一分钱多在子虚手里。每日同朋友在院中行走,与西门庆都是前日结拜的弟兄。终日与应伯爵、谢希大一班十数个,每月会在一处,叫些唱的,花攒锦簇顽耍。众人又见花子虚乃是内臣家勤儿,手里使钱撒漫,哄着他在院中请婊子,整三五夜不归。正是:

\[
紫陌春光好,红楼醉管弦。
人生能有几?不乐是徒然。
\]

此事表过不题。且说当日西门庆率同妻妾,合家欢乐,在芙蓉亭上饮酒,至晚方散。归来潘金莲房中,已有半酣,乘着酒兴,要和妇人云雨。妇人连忙熏香打铺,和他解衣上床。西门庆且不与他云雨,明知妇人第一好品箫,于是坐在青纱帐内,令妇人马爬在身边,双手轻笼金钏,捧定那话,往口里吞放。西门庆垂首玩其出入之妙,鸣咂良久,淫情倍增,因呼春梅进来递茶。妇人恐怕丫头看见,连忙放下帐子来。西门庆道:“怕怎么的?”因说起:“隔壁花二哥房里到有两个好丫头,今日送花来的是小丫头。还有一个也有春梅年纪,也是花二哥收用过了。但见他娘在门首站立,他跟出来,却是生得好模样儿。谁知这花二哥年纪小小的,房里恁般用人!”妇人听了,瞅了他一眼,说道:“怪行货子,我不好骂你,你心里要收这个丫头,收他便了,如何远打周折,指山说磨,拿人家来比奴。奴不是那样人,他又不是我的丫头!既然如此,明日我往后边坐一回,腾个空儿,你自在房中叫他来,收他便了。”西门庆听了,欢喜道:“我的儿,你会这般解趣,怎教我不爱你!”二人说得情投意洽,更觉美爱无加,慢慢的品箫过了,方才抱头交股而寝。正是:

\[
自有内事迎郎意,殷勤快把紫箫吹。
\]
有《西江月》为证:

\[
纱帐香飘兰麝,娥眉惯把箫吹。雪莹玉体透房帏,禁不住魂飞魄碎。玉腕款笼金钏,两情如醉如痴。才郎情动嘱奴知,慢慢多咂一会。
\]

到次日,果然妇人往孟玉楼房中坐了。西门庆叫春梅到房中,收用了这妮子。正是:

\[
春点杏桃红绽蕊,风欺杨柳绿翻腰。
\]

潘金莲自此一力抬举他起来,不令他上锅抹灶,只叫他在房中铺床叠被,递茶水,衣服首饰拣心爱的与他,缠得两只脚小小的。原来春梅比秋菊不同,性聪慧,喜谑浪,善应对,生的有几分颜色,西门庆甚是宠他。秋菊为人浊蠢,不谙事体,妇人常常打的是他。正是:

\[
燕雀池塘语话喧,蜂柔蝶嫩总堪怜。
虽然异数同飞鸟,贵贱高低不一般。
\]

\newpage
%# -*- coding:utf-8 -*-
%%%%%%%%%%%%%%%%%%%%%%%%%%%%%%%%%%%%%%%%%%%%%%%%%%%%%%%%%%%%%%%%%%%%%%%%%%%%%%%%%%%%%


\chapter{潘金莲激打孙雪娥\KG 西门庆梳笼李桂姐}


诗曰:

\[
六街箫鼓正喧阗,初月今朝一线添。
睡去乌衣惊玉剪,斗来宵烛浑朱帘。
香绡染处红余白,翠黛攒来苦味甜。
阿姐当年曾似此,纵他戏汝不须嫌。
\]

话说潘金莲在家恃宠生骄,颠寒作热,镇日夜不得个宁静。性极多疑,专一听篱察壁。那个春梅,又不是十分耐烦的。一日,金莲为些零碎事情不凑巧,骂了春梅几句。春梅没处出气,走往后边厨房下去,槌台拍凳闹狠狠的模样。那孙雪娥看不过,假意戏他道:“怪行货子!想汉子便别处去想,怎的在这里硬气?”春梅正在闷时,听了这句,不一时暴跳起来:“那个歪斯缠我哄汉子?”雪娥见他性不顺,只做不听得。春梅便使性做几步走到前边来,一五一十,又添些话头,道:“他还说娘教爹收了我,俏一帮儿哄汉子。”挑拨与金莲知道。金莲满肚子不快活。因送吴月娘出去送殡,起身早些,有些身子倦,睡了一觉,走到亭子上。只见孟玉楼摇飐的走来,笑嘻嘻道:“姐姐如何闷闷的不言语?”金莲道:“不要说起,今早倦的了不得。三姐你在那里去来?”玉楼道:“才到后面厨房里走了走来。”金莲道:“他与你说些甚么来?”玉楼道:“姐姐没言语。”金莲心虽怀恨,口里却不说出。两个做了一回针指。只见春梅拿茶来,吃毕,两个闷倦,就放桌儿下棋耍子。忽见看园门小厮琴童走来,报道:“爹来了。”慌的两个妇人收棋子不迭。西门庆恰进门槛,看见二人家常都带着银丝鬒髻,露着四鬓,耳边青宝石坠子,白纱衫儿,银红比甲,挑线裙子,双弯尖翘,红鸳瘦小,一个个粉妆玉琢,不觉满面堆笑,戏道:“好似一对儿粉头,也值百十两银子!”潘金莲说道:“俺们倒不是粉头,你家正有粉头在后边哩!”那玉楼抽身就往后走,被西门庆一手拉住,说道:“你往那里去?我来了,你倒要脱身去了。实说,我不在家,你两个在这里做甚么?”金莲道:“俺俩个闷的慌,在这里下了两盘棋,时没做贼,谁知道你就来了。”一面替他接了衣服,说道:“你今日送殡来家早。”西门庆道:“今日斋堂里都是内相同官,天气又热,我不耐烦,先来家。”玉楼问道:“他大娘怎的还不来?”西门庆道:“他的轿子也待进城,我先回,使两个小厮接去了。”一面坐下。因问:“你两个下棋赌些甚么?”金莲道:“俺两个自下一盘耍子,平白赌什么?”西门庆道:“等我和你们下一盘,那个输了,拿出一两银子做东道。”金莲道:“俺们没银子。”西门庆道:“你没银子,拿簪子问我当,也是一般。”于是摆下棋子,三人下了一盘。潘金莲输了。西门庆才数子儿,被妇人把棋子扑撒乱了。一直走到瑞香花下,倚着湖山,推掐花儿。西门庆寻到那里,说道:“好小油嘴儿!你输了棋子,却躲在这里。”那妇人见西门庆来,昵笑不止,说道:“怪行货子!孟三儿输了,你不敢禁他,却来缠我!”将手中花撮成瓣儿,洒西门庆一身。被西门庆走向前,双关抱住,按在湖山畔,就口吐丁香,舌融甜唾,戏谑做一处。不防玉楼走到根前,叫道:“六姐,他大娘来家了。咱后边去来。”这妇人撇了西门庆,说道:“哥儿,我回来和你答话。”遂同玉楼到后边,与月娘道了万福。月娘问:“你们笑甚么?”玉楼道:“六姐今日和他爹下棋,输了一两银子,到明日整治东道,请姐姐耍子。”月娘笑了。金莲只在月娘面前打了个照面儿,就走来前边陪伴西门庆。分付春梅房中薰香,预备澡盆浴汤,准备晚间效鱼水之欢。看官听说:家中虽是吴月娘居大,常有疾病,不管家事。只是人情来往,出入银钱,都在李娇儿手里。孙雪儿单管率领家人媳妇,在厨中上灶,打发各房饮食。譬如西门庆在那房里宿歇,或吃酒,或吃饭,造甚汤水,俱经雪娥手中整理,那房里丫头自往厨下去拿。此不必说。当晚西门庆在金莲房中,吃了回酒,洗毕澡,两人歇了。

次日,也是合当有事。西门庆许下金莲,要往庙上替他买珠子穿箍儿戴。早起来,等着要吃荷花饼、银丝鲊汤,使春梅往厨下说去。那春梅只顾不动身。金莲道:“你休使他。有人说我纵容他,教你收了,俏成一帮儿哄汉子。百般指猪骂狗,欺负俺娘儿们。你又使他后边做甚么去?”西门庆便问:“是谁说的?你对我说。”妇人道:“说怎的!盆罐都有耳朵,你只不叫他后边去,另使秋菊去便了。”这西门庆遂叫过秋菊,分付他往厨下对雪娥说去。约有两顿饭时,妇人已是把桌儿放了,白不见拿来。急的西门庆只是暴跳。妇人见秋菊不来,使春梅:“你去后边瞧瞧那奴才,只顾生根长苗的不见来。”

春梅有几分不顺,使性子走到厨下。只见秋菊正在那里等着哩,便骂道:“贼奴才,娘要卸你那腿哩!说你怎的就不去了。爹等着吃了饼,要往庙上去。急的爹在前边暴跳,叫我采了你去哩!”这孙雪娥不听便罢,听了心中大怒,骂道:“怪小淫妇儿!马回子拜节——来到的就是?锅儿是铁打的,也等慢慢儿的来,预备下熬的粥儿又不吃,忽剌八新兴出来要烙饼做汤。那个是肚里蛔虫!”春梅不忿他骂,说道:“没的扯屄淡!主子不使了来,那个好来问你要。有与没,俺们到前边只说的一声儿,有那些声气的?”一只手拧着秋菊的耳朵,一直往前边来。雪娥道:“主子奴才,常远似这等硬气,有时道着!”春梅道:“有时道没时道,没的把俺娘儿两个别变了罢!”于是气狠狠走来。妇人见他脸气得黄黄的,拉着秋菊进门,便问:“怎的来了?”春梅道:“你问他。我去时还在厨房里雌着,等他慢条厮礼儿才和面儿。我自不是,说了一句‘爹在前边等着,娘说你怎的就不去了?’倒被那小院儿里的,千奴才、万奴才骂了我恁一顿。说爹马回子拜节——走到的就是!只象那个调唆了爹一般,预备下粥儿不吃,平白新生发起要甚饼和汤。只顾在厨房里骂人,不肯做哩。”妇人在旁便道:“我说别要使他去,人自恁和他合气。说俺娘儿两个霸拦你在这屋里,只当吃人骂将来。”这西门庆听了大怒,走到后边厨房里,不由分说,向雪娥踢了几脚,骂道:“贼歪剌骨!我使他来要饼,你如何骂他?你骂他奴才,你如何不溺泡尿把你自家照照!”雪娥被西门庆踢骂了一顿,敢怒而不敢言。西门庆刚走出厨房外,孙雪娥对着来昭妻一丈青说道:“你看,我今日晦气!早是你在旁听,我又没曾说什么。他走将来凶神似一般,大吆小喝,把丫头采的去了,反对主子面前轻事重报,惹的走来平白地把恁一场儿。我洗着眼儿,看着主子奴才长远恁硬气着,只休要错了脚儿!”不想被西门庆听见了,复回来又打了几拳,骂道:“贼奴才淫妇!你还说不欺负他,亲耳朵听见你还骂他。”打的雪娥疼痛难忍,西门庆便往前边去了。那雪娥气的在厨房里两泪悲流,放声大哭。吴月娘正在上房,才起来梳头,因问小玉:“厨房里乱些甚么?”小玉回道:“爹要饼吃了往庙上去,说姑娘骂五娘房里春梅来,被爹听见了,踢了姑娘几脚,哭起来。”月娘道:“也没见他,要饼吃连忙做了与他去就罢了,平白又骂他房里丫头怎的!”于是使小玉走到厨房,撺掇雪娥和家人媳妇忙造汤水,打发西门庆吃了,往庙上去,不题。

这雪娥气愤不过,正走到月娘房里告诉此事。不妨金莲蓦然走来,立于窗下潜听。见雪娥在房里对月娘、李娇儿说他怎的霸拦汉子,背地无所不为:“娘,你还不知淫妇,说起来比养汉老婆还浪,一夜没汉子也不成的。背地干的那茧儿,人干不出,他干出来。当初在家,把亲汉子用毒药摆死了,跟了来。如今把俺们也吃他活埋了。弄的汉子乌眼鸡一般,见了俺们便不待见。”月娘道:“也没见你,他前边使了丫头要饼,你好好打发与他去便了。平白又骂他怎的?”孙雪娥道:“我骂他秃也瞎也来?那顷,这丫头在娘房里着紧不听手。俺没曾在灶上把刀背打他,娘尚且不言语。可可今日轮到他手里,便骄贵的这等了。”正说着,只见小玉走到,说:“五娘在外边。”少倾,金莲进房,望着雪娥说道:“比如我当初摆死亲夫,你就不消叫汉子娶我来家,省得我霸拦着他,撑了你的窝儿。论起春梅,又不是我的丫头,你气不愤,还教他伏侍大娘就是了。省得你和他合气,把我扯在里头。那个好意死了汉子嫁人?如今也不难的勾当,等他来家,与我一纸休书,我去就是了。”月娘道:“我也不晓的你们底事。你们大家省言一句儿便了。”孙雪娥道:“娘,你看他嘴似淮洪也一般,随问谁也辩他不过。明在汉子根前戳舌儿,转过眼就不认了。依你说起来,除了娘,把俺们都撵,只留着你罢!”那吴月娘坐着,由着他那两个你一句我一句,只不言语。后来见骂起来,雪娥道:“你骂我奴才!你便是真奴才!”险些儿不曾打起来。月娘看不上,使小玉把雪娥拉往后边去。这潘金莲一直归到前边,卸了浓妆,洗了脂粉,乌云散乱,花容不整,哭得两眼如桃,躺在床上。

到日西时分,西门庆庙上来,袖着四两珠子,进入房中,一见便问:“怎的来?”妇人放声号哭起来,问西门庆要休书。如此这般告诉一遍:“我当初又不曾图你钱财,自恁跟了你来。如何今日教人这等欺负?千也说我摆杀汉子,万也说我摆杀汉子!没丫头便罢了,如何要人房里丫头伏侍?吃人指骂!”这西门庆不听便罢,听了时,三尸神暴跳,五脏气冲天。一阵风走到后边,采过雪娥头发来,尽力拿短棍打了几下。多亏吴月娘向前拉住了,说道:“没得大家省些事儿罢了!好交你主子惹气!”西门庆便道:“好贼歪剌骨,我亲自听见你在厨房里骂,你还搅缠别人。我不把你下截打下来也不算。”看官听说:不争今日打了孙雪娥,管教潘金莲从前作过事,没兴一齐来。正是:

\[
自古感恩并积恨,万年千载不生尘。
\]

当下西门庆打了雪娥,走到前边,窝盘住了金莲,袖中取出庙上买的四两珠子,递与他。妇人见汉子与他做主,出了气,如何不喜。由是要一奉十,宠爱愈深。

话休饶舌,一日正轮该花子虚家摆酒会茶,这花家就在西门庆紧隔壁。内官家摆酒,甚是丰盛。众兄弟都到了。因西门庆有事,约午后才来,都等他,不肯先坐。少顷,西门庆来到,然后叙礼让坐,东家安西门庆居首席。两个妓女,琵琶筝秦在席前弹唱。端的说不尽梨园娇艳,色艺双全。但见:

\[
罗衣叠雪,宝髻堆云。樱桃口,杏脸桃腮;杨柳腰,兰心蕙性。歌喉宛转,声如枝上流莺;舞态蹁跹,影似花间凤转。腔依古调,音出天然。舞回明月坠秦楼,歌遏行云遮楚馆。高低紧慢按宫商,轻重疾徐依格调,筝排雁柱声声慢,板拍红牙字字新。
\]

少顷,酒过三巡,歌吟两套,两个唱的放下乐器,向前花枝摇飐般来磕头。西门庆呼玳安书袋内取两封赏赐,每人二钱,拜谢了下去。因问东家花子虚道:“这位姐儿上姓?端的会唱。”东家未及答应,应伯爵插口道:“大官人多忘事,就不认的了?这弹筝的是花二哥令翠——勾栏后巷吴银儿。这弹琵琶的,就是我前日说的李三妈的女儿、李桂卿的妹子,小名叫做桂姐。你家中见放着他的亲姑娘。如何推不认的?”西门庆笑道:“元来就是他,我六年不见,不想就出落得恁般成人了!”落后酒阑,上席来递酒。这桂姐殷勤劝酒,情话盘桓。西门庆因问:“你三妈与姐姐桂卿,在家做甚么?怎的不来我家看看你姑娘?”桂姐道:“俺妈从去岁不好了一场,至今腿脚半边通动不的,只扶着人走。俺姐姐桂卿被准上一个客人包了半年,常接到店里住,两三日不放来家。家中好不无人,只靠着我逐日出来供唱,好不辛苦!时常也想着要往宅里看看姑娘,白不得个闲。爹许久怎的也不在里边走走?几时放姑娘家去看看俺妈也好。”西门庆见他一团和气,说话儿乖觉伶变,就有几分留恋之意,说道:“我今日约两位好朋友送你家去。你意下如何?”桂姐道:“爹休哄我。你肯贵人脚儿踏俺贱地?”西门庆道:“我不哄你。”便向袖中取出汗巾连挑牙与香茶盒儿,递与桂姐收了。桂姐道:“多咱去?如今使保儿先家去先说一声,作个预备。”西门庆道:“直待人散,一同起身。”少顷,递毕酒,约掌灯人散时分,西门庆约下应伯爵、谢希大,也不到家,骡马同送桂姐,迳进勾栏往李家去。正是:

\[
陷人坑,土窖般暗开掘;迷魂洞,囚牢般巧砌叠;检尸场,屠铺般明排列。整一味死温存活打劫。招牌儿大字书者:买俏金,哥哥休扯;缠头锦,婆婆自接;卖花钱,姐姐不赊。
\]

西门庆等送桂姐轿子到门首,李桂卿迎门接入堂中。见毕礼数,请老妈出来拜见。不一时,虔婆扶拐而出,半边胳膊都动弹不得,见了西门庆,道了万福。说道:“天么,天么!姐夫贵人,那阵风儿刮得你到这里?”西门庆笑道:“一向穷冗,没曾来得,老妈休怪。”虔婆又向应、谢二人说道:“二位怎的也不来走走?”伯爵道:“便是白不得闲,今日在花家会茶,遇见桂姐,因此同西门爹送回来。快看酒来,俺们乐饮三杯。”虔婆让三位上首坐了。一面点茶,一面打抹春台,收拾酒菜。少顷,掌上灯烛,酒肴罗列。桂姐从新房中打扮出来,旁边陪坐,免不得姐妹两个金樽满泛,玉阮同调,歌唱递酒。正是:

\[
琉璃锺,琥珀浓,小槽酒滴珍珠红。烹龙炮凤玉脂泣,罗帏绣幄围香风。吹龙笛,击鼍鼓。皓齿歌,细腰舞。况是青春莫虚度,银缸掩映娇娥语,不到刘伶坟上去。
\]

当下姊妹两个唱了一套,席上觥筹交错饮酒。西门庆向桂卿道:“今日二位在此,久闻桂姐善舞能歌南曲,何不请歌一词,奉劝二位一杯儿酒!”应伯爵道:“我又不当起动,借大官人余光,洗耳愿听佳音。”那桂姐坐着只是笑,半晌不动身。原来西门庆有心要梳笼桂姐,故先索落他唱。那院中婆娘见识精明,早已看破了八九分。桂卿在旁,就先开口说道:“我家桂姐从小儿养得娇,自来生得腼腆,不肯对人胡乱便唱。”于是西门庆便叫玳安书袋内取出五两一锭银子来,放在桌上,说道:“这些不当甚么,权与桂姐为脂粉之需,改日另送几套织金衣服。”桂姐连忙起身谢了。先令丫鬟收去,方才下席来唱。这桂姐虽年纪不多,却色艺过人,当下不慌不忙,轻扶罗袖,摆动湘裙,袖口边搭剌着一方银红撮穗的落花流水汗巾儿,歌唱道:

\[
\cipaim{驻云飞}举止从容,压尽勾栏占上风。行动香风送,频使人钦重。嗏!玉杵污泥中,岂凡庸?一曲宫商,满座皆惊动。胜似襄王一梦中,胜似襄王一梦中。
\]

唱毕,把个西门庆喜欢的没入脚处。分付玳安回马家去,晚夕就在李桂卿房里歇了一宿。紧着西门庆要梳笼这女子,又被应伯爵、谢希大两个一力撺掇,就上了道儿。次日,使小厮往家去拿五十两银子,段铺内讨四件衣裳,要梳笼桂姐。那李娇儿听见要梳笼他的侄女儿,如何不喜?连忙拿了一锭大元宝付与玳安,拿到院中打头面,做衣服,定桌席,吹弹歌舞,花攒锦簇,饮三日喜酒。应伯爵、谢希大又约会了孙寡嘴、祝实念、常峙节,每人出五分分子,都来贺他。铺的盖的都是西门庆出。每日大酒大肉,在院中玩耍,不在话下。

\[
舞裙歌板逐时新,散尽黄金只此身。
寄语富儿休暴殄,俭如良药可医贫。
\]

\newpage
%# -*- coding:utf-8 -*-
%%%%%%%%%%%%%%%%%%%%%%%%%%%%%%%%%%%%%%%%%%%%%%%%%%%%%%%%%%%%%%%%%%%%%%%%%%%%%%%%%%%%%


\chapter{潘金莲私仆受辱\KG 刘理星魇胜求财}


诗曰:

\[
可怜独立树,枝轻根亦摇。虽为露所浥,复为风所飘。
锦衾襞不开,端坐夜及朝。是妾愁成瘦,非君重细腰。
\]

话说西门庆在院中贪恋桂姐姿色,约半月不曾来家。吴月娘使小厮拿马接了数次,李家把西门庆衣帽都藏过,不放他起身。丢的家中这些妇人都闲静了。别人犹可,惟有潘金莲这妇人,青春未及三十岁,欲火难禁一丈高。每日打扮的粉妆玉琢,皓齿朱唇,无日不在大门首倚门而望,只等到黄昏。到晚来归入房中,粲枕孤帏,凤台无伴,睡不着,走来花园中,款步花苔。看见那月洋水底,便疑西门庆情性难拿;偶遇着玳瑁猫儿交欢,越引逗的他芳心迷乱。当时玉楼带来一个小厮,名唤琴童,年约十六岁,才留起头发,生的眉目清秀,乖滑伶俐。西门庆教他看管花园,晚夕就在花园门首一间小耳房内安歇。金莲和玉楼白日里常在花园亭子上一处做针指或下棋。这小厮专一献小殷勤,常观见西门庆来,就先来告报。以此妇人喜他,常叫他入房,赏酒与他吃。两个朝朝暮暮,眉来眼去,都有意了。

不想到了七月,西门庆生日将近。吴月娘见西门庆留恋烟花,因使玳安拿马去接。这潘金莲暗暗修了一柬帖,交付玳安,教:“悄悄递与你爹,说五娘请爹早些家去罢。”这玳安儿一直骑马到李家,只见应伯爵、谢希大、祝实念,孙寡嘴,常峙节众人,正在那里伴着西门庆,搂着粉头欢乐饮酒。西门庆看见玳安来到,便问:“你来怎麽?家中没事?”玳安道:“家中没事。”西门庆道:“前边各项银子,叫傅二叔讨讨,等我到家算帐。”玳安道:“这两日傅二叔讨了许多,等爹到家上帐。”西门庆道:“你桂姨那一套衣服,捎来不曾?”玳安道:“已捎在此。”便向毡包内取出一套红衫蓝裙,递与桂姐。桂姐道了万福,收了,连忙分付下边,管待玳安酒饭。那小厮吃了酒饭,复走来上边伺候。悄悄向西门庆耳边说道:“五娘使我捎了个帖儿在此。请爹早些家去。”西门庆才待用手去接,早被李桂姐看见,只道是西门庆那个表子寄来的情书,一手挝过来,拆开观看,却是一幅回文锦笺,上写着几行墨迹。桂姐递与祝实念,教念与他听。这祝实念见上面写词一首,名《落梅风》,念道:

\[
黄昏想,白日思,盼杀人多情不至。因他为他憔悴死,可怜也绣衾独自!灯将残,人睡也,空留得半窗明月。眠心硬,浑似铁,这凄凉怎捱今夜?
\]
下书:“爱妾潘六儿拜。”那桂姐听毕,撇了酒席,走入房中,倒在床上,面朝里边睡了。西门庆见桂姐恼了,把帖子扯的稀烂,众人前把玳安踢了两脚。请桂姐两遍不来,慌的西门庆亲自进房,抱出他来,说道:“分付带马回去,家中那个淫妇使你来,我这一到家,都打个臭死!”玳安只得含泪回家。西门庆道:“桂姐,你休恼,这帖子不是别人的,乃是我第五个小妾寄来,请我到家有些事儿计较,再无别故。”祝实念在旁戏道:“桂姐,你休听他哄你哩!这个潘六儿乃是那边院里新叙的一个表子,生的一表人物。你休放他去。”西门庆笑赶着打,说道:“你这贱天杀的,单管弄死了人,紧着他恁麻犯人,你又胡说。”李桂卿道:“姐夫差了,既然家中有人拘管,就不消梳笼人家粉头,自守着家里的便了。才相伴了多少时,便就要抛离了去。”应伯爵插口道:“说的有理。你两人都依我,大官人也不消家去,桂姐也不必恼。今日说过,那个再恁,每人罚二两银子,买酒咱大家吃。”于是西门庆把桂姐搂在怀中陪笑,一递一口儿饮酒。少倾,拿了七锺茶来,馨香可掬,每人面前一盏。应伯爵道:“我有个曲儿,单道这茶好处:

\[
\cipaim{朝天子}这细茶的嫩芽,生长在春风下。不揪不采叶儿楂,但煮着颜色大。绝品清奇,难描难画。口里儿常时呷,醉了时想他,醒来时爱他。原来一篓儿千金价。”
\]

谢希大笑道:“大官人使钱费物,不图这‘一搂儿’,却图些甚的?如今每人有词的唱词,不会词,每人说个笑话儿,与桂姐下酒。”就该谢希大先说,因说道:“有一个泥水匠,在院中墁地。老妈儿怠慢了他,他暗把阴沟内堵上块砖。落后天下雨,积的满院子都是水。老妈慌了,寻的他来,多与他酒饭,还秤了一钱银子,央他打水平。那泥水匠吃了酒饭,悄悄去阴沟内把那块砖拿出,那水登时出的罄尽。老妈便问作头:‘此是那里的病?’泥水匠回道:‘这病与你老人家的病一样,有钱便流,无钱不流。’”桂姐见把他家来伤了,便道:“我也有个笑话,回奉列位。有一孙真人,摆着筵席请人,却教座下老虎去请。那老虎把客人都路上一个个吃了。真人等至天晚,不见一客到。不一时老虎来,真人便问:‘你请的客人都那里去了?’老虎口吐人言:‘告师父得知,我从来不晓得请人,只会白嚼人。’”当下把众人都伤了。应伯爵道:“可见的俺们只是白嚼,你家孤老就还不起个东道?”于是向头上拨下一根闹银耳斡儿来,重一钱;谢希大一对镀金网巾圈,秤了秤重九分半;祝实念袖中掏出一方旧汗巾儿,算二百文长钱;孙寡嘴腰间解下一条白布裙,当两壶半酒;常峙节无以为敬,问西门庆借了一钱银子。都递与桂卿,置办东道,请西门庆和桂姐。那桂卿将银钱都付与保儿,买了一钱猪肉,又宰了一只鸡,自家又陪些小菜儿,安排停当。大盘小碗拿上来,众人坐下,说了一声动箸吃时,说时迟,那时快,但见:

\[
人人动嘴,个个低头。遮天映日,犹如蝗蚋一齐来;挤眼掇肩,好似饿牢才打出。这个抢风膀臂,如经年未见酒和肴;那个连三筷子,成岁不筵与席。一个汗流满面,却似与鸡骨秃有冤仇;一个油抹唇边,把猪毛皮连唾咽。吃片时,杯盘狼藉;啖顷刻,箸子纵横。这个称为食王元帅,那个号作净盘将军。酒壶番晒又重斟,盘馔已无还去探。正是:珍羞百味片时休,果然都送入五脏庙。
\]
当下众人吃得个净光王佛。西门庆与桂姐吃不上两锺酒,拣了些菜蔬,又被这伙人吃去了。那日把席上椅子坐折了两张,前边跟马的小厮,不得上来掉嘴吃,把门前供养的土地翻倒来,便剌了一泡屯谷都的热屎。临出门来,孙寡嘴把李家明间内供养的镀金铜佛,塞在裤腰里;应伯爵推斗桂姐亲嘴,把头上金琢针儿戏了;谢希大把西门庆川扇儿藏了;祝实念走到桂卿房里照面,溜了他一面水银镜子。常峙节借的西门庆一钱银子,竞是写在嫖账上了。原来这起人,只伴着西门庆玩耍,好不快活。有诗为证:

\[
工妍掩袖媚如猱,乘兴闲来可暂留。
若要死贪无厌足,家中金钥教谁收?
\]

按下众人簇拥着西门庆饮酒不题。单表玳安回马到家,吴月娘和孟玉楼、潘金莲正在房坐的,见了便问玳安:“你去接爹来了不曾?”玳安哭的两眼红红的,说道:被爹踢骂了小的来了。爹说那个再使人接,来家都要骂。”月娘便道:“你看恁不合理,不来便了,如何又骂小厮?”孟玉楼道:“你踢将小厮便罢了,如何连俺们都骂将来?”潘金莲道:“十个九个院中淫妇,和你有甚情实!常言说的好:船载的金银,填不满烟花寨。”金莲只知说出来,不防李娇儿见玳安自院中来家,便走来窗下潜听。见金莲骂他家千淫妇万淫妇,暗暗怀恨在心。从此二人结仇,不在话下。正是:

\[
甜言美语三冬暖,恶语伤人六月寒。
\]

不说李娇儿与潘金莲结仇。单表金莲归到房中,捱一刻似三秋,盼一时如半夏。知道西门庆不来家,把两个丫头打发睡了,推往花园中游玩,将琴童叫进房与他酒吃。把小厮灌醉了,掩上房门,褪衣解带,两个就干做一处。但见:

\[
一个不顾纲常贵贱,一个那分上下高低。一个色胆歪邪,管甚丈夫利害;一个淫心荡漾,纵他律法明条。百花园内,翻为快活排场;主母房中,变作行乐世界。霎时一滴驴精髓,倾在金莲玉体中。
\]
自此为始,每夜妇人便叫琴童进房如此。未到天明,就打发出来。背地把金裹头簪子两三根带在头上,又把裙边带的锦香囊葫芦儿也与了他。岂知这小厮不守本分,常常和同行小厮街上吃酒耍钱,颇露机关。常言:若要不知,除非莫为。有一日,风声吹到孙雪娥、李娇儿耳朵内,说道:“贼淫妇,往常假撇清,如何今日也做出来了?”齐来告月娘。月娘再三不信,说道:“不争你们和他合气,惹的孟三姐不怪?只说你们挤撮他的小厮。”说的二人无言而退。落后妇人夜间和小厮在房中行事,忘记关厨房门,不想被丫头秋菊出来净手,看见了。次日传与后边小玉,小玉对雪娥说。雪娥同李娇儿又来告诉月娘如此这般:“他屋里丫头亲口说出来,又不是俺们葬送他。大娘不说,俺们对他爹说。若是饶了这个淫妇,非除饶了蝎子!”

此时正值七月二十七日,西门庆从院中来家上寿。月娘道:“他才来家,又是他好日子,你们不依我,只顾说去!等他反乱将起来,我不管你。”二人不听月娘,约的西门庆进入房中,齐来告诉金莲在家怎的养小厮一节。这西门庆不听万事皆休,听了怒从心上起,恶向胆边生。走到前边坐下,一片声叫琴童儿。早有人报与潘金莲。金莲慌了手脚,使春梅忙叫小厮到房中,嘱咐千万不要说出来,把头上簪子都拿过来收了。着了慌,就忘解了香囊葫芦下来。被西门庆叫到前厅跪下,分付三四个小厮,选大板子伺候。西门庆道:“贼奴才,你知罪么?”那琴童半日不敢言语。西门庆令左右:“拨下他簪子来,我瞧!”见没了簪子,因问:“你戴的金裹头银簪子,往那里去了?”琴童道:“小的并没甚银簪子。”西门庆道:“奴才还捣鬼!与我旋剥了衣服,拿板子打!”当下两三个小厮扶侍一个,剥去他衣服,扯了裤子。见他身底下穿着玉色绢裈儿,裈儿带上露出锦香囊葫芦儿。西门庆一眼看见,便叫:“拿上来我瞧!”认的是潘金莲裙边带的物件,不觉心中大怒,就问他:“此物从那里得来?你实说是谁与你的?”唬的小厮半日开口不得,说道:“这是小的某日打扫花园,在花园内拾的。并不曾有人与我。”西门庆越怒,切齿喝令:“与我捆起来着实打!”当下把琴童绷子绷着,打了三十大棍,打得皮开肉绽,鲜血顺腿淋漓。又叫来保:“把奴才两个鬓毛与我撏了!赶将出去,再不许进门!”那琴童磕了头,哭哭啼啼出门去了。

潘金莲在房中听见,如提冷水盆内一般。不一时,西门庆进房来,吓的战战兢兢,浑身无了脉息,小心在旁扶侍接衣服,被西门庆兜脸一个耳刮子,把妇人打了一交。分付春梅:“把前后角门顶了,不放一个人进来!”拿张小椅儿,坐在院内花架儿底下,取了一根马鞭子,拿在手里,喝令:“淫妇,脱了衣裳跪着!”那妇人自知理亏,不敢不跪,真个脱去了上下衣服,跪在面前,低垂粉面,不敢出一声儿。西门庆便问:“贼淫妇,你休推梦里睡里,奴才我已审问明白,他一一都供出来了。你实说,我不在家,你与他偷了几遭?”妇人便哭道:“天那,天那!可不冤屈杀了我罢了!自从你不在家半个来月,奴白日里只和孟三儿一处做针指,到晚夕早关了房门就睡了。没勾当,不敢出这角门边儿来。你不信,只问春梅便了。有甚和盐和醋,他有个不知道的?”因叫春梅:“姐姐你过来,亲对你爹说。”西门庆骂道:“贼淫妇!有人说你把头上金裹头簪子两三根都偷与了小厮,你如何不认?”妇人道:“就屈杀了奴罢了!是那个不逢好死的嚼舌根的淫妇,嚼他那旺跳身子。见你常时进奴这屋里来歇,无非都气不愤,拿这有天没日头的事压枉奴。就是你与的簪子,都有数儿,一五一十都在,你查不是!我平白想起甚么来与那奴才?好成材的奴才,也不枉说的,恁一个尿不出来的毛奴才,平空把我篡一篇舌头!”西门庆道:“簪子有没罢了。”因向袖中取出那香囊来,说道:“这个是你的物件儿,如何打小厮身底下捏出来?你还口强甚么?”说着纷纷的恼了,向他白馥馥香肌上,飕的一马鞭子来,打的妇人疼痛难忍,眼噙粉泪,没口子叫道:“好爹爹,你饶了奴罢!你容奴说便说,不容奴说,你就打死了奴,也只臭烂了这块地。这个香囊葫芦儿,你不在家,奴那日同孟三姐在花园里做生活,因从木香棚下过,带儿系不牢,就抓落在地,我那里没寻,谁知这奴才拾了。奴并不曾与他。”只这一句,就合着琴童供称一样的话,又见妇人脱的光赤条条,花朵儿般身子,娇啼嫩语,跪在地下,那怒气早已钻入爪洼国去了,把心已回动了八九分,因叫过春梅,搂在怀中,问他:“淫妇果然与小厮有首尾没有?你说饶了淫妇,我就饶了罢。”那春梅撒娇撒痴,坐在西门庆怀里,说道:“这个,爹你好没的说!我和娘成日唇不离腮,娘肯与那奴才?这个都是人气不愤俺娘儿们,做作出这样事来。爹,你也要个主张,好把丑名儿顶在头上,传出外边去好听?”几句把西门庆说的一声儿没言语,丢了马鞭子,一面叫金莲起来,穿上衣服,分付秋菊看菜儿,放桌儿吃酒。这妇人满斟了一杯酒,双手递上去,跪在地下,等他锺儿。西门庆分付道:“我今日饶了你。我若但凡不在家,要你洗心改正,早关了门户,不许你胡思乱想。我若知道,并不饶你!”妇人道:“你分付,奴知道了。”又与西门庆磕了四个头,方才安坐儿,在旁陪坐饮酒。潘金莲平日被西门庆宠的狂了,今日讨这场羞辱在身上。正是:

\[
为人莫作妇人身,百年苦乐由他人。
\]

当下西门庆正在金莲房中饮酒,忽小厮打门,说:“前边有吴大舅、吴二舅、傅伙计、女儿、女婿,众亲戚送礼来祝寿。”方才撇了金莲,出前边陪待宾客。那时应伯爵、谢希大众人都有人情,院中李桂姐家亦使保儿送礼来。西门庆前边乱着收人家礼物,发柬请人,不在话下。

且说孟玉楼打听金莲受辱,约的西门庆不在房里,瞒着李娇儿、孙雪娥,走来看望。见金莲睡在床上,因问道:“六姐,你端的怎么缘故?告我说则个。”那金莲满眼流泪哭道:“三姐,你看小淫妇,今日在背地里白唆调汉子,打了我恁一顿。我到明日,和这两个淫妇冤仇结得有海深。”玉楼道:“你便与他有瑕玷,如何做作着把我的小厮弄出去了?六姐,你休烦恼,莫不汉子就不听俺们说句话儿?若明日他不进我房里来便罢,但到我房里来,等我慢慢劝他。”金莲道:“多谢姐姐费心。”一面叫春梅看茶来吃。坐着说了回话,玉楼告回房去了。至晚,西门庆因上房吴大妗子来了,走到玉楼房中宿歇。玉楼因说道:“你休枉了六姐心,六姐并无此事,都是日前和李娇儿、孙雪娥两个有言语,平白把我的小厮扎罚了。你不问个青红皂白,就把他屈了,却不难为他了!我就替他赌个大誓,若果有此事,大姐姐有个不先说的?”西门庆道:“我问春梅,他也是这般说。”玉楼道:“他今在房中不好哩,你不去看他看去?”西门庆道:“我知道,明日到他房中去。”当晚无话。

到第二日,西门庆正生日。有周守备、夏提刑、张团练、吴大舅许多官客饮酒,拿轿子接了李桂姐并两个唱的,唱了一日。李娇儿见他侄女儿来,引着拜见月娘众人,在上房里坐吃茶。请潘金莲见,连使丫头请了两遍,金莲不出来,只说心中不好。到晚夕,桂姐临家去,拜辞月娘。月娘与他一件云绢比甲儿、汗巾花翠之类,同李娇儿送出门首。桂姐又亲自到金莲花园角门首:“好歹见见五娘。”那金莲听见他来,使春梅把角门关得铁桶相似,说道:“娘分付,我不敢开。”这花娘遂羞讪满面而回,不题。

单表西门庆至晚进入金莲房内来,那金莲把云鬓不整,花容倦淡,迎接进房,替他脱衣解带,伺候茶汤脚水,百般殷勤扶侍。到夜里枕席欢娱,屈身忍辱,无所不至,说道:“我的哥哥,这一家谁是疼你的?都是露水夫妻,再醮货儿。惟有奴知道你的心,你知道奴的意。旁人见你这般疼奴,在奴身边的多,都气不愤,背地里驾舌头,在你跟前唆调。我的傻冤家!你想起甚么来,中人的拖刀之计,把你心爱的人儿这等下无情的折挫!常言道:家鸡打的团团转,野鸡打的贴天飞。你就把奴打死了,也只在这屋里。就是前日你在院里踢骂了小厮来,早是有大姐姐、孟三姐在跟前,我自不是说了一声,恐怕他家粉头掏渌坏了你身子,院中唱的一味爱钱,有甚情节?谁人疼你?谁知被有心的人听见,两个背地做成一帮儿算计我。自古人害人不死,天害人才害死了。往后久而自明,只要你与奴做个主儿便了。”几句把西门庆窝盘住了。是夜与他淫欲无度。

过了几日,西门庆备马,玳安、平安两个跟随,往院中来。却说李桂姐正打扮着陪人坐的,听见他来,连忙走进房去,洗了浓妆,除了簪环,倒在床上裹衾而卧。西门庆走到,坐了半日,老妈才出来,道了万福,让西门庆坐下,问道:“怎的姐夫连日不进来走走?”西门庆道:“正是因贱日穷冗,家中无人。”虔婆道:“姐儿那日打搅。”西门庆道:“怎的那日桂卿不来走走?”虔婆道:“桂卿不在家,被客人接去店里。这几日还不放了来。”说了半日话,才拿茶来陪着吃了。西门庆便问:“怎的不见桂姐?”虔婆道:“姐夫还不知哩,小孩儿家,不知怎的,那日着了恼,来家就不好起来,睡倒了。房门儿也不出,直到如今。姐夫好狠心,也不来看看姐儿。”西门庆道:“真个?我通不知。”因问:“在那边房里?我看看去。”虔婆道:“在他后边卧房里睡。”慌忙令丫鬟掀帘子。西门庆走到他房中,只见粉头乌云散乱,粉面慵妆,裹被坐在床上,面朝里,见了西门庆,不动一动儿。西门庆道:“你那日来家,怎的不好?”也不答应。又问:“你着了谁人恼,你告我说。”问了半日,那桂姐方开言说道:“左右是你家五娘子。你家中既有恁好的迎欢卖俏,又来稀罕俺们这样淫妇做甚么?俺们虽是门户中出身,跷起脚儿,比外边良人家不成的货色儿高好些!我前日又不是供唱,我也送人情去。大娘到见我甚是亲热,又与我许多花翠衣服。待要不请他见,又说俺院中没礼法。闻说你家有五娘子,当即请他拜见,又不出来。家来同俺姑娘又辞他去,他使丫头把房门关了。端的好不识人敬重!”西门庆道:“你到休怪他。他那日本等心中不自在,他若好时,有个不出来见你的?这个淫妇,我几次因他咬群儿,口嘴伤人,也要打他哩!”桂姐反手向西门庆脸上一扫,说道:“没羞的哥儿,你就打他?”西门庆道:“你还不知我手段,除了俺家房下,家中这几个老婆丫头,但打起来也不善,着紧二三十马鞭子还打不下来。好不好还把头发都剪了。”桂姐道:“我见砍头的,没见吹嘴的,你打三个官儿,唱两个喏,谁见来?你若有本事,到家里只剪下一柳子头发,拿来我瞧,我方信你是本司三院有名的子弟。”西门庆道:“你敢与我排手?”那桂姐道:“我和你排一百个手。”当日西门庆在院中歇了一夜,到次日黄昏时分,辞了桂姐,上马回家。桂姐道:“哥儿,你这一去,没有这物件儿,看你拿甚嘴脸见我!”

这西门庆吃他激怒了几句话,归家已是酒酣,不往别房里去,迳到潘金莲房内来。妇人见他有酒了,加意用心伏侍。问他酒饭都不吃。分付春梅把床上枕席拭抹干净,带上门出去。他便坐在床上,令妇人脱靴。那妇人不敢不脱。须臾,脱了靴,打发他上床。西门庆且不睡,坐在一只枕头上,令妇人褪了衣服,地下跪着。那妇人吓的捏两把汗,又不知因为甚么,于是跪在地下,柔声痛哭道:“我的爹爹!你透与奴个伶俐说话,奴死也甘心。饶奴终日恁提心吊胆,陪着一千个小心,还投不着你的机会,只拿钝刀子锯处我,教奴怎生吃受?”西门庆骂道:“贱淫妇,你真个不脱衣裳,我就没好意了!”因叫春梅:“门背后有马鞭子,与我取了来!”那春梅只顾不进房来,叫了半日,才慢条厮礼推开房门进来。看见妇人跪在床地平上,向灯前倒着桌儿下,由西门庆使他,只不动身。妇人叫道:“春梅,我的姐姐,你救我救儿,他如今要打我。”西门庆道:“小油嘴儿,你不要管他。你只递马鞭子与我打这淫妇。”春梅道:“爹,你怎的恁没羞!娘干坏了你甚么事儿?你信淫妇言语,平地里起风波,要便搜寻娘?还教人和你一心一计哩!你教人有那眼儿看得上你!倒是我不依你。”拽上房门,走在前边去了。那西门庆无法可处,倒呵呵笑了,向金莲道:“我且不打你。你上来,我问你要椿物儿,你与我不与我?”妇人道:“好亲亲,奴一身骨朵肉儿都属了你,随要甚么,奴无有不依随的。不知你心里要甚么儿?”西门庆道:“我要你顶上一柳儿好头发。”妇人道:“好心肝!奴身上随你怎的拣着烧遍了也依,这个剪头发却依不的,可不吓死了我罢了。奴出娘胞儿,活了二十六岁,从没干这营生。打紧我顶上这头发近来又脱了好些,只当可怜见我罢。”西门庆道:“你只怪我恼,我说的你就不依。”妇人道:“我不依你,再依谁?”因问:“你实对奴说,要奴这头发做甚么?”西门庆道:“我要做网巾。”妇人道:“你要做网巾,奴就与你做,休要拿与淫妇,教他好压镇我。”西门庆道:“我不与人便了,要你发儿做顶线儿。”妇人道:“你既要做顶线,待奴剪与你。”当下妇人分开头发,西门庆拿剪刀,按妇人顶上,齐臻臻剪下一大柳来,用纸包放在顺袋内。妇人便倒在西门庆怀中,娇声哭道:“奴凡事依你,只愿你休忘了心肠,随你前边和人好,只休抛闪了奴家!”是夜与他欢会异常。

到次日,西门庆起身,妇人打发他吃了饭,出门骑马,迳到院里。桂姐便问:“你剪的他头发在那里?”西门庆道:“有,在此。”便向茄袋内取出,递与桂姐。打开看,果然黑油也一般好头发,就收在袖中。西门庆道:“你看了还与我,他昨日为剪这头发,好不烦难,吃我变了脸恼了,他才容我剪下这一柳子来。我哄他,只说要做网巾顶线儿,迳拿进来与你瞧。可见我不失信。”桂姐道:“甚么稀罕货,慌的恁个腔儿!等你家去,我还与你。比是你恁怕他,就不消剪他的来了。”西门庆笑道:“那里是怕他!恁说我言语不的了。”桂姐一面叫桂卿陪着他吃酒,走到背地里,把妇人头发早絮在鞋底下,每日踹踏,不在话下。却把西门庆缠住,连过了数日,不放来家。

金莲自从头发剪下之后,觉道心中不快,每日房门不出,茶饭慵餐。吴月娘使小厮请了家中常走看的刘婆子来看视,说:“娘子着了些暗气,恼在心中,不能回转,头疼恶心,饮食不进。”一面打开药包来,留了两服黑丸子药儿:“晚上用姜汤吃。”又说:“我明日叫我老公来,替你老人家看看今岁流年,有灾没灾。”金莲道:“原来你家老公也会算命?”刘婆道:“他虽是个瞽目人,到会两三椿本事:第一善阴阳算命,与人家禳保;第二会针灸收疮;第三椿儿不可说,——单管与人家回背。”妇人问道:“怎么是回背?”刘婆子道:“比如有父子不和,兄弟不睦,大妻小妻争斗,教了俺老公去说了,替他用镇物安镇,画些符水与他吃了,不消三日,教他父子亲热,兄弟和睦,妻妾不争。若人家买卖不顺溜,田宅不兴旺者,常与人开财门发利市。治病洒扫,禳星告斗都会。因此人都叫他做刘理星。也是一家子,新娶个媳妇儿是小人家女儿,有些手脚儿不稳,常偷盗婆婆家东西往娘家去。丈夫知道,常被责打。俺老公与他回背,画了一道符,烧灰放在水缸下埋着,合家大小吃了缸内水,眼看媳妇偷盗,只象没看见一般。又放一件镇物在枕头内,男子汉睡了那枕头,好似手封住了的,再不打他了。”那金莲听见遂留心,便呼丫头,打发茶汤点心与刘婆吃。临去,包了三钱药钱,另外又秤了五钱,要买纸扎信信物。明日早饭时叫刘瞎来烧神纸。那婆子作辞回家。

到次日,果然大清早晨,领贼瞎迳进大门往里走。那日西门庆还在院中,看门小厮便问:“瞎子往那里走?”刘婆道:“今日与里边五娘烧纸。”小厮道:“既是与五娘烧纸,老刘你领进去。仔细看狗。”这婆子领定,迳到潘金莲卧房明间内,等了半日,妇人才出来。瞎子见了礼,坐下。妇人说与他八字,贼瞎用手捏了捏,说道:“娘子庚辰年,庚寅月,乙亥日,己丑时。初八日立春,已交正月算命。依子平正论,娘子这八字,虽故清奇,一生不得夫星济,子上有些防碍。乙木生在正月间,亦作身旺论,不克当自焚。又两重庚金,羊刃大重,夫星难为,克过两个才好。”妇人道:“已克过了。”贼瞎子道:“娘子这命中,休怪小人说,子平虽取煞印格,只吃了亥中有癸水,丑中又有癸水,水太多了,冲动了只一重巳土,官煞混杂。论来,男人煞重掌威权,女子煞重必刑夫。所以主为人聪明机变,得人之宠。只有一件,今岁流年甲辰,岁运并临,灾殃立至。命中又犯小耗勾绞,两位星辰打搅,虽不能伤,却主有比肩不和,小人嘴舌,常沾些啾唧不宁之状。”妇人听了,说道:“累先生仔细用心,与我回背回背。我这里一两银子相谢先生,买一盏茶吃。奴不求别的,只愿得小人离退,夫主爱敬便了。”一面转入房中,拔了两件首饰递与贼瞎。贼瞎收入袖中,说道:“既要小人回背,用柳木一块,刻两个男女人形,书着娘子与夫主生辰八字,用七七四十九根红线扎在一处。上用红纱一片,蒙在男子眼中,用艾塞其心,用针钉其手,下用胶粘其足,暗暗埋在睡的枕头内。又朱砂书符一道烧灰,暗暗搅茶内。若得夫主吃了茶,到晚夕睡了枕头,不过三日,自然有验。”妇人道:“请问先生,这四椿儿是怎的说?”贼瞎道:“好教娘子得知:用纱蒙眼,使夫主见你一似西施娇艳;用艾塞心,使他心爱到你;用针钉手,随你怎的不是,使他再不敢动手打你;用胶粘足者,使他再不往那里胡行。”妇人听言,满心欢喜。当下备了香烛纸马,替妇人烧了纸。到次日,使刘婆送了符水镇物与妇人,如法安顿停当,将符烧灰,顿下好茶,待的西门庆家来,妇人叫春梅递茶与他吃。到晚夕,与他共枕同床,过了一日两,两日三,似水如鱼,欢会异常。看观听说:但凡大小人家,师尼僧道,乳母牙婆,切记休招惹他,背地什么事不干出来?古人有四句格言说得好:

\[
堂前切莫走三婆,后门常锁莫通和。
院内有井防小口,便是祸少福星多。
\]

\newpage
%# -*- coding:utf-8 -*-
%%%%%%%%%%%%%%%%%%%%%%%%%%%%%%%%%%%%%%%%%%%%%%%%%%%%%%%%%%%%%%%%%%%%%%%%%%%%%%%%%%%%%


\chapter{李瓶姐墙头密约\KG 迎春儿隙底私窥}


词曰:

\[
绣面芙蓉一笑开,斜飞宝鸭衬香腮。眼波才动被人猜。一面风情深有韵,半笺娇恨寄幽怀。月移花影约重来。
\]

话说一日西门庆往前边走来,到月娘房中。月娘告说:“今日花家使小厮拿帖来,请你吃酒。”西门庆观看帖子,写着:“即午院中吴银家一叙,希即过我同往,万万!”少顷,打选衣帽,叫了两个跟随,骑匹骏马,先迳到花家。不想花子虚不在家了。他浑家李瓶儿,夏月间戴着银丝鬒髻,金镶紫瑛坠子,藕丝对衿衫,白纱挑线镶边裙,裙边露一对红鸳凤嘴尖尖翘翘小脚,立在二门里台基上。那西门庆三不知走进门,两下撞了个满怀。这西门庆留心已久,虽故庄上见了一面,不曾细玩。今日对面见了,见他生的甚是白净,五短身才,瓜子面儿,细湾湾两道眉儿,不觉魂飞天外,忙向前深深作揖。妇人还了万福,转身入后边去了。使出一个头发齐眉的丫鬟来,名唤绣春,请西门庆客位内坐。他便立在角门首,半露娇容说:“大官人少坐一时。他适才有些小事出去了,便来也。”丫鬟拿出一盏茶来,西门庆吃了。妇人隔门说道:“今日他请大官人往那边吃酒去,好歹看奴之面,劝他早些回家。两个小厮又都跟去了,止是这两个丫鬟和奴,家中无人。”西门庆便道:“嫂子见得有理,哥家事要紧。嫂子既然分付在下,在下一定伴哥同去同来。”

正说着,只见花子虚来家,妇人便回房去了。花子虚见西门庆叙礼说道:“蒙哥下降,小弟适有些不得已小事出去,失迎,恕罪!”于是分宾主坐下,便叫小厮看茶。须臾,茶罢。又分付小厮:“对你娘说,看菜儿来,我和西门爹吃三杯起身。今日六月二十四,是院内吴银姐生日,请哥同往一乐。”西门庆道:“二哥何不早说?”即令玳安:“快家去,讨五钱银子封了来。”花子虚道:“哥何故又费心?小弟到不是了。”西门庆见左右放桌儿,说道:“不消坐了,咱往里边吃去罢。”花子虚道:“不敢久留,哥略坐一回。”少倾,就是齐整肴馔拿将上来,银高脚葵花锺,每人三锺,又是四个卷饼,吃毕收下来与马上人吃。

少倾,玳安取了分资来,一同起身上马,迳往吴四妈家与吴银儿做生日。到那里,花攒锦簇,歌舞吹弹,饮酒至一更时分方散。西门庆留心,把子虚灌得酩酊大醉。又因李瓶儿央浼之言,相伴他一同来家。小厮叫开大门,扶到他客位坐下。李瓶儿同丫鬟掌着灯烛出来,把子虚搀扶进去。

西门庆交付明白,就要告回。妇人旋走出来,拜谢西门庆,说道:“拙夫不才贪酒,多累看奴薄面,姑待来家,官人休要笑话。”那西门庆忙屈身还喏,说道:“不敢。嫂子这里分付,在下敢不铭心刻骨,同哥一搭里来家!非独嫂子耽心,显的在下干事不的了。方才哥在他家,被那些人缠住了,我强着催哥起身。走到乐星堂儿门首粉头郑爱香儿家,——小名叫做郑观音,生的一表人物,哥就要往他家去,被我再三拦住,劝他说道:‘恐怕家中嫂子放心不下。’方才一直来家。若到郑家,便有一夜不来。嫂子在上,不该我说,哥也糊涂,嫂子又青年,偌大家室,如何就丢了,成夜不在家?是何道理!”妇人道:“正是如此,奴为他这等在外胡行,不听人说,奴也气了一身病痛在这里。往后大官人但遇他在院中,好歹看奴薄面,劝他早早回家。奴恩有重报,不敢有忘。”这西门庆是头上打一下脚底板响的人,积年风月中走,甚么事儿不知道?今日妇人到明明开了一条大路,教他入港,岂不省腔!于是满面堆笑道:“嫂子说那里话!相交朋友做甚么?我一定苦心谏哥,嫂子放心。”妇人又道了万福,又叫小丫鬟拿了一盏果仁泡茶来。西门庆吃毕茶,说道:“我回去罢,嫂子仔细门户。”遂告辞归家。

自此西门庆就安心设计,图谋这妇人,屡屡安下应伯爵、谢希大这伙人,把子虚挂住在院里饮酒过夜。他便脱身来家,一径在门首站立。这妇人亦常领着两个丫鬟在门首。西门庆看见了,便扬声咳嗽,一回走过东来,又往西去,或在对门站立,把眼不住望门里睃盼。妇人影身在门里,见他来便闪进里面,见他过去了,又探头去瞧。两个眼意心期,已在不言之表。一日,西门庆正站在门首,忽见小丫鬟绣春来请。西门庆故意问道:“姐姐请我做甚么?你爹在家里不在?”绣春道:“俺爹不在家,娘请西门庆爹问句话儿。”这西门庆得不的一声,连忙走过来,到客位内坐下。良久,妇人出来,道了万福,便道:“前日多承官人厚意,奴铭刻于心,知感不尽。他从昨日出去,一连两日不来家了,不知官人曾会见他来不曾?”西门庆道:“他昨日同三四个在郑家吃酒,我偶然有些小事就来了。今日我不曾得进去,不知他还在那里没在。若是我在那里,恐怕嫂子忧心,有个不催促哥早早来家的?”妇人道:“正是这般说。奴吃煞他不听人说、在外边眠花卧柳不顾家事的亏。”西门庆道:“论起哥来,仁义上也好,只是有这一件儿。”说着,小丫鬟拿茶来吃了。西门庆恐子虚来家,不敢久恋,就要告归。妇人又千叮万嘱,央西门庆:“不拘到那里,好歹劝他早来家,奴一定恩有重报,决不敢忘官人!”西门庆道:“嫂子没的说,我与哥是那样相交!”说毕,西门庆家去了。

到次日,花子虚自院中回家,妇人再三埋怨说道:“你在外边贪酒恋色,多亏隔壁西门大官人,两次三番顾睦你来家。你买分礼儿谢谢他,方不失了人情。”那花子虚连忙买了四盒礼物,一坛酒,使小厮天福儿送到西门庆家。西门庆收下,厚赏来人去了。吴月娘便问说:“花家如何送你这礼?”西门庆道:“花二哥前日请我们在院中与吴银儿做生日,醉了,被我搀扶了他来家;又见常时院中劝他休过夜,早早来家。他娘子儿因此感我的情,想对花二哥说,故买此礼来谢我。”吴月娘听了,与他打个问讯,说道:“我的哥哥,你自顾了你罢,又泥佛劝土佛!你也成日不着个家,在外养女调妇,反劝人家汉子!”又道:“你莫不白受他这礼?”因问:“他帖上儿写着谁的名字?若是他娘子的名字,今日写我的帖儿,请他娘子过来坐坐,他也只恁要来咱家走走哩。若是他男子汉名字,随你请不请,我不管你。”西门庆道:“是花二哥名字,我明日请他便了。”次日,西门庆果然治酒,请过花子虚来,吃了一日酒。归家,李瓶儿说:“你不要差了礼数。咱送了他一分礼,他到请你过去吃了一席酒,你改日还该治一席酒请他,只当回席。”

光阴迅速,又早九月重阳。花子虚假着节下,叫了两个妓者,具柬请西门庆过来赏菊。又邀应伯爵、谢希大、祝实念、孙天化四人相陪。传花击鼓,欢乐饮酒。有诗为证:

\[
乌兔循环似箭忙,人间佳节又重阳。
千枝红树妆秋色,三径黄花吐异香。
不见登高乌帽客,还思捧酒绮罗娘。
秀帘琐闼私相觑,从此恩情两不忘。
\]

当日,众人饮酒到掌灯之后,西门庆忽下席来外边解手。不防李瓶儿正在遮槅子边站立偷觑,两个撞了个满怀,西门庆回避不及。妇人走到西角门首,暗暗使绣春黑影里走到西门庆跟前,低声说道:“俺娘使我对西门爹说,少吃酒,早早回家。晚夕,娘如此这般要和西门爹说话哩。”西门庆听了,欢喜不尽。小解回来,到席上连酒也不吃,唱的左右弹唱递酒,只是装醉不吃。看看到一更时分,那李瓶儿不住走来廉外,见西门庆坐在上面,只推做打盹。那应伯爵、谢希大,如同钉在椅子上,白不起身。熬的祝实念、孙寡嘴也去了,他两个还不动。把个李瓶儿急的要不的。西门庆已是走出来,被花子虚再不放,说道:“今日小弟没敬心,哥怎的白不肯坐?”西门庆道:“我本醉了,吃不去。”于是故意东倒西歪,教两个扶归家去了。应伯爵道:“他今日不知怎的,白不肯吃酒,吃了不多酒就醉了。既是东家费心,难为两个姐儿在此,拿大锺来,咱每再周四五十轮,散了罢。”李瓶儿在帘外听见,骂“涎脸的囚根子”不绝。暗暗使小厮天喜儿请下花子虚来,分付说:“你既要与这伙人吃,趁早与我院里吃去。休要在家里聒噪。我半夜三更,熬油费火,我那里耐烦!”花子虚道:“这咱晚我就和他们院里去,也是来家不成,你休再麻犯我。”妇人道:“你去,我不麻犯便了。”这花子虚得不的这一声,走来对众人说:“我们往院里去。”应伯爵道:“真个?休哄我。你去问声嫂子来,咱好起身。”子虚道:“房下刚才已是说了,教我明日来家。”谢希大道:“可是来,自吃应花子这等唠叨。哥刚才已是讨了老脚来,咱去的也放心。”于是连两个唱的,都一齐起身进院。此时已是二更天气,天福儿、天喜儿跟花子虚等三人,从新又到后巷吴银儿家去吃酒不题。

单表西门庆推醉到家,走到金莲房里,刚脱了衣裳,就往前边花园里去坐,单等李瓶儿那边请他。良久,只听得那边赶狗关门。少倾,只见丫鬟迎春黑影影里扒着墙,推叫猫,看见西门庆坐在亭子上,递了话。这西门庆就掇过一张桌凳来踏着,暗暗扒过墙来,这边已安下梯子。李瓶儿打发子虚去了,已是摘了冠儿,乱挽乌云,素体浓妆,立在穿廊下。看见西门庆过来,欢喜无尽,忙迎接进房中。灯烛下,早已安排一桌齐整酒肴果菜,壶内满贮香醪。妇人双手高擎玉斝,亲递与西门庆,深深道个万福:“奴一向感谢官人,蒙官人又费心酬答,使奴家心下不安。今日奴自治了这杯淡酒,请官人过来,聊尽奴一点薄情。又撞着两个天杀的涎脸,只顾坐住了,急的奴要不的。刚才吃我都打发到院里去了。”西门庆道:“只怕二哥还来家么?”妇人道:“奴已分付过夜不来了。两个小厮都跟去了。家里再无一人,只是这两个丫头,一个冯妈妈看门首,他是奴从小儿养娘心腹人。前后门都已关闭了。”西门庆听了,心中甚喜。两个于是并肩叠股,交杯换盏,饮酒做一处。迎春旁边斟酒,绣春往来拿菜儿。吃得酒浓时,锦帐中香熏鸳被,设放珊瑚,两个丫鬟撤开酒桌,拽上门去了。两人上床交欢。

原来大人家有两层窗寮,外面为窗,里面为寮。妇人打发丫鬟出去,关上里面两扇窗寮,房中掌着灯烛,外边通看不见。这迎春丫头,今年已十七岁,颇知事体,见他两个今夜偷期,悄悄向窗下,用头上簪子挺签破窗寮上纸,往里窥觑。端的二人怎样交接?但见:

\[
灯光影里,鲛绡帐中,一个玉臂忙摇,一个金莲高举。一个莺声呖呖,一个燕语喃喃。好似君瑞遇莺娘,犹若宋玉偷神女。山盟海誓,依稀耳中;蝶恋蜂恣,未能即罢。正是:被翻红浪,灵犀一点透酥胸;帐挽银钩,眉黛两弯垂玉脸。
\]

房中二人云雨,不料迎春在窗外,听看得明明白白。听见西门庆问妇人多少青春。李瓶儿道:“奴今年二十三岁。”因问:“他大娘贵庚?”西门庆道:“房下二十六岁了。”妇人道:“原来长奴三岁,到明日买分礼儿过去,看看大娘,只怕不好亲近。”西门庆道:“房下自来好性儿。”妇人又问:“你头里过这边来,他大娘知道不知?倘或问你时,你怎生回答?”西门庆道:“俺房下都在后边第四层房子里,惟有我第五个小妾潘氏,在这前边花园内,独自一所楼房居住,他不敢管我。”妇人道:“他五娘贵庚多少?”西门庆道:“他与大房下同年。”妇人道:“又好了,若不嫌奴有玷,奴就拜他五娘做个姐姐罢。到明日,讨他大娘和五娘的脚样儿来,奴亲自做两双鞋儿过去,以表奴情。”说着,又将头上关顶的金簪儿拨下两根来,替西门庆带在头上,说道:“若在院里,休要叫花子虚看见。”西门庆道:“这理会得。”当下二人如胶似漆,盘桓到五更时分。窗外鸡叫,东方渐白,西门庆恐怕子虚来家,整衣而起,照前越墙而过。两个约定暗号儿,但子虚不在家,这边就使丫鬟在墙头上暗暗以咳嗽为号,或先丢块瓦儿,见这边无人,方才上墙,这边西门庆便用梯凳扒过墙来。两个隔墙酬和,窃玉偷香,不由大门行走,街房邻舍怎的晓得?有诗为证:

\[
月落花阴夜漏长,相逢疑是梦高唐。
夜深偷把银缸照,犹恐憨奴瞰隙光。
\]

却说西门庆扒过墙来,走到潘金莲房里。金莲还睡未起,因问:“你昨日也不知又往那里去了这一夜?也不对奴说一声儿。”西门庆道:“花二哥又使小厮邀我往院里去,吃了半夜酒,才脱身走来家。”金莲虽故信了,还有几分疑影在心。一日,同孟玉楼饭后在花园亭子上做针指,猛可见一块瓦儿打在面前。那孟玉楼低着头纳鞋,没看见。这潘金莲单单把眼四下观看,影影绰绰只见隔壁墙头上一个白面探了一探,就下去了。金莲忙推玉楼,指与他瞧,说道:“三姐姐,你看这个,是隔壁花家那大丫头,想是上墙瞧花儿,看见俺们在这里,他就下去了。”说毕,也就罢了。到晚夕,西门庆自外赴席来家,进金莲房中。金莲与他接了衣裳,问他。饭不吃,茶也不吃,趔趄着脚儿,只往前边花园里走。这潘金莲贼留心,暗暗看着他。坐了好一回,只见先头那丫头在墙头上打了个照面,这西门庆就踏着梯凳过墙去了。那边李瓶儿接入房中,两个厮会不题。

这潘金莲归到房中,翻来复去,通一夜不曾睡。将到天明,只见西门庆过来,推开房门,妇人睡在床上,不理他。那西门庆先带几分愧色,挨近他床上坐下。妇人见他来,跳起来坐着,一手撮着他耳朵,骂道:“好负心的贼!你昨日端的那里去来?把老娘气了一夜!你原来干的那茧儿,我已是晓得不耐烦了!趁早实说,从前已往,与隔壁花家那淫妇偷了几遭?一一说出来,我便罢休。但瞒着一字儿,到明日你前脚儿过去,后脚我就吆喝起来,教你负心的囚根子死无葬身之地!你安下人标住他汉子在院里过夜,却这里要他老婆。我教你吃不了包着走!嗔道昨日大白日里,我和孟三姐在花园里做生活,只见他家那大丫头在墙那边探头舒脑的,原来是那淫妇使的勾使鬼来勾你来了。你还哄我老娘!前日他家那忘八,半夜叫了你往院里去,原来他家就是院里!”西门庆听了,慌的装矮子,只跌脚跪在地下,笑嘻嘻央及说道:“怪小油嘴儿,禁声些!实不瞒你,他如此这般问了你两个的年纪,到明日讨了鞋样去,每人替你做双鞋儿,要拜认你两个做姐姐,他情愿做妹子。”金莲道:“我是不要那淫妇认甚哥哥姐姐的。他要了人家汉子,又来献小殷勤儿,我老娘眼里是放不下砂子的人,肯叫你在我跟前弄了鬼儿去!”说着一只手把他裤子扯开,只见那话软仃当,银托子还带在上面,问道:“你实说,与淫妇弄了几遭?”西门庆道:“弄到有数儿的,只一遭。”妇人道:“你赌个誓,一遭就弄的他恁软如鼻涕浓如酱,却如风瘫了一般的!有些硬朗气儿也是人心。”说着把托子一揪,挂下来,骂道:“没羞的强盗,嗔道教我那里没寻,原来把这行货子悄地带出,和那淫妇肏捣去了。”西门庆满脸儿陪笑说道:“怪小淫妇儿,麻犯人死了,他再三教我捎了上覆来,他到明日过来与你磕头,还要替你做鞋。昨日使丫头替了吴家的样子去了。今日教我捎了这一对寿字簪儿送你。”于是除了帽子,向头上拔将下来,递与金莲。金莲接在手内观看,却是两根番石青填地、金玲珑寿字簪儿,乃御前所制,宫里出来的,甚是奇巧。金莲满心欢喜,说道:“既是如此,我不言语便了。等你过那边去,我这里与你两个观风,教你两个自在肏捣。你心下如何?”那西门庆欢喜的双手搂抱着说道:“我的乖乖的儿,正是如此。不枉的养儿,——不在屙金溺银,只要见景生情。我到明日梯己买一套妆花衣服谢你。”妇人道:“我不信那蜜嘴糖舌,既要老娘替你二人周旋,要依我三件事。”西门庆道:“不拘几件,我都依。”妇人道:“头一件不许你往院里去;第二件要依我说话;第三件你过去和他睡了,来家就要告我说,一字不许你瞒我。”西门庆道:“这个不打紧,都依你便了。”

自此为始,西门庆过去睡了来,就告妇人说:“李瓶儿怎的生得白净,身软如绵花,好风月,又善饮。俺两个帐子里放着果盒,看牌饮酒,常玩耍半夜不睡。”又向袖中取出一个物件儿来,递与金莲瞧,道:“此是他老公公内府画出来的,俺两个点着灯,看着上面行事。”金莲接在手中,展开观看。有词为证:

\[
内府衢花绫裱,牙签锦带妆成。大青小绿细描金,镶嵌斗方干净。女赛巫山神女,男如宋玉郎君,双双帐内惯交锋。解名二十四,春意动关情。
\]
金莲从前至尾看了一遍,不肯放手,就交与春梅道:“好生收在我箱子内,早晚看着耍子。”西门庆道:“你看两日,还交与我。此是人的爱物儿,我借了他来家瞧瞧,还与他。”金莲道:“他的东西,如何到我家?我又不曾从他手里要将来。就是打也打不出去。”西门庆道:“怪小奴才儿,休要耍问。”赶着夺那手卷。金莲道:“你若夺一夺儿,赌个手段,我就把他扯得稀烂,大家看不成。”西门庆笑道:“我也没法了,随你看完了与他罢么。你还了他这个去,他还有个稀奇物件儿哩,到明日我要了来与你。”金莲道:“我儿,谁养得你恁乖?你拿了来,我方与你这手卷去。”两个絮聒了一回。晚夕,金莲在房中香薰鸳被,款设银灯,艳妆澡牝,与西门庆展开手卷,在锦帐之中效“于飞”之乐。看观听说:巫蛊魇昧之物,自古有之。金莲自从叫刘瞎子回背之后,不上几时,使西门庆变嗔怒而为宠爱,化忧辱而为欢娱,再不敢制他。正是:

\[
饶你奸似鬼,也吃洗脚水。
\]
有词为证:
\[
记得书斋乍会时,云踪雨迹少人知。晓来鸾凤栖双枕,剔尽银灯半吐辉。思往事,梦魂迷,今宵喜得效于飞。颠鸾倒凤无穷乐,从此双双永不离。
\]

\newpage
%# -*- coding:utf-8 -*-
%%%%%%%%%%%%%%%%%%%%%%%%%%%%%%%%%%%%%%%%%%%%%%%%%%%%%%%%%%%%%%%%%%%%%%%%%%%%%%%%%%%%%


\chapter{花子虚因气丧身\KG 李瓶儿迎奸赴会}


诗曰:

\[
眼意心期未即休,不堪拈弄玉搔头。
春回笑脸花含媚,黛蹙娥眉柳带愁。
粉晕桃腮思伉俪,寒生兰室盼绸缪。
何如得遂相如意,不让文君咏白头。
\]

话说一日吴月娘心中不快,吴大妗子来看,月娘留他住两日。正陪在房中坐的,忽见小厮玳安抱进毡包来,说:“爹来家了。”吴大妗子便往李娇儿房里去了。西门庆进来,脱了衣服坐下。小玉拿茶来也不吃。月娘见他面色改常,便问:“你今日会茶,来家恁早?”西门庆道:“今该常二哥会,他家没地方,请俺们在城外永福寺去耍子。有花二哥邀了应二哥,俺们四五个,往院里郑爱香儿家吃酒。正吃着,忽见几个做公的进来,不由分说,把花二哥拿的去了。把众人吓了一惊。我便走到李桂姐躲了半日,不放心,使人打听。原来是花二哥内臣家房族中告家财,在东京开封府递了状子,批下来,着落本县拿人。俺们才放心,各人散归家来。”月娘闻言,便道:“这是正该的,你整日跟着这伙人,不着个家,只在外边胡撞;今日只当丢出事来,才是个了手。你如今还不心死。到明日不吃人挣锋厮打,群到那日是个烂羊头,你肯断绝了这条路儿!正经家里老婆的言语说着你肯听?只是院里淫妇在你跟前说句话儿,你到着个驴耳朵听他。正是:家人说着耳边风,外人说着金字经。”西门庆笑道:“谁人敢七个头八个胆打我!”月娘道:“你这行货子,只好家里嘴头子罢了。”

正说着,只见玳安走来说:“隔壁花二娘使天福儿来,请爹过去说话。”这西门庆听了,趔趄脚儿就往外走。月娘道:“明日没的教人讲你把。”西门庆道:“切邻间不防事。我去到那里,看他有甚么话说。”当下走过花子虚家来,李瓶儿使小厮请到后边说话,只见妇人罗衫不整,粉面慵妆,从房里出来,脸吓的蜡渣也似黄,跪着西门庆,再三哀告道:“大官人没奈何,不看僧面看佛面,常言道:家有患难,邻里相助。因他不听人言,把着正经家事儿不理,只在外边胡行。今日吃人暗算,弄出这等事来。这时节方对小厮说将来,教我寻人情救他。我一个妇人家没脚的,那里寻那人情去。发狠起来,想着他恁不依说,拿到东京,打的他烂烂的,也不亏他。只是难为过世老公公的姓字。奴没奈何,请将大官人过来,央及大官人,把他不要提起罢,千万看奴薄面,有人情好歹寻一个儿,只不教他吃凌逼便了。”西门庆见妇人下礼,连忙道:“嫂子请起来,不妨,我还不知为了甚勾当。”妇人道:“正是一言难尽。俺过世老公公有四个侄儿,大侄儿唤做花子由,第三个唤花子光,第四个叫花子华,俺这个名花子虚,都是老公公嫡亲的。虽然老公公挣下这一分钱财,见我这个儿不成器,从广南回来,把东西只交付与我手里收着。着紧还打倘棍儿,那三个越发打的不敢上前。去年老公公死了,这花大、花三、花四,也分了些床帐家伙去了,只现一分银子儿没曾分得。我常说,多少与他些也罢了,他通不理一理儿。今日手暗不通风,却教人弄下来了。”说毕,放声大哭。西门庆道:“嫂子放心,我只道是甚么事来,原来是房分中告家财事,这个不打紧。既是嫂子分付,哥的事就是我的事一般,随问怎的,我在下谨领。”妇人说道:“官人若肯时又好了。请问寻分上,要用多少礼儿,奴好预备。”西门庆道:“也用不多,闻得东京开封府杨府尹,乃蔡太师门生。蔡太师与我这四门亲家杨提督,都是当朝天子面前说得话的人。拿两个分上,齐对杨府尹说,有个不依的!不拘多大事情也了了。如今倒是蔡太师用些礼物。那提督杨爷与我舍下有亲,他肯受礼?”妇人便往房中开箱子,搬出六十锭大元宝,共计三千两,教西门庆收去寻人情,上下使用。西门庆道:“只一半足矣,何消用得许多!”妇人道:“多的大官人收了去。奴床后还有四箱柜蟒衣玉带,帽顶绦环,都是值钱珍宝之物,亦发大官人替我收去,放在大官人那里,奴用时来取。趁这时,奴不思个防身之计,信着他,往后过不出好日子来。眼见得三拳敌不得四手,到明日,没的把这些东西儿吃人暗算了去,坑闪得奴三不归!”西门庆道:“只怕花二哥来家寻问怎了?”妇人道:“这都是老公公在时,梯己交与奴收着之物,他一字不知。大官人只顾收去。”西门庆说道:“既是嫂子恁说,我到家教人来取。”于是一直来家,与月娘商议。月娘说:“银子便用食盒叫小厮抬来。那箱笼东西,若从大门里来,教两边街坊看着不惹眼?必须夜晚打墙上过来方隐密些。”西门庆听言大喜,即令玳安、来旺、来兴、平安四个小厮,两架食盒,把三千两银子先抬来家。然后到晚夕月上时分,李瓶儿那边同迎春、绣春放桌凳,把箱柜挨到墙上。西门庆这边,止是月娘、金莲、春梅,用梯子接着。墙头上铺衬毡条,一个个打发过来,都送到月娘房中去了。正是:

\[
富贵自是福来投,利名还有利名忧。
命里有时终须有,命里无时莫强求。
\]
西庆收下他许多细软金银宝物,邻舍街坊俱不知道。连夜打点驮装停当,求了他亲家陈宅一封书,差家人来保上东京。送上杨提督书礼,转求内阁蔡太师柬帖下与开封府杨府尹。这府尹名唤杨时,别号龟山,乃陕西弘农县人氏,由癸未进士升大理寺卿,今推开封府尹,极是清廉。况蔡太师是他旧时座主,杨戬又是当道时臣,如何不做分上!当日杨府尹升厅,监中提出花子虚来,一干人上厅跪下,审问他家财下落。此时花子虚已有西门庆捎书知会了,口口只说:“自从老公公死了,发送念经,都花费了。止有宅舍两所、庄田一处见在,其余床帐家火物件,俱被族人分散一空。”杨府尹道:“你们内官家财,无可稽考,得之易,失之易。既是花费无存,批仰清河县委官将花太监住宅二所、庄田一处,估价变卖,分给花子由等三人回缴。”花子由等又上前跪禀,还要监追子虚,要别项银两。被杨府尹大怒,都喝下来,说道:“你这厮少打!当初你那内相一死之时,你每不告做甚么来?如今事情已往,又来骚扰。”于是把花子虚一下儿也没打,批了一道公文,押发清河县前来估计庄宅,不在话下。

来保打听这消息,星夜回来,报知西门庆。西门庆听见分上准了,放出花子虚来家,满心欢喜。这里李瓶儿请过西门庆去计议,要叫西门庆拿几两银子,买了这所住的宅子:“到明日,奴不久也是你的人了。”西门庆归家与吴月娘商议。月娘道:“你若要他这房子,恐怕他汉子一时生起疑心来,怎了?”西门庆听记在心。那消几日,花子虚来家,清河县委下乐县丞丈估:太监大宅一所,坐落大街安庆坊,值银七百两,卖与王皇亲为业;南门外庄田一处,值银六百五十两,卖与守备周秀为业。止有住居小宅,值银五百四十两,因在西门庆紧隔壁,没人敢买。花子虚再三使人来说,西门庆只推没银子,不肯上帐。县中紧等要回文书,李瓶儿急了,暗暗使冯妈妈来对西门庆说,教拿他寄放的银子兑五百四十两买了罢。这西门庆方才依允。当官交兑了银两,花子由都画了字。连夜做文书回了上司,共该银一千八百九十五两,三人均分讫。

花子虚打了一场官司出来,没分的丝毫,把银两、房舍、庄田又没了,两箱内三千两大元宝又不见踪影,心中甚是焦躁。因问李瓶儿查算西门庆使用银两下落,今还剩多少,好凑着买房子。反吃妇人整骂了四五日,骂道:“呸!魉魉混沌,你成日放着正事儿不理,在外边眠花卧柳,只当被人弄成圈套,拿在牢里,使将人来教我寻人情。奴是个女妇人家,大门边儿也没走,晓得甚么?认得何人?那里寻人情?浑身是铁打得多少钉儿?替你添羞脸,到处求爹爹告奶奶。多亏了隔壁西门大官人,看日前相交之情,大冷天,刮得那黄风黑风,使了家下人往东京去,替你把事儿干得停停当当的。你今日了毕官司,两脚站在平川地,得命思财,疮好忘痛,来家到问老婆找起后帐儿来了,还说有也没有。你写来的帖子现在,没你的手字儿,我擅自拿出你的银子寻人情,抵盗与人便难了!”花子虚道:“可知是我的帖子来说,实指望还剩下些,咱凑着买房子过日子。”妇人道:“呸!浊蠢才!我不好骂你的。你早仔细好来,咊头儿上不算计,圈底儿下却算计。千也说使多了,万也说使多了,你那三千两银子能到的那里?蔡太师、杨提督好小食肠儿!不是恁大人情,平白拿了你一场,当官蒿条儿也没曾打在你这忘八身上,好好儿放出来,教你在家里恁说嘴!人家不属你管辖,你是他甚么着疼的亲?平白怎替你南上北下走跳,使钱教你!你来家也该摆席酒儿,请过人来,知谢人一知谢儿,还一扫帚扫得人光光的,到问人找起后帐儿来了!”几句连搽带骂,骂的子虚闭口无言。

到次日,西门庆使玳安送了一分礼来与子虚压惊。子虚这里安排了一席,请西门庆来知谢,就要问他银两下落。依着西门庆,还要找过几百两银子与他凑买房子。到是李瓶儿不肯,暗地使冯妈妈过来对西门庆说:“休要来吃酒,只开送一篇花帐与他,说银子上下打点都使没了。”花子虚不识时,还使小厮再三邀请。西门庆躲的一径往院里去了,只回不在家。花子虚气的发昏,只是跌脚。看观听说:大凡妇人更变,不与男子汉一心,随你咬折铁钉般刚毅之夫,也难测其暗地之事。自古男治外而女治内,往往男子之名都被妇人坏了者为何?皆由御之不得其道。要之在乎容德相感,缘分相投,夫唱妇随,庶可保其无咎。若似花子虚落魄飘风,谩无纪律,而欲其内人不生他意,岂可得乎!正是:

\[
自意得其垫,无风可动摇。
\]

话休饶舌。后来子虚只摈凑了二百五十两银子,买了狮子街一所房屋居住。得了这口重气,刚搬到那里,又不幸害了一场伤寒,从十一月初旬,睡倒在床上,就不曾起来。初时还请太医来看,后来怕使钱,只挨着。一日两,两日三,挨到二十头,呜呼哀哉,断气身亡,亡年二十四岁。那手下的大小厮天喜儿,从子虚病倒之时,就拐了五两银子走的无踪。子虚一倒了头,李瓶儿就使冯妈妈请了西门庆过去,与他商议买棺入殓,念经发送,到坟上安葬。那花大、花三、花四一般儿男妇,也都来吊孝送殡。西门庆那日也教吴月娘办了一张桌席,与他山头祭奠。当日妇人轿子归家,也设了一个灵位,供养在房中。虽是守灵,一心只想着西门庆。从子虚在日,就把两个丫头教西门庆耍了,子虚死后,越发通家往还。

一日,正值正月初九,李瓶儿打听是潘金莲生日,未曾过子虚五七,李瓶儿就买礼物坐轿子,穿白绫袄儿,蓝织金裙,白纻布鬒髻,珠子箍儿,来与金莲做生日。冯妈妈抱毡包,天福儿跟轿。进门先与月娘磕了四个头,说道:“前日山头多劳动大娘受饿,又多谢重礼。”拜了月娘,又请李娇儿、孟玉楼拜见了。然后潘金莲来到,说道:“这位就是五娘?”又要磕下头去,一口一声称呼:“姐姐,请受奴一礼儿。”金莲那里肯受,相让了半日,两个还平磕了头。金莲又谢了他寿礼。又有吴大妗子、潘姥姥一同见了。李瓶儿便请西门庆拜见。月娘道:“他今日往门外玉皇庙打醮去了。”一面让坐了,唤茶来吃了。良久,只见孙雪娥走过来。李瓶儿见他妆饰少次于众人,便起身来问道:“此位是何人?奴不知,不曾请见得。”月娘道:“此是他姑娘哩。”李瓶儿就要行礼。月娘道:“不劳起动二娘,只是平拜拜儿罢。”于是彼此拜毕,月娘就让到房中,换了衣裳,分付丫鬟,明间内放桌儿摆茶。须臾,围炉添炭,酒泛羊羔,安排上酒来。让吴大妗子、潘姥姥、李瓶儿上坐,月娘和李娇儿主席,孟玉楼和潘金莲打横。孙雪娥回厨下照管,不敢久坐。月娘见李瓶儿锺锺酒都不辞,于是亲自递了一遍酒,又令李娇儿众人各递酒一遍,因嘲问他话儿道:“花二娘搬的远了,俺姊妹们离多会少,好不思想。二娘狠心,就不说来看俺们看见?”孟玉楼便道:“二娘今日不是因与六姐做生日还不来哩!”李瓶儿道:“好大娘,三娘,蒙众娘抬举,奴心里也要来,一者热孝在身,二者家下没人。昨日才过了他五七,不是怕五娘怪,还不敢来。”因问:“大娘贵降在几时?”月娘道:“贱日早哩。”潘金莲接过来道:“大娘生日是八月十五,二娘好歹来走走。”李瓶儿道:“不消说,一定都来。”孟玉楼道:“二娘今日与俺姊妹相伴一夜儿,不往家去罢了。”李瓶儿道:“奴可知也要和众位娘叙些话儿。不瞒众位娘说,小家儿人家,初搬到那里,自从他没了,家下没人,奴那房子后墙紧靠着乔皇亲花园,好不空!晚夕常有狐狸抛砖掠瓦,奴又害怕。原是两个小厮,那个大小厮又走了,止是这个天福儿小厮看守前门,后半截通空落落的。倒亏了这个老冯,是奴旧时人,常来与奴浆洗些衣裳。”月娘因问:“老冯多少年纪?且是好个恩实妈妈儿,高大言也没句儿。”李瓶儿道:“他今年五十六岁,男花女花都没,只靠说媒度日。我这里常管他些衣裳。昨日拙夫死了,叫过他来与奴做伴儿,晚夕同丫头一炕睡。”潘金莲嘴快,说道:“既有老冯在家里看家,二娘在这里过一夜也不妨,左右你花爹没了,有谁管着你!”玉楼道:“二娘只依我,叫老冯回了轿子,不去罢。”那李瓶儿只是笑,不做声。话说中间,酒过数巡。潘姥姥先起身往前边去了。潘金莲随跟着他娘往房里去了。李瓶儿再三辞道:“奴的酒勾了。”李娇儿道:“花二娘怎的,在他大娘、三娘手里肯吃酒,偏我递酒,二娘不肯吃?显的有厚薄。”遂拿个大杯斟上。李瓶儿道:“好二娘,奴委的吃不去了,岂敢做假!”月娘道:“二娘,你吃过此杯,略歇歇儿罢。”那李瓶儿方才接了,放在面前,只顾与众人说话。孟玉楼见春梅立在旁边,便问春梅:“你娘在前边做甚么哩?你去连你娘、潘姥姥快请来,就说大娘请来陪你花二娘吃酒哩。”春梅去不多时,回来道:“姥姥害身上疼,睡哩。俺娘在房里匀脸,就来。”月娘道:“我倒也没见,他倒是个主人家,把客人丢了,三不知往房里去了。诸般都好,只是有这些孩子气。”有诗为证:

\[
倦来汗湿罗衣彻,楼上人扶上玉梯。
归到院中重洗面,金盆水里发红泥。
\]

正说着,只见潘金莲走来。玉楼在席上看见他艳抹浓妆,从外边摇摆将来,戏道:“五丫头,你好人儿!今日是你个驴马畜,把客人丢在这里,你躲到房里去了,你可成人养的!”那金莲笑嘻嘻向他身上打了一下。玉楼道:“好大胆的五丫头!你还来递一锺儿。”李瓶儿道:“奴在三娘手里吃了好少酒儿,也都勾了。”金莲道:“他手里是他手里帐,我也敢奉二娘一锺儿。”于是满斟一大锺递与李瓶儿。李瓶儿只顾放着不肯吃。月娘因看见金莲鬓上撇着一根金寿字簪儿,便问:“二娘,你与六姐这对寿字簪儿,是那里打造的?倒好样儿。到明日俺每人照样也配恁一对儿戴。”李瓶儿道:“大娘既要,奴还有几对,到明日每位娘都补奉上一对儿。此是过世老公公御前带出来的,外边那里有这样范!”月娘道:“奴取笑斗二娘耍子。俺姐妹们人多,那里有这些相送!”众女眷饮酒欢笑。

看看日西时分,冯妈妈在后边雪娥房里管待酒,吃的脸红红的出来,催逼李瓶儿道:“起身不起身?好打发轿子回去。”月娘道:“二娘不去罢,叫老冯回了轿子家去罢。”李瓶儿说:“家里无人,改日再奉看众位娘,有日子住哩。”孟玉楼道:“二娘好执古,俺众人就没些儿分上?如今不打发轿子,等住回他爹来,少不的也要留二娘。”自这说话,逼迫的李瓶儿就把房门钥匙递与冯妈妈,说道:“既是他众位娘再三留我,显的奴不识敬重。分付轿子回去,教他明日来接罢。你和小厮家去,仔细门户。”又教冯妈妈附耳低言:“教大丫头迎春,拿钥匙开我床房里头一个箱子,小描金头面匣儿里,拿四对金寿字簪儿。你明日早送来,我要送四位娘。”那冯妈妈得了话,拜辞了月娘,一面出门,不在话下。

少顷,李瓶儿不肯吃酒,月娘请到上房,同大妗子一处吃茶坐的。忽见玳安抱进毡包,西门庆来家,掀开帘子进来,说道:“花二娘在这里!”慌的李瓶儿跳起身来,两个见了礼,坐下。月娘叫玉箫与西门庆接了衣裳。西门庆便对吴大妗子、李瓶儿说道:“今日门外玉皇庙圣诞打醮,该我年例做会首,与众人在吴道官房里算帐。七担八柳缠到这咱晚。”因问:“二娘今日不家去罢了?”玉楼道:“二娘再三不肯,要去,被俺众姐妹强着留下。”李瓶儿道:“家里没人,奴不放心。”西门庆道:“没的扯淡,这两日好不巡夜的甚紧,怕怎的!但有些风吹草动,拿我个帖儿送与周大人,点到奉行。”又道:“二娘怎的冷清清坐着?用了些酒儿不曾?”孟玉楼道:“俺众人再三劝二娘,二娘只是推不肯吃。”西门庆道:“你们不济,等我劝二娘。二娘好小量儿!”李瓶儿口里虽说:“奴吃不去了。”只不动身。一面分付丫鬟,从新房中放桌儿,都是留下伺候西门庆的嗄饭菜蔬、细巧果仁,摆了一张桌子。吴大妗子知局,推不用酒,因往李娇儿房里去了。当下李瓶儿上坐,西门庆关席,吴月娘在炕上跐着炉壶儿。孟玉楼、潘金莲两边打横。五人坐定,把酒来斟,也不用小锺儿,都是大银衢花锺子,你一杯,我一盏。常言:风流茶说合,酒是色媒人。吃来吃去,吃的妇人眉黛低横,秋波斜视。正是:

\[
两朵桃花上脸来,眉眼施开真色相。
\]

月娘见他二人吃得饧成一块,言颇涉邪,看不上,往那边房里陪吴大妗子坐去了,由着他四个吃到三更时分。李瓶儿星眼乜斜,立身不住,拉金莲往后边净手。西门庆走到月娘房里,亦东倒西歪,问月娘打发他那里歇。月娘道:“他来与那个做生日,就在那个房儿里歇。”西门庆道:“我在那里歇?”月娘道:“随你那里歇,再不你也跟了他一处去歇罢。”西门庆忍不住笑道:“岂有此理!”因叫小玉来脱衣:“我在这房里睡了。”月娘道:“就别要汗邪,休要惹我那没好口的骂出来!你在这里,他大妗子那里歇?”西门庆道:“罢,罢!我往孟三儿房里歇去罢,于是往玉楼房中歇了。

潘金莲引着李瓶儿净了手,同往他前边来,就和姥姥一处歇卧。到次日起来,临镜梳妆,春梅伏侍。他因见春梅灵变,知是西门庆用过的丫头,与了他一副金三事儿。那春梅连忙就对金莲说了。金莲谢了又谢,说道:“又劳二娘赏赐他。”李瓶儿道:“不枉了五娘有福,好个姐姐!”梳妆毕,金莲领着他同潘姥姥,叫春梅开了花园门,各处游看。李瓶儿看见他那边墙头开了个便门,通着他那壁,便问:“西门爹几时起盖这房子?”金莲道:“前者阴阳看来,说到这二月间兴工动土,要把二娘那房子打开,通做一处,前面盖山子卷棚,展一个大花园;后面还盖三间玩花楼,与奴这三间楼做一条边。”这李瓶儿听了在心。只见月娘使了小玉来请后边吃茶。三人同来到上房。吴月娘、李娇儿、孟玉楼陪着吴大妗子,摆下茶等着哩。众人正吃点心,只见冯妈妈进来,向袖中取出一方旧汗巾,包着四对金寿字簪儿,递与李瓶儿。李瓶儿先奉了一对与月娘,然后李娇儿、孟玉楼、孙雪娥每人都是一对。月娘道:“多有破费二娘,这个却使不得。”李瓶儿笑道:“好大娘,甚么稀罕之物,胡乱与娘们赏人便了。”月娘众人拜谢了,方才各人插在头上。月娘道:“闻说二娘家门首就是灯市,好不热闹。到明日我们看灯,就往二娘府上望望,休要推不在家。”李瓶儿道:“奴到那日,奉请众位娘。”金莲道:“姐姐还不知,奴打听来,这十五日是二娘生日。”月娘道:“今日说过,若是二娘贵降的日子,俺姊妹一个也不少,来与二娘祝寿。”李瓶儿笑道:“蜗居小室,娘们肯下降,奴一定奉请。”不一时吃罢早饭,摆上酒来饮酒。看看留连到日西时分,轿子来接,李瓶儿告辞归家。众姐妹款留不住。临出门,请西门庆拜见。月娘道:“他今日早起身,出门与人家送行去了。”妇人千恩万谢,方才上轿来家。正是:

\[
合欢核桃真堪爱,里面原来别有仁。
\]

\newpage
%# -*- coding:utf-8 -*-
%%%%%%%%%%%%%%%%%%%%%%%%%%%%%%%%%%%%%%%%%%%%%%%%%%%%%%%%%%%%%%%%%%%%%%%%%%%%%%%%%%%%%


\chapter{佳人笑赏玩灯楼\KG 狎客帮嫖丽春院}


诗曰:

\[
楼上多娇艳,当窗并三五。争弄游春陌,相邀开绣户。
转态结红裾,含娇入翠羽。留宾乍拂弦,托意时移住。
\]

话说光阴迅速,又早到正月十五日。西门庆先一日差玳安送了四盘羹菜、一坛酒、一盘寿桃、一盘寿面、一套织金重绢衣服,写吴月娘名字,送与李瓶儿做生日。李瓶儿才起来梳妆,叫了玳安儿到卧房里,说道:“前日打搅你大娘,今日又教你大娘费心送礼来。”玳安道:“娘多上覆,爹也上覆二娘,不多些微礼,送二娘赏人。”李瓶儿一面分付迎春罢四盘茶食管待玳安。临出门与二钱银子、一方闪色手帕:“到家多上覆你家列位娘,我这里就使老冯拿帖儿来请。好歹明日都要光降走走。”玳安磕头出门,两个抬盒子的与一百文钱。李瓶儿随即使老冯拿着五个柬帖儿,十五日请月娘和李娇儿、孟玉楼、孙雪娥、潘金莲,又捎了一个帖儿,暗暗请西门庆那日晚夕赴席。

月娘到次日,留下孙雪娥看家,同李娇儿、孟玉楼、潘金莲四顶轿子出门,都穿着妆花锦绣衣服,来兴、来安、玳安、画童四个小厮跟随着,竟到狮子街灯市李瓶儿新买的房子里来。这房子门面四间,到底三层:临街是楼;仪门内两边厢房,三间客坐,一间梢间;过道穿进去,第三层三间卧房,一间厨房。后边落地紧靠着乔皇亲花园。李瓶儿知月娘众人来看灯,临街楼上设放围屏桌席,悬挂许多花灯。先迎接到客位内,见毕礼数,次让入后边明间内待茶,不必细说。到午间,客位内设四张桌席,叫了两个唱的——董娇儿、韩金钏儿,弹唱饮酒。前边楼上设着细巧添换酒席,又请月娘众人登楼看灯玩耍。楼檐前挂着湘帘,悬着灯彩。吴月娘穿着大红妆花通袖袄儿,娇绿段裙,貂鼠皮袄。李娇儿、孟玉楼、潘金莲都是白绫袄儿,蓝段裙。李娇儿是沉香色遍地金比甲,孟玉楼是绿遍地金比甲,潘金莲是大红遍地金比甲,头上珠翠堆盈,凤钗半卸。俱搭伏定楼窗观看。那灯市中人烟凑集,十分热闹。当街搭数十座灯架,四下围列诸般买卖,玩灯男女,花红柳绿,车马轰雷。但见:

\[
山石穿双龙戏水,云霞映独鹤朝天。金屏灯、玉楼灯见一片珠玑;荷花灯、芙蓉灯散千围锦绣。绣球灯皎皎洁洁,雪花灯拂拂纷纷。秀才灯揖让进止,存孔孟之遗风;媳妇灯容德温柔,效孟姜之节操。和尚灯月明与柳翠相连,判官灯锺馗共小妹并坐。师婆灯挥羽扇假降邪神,刘海灯背金蟾戏吞至宝。骆驼灯、青狮灯驮无价之奇珍;猿猴灯、白象灯进连城之秘宝。七手八脚螃蟹灯倒戏清波,巨大口髯鲇鱼灯平吞绿藻。银蛾斗彩,雪柳争辉。鱼龙沙戏,七真五老献丹书;吊挂流苏,九夷八蛮来进宝。村里社鼓,队队喧阗;百戏货郎,桩桩斗巧。转灯儿一来一往,吊灯儿或仰或垂。琉璃瓶映美女奇花,云母障并瀛州阆苑。王孙争看小栏下,蹴鞠齐云;仕女相携高楼上,娇娆炫色。卦肆云集,相幄星罗:讲新春造化如何,定一世荣枯有准。又有那站高坡打谈的,词曲杨恭;到看这扇响钹游脚僧,演说三藏。卖元宵的高堆果馅,粘梅花的齐插枯枝。剪春娥,鬓边斜插闹东风;祷凉钗,头上飞金光耀日。围屏画石崇之锦帐,珠帘绘梅月之双清。虽然览不尽鳌山景,也应丰登快活年。
\]

月娘看了一回,见楼下人乱,就和李娇儿各归席上吃酒去了。惟有潘金莲、孟玉楼同两个唱的,只顾搭伏着楼窗子望下观看。那潘金莲一径把白绫袄袖子儿搂着,显他那遍地金掏袖儿,露出那十指春葱来,带着六个金马镫戒指儿,探着半截身子,口中磕瓜子儿,把磕的瓜子皮儿都吐落在人身上,和玉楼两个嘻笑不止。一回指道:“大姐姐,你来看,那家房檐下挂的两盏绣球灯,一来一往,滚上滚下,倒好看。”一回又道:“二姐姐,你来看,这对门架子上,挑着一盏大鱼灯,下面还有许多小鱼鳖蟹儿,跟着他倒好耍子。”一回又叫:“三姐姐,你看,这首里这个婆儿灯,那个老儿灯。”正看着,忽然一阵风来,把个婆儿灯下半截割了一个大窟窿。妇人看见,笑个不了,引惹的那楼下看灯的人,挨肩擦背,仰望上瞧,通挤匝不开,都压倮倮儿。内中有几个浮浪子弟,直指着谈论。一个说道:“一定是那公侯府里出来的宅眷。”一个又猜:“是贵戚王孙家艳妾,来此看灯。不然如何内家妆束?”又一个说道:“莫不是院中小娘儿?是那大人家叫来这里看灯弹唱。”又一个走过来说道:“只我认的,你们都猜不着。这两个妇人,也不是小可人家的,他是阎罗大王的妻,五道将军的妾,是咱县门前开生药铺、放官吏债西门大官人的妇女。你惹他怎的?想必跟他大娘来这里看灯。这个穿绿遍地金比甲的,我不认的。那穿大红遍地金比甲儿,上戴着个翠面花儿的,倒好似卖炊饼武大郎的娘子。大郎因为在王婆茶坊内捉奸,被大官人踢死了。把他娶在家里做妾。后次他小叔武松告状,误打死了皂隶李外傅,被大官人垫发充军去了。如今一二年不见出来,落的这等标致了。”正说着,吴月娘见楼下围的人多了,叫了金莲、玉楼席坐下,听着两个粉头弹唱灯词,饮酒。

坐了一回,月娘要起身,说道:“酒勾了,我和二娘先行一步,留下他姊妹两个再坐一回儿,以尽二娘之情。今日他爹不在家,家里无人,光丢着些丫头们,我不放心。”这李瓶儿那里肯放,说道:“好大娘,奴没尽心也是的。今日大节间,灯儿也没点,饭儿也没上,就要家去,就是西门爹不在家中,还有他姑娘们哩,怕怎的?待月色上来,奴送四位娘去。”月娘道:“二娘,不是这等说。我又不大十分用酒,留下他姊妹两个,就同我一般。”李瓶儿道:“大娘不用,二娘也不吃一锺,也没这个道理。想奴前日在大娘府上,那等锺锺不辞,众位娘竟不肯饶我。今日来到奴这湫窄之处,虽无甚物供献,也尽奴一点劳心。”于是拿大银锺递与李娇儿,说道:“二娘好歹吃一杯儿。大娘,奴不敢奉大杯,只奉小杯儿罢。”于是满斟递与月娘。两个唱的,月娘每人与他二钱银子。待的李娇儿吃过酒,月娘就起身,又嘱咐玉楼、金莲道:“我两个先去,就使小厮拿灯笼来接你们,也就来罢。家里没人。”玉楼应诺。李瓶儿送月娘、李娇儿到门首,上轿去了。归到楼上,陪玉楼、金莲饮酒,看看天晚,楼上点起灯来,两个唱的弹唱饮酒,不在话下。

却说西门庆那日同应伯爵、谢希大两个,家中吃了饭,同往灯市里游玩。到了狮子街东口,西门庆因为月娘众人都在李瓶儿家吃酒,恐怕他两个看见,就不往西街去看大灯,只到卖纱灯的跟前就回了。不想转过湾来,撞遇孙寡嘴、祝实念,唱喏说道:“连日不会哥,心中渴想。”见了应伯爵、谢希大骂道:“你两个天杀的好人儿,你来和哥游玩,就不说叫俺一声儿!”西门庆道:“祝兄弟,你错怪了他两个,刚才也是路上相遇。”祝实念道:“如今看了灯往那里去?”西门庆道:“同众位兄弟到大酒楼上吃三杯儿,不是也请众兄弟家去,今日房下们都往人家吃酒去了。”祝实念道:“比是哥请俺每到酒楼上,何不往里边望望李桂姐去?只当大节间拜拜年,去混他混。前日俺两个在他家,他望着俺们好不哭哩!说他从腊里不好到如今,大官人通影边儿不进去看他看。哥今日倒闲,俺们情愿相伴哥进去走走。”西门庆因记挂晚夕李瓶儿有约,故推辞道:“今日我还有小事,明日去罢。”怎禁这伙人死拖活拽,于是同进院中去。正是:

\[
柳底花阴压路尘,一回游赏一回新。
不知买尽长安笑,活得苍生几户贫?
\]

西门庆同众人到了李家,桂卿正打扮着在门首站立,一面迎接入中堂相见了。祝实念就高叫道:“快请三妈出来!还亏俺众人,今日请的大官人来了。”少顷,老虔婆扶拐而出,与西门庆见礼毕,说道:“老身又不曾怠慢了姐夫,如何一向不进来看看姐儿?想必别处另叙了新表子来。”祝实念插口道:“你老人家会猜算,俺大官人近日相了个绝色的表子,每日只在那里走,不想你家桂姐儿。刚才不是俺二人在灯市里撞见,拉他来,他还不来哩!妈不信,问孙伯修就是了。”因指着应伯爵、谢希大说道:“这两个天杀的,和他都是一路神衹。”老虔婆听了,哈哈笑道:“好应二哥,俺家没恼着你,如何不在姐夫面前美言一句儿?虽故姐夫里边头絮儿多,常言道:好子弟不嫖一个粉头,天下钱眼儿都一样。不是老身夸口说,我家桂姐也不丑,姐夫自有眼,今也不消人说。”孙寡嘴道:“我是老实说,哥如今新叙的这个表子,不是里面的,是外面的表子。”西门庆听了,赶着孙寡嘴只顾打,说道:“老妈,你休听这天灾人祸的老油嘴,老杀才!”孙寡嘴和众人笑成一块。西门庆向袖中掏出三两银子来,递与桂卿:“大节间,我请众朋友。”桂卿不肯接,递与老妈。老妈说道:“怎么的?姐夫就笑话我家,大节下拿不出酒菜儿管待列位老爹?又教姐夫坏钞,拿出银子。显的俺们院里人家只是爱钱了。”应伯爵走过来说道:“老妈,你依我收了,快安排酒来俺们吃。”那虔婆说道:“这个理上却使不得。”一壁推辞,一壁把银子接来袖了,深深道了个万福,说道:“谢姐夫的布施。”应伯爵道:“妈,你且住。我说个笑话儿你听:一个子弟在院中嫖小娘儿。那一日做耍,装做贫子进去。老妈见他衣服褴缕,不理他。坐了半日,茶也不拿出来。子弟说:‘妈,我肚饥,有饭寻些来吃。’老妈道:‘米囤也晒,那讨饭来?’子弟又道:‘既没饭,有水拿些来,我洗脸。’老妈道:‘少挑水钱,连日没送水来。’这子弟向袖中取出十两一锭银子,放在桌上,教买米雇水去。慌的老妈没口子道:‘姐夫吃了脸洗饭,洗了饭吃脸!’”把众人都笑了。虔婆道:“你还是这等快取笑,可可儿的来,自古有恁说没这事。”应伯爵道:“你拿耳朵来,我对你说:大官人新近请了花二哥表子——后巷的吴银儿了,不要你家桂姐哩!”虔婆笑道:“我不信,俺桂姐今日不是强口,比吴银儿还比得过。我家与姐夫是快刀儿割不断的亲戚。姐夫是何等人儿?他眼里见得多,着紧处,金子也估出个成色来!”说毕,入去收拾酒菜去了。

少顷,李桂姐出来,家常挽着一窝丝杭州攒,金缕丝钗,翠梅花钿儿,珠子箍儿,金笼坠子,上穿白绫对襟袄儿,下着红罗裙子,打扮的粉妆玉琢,望下道了万福,与桂卿一边一个打横坐下。须臾,泡出茶来,桂卿、桂姐每人递了一盏,陪着吃毕。保儿就来打抹春台,才待收拾摆放案酒,忽见帘子外探头舒脑,有几个穿褴缕衣者——谓之架儿,进来跪下,手里拿着三四升瓜子儿:“大节间,孝顺大老爹。”西门庆只认头一个叫于春儿,问:“你们那几个在这里?”于春道:“还有段绵纱、青聂钺,在外边伺候。”段绵纱进来,看见应伯爵在里,说道:“应爹也在这里。”连忙磕了头。西门庆分付收了他瓜子儿,打开银包儿,捏一两一块银子掠在地下。于春儿接了,和众人扒在地下磕了个头,说道:“谢爹赏赐。”往外飞跑。有《朝天子》单道架儿行藏:

\[
这家子打和,那家子撮合。他的本分少,虚头大,一些儿不巧又腾挪,绕院里都踅过。席面上帮闲,把牙儿闲磕。攘一回才散伙,赚钱又不多。歪厮缠怎么?他在虎口里求津唾。
\]

西门庆打发架儿出门,安排酒上来吃。桂姐满泛金杯,双垂红袖,肴烹异品,果献时新,倚翠偎红,花浓酒艳。酒过两巡,桂卿、桂姐一个弹筝,一个琵琶,两个弹着唱了一套《霁景融和》。正唱在热闹处,见三个穿青衣黄板鞭者——谓之圆社,手里捧着一只烧鹅,提着两瓶老酒,大节间来孝顺大官人,向前打了半跪。西门庆平昔认的,一个唤白秃子,一个唤小张闲,一个是罗回子,因说道:“你们且外边候候,待俺们吃过酒,踢三跑。”于是向桌子上拾了四盘嗄饭、一大壶酒、一碟点心,打发众圆社吃了,整理气毬伺候。西门庆吃了一回酒,出来外面院子里,先踢了一跑。次教桂姐上来,与两个圆社踢。一个揸头,一个对障,勾踢拐打之间,无不假喝彩奉承。就有些不到处,都快取过去了。反来向西门庆面前讨赏钱,说:“桂姐的行头,就数一数二的,强如二条巷董官女儿数十倍。”当下桂姐踢了两跑下来,使的尘生眉畔,汗湿腮边,气喘吁吁,腰肢困乏。袖中取出春扇儿摇凉,与西门庆携手,看桂卿与谢希大、张小闲踢行头。白秃子、罗回子在旁虚撮脚儿等漏,往来拾毛。亦有《朝天子》一词,单表这踢圆的始末:

\[
在家中也闲,到处刮涎,生理全不干,气毬儿不离在身边,每日街头站。穷的又不趋,富贵他偏羡。从早晨只到晚,不得甚饱餐。转不得大钱,他老婆常被人包占。
\]

西门庆正看着众人在院内打双陆、踢气毬,饮酒,只见玳安骑马来接,悄悄附耳低言道:“大娘、二娘家去了。花二娘叫小的请爹早些过去哩!”这西门庆听了,暗暗叫玳安:“把马吊在后门边,等着我。”于是酒也不吃,拉桂姐到房中,只坐了一回儿,就出来推净手,于后门上马,一溜烟走了。应伯爵使保儿去拉扯,西门庆只说:“我家里有事。”那里肯转来!教玳安儿拿了一两五钱银子打发三个圆社。李家恐怕他又往后巷吴银儿家去,使丫鬟直跟至院门首方回。应伯爵等众人,还吃到二更才散。正是:

\[
笑骂由他笑骂,欢娱我且欢娱。
\]

\newpage
%# -*- coding:utf-8 -*-
%%%%%%%%%%%%%%%%%%%%%%%%%%%%%%%%%%%%%%%%%%%%%%%%%%%%%%%%%%%%%%%%%%%%%%%%%%%%%%%%%%%%%


\chapter{西门庆择吉佳期\KG 应伯爵追欢喜庆}


诗曰:

\[
倾城倾国莫相疑,巫水巫云梦亦痴。
红粉情多销骏骨,金兰谊薄惜蛾眉。
温柔乡里精神健,窈窕风前意态奇。
村子不知春寂寂,千金此夕故踟蹰。
\]

话说当日西门庆出离院门,玳安跟马,迳到狮子街李瓶儿家,见大门关着,就知堂客轿子家去了。玳安叫冯妈妈开了门,西门庆进来。李瓶儿在堂中秉烛,花冠齐整,素服轻盈,正倚帘栊盼望。见西门庆来,忙移莲步,款促湘裙,下阶迎接,笑道:“你早来些儿,他三娘、五娘还在这里,只刚才起身去了。今日他大娘去的早,说你不在家。那里去了?”西门庆道:“今日我和应二哥、谢子纯早晨看灯,打你门首过去来。不想又撞见两个朋友,拉去院里,撞到这咱晚。我恐怕你这里等候,小厮去时,教我推净手,打后门跑了。不然必吃他们挂住了,休想来的成。”李瓶儿道:“适间多谢你重礼。他娘们又不肯坐,只说家里没人,教奴到没意思的。”于是重筛美酒,再整佳肴,堂中把花灯都点上,放下暖帘来。金炉添兽炭,宝篆热龙涎。妇人递酒与西门庆,磕下头去说道:“拙夫已故,举眼无亲。今日此杯酒,只靠官人与奴作个主儿,休要嫌奴丑陋,奴情愿与官人铺床叠被,与众位娘子作个姊妹,奴自己甘心。不知官人心下如何?”说着满眼泪落。西门庆一手接酒,一手扯他道:“你请起来。既蒙你厚爱,我西门庆铭刻于心。待你孝服满时,我自有处,不劳你费心。今日是你的好日子,咱每且吃酒。”西门庆吃毕,亦满斟一杯回奉。妇人吃毕,安席坐下。冯妈妈单管厨下。须臾,拿面上来吃。西门庆因问道:“今日唱的是那两个?”李瓶儿道:“今日是董娇儿、韩金钏儿两个。临晚,送他三娘、五娘家中讨花儿去了。”两个在席上交杯换盏饮酒,绣春、迎春两个在旁斟酒下菜伏侍。只见玳安上来,与李瓶儿磕头拜寿。李瓶儿连忙起身还了个万福,分付迎春教老冯厨下看寿面点心下饭,拿一壶酒与玳安吃。西门庆分付:“吃了早些回家去罢。”李瓶儿道:“到家里,你娘问,休说你爹在这里。”玳安道:“小的知道,只说爹在里边过夜。明日早来接爹就是了。”西门庆点了点头儿,当下把李瓶儿喜欢的要不的,说道:“好个乖孩子,眼里说话。”又叫迎春拿二钱银子与他节间买瓜子儿磕:“明日你拿个样儿来,我替你做双好鞋儿穿。”那玳安连忙磕头说:“小的怎敢?”走到下边吃了酒饭,带马出门。冯妈妈把大门关上了拴。

李瓶儿同西门庆猜枚吃了一回,又拿一付三十二扇象牙牌儿,桌上铺茜红苫条,两个抹牌饮酒。吃一回,分付迎春房里秉烛。原来花子虚死了,迎春、绣春都已被西门庆耍了,以此凡事不避,教他收拾铺床,拿果盒杯酒。又在床上紫锦帐里,妇人露着粉般身子,西门庆香肩相并,玉体厮挨。两个看牌,拿大锺饮酒。因问西门庆:“你那边房子几时收拾?”西门庆道:“且待二月间兴工,连你这边一所通身打开,与那边花园取齐。前边起盖个山子卷棚,花园耍子。后边还盖三间玩花楼。”妇人因指道:“奴这床后茶叶箱内,还藏三四十斤沉香、二百斤白蜡、两罐子水银、八十斤胡椒。你明日都搬出来,替我卖了银子,凑着你盖房子使。你若不嫌奴丑陋,到家好歹对大娘说,奴情愿与娘们做个姊妹,随问把我做第几个也罢。亲亲,奴舍不的你。”说着,眼泪纷纷的落将下来。西门庆忙把汗巾儿抹拭,说道:“你的情意,我已尽知。待你这边孝服满,我那边房子盖了才好。不然娶你过去,没有住房。”妇人道:“既有实心娶奴家去,到明日好歹把奴的房盖的与他五娘在一处,奴舍不的他好个人儿,与后边孟家三娘,见了奴且亲热。两个天生的打扮,也不相两个姊妹,只相一个娘儿生的一般。惟有他大娘性儿不是好的,快眉眼里扫人。”西门庆说道:“俺吴家的这个拙荆,他到是好性儿哩。不然手下怎生容得这些人?明日这边与那边一样,盖三间楼与你居住,安两个角门儿出入。你心下如何?”妇人道:“我的哥哥,这等才可奴的意!”于是两个颠鸾倒凤,淫欲无度。狂到四更时分,方才就寝。枕上并肩交股,直睡到次日饭时不起来。

妇人且不梳头,迎春拿进粥来,只陪着西门庆吃了半盏粥儿,又拿酒来,二人又吃。原来李瓶儿好马爬着,教西门庆坐在枕上,他倒插花往来自动。两个正在美处,只见玳安儿外边打门,骑马来接。西门庆唤他在窗下问他话。玳安说:“家中有三个川广客人,在家中坐着。有许多细货要科兑与傅二叔,只要一百两银子押合同,约八月中找完银子。大娘使小的来请爹家去理会此事。”西门庆道:“你没说我在这里?”玳安道:“小的只说爹在桂姨家,没说在这里。”西门庆道:“你看不晓事!教傅二叔打发他便了,又来请我怎的?”玳安道:“傅二叔讲来,客人不肯,直等爹去,方才批合同。”李瓶儿道:“既是家中使孩子来请,买卖要紧,你不去,惹的大娘不怪么?”西门庆道:“你不知,贼蛮奴才,行市迟,货物没处发兑,才上门脱与人。若快时,他就张致了。满清河县,除了我家铺子大,发货多,随问多少时,不怕他不来寻我。”妇人道:“买卖不与道路为仇,只依奴到家打发了再来。往后日子多如柳叶儿哩。”西门庆于是依李瓶儿之言,慢慢起来,梳头净面,戴网巾,穿衣服。李瓶儿收拾饭与他吃了,西门庆一直带着个眼纱,骑马来家。

铺子里有四五个客人,等候秤货兑银。批了合同,打发去了。走到潘金莲房中,金莲便问:“你昨日往那里去来?实说便罢,不然我就嚷的尘邓邓的。”西门庆道:“你们都在花家吃酒,我和他们灯市里走了走,就同往里边吃酒,过一夜。今日小厮接我方才来家。”金莲道:“我知小厮去接,那院里有你魂儿?罢么,贼负心,你还哄我哩!那淫妇昨日打发俺们来了,弄神弄鬼的。晚夕叫了你去,肏捣了一夜,肏捣的了,才放来了。玳安这贼囚根子,久惯儿牢成,对着他大娘又一样话儿,对着我又是一样话儿。先是他回马来家,他大娘问他:‘你爹怎的不来?在谁家吃酒哩?’他回说:‘和傅二叔众人看了灯回来,都在院里李桂姨家吃酒,叫我明早接去哩。”落后我叫了问他,他笑不言语。问的急了,才说:‘爹在狮子街花二娘那里哩!’贼囚根,他怎的就知我和你一心一话!想必你叫他说来。”西门庆道:“我那里教他?”于是隐瞒不住,方才把李瓶儿“晚夕请我去到那里,与我递酒,说空过你们来了。又哭哭啼啼告诉我说,他没人手,后半截空,晚夕害怕,一心要教我娶他。问几时收拾这房子。他还有些香烛细货,也值几百两银子,教我会经纪,替他打发。银子教我收,凑着盖房子。上紧修盖,他要和你一处住,与你做个姊妹,恐怕你不肯。”妇人道:“我也不多着个影儿在这里,巴不的来总好。我这里也空落落的,得他来与老娘做伴儿。自古舡多不碍港,车多不碍路,我不肯招他,当初那个怎么招我来?搀奴甚么分儿也怎的?倒只怕人心不似奴心。你还问声大姐姐去。”西门庆道:“虽故是恁说,他孝服未满哩。”说毕,妇人与西门庆脱白绫袄,袖子里滑浪一声吊出个物件儿来,拿在手里沉甸甸的,弹子大,认了半日,竟不知甚么东西。但见:

\[
原是番兵出产,逢人荐转在京。身躯小内玲珑。得人轻借力,辗转作蝉鸣。解使佳人心颤,惯能助肾威风。号称金面勇先锋。战降功第一,扬名勉子铃。
\]
妇人认了半日,问道:“是甚么东西儿?怎和把人半边胳膊都麻了?”西门庆笑道:“这物件你就不知道了,名唤做勉铃,南方勉甸国出来的。好的也值四五两银子。”妇人道:“此物使到那里?”西门庆道:“先把他放入炉内,然后行事,妙不可言。”妇人道:“你与李瓶儿也干来?”西门庆于是把晚间之事,从头告诉一遍。说得金莲淫心顿起,两个白日里掩上房门,解衣上床交欢。正是:

\[
不知子晋缘何事,才学吹箫便作仙。
\]

话休饶舌。一日西门庆会了经纪,把李瓶儿的香蜡等物,都秤了斤两,共卖了三百八十两银子。李瓶儿只留下一百八十两盘缠,其余都付与西门庆收了,凑着盖房使。教阴阳择用二月初八日兴土动工。将五百两银子委付大家人来招并主管贲四,卸砖瓦木石,管工计帐。这贲四名唤贲第传,年少生的浮浪嚣虚,百能百巧。原是内相勤儿出身,因不守本分,被赶出来。初时跟着人做兄弟,次后投入大人家做家人,把人家奶子拐出来做了浑家,却在故衣行做经纪。琵琶箫管都会。西门庆见他这般本事,常照管他在生药铺中秤货讨人钱使。以此凡大小事情,少他不得。当日贲四、来招督管各作匠人兴工。先拆毁花家那边旧房,打开墙垣,筑起地脚,盖起卷棚山子、各亭台耍子去处。非止一日,不必尽说。

光阴迅速,日月如梭。西门庆起盖花园,约个月有余。却是三月上旬,乃花子虚百日。李瓶儿预先请过西门庆去,和他计议,要把花子虚灵烧了:“房子卖的卖,不的,你着人来看守。你早把奴娶过去罢!随你把奴作第几个,奴情愿伏侍你铺床叠被。”说着泪如雨下。西门庆道:“你休烦恼。我这话对房下和潘五姐也说过了,直待与你把房盖完,那时你孝服将满,娶你过门不迟。”李瓶儿道:“你既有真心娶奴,先早把奴房撺掇盖了。娶过奴去,到你家住一日,死也甘心。省得奴在这里度日如年。”西门庆道:“你的话,我知道了。”李瓶儿道:“再不的,我烧了灵,先搬在五娘那边住两日。等你盖了新房子,搬移不迟。你好歹到家和五娘说,我还等你的话。这三月初十日,是他百日,我好念经烧灵。”西门庆应诺,与妇人歇了一夜。

到次日来家,一五一十对潘金莲说了。金莲道:“可知好哩!奴巴不的腾两间房与他住。你还问声大姐姐去。我落得河水不洗船。”西门庆一直走到月娘房里来,月娘正梳头。西门庆把李瓶儿要嫁一节,从头至尾说一遍。月娘道:“你不好娶他的。他头一件,孝服不满;第二件,你当初和他男子汉相交;第三件,你又和他老婆有连手,买了他房子,收着他寄放的许多东西。常言:机儿不快梭儿快。我闻得人说,他家房族中花大是个刁徒泼皮。倘一时有些声口,倒没的惹虱子头上搔。奴说的是好话。赵钱孙李,你依不依随你!”几句说的西门庆闭口无言。走出前厅来,坐在椅子上沉吟:又不好回李瓶儿话,又不好不去的。寻思了半日,还进入金莲房里来。金莲问道:“大姐姐怎么说?”西门庆把月娘的话告诉了一遍。金莲道:“大姐姐说的也是。你又买了他房子,又娶他老婆,当初又与他汉子相交,既做朋友,没丝也有寸,交官儿也看乔了。”西门庆道:“这个也罢了。到只怕花大那厮没圈子跳,知道挟制他孝服不满,在中间鬼浑。怎生计较?我如今又不好回他的。”金莲道:“呸!有甚难处的事?你到那里只说:‘我到家对五娘说来,他的楼上堆着许多药料,你这家伙去到那里没处堆放,亦发再宽待些时,你这边房子也七八盖了,撺掇匠人早些装修油漆停当,你这里孝服也将满。那里娶你过去,却不齐备些。强似搬在五娘楼上,荤不荤,素不素,挤在一处甚么样子!’管情他也罢了。”

西门庆听言大喜,那里等的时分,就走到李瓶儿家。妇人便问:“所言之事如何?”西门庆道:“五娘说来,一发等收拾油漆你新房子,你搬去不迟。如今他那边楼上,堆的破零零的,你这些东西过去那里堆放?还有一件打搅,只怕你家大伯子说你孝服不满,如之奈何?”妇人道:“他不敢管我的事。休说各衣另饭,当官写立分单,已倒断开了,只我先嫁由爹娘,后嫁由自己。常言:嫂叔不通问,大伯管不的我暗地里事。我如今见过不的日子,他顾不的我。他但若放出个屁来,我教那贼花子坐着死不敢睡着死。大官人你放心,他不敢惹我。”因问:“你这房子,也得几时方收拾完备?”西门庆道:“我如今分付匠人,先替你盖出这三间楼来,及至油漆了,也到五月头上。”妇人道:“我的哥哥,你上紧些。奴情愿等到那时候也罢。”说毕,丫鬟摆上酒,两个欢娱饮酒过夜。西门庆自此,没三五日不来,俱不必细说。

光阴迅速,西门庆家中已盖了两月房屋。三间玩花楼,装修将完,只少卷棚还未安磉。一日,五月蕤宾时节,正是:

\[
家家门插艾叶,处处户挂灵符。
\]
李瓶儿治了一席酒,请过西门庆来,一者解粽,二者商议过门之事。择五月十五日,先请僧人念经烧灵,然后西门庆这边择娶妇人过门。西门庆因问李瓶儿道:“你烧灵那日,花大、花三、花四请他不请?”妇人道:“我每人把个帖子,随他来不来!”当下计议已定,单等五月十五日,妇人请了报恩寺十二众僧人,在家念经除灵。

西门庆那日封了三钱银子人情,与应伯爵做生日。早晨拿了五两银子与玳安,教他买办置酒,晚夕与李瓶儿除服。却教平安、画童两个跟马,约午后时分,往应伯爵家来。那日在席者谢希大、祝实念、孙天化、吴典恩、云理守、常峙节连新上会贲第传十个朋友,一个不少。又叫了两个小优儿弹唱。递毕酒,上坐之时,西门庆叫过两个小优儿,认的头一个是吴银儿兄弟,名唤吴惠。那一个不认的,跪下说道:“小的是郑爱香儿的哥,叫郑奉。”西门庆坐首席,每人赏二钱银子。吃到日西时分,只见玳安拿马来接,向西门庆耳边悄悄说道:“二娘请爹早些去。”西门庆与了他个眼色,就往下走。被应伯爵叫住问道:“贼狗骨头儿,你过来实说。若不实说,我把你小耳朵拧过一边来,你应爹一年有几个生日?恁日头半天里就拿马来,端的谁使你来?或者是你家中那娘使了你来?或者是里边十八子那里?你若不说,过一百年也不对你爹说,替你这小狗秃儿娶老婆。”玳安只说道:“委的没人使小的。小的恐怕夜紧,爹要起身早,拿马来伺候。”应伯爵奈何了他一回,见不说,便道:“你不说,我明日打听出来,和你这小油嘴儿算帐。”于是又斟了一锺酒,拿了半碟点儿,与玳安下边吃去。

良久,西门庆下来更衣,叫玳安到僻静处问他话:“今日花家有谁来?”玳安道:“花三往乡里去了。花四家里害眼,都没人来。只有花大家两口子来。吃了一日斋饭,他汉子先家去了,只有他老婆,临去,二娘叫到房里,与了他十两银子,两套衣服。还与二娘磕了头。”西门庆道:“他没说什么?”玳安道:“他一字没敢题甚么,只说到明日二娘过来,他三日要来爹家走走。”西门庆道:“他真个说此话来?”玳安道:“小的怎敢说谎。”西门庆听了,满心欢喜。又问:“斋供了毕不曾?”玳安道:“和尚老早就去了,灵位也烧了。二娘说请爹早些过去。”西门庆道:“我知道了,你处边看马去。”这玳安正往外走,不想应伯爵在过道内听,猛可叫了一声,把玳安吓了一跳。伯爵骂道:“贼小骨头儿!你不对我说,我怎的也听见了?原来你爹儿们干的好茧儿!”西门庆道:“怪狗才,休要倡扬。”伯爵道:“你央我央儿,我不说便了。”于是走到席上,如此这般,对众人说了一回。把西门庆拉着说道:“哥,你可成个人!有这等事,就挂口不对兄弟们说声儿?就是花大有些话说,哥只分付俺们一声,等俺们和他说,不怕他不依。他若敢道个不字,俺们就与他结下个大疙瘩。端的不知哥这亲事成了不曾?哥一一告诉俺们。比来相交朋友做甚么?哥若有使令去处,兄弟情愿火里火去,水里水去。弟兄们这等待你,哥还只瞒着不说。”谢希大接过说道:“哥若不说,俺们明日倡扬的里边李桂姐、吴银儿知道了,大家都不好意思的。”西门庆笑道:“我教众位得知罢,亲事已都停当了。”谢希大道:“哥到明日娶嫂子过门,俺们贺哥去。哥好歹叫上四个唱的,请俺们吃喜酒。”西门庆道:“这个不消说,一定奉请列位兄弟。”祝实念道:“比时明日与哥庆喜,不如咱如今替哥把一杯儿酒,先庆了喜罢。”于是叫伯爵把酒,谢希大执壶,祝实念捧菜,其余都陪跪。把两个小优儿也叫来跪着,弹唱一套《十三腔》“喜遇吉日”,一连把西门庆灌了三四锺酒。祝实念道:“哥,那日请俺们吃酒,也不要少了郑奉、吴惠两个。”因定下:“你二人好歹去。”郑奉掩口道:“小的们一定伺候。”须臾,递酒毕,各归席坐下。又吃了一回。看看天晚,那西门庆那里坐的住,赶眼错起身走了。应伯爵还要拦门不放,谢希大道:“应二哥,你放哥去罢。休要误了他的事,教嫂子见怪。”

那西门庆得手上马,一直走了。到了狮子街,李瓶儿摘去孝髻,换上一身艳服。堂中灯火荧煌,预备下一桌齐整酒席,上面独独安一张交椅,让西门庆上坐。丫鬟执壶,李瓶儿满斟一杯递上去,磕了四个头,说道:“今日灵已烧了,蒙大官人不弃,奴家得奉巾栉之欢,以遂于飞之愿。”行毕礼起来。西门庆下席来,亦回递妇人一杯,方才坐下。因问:“今日花大两口子没说什么?”李瓶儿道:“奴午斋后,叫他进到房中,就说大官人这边亲事。他满口说好,一句闲话也无。只说明日三日里,教他娘子儿来咱家走走。奴与他十两银子,两套衣服,两口子欢喜的要不的。临出门,谢了又谢。”西门庆道:“他既恁说,我容他上门走走也不差甚么。但有一句闲话,我不饶他。”李瓶儿道:“他若放辣骚,奴也不放过他。”于是银镶锺儿盛着南酒,绣春斟了送上,李瓶儿陪着吃了几杯。真个是年随情少,酒因境多。李瓶儿因过门日子近了,比常时益发欢喜,脸上堆下笑来,问西门庆道:“方才你在应家吃酒,玳安来请你,那边没人知道么?”西门庆道:“又被应花子猜着,逼勒小厮说了几句,闹混了一场。诸弟兄要与我贺喜,唤唱的,做东道,又齐攒的帮衬,灌上我几杯。我赶眼错就走出来,还要拦阻,又说好歹,放了我来。”李瓶儿道:“他们放了你,也还解趣哩。”西门庆看他醉态颠狂,情眸眷恋,一霎的不禁胡乱。两个口吐丁香,脸偎仙杏,李瓶儿把西门庆抱在怀里叫道:“我的亲哥!你既真心要娶我,可趁早些。你又往来不便,休丢我在这里日夜悬望。”说毕翻来倒去,搅做一团,真个是:

\[
情浓胸凑紧,款洽臂轻笼;
倦把银缸照,犹疑是梦中。
\]

\newpage
%# -*- coding:utf-8 -*-
%%%%%%%%%%%%%%%%%%%%%%%%%%%%%%%%%%%%%%%%%%%%%%%%%%%%%%%%%%%%%%%%%%%%%%%%%%%%%%%%%%%%%


\chapter{宇给事劾倒杨提督\KG 李瓶儿许嫁蒋竹山}


诗曰:

\[
早知君爱歇,本自无容妒;谁使恩情深,今来反相误。
愁眠罗帐晓,泣坐金闺暮;独有梦中魂,犹言意如故。
\]

话说五月二十日,帅府周守备生日。西门庆封五星分资、两方手帕,打选衣帽齐整,骑匹大白马,四个小厮跟随,往他家拜寿。席间也有夏提刑、张团练、荆千户、贺千户一班武官儿饮酒,鼓乐迎接,搬演戏文。玳安接了衣裳,回马来家。到日西时分,又骑马去接,走到西街口上,撞见冯妈妈,问道:“冯妈妈那里去?”冯妈妈道:“你二娘使我来请你爹。雇银匠整理头面完备,今日送来,请你爹那里瞧去。你二娘还和你爹说话哩!”玳安道:“俺爹今日在守备府周老爷处吃酒,我如今接去。你老人家回罢。等我到那里,对爹说就是了。”冯妈妈道:“累你好歹说声,你二娘等着哩!”这玳安打马迳到守备府。众官员正饮酒间,玳安走到西门庆席前,说道:“小的回马家来时,在街口撞遇冯妈妈,二娘使了来说,雇银匠送了头面来了,请爹瞧去,还要和爹说话哩。”西门庆听了,就要起身,那周守备那里肯放,拦门拿巨杯相劝。西门庆道:“蒙大人见赐,宁可饮一杯,还有些小事,不能尽情,恕罪,恕罪!”于是一饮而尽,辞周守备上马,迳到李瓶儿家。

妇人接着,茶汤毕,西门庆分付玳安回马家去,明日来接。玳安去了。李瓶儿叫迎春盒儿内取出头面来,与西门庆过目。黄烘烘火焰般一付好头面,收过去,单等二十四日行礼,出月初四日准娶。妇人满心欢喜,连忙安排酒来,和西门庆畅饮开怀。吃了一回,使丫鬟房中搽抹凉席干净。两个在纱帐之中,香焚兰麝,衾展鲛绡,脱去衣裳,并肩叠股,饮酒调笑。良久,春色横眉,淫心荡漾。西门庆先和妇人云雨一回,然后乘着酒兴,坐于床上,令妇人横躺于衽席之上,与他品箫。但见:

\[
不竹不丝不石,肉音别自唔咿。流苏瑟瑟碧纱垂,辨不出宫商角徵。一点樱桃欲绽,纤纤十指频移。深吞添吐两情痴,不觉灵犀味美。
\]
西门庆醉中戏问妇人:“当初花子虚在时,也和他干此事不干?”妇人道:“他逐日睡生梦死,奴那里耐烦和他干这营生!他每日只在外边胡撞,就来家,奴等闲也不和他沾身。况且老公公在时,和他另在一间房睡着,我还把他骂的狗血喷了头。好不好,对老公公说了,要打倘棍儿。奴与他这般顽耍,可不碜杀奴罢了!谁似冤家这般可奴之意,就是医奴的药一般。白日黑夜,教奴只是想你。”两个耍一回,又干了一回。傍边迎春伺候下一个小方盒,都是各样细巧果品,小金壶内满泛琼浆。从黄昏掌上灯烛,且干且歇,直耍到一更时分。只听外边一片声打的大门响,使冯妈妈开门瞧去,原来是玳安来了。西门庆道:“我分付明日来接,这咱晚又来做甚么?”因叫进来问他。那小厮慌慌张张走到房门首,因西门庆与妇人睡着,又不敢进来,只在帘外说道:“姐姐、姐夫都搬来了,许多箱笼在家中。大娘使我来请爹,快去计较话哩。”这西门庆听了,只顾犹豫:“这咱晚,端的有甚缘故?须得到家瞧瞧。”连忙起来。妇人打发穿上衣服,做了一盏暖酒与他吃。

打马一直到家,只见后堂中秉着灯烛,女儿女婿都来了,堆着许多箱笼床帐家伙,先吃了一惊,因问:“怎的这咱来家?”女婿陈敬济磕了头,哭说:“近日朝中,俺杨老爷被科道官参论倒了。圣旨下来,拿送南牢问罪。门下亲族用事人等,都问拟枷充军。昨日府中杨干办连夜奔来,透报与父亲知道。父亲慌了,教儿子同大姐和些家伙箱笼,且暂在爹家中寄放,躲避些时。他便起身往东京我姑娘那里,打听消息去了。待事宁之日,恩有重报,不敢有忘。”西门庆问:“你爹有书没有?”陈敬济道:“有书在此。”向袖中取出,递与西门庆。折开观看,上面写道:

\[
眷生陈洪顿首书奉大德西门庆亲家台览:余情不叙。兹因北虏犯边,抢过雄州地界,兵部王尚书不发救兵,失误军机,连累朝中杨老爷,俱被科道官参劾太重。圣旨恼怒,拿下南牢监禁,会同三法司审问。其门下亲族用事人等,俱照例发边卫充军。生一闻消息,举家惊惶,无处可投,先打发小儿、令爱,随身箱笼家活,暂借亲家府上寄寓。生即上京,投在姐夫张世廉处,打听示下。待事务宁帖之日,回家恩有重报,不敢有忘。诚恐县中有甚声色,生令小儿外具银五百两,相烦亲家费心处料,容当叩报没齿不忘。灯下草书,不宣。\named{仲夏二十日洪再拜}
\]

西门庆看了,慌了手脚,教吴月娘安排酒饭,管待女儿、女婿。就令家下人等,打扫厅前东厢房三间,与他两口儿居住。把箱笼细软都收拾月娘上房来。陈敬济取出他那五百两银子,交与西门庆打点使用。西门庆叫了吴主管来,与他五百两银子,教他连夜往县中承行房里,抄录一张东京行下来的文书邸报来看。上面端的写的是甚言语:

\[
兵科给事中宇文虚中等一本,恳乞宸断,亟诛误国权奸,以振本兵,以消虏患事:臣闻夷狄之祸,自古有之。周之猃狁,汉之匈奴,唐之突厥,迨及五代而契丹浸强,至我皇宋建国,大辽纵横中原者已非一日。然未闻内无夷狄而外萌夷狄之患者。语云:霜降而堂钟鸣,雨下而柱础润。以类感类,必然之理。譬若病夫,腹心之疾已久,元气内消,风邪外入,四肢百骸,无非受病,虽卢扁莫之能救,焉能久乎?今天下之势,正犹病夫兀羸之极矣。君犹元首也,辅臣犹腹心也,百官犹四肢也。陛下端拱于九重之上,百官庶政各尽职于下。元气内充,荣卫外扞,则虏患何由而至哉?今招夷虏之患者,莫如崇政殿大学士蔡京者:本以倹邪奸险之资,济以寡廉鲜耻之行,谗谄面谀,上不能辅君当道,赞元理化;下不能宣德布政,保爱元元。徒以利禄自资,希宠固位,树党怀奸,蒙蔽欺君,中伤善类。忠士为之解体,四海为之寒心。联翩朱紫,萃聚一门。迩者河湟失议,主议伐辽,内割三郡,郭药师之叛,卒使金虏背盟,凭陵中原。此皆误国之大者,皆由京之不职也。王黼贪庸无赖,行比俳优。蒙京汲引,荐居政府,未几谬掌本兵。惟事慕位苟安,终无一筹可展。乃者张达残于太原,为之张皇失散。今虏犯内地,则又挈妻子南下,为自全之计。其误国之罪,可胜诛戮?杨戬本以纨绔膏粱叨承祖荫,凭籍宠灵典司兵柄,滥膺阃外,大奸似忠,怯懦无比。此三臣者,皆朋党固结,内外蒙蔽,为陛下腹心之蛊者也。数年以来,招灾致异,丧本伤元,役重赋烦,生民离散,盗贼猖獗,夷虏犯顺,天下之膏腴已尽,国家之纲纪废弛,虽擢发不足以数京等之罪也。臣等待罪该科,备员谏职,徒以目击奸臣误国,而不为皇上陈之,则上辜君父之恩,下负平生所学。伏乞宸断,将京等一干党恶人犯,或下廷尉,以示薄罚;或致极典,以彰显戮;或照例枷号;或投之荒裔,以御魑魅。庶天意可回,人心畅快,国法以正,虏患自消。天下幸甚!臣民幸甚!奉圣旨:“蔡京姑留辅政。王黼、杨戬着拿送三法司,会问明白来说。钦此钦遵。”续该三法司会问过,并党恶人犯王黼、杨戬,本兵不职,纵虏深入,荼毒生民,损兵折将,失陷内地,律应处斩。手下坏事家人、书办、官掾、亲家董升、卢虎、杨盛、庞宣、韩宗仁、陈洪、黄玉、刘盛、赵弘道等,查出有名人犯,俱问拟枷号一个月,满日发边卫充军。
\]

西门庆不看,万事皆休;看了耳边厢只听飕的一声,魂魄不知往那里去了。就是:

\[
惊伤六叶连肝肺,吓坏三毛七孔心。
\]
当下即忙打点金银宝玩,驮装停当,把家人来保、来旺叫到卧房中,悄悄分付,如此这般:“雇头口星夜上东京打听消息。不消到你陈亲家老爹下处。但有不好声色,取巧打点停当,速来回报。”又与了他二人二十两银子。绝早五更雇脚夫起程,上东京去了,不在话下。

西门庆通一夜不曾睡着,到次日早,分付来昭、贲四,把花园工程止住,各项匠人都且回去,不做了。每日将大门紧闭,家下人无事亦不许往外去。西门庆只在房里走来走去,忧上加忧,闷上加闷,如热地蜒蚰一般,把娶李瓶儿的勾当丢在九霄云外去了。吴月娘见他愁眉不展,面带忧容,只得宽慰他,说道:“他陈亲家那边为事,各人冤有头债有主,你也不需焦愁如此。”西门庆道:“你妇人都知道些甚么?陈亲家是我的亲家,女儿、女婿两个孽障搬来咱家住着,平昔街坊邻舍恼咱的极多,常言:机儿不快梭儿快,打着羊驹驴战。倘有小人指搠,拔树寻根,你我身家不保。”正是:关门家里坐,祸从天上来。这里西门庆在家纳闷,不题。

且说李瓶儿等了一日两日,不见动静,一连使冯妈妈来了两遍,大门关得铁桶相似。等了半日,没一个人牙儿出来,竟不知怎的。看看到二十四日,李瓶儿又使冯妈妈送头面来,就请西门庆过去说话。叫门不开,立在对过房檐下等。少顷,只见玳安出来饮马,看见便问:“冯妈妈,你来做甚么?”冯妈妈说:“你二娘使我送头面来,怎的不见动静?请你爹过去说话哩。”玳安道:“俺爹连日有些事儿,不得闲。你老人家还拿头面去,等我饮马回来,对俺爹说就是了。”冯妈妈道:“好哥哥,我这在里等着,你拿进头面去和你爹说去。你二娘那里好不恼我哩!”这玳安一面把马拴下,走到里边,半日出来道:“对爹说了,头面爹收下了,教你上覆二娘,再待几日儿,我爹出来往二娘那里说话。”这冯妈妈一直走来,回了妇人话。妇人又等了几日,看看五月将尽,六月初旬,朝思暮盼,音信全无,梦攘魂劳,佳期间阻。正是:

\[
懒把蛾眉扫,羞将粉脸匀。
满怀幽恨积,憔悴玉精神。
\]

妇人盼不见西门庆来,每日茶饭顿减,精神恍惚。到晚夕,孤眠枕上展转踌蹰。忽听外边打门,仿佛见西门庆来到。妇人迎门笑接,携手进房,问其爽约之情,各诉衷肠之话。绸缪缱绻,彻夜欢娱。鸡鸣天晓,便抽身回去。妇人恍然惊觉,大呼一声,精魂已失。冯妈妈听见,慌忙进房来看。妇人说道:“西门他爹刚才出去,你关上门不曾?”冯妈妈道:“娘子想得心迷了,那里得大官人来?影儿也没有!”妇人自此梦境随邪,夜夜有狐狸假名抵姓,摄其精髓。渐渐形容黄瘦,饮食不进,卧床不起。冯妈妈向妇人说,请了大街口蒋竹山来看。其人年不上三十,生的五短身材,人物飘逸,极是轻浮狂诈。请入卧室,妇人则雾鬓云鬟,拥衾而卧,似不胜忧愁之状。茶汤已罢,丫鬟安放褥垫。竹山就床诊视脉息毕,因见妇人生有姿色,便开口说道:“学生适诊病源,娘子肝脉弦出寸口而洪大,厥阴脉出寸口久上鱼际,主六欲七情所致。阴阳交争,乍寒乍热,似有郁结于中而不遂之意也。似疟非疟,似寒非寒,白日则倦怠嗜卧,精神短少;夜晚神不守舍,梦与鬼交。若不早治,久而变为骨蒸之疾,必有属纩之忧矣。可惜,可惜!”妇人道:“有累先生,俯赐良剂。奴好了,重加酬谢。”竹山道:“学生无不用心,娘子若服了我的药,必然贵体全安。”说毕起身。这里送药金五星,使冯妈妈讨将药来。妇人晚间吃了药下去,夜里得睡,便不惊恐。渐渐饮食加添,起来梳头走动。那消数日,精神复旧。

一日,安排了一席酒肴,备下三两银子,使冯妈妈请过竹山来相谢。蒋竹山自从与妇人看病,怀觊觎之心已非一日。一闻其请,即具服而往。延之中堂,妇人盛妆出见,道了万福,茶汤两换,请入房中。酒肴已陈,麝兰香蔼。小丫鬟绣春在傍,描金盘内托出三两白金。妇人高擎玉盏,向前施礼,说道:“前日,奴家心中不好,蒙赐良剂,服之见效。今粗治了一杯水酒,请过先生来知谢知谢。”竹山道:“此是学生分内之事,理当措置,何必计较!”因见三两谢礼,说道:“这个学生怎么敢领?”妇人道:“些须微意,不成礼数,万望先生笑纳。”辞让了半日,竹山方才收了。妇人递酒,安下坐次。饮过三巡,竹山偷眼睃视妇人,粉妆玉琢,娇艳惊人,先用言以挑之,因道:“学生不敢动问,娘子青春几何?”妇人道:“奴虚度二十四岁。”竹山道:“似娘子这等妙年,生长深闺,处于富足,何事不遂,而前日有此郁结不足之病?”妇人听了,微笑道:“不瞒先生,奴因拙夫弃世,家事萧条,独自一身,忧愁思虑,何得无病!”竹山道:“原来娘子夫主殁了。多少时了?”妇人道:“拙夫从去岁十一月得伤寒病死了,今已八个月。”竹山道:“曾吃谁的药来?”妇人道:“大街上胡先生。”竹山道:“是那东街上刘太监房子住的胡鬼嘴儿?他又不是我太医院出身,知道甚么脉,娘子怎的请他?”妇人道:“也是因街坊上人荐举请他来看。还是拙夫没命,不干他事。”竹山又道:“娘子也还有子女没有?”妇人道:“儿女俱无。”竹山道:“可惜娘子这般青春妙龄之际,独自孀居,又无所出,何不寻其别进之路?甘为幽闷,岂不生病!”妇人道:“奴近日也讲着亲事,早晚过门。”竹山便道:“动问娘子与何人作亲?”妇人道:“是县前开生药铺西门大官人。”竹山听了道:“苦哉,苦哉!娘子因何嫁他?学生常在他家看病,最知详细。此人专在县中包揽说事,广放私债,贩卖人口,家中丫头不算,大小五六个老婆,着紧打倘棍儿,稍不中意,就令媒人领出卖了。就是打老婆的班头,坑妇女的领袖。娘子早是对我说,不然进入他家,如飞蛾投火一般,坑你上不上,下不下,那时悔之晚矣。况近日他亲家那边为事干连,在家躲避不出,房子盖的半落不合的,都丢下了。东京关下文书,坐落府县拿人。到明日他盖这房子,多是入官抄没的数儿。娘子没来由嫁他做甚?”一篇话把妇人说的闭口无言。况且许多东西丢在他家,寻思半晌,暗中跌脚:“嗔怪道一替两替请着他不来,他家中为事哩!”又见竹山语言活动,一团谦恭:“奴明日若嫁得恁样个人也罢了,不知他有妻室没有?”因说道:“既蒙先生指教,奴家感戴不浅,倘有甚相知人家,举保来说,奴无有个不依之理。”竹山乘机请问:“不知要何等样人家?学生打听的实,好来这里说。”妇人道:“人家到也不论大小,只要象先生这般人物的。”这蒋竹山不听便罢,听了此言,欢喜的满心痒,不知搔处,慌忙走下席来,双膝跪下告道:“不瞒娘子说,学生内帏失助,中馈乏人,鳏居已久,子息全无。倘蒙娘子垂怜,肯结秦晋之缘,足称平生之愿。学生虽衔环结草,不敢有忘。”妇人笑笑,以手携之,说道:“且请起,未审先生鳏居几时?贵庚多少?既要做亲,须得要个保山来说,方成礼数。”竹山又跪下哀告道:“学生行年二十九岁,正月二十七日卯时建生,不幸去年荆妻已故,家缘贫乏,实出寒微。今既蒙金诺之言,何用冰人之讲。”妇人笑道:“你既无钱,我这里有个妈妈姓冯,拉他做个媒证。也不消你行聘,择个吉日良时,招你进来,入门为赘。你意下若何?”这蒋竹山连忙倒身下拜:“娘子就如同学生重生父母,再长爹娘。夙世有缘,三生大幸矣!”一面两个在房中各递了一杯交欢酒,已成其亲事。竹山饮至天晚回家。

妇人这里与冯妈妈商议说:“西门庆如此这般为事,吉凶难保。况且奴家这边没人,不好了一场,险不丧了性命。为今之计,不如把这位先生招他进来,有何不可?”到次日,就使冯妈妈递信过去,择六月十八日大好日子,把蒋竹山倒踏门招进来,成其夫妻。过了三日,妇人凑了三百两银子,与竹山打开两间门面,店内焕然一新。初时往人家看病只是走,后来买了一匹驴儿骑着,在街上往来,不在话下。正是:

\[
一洼死水全无浪,也有春风摆动时。
\]

\newpage
%# -*- coding:utf-8 -*-
%%%%%%%%%%%%%%%%%%%%%%%%%%%%%%%%%%%%%%%%%%%%%%%%%%%%%%%%%%%%%%%%%%%%%%%%%%%%%%%%%%%%%


\chapter{赂相府西门脱祸\KG 见娇娘敬济销魂}


词曰:

\[
有个人人,海棠标韵,飞燕轻盈。酒晕潮红,羞蛾一笑生春。为伊无限伤心,更说甚巫山楚云!斗帐香销,纱窗月冷,着意温存。
\]

话分两头。不说蒋竹山在李瓶儿家招赘,单表来保、来旺二人上东京打点,朝登紫陌,暮践红尘,一日到东京,进了万寿门,投旅店安歇。到次日,街前打听,只听见街谈巷议,都说兵部王尚书昨日会问明白,圣旨下来,秋后处决。止有杨提督名下亲族人等,未曾拿完,尚未定夺。来保等二人把礼物打在身边,急来到蔡府门首。旧时干事来了两遍,道路久熟,立在龙德街牌楼底下,探听府中消息。少顷,只见一个青衣人,慌慌打府中出来,往东去了。来保认得是杨提督府里亲随杨干办,待要叫住问他一声事情如何,因家主不曾分付,以此不言语,放过他去了。迟了半日,两个走到府门前,望着守门官深深唱个喏:“动问一声,太师老爷在家不在?”那守门官道:“老爷朝中议事未回。你问怎的?”来保又问道:“管家翟爷请出来,小人见见,有事禀白。”那官吏道:“管家翟叔也不在了。”来保见他不肯实说,晓得是要些东西,就袖中取出一两银子递与他。那官吏接了便问:“你要见老爷,要见学士大爷?老爷便是大管家翟谦禀,大爷的事便是小管家高安禀,各有所掌。况老爷朝中未回,止有学士大爷在家。你有甚事,我替你请出高管家来,禀见大爷也是一般。”这来保就借情道:“我是提督杨爷府中,有事禀见。”官吏听了,不敢怠慢,进入府中。良久,只见高安出来。来保慌忙施礼,递上十两银子,说道:“小人是杨爷的亲,同杨干办一路来见老爷讨信。因后边吃饭,来迟了一步,不想他先来了。所以不曾赶上。”高安接了礼物,说道:“杨干办只刚才去了,老爷还未散朝。你且待待,我引你再见见大爷罢。”一面把来保领到第二层大厅傍边,另一座仪门进去。坐北朝南三间敞厅,绿油栏杆,朱红牌额,石青镇地,金字大书天子御笔钦赐“学士琴堂”四字。

原来蔡京儿子蔡攸,也是宠臣,见为祥和殿学士兼礼部尚书、提点太乙宫使。来保在门外伺候,高安先入,说了出来,然后唤来保入见,当厅跪下。蔡攸深衣软巾,坐于堂上,问道:“你是那里来的?”来保禀道:“小人是杨爷的亲家陈洪的家人,同府中杨干办来禀见老爷讨信。不想杨干办先来见了,小人赶来后见。”因向袖中取出揭帖递上。蔡攸见上面写着“白米五百石”,叫来保近前说道:“蔡老爷亦因言官论列,连日回避。阁中之事并昨日三法司会问,都是右相李爷秉笔。杨老爷的事,昨日内里有消息出来,圣上宽恩,另有处分了。其手下用事有名人犯,待查明问罪。你还到李爷那里去说。”来保只顾磕头道:“小的不认的李爷府中,望爷怜悯,看家杨老爷分上。”蔡攸道:“你去到天汉桥边北高坡大门楼处,问声当朝右相、资政殿大学士兼礼部尚书讳邦彦的你李爷,谁是不知道!也罢,我这里还差个人同你去。”即令祗候官呈过一缄,使了图书,就差管家高安同去见李爷,如此替他说。

那高安承应下了,同来保去了府门,叫了来旺,带着礼物,转过龙德街,迳到天汉桥李邦彦门首。正值邦彦朝散才来家,穿大红绉纱袍,腰系玉带,送出一位公卿上轿而去,回到厅上,门吏禀报说:“学士蔡大爷差管家来见。”先叫高安进去说了回话,然后唤来保、来旺进见,跪在厅台下。高安就在傍边递了蔡攸封缄,并礼物揭帖,来保下边就把礼物呈上。邦彦看了说道:“你蔡大爷分上,又是你杨老爷亲,我怎么好受此礼物?况你杨爷,昨日圣心回动,已没事。但只手下之人,科道参语甚重,一定问发几个。”即令堂候官取过昨日科中送的那几个名字与他瞧。上面写着:“王黼名下书办官董升,家人王廉,班头黄玉,杨戬名下坏事书办官卢虎,干办杨盛,府掾韩宗仁、赵弘道,班头刘成,亲党陈洪、西门庆、胡四等,皆鹰犬之徒,狐假虎威之辈。乞敕下法司,将一干人犯,或投之荒裔以御魍魉,或置之典刑,以正国法。”来保见了,慌的只顾磕头,告道:“小人就是西门庆家人,望老爷开天地之心,超生性命则个!”高安又替他跪禀一次。邦彦见五百两金银,只买一个名字,如何不做分上?即令左右抬书案过来,取笔将文卷上西门庆名字改作贾廉,一面收上礼物去。邦彦打发来保等出来,就拿回帖回学士,赏了高安、来保、来旺一封五两银子。

来保路上作辞高管家,回到客店,收拾行李,还了房钱,星夜回清河县。来家见西门庆,把东京所干的事,从头说了一遍。西门庆听了,如提在冷水盆内,对月娘说:“早时使人去打点,不然怎了!”正是,这回西门庆性命有如——

\[
落日已沉西岭外,却被扶桑唤出来。
\]
于是一块石头方才落地。过了两日,门也不关了,花园照旧还盖,渐渐出来街上走动。

一日,玳安骑马打狮子街过,看见李瓶儿门首开个大生药铺,里边堆着许多生熟药材。朱红小柜,油漆牌匾,吊着幌子,甚是热闹。归来告与西门庆说——还不知招赘蒋竹山一节,只说:“二娘搭了个新伙计,开了个生药铺。”西门庆听了,半信不信。

一日,七月中旬,金风淅淅,玉露泠泠。西门庆正骑马街上走着,撞见应伯爵、谢希大。两人叫住,下马唱喏,问道:“哥,一向怎的不见?兄弟到府上几遍,见大门关着,又不敢叫,整闷了这些时。端的哥在家做甚事?嫂子娶进来不曾?也不请兄弟们吃酒。”西门庆道:“不好告诉的。因舍亲陈宅那边为些闲事,替他乱了几日。亲事另改了日期了。”伯爵道:“兄弟们不知哥吃惊。今日既撞遇哥,兄弟二人肯空放了?如今请哥同到里边吴银姐那里吃三杯,权当解闷。”不由分说,把西门庆拉进院中来。正是:

\[
高榭樽开歌妓迎,漫夸解语一含情。
纤手传杯分竹叶,一帘秋水浸桃笙。
\]

当日西门庆被二人拉到吴银儿家,吃了一日酒。到日暮时分,已带半酣,才放出来。打马正走到东街口上,撞见冯妈妈从南来,走得甚慌。西门庆勒住马,问道:“你那里去?”冯妈妈道:“二娘使我往门外寺里鱼篮会,替过世二爷烧箱库去来。”西门庆醉中道:“你二娘在家好么?我明日和他说话去。”冯妈妈道:“还问甚么好?把个见见成成做熟了饭的亲事,吃人掇了锅儿去了。”西门庆听了失声惊问道:“莫不他嫁人去了?”冯妈妈道:“二娘那等使老身送过头面,往你家去了几遍不见你,大门关着。对大官儿说进去,教你早动身,你不理。今教别人成了,你还说甚的?”西门庆问:“是谁?”冯妈妈悉把半夜三更妇人被狐狸缠着,染病看看至死,怎的请了蒋竹山来看,吃了他的药怎的好了,某日怎的倒踏门招进来,成其夫妇,见今二娘拿出三百两银子与他开了生药铺,从头至尾说了一遍。这西门庆不听便罢,听了气的在马上只是跌脚,叫道:“苦哉!你嫁别人,我也不恼,如何嫁那矮王八!他有甚么起解?”于是一直打马来家。

刚下马进仪门,只见吴月娘、孟玉楼、潘金莲并西门大姐四个,在前厅天井内月下跳马索儿耍子。见西门庆来家,月娘、玉楼、大姐三个都往后走了。只有金莲不去,且扶着庭柱兜鞋,被西门庆带酒骂道:“淫妇们闲的声唤,平白跳甚么百索儿?”赶上金莲踢了两脚。走到后边,也不往月娘房中去脱衣裳,走在西厢一间书房内,要了铺盖,那里宿歇。打丫头,骂小厮,只是没好气。众妇人同站在一处,都甚是着恐,不知是那缘故。吴月娘埋怨金莲:“你见他进门有酒了,两三步叉开一边便了。还只顾在跟前笑成一块,且提鞋儿,却教他蝗虫蚂蚱一例都骂着。”玉楼道:“骂我们也罢,如何连大姐姐也骂起淫妇来了?没槽道的行货子!”金莲接过来道:“这一家子只是我好欺负的!一般三个人在这里,只踢我一个儿。那个偏受用着甚么也怎的?”月娘就恼了,说道:“你头里何不叫他连我踢不是?你没偏受用,谁偏受用?恁的贼不识高低货!我到不言语,你只顾嘴头子哗哩薄喇的!”金莲见月娘恼了,便把话儿来摭,说道:“姐姐,不是这等说。他不知那里因着甚么头由儿,只拿我煞气。要便睁着眼望着俺叫,千也要打个臭死,万也要打个臭死!”月娘道:“谁教你只要嘲他来?他不打你,却打狗不成!”玉楼道:“大姐姐,且叫小厮来问他声,今日在谁家吃酒来?早晨好好出去,如何来家恁个腔儿!”不一时,把玳安叫到跟前,月娘骂道:“贼囚根子!你不实说,教大小厮来拷打你和平安儿,每人都是十板。”玳安道:“娘休打,待小的实说了罢。爹今日和应二叔们都在院里吴家吃酒,散了来在东街口上,撞遇冯妈妈,说花二娘等爹不去,嫁了大街住的蒋太医了。爹一路上恼的要不的。”月娘道:“信那没廉耻的歪淫妇,浪着嫁了汉子,来家拿人煞气。”玳安道:“二娘没嫁蒋太医,把他倒踏门招进去了。如今二娘与他本钱,开了好不兴的生药铺。我来家告爹说,爹还不信。”孟玉楼道:“论起来,男子汉死了多少时儿?服也还未满,就嫁人,使不得的!”月娘道:“如今年程,论的甚么使的使不的。汉子孝服未满,浪着嫁人的,才一个儿?淫妇成日和汉子酒里眠酒里卧的人,他原守的甚么贞节!”看官听说:月娘这一句话,一棒打着两个人——孟玉楼与潘金莲都是孝服不曾满再醮人的,听了此言,未免各人怀着惭愧归房,不在话下。正是:

\[
不如意事常八九,可与人言无二三。
\]

却说西门庆当晚在前边厢房睡了一夜。到次日早,把女婿陈敬济安在他花园中,同贲四管工记帐,换下来招教他看守大门。西门大姐白日里便在后边和月娘众人一处吃酒,晚夕归到前边厢房中歇。陈敬济每日只在花园中管工,非呼唤不敢进入中堂,饮食都是内里小厮拿出来吃。所以西门庆手下这几房妇人都不曾见面。一日,西门庆不在家,与提刑所贺千户送行去了。月娘因陈敬济一向管工辛苦,不曾安排一顿饭儿酬劳他,向孟玉楼、李娇儿说:“待要管,又说我多揽事;我待欲不管,又看不上。人家的孩儿在你家,每日早起睡晚,辛辛苦苦,替你家打勤劳儿,那个与心知慰他一知慰儿也怎的?”玉楼道:“姐姐,你是个当家的人,你不上心谁上心!”月娘于是分付厨下,安排了一桌酒肴点心,午间请陈敬济进来吃一顿饭。这陈敬济撇了工程教贲四看管,迳到后边参见月娘,作揖毕,旁边坐下。小玉拿茶来吃了,安放桌儿,拿蔬菜按酒上来。月娘道:“姐夫每日管工辛苦,要请姐夫进来坐坐,白不得个闲。今日你爹不在家,无事,治了一杯水酒,权与姐夫酬劳。”敬济道:“儿子蒙爹娘抬举,有甚劳苦,这等费心!”月娘陪着他吃了一回酒。月娘使小玉:“请大姑娘来这里坐。”小玉道:“大姑娘使着手,就来。”少顷,只听房中抹得牌响。敬济便问:“谁人抹牌?”月娘道:“是大姐与玉箫丫头弄牌。”敬济道:“你看没分晓,娘这里呼唤不来,且在房中抹牌。”一不时,大姐掀帘子出来,与他女婿对面坐下,一周饮酒。月娘便问大姐:“陈姐夫也会看牌不会?”大姐道:“他也知道些香臭儿。”月娘只知敬济是志诚的女婿,却不道这小伙子儿诗词歌赋,双陆象棋,拆牌道字,无所不通,无所不晓。正是:

\[
自幼乖滑伶俐,风流博浪牢成。爱穿鸭绿出炉银,双陆象棋帮衬。琵琶笙筝箫管,弹丸走马员情。只有一件不堪闻:见了佳人是命。
\]

月娘便道:“既是姐夫会看牌,何不进去咱同看一看?”敬济道:“娘和大姐看罢,儿子却不当。”月娘道:“姐夫至亲间,怕怎的?”一面进入房中,只见孟玉楼正在床上铺茜红毡看牌,见敬济进来,抽身就要走。月娘道:“姐夫又不是别人,见个礼儿罢。”向敬济道:“这是你三娘哩。”那敬济慌忙躬身作揖,玉楼还了万福。当下玉楼、大姐三人同抹,敬济在傍边观看。抹了一回,大姐输了下来,敬济上来又抹。玉楼出了个天地分;敬济出了个恨点不到;吴月娘出了个四红沉八不就,双三不搭两么儿,和儿不出,左来右去配不着色头。只见潘金莲掀帘子进来,银丝鬒髻上戴着一头鲜花儿,笑嘻嘻道:“我说是谁,原来是陈姐夫在这里。”慌的陈敬济扭颈回头,猛然一见,不觉心荡目摇,精魂已失。正是:五百年冤家相遇,三十年恩爱一旦遭逢。月娘道:“此是五娘,姐夫也只见个长礼儿罢。”敬济忙向前深深作揖,金莲一面还了万福。月娘便道:“五姐你来看,小雏儿倒把老鸦子来赢了。”这金莲近前一手扶着床护炕儿,一只手拈着白纱团扇儿,在傍替月娘指点道:“大姐姐,这牌不是这等出了,把双三搭过来,却不是天不同和牌?还赢了陈姐夫和三姐姐。”众人正抹牌在热闹处,只见玳安抱进毡包来,说:“爹来家了。”月娘连忙撺掇小玉送姐夫打角门出去了。

西门庆下马进门,先到前边工上观看了一遍,然后踅到潘金莲房中来。金莲慌忙接着,与他脱了衣裳,说道:“你今日送行去来的早。”西门庆道:“提刑所贺千户新升新平寨知寨,合卫所相知都郊外送他来,拿帖儿知会我,不好不去的。”金莲道:“你没酒,教丫鬟看酒来你吃。”不一时,放了桌儿饮酒,菜蔬都摆在面前。饮酒中间,因说起后日花园卷棚上梁,约有许多亲朋都要来递果盒酒挂红,少不得叫厨子置酒管待。说了一回,天色已晚。春梅掌灯归房,二人上床宿歇。西门庆因起早送行,着了辛苦,吃了几杯酒就醉了。倒下头鼾睡如雷,齁齁不醒。那时正值七月二十头天气,夜间有些余热,这潘金莲怎生睡得着?忽听碧纱帐内一派蚊雷,不免赤着身子起来,执烛满帐照蚊。照一个,烧一个。回首见西门庆仰卧枕上,睡得正浓,摇之不醒。其腰间那话,带着托子,累垂伟长,不觉淫心辄起,放下烛台,用纤手扪弄。弄了一回,蹲下身去,用口吮之。吮来吮去,西门庆醒了,骂道:“怪小淫妇儿,你达达睡睡,就掴掍死了。”一面起来,坐在枕上,亦发叫他在下尽着吮咂;又垂首玩之,以畅其美。正是:

\[
怪底佳人风性重,夜深偷弄紫箫吹。
\]
又有蚊子双关《踏莎行》词为证:

\[
我爱他身体轻盈,楚腰腻细。行行一派笙歌沸。黄昏人未掩朱扉,潜身撞入纱厨内。款傍香肌,轻怜玉体。嘴到处,胭脂记。耳边厢造就百般声,夜深不肯教人睡。
\]

妇人顽了有一顿饭时,西门庆忽然想起一件事来,叫春梅筛酒过来,在床前执壶而立。将烛移在床背板上,教妇人马爬在他面前,那话隔山取火,托入牡中,令其自动,在上饮酒取乐。妇人骂道:“好个刁钻的强盗!从几时新兴出来的例儿,怪剌剌教丫头看答着,甚么张致!”西门庆道:“我对你说了罢,当初你瓶姨和我常如此干,叫他家迎春在傍执壶斟酒,到好耍子。”妇人道:“我不好骂出来的,甚么瓶姨鸟姨,题那淫妇做甚,奴好心不得好报。那淫妇等不的,浪着嫁汉子去了。你前日吃了酒来家,一般的三个人在院子里跳百索儿,只拿我煞气,只踢我一个儿,倒惹的人和我辨了回子嘴。想起来,奴是好欺负的!”西门庆问道:“你与谁辨嘴来?”妇人道:“那日你便进来了,上房的好不和我合气,说我在他跟前顶嘴来,骂我不识高低的货。我想起来为甚么?养虾蟆得水虫儿病,如今倒教人恼我!”西门庆道:“不是我也不恼,那日应二哥他们拉我到吴银儿家,吃了酒出来,路上撞见冯妈妈子,这般告诉我,把我气了个立睁。若嫁了别人,我到罢了。那蒋太医贼矮忘八,那花大怎不咬下他下截来?他有甚么起解?招他进去,与他本钱,教他在我眼面前开铺子,大剌剌的做买卖!”妇人道:“亏你脸嘴还说哩!奴当初怎么说来?先下米儿先吃饭。你不听,只顾来问大姐姐。常言:信人调,丢了瓢。你做差了,你埋怨那个?”西门庆被妇人几句话,冲得心头一点火起,云山半壁通红,便道:“你由他,教那不贤良的淫妇说去。到明日休想我理他!”看官听说:自古谗言罔行,君臣、父子、夫妇、昆弟之间,皆不能免。饶吴月娘恁般贤淑,西门庆听金莲衽席睥睨之间言,卒致于反目,其他可不慎哉!自是以后,西门庆与月娘尚气,彼此觌面,都不说话。月娘随他往那房里去,也不管他;来迟去早,也不问他;或是他进房中取东取西,只教丫头上前答应,也不理他。两个都把心冷淡了。正是:

\[
前车倒了千千辆,后车到了亦如然。
分明指与平川路,却把忠言当恶言。
\]

且说潘金莲自西门庆与月娘尚气之后,见汉子偏听,以为得志。每日抖擞着精神,妆饰打扮,希宠市爱。因为那日后边会着陈敬济一遍,见小伙儿生的乖猾伶俐,有心也要勾搭他。但只畏惧西门庆,不敢下手。只等西门庆往那里去,便使了丫鬟叫进房中,与他茶水吃,常时两个下棋做一处。一日西门庆新盖卷棚上梁,亲友挂红庆贺,递果盒。许多匠作,都有犒劳赏赐。大厅上管待客官,吃到午晌,人才散了。西门庆因起得早,就归后边睡去了。陈敬济走来金莲房中讨茶吃。金莲正在床上弹弄琵琶,道:“前边上梁,吃了这半日酒,你就不曾吃些甚么,还来我屋里要茶吃?”敬济道:“儿子不瞒你老人家说,从半夜起来,乱了这一五更,谁吃甚么来!”妇人问道:“你爹在那里?”敬济道:“爹后边睡去了。”妇人道:“你既没吃甚么,”叫春梅:“拣籹里拿我吃的那蒸酥果馅饼儿来,与你姐夫吃。”这小伙儿就在他炕桌儿上摆着四碟小菜,吃着点心。因见妇人弹琵琶,戏问道:“五娘,你弹的甚曲儿?怎不唱个儿我听。”妇人笑道:“好陈姐夫,奴又不是你影射的,如何唱曲儿你听?我等你爹起来,看我对你爹说不说!”那敬济笑嘻嘻,慌忙跪着央及道:“望乞五娘可怜见,儿子再不敢了!”那妇人笑起来了。自此这小伙儿和这妇人日近日亲,或吃茶吃饭,穿房入屋,打牙犯嘴,挨肩擦背,通不忌惮。月娘托以儿辈,放这样不老实的女婿在家,自家的事却看不见。正是:

\[
只晓采花成酿蜜,不知辛苦为谁甜。
\]

\newpage
%# -*- coding:utf-8 -*-
%%%%%%%%%%%%%%%%%%%%%%%%%%%%%%%%%%%%%%%%%%%%%%%%%%%%%%%%%%%%%%%%%%%%%%%%%%%%%%%%%%%%%


\chapter{草里蛇逻打蒋竹山\KG 李瓶儿情感西门庆}


诗曰:

\[
人靡不有初,想君能终之。别来历年岁,旧恩何可期。
重新而忘故,君子所犹讥。寄身虽在远,岂忘君须臾。
既厚不为薄,想君时见思。
\]

话说西门庆起盖花园卷棚,约有半年光阴,装修油漆完备,前后焕然一新。庆房的整吃了数日酒,俱不在话下。

一日,八月初旬,与夏提刑做生日,在新买庄上摆酒。叫了四个唱的、一起乐工、杂耍步戏。西门庆从巳牌时分,就骑马去了。吴月娘在家,整置了酒肴细果,约同李娇儿、孟玉楼、孙雪娥、大姐、潘金莲众人,开了新花园门游赏。里面花木庭台,一望无际,端的好座花园。但见:

\[
正面丈五高,周围二十板。当先一座门楼,四下几间台榭。假山真水,翠竹苍松。高而不尖谓之台,巍而不峻谓之榭。四时赏玩,各有风光:春赏燕游堂,桃李争妍;夏赏临溪馆,荷莲斗彩;秋赏叠翠楼,黄菊舒金;冬赏藏春阁,白梅横玉。更有那娇花笼浅径,芳树压雕栏,弄风杨柳纵蛾眉,带雨海棠陪嫩脸。燕游堂前,灯光花似开不开;藏春阁后,白银杏半放不放。湖山侧才绽金钱,宝槛边初生石笋。翩翩紫燕穿帘幕,呖呖黄莺度翠阴。也有那月窗雪洞,也有那水阁风亭。木香棚与荼蘼架相连,千叶桃与三春柳作对。松墙竹径,曲水方池,映阶蕉棕,向日葵榴。游渔藻内惊人,粉蝶花间对舞。正是:芍药展开菩萨面,荔枝擎出鬼王头。
\]

当下吴月娘领着众妇人,或携手游芳径之中,或斗草坐香茵之上。一个临轩对景,戏将红豆掷金鳞;一个伏槛观花,笑把罗纨惊粉蝶。月娘于是走在一个最高亭子上,名唤卧云亭,和孟玉楼、李娇儿下棋。潘金莲和西门大姐、孙雪娥都在玩花楼望下观看。见楼前牡丹花畔,芍药圃、海棠轩、蔷薇架、木香棚,又有耐寒君子竹、欺雪大夫松。端的四时有不谢之花,八节有长春之景。观之不足,看之有余。不一时摆上酒来,吴月娘居上,李娇儿对席,两边孟玉楼、孙雪娥、潘金莲、西门大姐,各依序而坐。月娘道:“我忘了请姐夫来坐坐。”一面使小玉:“前边快请姑夫来。”不一时,敬济来到,头上天青罗帽,身穿紫绫深衣,脚下粉头皂靴,向前作揖,就在大姐跟前坐下。传杯换盏,吃了一回酒,吴月娘还与李娇儿、西门大姐下棋。孙雪娥与孟玉楼却上楼观看。惟有金莲,且在山子前花池边,用白纱团扇扑蝴蝶为戏。不妨敬济悄悄在他背后戏说道:“五娘,你不会扑蝴蝶儿,等我替你扑。这蝴蝶儿忽上忽下心不定,有些走滚。”那金莲扭回粉颈,斜瞅了他一眼,骂道:“贼短命,人听着,你待死也!我晓得你也不要命了。”那敬济笑嘻嘻扑近他身来,搂他亲嘴。被妇人顺手只一推,把小伙儿推了一交。却不想玉楼在玩花楼远远瞧见,叫道:“五姐,你走这里来,我和你说话。”金莲方才撇了敬济,上楼去了。原来两个蝴蝶到没曾捉得住,到订了燕约莺期,则做了蜂须花嘴。正是:

\[
狂蜂浪蝶有时见,飞入梨花没寻处。
\]

敬济见妇人去了,默默归房,心中怏怏不乐。口占《折桂令》一词,以遣其闷:

\[
我见他斜戴花枝,朱唇上不抹胭脂,似抹胭脂。前日相逢,似有私情,未见私情。欲见许,何曾见许!似推辞,本是不推辞。约在何时?会在何时?不相逢,他又相思;既相逢,我又相思。
\]

且不说吴月娘等在花园中饮酒。单表西门庆从门外夏提刑庄子上吃了酒回家,打南瓦子巷里头过。平昔在三街两巷行走,捣子们都认的——宋时谓之捣子,今时俗呼为光棍。内中有两个,一名草里蛇鲁华,一名过街鼠张胜,常受西门庆资助,乃鸡窃狗盗之徒。西门庆见他两个在那里耍钱,就勒住马,上前说话。二人连忙走到跟前,打个半跪道:“大官人,这咱晚往那里去来?”西门庆道:“今日是提刑所夏老爹生日,门外庄上请我们吃了酒来。我有一椿事央烦你们,依我不依?”二人道:“大官人没的说,小人平昔受恩甚多,如有使令,虽赴汤蹈火,万死何辞!”西门庆道:“既是恁说,明日来我家,我有话分付你。”二人道:“那里等的到明日!你老人家说与小人罢,端的有甚么事?”西门庆附耳低言,便把蒋竹山要了李瓶儿之事说了一遍:“只要你弟兄二人替我出这口气儿便了!”因在马上搂起衣底顺袋中,还有四五两碎银子,都倒与二人。便道:“你两个拿去打酒吃。只要替我干得停当,还谢你二人。”鲁华那里肯接,说道:“小人受你老人家恩还少哩!我只道教俺两个往东洋大海里拔苍龙头上角,西华岳山中取猛虎口中牙,便去不的,这些小之事,有何难哉!这个银两,小人断不敢领。”西门庆道:“你不收,我也不央及你了。”教玳安接了银子,打马就走。又被张胜拦住说:“鲁华,你不知他老人家性儿?你不收,恰似咱每推脱的一般。”一面接了银子,扒到地下磕了头,说道:“你老人家只顾家里坐着,不消两日,管情稳日日教你笑一声。”张胜道:“只望大官人到明日,把小人送与提刑夏老爹那里答应,就勾了小人了。”西门庆道:“这个不打紧。”后来西门庆果然把张胜送在守备府做了个亲随。此系后事,表过不题。那两个捣子,得了银子,依旧耍钱去了。

西门庆骑马来家,已是日西时分。月娘等众人,听见他进门,都往后边去了,只有金莲在卷棚内看收家活。西门庆不往后边去,迳到花园里来,见妇人在亭子上收家伙,便问:“我不在,你在这里做甚么来?”金莲笑道:“俺们今日和大姐姐开门看了看,谁知你来的恁早。”西门庆道:“今日夏大人费心,庄子上叫了四个唱的,只请了五位客到。我恐怕路远,来的早。”妇人与他脱了衣裳,因说道:“你没酒,教丫头看酒来你吃。”西门庆分付春梅:“把别的菜蔬都收下去,只留下几碟细果子儿,筛一壶葡萄酒来我吃。”坐在上面椅子上,因看见妇人上穿沉香色水纬罗对襟衫儿,五色绉纱眉子,下着白碾光绢挑线裙儿,裙边大红段子白绫高低鞋儿。头上银丝鬒髻,金镶分心翠梅钿儿,云鬓簪着许多花翠。越显得红馥馥朱唇、白腻腻粉脸,不觉淫心辄起,搀着他两只手儿,搂抱在一处亲嘴。不一时,春梅筛上酒来,两个一递一口儿饮酒咂舌。妇人一面抠起裙子,坐在身上,噙酒哺在他口里,然后纤手拈了一个鲜莲蓬子,与他吃。西门庆道:“涩剌剌的,吃他做甚么?”妇人道:“我的儿,你就吊了造化了,娘手里拿的东西儿你不吃!”又口中噙了一粒鲜核桃仁儿,送与他,才罢了。西门庆又要玩弄妇人的胸乳。妇人一面摊开罗衫,露出美玉无瑕、香馥馥的酥胸,紧就就的香乳。揣摸良久,用口舐之,彼此调笑,曲尽“于飞”。

西门庆乘着欢喜,向妇人道:“我有一件事告诉你,到明日,教你笑一声。你道蒋太医开了生药铺,到明日管情教他脸上开果子铺来。”妇人便问怎么缘故。西门庆悉把今日门外撞遇鲁、张二人之事,告诉了一遍。妇人笑道:“你这个众生,到明日不知作多少罪业。”又问:“这蒋太医,不是常来咱家看病的么?我见他且是谦恭,见了人把头只低着,可怜见儿的,你这等做作他!”西门庆道:“你看不出他。你说他低着头儿,他专一看你的脚哩。”妇人道:“汗邪的油嘴!他可可看人家老婆的脚?我不信,他一个文墨人儿,也干这个营生?”西门庆道:“你看他迎面儿,就误了勾当,单爱外装老成内藏奸诈。”两个说笑了一回,不吃酒了,收拾了家活,归房宿歇,不在话下。

却说李瓶儿招赘了蒋竹山,约两月光景。初时蒋竹山图妇人喜欢,修合了些戏药,买了些景东人事、美女想思套之类,实指望打动妇人。不想妇人在西门庆手里狂风骤雨经过的,往往干事不称其意,渐生憎恶,反被妇人把淫器之物,都用石砸的稀碎丢掉了。又说:“你本虾鳝,腰里无力,平白买将这行货子来戏弄老娘!把你当块肉儿,原来是个中看不中吃腊枪头,死王八!”常被妇人半夜三更赶到前边铺子里睡。于是一心只想西门庆,不许他进房。每日躁聒着算帐,查算本钱。

这竹山正受了一肚气,走在铺子小柜里坐的,只见两个人进来,吃的浪浪跄跄,楞楞睁睁,走在凳子上坐下。先是一个问道:“你这铺中有狗黄没有?”竹山笑道:“休要作戏。只有牛黄,那有狗黄?”又问:“没有狗黄,你有冰灰也罢,拿来我瞧,我要买你几两。”竹山道:“生药行只有冰片,是南海波斯国地道出的,那讨冰灰来?”那一个说道:“你休问他,量他才开了几日铺子,那里有这两椿药材?只与他说正经话罢。蒋二哥,你休推睡里梦里。你三年前死了娘子儿,问这位鲁大哥借的那三十两银子,本利也该许多,今日问你要来了。俺们才进门就先问你要,你在人家招赘了,初开了这个铺子,恐怕丧了你行止,显的俺们没阴骘了。故此先把几句风话来教你认范。你不认范,他这银子你少不得还他。”竹山听了,吓了个立睁,说道:“我并没有借他甚么银子。”那人道:“你没借银,却问你讨?自古苍蝇不钻那没缝的蛋,快休说此话!”竹山道:“我不知阁下姓甚名谁,素不相识,如何来问我要银子?”那人道:“蒋二哥,你就差了!自古于官不贫,赖债不富。想着你当初不得地时,串铃儿卖膏药,也亏了这位鲁大哥扶持,你今日就到这田地来。”这个人道:“我便姓鲁,叫做鲁华,你某年借了我三十两银子,发送妻小,本利该我四十八两,少不的还我。”竹山慌道:“我那里借你银子来?就借你银子,也有文书保人。”张胜道:“我张胜就是保人。”因向袖中取出文书,与他照了照。把竹山气的脸腊查也似黄了,骂道:“好杀才,狗男女!你是那里捣子,走来吓诈我!”鲁华听了,心中大怒,隔着小柜,飕的一拳去,早飞到竹山面门上,就把鼻子打歪在半边,一面把架上药材撒了一街。竹山大骂:“好贼捣子!你如何来抢夺我货物?”因叫天福儿来帮助,被鲁华一脚踢过一边,那里再敢上前。张胜把竹山拖出小柜来,拦住鲁华手,劝道:“鲁大哥,你多日子也耽待了,再宽他两日儿,教他凑过与你便了。蒋二哥,你怎么说?”竹山道:“我几时借他银子来?就是问你借的,也等慢慢好讲,如何这等撒野?”张胜道:“蒋二哥,你这回吃了橄榄灰儿——回过味来了。你若好好早这般,我教鲁大哥饶让你些利钱儿,你便两三限凑了还他,才是话。你如何把硬话儿不认,莫不人家就不问你要罢?”那竹山听了道:“气杀我,我和他见官去!谁借他甚么钱来!”张胜道:“你又吃了早酒了!”不提防鲁华又是一拳,仰八叉跌了一交,险不倒栽入洋沟里,将发散开,巾帻都污浊了。竹山大叫“青天白日”起来,被保甲上来,都一条绳子拴了。李瓶儿在房中听见外边人嚷,走来帘下听觑,见地方拴的竹山去了,气的个立睁。使出冯妈妈来,把牌面幌子都收了。街上药材,被人抢了许多。一面关闭了门户,家中坐的。

早有人把这件事报与西门庆知道,即差人分付地方,明日早解提刑院。这里又拿帖子,对夏大人说了。次日早,带上人来,夏提刑升厅,看了地方呈状,叫上竹山去,问道:“你是蒋文蕙?如何借了鲁华银子不还,反行毁打他?甚情可恶!”竹山道:“小人通不认的此人,并没借他银子。小人以理分说,他反不容,乱行踢打,把小人货物都抢了。”夏提刑便叫鲁华:“你怎么说?”鲁华道:“他原借小的银两,发送丧妻,至今三年,延挨不还。小的今日打听他在人家招赘,做了大买卖,问他理讨,他倒百般辱骂小的,说小的抢夺他的货物。见有他借银子的文书在此,这张胜就是保人,望爷察情。”一面怀中取出文契,递上去。夏提刑展开观看,写道:

\[
立借票人蒋文蕙,系本县医生,为因妻丧,无钱发送,凭保人张胜,借到鲁华名下白银三十两,月利三分,入手用度。约至次年,本利交还,不致少欠。恐后无凭,立此借票存照。
\]

夏提刑看了,拍案大怒道:“可又来,见有保人、借票,还这等抵赖。看这厮咬文嚼字模样,就象个赖债的。”喝令左右:“选大板,拿下去着实打。”当下三、四个人,不由分说,拖翻竹山在地,痛责三十大板,打的皮开肉绽,鲜血淋漓。一面差两个公人,拿着白牌,押蒋竹山到家,处三十两银子交还鲁华。不然,带回衙门收监。

那蒋竹山打的两腿剌八着,走到家哭哭啼啼哀告李瓶儿,问他要银子,还与鲁华。又被妇人哕在脸上,骂道:“没羞的忘八,你递甚么银子在我手里,问我要银子?我早知你这忘八砍了头是个债椿,就瞎了眼也不嫁你这中看不中吃的忘八!”那四个人听见屋里嚷骂,不住催逼叫道:“蒋文蕙既没银子,不消只管挨迟了,趁早到衙门回话去罢。”竹山一面出来安抚了公人,又去里边哀告妇人。直蹶儿跪在地上,哭哭啼啼说道:“你只当积阴骘,四山五舍斋佛布施这三十两银子罢!不与这一回去,我这烂屁股上怎禁的拷打?就是死罢了。”妇人不得已拿出三十两雪花银子与他,当官交与鲁华,扯碎了文书,方才完事。

这鲁华、张胜得了三十两银子,迳到西门庆家回话。西门庆留在卷棚下,管待二人酒饭。把前事告诉了一遍。西门庆满心大喜说:“二位出了我这口气,足勾了。”鲁华把三十两银子交与西门庆,西门庆那里肯收:“你二人收去,买壶酒吃,就是我酬谢你了。后头还有事相烦。”二人临起身谢了又谢,拿着银子,自行耍钱去了。正是:

\[
常将压善欺良意,权作尤云殢雨心。
\]

却说蒋竹山提刑院交了银子,归到家中。妇人那里容他住,说道:“只当奴害了汗病,把这三十两银子问你讨了药吃了。你趁早与我搬出去罢!再迟些时,连我这两间房子,尚且不勾你还人!”这蒋竹山只知存身不住,哭哭啼啼,忍着两腿疼,自去另寻房儿。但是妇人本钱置的货物都留下,把他原旧的药材、药碾、药筛、药箱之物,即时催他搬去,两个就开交了。临出门,妇人还使冯妈妈舀了一盆水,赶着泼去,说道:“喜得冤家离眼睛!”当日打发了竹山出门。这妇人一心只想着西门庆,又打听得他家中没事,心中甚是懊悔。每日茶饭慵餐,娥眉懒画,把门儿倚遍,眼儿望穿,白盼不见一个人儿来。正是:

\[
枕上言犹在,于今恩爱沦。
房中人不见,无语自消魂。
\]

不说妇人思想西门庆,单表一日玳安骑马打门首经过,看见妇人大门关着,药铺不开,静落落的,归来告诉与西门庆。西门庆道:“想必那矮忘八打重了,在屋里睡哩,会胜也得半个月出不来做买卖。”遂把这事情丢下了。一日,八月十五日,吴月娘生日,家中有许多堂客来,在大厅上坐。西门庆因与月娘不说话,一迳来院中李桂姐家坐的,分付玳安:“早回马去罢,晚上来接我。”旋邀了应伯爵、谢希大来打双陆。那日桂卿也在家,姐妹两个陪侍劝酒。良久,都出来院子内投壶耍子。玳安约至日西时分,勒马来接。西门庆正在后边出恭,见了玳安问:“家中无事?”玳安道:“家中没事。大厅上堂客都散了,止有大妗子与姑奶奶众人,大娘邀的后边去了。今日狮子街花二娘那里,使了老冯与大娘送生日礼来:四盘羹果、两盘寿桃面、一匹尺头,又与大娘做了一双鞋。大娘与了老冯一钱银子,说爹不在家了。也没曾请去。”西门庆因见玳安脸红红的,便问:“你那里吃酒来?”玳安道:“刚才二娘使冯妈妈叫了小的去,与小的酒吃。我说不吃酒,强说着叫小的吃了两锺,就脸红起来。如今二娘到悔过来,对着小的好不哭哩。前日我告爹说,爹还不信。从那日提刑所出来,就把蒋太医打发去了。二娘甚是懊悔,一心还要嫁爹,比旧瘦了好些儿,央及小的好歹请爹过去,讨爹示下。爹若吐了口儿,还教小的回他一声。”西门庆道:“贼贱淫妇,既嫁汉子去罢了,又来缠我怎的?既是如此,我也不得闲去。你对他说,甚么下茶下礼,拣个好日子,抬了那淫妇来罢。”玳安道:“小的知道了。他那里还等着小的去回他话哩,教平安、画童儿这里伺候爹就是了。”西门庆道:“你去,我知道了。”这玳安出了院门,一直走到李瓶儿那里,回了妇人话。妇人满心欢喜,说道:“好哥哥,今日多累你对爹说,成就了此事。”于是亲自下厨整理蔬菜,管待玳安,说道:“你二娘这里没人,明日好歹你来帮扶天福儿,着人搬家伙过去。”次日雇了五六副扛,整抬运四五日。西门庆也不对吴月娘说,都堆在新盖的玩花楼上。择了八月二十日,一顶大轿,一匹段子红,四对灯笼,派定玳安、平安、画童、来兴四个跟轿,约后晌时分,方娶妇人过门。妇人打发两个丫鬟,教冯妈妈领着先来了,等的回去,方才上轿。把房子交与冯妈妈、天福儿看守。

西门庆那日不往那里去,在家新卷棚内,深衣幅巾坐的,单等妇人进门。妇人轿子落在大门首,半日没个人出去迎接。孟玉楼走来上房,对月娘说:“姐姐,你是家主,如今他已是在门首,你不去迎接迎接儿,惹的他爹不怪?他爹在卷棚内坐着,轿子在门首这一日了,没个人出去,怎么好进来的?”这吴月娘欲待出去接他,心中恼,又不下气;欲待不出去,又怕西门庆性子不是好的。沉吟了半晌,于是轻移莲步,款蹙湘裙,出来迎接。妇人抱着宝瓶,径往他那边新房去了。迎春、绣春两个丫鬟,又早在房中铺陈停当,单等西门庆晚夕进房。不想西门庆正因旧恼在心,不进他房去。到次日,叫他出来后边月娘房里见面,分其大小,排行他是六娘。一般三日摆大酒席,请堂客会亲吃酒,只是不往他房里去。头一日晚夕,先在潘金莲房中。金莲道:“他是个新人儿,才来头一日,你就空了他房?”西门庆道:“你不知淫妇有些眼里火,等我奈何他两日,慢慢的进去。”到了三日,打发堂客散了,西门庆又不进他房中,往后边孟玉楼房里歇去了。这妇人见汉子一连三夜不进他房来,到半夜打发两个丫鬟睡了,饱哭了一场,可怜走到床上,用脚带吊颈悬梁自缢。正是:

\[
连理未谐鸳帐底,冤魂先到九重泉。
\]

两个丫鬟睡了一觉醒来,见灯光昏暗,起来剔灯,猛见床上妇人吊着,吓慌了手脚。忙走出隔壁叫春梅说:“俺娘上吊哩!”慌的金莲起来这边看视,见妇人穿一身大红衣裳,直掇掇吊在床上。连忙和春梅把脚带割断,解救下来。过了半日,吐了一口清涎,方才苏醒。即叫春梅:“后边快请你爹来。”西门庆正在玉楼房中吃酒,还未睡哩。先是玉楼劝西门庆说道:“你娶将他来,一连三日不往他房里去,惹他心中不恼么?恰似俺们把这椿事放在头里一般,头上末下,就让不得这一夜儿。”西门庆道:“待过三日儿我去。你不知道,淫妇有些吃着碗里,看着锅里。想起来你恼不过我。未曾你汉子死了,相交到如今,甚么话儿没告诉我?临了招进蒋太医去!我不如那厮?今日却怎的又寻将我来?”玉楼道:“你恼的是。他也吃人骗了。”正说话间,忽一片声打仪门。玉楼使兰香问,说是春梅来请爹:“六娘在房里上吊哩!”慌的玉楼撺掇西门庆不迭,便道:“我说教你进他房中走走,你不依,只当弄出事来。”于是打着灯笼,走来前边看视。落后吴月娘、李娇儿听见,都起来,到他房中。见金莲搂着他坐的,说道:“五姐,你灌了他些姜汤儿没有?”金莲道:“我救下来时,就灌了些了。”那妇人只顾喉中哽咽了一回,方哭出声。月娘众人一块石头才落地,好好安抚他睡下,各归房歇息。

次日,晌午前后,李瓶儿才吃些粥汤儿。西门庆向李娇儿众人说道:“你们休信那淫妇装死吓人。我手里放不过他。到晚夕等我到房里去,亲看着他上个吊儿我瞧,不然吃我一顿好马鞭子。贼淫妇!不知把我当谁哩!”众人见他这般说,都替李瓶儿捏着把汗。到晚夕,见西门庆袖着马鞭子,进他房去了。玉楼、金莲分付春梅把门关了,不许一个人来,都立在角门首儿外悄悄听着。

且说西门庆见他睡在床上,倒着身子哭泣,见他进去不起身,心中就有几分不悦。先把两个丫头都赶去空房里住了。西门庆走来椅子上坐下,指着妇人骂道:“淫妇!你既然亏心,何消来我家上吊?你跟着那矮忘八过去便了,谁请你来!我又不曾把人坑了,你甚么缘故,流那屄尿怎的?我自来不曾见人上吊,我今日看着你上个吊儿我瞧!”于是拿一条绳子丢在他面前,叫妇人上吊。那妇人想起蒋竹山说西门庆是打老婆的班头,降妇女的领袖,思量我那世里晦气,今日大睁眼又撞入火坑里来了,越发烦恼痛哭起来。这西门庆心中大怒,教他下床来脱了衣裳跪着。妇人只顾延挨不脱,被西门庆拖翻在床地平上,袖中取出鞭子来抽了几鞭子,妇人方才脱去上下衣裳,战兢兢跪在地平上。西门庆坐着,从头至尾问妇人:“我那等对你说,教你略等等儿,我家中有些事儿,如何不依我,慌忙就嫁了蒋太医那厮?你嫁了别人,我倒也不恼!那矮忘八有甚么起解?你把他倒踏进门去,拿本钱与他开铺子,在我眼皮子跟前,要撑我的买卖!”妇人道:“奴不说的悔也是迟了。只因你一去了不见来,朝思暮想,奴想的心斜了。后边乔皇亲花园里常有狐狸,要便半夜三更假名托姓变做你,来摄我精髓,到天明鸡叫就去了。你不信只要问老冯、两个丫头便知。后来看看把奴摄得至死,才请这蒋太医来看。奴就象吊在麴糊盆内一般,吃那厮局骗了。说你家中有事,上东京去了,奴不得已才干下这条路。谁知这厮斫了头是个债椿,被人打上门来,经动官府。奴忍气吞声,丢了几两银子,吃奴即时撵出去了。”西门庆道:“说你叫他写状子,告我收着你许多东西。你如何今日也到我家来了!”妇人道:“你可是没的说。奴那里有这话,就把奴身子烂化了。”西门庆道:“就算有,我也不怕。你说你有钱,快转换汉子,我手里容你不得!我实对你说罢,前者打太医那两个人,是如此这般使的手段。只略施小计,教那厮疾走无门,若稍用机关,也要连你挂了到官,弄倒一个田地。”妇人道:“奴知道是你使的术儿。还是可怜见奴,若弄到那无人烟之处,就是死罢了。”看看说的西门庆怒气消下些来了。又问道:“淫妇你过来,我问你,我比蒋太医那厮谁强?”妇人道:“他拿甚么来比你!你是个天,他是块砖;你在三十三天之上,他在九十九地之下。休说你这等为人上之人,只你每日吃用稀奇之物,他在世几百年还没曾看见哩!他拿甚么来比你!莫要说他,就是花子虚在日,若是比得上你时,奴也不恁般贪你了。你就是医奴的药一般,一经你手,教奴没日没夜只是想你。”自这一句话,把西门庆旧情兜起,欢喜无尽,即丢了鞭子,用手把妇人拉将起来,穿上衣裳,搂在怀里,说道:“我的儿,你说的是。果然这厮他见甚么碟儿天来大!”即叫春梅:“快放桌儿,后边取酒菜儿来!”正是:东边日出西边雨,道是无情却有情。有诗为证:

\[
碧玉破瓜时,郎为情颠倒。
感君不羞赧,回身就郎抱。
\]

\newpage
%# -*- coding:utf-8 -*-
%%%%%%%%%%%%%%%%%%%%%%%%%%%%%%%%%%%%%%%%%%%%%%%%%%%%%%%%%%%%%%%%%%%%%%%%%%%%%%%%%%%%%


\chapter{傻帮闲趋奉闹华筵\KG 痴子弟争锋毁花院}


词曰:

\[
步花径,阑干狭。防人觑,常惊吓。荆刺抓裙钗,倒闪在荼蘼架。勾引嫩枝咿哑,讨归路,寻空罅,被旧家巢燕,引入窗纱。
\]


话说西门庆在房中,被李瓶儿柔情软语,感触的回嗔作喜,拉他起来,穿上衣裳,两个相搂相抱,极尽绸缪。一面令春梅进房放桌儿,往后边取酒去。

且说金莲和玉楼,从西门庆进他房中去,站在角门首窍听消息。他这边又闭着,止春梅一人在院子里伺候。金莲同玉楼两个打门缝儿往里张觑,只见房中掌着灯烛,里边说话,都听不见。金莲道:“俺到不如春梅贼小肉儿,他倒听的伶俐。”那春梅在窗下潜听了一回,又走过来。金莲悄问他房中怎的动静,春梅便隔门告诉与二人说:“俺爹怎的教他脱衣裳跪着,他不脱。爹恼了,抽了他几马鞭子。”金莲道:“打了他,他脱了不曾?”春梅道:“他见爹恼了,才慌了,就脱了衣裳,跪在地平上。爹如今问他话哩。”玉楼恐怕西门庆听见,便道:“五姐,咱过那边去罢。”拉金莲来西角门首。此时是八月二十头,月色才上来。两个站立在黑头里,一处说话,等着春梅出来问他话。潘金莲向玉楼道:“我的姐姐,只说好食果子,一心只要来这里。头儿没过动,下马威早讨了这几下在身上。俺这个好不顺脸的货儿,你若顺顺儿他倒罢了。属扭孤儿糖的,你扭扭儿也是钱,不扭也是钱。想着先前吃小妇奴才压枉造舌,我陪下十二分小心,还吃他奈何得我那等哭哩。姐姐,你来了几时,还不知他性格哩!”

二人正说话之间,只听开的角门响,春梅出来,一直迳往后边走。不防他娘站在黑影处叫他,问道:“小肉儿,那去?”春梅笑着只顾走。金莲道:“怪小肉儿,你过来,我问你话。慌走怎的?”那春梅方才立住了脚,方说:“他哭着对俺爹说了许多话。爹喜欢抱起他来,令他穿上衣裳,教我放了桌儿,如今往后边取酒去。”金莲听了,向玉楼说道:“贼没廉耻的货!头里那等雷声大雨点小,打哩乱哩。及到其间,也不怎么的。我猜,也没的想,管情取了酒来,教他递。贼小肉儿,没他房里丫头?你替他取酒去!到后边,又叫雪娥那小妇奴才屄声浪颡,我又听不上。”春梅道:“爹使我,管我事!”于是笑嘻嘻去了。金莲道:“俺这小肉儿,正经使着他,死了一般懒待动旦。若干猫儿头差事,钻头觅缝干办了要去,去的那快!现他房里两个丫头,你替他走,管你腿事!卖萝葡的跟着盐担子走——好个闲嘈心的小肉儿!”玉楼道:“可不怎的!俺大丫头兰香,我正使他做活儿,他便有要没紧的。爹使他行鬼头儿,听人的话儿,你看他走的那快!”

正说着,只见玉箫自后边蓦地走来,便道:“三娘还在这里?我来接你来了。”玉楼道:“怪狗肉,唬我一跳!”因问:“你娘知道你来不曾?”玉箫道:“我打发娘睡下这一日了,我来前边瞧瞧,刚才看见春梅后边要酒果去了。”因问:“俺爹到他屋里,怎样个动静儿?”金莲接过来伸着手道:“进他屋里去,齐头故事。”玉箫又问玉楼,玉楼便一一对他说。玉箫道:“三娘,真个教他脱了衣裳跪着,打了他五马鞭子来?”玉楼道:“你爹因他不跪,才打他。”玉箫道:“带着衣服打来,去了衣裳打来?亏他那莹白的皮肉儿上怎么挨得?”玉楼笑道:“怪小狗肉儿,你倒替古人耽忧!”正说着,只见春梅拿着酒,小玉拿着方盒,迳往李瓶儿那边去。金莲道:“贼小肉儿,不知怎的,听见干恁勾当儿,云端里老鼠——天生的耗。”分付:“快送了来,教他家丫头伺候去。你不要管他,我要使你哩!”那春梅笑嘻嘻同小玉进去了。一面把酒菜摆在桌上,就出来了,只是绣春、迎春在房答应。玉楼、金莲问了他话。玉箫道:“三娘,咱后边去罢。”二人一路去了。金莲叫春梅关上角门,归进房来,独自宿歇,不在话下。正是:

\[
可惜团圆今夜月,清光咫尺别人圆。
\]

不说金莲独宿,单表西门庆与李瓶儿两个相怜相爱,饮酒说话到半夜,方才被伸翡翠,枕设鸳鸯,上床就寝。灯光掩映,不啻镜中鸾凤和鸣;香气薰笼,好似花间蝴蝶对舞。正是:

\[
今宵胜把银缸照,只恐相逢是梦中。
\]
有词为证:

\[
淡画眉儿斜插梳,不忻拈弄倩工夫。云窗雾阁深深许,蕙性兰心款款呼。相怜爱,倩人扶,神仙标格世间无。从今罢却相思调,美满恩情锦不如。
\]

两个睡到次日饭时。李瓶儿恰待起来临镜梳头,只见迎春后边拿将饭来。妇人先漱了口,陪西门庆吃了半盏儿,又教迎春:“将昨日剩的金华酒筛来。”拿瓯子陪着西门庆,每人吃了两瓯子,方才洗脸梳妆。一面开箱子,打点细软首饰衣服,与西门庆过目。拿出一百颗西洋珠子与西门庆看,原是昔日梁中书家带来之物。又拿出一件金镶鸦青帽顶子,说是过世老公公的。起下来上等子秤,四钱八分重。李瓶儿教西门庆拿与银匠,替他做一对坠子。又拿出一顶金丝鬒髻,重九两。因问西门庆:“上房他大娘众人,有这鬒髻没有?”西门庆道:“他们银丝鬒髻倒有两三顶,只没编这鬒髻。”妇人道:“我不好戴出来的。你替我拿到银匠家毁了,打一件金九凤垫根儿,每个凤嘴衔一溜珠儿,剩下的再替我打一件,照依他大娘正面戴的金镶玉观音满池娇分心。”西门庆收了,一面梳头洗脸,穿了衣服出门。李瓶儿又说道:“那边房里没人,你好歹委付个人儿看守,替了小厮天福儿来家使唤。那老冯老行货子,啻啻磕磕的,独自在那里,我又不放心。”西门庆道:“我知道了。”袖着鬒髻和帽顶子,一直往外走。不妨金莲鬒着头,站在东角门首,叫道:“哥,你往那去?这咱才出来?”西门庆道:“我有勾当去。”妇人道:“怪行货子,慌走怎的?我和你说话。”那西门庆见他叫的紧,只得回来。被妇人引到房中,妇人便坐在椅子上,把他两只手拉着说道:“我不好骂出来的,怪火燎腿三寸货,那个拿长锅镬吃了你!慌往外抢的是些甚的?你过来,我且问你。”西门庆道:“罢么,小淫妇儿,只顾问甚么!我有勾当哩,等我回来说。”说着,往外走。妇人摸见袖子里重重的,道:“是甚么?拿出来我瞧瞧。”西门庆道:“是我的银子包。”妇人不信,伸手进袖子里就掏,掏出一顶金丝鬒髻来,说道:“这是他的鬒髻,你拿那去?”西门庆道:“他问我,知你每没有,说不好戴的,教我到银匠家替他毁了,打两件头面戴。”金莲问道:“这鬒髻多少重?他要打甚么?”西门庆道:“这鬒髻重九两,他要打一件九凤甸儿,一件照依上房娘的正面那一件玉观音满池娇分心。”金莲道:“一件九凤甸儿,满破使了三两五六钱金子勾了。大姐姐那件分心,我秤只重一两六钱,把剩下的,好歹你替我照依他也打一件九凤甸儿。”西门庆道:“满池娇他要揭实枝梗的。”金莲道:“就是揭实枝梗,使了三两金子满顶了。还落他二三两金子,勾打个甸儿了。”西门庆笑骂道:“你这小淫妇儿!单管爱小便宜儿,随处也捏个尖儿。”金莲道:“我儿,娘说的话,你好歹记着。你不替我打将来,我和你答话!”那西门庆袖了鬒髻,笑着出门。金莲戏道:“哥儿,你干上了。”西门庆道:“我怎的干上了?”金莲道:“你既不干上,昨日那等雷声大雨点小,要打着教他上吊。今日拿出一顶鬒髻来,使的你狗油嘴鬼推磨,不怕你不走。”西门庆笑道:“这小淫妇儿,单只管胡说!”说着往外去了。

却说吴月娘和孟玉楼、李娇儿在房中坐的,忽听见外边小厮一片声寻来旺儿,寻不着。只见平安来掀帘子,月娘便问:“寻他做甚么?”平安道:“爹紧等着哩。”月娘半日才说:“我使他有勾当去了。”原来月娘早晨分付下他,往王姑子庵里送香油白米去了。平安道:“小的回爹,只说娘使他有勾当去了。”月娘骂道:“怪奴才,随你怎么回去!”平安慌的不敢言语,往外走了。月娘便向玉楼众人说道:“我开口,又说我多管。不言语,我又憋的慌。一个人也拉剌将来了,那房子卖掉了就是了。平白扯淡,摇铃打鼓的,看守甚么?左右有他家冯妈妈子,再派一个没老婆的小厮,同在那里就是了,怕走了那房子也怎的?巴巴叫来旺两口子去!他媳妇子七病八痛,一时病倒了在那里,谁扶侍他?”玉楼道:“姐姐在上,不该我说。你是个一家之主,不争你与他爹两个不说话,就是俺们不好主张的,下边孩子每也没投奔。他爹这两日隔二骗三的,也甚是没意思。姐姐依俺每一句话儿,与他爹笑开了罢。”月娘道:“孟三姐,你休要起这个意。我又不曾和他两个嚷闹,他平白的使性儿。那怕他使的那脸阁,休想我正眼看他一眼儿!他背地对人骂我不贤良的淫妇,我怎的不贤良?如今耸七八个在屋里,才知道我不贤良!自古道,顺情说好话,干直惹人嫌。我当初说着拦你,也只为好来。你既收了他许多东西,又买他房子,今日又图谋他老婆,就着官儿也看乔了。何况他孝服不满,你不好娶他的。谁知道人在背地里把圈套做的成成的,每日行茶过水,只瞒我一个儿,把我合在缸底下。今日也推在院里歇,明日也推在院里歇,谁想他只当把个人儿歇了家里来,端的好在院里歇!他自吃人在他跟前那等花丽狐哨,乔龙画虎的,两面刀哄他,就是千好万好了。似俺每这等依老实,苦口良言,着他理你理儿!你不理我,我想求你?一日不少我三顿饭,我只当没汉子,守寡在这里。随我去,你每不要管他。”几句话说的玉楼众人讪讪的。

良久,只见李瓶儿梳妆打扮,上穿大红遍地金对襟罗衫儿,翠盖拖泥妆花罗裙,迎春抱着银汤瓶,绣春拿着茶盒,走来上房,与月娘众人递茶。月娘叫小玉安放座儿与他坐。落后孙雪娥也来到,都递了茶,一处坐地。潘金莲嘴快,便叫道:“李大姐,你过来,与大姐姐下个礼儿。实和你说了罢,大姐姐和他爹好些时不说话,都为你来!俺每刚才替你劝了恁一日。你改日安排一席酒儿,央及央及大姐姐,教他两个老公婆笑开了罢。”李瓶儿道:“姐姐分付,奴知道。”于是向月娘面前插烛也似磕了四个头。月娘道:“李大姐,他哄你哩。”又道:“五姐,你每不要来撺掇。我已是赌下誓,就是一百年也不和他在一答儿哩。”以此众人再不敢复言。金莲在旁拿把抿子与李瓶儿抿头,见他头上戴着一副金玲珑草虫儿头面,并金累丝松竹梅岁寒三友梳背儿,因说道:“李大姐,你不该打这碎草虫头面,有些抓头发,不如大姐姐戴的金观音满池娇,是揭实枝梗的好。”这李瓶儿老实,就说道:“奴也照样儿要教银匠打恁一件哩!”落后小玉、玉箫来递茶,都乱戏他。先是玉箫问道:“六娘,你家老公公当初在皇城内那衙门来?”李瓶儿道:“先在惜薪司掌厂。”玉箫笑道:“嗔道你老人家昨日挨得好柴!”小玉又道:“去年许多里长老人,好不寻你,教你往东京去。”妇人不省,说道:“他寻我怎的?”小玉笑道:“他说你老人家会告的好水灾。”玉箫又道:“你老人家乡里妈妈拜千佛,昨日磕头磕勾了。”小玉又说道:“昨日朝廷差四个夜不收,请你往口外和番,端的有这话么?”李瓶儿道:“我不知道。”小玉笑道:“说你老人家会叫的好达达!”把玉楼、金莲笑的不了。月娘骂道:“怪臭肉每,干你那营生去,只顾奚落他怎的?”于是把个李瓶儿羞的脸上一块红、一块白,站又站不得,坐又坐不住,半日回房去了。

良久,西门庆进房来,回他雇银匠家打造生活。就计较发柬,二十五日请官客吃会亲酒,少不的请请花大哥。李瓶儿道:“他娘子三日来,再三说了。也罢,你请他请罢。”李瓶儿又说:“那边房子左右有老冯看守,你这里再教一个和天福儿轮着上宿就是,不消叫旺官去罢。上房姐姐说,他媳妇儿有病,去不的。”西门庆道:“我不知道。”即叫平安,分付:“你和天福儿两个轮,一递一日,狮子街房子里上宿。”不在言表。

不觉到二十五日,西门庆家中吃会亲酒,安排插花筵席,一起杂耍步戏。四个唱的,李桂姐、吴银儿、董玉仙、韩金钏儿,从晌午就来了。官客在卷棚内吃了茶,等到齐了,然后大厅上坐席。头一席花大舅、吴大舅;第二席吴二舅、沈姨夫;第三席应伯爵、谢希大;第四席祝实念、孙天化;第五席常峙节、吴典恩;第六席云里守、白赉光。西门庆主位,其余傅自新、贲第传、女婿陈敬济两边列坐。乐人撮弄杂耍数回,就是笑乐院本。下去,李铭、吴惠两个小优上来弹唱,间着清吹。下去,四个唱的出来,筵外递酒。应伯爵在席上先开言说道:“今日哥的喜酒,是兄弟不当斗胆,请新嫂子出来拜见拜见,足见亲厚之情。俺每不打紧,花大尊亲,并二位老舅、沈姨丈在上,今日为何来?”西门庆道:“小妾丑陋,不堪拜见,免了罢。”谢希大道:“哥,这话难说。当初有言在先,不为嫂子,俺每怎么儿来?何况见有我尊亲花大哥在上,先做友,后做亲,又不同别人。请出来见见怕怎的?”西门庆笑不动身。应伯爵道:“哥,你不要笑,俺每都拿着拜见钱在这里,不白教他出来见。”西门庆道:“你这狗才,单管胡说。”吃他再三逼迫不过,叫过玳安来,教他后边说去。半日,玳安出来回说:“六娘道,免了罢。”应伯爵道:“就是你这小狗骨秃儿的鬼!你几时往后边去,就来哄我?”玳安道:“小的莫不哄应二爹!二爹进去问不是?”伯爵道:“你量我不敢进去?左右花园中熟径,好不好我走进去,连你那几位娘都拉了出来。”玳安道:“俺家那大猱狮狗,好不利害。倒没有把应二爹下半截撕下来。”伯爵故意下席,赶着玳安踢两脚,笑道:“好小狗骨秃儿,你伤的我好!趁早与我后边请去。请不将来,打二十栏杆。”把众人、四个唱的都笑了。玳安走到下边立着,把眼只看着他爹不动身。西门庆无法可处,只得叫过玳安近前,分付:“对你六娘说,收拾了出来见见罢。”那玳安去了半日出来,复请了西门庆进去。然后才把脚下人赶出去,关上仪门。孟玉楼、潘金莲百方撺掇,替他抿头,戴花翠,打发他出来。厅上铺下锦毡绣毯,四个唱的,都到后边弹乐器,导引前行。麝兰叆叇,丝竹和鸣。妇人身穿大红五彩通袖罗袍,下着金枝线叶沙绿百花裙,腰里束着碧玉女带,腕上笼着金压袖。胸前缨落缤纷,裙边环佩叮当,头上珠翠堆盈,鬓畔宝钗半卸,粉面宜贴翠花钿,湘裙越显红鸳小。正是:

\[
恍似姮嫦离月殿,犹如神女到筵前。
\]

当下四个唱的,琵琶筝弦,簇拥妇人,花枝招展,绣带飘摇,望上朝拜。慌的众人都下席来,还礼不迭。却说孟玉楼、潘金莲、李娇儿簇拥着月娘都在大厅软壁后听觑,听见唱“喜得功名遂”,唱到“天之配合一对儿,如鸾似凤”,直至“永团圆,世世夫妻”。金莲向月娘说道:“大姐姐,你听唱的!小老婆今日不该唱这一套,他做了一对鱼水团圆,世世夫妻,把姐姐放到那里?”那月娘虽故好性儿,听了这两句,未免有几分恼在心头。又见应伯爵、谢希大这伙人,见李瓶儿出来上拜,恨不得生出几个口来夸奖奉承,说道:“我这嫂子,端的寰中少有,盖世无双!休说德性温良,举止沉重,自这一表人物,普天之下,也寻不出来。那里有哥这样大福?俺每今日得见嫂子一面,明日死也得好处。”因唤玳安儿:“快请你娘回房里,只怕劳动着,倒值了多的。”吴月娘众人听了,骂扯淡轻嘴的囚根子不绝。良久,李瓶儿下来。四个唱的见他手里有钱,都乱趋奉着他,娘长娘短,替他拾花翠,叠衣裳,无所不至。

月娘归房,甚是不乐。只见玳安、平安接了许多拜钱,也有尺头、衣服并人情礼,盒子盛着,拿到月娘房里。月娘正眼也不看,骂道:“贼囚根子!拿送到前头就是了,平白拿到我房里来做甚么?”玳安道:“爹分付拿到娘房里来。”月娘叫玉箫接了,掠在床上去。不一时,吴大舅吃了第二道汤饭,走进后边来见月娘。月娘见他哥进房来,连忙与他哥哥行礼毕,坐下。吴大舅道:“昨日你嫂子在这里打搅,又多谢姐夫送了桌面去。到家对我说,你与姐夫两下不说话。我执着要来劝你,不想姐夫今日又请。姐姐,你若这等,把你从前一场好都没了。自古痴人畏妇,贤女畏夫。三从四德,乃妇道之常。今后他行的事,你休要拦他,料姐夫他也不肯差了。落的做好好先生,才显出你贤德来。”月娘道:“早贤德好来,不教人这般憎嫌。他有了他富贵的姐姐,把我这穷官儿家丫头,只当忘故了的算帐。你也不要管他,左右是我,随他把我怎么的罢!贼强人,从几时这等变心来?”说着,月娘就哭了。吴大舅道:“姐姐,你这个就差了。你我不是那等人家,快休如此。你两口儿好好的,俺每走来也有光辉些!”劝月娘一回。小玉拿茶来。吃毕茶,只见前边使小厮来请,吴大舅便作辞月娘出来。当下众人吃至掌灯以后,就起身散了。四个唱的,李瓶儿每人都是一方销金汗巾儿,五钱银子,欢喜回家。自此西门庆连在瓶儿房里歇了数夜。别人都罢了,只有潘金莲恼的要不的,背地唆调吴月娘与李瓶儿合气。对着李瓶儿,又说月娘容不的人。李瓶儿尚不知堕他计中,每以姐姐呼之,与他亲厚尤密。正是:

\[
逢人且说三分话,未可全抛一片心。
\]

西门庆自娶李瓶儿过门,又兼得了两三场横财,家道营盛,外庄内宅,焕然一新。米麦陈仓,骡马成群,奴仆成行。把李瓶儿带来小厮天福儿,改名琴童。又买了两个小厮,一名来安儿,一名棋童儿。把金莲房中春梅、上房玉箫、李瓶儿房中迎春、玉楼房中兰香,一般儿四个丫头,衣服首饰妆束起来,在前厅西厢房,教李娇儿兄弟乐工李铭来家,教演习学弹唱。春梅琵琶,玉箫学筝,迎春学弦子,兰香学胡琴。每日三茶六饭,管待李铭,一月与他五两银子。又打开门面两间,兑出二千两银子来,委傅伙计、贲第传开解当铺。女婿陈敬济只掌钥匙,出入寻讨。贲第传只写帐目,秤发货物。傅伙计便督理生药、解当两个铺子,看银色,做买卖。潘金莲这边楼上,堆放生药。李瓶儿那边楼上,厢成架子,搁解当库衣服、首饰、古董、书画、玩好之物。一日也当许多银子出门。

陈敬济每日起早睡迟,带着钥匙,同伙计查点出入银钱,收放写算皆精。西门庆见了,喜欢的要不的。一日在前厅与他同桌儿吃饭,说道:“姐夫,你在我家这等会做买卖,就是你父亲在东京知道,他也心安,我也得托了。常言道:有儿靠儿,无儿靠婿。我若久后没出,这分儿家当,都是你两口儿的。”那敬济说道:“儿子不幸,家遭官事,父母远离,投在爹娘这里。蒙爹娘抬举,莫大之恩,生死难报。只是儿子年幼,不知好歹,望爹娘耽待便了,岂敢非望。”西门庆听见他说话儿聪明乖觉,越发满心欢喜。但凡家中大小事务、出入书柬、礼帖,都教他写。但凡客人到,必请他席侧相陪。吃茶吃饭,一时也少不的他。谁知道这小伙儿绵里之针,肉里之刺。

\[
常向绣帘窥贾玉,每从绮阁窃韩香。
\]

光阴似箭,不觉又是十一月下旬。西门庆在常峙节家会茶散的早,未掌灯就起身,同应伯爵、谢希大、祝实念三个并马而行。刚出了门,只见天上彤云密布,又早纷纷扬扬飘下一天雪花来。应伯爵便道:“哥,咱这时候就家去,家里也不收。我每许久不曾进里边看看桂姐,今日趁着落雪,只当孟浩然踏雪寻梅,望他望去。”祝实念道:“应二哥说的是。你每月风雨不阻,出二十银子包钱包着他,你不去,落的他自在。”西门庆吃三人你一言我一句,说的把马迳往东街勾栏来了。来到李桂姐家,已是天气将晚。只见客位里掌着灯,丫头正扫地。老妈并李桂卿出来,见礼毕,上面列四张交椅,四人坐下。老虔婆便道:“前者桂姐在宅里来晚了,多有打搅。又多谢六娘,赏汗巾花翠。”西门庆道:“那日空过他。我恐怕晚了他们,客人散了,就打发他来了。”说着,虔婆一面看茶吃了,丫鬟就安放桌儿,设放案酒。西门庆道:“怎么桂姐不见?”虔婆道:“桂姐连日在家伺候姐夫,不见姐夫来。今日是他五姨妈生日,拿轿子接了与他五姨妈做生日去了。”原来李桂姐也不曾往五姨家做生日去。近日见西门庆不来,又接了杭州贩绸绢的丁相公儿子丁二官人,号丁双桥,贩了千两银子绸绢,在客店里,瞒着他父亲来院中嫖。头上拿十两银子、两套杭州重绢衣服请李桂姐,一连歇了两夜。适才正和桂姐在房中吃酒,不想西门庆到。老虔婆忙教桂姐陪他到后边第三层一间僻静小房坐去了。当下西门庆听信虔婆之言,便道:“既是桂姐不在,老妈快看酒来,俺每慢慢等他。”这老虔婆在下面一力撺掇,酒肴蔬菜齐上,须臾,堆满桌席。李桂卿不免筝排雁柱,歌按新腔,众人席上猜枚行令。正饮时,不妨西门庆往后边更衣去。也是合当有事,忽听东耳房有人笑声。西门庆更毕衣,走至窗下偷眼观觑,正见李桂姐在房内陪着一个戴方巾的蛮子饮酒。由不的心头火起,走到前边,一手把吃酒桌子掀翻,碟儿盏儿打的粉碎。喝令跟马的平安、玳安、画童、琴童四个小厮上来,把李家门窗户壁床帐都打碎了。应伯爵、谢希大、祝实念向前拉劝不住。西门庆口口声声只要采出蛮囚来,和粉头一条绳子墩锁在门房内。那丁二官又是个小胆之人,见外边嚷斗起来,慌的藏在里间床底下,只叫:“桂姐救命!”桂姐道:“呸!好不好,还有妈哩!这是俺院中人家常有的,不妨事,随他发作叫嚷,你只休要出来。”老虔婆见西门庆打的不象模样,还要架桥儿说谎,上前分辨。西门庆那里还听他,只是气狠狠呼喝小厮乱打,险些不曾把李老妈打起来。多亏了应伯爵、谢希大、祝实念三人死劝,活喇喇拉开了手。西门庆大闹了一场,赌誓再不踏他门来,大雪里上马回家。正是:

\[
宿尽闲花万万千,不如归家伴妻眠。
虽然枕上无情趣,睡到天明不要钱。
\]

\newpage
%# -*- coding:utf-8 -*-
%%%%%%%%%%%%%%%%%%%%%%%%%%%%%%%%%%%%%%%%%%%%%%%%%%%%%%%%%%%%%%%%%%%%%%%%%%%%%%%%%%%%%


\chapter{吴月娘扫雪烹茶\KG 应伯爵替花邀酒}


词曰:

\[
并刀如水,吴盐胜雪,纤手破新橙。锦幄初温,兽烟不断,相对坐调笙。低声问向谁行宿,城上已三更。马滑霜浓,不如休去,直至少人行。
\]

话说西门庆从院中归家,已一更天气,到家门首,小厮叫开门,下了马,踏着那乱琼碎玉,到于后边仪门首。只仪门半掩半开,院内悄无人声。西门庆心内暗道:“此必有跷蹊。”于是潜身立于仪门内粉壁前,悄悄听觑。只见小玉出来,穿廊下放桌儿。原来吴月娘自从西门庆与他反目以来,每月吃斋三次,逢七拜斗焚香,保佑夫主早早回心,西门庆还不知。只见小玉放毕香桌儿。少顷,月娘整衣出来,向天井内满炉炷香,望空深深礼拜。祝曰:“妾身吴氏,作配西门。奈因夫主留恋烟花,中年无子。妾等妻妾六人,俱无所出,缺少坟前拜扫之人。妾夙夜忧心,恐无所托。是以发心,每夜于星月之下,祝赞三光,要祈佑儿夫,早早回心。弃却繁华,齐心家事。不拘妾等六人之中,早见嗣息,以为终身之计,乃妾之素愿也。”正是:

\[
私出房栊夜气清,一庭香雾雪微明。
拜天诉尽衷肠事,无限徘徊独自惺。
\]

这西门庆不听便罢,听了月娘这一篇言语,不觉满心惭感道:“原来我一向错恼了他。他一篇都是为我的心,还是正经夫妻。”忍不住从粉壁前叉步走来,抱住月娘。月娘不防是他大雪里来到,吓了一跳,就要推开往屋里走,被西门庆双关抱住,说道:“我的姐姐!我西门庆死也不晓的,你一片好心,都是为我的。一向错见了,丢冷了你的心,到今悔之晚矣。”月娘道:“大雪里,你错走了门儿了,敢不是这屋里。我是那不贤良的淫妇,和你有甚情节?那讨为你的来?你平白又来理我怎的?咱两个永世千年休要见面!”西门庆把月娘一手拖进房来。灯前看见他家常穿着:大红路绸对衿袄儿,软黄裙子;头上戴着貂鼠卧兔儿,金满池娇分心,越显出他:

\[
粉妆玉琢银盆脸,蝉髻鸦鬟楚岫云。
\]

那西门庆如何不爱?连忙与月娘深深作了个揖,说道:“我西门庆一时昏昧,不听你之良言,辜负你之好意。正是有眼不识荆山玉,拿着顽石一样看。过后方知君子,千万饶恕我则个。”月娘道:“我又不是你那心上的人儿,凡是投不着你的机会,有甚良言劝你?随我在这屋里自生自活,你休要理他。我这屋里也难安放你,趁早与我出去,我不着丫头撵你。”西门庆道:“我今日平白惹一肚子气,大雪里来家,迳来告诉你。”月娘道:“惹气不惹气,休对我说。我不管你,望着管你的人去说。”西门庆见月娘脸儿不瞧,就折叠腿装矮子,跪在地下,杀鸡扯脖,口里姐姐长,姐姐短。月娘看不上,说道:“你真个恁涎脸涎皮的!我叫丫头进来。”一面叫小玉。那西门庆见小玉进来,连忙立起来,无计支出他去,说道:“外边下雪了,一张香桌儿还不收进来?”小玉道:“香桌儿头里已收进来了。”月娘忍不住笑道:“没羞的货,丫头跟前也调个谎儿。”小玉出去,那西门庆又跪下央及。月娘道:“不看世人面上,一百年不理才好。”说毕,方才和他坐在一处,教玉箫捧茶与他吃。西门庆因他今日常家茶会,散后同邀伯爵到李家如何嚷闹,告诉一遍:“如今赌了誓,再不踏院门了。”月娘道:“你踹不踹,不在于我。你拿响金白银包着他,你不去,可知他另接了别个汉子?养汉老婆的营生,你拴住他身,拴不住他心。你长拿封皮封着他也怎的?”西门庆道:“你说的是。”于是打发丫鬟出去,脱衣上床,要与月娘求欢。月娘道:“教你上炕就捞食儿吃,今日只容你在我床上就勾了,要思想别的事,却不能勾。”西门庆把那话露将出来,向月娘戏道:“都是你气的他,中风不语了。大睁着眼儿,说不出话来。”月娘骂道:“好个汗邪的货,教我有半个眼儿看的上!”西门庆不由分说,把月娘两只白生生腿扛在肩膀上,那话插入牝中,一任其莺恣蝶采,殢雨尤云,未肯即休。正是得多少——

\[
海棠枝上莺梭急,翡翠梁间燕语频。
\]
不觉到灵犀一点,美爱无加,麝兰半吐,脂香满唇。西门庆情极,低声求月娘叫达达;月娘亦低声睥帏睨枕,态有余妍,口呼亲亲不绝。是夜,两人雨意云情,并头交颈而睡。正是:

\[
乱鬓双横兴已饶,情浓犹复厌通宵。
晚来独向妆台立,淡淡春山不用描。
\]

当夜夫妻交欢不题。却表次日清晨,孟玉楼走到潘金莲房中,未曾进门,先叫道:“六丫头,起来了不曾?”春梅道:“俺娘才起来梳头哩。三娘进屋里坐。”玉楼进来,只见金莲正在梳台前整掠香云。因说道:“我有椿事儿来告诉你,你知道不知?”金莲道:“我在这背哈喇子,谁晓的!”因问:“甚么事?”玉楼道:“他爹昨夜二更来家,走到上房里,和吴家的好了,在他房里歇了一夜。”金莲道:“俺们何等劝着,他说一百年二百年,又怎的平白浪着,自家又好了?又没人劝他!”玉楼道:“今早我才知道。俺大丫头兰香,在厨房内听见小厮们说,昨日他爹同应二在院里李桂儿家吃酒,看出淫妇的甚么破绽,把淫妇门窗户壁都打了。大雪里着恼来家,进仪门,看见上房烧夜香,想必听见些甚么话儿,两个才到一搭哩。碜死了。相他这等就没的话说。若是别人,又不知怎的说浪!”金莲接说道:“早是与人家做大老婆,还不知怎样久惯牢成!一个烧夜香,只该默默祷祝,谁家一径倡扬,使汉子知道了。又没人劝,自家暗里又和汉子好了。硬到底才好,干净假撇清!”玉楼道:“也不是假撇清,他有心也要和,只是不好说出来的。他说他是大老婆不下气,到叫俺们做分上,怕俺们久后玷言玷语说他,敢说你两口子话差,也亏俺们说和。如今你我休教他买了乖儿去。你快梳了头,过去和李瓶儿说去。咱两个每人出五钱银子,叫李瓶儿拿出一两来,原为他的事起。今日安排一席酒,一者与他两个把一杯,二者当家儿只当赏雪,耍戏一日,有何不可?”金莲道:“说的是。不知他爹今日有勾当没有?”玉楼道:“大雪里有甚勾当?我来时两口子还不见动静,上房门儿才开,小玉拿水进去了。”这金莲慌忙梳毕头,和玉楼同过李瓶儿这边来。李瓶儿还睡着在床上,迎春说:“三娘、五娘来了。”玉楼、金莲进来,说道:“李大姐,好自在。这咱时懒龙才伸腰儿。”金莲说舒进手去被窝里,摸见薰被的银香球儿,道:“李大姐生了蛋了。”就掀开被,见他一身白肉。那李瓶儿连忙穿衣不迭。玉楼道:“五姐,休鬼混他。李大姐,你快起来,俺们有椿事来对你说。如此这般,他爹昨日和大姐姐好了,咱每人五钱银子,你便多出些儿,当初因为你起来。今日大雪里,只当赏雪,咱安排一席酒儿,请他爹和大姐姐坐坐儿,好不好?”李瓶儿道:“随姐姐教我出多少,奴出便了。”金莲道:“你将就只出一两儿罢。你秤出来,俺好往后边问李娇儿、孙雪娥要去。”这李瓶儿一面穿衣缠脚,叫迎春开箱子,拿出银子。拿了一块,金莲上等子秤,重一两二钱五分。玉楼叫金莲伴着李瓶儿梳头:“等我往后边问李娇儿和孙雪娥要银子去。”金莲看着李瓶儿梳头洗面,约一个时辰,只见玉楼从后边来说道:“我早知也不干这营生。大家的事,相白要他的。小淫妇说:‘我是没时运的人,汉子再不进我房里来,我那讨银子?’求了半日,只拿出这根银簪子来,你秤秤重多少?”金莲取过等子来秤,只重三钱七分。因问:“李娇儿怎的?”玉楼道:“李娇儿初时只说没有,‘虽是钱日逐打我手里使,都是叩数的。使多少交多少,那里有富余钱?’我说:‘你当家还说没钱,俺们那个是有的?六月日头,没打你门前过也怎的?大家的事,你不出罢!’教我使性子走了出来,他慌了,使丫头叫我回去,才拿出这银子与我。没来由,教我恁惹气剌剌的!”金莲拿过李娇儿银子来秤了秤,只四钱八分。因骂道:“好个奸滑的淫妇!随问怎的,绑着鬼也不与人家足数,好歹短几分。”玉楼道:“只许他家拿黄捍等子秤人的。人问他要,只相打骨秃出来一般,不知教人骂了多少!”一面连玉楼、金莲共凑了三两一钱;一面使绣春叫了玳安来。金莲先问他:“你昨日跟了你爹去,在李家为什么着了恼来?”玳安悉把在常家会茶散的早,邀应二爹和谢爹同到李家,他鸨子回说不在家,往五姨妈家做生日去了。“不想落后爹净手,到后边亲看见粉头和一个蛮子吃酒,爹就恼了。不由分说,叫俺众人把淫妇性骑马回家,在路上发狠,到明日还要摆布淫妇哩。”金莲道:“贼淫妇!我只道蜜罐儿长年拿的牢牢的,如何今日也打了?”又问玳安:“你爹真个恁说来?”玳安道:“莫是小的敢哄娘!”金莲道:“贼囚根子,他不揪不采,也是你爹的婊子,许你骂他?想着迎头儿我们使着你,只推不得闲,‘爹使我往桂姨家送银子去哩!’叫的桂姨那甜!如今他败落了来,你主子恼了,连你也叫他淫妇来了!看我明日对你爹说不说。”玳安道:“耶乐!五娘这回日头打西出来,从新又护起他家来了!莫不爹不在路上骂他淫妇,小的敢骂他?”金莲道:“许你爹骂他罢了,原来也许你骂他?”玳安道:“早知五娘麻犯小的,小的也不对五娘说。”玉楼便道:“小囚儿,你别要说嘴。这里三两一钱银子,你快和来兴儿替我买东西去。今日俺们请你爹和大娘赏雪。你将就少落我们些儿,我教你五娘不告你爹说罢。”玳安道:“娘使小的,小的敢落钱?”于是拿了银子同来兴儿买东西去了。

且说西门庆起来,正在上房梳洗。只见大雪里,来兴买了鸡鹅嗄饭,迳往厨房里去了。玳安又提了一坛金华酒进来。便问玉箫:“小厮的东西,是那里的?”玉箫回道:“今日众娘置酒,请爹娘赏雪。”西门庆道:“金华酒是那里的?”玳安道:“是三娘与小的银子买的。”西门庆道:“啊呀!家里见放着酒,又去买!”分付玳安:“拿钥匙,前边厢房有双料茉莉酒,提两坛搀着这酒吃。”于是在后厅明间内,设锦帐围屏,放下梅花暖帘,炉安兽炭,摆列酒席。不一时,整理停当。李娇儿、孟玉楼、潘金莲、李瓶儿来到,请西门庆、月娘出来。当下李娇儿把盏,孟玉楼执壶,潘金莲捧菜,李瓶儿陪跪,头一锺先递了与西门庆。西门庆接酒在手,笑道:“我儿,多有起动,孝顺我老人家常礼儿罢!”那潘金莲嘴快,插口道:“好老气的孩儿!谁这里替你磕头哩?俺们磕着你,你站着。羊角葱靠南墙——越发老辣!若不是大姐姐带携你,俺们今日与你磕头?”一面递了西门庆,从新又满满斟了一盏,请月娘转上,递与月娘。月娘道:“你们也不和我说,谁知你们平白又费这个心。”玉楼笑道:“没甚么。俺们胡乱置了杯水酒儿,大雪,与你老公婆两个散闷而已。姐姐请坐,受俺们一礼儿。”月娘不肯,亦平还下礼去。玉楼道:“姐姐不坐,我们也不起来。”相让了半日,月娘才受了半礼。金莲戏道:“对姐姐说过,今日姐姐有俺们面上,宽恕了他。下次再无礼,冲撞了姐姐,俺们也不管了。”望西门庆说道:“你装憨打势,还在上首坐,还不快下来,与姐姐递个锺儿,陪不是哩!”西门庆又是笑。良久,递毕,月娘转下来,令玉箫执壶,亦斟酒与众姊妹回酒。惟孙雪娥跪着接酒,其余都平叙姊妹之情。

于是西门庆与月娘居上座,其余李娇儿、孟玉楼、潘金莲、李瓶儿、孙雪娥并西门大姐,都两边打横。金莲便道:“李大姐,你也该梯己与大姐姐递杯酒儿,当初因为你的事起来,你做了老林,怎么还恁木木的!”那李瓶儿真个就就走下席来要递酒。被西门庆拦住,说道:“你休听那小淫妇儿,他哄你。已是递过一遍酒罢了,递几遍儿?”那李瓶儿方不动了。当下春梅、迎春、玉箫、兰香一般儿四个家乐,琵琶、筝、弦子、月琴,一面弹唱起来,唱了一套《南石榴花》“佳期重会”。西门庆听了,便问:“谁叫他唱这一套词来?”玉箫道:“是五娘分咐唱来。”西门庆就看着潘金莲说道:“你这小淫妇,单管胡枝扯叶的!”金莲道:“谁教他唱他来?没的又来缠我。”月娘便道:“怎的不请陈姐夫来坐坐?”一面使小厮前边请去。不一时,敬济来到,向席上都作了揖,就在大姐下边坐了。月娘令小玉安放了锺箸,合家欢饮。西门庆把眼观看帘前那雪,如撏绵扯絮,乱舞梨花,下的大了。端的好雪。但见:

初如柳絮,渐似鹅毛。唰唰似数蟹行沙上,纷纷如乱琼堆砌间。但行动衣沾六出,只顷刻拂满蜂鬓。衬瑶台,似玉龙翻甲绕空舞;飘粉额,如白鹤羽毛连地落。正是:冻合玉楼寒起粟,光摇银海烛生花。

吴月娘见雪下在粉壁间太湖石上甚厚。下席来,教小玉拿着茶罐,亲自扫雪,烹江南凤团雀舌牙茶与众人吃。正是:

\[
白玉壶中翻碧浪,紫金杯内喷清香。
\]

正吃茶中间,只见玳安进来,说道:“李铭来了,在前边伺候。”西门庆道:“教他进来。”不一时,李铭进来向众人磕了头,走在旁边。西门庆问道:“你往那里去来?来得正好。”李铭道:“小的没往那里去,北边酒醋门刘公公那里,教了些孩子,小的瞧了瞧。记挂着爹娘内姐儿们,还有几段唱未合拍,来伺候。”西门庆就将手内吃的那一盏木樨茶,递与他吃。说道:“你吃了休去,且唱一个我听。”李铭道:“小的知道。”一面下边吃了茶上来,把筝弦调定,顿开喉音,并足朝上,唱了一套《冬景·绛都春》。唱毕,西门庆令李铭近前,赏酒与他吃,教小玉拿壶满斟,倾在银珐琅桃儿锺内。那李铭跪在地下,满饮三杯。西门庆又叫在桌上拿了四碟菜,用盘子托着与李铭。那李铭走到下边吃了,用绢儿把嘴抹了,走到上边,直竖竖的靠着槅子站立。西门庆因把昨日桂姐家之事,告诉一遍。李铭道:“小的并不知道,一向也不过那边去。想起来不干桂姐事,都是俺三妈干的营生。爹也别要恼他,等小的见他说他便了。”当日饮酒到一更时分,妻妾俱各欢乐。先是陈敬济、大姐往前边去了。落后酒阑,西门庆又赏李铭酒,打发出门,分咐:“你到那边,休说今日在我这里。”李铭道:“爹分咐,小的知道。”西门庆令左右送他出门,于是妻妾各散。西门庆还在月娘上房歇了。有诗为证:

\[
赤绳缘分莫疑猜,扊扅夫妻共此怀。
鱼水相逢从此始,两情愿保百年谐。
\]

却说次日雪晴,应伯爵、谢希大受了李家烧鹅瓶酒,恐怕西门庆摆布他家,迳来邀请西门庆进里边陪礼。月娘早晨梳妆毕,正和西门庆在房中吃饼,只见玳安来说:“应二爹和谢爹来了。”西门庆放下饼,就要往前走。月娘道:“两个勾使鬼,又不知来做甚么。你亦发吃了出去,教他外头等着去。慌的恁没命的一般往外走怎的?大雪里又不知勾了那去?”西门庆道:“你叫小厮把饼拿到前边,我和他两个吃罢。”说着,起身往外来。月娘分咐:“你和他吃了,别要信着又勾引的往那里去了。今日孟三姐晚夕上寿哩。”西门庆道:“我知道。”于是与应、谢二人相见声喏,说道:“哥昨日着恼家来了,俺们甚是怪说他家:‘从前已往,在你家使钱费物,虽故一时不来,休要改了腔儿才好,许你家粉头背地偷接蛮子?冤家路儿窄,又被他亲眼看见,他怎的不恼!休说哥恼,俺们心里也看不过!’尽力说了他娘儿几句,他也甚是没意思。今日早请了俺两个到家,娘儿们哭哭啼啼跪着,恐怕你动意,置了一杯水酒儿,好歹请你进去陪个不是。”西门庆道:“我也不动意。我再也不进去了。”伯爵道:“哥恼有理。但说起来,也不干桂姐事。这个丁二官原先是他姐姐桂卿的孤老,也没说要请桂姐。只因他父亲货船搭在他乡里陈监生船上,才到了不多两日。这陈监生号两淮,乃是陈参政的儿子。丁二官拿了十两银子,在他家摆酒请陈监生。才送这银子来,不想你我到了他家,就慌了,躲不及,把个蛮子藏在后边,被你看见了。实告不曾和桂姐沾身。今日他娘儿们赌身发咒,磕头礼拜,央俺二人好歹请哥到那里,把这委屈情由也对哥表出,也把恼解了一半。”西门庆道:“我已是对房下赌誓,再也不去,又恼甚么?你上覆他家,到不消费心。我家中今日有些小事,委的不得去。”慌的二人一齐跪下,说道:“哥,甚么话!不争你不去,显的我们请不得哥去,没些面情了。到那里略坐坐儿就来也罢。”当下二人死告活央,说的西门庆肯了。不一时,放桌儿,留二人吃饼。须臾吃毕,令玳安取衣服去。月娘正和孟玉楼坐着,便问玳安:“你爹要往那去?”玳安道:“小的不知,爹只叫小的取衣服。”月娘骂道:“贼囚根子,你还瞒着我不说!今日你三娘上寿哩。你爹但来晚了,我只打你这个贼囚根子。”玳安道:“娘打小的,管小的甚事?”月娘道:“不知怎的,听见他这老子每来,恰似奔命的一般,吃着饭,丢下饭碗,往外不迭。又不知勾引游魂撞尸,撞到多咱才来!”家中置酒等候不题。

且说西门庆被两个邀请到李家,又早堂中置了一席齐整酒肴,叫了两个妓女弹唱。李桂姐与桂卿两个打扮迎接。老虔婆出来,跪着陪礼。姐儿两个递酒。应伯爵、谢希大在旁打诨耍笑,向桂姐道:“还亏我把嘴头上皮也磨了半边去,请了你家汉子来。就连酒儿也不替我递一杯儿,只递你家汉子!刚才若他撅了不来,休说你哭瞎了你眼,唱门词儿,到明日诸人不要你,只我好说话儿将就罢了。”桂姐骂道:“怪应花子,汗邪了你!我不好骂出来的。可可儿的我唱门词儿来?”应伯爵道:“你看贼小淫妇儿!念了经打和尚,他不来慌的那腔儿,这回就翅膀毛儿干了。你过来,且与我个嘴温温寒着。”于是不由分说,搂过脖子来就亲了个嘴。桂姐笑道:“怪攮刀子的,看推撒了酒在爹身上。”伯爵道:“小淫妇儿,会乔张致的,这回就疼汉子。‘看撒了爹身上酒!’叫你爹那甜。我是后娘养的?怎的不叫我一声儿?”桂姐道:“我叫你是我的孩儿。”伯爵道:“你过来,我说个笑话儿你听:一个螃蟹与田鸡结为兄弟,赌跳过水沟儿去便是大哥。田鸡几跳,跳过去了。螃蟹方欲跳,撞遇两个女子来汲水,用草绳儿把他拴住,打了水带回家去。临行忘记了,不将去。田鸡见他不来,过来看他,说道:‘你怎的就不过去了?’螃蟹说:‘我过的去,倒不吃两个小淫妇捩的恁样了!’”桂姐两个听了,一齐赶着打,把西门庆笑的要不的。

不说这里调笑顽耍,且说家中吴月娘一者置酒回席,二者又是玉楼上寿,吴大妗子、杨姑娘并两个姑子,都在上房里坐的。看看等到日落时分,不见西门庆来家,急的月娘要不的。金莲拉着李瓶儿,笑嘻嘻向月娘说道:“大姐姐,他这咱不来,俺们往门首瞧他瞧去。”月娘道:“耐烦瞧他怎的!”金莲又拉玉楼说:“咱三个打伙儿走走去。”玉楼道:“我这里听大师父说笑话儿哩,等听说了笑话儿咱去。”那金莲方住了脚,围着两个姑子听说笑话儿,因说道:“大师父,你有,快些说。”那王姑子坐在坑上,就说了一个。金莲道:“这个不好。再说一个。”王姑子又道:“一家三个媳妇儿,与公公上寿。先是大媳妇递酒说:‘公公好象一员官。’公公云:‘我如何象官?’媳妇云:‘坐在上面,家中大小都怕你,如何不象官?’次该二媳妇上来递酒,说:‘公公象虎威皂隶。’公公曰:‘我如何象虎威皂隶?’媳妇云:‘你喝一声,家中大小都吃一惊,怎不象皂隶?’公公道:‘你说的我好!’该第三媳妇递酒,上来说:‘公公也不象官,也不象皂隶。’公公道:‘却象甚么?’媳妇道:‘公公象个外郎!’公公道:‘我如何象个外郎?’媳妇道:‘不象外郎,如何六房里都串到?’”把众人都笑了。金莲道:“好秃子!把俺们都说在里头。那个外郎敢恁大胆!”说罢,金莲、玉楼、李瓶儿同来到前边大门首,瞧西门庆。玉楼问道:“今日他爹大雪里那里去了?”金莲道:“我猜他一定往院中李桂儿那淫妇家去了。”玉楼道:“打了一场,赌誓再不去,如何又去?咱每赌甚么?管情不在他家。”金莲道:“李大姐做证见,你敢和我拍手么?我说今日往他家去了。前日打了淫妇家,昨日李铭那忘八先来打探子儿。今日应二和姓谢的,大清早晨,勾使鬼勾了他去。我猜老虔婆和淫妇铺谋定计叫了去,不知怎的撮弄,陪着不是,还要回炉复帐,不知涎缠到多咱时候。有个来的成来不成,大姐姐还只顾等着他!”玉楼道:“就不来,小厮也该来家回一声儿。”正说着,只见卖瓜子的过来,两个正在门首买瓜子儿,忽然西门庆从东来了,三个往后跑不迭。

西门庆在马上,教玳安先头里走:“你瞧是谁在大门首?”玳安走了两步,说道:“是三娘、五娘、六娘在门首买瓜子哩。”西门庆到家下马,进入后边仪门首。玉楼、李瓶儿先去上房报月娘去了。独有金莲藏在粉壁背后黑影里。西门庆撞见,吓了一跳,说道:“怪小淫妇儿,猛可唬我一跳!你们在门首做甚么来?”金莲道:“你还敢说哩。你在那里?这时才来,教娘们只顾在门首等着你。”西门庆进房中,月娘安排酒肴,教玉箫执壶,大姐递酒。先递了西门庆,然后众姊妹都递了,安席坐下。春梅、迎春下边弹唱,吃了一回,都收下去。从新摆上玉楼上寿的酒,并四十样细巧各样的菜碟儿上来。壶斟美酝,盏泛流霞。让吴大妗子上坐。吃到起更时分,大妗子吃不多酒,归后边去了。止是吴月娘同众人陪西门庆掷骰猜枚行令。轮到月娘跟前,月娘道:“既要我行令,照依牌谱上饮酒:一个牌儿名,两个骨牌名,合《西厢》一句。”月娘先说:“六娘子醉杨妃,落了八珠环,游丝儿抓住荼蘼架。”不遇。该西门庆掷,说:“虞美人,见楚汉争锋,伤了正马军,只听耳边金鼓连天震。”果然是个正马军,吃了一杯。该李娇儿,说:“水仙子,因二士入桃源,惊散了花开蝶满枝,只做了落红满地胭脂冷。”不遇。次该金莲掷,说道:“鲍老儿,临老入花丛,坏了三纲五常,问他个非奸做贼拿。”果然是三纲五常,吃了一杯。轮该李瓶儿掷,说:“端正好,搭梯望月,等到春分昼夜停,那时节隔墙儿险化做望夫山。”不遇。该孙雪娥,说:“麻郎儿,见群鸦打凤,绊住了折足雁,好教我两下里做人难。”不遇。落后该玉楼完令,说:“念奴娇,醉扶定四红沉,拖着锦裙栏,得多少春风夜月销金帐。”正掷了四红沉。月娘满令,叫小玉:“斟酒与你三娘吃。”说道:“你吃三大杯才好!今晚你该伴新郎宿歇。”因对李瓶儿、金莲众人说:“吃毕酒,咱送他两个归房去。”金莲道:“姐姐严令,岂敢不依!”把玉楼羞的要不的。

少顷酒阑,月娘等相送西门庆到玉楼房首方回。玉楼让众人坐,都不坐。金莲便戏玉楼道:“我儿,好好儿睡罢。你娘明日来看你,休要淘气!”因向月娘道:“亲家,孩儿小哩,看我面上,凡是担待些儿罢。”玉楼道:“六丫头,你老米醋,挨着做。我明日和你答话。”金莲道:“我媒人婆上楼子——老娘好耐惊耐怕儿。”于是和李瓶儿、西门大姐一路去了。刚走到仪门首,不想李瓶儿被地滑了一交。这金莲遂怪乔叫起来道:“这个李大姐,只象个瞎子,行动一磨子就倒了。我搊你去,倒把我一只脚踩在雪里,把人的鞋儿也踹泥了!”月娘听见,说道:“就是仪门首那堆子雪。我分咐了小厮两遍,贼奴才,白不肯抬,只当还滑倒了。”因叫小玉:“你拿个灯笼送送五娘、六娘去。”西门庆在房里向玉楼道:“你看贼小淫妇儿!他踹在泥里把人绊了一交,他还说人踹泥了他的鞋,恰是那一个儿,就没些嘴抹儿。恁一个小淫妇!昨日叫丫头们平白唱‘佳期重会’,我就猜是他干的营生。”玉楼道:“‘佳期重会’是怎的说?”西门庆道:“他说吴家的不是正经相会,是私下相会。恰似烧夜香,有心等着我一般。”玉楼道:“六姐他诸般曲儿到都知道,俺们却不晓的。”西门庆道:“你不知,这淫妇单管咬群儿。”

不说西门庆在玉楼房中宿歇。单表潘金莲、李瓶儿两个走着说话,走到仪门,大姐便归前边厢房去了。小玉打着灯笼,送二人到花园内。金莲已带半酣,拉着李瓶儿道:“二娘,我今日有酒了,你好歹送到我房里。”李瓶儿道:“姐姐,你不醉。”须臾,送到金莲房内。打发小玉回后边,留李瓶儿坐,吃茶。金莲又道:“你说你那咱不得来,亏了谁?谁想今日咱姊妹在一个跳板儿上走,不知替你顶了多少瞎缸,教人背地好不说我!奴只行好心,自有天知道罢了。”李瓶儿道:“奴知道姐姐费心,恩当重报,不敢有忘。”金莲道:“得你知道,好了。”不一时,春梅拿茶来吃了,李瓶儿告辞归房。金莲独自歇宿,不在话下。正是:

\[
空庭高楼月,非复三五圆。
何须照床里,终是一人眠。
\]

\newpage
%# -*- coding:utf-8 -*-
%%%%%%%%%%%%%%%%%%%%%%%%%%%%%%%%%%%%%%%%%%%%%%%%%%%%%%%%%%%%%%%%%%%%%%%%%%%%%%%%%%%%%


\chapter{蕙莲儿偷期蒙爱\KG 春梅姐正色闲邪}


词曰:

\[
今宵何夕?月痕初照。等闲间一见犹难,平白地两边凑巧。向灯前见他,向灯前见他,一似梦中来到。何曾心料,他怕人瞧。惊脸儿红还白,热心儿火样烧。
\]

话说次日,有吴大妗子、杨姑娘、潘姥姥众堂客,因来与孟玉楼做生日,月娘都留在后厅饮酒,其中惹出一件事儿。那来旺儿,因他媳妇痨病死了,月娘新又与他娶了一房媳妇,乃是卖棺材宋仁的女儿,也名唤金莲。当先卖在蔡通判家房里使唤,后因坏了事出来,嫁与厨役蒋聪为妻。这蒋聪常在西门庆家答应,来旺儿早晚到蒋聪家叫他去,看见这个老婆,两个吃酒刮言,就把这个老婆刮上了。一日,不想这蒋聪因和一般厨役分财不均,酒醉厮打,动起刀杖来,把蒋聪戳死在地,那人便越墙逃走了。老婆央来旺儿对西门庆说了,替他拿帖儿县里和县丞说,差人捉住正犯,问成死罪,抵了蒋聪命。后来,来旺儿哄月娘,只说是小人家媳妇儿,会做针指。月娘使了五两银子,两套衣服,四匹青红布,并簪环之类,娶与他为妻。月娘因他叫金莲,不好称呼,遂改名为蕙莲。这个妇人小金莲两岁,今年二十四岁,生的白净,身子儿不肥不瘦,模样儿不短不长,比金莲脚还小些儿。性明敏,善机变,会妆饰,就是嘲汉子的班头,坏家风的领袖。若说他底的本事,他也曾:

\[
斜倚门儿立,人来侧目随。托腮并咬指,无故整衣裳。坐立频摇腿,无人曲唱低。开窗推户牖,停针不语时。未言先欲笑,必定与人私。
\]

初来时,同众媳妇上灶,还没甚么妆饰。后过了个月有余,因看见玉楼、金莲打扮,他便把\textuni{4BFC}髻垫的高高的,头发梳的虚笼笼的,水鬓描的长长的,在上边递茶递水,被西门庆睃在眼里。一日,设了条计策,教来旺儿押了五百两银子,往杭州替蔡太师制造庆贺生辰锦绣蟒衣,并家中穿的四季衣服,往回也有半年期程。从十一月半头,搭在旱路车上起身去了。西门庆安心早晚要调戏他这老婆,不期到此正值孟玉楼生日,月娘和众堂客在后厅吃酒。西门庆那日没往那去,月娘分咐玉箫:“房中另放桌儿,打发酒菜你爹吃。”西门庆因打帘内看见蕙莲身上穿着红绸对襟袄、紫绢裙子,在席上斟酒,问玉箫道:“那个是新娶的来旺儿的媳妇子蕙莲?怎的红袄配着紫裙子,怪模怪样?到明日对你娘说,另与他一条别的颜色裙子配着穿。”玉箫道:“这紫裙子,还是问我借的。”说着就罢了。

须臾,过了玉楼生日。一日,月娘往对门乔大户家吃酒去了。约后晌时分,西门庆从外来家,已有酒了,走到仪门首,这蕙莲正往外走,两个撞个满怀。西门庆便一手搂过脖子来,就亲了个嘴,口中喃喃呐呐说道:“我的儿,你若依了我,头面衣服,随你拣着用。”那妇人一声儿没言语,推开西门庆手,一直往前走了。西门庆归到上房,叫玉箫送了一匹蓝缎子到他屋里,如此这般对他说:“爹昨日见你穿着红袄,配着紫裙子,怪模怪样的不好看,才拿了这匹缎子,使我送与你,教你做裙子穿。”这蕙莲开看,却是一匹翠蓝兼四季团花喜相逢缎子。说道:“我做出来,娘见了问怎了?”玉箫道:“爹到明日还对娘说,你放心。爹说来,你若依了这件事,随你要甚么,爹与你买。今日赶娘不在家,要和你会会儿,你心下如何?”那妇人听了,微笑不言,因问:“爹多咱时分来?我好在屋里伺候。”玉箫道:“爹说小厮们看着,不好进你屋里来的。教你悄悄往山子底下洞儿里,那里无人,堪可一会。”老婆道:“只怕五娘、六娘知道了,不好意思的。”玉箫道:“三娘和五娘都在六娘屋里下棋,你去不妨事。”当下约会已定,玉箫走来回西门庆说话。两个都往山子底下成事,玉箫在门首与他观风。正是:

\[
解带色已战,触手心愈忙。
那识罗裙内,销魂别有香。
\]

不想金莲、玉楼都在李瓶儿房里下棋,只见小鸾来请玉楼,说:“爹来家了。”三人就散了,玉楼回后边去了。金莲走到房中,匀了脸,亦往后边来。走入仪门,只见小玉立在上房门首。金莲问:“你爹在屋里?”小玉摇手儿,往前指。金莲就知其意,走到前边山子角门首,只见玉箫拦着门。金莲只猜玉箫和西门庆在此私狎,便顶进去。玉箫慌了,说道:“五娘休进去,爹在里头有勾当哩!”金莲骂道:“怪狗肉,我又怕你爹了?”不由分说,进入花园里来,各处寻了一遍。走到藏春坞山子洞儿里,只见他两个人在里面才了事。妇人听见有人来,连忙系上裙子往外走,看见金莲,把脸通红了。金莲问道:“贼臭肉,你在这里做甚么?”蕙莲道:“我来叫画童儿。”说着,一溜烟走了。金莲进来,看见西门庆在里边系裤子,骂道:“贼没廉耻的货,你和奴才淫妇大白日里在这里,端的干这勾当儿,刚才我打与淫妇两个耳刮子才好,不想他往外走了。原来你就是画童儿,他来寻你!你与我实说,和这淫妇偷了几遭?若不实说,等住回大姐姐来家,看我说不说。我若不把奴才淫妇脸打的胀猪,也不算。俺们闲的声唤在这里,你也来插上一把子。老娘眼里却放不过!”西门庆笑道:“怪小淫妇儿,悄悄儿罢,休要嚷的人知道。我实对你说,如此这般,连今日才第一遭。”金莲道:“一遭二遭,我不信。你既要这奴才淫妇,两个瞒神谎鬼弄刺子儿,我打听出来,休怪了,我却和你们答话!”那西门庆笑的出去了。

金莲到后边,听见众丫头们说:“爹来家,使玉箫手巾裹着一匹蓝缎子往前边去,不知与谁。”金莲就知是与蕙莲的,对玉楼也不题起此事。这妇人每日在那边,或替他造汤饭,或替他做针指鞋脚,或跟着李瓶儿下棋,常贼乖趋附金莲。被西门庆撞在一处,无人,教他两个苟合,图汉子喜欢。蕙莲自从和西门庆私通之后,背地与他衣服、首饰、香茶之类不算,只银子成两家带在身边,在门首买花翠胭脂,渐渐显露,打扮的比往日不同。西门庆又对月娘说,他做的好汤水,不教他上大灶,只教他和玉箫两个,在月娘房里后边小灶上,专顿茶水,整理菜蔬,打发月娘房里吃饭,与月娘做针指,不必细说。看官听说:凡家主,切不可与奴仆并家人之妇苟且私狎,久后必紊乱上下,窃弄奸欺,败坏风俗,殆不可制。

一日,腊月初八日,西门庆早起,约下应伯爵,与大街坊尚推官家送殡。叫小厮马也备下两匹,等伯爵白不见到,一面李铭来了。西门庆就在大厅上围炉坐的,教春梅、玉箫、兰香、迎春一般儿四个,都打扮出来,看着李铭指拨、教演他弹唱。女婿陈敬济,在傍陪着说话。正唱《三弄梅花》,还未了,只见伯爵来,应保夹着毡包进门。那春梅等四个就要往后走,被西门庆喝住,说道:“左右只是你应二爹,都来见见罢,躲怎的!”与伯爵两个相见作揖,才待坐下,西门庆令四个过来:“与应二爹磕头。”那春梅等朝上磕头下去,慌的伯爵还喏不迭,夸道:“谁似哥有福,出落的恁四个好姐姐,水葱儿的一般,一个赛一个。却怎生好?你应二爹今日素手,促忙促急,没曾带的甚么在身边,改日送胭脂钱来罢。”春梅等四人,见了礼去了。陈敬济向前作揖,一同坐下。西门庆道:“你如何今日这咱才来?”应伯爵道:“不好告诉你的。大小女病了一向,近日才好些。房下记挂着,今日接了他家来散心住两日。乱着,旋叫应保叫了轿子,买了些东西在家,我才来了。”西门庆道:“教我只顾等着你。咱吃了粥,好去了。”随即分付后边看粥来吃。只见李铭,见伯爵打了半跪。伯爵道:“李日新,一向不见你。”李铭道:“小的有。连日小的在北边徐公公那里答应来。”说着,小厮放桌儿,拿粥来吃。西门庆陪应伯爵、陈敬济吃了。就拿小银锺筛金华酒,每人吃了三杯。壶里还剩下上半壶酒,分付画童儿:“连桌儿抬去厢房内,与李铭吃。”就穿衣服起身,同伯爵并马而行,与尚推官送殡去了。只落下李铭在西厢房,吃毕酒饭。

玉箫和兰香众人,打发西门庆出了门,在厢房内厮乱,顽成一块。一回,都往对过东厢房西门大姐房里掴混去了,止落下春梅一个,和李铭在这边教演琵琶。李铭也有酒了。春梅袖口子宽,把手兜住了。李铭把他手拿起,略按重了些。被春梅怪叫起来,骂道:“好贼忘八!你怎的捻我的手,调戏我?贼少死的忘八,你还不知道我是谁哩!一日好酒好肉,越发养活的你这忘八圣灵儿出来了,平白捻我的手来了。贼忘八,你错下这个锹撅了。你问声儿去,我手里你来弄鬼!爹来家等我说了,把你这贼忘八,一条棍撵的离门离户!没你这忘八,学不成唱了?愁本司三院寻不出忘八来?撅臭了你这忘八了!”被他千忘八,万忘八,骂的李铭拿着衣服,往外走不迭。正是:

\[
两手劈开生死路,翻身跳出是非门。
\]

当下春梅气狠狠,直骂进后边来。金莲正和孟玉楼、李瓶儿并宋蕙莲在房里下棋,只听见春梅从外骂将来。金莲便问道:“贼小肉儿,你骂谁哩,谁惹你来?”春梅道:“情知是谁,叵耐李铭那忘八!爹临去,好意分付小厮,留下一桌菜并粳米粥儿与他吃。也有玉箫他们,你推我,我打你,顽成一块,对着忘八,呲牙露嘴的,狂的有些褶儿也怎的。顽了一回,都往大姐那边去了。忘八见无人,尽力把我手上捻一下。吃的醉醉的,看着我嗤嗤呆笑。那忘八见我吆喝骂起来,他就夹着衣裳往外走了。刚才打与贼忘八两个耳刮子才好!贼忘八,你也看个人儿行事,我不是那不三不四的邪皮行货,教你这个忘八在我手里弄鬼。我把忘八脸打绿了!”金莲道:“怪小肉儿,学不学没要紧,把脸气的黄黄的,等爹来家说了,把贼忘八撵了去就是了。那里紧等着供唱撰钱哩,怎的教忘八调戏我这丫头!我知道贼忘八业罐子满了。”春梅道:“他就倒运,着量二娘的兄弟。那怕他!二娘莫不挟仇打我五棍儿?”宋蕙莲道:“论起来,你是乐工,在人家教唱,也不该调戏良人家女子!照顾你一个钱,也是养身父母,休说一日三茶六饭儿扶侍着。”金莲道:“扶侍着,临了还要钱儿去了。按月儿,一个月与他五两银子。贼忘八,错上了坟。你问声家里这些小厮们,那个敢望着他呲牙笑一笑儿,吊个嘴儿?遇喜欢骂两句;若不欢喜,拉倒他主子跟前就是打。贼忘八,造化低,你惹他生姜,你还没曾经着他辣手!”因向春梅道:“没见你,你爹去了,你进来便罢了,平白只顾和他那房里做甚么?却教那忘八调戏你!”春梅道:“都是玉箫和他们,只顾还笑成一块,不肯进来。”玉楼道:“他三个如今还在那屋里?”春梅道:“都往大姐房里去了。”玉楼道:“等我瞧瞧去。”那玉楼起身去了。良久,李瓶儿亦回房,使绣春叫迎春去。至晚,西门庆来家,金莲一五一十告诉西门庆。西门庆分付来兴儿,今后休放进李铭来走动。自此断了路儿,不敢上门。正是:

\[
习教歌妓逞家豪,每日闲庭弄锦槽。
不是朱颜容易变,何由声价竞天高。
\]

\newpage
%# -*- coding:utf-8 -*-
%%%%%%%%%%%%%%%%%%%%%%%%%%%%%%%%%%%%%%%%%%%%%%%%%%%%%%%%%%%%%%%%%%%%%%%%%%%%%%%%%%%%%


\chapter{赌棋枰瓶儿输钞\KG 觑藏春潘氏潜踪}


词曰:

\[
心中难自泄,暗里深深谢。未必娘行,恁地能贤哲。衷肠怎好和君说?说不愿丫头,愿做官人的侍妾。他坚牢望我情真切。岂想风波,果应了他心料者。
\]

话说一日腊尽春回,新正佳节,西门庆贺节不在家,吴月娘往吴大妗子家去了。午间孟玉楼、潘金莲都在李瓶儿房里下棋。玉楼道:“咱们今日赌甚么好?”金莲道:“咱们赌五钱银子东道,三钱银子买金华酒儿,那二钱买个猪头来,教来旺媳妇子烧猪头咱们吃。说他会烧的好猪头,只用一根柴禾儿,烧的稀烂。”玉楼道:“大姐姐不在家,却怎的计较?”存下一分儿,送在他屋里,也是一般。”说毕,三人下棋。下了三盘,李瓶儿输了五钱。金莲使绣春儿叫将来兴儿来,把银子递与他,教他买一坛金华酒,一个猪首,连四只蹄子,分咐:“送到后边厨房里,教来旺儿媳妇蕙莲快烧了,拿到你三娘屋里等着,我们就去。”玉楼道:“六姐,教他烧了拿盒子拿到这里来吃罢。在后边,李娇儿、孙雪娥两个看着,是请他不请他?”金莲遂依玉楼之言。

不一时,来兴儿买了酒和猪首,送到厨下。蕙莲正在后边和玉箫在石台基上坐着,挝瓜子耍子哩。来兴儿便叫他:“蕙莲嫂子,五娘、三娘都上覆你,使我买了酒、猪头连蹄子,都在厨房里,教你替他烧熟了,送到前边六娘房里去。”蕙莲道:“我不得闲,与娘纳鞋哩。随问教那个烧烧儿罢,巴巴坐名儿教我烧?”来兴儿道:“你烧不烧随你,交与你,我有勾当去。”说着,出去了。玉箫道:“你且丢下,替他烧烧罢。你晓的五娘嘴头子,又惹的声声气气的。”蕙莲笑道:“五娘怎么就知道我会烧猪头,栽派与我!”于是起到大厨灶里,舀了一锅水,把那猪首蹄子剃刷干净,只用的一根长柴禾安在灶内,用一大碗油酱,并茴香大料,拌的停当,上下锡古子扣定。那消一个时辰,把个猪头烧的皮脱肉化,香喷喷五味俱全。将大冰盘盛了,连姜蒜碟儿,用方盒拿到前边李瓶儿房里,旋打开金华酒来。玉楼拣齐整的,留下一大盘子,并一壶金华酒,使丫头送到上房里,与月娘吃。其余三人坐定,斟酒共酌。

正吃中间,只见蕙莲笑嘻嘻走到跟前,说道:“娘们试尝这猪头,今日烧的好不好?”金莲道:“三娘刚才夸你倒好手段儿!烧的且是稀烂。”李瓶儿问道:“真个你只用一根柴禾儿?”蕙莲道:“不瞒娘们说,还消不得一根柴禾儿哩!若是一根柴禾儿,就烧的脱了骨。”玉楼叫绣春:“你拿个大盏儿,筛一盏儿与你嫂子吃。”李瓶儿连忙叫绣春斟酒,他便取碟儿拣了一碟猪头肉儿递与蕙莲,说道:“你自造的,你试尝尝。”蕙莲道:“小的自知娘们吃不的咸,没曾好生加酱,胡乱罢了。下次再烧时,小的知道了。”便磕了三个头,方才在桌头旁边立着,做一处吃酒。

到晚夕月娘来家,众妇人见了月娘,小玉悉将送来猪头,拿与月娘看。玉楼笑道:“今日俺们下棋耍子,赢的李大姐猪头,留与姐姐吃。”月娘道:“这般有些不均了。各人赌胜,亏了一个就不是了。咱们这等计较:只当大节下,咱姊妹这几人每人轮流治一席酒儿,叫将郁大姐来,晚间耍耍,有何妨碍?强如赌胜负,难为一个人。我主张的好不好?”众人都说:“姐姐主张的是!”月娘道:“明日初五日,就是我起先罢。”李娇儿占了初六,玉楼占了初七,金莲占了初八。金莲道:“只我便宜,那日又是我的寿酒,却一举而两得。”问着孙雪娥,孙雪娥半日不言语。月娘道:“他罢,你们不要缠他了,教李大姐挨着罢。”玉楼道:“初九日又是六姐生日,只怕有潘姥姥和他妗子来。”月娘道:“初九日不得闲,教李大姐挪在初十罢了。”众人计议已定。

话休絮烦。先是初五日,西门庆不在家,往邻家赴席去了。月娘在上房摆酒,郁大姐供唱,请众姐妹欢饮了一日方散。到第二日,却该李娇儿,就挨着玉楼、金莲,都不必细说。须臾,过了金莲生日,潘姥姥、吴大妗子,都在这里过节顽耍。看看到初十日,该李瓶儿摆酒,使绣春往后边请雪娥去。一连请了两替,答应着来,只顾不来。玉楼道:“我就说他不来,李大姐只顾强去请他。可是他对着人说的:‘你每有钱的,都吃十轮酒儿,没的俺们去赤脚绊驴蹄。’似他这等说,俺们罢了,把大姐姐都当驴蹄看承!”月娘道:“他是恁不成材的行货子,都不消理他了,又请他怎的!”于是摆上酒来,众人都来前边李瓶儿房里吃酒。郁大姐在旁弹唱。当下,吴大妗子和西门大姐,共八个人饮酒。只因西门庆不在,月娘分咐玉箫:“等你爹来家要吃酒,你打发他吃就是了。”玉箫应诺。

后晌时分,西门庆来家,玉箫替他脱了衣裳。西门庆便问:“娘往那去了?”玉箫回道:“都在六娘房里和大妗子、潘姥姥吃酒哩。”西门庆问道:“吃的是甚么酒?”玉箫道:“是金华酒。”西门庆道:“还有年下你应二爹送的那一坛茉莉花酒,打开吃。”一面教玉箫把茉莉花酒打开,西门庆尝了尝,说道:“正好你娘们吃。”教小玉、玉箫两个提着,送到前边李瓶儿房里。蕙莲正在月娘旁边侍立斟酒,见玉箫送酒来,蕙莲俐便,连忙走下来接酒。玉箫便递了个眼色与他,向他手上捏了一把,这婆娘就知其意。月娘问玉箫:“谁使你送酒来?”玉箫道:“爹使我来。”月娘道:“你爹来家多大回了?”玉箫道:“爹刚才来家。因问娘们吃酒,教我把这一坛茉莉花酒,拿来与娘们吃。”月娘问:“你爹若吃酒,房中放桌儿,有见成菜儿打发他吃。”玉箫应的,往后边去了。

这蕙莲在席上站了一回,推说道:“我后边看茶来,与娘们吃。”月娘分咐道:“对你姐说,上房拣妆里有六安茶,顿一壶来俺们吃。”这老婆一个猎古调走到后边,玉箫站在堂屋门首,努了个嘴儿与他。老婆掀开帘子,进月娘房来,只见西门庆坐在椅子上吃酒。走向前,一屁股就坐在他怀里,两个就亲嘴咂舌做一处。婆娘一面用手攥着他那话,一面在上噙酒哺与他吃。便道:“爹,你有香茶再与我些,前日与我的都没了。我少薛嫂儿几钱花儿钱,你有银子与我些儿。”西门庆道:“我茄袋内还有一二两,你拿去。”说着。西门庆要解他裤子。妇人道:“不好,只怕人来看见。”西门庆道:“你今日不出去,晚夕咱好生耍耍。”蕙莲摇头说道:“后边惜薪司挡路儿——柴众。咱不如还在五娘那里,色丝子女。”于是玉箫在堂屋门首观风,由他二人在屋里做一处顽耍。不防孙雪娥从后来,听见房里有人笑,只猜玉箫在房里和西门庆说笑,不想玉箫又在穿廊下坐的,就立住了脚。玉箫恐怕他进屋里去,便支他说:“前边六娘请姑娘,怎的不去?”雪娥鼻子里冷笑道:“俺们是没时运的人儿,骑着快马也赶他不上,拿甚么伴着他吃十轮酒儿?自己穷的伴当儿伴的没裤儿!”正说着,被西门庆房中咳嗽了一声,雪娥就往厨房里去了。

这玉箫把帘子欣开,婆娘见无人,急伶俐两三步就叉出来,往后边看茶去。须臾,小玉从后边走来叫:“蕙莲嫂子,娘说你怎的取茶就不去了?”妇人道:“茶有了,着姐拿果仁儿来。”不一时,小玉拿着盏托,他提着茶,一直来到前边。月娘问道:“怎的茶这咱才来?”蕙莲道:“爹在房里吃酒,小的不敢进去。等着姐屋里取茶叶,剥果仁儿来。”众人吃了茶,这蕙莲在席上,斜靠桌儿站立,看着月娘众人掷骰儿,故作扬声说道:“娘,把长么搭在纯六,却不是天地分?还赢了五娘。”又道:“你这六娘,骰子是锦屏风对儿。我看三娘这么三配纯五,只是十四点儿,输了。”被玉箫恼了,说道:“你这媳妇子,俺们在这里掷骰儿,插嘴插舌,有你甚么说处?”把老婆羞的站又站不住,立又立不住,绯红了面皮,往下去了。正是:

\[
谁人汲得西江水,难洗今朝一面羞。
\]

这里众妇人饮酒,至掌灯时分,只见西门庆掀帘子进来,笑道:“你们好吃!”吴大妗子跳起来,说道:“姐夫来了!”连忙让座儿与他坐。月娘道:“你在后边吃酒罢了,女妇男子汉,又走来做甚么?”西门庆道:“既是恁说,我去罢。”于是走过金莲这边来,金莲随即跟了来。西门庆吃得半醉,拉着金莲说道:“小油嘴,我有句话儿和你说。我要留蕙莲在后边一夜儿,后边没地方。看你怎的容他在你这边歇一夜儿罢?”金莲道:“我不好骂的,没的那汗邪的胡乱!随你和他那里\textuni{34B2}捣去,好娇态,教他在我这里!我是没处安放他。我就算依了你,春梅贼小肉儿他也不容。你不信,叫了春梅问他,他若肯了,我就容你。”西门庆道:“既是你娘儿们不肯,罢!我和他往山子洞儿那里过一夜。你分咐丫头拿床铺盖,生些火儿。不然,这一冷怎么当。”金莲忍不住笑了:“我不好骂出你来的,贼奴才淫妇,他是养你的娘?你是王祥,寒冬腊月行孝顺,在那石头床上卧冰哩。”西门庆笑道:“怪小油嘴儿,休奚落我。罢么,好歹叫丫头生个火儿。”金莲道:“你去,我知道。”当晚众人席散,金莲分咐秋菊,果然抱铺盖、笼火,在山子底下藏春坞雪洞里。

蕙莲送月娘、李娇儿、玉楼进到后边仪门首,故意说道:“娘,小的不送,往前边去罢。”月娘道:“也罢,你前边睡去罢。”这婆娘打发月娘进内,还在仪门首站立了一回,见无人,一溜烟往山子底下去了。正是:

\[
莫教襄王劳望眼,巫山自送雨云来。
\]
这宋蕙莲走到花园门首,只说西门庆还未进来,就不曾扣门子,只虚掩着。来到藏春坞洞儿内,只见西门庆早在那里秉烛而坐。婆娘进到里面,但觉冷气侵人,尘嚣满榻。于是袖中取出两枝棒儿香,灯上点了,插在地下。虽故地下笼着一盆碳火儿,还冷的打兢。婆娘在床上先伸下铺,上面还盖着一件貂鼠禅衣。掩上双扉,两个上床就寝。西门庆脱去上衣白绫道袍,坐在床上,把妇人褪了裤,抱在怀里,两只脚跷在两边,那话突入牝中。两个搂抱,正做得好。却不防潘金莲打听他二人入港了,在房中摘去冠儿,轻移莲步,悄悄走来窃听。到角门首,推开门,遂潜身悄步而入。也不怕苍苔冰透了凌波,花刺抓伤了裙褶,蹑迹隐身,在藏春坞月窗下站听。良久,只见里面灯烛尚明,婆娘笑声说:“冷铺中舍冰,把你贼受罪不济的老花子,就没本事寻个地方儿,走在这寒冰地狱里来了!口里衔着条绳子,冻死了往外拉。”又道:“冷合合的,睡了罢,怎的只顾端详我的脚?你看过那小脚儿的来,象我没双鞋面儿,那个买与我双鞋面儿也怎的?看着人家做鞋,不能彀做!”西门庆道:“我儿,不打紧,到明日替你买几钱的各色鞋面。谁知你比你五娘脚儿还小!”妇人道:“拿甚么比他!昨日我拿他的鞋略试了试,还套着我的鞋穿。倒也不在乎大小,只是鞋样子周正才好。”金莲在外听了:“这个奴才淫妇!等我再听一回,他还说甚么。”又听彀多时,只听老婆问西门庆说:“你家第五的秋胡戏,你娶他来家多少时了?是女招的,是后婚儿来?”西门庆道:“也是回头人儿。”妇人说:“嗔道恁久惯牢成!原来也是个意中人儿,露水夫妻。”这金莲不听便罢,听了气的在外两只胳膊都软了,半日移脚不动,说道:“若教这奴才淫妇在里面,把俺们都吃他撑下去了!”待要那时就声张骂起来,又恐怕西门庆性子不好,逞了淫妇的脸。待要含忍了他,恐怕他明日不认。“罢罢!留下个记儿,使他知道,到明日我和他答话。”于是走到角门首,拔下头上一根银簪儿,把门倒销了,懊恨归房。晚景题过。

到次日清早晨,婆娘先起来,穿上衣裳,蓬着头走出来。见角门没插,吃了一惊,又摇门,摇了半日摇不开。走去见西门庆,西门庆隔壁叫迎春替他开了。因看见簪销着门,知是金莲的簪子,就知晚夕他听了出去。这妇人怀着鬼胎,走到前边,正开房门,只见平安从东净里出来,看见他只是笑。蕙莲道:“怪囚根子,谁和你呲那牙笑哩?”平安儿道:“嫂子,俺们笑笑儿也嗔?”蕙莲道:“大清早晨,平白笑的是甚么?”平安道:“我笑嫂子三日没吃饭,眼前花。我猜你昨日一夜不来家!”妇人听了此言,便把脸红了,骂道:“贼提口拔舌见鬼的囚根子,我那一夜不在屋里睡?怎的不来家?”平安道:“我刚才还看见嫂子锁着门,怎的赖得过?”蕙莲道:“我早起身,就往五娘屋里,只刚才出来。你这囚在那里来?”平安道:“我听见五娘教你腌螃蟹,说你会劈的好腿儿。嗔道五娘使你门首看着卖簸箕的,说你会咂得好舌头。”把妇人说的急了,拿起条门闩来,赶着平安儿绕院子骂道:“贼汗邪囚根子,看我到明日对他说不说。不与你个功德也不怕,狂的有些褶儿也怎的?”那平安道:“耶嚛,嫂子,将就着些儿罢。对谁说?我晓得你往高枝儿上去了。”那蕙莲急起来,只赶着他打。不料玳安正在印子铺走出来,一把手将闩夺住了,说道:“嫂子为甚么打他?”蕙莲道:“你问那呲牙囚根子,口里白说六道的,把我的胳膊都气软了!”那平安得手往外跑了。玳安推着他说:“嫂子,你少生气着恼,且往屋里梳头去罢。”妇人便向腰间荷包里,取出三四分银子来,递与玳安道:“累你替我拿大碗烫两个合汁来我吃,把汤盛在铫子里罢。”玳安道:“不打紧,等我去。”一手接了。连忙洗了脸,替他烫了合汁来。妇人让玳安吃了一碗,他也吃了一碗,方才梳了头,锁上门,先到后边月娘房里打了卯儿,然后来金莲房里。

金莲正临镜梳头。蕙莲小意儿,在旁拿抵镜、掇洗手水,殷情侍奉。金莲正眼也不瞧他。蕙莲道:“娘的睡鞋裹脚,我卷平收了去?”金莲道:“由他。你放着,叫丫头进来收。”便叫秋菊:“贼奴才,往那去了?”蕙莲道:“秋菊扫地哩。春梅姐在那里梳头哩。”金莲道:“你别要管他,丢着罢,亦发等他们来收拾。歪蹄泼脚的,没的沾污了嫂子的手。你去扶侍你爹,爹也得你恁个人儿扶侍他,才可他的心。俺们都是露水夫妻,再醮货儿。只嫂子是正名正顶轿子娶将来的,是他的正头老婆,秋胡戏。”这妇人听了,正道着昨日晚夕他的真病,于是向前双膝跪下,说道:“娘是小的一个主儿,娘不高抬贵手,小的一时儿存站不的。当初不因娘宽恩,小的也不肯依随爹。就是后边大娘,无过只是个大纲儿。小的还是娘抬举多,莫不敢在娘面前欺心?随娘查访,小的但有一字欺心,到明日不逢好死,一个毛孔儿里生下一个疔疮。”金莲道:“不是这等说。我眼里放不下砂子的人。汉子既要了你,俺们莫不与争?不许你在汉子跟前弄鬼,轻言轻语的。你说你把俺们踩下去了,你要在中间踢跳,我的姐姐,对你说,把这样心儿且吐了些儿罢!”蕙莲道:“娘再访,小的并不敢欺心,到只怕昨日晚夕娘错听了。”金莲道:“傻嫂子,我闲的慌,听你怎的?我对你说了罢,十个老婆买不住一个男子汉的心。你爹虽故家里有这几个老婆,或是外边请人家的粉头,来家通不瞒我一些儿,一五一十就告我说。你大娘当时和他一个鼻子眼儿里出气,甚么事儿来家不告诉我?你比他差些儿。”说得老婆闭口无言,在房中立了一回,走出来了。刚到仪门夹道内,撞见西门庆,说道:“你好人儿,原来昨日人对你说的话儿,你就告诉与人。今日教人下落了我恁一顿!我和你说的话儿,只放在你心里,放烂了才好。为甚么对人说?干净你这嘴头子就是个走水的槽。有话到明日不告你说了。”西门庆道:“甚么话?我并不知道。”那妇人瞅了一眼,往前边去了。

这妇人嘴儿乖,常在门前站立,买东买西,赶着傅伙计叫傅大郎,陈敬济叫姑夫,贲四叫老四。因和西门庆勾搭上了,越发在人前花哨起来,常和众人打牙犯嘴,全无忌惮。或一时叫:“傅大郎,我拜你拜,替我门首看着卖粉的。”那傅伙计老成,便惊心儿替他门首看着,过来叫住,请他出来买。玳安故意戏他,说道:“嫂子,卖粉的早晨过去了,你早出来,拿秤称他的好来!”婆娘骂道:“贼猴儿,里边五娘、六娘使我要买搽的粉,你如何说拿秤称二斤胭脂三斤粉,教那淫妇搽了又搽?看我进里边对他说不说?”玳安道:“耶嚛,嫂子,行动只拿五娘吓我!”一回又叫:“贲老四,我对你说,门首看着卖梅花菊花的,我要买两对儿戴。”那贲四误了买卖,好歹专心替他看着卖的叫住,请他出来买。妇人立在二层门里,打门厢儿拣,要了他两对鬓花大翠,又是两方紫绫闪色销金汗巾儿,共该他七钱五分银子。妇人向腰里摸出半侧银子儿来,央及贲四替他凿,称七钱五分与他。那贲四正写着帐,丢下走来替他锤。只见玳安来说道:“等我与嫂子凿。”一面接过银子在手,且不凿,只顾瞧这银子。妇人道:“贼猴儿,不凿,只顾端详甚么?你半夜没听见狗咬?是偷来的银子!”玳安道:“偷到不偷。这银子到有些眼熟,倒象爹银子包儿里的。前日爹在灯市里,凿与卖勾金蛮子的银子,还剩了一半,就是这银子。我记得千真万真。”妇人道:“贼囚,一个天下,人还有一样的,爹的银子怎的到得我手里?”玳安笑道:“我知道甚么帐儿!”妇人便赶着打。玳安把银子凿下七钱五分,交与卖花翠的,把剩的银子拿在手里,不与他去了。妇人道:“贼囚根子!你敢拿了去,我算你好汉!”玳安道:“我不拿你的。你把剩下的,与我些儿买果子吃。”那妇人道:“贼猴儿,你递过来,我与你。”哄和玳安递到他手里,只掠了四五分一块与他,别的还塞在腰里,一直进去了。

自此以后,常在门首成两价拿银钱买剪截花翠汗巾之类,甚至瓜子儿四五升里进去,分与各房丫鬟并众人吃。头上治的珠子箍儿,金灯笼坠子,黄烘烘的。衣服底下穿着红\textSiLuo 绸裤儿,线捺护膝。又大袖子袖着香茶、香桶子三四个,带在身边。见一日也花消二三钱银子,都是西门庆背地与他的,此事不必细说。这妇人自从金莲识破他机关,每日只在金莲房里,把小意儿贴恋,与他顿茶顿水,做鞋脚针指,不拿强拿,不动强动。正经月娘后边,每日只打个到面儿,就到金莲这边来。每日和金莲、瓶儿两个下棋、抹牌,行成伙儿。或一时撞见西门庆来,金莲故意令他旁边斟酒,教他一处坐了顽耍,只图汉子喜欢。正是:

\[
颠狂柳絮随风舞,轻薄桃花逐水流。
\]

\newpage
%# -*- coding:utf-8 -*-
%%%%%%%%%%%%%%%%%%%%%%%%%%%%%%%%%%%%%%%%%%%%%%%%%%%%%%%%%%%%%%%%%%%%%%%%%%%%%%%%%%%%%


\chapter{敬济元夜戏娇姿\KG 惠祥怒詈来旺妇}


诗曰:

\[
银烛高烧酒乍醺,当筵且喜笑声频。
蛮腰细舞章台柳,素口轻歌上苑春。
香气拂衣来有意,翠花落地拾无声。
不因一点风流趣,安得韩生醉后醒。
\]

话说一日,天上元宵,人间灯夕,西门庆在厅上张挂花灯,铺陈绮席。正月十六,合家欢乐饮酒。西门庆与吴月娘居上,其余李娇儿、孟玉楼、潘金莲、李瓶儿、孙雪娥、西门大姐都在两边同坐,都穿着锦绣衣裳。春梅、玉箫、迎春、兰香一般儿四个家乐,在旁\textShouLuan 筝歌板,弹唱灯词。独于东首设一席与女婿陈敬济坐。果然食烹异品,果献时新。小玉、元宵、小鸾、绣春都在上面斟酒。那来旺儿媳妇宋蕙莲却坐在穿廊下一张椅儿上,口里磕瓜子儿。等的上边呼唤要酒,他便扬声叫:“来安儿,画童儿,上边要热酒,快趱酒上来!贼囚根子,一个也没在这里伺候,都不知往那去了!”只见画童烫酒上去。西门庆就骂道:“贼奴才,一个也不在这里伺候,往那去来?贼少打的奴才!”小厮走来说道:“嫂子,谁往那去来?就对着爹说,吆喝教爹骂我。”蕙莲道:“上头要酒,谁教你不伺候?关我甚事!不骂你骂谁?”画童儿道:“这地上干干净净的,嫂子磕下恁一地瓜子皮,爹看见又骂了。”蕙莲道:“贼囚根子!六月债儿热,还得快就是。甚么打紧,便当你不扫,丢着,另教个小厮扫。等他问我,只说得一声。”画童儿道:“耶嚛,嫂子,将就些罢了,如何和我合气!”于是取了笤帚来,替他扫瓜子皮儿,不题。

却说西门庆席上,见女婿陈敬济没酒,分咐潘金莲去递一巡儿。这金莲连忙下来,满斟杯酒,笑嘻嘻递与敬济,说道:“姐夫,你爹分咐,好歹饮奴这杯酒儿。”敬济一壁接酒,一面把眼儿斜溜妇人,说:“五娘请尊便,等儿子慢慢吃!”妇人将身子把灯影着,左手执酒,刚待的敬济将手来接,右手向他手背只一捻,这敬济一面把眼瞧着众人,一面在下戏把金莲小脚儿踢了一下。妇人微笑,低声道:“怪油嘴,你丈人瞧着待怎么?”两个在暗地里调情顽耍,众人倒不曾看出来。不料宋蕙莲这婆娘,在槅子外窗眼里,被他瞧了个不耐烦。口中不言,心下自忖:“寻常在俺们跟前,到且是精细撇清,谁想暗地却和这小伙子儿勾搭。今日被我看出破绽,到明日再搜求我,自有话说。”正是:

\[
谁家院内白蔷薇,暗暗偷攀三两枝。
罗袖隐藏人不见,馨香惟有蝶先知。
\]

饮酒多时,西门庆忽被应伯爵差人请去赏灯。分咐月娘:“你们自在耍耍,我往应二哥家吃酒去来。”玳安、平安两个跟随去了。

月娘与众姊妹吃了一回,但见银河清浅,珠斗烂斑,一轮团圆皎月从东而出,照得院宇犹如白昼。妇人或有房中换衣者,或有月下整妆者,或有灯前戴花者。惟有玉楼、金莲、李瓶儿三个并蕙莲,在厅前看敬济放花儿。李娇儿、孙雪娥、西门大姐都随月娘后边去了。金莲便向二人说道:“他爹今日不在家,咱对大姐姐说,往街上走走去。”蕙莲在旁说道:“娘们去,也携带我走走。”金莲道:“你既要去,你就往后边问声你大娘和你二娘,看他去不去,俺们在这里等着你。”那蕙莲连忙往后边去了。玉楼道:“他不济事,等我亲自问他声去。”李瓶儿道:“我也往屋里穿件衣裳,只怕夜深了冷。”金莲道:“李大姐,你有披袄子,带件来我穿,省得我往屋里去。”那李瓶儿应诺去了。独剩下金莲一个,看着敬济放花儿。见无人,走向敬济身上捏了一把,笑道:“姐夫原来只穿恁单薄衣裳,不害冷么?”只见家人儿子小铁棍儿笑嘻嘻在跟前,舞旋旋的且拉着敬济,要炮丈放。这敬济恐怕打搅了事,巴不得与了他两个元宵炮丈,支他外边耍去了。于是和金莲嘲戏说道:“你老人家见我身上单薄,肯赏我一件衣裳儿穿穿也怎的?”金莲道:“贼短命,得其惯便了,头里头蹑我的脚儿,我不言语,如今大胆,又来问我要衣服穿!我又不是你影射的,何故把与你衣服穿?”敬济道:“你老人家不与就罢了,如何扎筏子来唬我?”妇人道:“贼短命,你是城楼上雀儿,好耐惊耐怕的虫蚁儿!”正说着,见玉楼和蕙莲出来,向金莲说道:“大娘因身上不方便,大姐不自在,故不去了。教娘们走走,早些来家。李娇儿害腿疼,也不走。孙雪娥见大姐姐不走,恐怕他爹来家嗔他,也不出门。”金莲道:“都不去罢,只咱和李大姐三个去罢。等他爹来家,随他骂去!再不,把春梅小肉儿和上房里玉箫,你房里兰香,李大姐房里迎春,都带了去。”小玉走来道:“俺奶奶已是不去,我也跟娘们走走。”玉楼道:“对你奶奶说了去,我前头等着你。”良久,小玉问了月娘,笑嘻嘻出来。当下三个妇人,带领着一簇男女。来安、画童两个小厮,打着一对纱吊灯跟随。女婿陈敬济踹着马台,放烟火花炮,与众妇人瞧。宋蕙莲道:“姑夫,你好歹略等等儿。娘们携带我走走,我到屋里搭搭头就来。”敬济道:“俺们如今就行。”蕙莲道:“你不等,我就恼你一生!”于是走到屋里,换了一套绿闪红缎子对衿衫儿、白挑线裙子。又用一方红销金汗巾子搭着头,额角上贴着飞金并面花儿,金灯笼坠耳,出来跟着众人走百媚儿。月色之下,恍若仙娥,都是白绫袄儿,遍地金比甲。头上珠翠堆满,粉面朱唇。敬济与来兴儿,左右一边一个,随路放慢吐莲、金丝菊、一丈兰、赛月明。出的大街市上,但见香尘不断,游人如蚁,花炮轰雷,灯光杂彩,箫鼓声喧,十分热闹。游人见一对纱灯引道,一簇男女过来,皆披红垂绿,以为出于公侯之家,莫敢仰视,都躲路而行。那宋蕙莲一回叫:“姑夫,你放个桶子花我瞧。”一回又道:“姑夫,你放个元宵炮丈我听。”一回又落了花翠,拾花翠;一回又吊了鞋,扶着人且兜鞋;左来右去,只和敬济嘲戏。玉楼看不上,说了两句:“如何只见你吊了鞋?”玉箫道:“他怕地下泥,套着五娘鞋穿着哩!”玉楼道:“你叫他过来我瞧,真个穿着五娘的鞋儿?”金莲道:“他昨日问我讨了一双鞋,谁知成精的狗肉,套着穿!”蕙莲抠起裙子来,与玉楼看。看见他穿着两双红鞋在脚上,用纱绿线带儿扎着裤腿,一声儿也不言语。

须臾,走过大街,到灯市里。金莲向玉楼道:“咱如今往狮子街李大姐房子里走走去。”于是分咐画童、来安儿打灯先行,迤逦往狮子街来。小厮先去打门,老冯已是歇下,房中有两个人家卖的丫头,在炕上睡。慌的老冯连忙开了门,让众妇女进来,旋戳开炉子顿茶,挈着壶往街上取酒。孟玉楼道:“老冯你且住,不要去打酒,俺们在家酒饭吃得饱饱来,你有茶,倒两瓯子来吃罢。”金莲道:“你既留人吃酒,先订下菜儿才好。”李瓶儿道:“妈妈子,一瓶两瓶取来了,打水不浑的,勾谁吃?要取一两坛儿来。”玉楼道:“他哄你,不消取,只看茶来罢。”那婆子方才不动身。李瓶儿道:“妈妈子,怎的不往那边去走走,端的在家做些甚么?”婆子道:“奶奶,你看丢下这两个业障在屋里,谁看他?”玉楼便问道:“两个丫头是谁家卖的?”婆子道:“一个是北边人家房里使女,十三岁,只要五两银子;一个是汪序班家出来的家人媳妇,家人走了,主子把\textuni{4BFC}髻打了,领出来卖,要十两银子。”玉楼道:“妈妈,我说与你,有一个人要,你赚他些银子使。”婆子道:“三娘,果然是谁要?告我说。”玉楼道:“如今你二娘房里,只元宵儿一个,不勾使,还寻大些的丫头使唤。你倒把这大的卖与他罢。”因问:“这个丫头十几岁?”婆子道:“他今年十七岁了。”说着,拿茶来,众人吃了茶。那春梅、玉箫并蕙莲都前边瞧了一遍,又到临街楼上推开窗看了一遍。陈敬济催逼说:“夜深了,看了快些家去罢。”金莲道:“怪短命,催的人手脚儿不停住,慌的是些甚么!”乃叫下春梅众人来,方才起身。冯妈妈送出门,李瓶儿因问:“平安往那去了?”婆子道:“今日这咱还没来,叫老身半夜三更开门闭户等着他。”来安儿道:“今日平安儿跟了爹往应二爹家去了。”李瓶儿分咐妈妈子:“早些关了门,睡了罢!他多也是不来,省的误了你的困头。明日早来宅里,送丫头与二娘来。你是石佛寺长老,请着你就张致了。”说毕,看着他关了大门,这一簇男女方才回家。

走到家门首,只听见住房子的韩回子老婆韩嫂儿声唤。因他男子汉答应马房内臣,他在家跟着人走百病儿去了,醉回来家,说有人挖开他房门,偷了狗,又不见了些东西,坐在当街上撒酒疯骂人。众妇人方才立住了脚。金莲使来安儿把韩嫂儿叫到当面,问道:“你为甚么来?”韩嫂儿叉手向前,拜了两拜,说道:“三位娘子在上,听小媳妇告诉。”于是从头说了一遍。玉楼众人听了,每人掏袖中些钱果子与他,叫来安儿:“你叫你陈姐夫送他进屋里。”那敬济且顾和蕙莲两个嘲戏,不肯搊他去。金莲使来安儿扶到他家中,分咐教他明日早来宅内浆洗衣裳:“我对你爹说,替你出气。”那韩嫂儿千恩万谢回家去了。

玉楼等刚走过门首来,只见贲四娘子,在大门首笑嘻嘻向前道了万福,说道:“三位娘那里走了走?请不弃到寒家献茶。”玉楼道:“方才因韩嫂儿哭,俺站住问了他声。承嫂子厚意,天晚了,不到罢。”贲四娘子道:“耶嚛,三位娘上门怪人家,就笑话俺小家人家茶也奉不出一杯儿来?”生死拉到屋里。原来上边供养观音八难并关圣贤,当门挂着雪花灯儿一盏。掀开门帘,摆设春台,与三人坐。连忙教他十四岁女儿长姐过来,与三位娘磕头递茶。玉楼、金莲每人与了他两枝花儿。李瓶儿袖中取了一方汗巾,又是一钱银子,与他买瓜子儿磕。喜欢的贲四娘子拜谢了又拜。款留不住,玉楼等起身。到大门首,小厮来兴在门首迎接。金莲就问:“你爹来家不曾?”来兴道:“爹未回家哩。”三个妇人,还看着陈敬济在门首放了两个一丈菊和一筒大烟兰、一个金盏银台儿,才进后边去了。西门庆直至四更来家。正是:

\[
醉后不知天色暝,任他明月下西楼。
\]

却说那陈敬济因走百病,与金莲等众妇人嘲戏了一路儿,又和蕙莲两个言来语去,都有意了。次日早晨梳洗毕,也不到铺子内,迳往后边吴月娘房里来。只见李娇儿、金莲陪着吴大妗子,放炕桌儿,才摆茶吃。月娘便往佛堂中烧香去了。这小伙儿向前作了揖,坐下。金莲便说道:“陈姐夫,你好人儿!昨日教你送送韩嫂儿,你就不动,只当还教小厮送去了。且和媳妇子打牙犯嘴,不知甚么张致!等你大娘烧了香来,看我对他说不说!”敬济道:“你老人家还说哩,昨日险些儿子腰梁㿚疡了哩!跟你老人家走了一路儿,又到狮子街房里回来,该多少里地?人辛苦走了,还教我送韩回子老婆!教小厮送送也罢了。睡了多大回就天晓了,今早还扒不起来。”正说着,吴月娘烧了香来,敬济作了揖。月娘便问:“昨日韩嫂儿为甚么撒酒疯骂人?”敬济把因走百病,被人挖开门,不见了狗,坐在当街哭喊骂人,“今早他汉子来家,一顿好打的,这咱还没起来哩。”金莲道:“不是俺们回来,劝的他进去了,一时你爹来家撞见,甚么样子!”说毕,玉楼、李瓶儿、大姐都到月娘屋里吃茶,敬济也陪着吃了茶。后次大姐回房,骂敬济:“不知死的囚根子!平白和来旺媳妇子打牙犯嘴,倘忽一时传的爹知道了,淫妇便没事,你死也没处死!”

却说那日,西门庆在李瓶儿房里宿歇,起来的迟。只见荆千户——新升一处兵马都监——来拜。西门庆才起来梳头,包网巾,整衣出来,陪荆都监在厅上说话。一面使平安儿进后边要茶。宋蕙莲正和玉箫、小玉在后边院子里挝子儿,赌打瓜子,顽成一块。那小玉把玉箫骑在底下,笑骂道:“贼淫妇,输了瓜子,不教我打!”因叫蕙莲:“嫂子你过来,扯着淫妇一只腿,等我\textuni{34B2}这淫妇一下子。”正顽着,只见平安走来,叫:“玉箫姐,前边荆老爹来,使我进来要茶哩。”那玉箫也不理他,且和小玉厮打顽耍。那平安儿只顾催逼说:“人坐下这一日了。”宋蕙莲道:“怪囚根子,爹要茶,问厨房里上灶的要去,如何只在俺这里缠?俺这后边只是预备爹娘房里用的茶,不管你外边的帐。”那平安儿走到厨房下。那日该来保妻蕙祥,蕙祥道:“怪囚,我这里使着手做饭,你问后边要两锺茶出去就是了,巴巴来问我要茶!”平安道:“我到后头来,后边不打发茶。蕙莲嫂子说,该是上灶的首尾。”蕙祥便骂道:“贼淫妇,他认定了他是爹娘房里人,俺天生是上灶的来?我这里又做大家伙里饭,又替大妗子炒素菜,几只手?论起就倒倒茶儿去也罢了,巴巴坐名儿来寻上灶的,上灶的是你叫的?误了茶也罢,我偏不打发上去。”平安儿道:“荆老爹来了这一日,嫂子快些打发茶,我拿上去罢。迟了又惹爹骂!”

当下这里推那里,那里推这里,就耽误了半日。比及又等玉箫取茶果、茶匙儿出来,平安儿拿茶出去,那荆都监坐的久了,再三要起身,被西门庆留住。嫌茶冷不好吃,喝骂平安另换茶上去吃了,荆都监才起身去了。西门庆进来,问:“今日茶是谁顿的?”平安道:“是灶上顿的茶。”西门庆回到上房,告诉月娘:“今日顿这样茶出去,你往厨下查那个奴才老婆上灶?采出来问他,打与他几下。”小玉道:“今日该蕙祥上灶。”慌的月娘说道:“这歪剌骨待死!越发顿恁样茶上去了。”一面使小玉叫将蕙祥当院子跪着,问他要打多少。蕙祥答道:“因做饭,炒大妗子素菜,使着手,茶略冷了些。”被月娘数骂了一回,饶了他起来。分咐:“今后但凡你爹前边人来,教玉箫和蕙莲后边顿茶,灶上只管大家茶饭。”

这蕙祥在厨下忍气不过,刚等的西门庆出去了,气狠狠走来后边,寻着蕙莲,指着大骂:“贼淫妇,趁了你的心了!罢了,你天生的就是有时运的爹娘房里人,俺们是上灶的老婆来?巴巴使小厮坐名问上灶要茶,上灶的是你叫的?你识我见的,促织不吃癞蛤蟆肉——都是一锹土上人。你恒数不是爹的小老婆就罢了。就是爹的小老婆,我也不怕你!”蕙莲道:“你好没要紧,你顿的茶不好,爹嫌你,管我甚事?你如何拿人撒气?”蕙祥听了,越发恼了,骂道:“贼淫妇!你刚才调唆打我几棍儿好来,怎的不教打我?你在蔡家养的汉数不了,来这里还弄鬼哩!”蕙莲道:“我养汉,你看见来?没的扯臊淡哩!嫂子,你也不是甚么清净姑姑儿!”蕙祥道:“我怎不是清净姑姑儿?跷起脚儿来,比你这淫妇好些儿。你汉子有一拿小米数儿!你在外边,那个不吃你嘲过?你背地干的那营生儿,只说人不知道。你把娘们还放不到心上,何况以下的人!”蕙莲道:“我背地里说甚么来?怎的放不到心上?随你压我,我不怕你!”蕙祥道:“有人与你做主儿,你可知不怕哩!”两个正拌嘴,被小玉请的月娘来,把两个都喝开了:“贼臭肉们,不干那营生去,都拌的是些甚么?教你主子听见又是一场儿。头里不曾打的成,等住回却打的成了!”蕙祥道:“若打我一下儿,我不把淫妇口里肠勾了也不算!我拚着这命,摈兑了你也不差厮甚么。咱大家都离了这门罢!”说着往前去了。后次这宋蕙莲越发猖狂起来,仗西门庆背地和他勾搭,把家中大小都看不到眼里,逐日与玉楼、金莲、李瓶儿、西门大姐、春梅在一处顽耍。

那日冯妈妈送了丫头来,约十三岁,先到李瓶儿房里看了,送到李娇儿房里。李娇儿用五两银子买下,房中伏侍,不在话下。正是:

\[
外作禽荒内色荒,连沾些子又何妨。
早晨跨得雕鞍去,日暮归来红粉香。
\]

\newpage
%# -*- coding:utf-8 -*-
%%%%%%%%%%%%%%%%%%%%%%%%%%%%%%%%%%%%%%%%%%%%%%%%%%%%%%%%%%%%%%%%%%%%%%%%%%%%%%%%%%%%%


\chapter{吴月娘春昼秋千\KG 来旺儿醉中谤仙}


词曰:

\[
蹴罢秋千,起来整顿纤纤手。露浓花瘦,薄汗轻衣透。见客入来,袜刬金钗溜。和羞走,倚门回首,却把青梅嗅。
\]

话说灯节已过,又早清明将至。西门庆有应伯爵早来邀请,说孙寡嘴作东,邀了郊外耍子去了。

先是吴月娘花园中,扎了一架秋千。这日见西门庆不在家,闲中率众姊妹游戏,以消春困。先是月娘与孟玉楼打了一回,下来教李娇儿和潘金莲打。李娇儿辞说身体沉重,打不的,却教李瓶儿和金莲打。打了一回,玉楼便叫:“六姐过来,我和你两个打个立秋千。”分咐:“休要笑。”当下两个玉手挽定彩绳,将身立于画板之上。月娘却教蕙莲、春梅两个相送。正是:

\[
红粉面对红粉面,玉酥肩并玉酥肩。
两双玉腕挽复挽,四只金莲颠倒颠。
\]

那金莲在上面笑成一块。月娘道:“六姐你在上头笑不打紧,只怕一时滑倒,不是耍处。”说着,不想那画板滑,又是高底鞋,跐不牢,只听得滑浪一声把金莲擦下来,早是扶住架子不曾跌着,险些没把玉楼也拖下来。月娘道:“我说六姐笑的不好,只当跌下来。”因望李娇儿众人说道:“这打秋千,最不该笑。笑多了,一定腿软了,跌下来。咱在家做女儿时,隔壁周台官家花园中扎着一座秋千。也是三月佳节,一日他家周小姐和俺一般三四个女孩儿,都打秋千耍子,也是这等笑的不了,把周小姐滑下来,骑在画板上,把身子喜抓去了。落后嫁与人家,被人家说不是女儿,休逐来家,今后打秋千,先要忌笑。”金莲道:“孟三儿不济,等我和李大姐打个立秋千。”月娘道:“你两个仔细打。”却教玉箫、春梅在旁推送。才待打时,只见陈敬济自外来,说道:“你每在这里打秋千哩。”月娘道:“姐夫来的正好,且来替你二位娘送送儿。丫头每气力少。”这敬济老和尚不撞钟——得不的一声,于是拨步撩衣,向前说:“等我送二位娘。”先把金莲裙子带住,说道:“五娘站牢,儿子送也。”那秋千飞在半空中,犹若飞仙相似。李瓶儿见秋千起去了,唬的上面怪叫道:“不好了,姐夫你也来送我送儿。”敬济道:“你老人家到且性急,也等我慢慢儿的打发将来。这里叫,那里叫,把儿子手脚都弄慌了。”于是把李瓶儿裙子掀起,露着他大红底衣,推了一把。李瓶儿道:“姐夫,慢慢着些!我腿软了!”敬济道:“你老人家原来吃不得紧酒。”金莲又说:“李大姐,把我裙子又兜住了。”两个打到半中腰里,都下来了。却是春梅和西门大姐两个打了一回。然后,教玉箫和蕙莲两个打立秋千。这蕙莲手挽彩绳,身子站的直屡屡的,脚跐定下边画板,也不用人推送,那秋千飞在半天云里,然后忽地飞将下来,端的却是飞仙一般,甚可人爱。月娘看见,对玉楼、李瓶儿说:“你看媳妇子,他倒会打。”这里月娘众人打秋千不题。

话分两头。却表来旺儿往杭州织造蔡太师生辰衣服回来,押着许多驮垛箱笼船上,先走来家。到门首,下了头口,收卸了行李,进到后边。只见雪娥正在堂屋门首,作了揖。那雪娥满面微笑,说道:“好呀,你来家了。路上风霜,多有辛苦!几时没见,吃得黑胖了。”来旺因问:“爹娘在那里?”雪娥道:“你爹今日被应二众人,邀去门外耍子去了。你大娘和大姐,都在花园中打秋千哩。”来旺儿道:“啊呀,打他则甚?”雪娥便倒了一盏茶与他吃,因问:“媳妇子在灶上,怎的不见?”那雪娥冷笑了一声,说道:“你的媳妇子,如今还是那时的媳妇儿哩?好不大了!他每日只跟着他娘每伙儿里下棋,挝子儿,抹牌顽耍。他肯在灶上做活哩!”正说着,小玉走到花园中,报与月娘。月娘自前边走来,来旺儿向前磕了头,立在旁边。问了些路上往回的话,月娘赏了两瓶酒。吃一回,他媳妇宋蕙莲来到。月娘道:“也罢,你辛苦了,且往房里洗洗头面,歇宿歇宿去。等你爹来,好见你爹回话。”那来旺儿便归房里。蕙莲先付钥匙开了门,又舀些水与他洗脸摊尘,收拾褡裢去,说道:“贼黑囚,几时没见,便吃得这等肥肥的。”又替他换了衣裳,安排饭食与他吃。睡了一觉起来,已是日西时分。

西门庆来家,来旺儿走到跟前参见,说道:“杭州织造蔡太师生辰的尺头并家中衣服,俱已完备,打成包裹,装了四箱,搭在官船上来家,只少雇夫过税。”西门庆满心欢喜,与了他赶脚银两,明日早装载进城。又赏银五两,房中盘缠;又教他管买办东西。这来旺儿私已带了些人事,悄悄送了孙雪娥两方绫汗巾,两只装花膝裤,四匣杭州粉,二十个胭脂。雪娥背地告诉来旺儿说:“自从你去了四个月,你媳妇怎的和西门庆勾搭,玉箫怎的做牵头,金莲屋里怎的做窝窠。先在山子底下,落后在屋里,成日明睡到夜,夜睡到明。与他的衣服、首饰、花翠、银钱,大包带在身边。使小厮在门首买东西,见一日也使二三钱银子。”来旺道:“怪道箱子里放着衣服、首饰!我问他,他说娘与他的。”雪娥道:“那娘与他?到是爷与他的哩!”这来旺儿遂听记在心。

到晚夕,吃了几锺酒,归到房中。常言酒发顿腹之言,因开箱子,看见一匹蓝缎子,甚是花样奇异,便问老婆:“是那里的缎子?谁人与你的?趁上实说。”老婆不知就里,故意笑着,回道:“怪贼囚,问怎的?此是后边见我没个袄儿,与了这匹缎子,放在箱中,没工夫做。端的谁肯与我?”来旺儿骂道:“贼淫妇!还捣鬼哩!端的是那个与你的?”又问:“这些首饰是那里的?”妇人道:“呸!怪囚根子,那个没个娘老子,就是石头罅剌儿里迸出来,也有个窝巢儿,为人就没个亲戚六眷?此是我姨娘家借来的钗梳。是谁与我的!”被来旺儿一拳,险不打了一交,说:“贼淫妇,还说嘴哩!有人亲看见你和那没人伦的猪狗有首尾!玉箫丫头怎的牵头,送缎子与你,在前边花园内两个干,落后吊在潘家那淫妇屋里明干,成日\textuni{34B2}的不值了。贼淫妇,你还要我手里吊子曰儿。”那妇人便大哭起来,说道:“贼不逢好死的囚根子!你做甚么来家打我?我干坏了你甚么事来?你恁是言不是语,丢块砖瓦儿也要个下落。是那个嚼舌根的,没空生有,调唆你来欺负老娘?我老娘不是那没根基的货!教人就欺负死,也拣个干净地方。你问声儿,宋家的丫头,若把脚略趄儿,把‘宋’字儿倒过来!你这贼囚根子,得不个风儿就雨儿。万物也要个实。人教你杀那个人,你就杀那个人?”几句说的来旺儿不言语了。妇人又道:“这匹蓝缎子,越发我和你说了罢,也是去年十一月里三娘生日,娘见我上穿着紫袄,下边借了玉箫的裙子穿着,说道:‘媳妇子怪剌剌的,甚么样子?’才与了我这匹缎子。谁得闲做他?那个是不知道!就纂我恁一遍舌头。你错认了老娘,老娘不是个饶人的。明日我咒骂个样儿与他听。破着我一条性命,自恁寻不着主儿哩。”来旺儿道:“你既没此事,平白和人合甚气?快些打铺我睡。”这妇人一面把铺伸下,说道:“怪倒路的囚根子,噇了那黄汤,挺你那觉!平白惹老娘骂。”把来旺掠翻在炕上,鼾声如雷。看官听说:但凡世上养汉的婆娘,饶他男子汉十八分精细,吃他几句左话儿右说,十个九个都着了道儿。正是:东净里砖儿——又臭又硬。

这宋蕙莲窝盘住来旺儿,过了一宿。到次日,往后边问玉箫,谁人透露此事,终莫知其所由,只顾海骂。一日,月娘使小玉叫雪娥,一地里寻不着。走到前边,只见雪娥从来旺儿房里出来,只猜和他媳妇说话,不想走到厨下,蕙莲又在里面切肉,良久,西门庆前边陪着乔大户说话,只为扬州盐商王四峰,被按抚使送监在狱中,许银二千两,央西门庆对蔡太师讨人情释放。刚打发大户去了,西门庆叫来旺,来旺从他屋里跑出来。正是:

\[
雪隐鹭莺飞始见,柳藏鹦鹉语方知。
\]
以此都知雪娥与来旺儿有尾首。

一日,来旺儿吃醉了,和一般家人小厮在前边恨骂西门庆,说怎的我不在家,使玉箫丫头拿一匹蓝缎子,在房里哄我老婆。把他吊在花园奸耍,后来潘金莲怎的做窝主:“由他,只休要撞到我手里。我教他白刀子进去,红刀子出来。好不好,把潘家那淫妇也杀了,也只是个死。你看我说出来做的出来。潘家那淫妇,想着他在家摆死了他汉子武大,他小叔武松来告状,多亏了谁替他上东京打点,把武松垫发充军去了?今日两脚踏住平川路,落得他受用,还挑拨我的老婆养汉。我的仇恨,与他结的有天来大。常言道,一不做,二不休,到跟前再说话。破着一命剐,便把皇帝打!”这来旺儿自知路上说话,不知草里有人,不想被同行家人来兴儿听见。这来兴儿在家,西门庆原派他买办食用撰钱过日,只因与来旺媳妇勾搭,把买办夺了,却教来旺儿管领。来兴儿就与来旺不睦,听见发此言语,就悄悄走来潘金莲房里告诉。

金莲正和孟玉楼一处坐的,只见来兴儿掀帘子进来,金莲便问来兴儿:“你来有甚事?你爹今日往谁家吃酒去了?”来兴道:“今日俺爹和应二爹往门外送殡去了。适有一件事,告诉老人家,只放在心里,休说是小的来说。”金莲道:“你有甚事,只顾说,不妨事!”来兴儿道:“别无甚事,叵耐来旺儿,昨日不知那里吃的醉稀稀的,在前边大吆小喝,指猪骂狗,骂了一日。又逻着小的厮打,小的走来一边不理,他对着家中大小,又骂爹和五娘。”潘金莲就问:“贼囚根子,骂我怎的?”来兴说:“小的不敢说。三娘在这里,也不是别人。那厮说爹怎的打发他不在家,耍了他的老婆,说五娘怎的做窝主,赚他老婆在房里和爹两个明睡到夜,夜睡到明。他打下刀子,要杀爹和五娘,白刀子进去,红刀子出来。又说,五娘那咱在家,毒药摆杀了亲夫,多亏了他上东京去打点,救了五娘一命。说五娘恩将仇报,挑拨他老婆养汉。小的穿青衣抱黑住,先来告诉五娘说声,早晚休吃那厮暗算。”玉楼听了,如提在冷水盆内一般,吃了一惊。这金莲不听便罢,听了,粉面通红,银牙咬碎,骂道:“这犯死的奴才!我与他往日无冤近日无仇,他主子要了他的老婆,他怎的缠我?我若教这奴才在西门庆家,永不算老婆!怎的我亏他救活了性命?”因分咐来兴儿:“你且去,等你爹来家问你时,你也只照恁般说。”来兴儿说:“五娘说那里话!小的又不赖他,有一句说一句。随爹怎的问,也只是这等说。”说毕,往前边去了。

玉楼便问金莲:“真个他爹和这媳妇子有?”金莲道:“你问那没廉耻的货!甚的好老婆,也不枉了教奴才这般挟制了。在人家使过了的奴才淫妇,当初在蔡通判家,和大婆作弊养汉,坏了事,才打发出来,嫁了蒋聪。岂止见过一个汉子儿?有一拿小米数儿,甚么事儿不知道!贼强人瞒神吓鬼,使玉箫送缎子儿与他做袄儿穿。一冬里,我要告诉你,没告诉你。那一日,大姐姐往乔大户家吃酒,咱每都不在前边下棋?只见丫头说他爹来家,咱每不散了?落后我走到后边仪门首,见小玉立在穿廊下,我问他,小玉望着我摇手儿。我刚走到花园前,只见玉箫那狗肉在角门首站立,原来替他观风。我还不知,教我径往花园里走。玉箫拦着我,不教我进去,说爹在里面。教我骂了两句。我到疑影和他有些甚么查子帐,不想走到里面,他和媳妇子在山洞里干营生。媳妇子见我进去,把脸飞红的走出来了。他爹见了我,讪讪的,吃我骂了两句没廉耻。落后媳妇子走到屋里,打旋磨跪着我,教我休对他娘说。落后正月里,他爹要把淫妇安托在我屋里过一夜儿,吃我和春梅折了两句,再几时容他傍个影儿!贼万杀的奴才,没的把我扯在里头。好娇态的奴才淫妇,我肯容他在那屋里头弄碜儿?就是我罢了,俺春梅那小肉儿,他也不肯容他。”玉楼道:“嗔道贼臭肉在那里坐着,见了俺每意意似似,待起不起的,谁知原来背地有这本帐!论起来,他爹也不该要他。那里寻不出老婆来,教奴才在外边倡扬,甚么样子?”金莲道:“左右的皮靴儿没番正,你要奴才老婆,奴才暗地里偷你的小娘子,彼此换着做!贼小妇奴才,千也嘴头子嚼说人,万也嚼说,今日打了嘴,也不说的!”玉楼向金莲道:“这椿事,咱对他爹说好,不说好?大姐姐又不管。倘忽那厮真个安心,咱每不言语,他爹又不知道,一时遭了他手怎了?六姐,你还该说说。”金莲道:“我若是饶了这奴才,除非是他\textuni{34B2}出我来。”正是:

\[
平生不作皱眉事,世上应无切齿人。
\]

西门庆至晚来家,只见金莲在房中云鬟不整,睡揾香腮,哭的眼坏坏的。问其所以,遂把来旺儿醉酒发言,要杀主之事诉说一遍:“见有来兴儿亲自听见,思想起来,你背地图他老婆,他便背地要你家小娘子。你的皮靴儿没番正。那厮杀你便该当,与我何干?连我一例也要杀!趁早不为之计,夜头早晚,人无后眼,只怕暗遭他毒手。”西门庆因问:“谁和那厮有首尾?”金莲道:“你休来问我,只问小玉便知。”又说:“这奴才欺负我,不是一遭儿了。说我当初怎的用药摆杀汉子,你娶了我来,亏他寻人情搭救我性命来。在外边对人揭条。早是奴没生下儿没长下女,若是生下儿女,教贼奴才揭条着好听?敢说:‘你家娘当初在家不得地时,也亏我寻人情救了他性命。’恁说在你脸上也无光了!你便没羞耻,我却成不的,要这命做甚么?”西门庆听了妇人之言,走到前边,叫将来兴儿到无人处,问他始末缘由。这小厮一五一十说了一遍。又走到后边,摘问了小玉口词,与金莲所说无差:委的某日,亲眼看见雪娥从来旺儿屋里出来,他媳妇儿不在屋里,的有此事。这西门庆心中大怒,把孙雪娥打了一顿,被月娘再三劝了,拘了他头面衣服,只教他伴着家人媳妇上灶,不许他见人。此事表过不题。

西门庆在后边,因使玉箫叫了宋蕙莲,背地亲自问他。这婆娘便道:“啊呀,爹,你老人家没的说,他是没有这个话。我就替他赌了大誓。他酒便吃两锺,敢恁七个头八个胆,背地里骂爹?又吃纣王水土,又说纣王无道!他靠那里过日子?爹,你不要听人言语。我且问爹,听见谁说这个话来?”那西门庆被婆娘一席话儿,闭口无言。问的急了,说:“是来兴儿告诉我说的。”蕙莲道:“来兴儿因爹叫俺这一个买办,说俺每夺了他的,不得赚些钱使,结下这仇恨儿,平空拿这血口喷他,爹就信了。他有这个欺心的事,我也不饶他。爹你依我,不要教他在家里,与他几两银子本钱,教他信信脱脱,远离他乡,做买卖去。他出去了,早晚爹和我说句话儿也方便些。”西门庆听了满心欢喜,说道:“我的儿,说的是。我有心要叫他上东京,与盐商王四峰央蔡太师人情,回来,还要押送生辰担去,只因他才从杭州来家,不好又使他的,打帐叫来保去。既你这样说,我明日打发他去便了。回来,我教他领一千两银子,同主管往杭州贩买绸绢丝线做买卖。你意下如何?”老婆心中大喜,说道:“爹若这等才好。”正说着,西门庆见无人,就搂他过来亲嘴。婆娘忙递舌头在他口里,两个咂做一处。妇人道:“爹,你许我编\textuni{4BFC}髻,怎的还不替我编?恁时候不戴到几时戴?只教我成日戴这头发壳子儿?”西门庆道:“不打紧,到明日将八两银子,往银匠家替你拔丝去。”西门庆又道:“怕你大娘问,怎生回答?”妇人道:“不打紧,我自有话打发他,只说问我姨娘家借来戴戴,怕怎的?”当下二人说了一回话,各自分散了。

到了次日,西门庆在厅上坐着,叫过来旺儿来:“你收拾衣服行李,赶明日三月二十八日起身,往东京央蔡太师人情。回来,我还打发你杭州做买卖去。”这来旺心中大喜,应诺下来,回房收拾行李,在外买人事。来兴儿打听得知,就来告报金莲知道。金莲打听西门庆在花园卷棚内,走到那里,不见西门庆,只见陈敬济在那里封礼物。金莲便道:“你爹在那里?你封的是甚么?”敬济道:“爹刚才在这里,往大娘那边兑盐商王四峰银子去了。我封的是往东京央蔡太师的礼。”金莲问:“打发谁去?”敬济道:“我听见昨日爹分咐来旺儿去。”这金莲才待下台基,往花园那条路上走,正撞见西门庆拿了银子来。叫到屋里,问他:“明日打发谁往东京去?”西门庆道:“来旺儿和吴主管二人同去。因有盐商王四峰一千干事的银两,以此多着两个去。”妇人道:“随你心下,我说的话儿你不依,到听那奴才淫妇一面儿言语。他随问怎的,只护他的汉子。那奴才有话在先,不是一日儿了。左右破着老婆丢与你,坑了你这银子,拐的往那头里停停脱脱去了,看哥哥两眼儿空哩。你的白丢了罢了,难为人家一千两银子,不怕你不赔他。我说在你心里,也随你。老婆无故只是为他。不争你贪他这老婆,你留他在家里也不好,你就打发他出去做买卖也不好。你留他在家里,早晚没这些眼防范他。你打发他外边去,他使了你本钱,头一件你先说不得他。你若要他这奴才老婆,不如先把奴才打发他离门离户。常言道:剪草不除根,萌芽依旧生;剪草若除根,萌芽再不生。就是你也不耽心,老婆他也死心塌地。”一席话儿,说得西门庆如醉方醒。正是:

\[
数语拨开君子路,片言提醒梦中人。
\]

\newpage
%# -*- coding:utf-8 -*-
%%%%%%%%%%%%%%%%%%%%%%%%%%%%%%%%%%%%%%%%%%%%%%%%%%%%%%%%%%%%%%%%%%%%%%%%%%%%%%%%%%%%%


\chapter{来旺儿递解徐州\KG 宋蕙莲含羞自缢}


诗曰:

\[
与君形影分吴越,玉枕经年对离别。
登台北望烟雨深,回身哭向天边月。
\]

又:

\[
夜深闷到戟门边,却绕行廊又独眠。
闺中只是空相忆,魂归漠漠魄归泉。
\]

话说西门庆听了金莲之言,又变了卦。到次日,那来旺儿收拾行李伺候,到日中还不见动静。只见西门庆出来,叫来旺儿到跟前说道:“我夜间想来,你才打杭州来家多少时儿,又教你往东京去,忒辛苦了,不如叫来保替你去罢。你且在家歇宿几日,我到明日,家门首生意寻一个与你做罢。”自古物听主裁,那来旺儿那里敢说甚的,只得应诺下来。西门庆就把银两书信,交付与来保和吴主管,三月念八日起身往东京去了。不在话下。

这来旺儿回到房中,心中大怒,吃酒醉倒房中,口内胡说,怒起宋蕙莲来,要杀西门庆。被宋蕙莲骂了他几句:“你咬人的狗儿不露齿,是言不是语,墙有缝,壁有耳。噇了那黄汤,挺那两觉。”打发他上床睡了。到次日,走到后边,串玉箫房里请出西门庆。两个在厨房后墙底下僻静处说话,玉箫在后门首替他观风。婆娘甚是埋怨,说道:“你是个人?你原说教他去,怎么转了靶子,又教别人去?你干净是个毬子心肠——滚上滚下,灯草拐棒儿——原拄不定把。你到明日盖个庙儿,立起个旗杆来,就是个谎神爷!我再不信你说话了。我那等和你说了一场,就没些情分儿!”西门庆笑道:“到不是此说。我不是也叫他去,恐怕他东京蔡太师府中不熟,所以教来保去了。留下他,家门首寻个买卖与他做罢!”妇人道:“你对我说,寻个甚么买卖与他做?”西门庆道:“我教他搭个主管,在家门首开酒店。”妇人听言满心欢喜,走到屋里一五一十对来旺儿说了,单等西门庆示下。

一日,西门庆在前厅坐下,着人叫来旺儿近前,桌上放下六包银两,说道:“孩儿!你一向杭州来家辛苦。教你往东京去,恐怕你蔡府中不十分熟,所以教来保去了。今日这六包银子三百两,你拿去搭上个主管,在家门首开酒店,月间寻些利息孝顺我,也是好处。”那来旺连忙趴在地下磕头,领了六包银两。回到房中,告与老婆说:“他倒拿买卖来窝盘我,今日与了我这三百两银子,教我搭主管,开酒店做买卖。”老婆道:“怪贼黑囚!你还嗔老婆说。一锹就掘了井?也等慢慢来。如何今日也做上买卖了!你安分守己,休再吃了酒,口里六说白道!”来旺儿叫老婆把银两收在箱中:“我在街上寻伙计去也!”于是走到街上寻主管。寻到天晚,主管也不成,又吃的大醉来家。老婆打发他睡了,就被玉箫走来,叫到后边去了。

来旺儿睡了一觉,约一更天气,酒还未醒,正朦朦胧胧睡着,忽听的窗外隐隐有人叫他道:“来旺哥!还不起来看看,你的媳妇子又被那没廉耻的勾引到花园后边,干那营生去了。亏你倒睡的放心!”来旺儿猛可惊醒,睁开眼看看,不见老婆在房里,只认是雪娥看见甚动静来递信与他,不觉怒从心上起,道:“我在面前就弄鬼儿!”忙跳起身来,开了房门,迳扑到花园中来。刚到厢房中角门首,不防黑影里抛出一条凳子来,把来旺儿绊了一交,只见响亮一声,一把刀子落地。左右闪过四五个小厮,大叫:“有贼!”一齐向前,把来旺儿一把捉住了。来旺儿道:“我是来旺儿,进来寻媳妇子,如何把我拿住了?”众人不由分说,一步一棍,打到厅上。只见大厅上灯烛荧煌,西门庆坐在上面,即叫:“拿上来!”来旺儿跪在地下,说道:“小的睡醒了,不见媳妇在房里,进来寻他。如何把小的做贼拿?”那来兴儿就把刀子放在面前,与西门庆看。西门庆大怒,骂道:“众生好度人难度,这厮真是个杀人贼!我倒见你杭州来家,叫你领三百两银子做买卖,如何夤夜进内来要杀我?不然拿这刀子做甚么?”喝令左右:“与我押到他房中,取我那三百两银子来!”众小厮随即押到房中。蕙莲正在后边同玉箫说话,忽闻此信,忙跑到房里。看见了,放声大哭,说道:“你好好吃了酒睡罢,平白又来寻我做甚么?只当暗中了人的拖刀之计。”一面开箱子,取出六包银子来,拿到厅上。西门庆灯下打开观看,内中止有一包银两,余者都是锡铅锭子。西门庆大怒,因问:“如何抵换了!我的银两往那里去了?趁早实说!”那来旺儿哭道:“爹抬举小的做买卖,小的怎敢欺心抵换银两?”西门庆道:“你打下刀子,还要杀我。刀子现在,还要支吾甚么?”因把来兴儿叫来,面前跪下,执证说:“你从某日,没曾在外对众发言要杀爹,嗔爹不与你买卖做?”这来旺儿只是叹气,张开口儿合不的。西门庆道:“既赃证刀杖明白,叫小厮与我拴锁在门房内。明日写状子,送到提刑所去!”只见宋蕙莲云鬟撩乱,衣裙不整,走来厅上向西门庆跪下,说道:“爹,此是你干的营生!他好好进来寻我,怎把他当贼拿了?你的六包银子,我收着,原封儿不动,平白怎的抵换了?恁活埋人,也要天理。他为甚么?你只因他甚么?打与他一顿。如今拉着送他那里去?”西门庆见了他,回嗔作喜道:“媳妇儿,关你甚事?你起来。他无礼胆大不是一日,见藏着刀子要杀我,你不得知道。你自安心,没你之事。”因令来安儿:“好搀扶你嫂子回房去,休要慌吓他。”那蕙莲只顾跪着不起来,说:“爹好狠心!你不看僧面看佛面,我恁说着,你就不依依儿?他虽故吃酒,并无此事。”缠得西门庆急了,教来安儿搊他起来,劝他回房去了。

到天明,西门庆写了柬帖,叫来兴儿做干证,揣着状子,押着来旺儿往提刑院去,说某日酒醉,持刀夤夜杀害家主,又抵换银两等情。才待出门,只见吴月娘走到前厅,向西门庆再三将言劝解,说道:“奴才无礼,家中处分他便了。又要拉出去,惊官动府做甚么?”西门庆听言,圆睁二目,喝道:“你妇人家,不晓道理!奴才安心要杀我,你倒还教饶他罢!”于是不听月娘之言,喝令左右把来旺儿押送提刑院去了。月娘当下羞赧而退,回到后边,向玉楼众人说道:“如今这屋里乱世为王,九尾狐狸精出世。不知听信了甚么人言语,平白把小厮弄出去了。你就赖他做贼,万物也要个着实才好,拿纸棺材糊人,成何道理?恁没道理昏君行货!”宋蕙莲跪在当面哭泣。月娘道:“孩儿你起来,不消哭。你汉子恒数问不的他死罪。贼强人,他吃了迷魂汤了,俺们说话不中听,老婆当军——充数儿罢了。”玉楼向蕙莲道:“你爹正在个气头上,待后慢慢的俺每再劝他。你安心回房去罢。”按下这里不提。

单表来旺儿押到提刑院,西门庆先差玳安送了一百石白米与夏提刑、贺千户。二人受了礼物,然后坐厅。来兴儿递上呈状,看了,已知来旺儿先因领银做买卖,见财起意,抵换银两,恐家主查算,夤夜持刀突入后厅,谋杀家主等情。心中大怒,把来旺叫到当厅跪下。这来旺儿告道:“望天官爷察情!容小的说,小的便说;不容小的说,小的不敢说。”夏提刑道:“你这厮!见获赃证明白,勿得推调,从实与我说来,免我动刑。”来旺儿悉把西门庆初时令某人将蓝缎子,怎的调戏他媳妇儿宋氏成奸,如今故入此罪,要垫害图霸妻子一节,诉说一遍。夏提刑大喝了一声,令左右打嘴巴,说:“你这奴才欺心背主!你这媳妇也是你家主娶的配与你为妻,又把资本与你做买卖,你不思报本,却倚醉夤夜突入卧房,持刀杀害。满天下人都象你这奴才,也不敢使人了。”来旺儿口还叫冤屈,被夏提刑叫过来兴儿过来执证。那来旺儿有口说不得了。正是:

\[
会施天上计,难免目前灾。
\]
夏提刑即令左右选大夹棍上来,把来旺儿夹了一夹,打了二十大棍,打的皮开肉绽,鲜血淋漓。分咐狱卒,带下去收监。来兴儿、钺安儿来家,回覆了西门庆话。西门庆满心欢喜,分咐家中小厮:“铺盖、饭食,一些都不许与他送进去。但打了,休来家对你嫂子说,只说衙门中一下儿也没打他,监几日便放出来。”众小厮应诺了。

这宋蕙莲自从拿了来旺儿去,头也不梳,脸也不洗,黄着脸儿,只是关闭房门哭泣,茶饭不吃。西门庆慌了,使玉箫并贲四娘子儿再三进房解劝他,说道:“你放心,爹因他吃酒狂言,监他几日,耐他性儿,不久也放他出来。”蕙莲不信,使小厮来安儿送饭进监去,回来问他,也是这般说:“哥见官,一下儿也不打。一两日就来家,教嫂子在家安心。”这蕙莲听了此言,方才不哭了。每日淡扫娥眉,薄施脂粉,出来走跳。西门庆要便来回打房门首走,老婆在檐下叫道:“房里无人,爹进来坐坐不是!”西门庆进入房里,与老婆做一处说话。西门庆哄他说道:“我儿,你放心。我看你面上,写了帖儿对官府说,也不曾打他一下儿。监他几日,耐耐他性儿,还放他出来,还叫他做买卖。”妇人搂抱着西门庆脖子,说道:“我的亲达达!你好歹看奴之面,奈何他两日,放他出来。随你教他做买卖不教他做买卖也罢,这一出来,我教他把酒断了,随你去近到远使他,他敢不去?再不你若嫌不自便,替他寻上个老婆,他也罢了。我常远不是他的人了。”西门庆道:“我的心肝,你话是了。我明日买了对过乔家房,收拾三间房子与你住,搬你那里去,咱两个自在顽耍。”妇人道:“着来,亲亲!随你张主便了。”说毕,两个闭了门儿。原来妇人夏月常不穿裤儿,只单吊着两条裙子,遇见西门庆在那里,便掀开裙子就干。于是二人解佩露甄妃之玉,齐眉点汉署之香,双凫飞肩,云雨一席。妇人将身带的白银条纱挑线香袋儿——里边装着松柏儿并排草,挑着“娇香美爱”四个字,把与西门庆。喜的心中要不的,恨不的与他誓共死生,向袖中即掏出一二两银子,与他买果子吃。再三安抚他:“不消忧虑,只怕忧虑坏了你。我明日写帖子对夏大人说,就放他出来。”说了一回,西门庆恐有人来,连忙出去了。

这妇人得了西门庆此话,到后边对众丫鬟媳妇词色之间未免轻露,孟玉楼早已知道,转来告潘金莲说,他爹怎的早晚要放来旺儿出来,另替他娶一个;怎的要买对门乔家房子,把媳妇子吊到那里去,与他三间房住,又买个丫头伏侍他;与他编银丝\textuni{4BFC}髻,打头面。一五一十说了一遍:“就和你我辈一般,甚么张致!大姐姐也就不管管儿!”潘金莲不听便罢,听了时:

\[
忿气满怀无处着,双腮红上更添红。
\]

说道:“真个由他,我就不信了!今日与你说的话,我若教贼奴才淫妇,与西门庆放了第七个老婆,我不喇嘴说,就把潘字倒过来!”玉楼道:“汉子没正条的,大姐姐又不管,咱每能走不能飞,到的那些儿?”金莲道:“你也忒不长俊,要这命做甚么?活一百岁杀肉吃!他若不依我,拚着这命摈兑在他手里也不差甚么!”玉楼笑道:“我是小胆儿,不敢惹他,看你有本事和他缠。”

到晚,西门庆在花园中翡翠轩书房里坐的,正要教陈敬济来写帖子,往夏提刑处说,要放来旺儿出来。被金莲蓦地走到跟前,搭伏着书桌儿,问:“你教陈姐夫写甚么帖子?”西门庆不能隐讳,因说道:“我想把来旺儿责打与他几下,放他出来罢。”妇人止住小厮:“且不要叫陈姐夫来。”坐在旁边,因说道:“你空耽着汉子的名儿,原来是个随风倒舵、顺水推船的行货子!我那等对你说的话儿你不依,倒听那贼奴才淫妇话儿。随你怎的逐日沙糖拌蜜与他吃,他还只疼他的汉子。依你如今把那奴才放出来,你也不好要他这老婆了,教他奴才好藉口,你放在家里不荤不素,当做甚么人儿看成?待要把他做你小老婆,奴才又见在;待要说道奴才老婆,你见把他逞的恁没张致的,在人跟前上头上脸有些样儿!就算另替那奴才娶一个,着你要了他这老婆,往后倘忽你两个坐在一答里,那奴才或走来跟前回话,或做甚么,见了有个不气的?老婆见了他,站起来是,不站起来是?先不先,只这个就不雅相。传出去,休说六邻亲戚笑话,只家中大小,把你也不着在意里。正是上梁不正下梁歪。你既要干这营生,不如一狠二狠,把奴才结果了,你就搂着他老婆也放心。”几句又把西门庆念翻转了,反又写帖子送与夏提刑,教夏提刑限三日提出来,一顿拷打,拷打的通不象模样。提刑两位官并上下观察、缉捕、排军,监狱中上下,都受了西门庆财物,只要重不要轻。

内中有一当案的孔目阴先生,名唤阴骘,乃山西孝义县人,极是个仁慈正直之士。因见西门庆要陷害此人,图谋他妻子,再三不肯做文书送问,与提刑官抵面相讲。两位提刑官以此掣肘难行,延挨了几日,人情两尽,只把他当厅责了四十,论个递解原籍徐州为民。当查原赃,花费十七两,铅锡五包,责令西门庆家人来兴儿领回。差人写个帖子,回覆了西门庆,随教即日押发起身。这里提刑官当厅押了一道公文,差两个公人把来旺儿取出来,已是打的稀烂,钉了扭,上了封皮,限即日起程,迳往徐州管下交割。

可怜这来旺儿,在监中监了半月光景,没钱使用,弄的身体狼狈,衣服蓝褛,没处投奔。哀告两个公人说:“两位哥在上,我打了一场屈官司,身上分文没有,要凑些脚步钱与二位,望你可怜见,押我到我家主处,有我的媳妇儿并衣服箱笼,讨出来变卖了,知谢二位,并路途盘费,也讨得一步松宽。”那两个公人道:“你好不知道理!你家主既摆布了一场,他又肯发出媳妇并箱笼与你?你还有甚亲故,俺们看阴师父面上,瞒上不瞒下,领你到那里,胡乱讨些钱米,勾你路上盘费便了。谁指望你甚脚步钱儿!”来旺道:“二位哥哥,你只可怜引我先到我家主门首,我央浼两三位亲邻,替我美言讨讨儿,无多有少。”两个公人道:“也罢,我们就押你去。”这来旺儿先到应伯爵门首,伯爵推不在家。又央了左邻贾仁清、伊勉慈二人来西门庆家,替来旺儿说讨媳妇箱笼。西门庆也不出来,使出五六个小厮,一顿棍打出来,不许在门首缠扰。把贾、伊二人羞的要不的。他媳妇儿宋蕙莲,在屋里瞒的铁桶相似,并不知一字。西门庆分咐:“那个小厮走漏消息,决打二十板!”两个公人又同到他丈人——卖棺材的宋仁家,来旺儿如此这般对宋仁哭诉其事,打发了他一两银子,与两个公人一吊铜钱、一斗米,路上盘缠。哭哭啼啼,从四月初旬离了清河县,往徐州大道而来。正是:

\[
若得苟全痴性命,也甘饥饿过平生。
\]

不说来旺儿递解徐州去了。且说宋蕙莲在家,每日只盼他出来。小厮一般的替他送饭,到外边,众人都吃了。转回来蕙莲问着他,只说:“哥吃了,监中无事。若不是也放出来了,连日提刑老爷没来衙门中问事,也只在一二日来家。”西门庆又哄他说:“我差人说了,不久即出。”妇人以为信实。一日风里言风里语,闻得人说,来旺儿押出来,在门首讨衣箱,不知怎的去了。这妇人几次问众小厮,都不说。忽见钺安儿跟了西门庆马来家,叫住问他:“你旺哥在监中好么?几时出来?”钺安道:“嫂子,我告你知了罢,俺哥这早晚到流沙河了。”蕙莲问其故,这钺安千不合万不合,如此这般:“打了四十板,递解原籍徐州家去了。只放你心里,休题我告你说。”这妇人不听万事皆休,听了此言,关闭了房间,放声大哭道:“我的人嚛!你在他家干坏了甚么事来?被人纸棺材暗算计了你!你做奴才一场,好衣服没曾挣下一件在屋里。今日只当把你远离他乡,弄的去了,坑得奴好苦也!你在路上死活未知。我就如合在缸底下一般,怎的晓得?”哭了一回,取一条长手巾拴在卧房门枢上,悬梁自缢。不想来昭妻一丈青,住房正与他相连,从后来听见他屋里哭了一回,不见动静,半日只听喘息之声。扣房门叫他不应,慌了手脚,教小厮平安儿撬开窗户进去。见妇人穿着随身衣服,在门枢上正吊得好。一面解救下来,并了房门,取姜汤撅灌。须臾,嚷的后边知道。吴月娘率领李娇儿、孟玉楼、西门大姐、李瓶儿、玉箫、小玉都来看视,贲四娘子儿也来瞧。一丈青搊扶他坐在地下,只顾哽咽,白哭不出声来。月娘叫着他,只是低着头,口吐涎痰,不答应。月娘便道:“原来是个傻孩子!你有话只顾说便好,如何寻起这条路起来!”又令玉箫扶着他,亲叫道:“蕙莲孩儿,你有甚么心事,越发老实叫上几声,不妨事。”问了半日,那妇人哽咽了一回,大放声排手拍掌哭起来。月娘叫玉箫扶他上炕,他不肯上炕。月娘众人劝了半日,回后边去了。止有贲四嫂同玉箫相伴在屋里。

只见西门庆掀帘子进来,看见他坐在冷地下哭泣,令玉箫:“你搊他炕上去罢。”玉箫道:“刚才娘教他上去,他不肯去。”西门庆道:“好强孩子,冷地下冰着你。你有话对我说,如何这等拙智!”蕙莲把头摇着说道:“爹,你好人儿,你瞒着我干的好勾当儿!还说甚么孩子不孩子!你原来就是个弄人的刽子手,把人活埋惯了,害死人还看出殡的!你成日间只哄着我,今日也说放出来,明日也说放出来。只当端的好出来。你如递解他,也和我说声儿,暗暗不通风,就解发远远的去了。你也要合凭个天理!你就信着人干下这等绝户计,把圈套儿做的成成的,你还瞒着我。你就打发,两个人都打发了,如何留下我做甚么?”西门庆笑道:“孩儿,不关你事。那厮坏了事,所以打发他。你安心,我自有处。”因令玉箫:“你和贲四娘子相伴他一夜儿,我使小厮送酒来你每吃。”说毕,往外去了。贲四嫂良久扶他上炕坐的,和玉箫将话儿劝解他。

西门庆到前边铺子里,问傅伙计支了一吊钱,买了一钱酥烧,拿盒子盛了,又是一瓶酒,使来安儿送到蕙莲屋里,说道:“爹使我送这个与嫂子吃。”蕙莲看见,一头骂:“贼囚根子!趁早与我拿了去,省的我摔一地。”来安儿道:“嫂子收了罢,我拿回去,爹又要打我。”便就放在桌子上。蕙莲跳下来,把酒拿起来,才待赶着摔了去,被一丈青拦住了。那贲四嫂看着一丈青咬指头儿。正相伴他坐的,只见贲四嫂家长儿走来,叫他妈道:“爹门外头来家,要吃饭。”贲四嫂和一丈青走出来。到一丈青门首,只见西门大姐在那里,和来保儿媳妇惠祥说话。因问贲四嫂那里去,贲四嫂道:“俺家的门外头来了,要饭吃。我到家瞧瞧就来。我只说来看看,吃他大爹再三央,陪伴他坐坐儿,谁知倒把我挂住了。”惠祥道:“刚才爹在屋里,他说甚么来?”贲四嫂只顾笑,说道:“看不出他旺官娘子,原来也是个辣菜根子,和他大爹白搽白折的平上。谁家媳妇儿有这个道理!”惠祥道:“这个媳妇儿比别的媳妇儿不同,从公公身上拉下来的媳妇儿,这一家大小谁如他?”说毕惠祥去了。一丈青道:“四嫂,你到家快来。”贲四嫂道:“甚么话,我若不来,惹他大爹就怪死了。”

却说西门庆白日教贲四嫂和一丈青陪他坐,晚夕教玉箫伴他睡,慢慢将言词劝他,说道:“宋大姐,你是个聪明的,趁恁妙龄之时,一朵花初开,主子爱你,也是缘法相投。你如今将上不足,比下有余,守着主子,强如守着奴才。他已是去了,你恁烦恼不打紧,一时哭的有好歹,却不亏负了你的性命?常言道:做一日和尚撞一日钟,往后贞节轮不到你身上了。”那蕙莲听了,只是哭泣,每日粥饭也不吃。玉箫回了西门庆话。西门庆又令潘金莲亲来对他说,也不依。金莲恼了,向西门庆道:“贼淫妇,他一心只想他汉子,千也说一夜夫妻百夜恩,万也说相随百步,也有个徘徊意,这等贞节的妇人,却拿甚么拴的住他心?”西门庆笑道:“你休听他摭说,他若早有贞节之心,当初只守着厨子蒋聪不嫁来旺儿了。”一面坐在前厅上,把众小厮都叫到跟前审问:“来旺儿递解去时,是谁对他说来?趁早举出来,我也一下不打他。不然,我打听出来,每人三十板,即与我离门离户。”忽有画童跪下,说道:“那日小的听见钺安跟了爹马来家,在夹道内,嫂子问他,他走了口对嫂子说。”西门庆听了大怒,一片声使人寻钺安儿。

这钺安早知消息,一直躲到潘金莲房里去。金莲正洗脸,小厮走到屋里,跪着哭道:“五娘救小的则个!”金莲骂道:“贼囚!猛可走来,吓我一跳!你又不知干下甚么事!”钺安道:“爹因为小的告嫂子说了旺哥去了,要打我。娘好歹劝劝爹。若出去,爹在气头里,小的就是死罢了!”金莲道:“怪囚根子,唬的鬼也似的!我说甚么勾当来,恁惊天动地的?原来为那奴才淫妇。”分咐:“你在我这屋里,不要出去。”于是藏在门背后。西门庆见叫不将钺安去,在前厅暴叫如雷。一连使了两替小厮来金莲房里寻,都被金莲骂的去了。落后,西门庆一阵风自家走来,手里拿着马鞭子,问:“奴才在那里?”金莲不理他,被西门庆绕屋寻遍,从门背后采出钺安来要打。吃金莲向前,把马鞭子夺了,掠在床顶上。说道:“没廉耻的货儿,你脸做主了!那奴才淫妇想他汉子上吊,羞急拿小厮来煞气,关小厮甚事!”那西门庆气的睁睁的。金莲叫小厮:“你往前头干你那营生去,不要理他。等他再打你,有我哩!”那钺安得手,一直往前去了。正是:

\[
两手劈开生死路,翻身跳出是非门。
\]

这潘金莲见西门庆留意在宋蕙莲身上,乃心生一计。在后边唆调孙雪娥,说来旺儿媳妇子怎的说你要了他汉子,备了他一篇是非,他爹恼了,才把他汉子打发了:“前日打了你那一顿,拘了你头面衣服,都是他过嘴告说的。”这孙雪娥听了个耳满心满。掉了雪娥口气儿,走到前边,向蕙莲又是一样话说,说孙雪娥怎的后边骂你是蔡家使喝的奴才,积年转主子养汉,不是你背养主子,你家汉子怎的离了他家门?说你眼泪留着些脚后跟。说的两下都怀仇恨。

一日,也是合当有事。四月十八日,李娇儿生日,院中李妈妈并李桂姐,都来与他做生日。吴月娘留他同众堂客在后厅饮酒,西门庆往人家赴席不在家。这宋蕙莲吃了饭儿,从早晨在后边打了个幌儿,走到屋里直睡到日西。由着后边一替两替使了丫鬟来叫,只是不出来。雪娥寻不着这个由头儿,走来他房里叫他,说道:“嫂子做了玉美人了,怎的这般难请?”那蕙莲也不理他,只顾面朝里睡。这雪娥又道:“嫂子,你思想你家旺官儿哩。早思想好来!不得你他也不得死,还在西门庆家里。”这蕙莲听了他这一句话,打动潘金莲说的那情由,翻身跳起来,望雪娥说道:“你没的走来浪声颡气!他便因我弄出去了。你为甚么来?打你一顿,撵的不容上前。得人不说出来,大家将就些便罢了,何必撑着头儿来寻趁人!”这雪娥心中大怒,骂道:“好贼奴才,养汉淫妇!如何大胆骂我?”蕙莲道:“我是奴才淫妇,你是奴才小妇!我养汉养主子,强如你养奴才!你倒背地偷我汉子,你还来倒自家掀腾?”这几句话,说的雪娥急了,宋蕙莲不防,被他走向前,一个巴掌打在脸上,打的脸上通红。说道:“你如何打我?”于是一头撞将去,两个就揪扭打在一处。慌的来昭妻一丈青走来劝解,把雪娥拉的后走,两个还骂不绝口。吴月娘走来骂了两句:“你每都没些规矩儿!不管家里有人没人,都这等家反宅乱的!等你主子回来,看我对你主子说不说!”当下雪娥就往后边去了。月娘见蕙莲头发揪乱,便道:“还不快梳了头,往后边来哩!”蕙莲一声儿不答话。打发月娘后边去了,走到房内,倒插了门,哭泣不止。哭到掌灯时分,众人乱着,后边堂客吃酒,可怜这妇人忍气不过,寻了两条脚带,拴在门楹上,自缢身死,亡年二十五岁。正是:

\[
世间好物不坚牢,彩云易散琉璃脆。
\]

落后,月娘送李妈妈、桂姐出来,打蕙莲门首过,房门关着,不见动静,心中甚是疑影。打发李妈妈娘儿上轿去了,回来叫他门不开,都慌了手脚。还使小厮打窗户内跳进去,割断脚带,解卸下来,撅救了半日,不知多咱时分,呜呼哀哉死了。但见:

\[
四肢冰冷,一气灯残。香魂眇眇,已赴望乡台;星眼瞑瞑,尸犹横地下。不知精爽逝何处,疑是行云秋水中。
\]
月娘见救不活,慌了。连忙使小厮来兴儿,骑头口往门外请西门庆来家。雪娥恐怕西门庆来家拔树寻根,归罪于己,在上房打旋磨儿跪着月娘,教休题出和他嚷闹来。月娘见他吓得那等腔儿,心中又下般不得,因说道:“此时你恁害怕,当初大家省言一句儿便了。”至晚,等的西门庆来家,只说蕙莲因思想他汉子,哭了一日,赶后边人乱,不知多咱寻了自尽。西门庆便道:“他恁个拙妇,原来没福。”一面差家人递了一纸状子,报到县主李知县手里,只说本妇因本家请堂客吃酒,他管银器家伙,因失落一件银锺,恐家主查问见责,自缢身死。又送了知县三十两银子。知县自恁要作分上,胡乱差了一员司吏带领几个仵作来看了。自买了一具棺材,讨了一张红票,贲四、来兴儿同送到门外地藏寺。与了火家五钱银子,多架些柴薪。才待发火烧毁,不想他老子卖棺材宋仁打听得知,走来拦住,叫起屈来。说他女儿死的不明白,称西门庆因倚强奸他:“我女贞节不从,威逼身死。我还要抚按告状,谁敢烧化尸首!”那众火家都乱走了,不敢烧。贲四、来兴少不的把棺材停在寺里来回话。正是:

\[
青龙与白虎同行,吉凶事全然未保。
\]

\newpage
%# -*- coding:utf-8 -*-
%%%%%%%%%%%%%%%%%%%%%%%%%%%%%%%%%%%%%%%%%%%%%%%%%%%%%%%%%%%%%%%%%%%%%%%%%%%%%%%%%%%%%


\chapter{李瓶儿私语翡翠轩\KG 潘金莲醉闹葡萄架}


词曰:

\[
锦帐鸳鸯,绣衾鸾凤。一种风流千种态:看香肌双莹,玉箫暗品,鹦舌偷尝。屏掩犹斜香冷,回娇眼,盼檀郎。道千金一刻须怜惜,早漏催银箭,星沉网户,月转回廊。
\]

话说来保正从东京来,在卷棚内回西门庆话,具言:“到东京先见禀事的管家,下了书,然后引见。太师老爷看了揭帖,把礼物收进去,交付明白。老爷分咐:不日写书,马上差人下与山东巡按侯爷,把山东沧州盐客王霁云等一十二名寄监者,尽行释放。翟叔多上覆爹:老爷寿诞六月十五日,好歹教爹上京走走,他有话和爹说。”这西门庆听了,满心欢喜,旋即使他回乔大户话去。只见贲四、来兴走来,见西门庆和来保说话,立在旁边。来保便往乔大户家去了。西门庆问贲四:“你每烧了回来了?”那贲四不敢言语。来兴儿向前,附耳低言说道:“宋仁走到化人场上,拦着尸首,不容烧化,声言甚是无礼,小的不敢说。”这西门庆不听万事皆休,听了心中大怒,骂道:“这少死光棍,这等可恶!”即令小厮:“请你姐夫来写帖儿。”就差来安儿送与李知县。随即差了两个公人,一条索子把宋仁拿到县里,反问他打纲诈财,倚尸图赖。当厅一夹二十大板,打的鲜血顺腿淋漓。写了一纸供状,再不许到西门庆家缠扰。并责令地方火甲,眼同西门庆家人,即将尸烧化讫。那宋仁打的两腿棒疮,归家着了重气,害了一场时疫,不上几日,呜呼哀哉死了。正是:

\[
失晓人家逢五道,溟泠饥鬼撞钟馗。
\]

西门庆刚了毕宋蕙莲之事,就打点三百两金银,交顾银率领许多银匠,在家中卷棚内打造蔡太师上寿的四阳捧寿的银人,每一座高尺有余。又打了两把金寿字壶。寻了两副玉桃杯、两套杭州织造的大红五彩罗缎纻丝蟒衣,只少两匹玄色焦布和大红纱蟒,一地里拿银子寻不出来。李瓶儿道:“我那边楼上还有几件没裁的蟒,等我瞧去。”西门庆随即与他同往楼上去寻,拣出四件来:两件大红纱,两件玄色焦布,俱是织金莲五彩蟒衣,比织来的花样身分更强几倍,把西门庆欢喜的要不的。于是打包,还着来保同吴主管五月二十八日离清河县,上东京去了,不在话下。

过了两日,却是六月初一日,天气十分炎热。到了那赤鸟当午的时候,一轮火伞当空,无半点云翳,真乃烁石流金之际。有一词单道这热:

\[
祝融南来鞭火龙,火云焰焰烧天空。日轮当午凝不去,万国如在红炉中。五岳翠干云彩灭,阳侯海底愁波渴。何当一夕金风发,为我扫除天下热。
\]
这西门庆近来遇见天热,不曾出门,在家撒发披襟避暑。在花园中翡翠轩卷棚内,看着小厮每打水浇花草。只见翡翠轩正面栽着一盆瑞香花,开得甚是烂漫。西门庆令来安儿拿着小喷壶儿,看着浇水。只见潘金莲和李瓶儿家常都是白银条纱衫儿,密合色纱挑线缕金拖泥裙子。李瓶儿是大红焦布比甲,金莲是银红比甲。唯金莲不戴冠儿,拖着一窝子杭州撵翠云子网儿,露着四鬓,额上贴着三个翠面花儿,越显出粉面油头,朱唇皓齿。两个携着手儿,笑嘻嘻蓦地走来。看见西门庆浇花儿,说道:“你原来在这里浇花儿哩!怎的还不梳头去?”西门庆道:“你教丫头拿水来,我这里洗头罢。”金莲叫来安:“你且放下喷壶,去屋里对丫头说,教他快拿水拿梳子来。”来安应诺去了。金莲看见那瑞香花,就要摘来戴。西门庆拦住道:“怪小油嘴,趁早休动手,我每人赏你一朵罢。”原来西门庆把旁边少开头,早已摘下几朵来,浸在一只翠磁胆瓶内。金莲笑道:“我儿,你原来掐下恁几朵来放在这里,不与娘戴。”于是先抢过一枝来插在头上。西门庆递了枝与李瓶儿。只见春梅送了抿镜梳子来,秋菊拿着洗面水。西门庆递了三枝花,教送与月娘、李娇儿、孟玉楼戴:“就请你三娘来,教他弹回月琴我听。”金莲道:“你把孟三儿的拿来,等我送与他,教春梅送他大娘和李娇儿的去。回来你再把一朵花儿与我——我只替你叫唱的,也该与我一朵儿。”西门庆道:“你去,回来与你。”金莲道:“我的儿,谁养的你恁乖!你哄我替你叫了孟三儿来,你却不与我。我不去!你与了我,我才叫去。”西门庆笑道:“贼小淫妇儿,这上头也掐个先儿。”于是又与了他一朵。金莲簪于云鬓之旁,方才往后边去了。

止撇下李瓶儿,西门庆见他纱裙内罩着大红纱裤儿,日影中玲珑剔透,露出玉骨冰肌,不觉淫心辄起。见左右无人,且不梳头,把李瓶儿按在一张凉椅上,揭起湘裙,红裤初褪,倒掬着隔山取火干了半晌,精还不泄。两人曲尽“于飞”之乐。不想金莲不曾往后边叫玉楼去,走到花园角门首,想了想,把花儿递与春梅送去,回来悄悄蹑足,走在翡翠轩槅子外潜听。听勾多时,听见他两个在里面正干得好,只听见西门庆向李瓶儿道:“我的心肝,你达不爱别的,爱你好个白屁股儿。今日尽着你达受用。”良久,又听的李瓶儿低声叫道:“亲达达,你省可的\textShan 罢。奴身上不方便,我前番吃你弄重了些,把奴的小肚子疼起来,这两日才好些儿。”西门庆因问:“你怎的身上不方便?”李瓶儿道:“不瞒你说,奴身中已怀临月孕,望你将就些儿。”西门庆听言,满心欢喜,说道:“我的心肝,你怎不早说,既然如此,你爹胡乱耍耍罢。”于是乐极情浓,怡然感之,两手抱定其股,一泄如注。妇人在下躬股承受其精。良久,只闻得西门庆气喘吁吁,妇人莺莺声软,都被金莲在外听了。

正听之间,只见玉楼从后蓦地走来,便问:“五丫头,在这里做甚么儿?”那金莲便摇手儿。两个一齐走到轩内,慌的西门庆凑手脚不迭。问西门庆:“我去了这半日,你做甚么?恰好还没曾梳头洗脸哩!”西门庆道:“我等着丫头取那茉莉花肥皂来我洗脸。”金莲道:“我不好说的,巴巴寻那肥皂洗脸,怪不的你的脸洗的比人家屁股还白!”那西门庆听了,也不着在意里。落后梳洗毕,与玉楼一同坐下,因问:“你在后边做甚么?带了月琴来不曾?”玉楼道:“我在后边替大姐姐穿珠花来,到明日与吴舜臣媳妇儿郑三姐下茶去戴。月琴春梅拿了来。”不一时,春梅来到,说:“花儿都送与大娘、二娘收了。”西门庆令他安排酒来。不一时冰盆内沉李浮瓜,凉亭上偎红倚翠。玉楼道:“不使春梅请大姐姐?”西门庆道:“他又不饮酒,不消邀他去。”当下西门庆上坐,三个妇人两边打横。正是:得多少壶斟美酿,盘列珍羞。那潘金莲放着椅儿不坐,只坐豆青磁凉墩儿。孟玉楼叫道:“五姐,你过这椅儿上坐,那凉墩儿只怕冷。”金莲道:“不妨事,我老人家不怕冰了胎,怕甚么?”

须臾,酒过三巡,西门庆叫春梅取月琴来,教与玉楼,取琵琶,教金莲弹:“你两个唱一套‘赤帝当权耀太虚’我听。”金莲不肯,说道:“我儿,谁养的你恁乖!俺每唱,你两人到会受用快活,我不!也教李大姐拿了椿乐器儿。”西门庆道:“他不会弹甚么。”金莲道:“他不会,教他在旁边代板。”西门庆笑道:“这小淫妇单管咬蛆儿。”一面令春梅旋取了一副红牙象板来,教李瓶儿拿着。他两个方才轻舒玉指,款跨鲛绡,合着声唱《雁过沙》。丫鬟绣春在旁打扇。须臾唱毕,西门庆每人递了一杯酒,与他吃了。潘金莲不住在席上只呷冰水,或吃生果子。玉楼道:“五姐,你今日怎的只吃生冷?”金莲笑道:“我老人家肚里没闲事,怕甚么冷糕么?”羞的李瓶儿在旁,脸上红一块白一块。西门庆瞅了他一眼,说道:“你这小淫妇,单管只胡说白道的。”金莲道:“哥儿,你多说了话。老妈妈睡着吃干腊肉——是恁一丝儿一丝儿的。你管他怎的?”

正饮酒中间,忽见云生东南,雾障西北,雷声隐隐,一阵大雨来,轩前花草皆湿。正是:

\[
江河淮海添新水,翠竹红榴洗濯清。
\]
少顷雨止,天外残虹,西边透出日色来。得多少:微雨过碧矶之润,晚风凉落院之清。只见后边小玉来请玉楼。玉楼道:“大姐姐叫,有几朵珠花没穿了,我去罢,惹的他怪。”李瓶儿道:“咱两个一答儿里去,奴也要看姐姐穿珠花哩。”西门庆道:“等我送你们一送。”于是取过月琴来,教玉楼弹着,西门庆排手,众人齐唱:

\[
\cipaim{梁州序}向晚来雨过南轩,见池面红妆零乱。渐轻雷隐隐,雨收云散。但闻荷香十里,新月一钩,此佳景无限。兰汤初浴罢,晚妆残。深院黄昏懒去眠。(合)金缕唱,碧筒劝,向冰山雪槛排佳宴。清世界,几人见?
\]
又:

\[
柳阴中忽噪新蝉,见流萤飞来庭院。听菱歌何处?画船归晚。只见玉绳低度,朱户无声,此景犹堪羡。起来携素手,整云鬟。月照纱厨人未眠。(合前)
\cipaim{节节高}涟漪戏彩鸳,绿荷翻。清香泻下琼珠溅。香风扇,芳草边,闲亭畔,坐来不觉神清健。蓬莱阆苑何足羡!(合)只恐西风又惊秋,暗中不觉流年换。
\]

众人唱着不觉到角门首。玉楼把月琴递与春梅,和李瓶儿往后去了。

潘金莲遂叫道:“孟三儿,等我等儿,我也去。”才待撇了西门庆走,被西门庆一把手拉住了,说道:“小油嘴儿,你躲滑儿,我偏不放你。”拉着只一轮,险些不轮了一交。妇人道:“怪行货子,他两个都走去了,我看你留下我做甚么?”西门庆道:“咱两个在这太湖石下,取酒来,投个壶儿耍子,吃三杯。”妇人道:“怪行货子,放着亭子上不去投,平白在这里做甚么?你不信,使春梅小肉儿,他也不替你取酒来。”西门庆因使春梅。春梅越发把月琴丢与妇人,扬长的去了。妇人接过月琴,弹了一回,说道:“我问孟三儿,也学会了几句儿了。”一壁弹着,见太湖石畔石榴花经雨盛开,戏折一枝,簪于云鬓之旁,说道:“我老娘带个三日不吃饭——眼前花。”被西门庆听见,走向前把他两只小金莲扛将起来,戏道:“我把这小淫妇,不看世界面上,就\textuni{34B2}死了。”那妇人便道:“怪行货子,且不要发讪,等我放下这月琴着。”于是把月琴顺手倚在花台边,因说道:“我的儿,适才你和李瓶儿\textuni{34B2}捣去罢,没地扯嚣儿,来缠我做甚么?”西门庆道:“怪奴才,单管只胡说,谁和他有甚事。”妇人道:“我儿,你但行动,瞒不过当方土地。老娘是谁?你来瞒我!我往后边送花儿去,你两个干的好营生儿!”西门庆道:“怪小淫妇儿,休胡说!”于是按在花台上就新嘴。那妇人连忙吐舌头在他口里。西门庆道:“你教我声亲达达,我饶了你,放你起来罢。”那妇人强不过,叫了他声亲达达:“我不是你那可意的,你来缠我怎的?”两个正是:

\[
弄晴莺舌于中巧,着雨花枝分外妍。
\]

两个顽了一回,妇人道:“咱往葡萄架那里投壶耍子儿去。”因把月琴跨在胳膊上,弹着找《梁州序》后半截:

\[
\cipaim{节节高}清宵思爽然,好凉天。瑶台月下清虚殿,神仙眷,开玳筵。重欢宴,任教玉漏催银箭,水晶宫里笙歌按。(合前)
\cipaim{尾声}光阴迅速如飞电,好良宵,可惜惭阑,拚取欢娱歌声喧。
\]
两人并肩而行,须臾,转过碧池,抹过木香亭,从翡翠轩前穿过来,到葡萄架下观看,端的好一座葡萄架。但见:

\[
四面雕栏石甃,周围翠叶深稠。迎眸霜色,如千枝紫弹坠流苏:喷鼻秋香,似万架绿云垂绣带。缒缒马乳,水晶丸里浥琼浆;滚滚绿珠,金屑架中含翠渥。乃西域移来之种,隐甘泉珍玩之芳。端的四时花木衬幽葩,明月清风无价买。
\]
二人到于架下,原来放着四个凉墩,有一把壶在旁。金莲把月琴倚了,和西门庆投壶。只见春梅拿着酒,秋菊掇着果盒,盒子上一碗冰湃的果子。妇人道:“小肉儿,你头里使性儿去了,如何又送将来了?”春梅道:“教人还往那里寻你每去,谁知蓦地这里来。”秋菊放下去了。西门庆一面揭开,盒里边攒就的八槅细巧果菜,一小银素儿葡萄酒,两个小金莲蓬锺儿,两双牙筋儿,安放一张小凉杌儿上。西门庆与妇人对面坐着,投壶耍子。须臾,过桥翎花,倒入飞双雁,连科及第,二乔观书,杨妃春睡,乌龙入洞,珍珠倒卷帘,投了十数壶。把妇人灌的醉了,不觉桃花上脸,秋波斜睨。西门庆要吃药五香酒,又叫春梅取酒去。金莲说道:“小油嘴儿,再央你央儿,往房内把凉席和枕头取了来。我困的慌,这里略躺躺儿。”那春梅故作撒娇,说道:“罢么,偏有这些支使人的,谁替你又拿去!”西门庆道:“你不拿,教秋菊抱了来,你拿酒就是了。”那春梅摇着头儿去了。

迟了半日,只见秋菊儿抱了凉席枕衾来。妇人分咐:“放下铺盖,拽上花园门,往房里看去,我叫你便来。”那秋菊应诺,放下衾枕,一直去了。这西门庆起身,脱下玉色纱\textYiXuan 儿,搭在栏杆上,迳往牡丹台畔花架下,小净手去了。回来见妇人早在架儿底下,铺设凉簟枕衾停当,脱的上下没条丝,仰卧于衽席之上,脚下穿着大红鞋儿,手弄白纱扇儿摇凉。西门庆看见,怎不触动淫心,于是剩着酒兴,亦脱去上下衣,坐在一凉墩上,先将脚指挑弄其花心,挑的淫精流出,如蜗之吐涎。一面又将妇人红绣花鞋儿摘取下来,戏把他两条脚带解下来,拴其双足,吊在两边葡萄架儿上,如金龙探爪相似,使牝户大张,红钩赤露,鸡舌内吐。西门庆先倒覆着身子,执麈柄抵牝口,卖了个倒入翎花,一手据枕,极力而提之,提的阴中淫气连绵,如数鳅行泥淖中相似。妇人在下没口子呼叫达达不绝。正干在美处,只见春梅烫了酒来,一眼看见,把酒注子放下,一直走到假山顶上卧云亭那里,搭伏着棋桌儿,弄棋子耍子。西门庆抬头看见,点手儿叫他,不下来,说道:“小油嘴,我拿不下你来就罢了。”于是撇了妇人,大叉步从石磴上走到亭子上来。那春梅早从右边一条小道儿下去,打藏春坞雪洞儿里穿过去,走到半中腰滴翠山丛、花木深处,欲待藏躲,不想被西门庆撞见,黑影里拦腰抱住,说道:“小油嘴,我却也寻着你了。”遂轻轻抱到葡萄架下,笑道:“你且吃锺酒着。”一面搂他坐在腿上,两个一递一口饮酒。春梅见妇人两腿拴吊在架上,便说道:“不知你每甚么张致!大青天白日里,一时人来撞见,怪模怪样的。”西门庆问道:“角门子关上了不曾?”春梅道:“我来时扣上了。”西门庆道:“小油嘴,看我投个肉壶,名唤金弹打银鹅,你瞧,若打中一弹,我吃一锺酒。”于是向冰碗内取了枚玉黄李子,向妇人牝中,一连打了三个,皆中花心。这西门庆一连吃了三锺药五香酒,旋令春梅斟了一锺儿,递与妇人吃。又把一个李子放在牝内,不取出来,又不行事,急的妇人春心没乱,淫水直流。只是朦胧星眼,四肢軃然于枕簟之上,口中叫道:“好个作怪的冤家,捉弄奴死了。”莺声颤掉。那西门庆叫春梅在旁打着扇,只顾只酒不理他,吃来吃去,仰卧在醉翁椅儿上打睡,就睡着了。春梅见他醉睡,走来摸摸,打雪洞内一溜烟往后边去了。听见有人叫角门,开了门,原来是李瓶儿。

由着西门庆睡了一个时辰,睁开眼醒来,看见妇人还吊在架上,两只白生生腿儿跷在两边,兴不可遏。因见春梅不在跟前,向妇人道:“淫妇,我丢与你罢。”于是先抠出牝中李子,教妇人吃了。坐在一只枕头上,向纱褶子顺带内取出淫器包儿来,使上银托子,次用硫黄圈束着根子,初时不肯深入,只在牝口子来回擂晃,急的妇人仰身迎播,口中不住声叫:“达达!快些进去罢,急坏了淫妇了,我晓的你恼我,为李瓶儿故意使这促恰来奈何我,今日经着你手段,再不敢惹你了。”西门庆笑道:“小淫妇儿!你知道就好说话儿了。”于是一壁幌着他心子,把那话拽出来,向袋中包儿里打开,捻了些“闺艳声娇”涂在蛙口内,顶入牝中,送了几送。须臾,那话昂健奢棱,暴怒起来,垂首玩着往来抽拽,玩其出入之势。那妇人在枕畔,朦胧星眼,呻吟不已,没口子叫:“大\textMaoJi \textMaoBa 达达,你不知使了甚么行货子进去。罢了,淫妇的\textuni{6BF4}心痒到骨髓里去了。可怜见饶了罢。”淫妇口里碜死的言语都叫了出来,这西门庆一上手,就是三四百回,两只手倒按住枕席,仰身竭力迎播掀干,抽没至胫复送至根者,又约一百余下。妇人以帕不住在下抹拭牝中之津,随拭随出,衽席为之皆湿。西门庆行货子,没棱露脑,往来逗留不已。因向妇人说道:“我要耍个老和尚撞钟。”忽然仰身望前只一送,那话攮进去了,直抵牝屋之上。牝屋者,乃妇人牝中深极处,有屋如含苞花蕊,到此处,男子茎首,觉翕然畅美不可言。妇人触疼,急跨其身,只听磕碴响了一声,把个硫黄圈子折在里面。妇人则目瞑气息,微有声嘶,舌尖冰冷,四肢收軃于衽席之上。西门庆慌了,急解其缚,向牝中抠出硫黄圈来,折做两截。于是把妇人扶坐,半日,星眸惊闪,苏醒过来。因向西门庆作娇泣声,说道:“我的达达,你今日怎的这般大恶,险不丧了奴的性命!今后再不可这般所为,不是耍处。我如今头目森森然,莫知所之。”西门庆见日色已西,连忙替他披上衣裳。叫了春梅、秋菊来,收拾衾枕,同扶他归房。

春梅回来,看着秋菊收了吃酒的家伙,才待开花园门,来昭的儿子小铁棍儿从花架下钻出来,赶着春梅,问姑娘要果子吃。春梅道:“小囚儿,你在那里来?”把了几个桃子、李子与他,说道:“你爹醉了,还不往前边去,只怕他看见打你。”那猴子接了果子,一直去了。春梅开了花园门回来,打发西门庆与妇人上床就寝。正是:

\[
朝随金谷宴,暮伴红楼娃。
休道欢娱处,流光逐暮霞。
\]

\newpage
%# -*- coding:utf-8 -*-
%%%%%%%%%%%%%%%%%%%%%%%%%%%%%%%%%%%%%%%%%%%%%%%%%%%%%%%%%%%%%%%%%%%%%%%%%%%%%%%%%%%%%


\chapter{陈敬济徼幸得金莲\KG 西门庆糊涂打铁棍}


诗曰:

\[
几日深闺绣得成,看来便觉可人情。
一湾暖玉凌波小,两瓣秋莲落地轻。
南陌踏青春有迹,西厢立月夜无声。
看花又湿苍苔露,晒向窗前趁晚晴。
\]

话说西门庆扶妇人到房中,脱去上下衣裳,赤着身子,妇人止着红纱抹胸儿。两个并肩叠股而坐,重斟杯酌。西门庆一手搂过他粉颈,一递一口和他吃酒,极尽温存之态。睨视妇人云鬟斜軃,酥胸半露,娇眼乜斜,犹如沉酒杨妃一般,纤手不住只向他腰里摸弄那话。那话因惊,银托子还带在上面,软叮当毛都鲁的累垂伟长。西门庆戏道:“你还弄他哩,都是你头里唬出他风病来了。”妇人问:“怎的风病。”西门庆道:“既不是疯病,如何这软瘫热化,起不来了,你还不下去央及他央及儿哩。”妇人笑瞅了他一眼。一面蹲下身子去,枕着他一只腿,取过一条裤带儿来,把那话拴住,用手提着,说道:“你这厮!头里那等头睁睁,股睁睁,把人奈何昏昏的,这咱你推风症装佯死儿。”提弄了一回,放在粉脸上偎晃良久,然后将口吮之,又用舌尖挑砥其蛙口。那话登时暴怒起来,裂瓜头凹眼睁圆,落腮胡挺身直竖。西门庆亦发坐在枕头上,令妇人马爬在纱帐内,尽着吮咂,以畅其美。俄尔淫思益炽,复与妇人交接。妇人哀告道:“我的达达,你饶了奴罢,又要捉弄奴也!”是夜,二人淫乐为之无度。有词为证:

\[
战酣乐极,云雨歇,娇眼乜斜。手持玉茎犹坚硬,告才郎将就些些。满饮金杯频劝,两情似醉如痴。
\]

一夜晚景题过。到次日,西门庆往外边去了。妇人约饭时起来,换睡鞋,寻昨日脚上穿的那双红鞋,左来右去少一只。问春梅,春梅说:“昨日我和爹搊扶着娘进来,秋菊抱娘的铺盖来。”妇人叫了秋菊来问。秋菊道:“我昨日没见娘穿着鞋进来。”妇人道:“你看胡说!我没穿鞋进来,莫不我精着脚进来了?”秋菊道:“娘你穿着鞋,怎的屋里没有?”妇人骂道:“贼奴才,还装憨儿!无过只在这屋里,你替我老实寻是的!”这秋菊三间屋里,床上床下,到处寻了一遍,那里讨那只鞋来?妇人道:“端的我这屋里有鬼,摄了我这只鞋去了。连我脚上穿的鞋都不见了,要你这奴才在屋里做甚么!”秋菊道:“倒只怕娘忘记落在花园里,没曾穿进来。”妇人道:“敢是\textuni{34B2}昏了,我鞋穿在脚上没穿在脚上,我不知道?”叫春梅:“你跟着这奴才,往花园里寻去。寻出来便罢,若寻不出来,叫他院子里顶石头跪着。”这春梅真个押着他,花园到处并葡萄架跟前,寻了一遍儿,那里得来!正是:

\[
都被六丁收拾去,芦花明月竟难寻。
\]

两个寻了一遍回来,春梅骂道:“奴才,你媒人婆迷了路儿——没的说了,王妈妈卖了磨——推不的了。”秋菊道:“不知甚么人偷了娘的这只鞋去了,我没曾见娘穿进屋里去。敢是你昨日开花园门放了那个,拾了娘的这只鞋去了。”被春梅一口稠唾沫哕了去,骂道:“贼见鬼的奴才,又搅缠起我来了!六娘叫门,我不替他开?可可儿的就放进人来了?你抱着娘的铺盖就不经心瞧瞧,还敢说嘴儿!”一面押他到屋里,回妇人说没有鞋。妇人叫踩出他院子里跪着。秋菊把脸哭丧下水来,说:“等我再往花园里寻一遍,寻不着随娘打罢。”春梅道:“娘休信他。花园里地也扫得干干净净的,就是针也寻出来,那里讨鞋来?”秋菊道:“等我寻不出来,教娘打就是了。你在旁戳舌儿怎的!”妇人向春梅道:“也罢,你跟着这奴才,看他那里寻去!”

这春梅又押着他,在花园山子底下,各处花池边,松墙下,寻了一遍,没有。他也慌了,被春梅两个耳刮子,就拉回来见妇人。秋菊道:“还有那个雪洞里没寻哩。”春梅道:“那藏春坞是爹的暖房儿,娘这一向又没到那里。我看寻不出来和你答话!”于是押着他,到于藏春坞雪洞内。正面是张坐床,旁边香几上都寻到,没有。又向书箧内寻,春梅道:“这书箧内都是他的拜帖纸,娘的鞋怎的到这里?没的摭溜子捱工夫儿!翻的他恁乱腾腾的,惹他看见又是一场儿,你这歪刺骨可死的成了!”良久,只见秋菊说道:“这不是娘的鞋!”在一个纸包内,裹着些棒儿香与排草,取出来与春梅瞧:“可怎的有了,刚才就调唆打我!”春梅看见,果是一只大红平底鞋儿,说道:“是娘的,怎生得到这书箧内?好蹊跷的事!”于是走来见妇人。妇人问:“有了我的鞋,端的在那里?”春梅道:“在藏春坞,爹暖房书箧内寻出来,和些拜帖子纸、排草、安息香包在一处。”妇人拿在手内,取过他的那只来一比,都是大红四季花缎子白绫平底绣花鞋儿,绿提根儿,蓝口金儿。惟有鞋上锁线儿差些,一只是纱绿锁线,一只是翠蓝锁线,不仔细认不出来。妇人登在脚上试了试,寻出来这一只比旧鞋略紧些,方知是来旺儿媳妇子的鞋:“不知几时与了贼强人,不敢拿到屋里,悄悄藏放在那里。不想又被奴才翻将出来。”看了一回,说道:“这鞋不是我的。奴才,快与我跪着去!”分咐春梅:“拿块石头与他顶着。”那秋菊哭起来,说道:“不是娘的鞋,是谁的鞋?我饶替娘寻出鞋来,还要打我;若是再寻不出来,不知还怎的打我哩!”妇人骂道:“贼奴才,休说嘴!”春梅一面掇了块大石头顶在他头上。妇人又另换了一双鞋穿在脚上,嫌房里热,分咐春梅把妆台放在玩花楼上,梳头去了,不在话下。

却说陈敬济早晨从铺子里进来寻衣服,走到花园角门首。小铁棍儿在那里正顽着,见陈敬济手里拿着一副银网巾圈儿,便问:“姑夫,你拿的甚么?与了我耍子罢。”敬济道:“此是人家当的网巾圈儿,来赎,我寻出来与他。”那小猴子笑嘻嘻道:“姑夫,你与了我耍子罢,我换与你件好物件儿。”敬济道:“傻孩子,此是人家当的。你要,我另寻一副儿与你耍子。你有甚么好物件,拿来我瞧。”那猴子便向腰里掏出一只红绣花鞋儿与敬济看。敬济便问:“是那里的?”那猴子笑嘻嘻道:“姑夫,我对你说了罢!我昨日在花园里耍子,看见俺爹吊着俺五娘两只腿儿,在葡萄架儿底下,摇摇摆摆。落后俺爹进去了,我寻俺春梅姑娘要果子吃,在葡萄架底下拾了这只鞋。”敬济接在手里:曲是天边新月,红如退瓣莲花,把在掌中,恰刚三寸。就知是金莲脚上之物,便道:“你与了我,明日另寻一对好圈儿与你耍子。”猴子道:“姑夫你休哄我,我明日就问你要哩。”敬济道:“我不哄你。”那猴子一面笑的耍去了。

这敬济把鞋褪在袖中,自己寻思“我几次戏他,他口儿且是活,及到中间,又走滚了。不想天假其便,此鞋落在我手里。今日我着实撩逗他一番,不怕他不上帐儿。”正是:

\[
时人不用穿针线,那得工夫送巧来?
\]

陈敬济袖着鞋,迳往潘金莲房来。转过影壁,只见秋菊跪在院内,便戏道:“小大姐,为甚么来?投充了新军,又掇起石头来了?”金莲在楼上听见,便叫春梅问道:“是谁说他掇起石头来了?干净这奴才没顶着?”春梅道:“是姑夫来了。秋菊顶着石头哩。”妇人便叫:“陈姐夫,楼上没人,你上来。”这小伙儿打步撩衣上的楼来。只见妇人在楼上,前面开了两扇窗儿,挂着湘帘,那里临镜梳妆。这陈敬济走到旁边一个小杌儿坐下,看见妇人黑油般头发,手挽着梳,还拖着地儿,红丝绳儿扎着一窝丝,缵上戴着银丝\textuni{4BFC}髻,还垫出一丝香云,鬓髻内安着许多玫瑰花瓣儿,露着四鬓,打扮的就是活观音。须臾,妇人梳了头,掇过妆台去,向面盘内洗了手,穿上衣裳,唤春梅拿茶来与姐夫吃。那敬济只是笑,不做声。妇人因问:“姐夫,笑甚么?”敬济道:“我笑你管情不见了些甚么儿?”妇人道:“贼短命!我不见了,关你甚事?你怎的晓得?”敬济道:“你看,我好心倒做了驴肝肺,你倒讪起我来。恁说,我去了。”抽身往楼下就走。被妇人一把手拉住,说道:“怪短命,会张致的!来旺儿媳妇子死了,没了想头了,却怎么还认的老娘。”因问:“你猜着我不见了甚么物件儿?”这敬济向袖中取出来,提着鞋拽靶儿,笑道:“你看这个是谁的?”妇人道:“好短命,原来是你偷拿了我的鞋去了!教我打着丫头,绕地里寻。”敬济道:“你怎的到得我手里?”妇人道:“我这屋里再有谁来?敢是你贼头鼠脑,偷了我这只鞋去了。”敬济道:“你老人家不害羞。我这两日又不往你屋里来,我怎生偷你的?”妇人道:“好贼短命,等我对你爹说,你倒偷了我鞋,还说我不害羞。”敬济道:“你只好拿爹来唬我罢了。”妇人道:“你好小胆儿,明知道和来旺儿媳妇子七个八个,你还调戏他,你几时有些忌惮儿的!既不是你偷了我的鞋,这鞋怎落在你手里?趁早实供出来,交还与我鞋,你还便宜。自古物见主,必索取。但道半个不字,教你死在我手里。”敬济道:“你老人家是个女番子,且是倒会的放刁。这里无人,咱们好讲:你既要鞋,拿一件物事儿,我换与你,不然天雷也打不出去。”妇人道:“好短命!我的鞋应当还我,教换甚物事儿与你?”敬济笑道:“五娘,你拿你袖的那方汗巾儿赏与儿子,儿子与了你的鞋罢。”妇人道:“我明日另寻一方好汗巾儿,这汗巾儿是你爹成日眼里见过,不好与你的。”敬济道:“我不。别的就与我一百方也不算,我一心只要你老人家这方汗巾儿。”妇人笑道:“好个牢成久惯的短命!我也没气力和你两个缠。”于是向袖中取出一方细撮穗白绫挑线莺莺烧夜香汗巾儿,上面连银三字儿都掠与他。有诗为证:

\[
郎君见妾下兰阶,来索纤纤红绣鞋。
不管露泥藏袖里,只言从此事堪谐。
\]
这陈敬济连忙接在手里,与他深深的唱个喏。妇人分咐:“好生藏着,休教大姐看见,他不是好嘴头子。”敬济道:“我知道。”一面把鞋递与他,如此这般:“是小铁棍儿昨日在花园里拾的,今早拿着问我换网巾圈儿耍子。”如此这般,告诉了一遍。妇人听了,粉面通红,说道:“你看贼小奴才,把我这鞋弄的恁漆黑的!看我教他爹打他不打他。”敬济道:“你弄杀我!打了他不打紧,敢就赖着我身上,是我说的。千万休要说罢。”妇人道:“我饶了小奴才,除非饶了蝎子。”

两个正说在热闹处,忽听小厮来安儿来寻:“爹在前厅请姐夫写礼帖儿哩。”妇人连忙撺掇他出去了。下的楼来,教春梅取板子来,要打秋菊。秋菊不肯躺,说道:“寻将娘的鞋来,娘还要打我!”妇人把陈敬济拿的鞋递与他看,骂道:“贼奴才,你把那个当我的鞋,将这个放在那里?”秋菊看见,把眼瞪了半日,说道:“可是作怪的勾当,怎生跑出娘三只鞋来了?”妇人道:“好大胆奴才!你拿谁的鞋来搪塞我,倒说我是三只脚的蟾?”不由分说,教春梅拉倒,打了十下。打有秋菊抱股而哭,望着春梅道:“都是你开门,教人进来,收了娘的鞋,这回教娘打我。”春梅骂道:“你倒收拾娘铺盖,不见了娘的鞋,娘打了你这几下儿,还敢抱怨人!早是这只旧鞋,若是娘头上的簪环不见了,你也推赖个人儿就是了?娘惜情儿,还打的你少。若是我,外边叫个小厮,辣辣的打上他二三十板,看这奴才怎么样的!”几句骂得秋菊忍气吞声,不言语了。

且说西门庆叫了敬济到前厅,封尺头礼物,送贺千户新升了淮安提刑所掌刑正千户。本卫亲识,都与他送行在永福寺,不必细说。西门庆差了钺安送去,厅上陪着敬济吃了饭,归到金莲房中。这金莲千不合万不合,把小铁棍儿拾鞋之事告诉一遍,说道:“都是你这没才料的货平白干的勾当!教贼万杀的小奴才把我的鞋拾了,拿到外头,谁是没瞧见。被我知道,要将过来了。你不打与他两下,到明日惯了他。”西门庆就不问:“谁告你说来。”一冲性子走到前边。那小猴儿不知,正在石台基顽耍,被西门庆揪住顶角,拳打脚踢,杀猪也似叫起来,方才住了手。这小猴子躺在地下,死了半日,慌得来昭两口子走来扶救,半日苏醒。见小厮鼻口流血,抱他到房里慢慢问他,方知为拾鞋之事惹起事来。这一丈青气忿忿的走到后边厨下,指东骂西,一顿海骂道:“贼不逢好死的淫妇,王八羔子!我的孩子和你有甚冤仇?他才十一二岁,晓的甚么?知道屄也在那块儿?平白地调唆打他恁一顿,打的鼻口中流血。假若死了,淫妇、王八儿也不好!称不了你甚么愿!”厨房里骂了,到前边又骂,整骂了一二日还不定。因金莲在房中陪西门庆吃酒,还不知。

晚夕上床宿歇,西门庆见妇人脚上穿着两只绿绸子睡鞋,大红提根儿,因说道:“啊呀,如何穿这个鞋在脚?怪怪的不好看。”妇人道:“我只一双红睡鞋,倒吃小奴才将一只弄油了,那里再讨第二双来?”西门庆道:“我的儿,你到明日做一双儿穿在脚上。你不知,我达达一心欢喜穿红鞋儿,看着心里爱。”妇人道:“怪奴才!可可儿的来想起一件事来,我要说,又忘了。”因令春梅:“你取那只鞋来与他瞧。”——“你认的这鞋是谁的鞋?”西门庆道:“我不知是谁的鞋。”妇人道:“你看他还打张鸡儿哩!瞒着我,黄猫黑尾,你干的好茧儿!来旺儿媳妇子的一只臭蹄子,宝上珠也一般,收藏在藏春坞雪洞儿里拜帖匣子内,搅着些字纸和香儿一处放着。甚么稀罕物件,也不当家化化的!怪不的那贼淫妇死了,堕阿鼻地狱!”又指着秋菊骂道:“这奴才当我的鞋,又翻出来,教我打了几下。”分咐春梅:“趁早与我掠出去!”春梅把鞋掠在地下,看着秋菊说道:“赏与你穿了罢!”那秋菊拾在手里,说道:“娘这个鞋,只好盛我一个脚指头儿罢了。”妇人骂道:“贼奴才,还教甚么屄娘哩,他是你家主子前世的娘!不然,怎的把他的鞋这等收藏的娇贵?到明日好传代!没廉耻的货!”秋菊拿着鞋就往外走,被妇人又叫回来,分咐:“取刀来,等我把淫妇剁作几截子,掠到茅厕里去!叫贼淫妇阴山背后,永世不得超生!”因向西门庆道:“你看着越心疼,我越发偏剁个样儿你瞧。”西门庆笑道:“怪奴才,丢开手罢了。我那里有这个心!”妇人道:“你没这个心,你就赌了誓。淫妇死的不知往那去了,你还留着他的鞋做甚么?早晚有省,好思想他。正以俺每和你恁一场,你也没恁个心儿,还要人和你一心一计哩!”西门庆笑道:“罢了,怪小淫妇儿,偏有这些儿的!他就在时,也没曾在你跟前行差了礼法。”于是搂过粉项来就亲了个嘴,两个云雨做一处。正是:动人春色娇还媚,惹蝶芳心软又浓。有诗为证:

\[
漫吐芳心说向谁?欲于何处寄想思?
想思有尽情难尽,一日都来十二时。
\]

\newpage
%# -*- coding:utf-8 -*-
%%%%%%%%%%%%%%%%%%%%%%%%%%%%%%%%%%%%%%%%%%%%%%%%%%%%%%%%%%%%%%%%%%%%%%%%%%%%%%%%%%%%%


\chapter{吴神仙冰鉴定终身\KG 潘金莲兰汤邀午战}


词曰:

\[
新凉睡起,兰汤试浴郎偷戏。去曾嗔怒,来便生欢喜。奴道无心,郎道奴如此。情如水,易开难断,若个知生死。
\]

话说到次日,潘金莲早起,打发西门庆出门。记挂着要做那红鞋,拿着针线筐儿,往翡翠轩台基儿上坐着,描画鞋扇。使春梅请了李瓶儿来到。李瓶儿问道:“姐姐,你描金的是甚么?”金莲道:“要做一双大红鞋素缎子白绫平底鞋儿,鞋尖上扣绣鹦鹉摘桃。”李瓶儿道:“我有一方大红十样锦缎子,也照依姐姐描恁一双儿。我做高低的罢。”于是取了针线筐,两个同一处做。金莲描了一只丢下,说道:“李大姐,你替我描这一只,等我后边把孟三姐叫了来。他昨日对我说,他也要做鞋哩。”一直走到后边。玉楼在房中倚着护炕儿,也衲着一只鞋儿哩。看见金莲进来,说道:“你早办!”金莲道:“我起来的早,打发他爹往门外与贺千户送行去了。教我约下李大姐,花园里赶早凉做些生活。我才描了一只鞋,教李大姐替我描着,迳来约你同去,咱三个一搭儿里好做。”因问:“你手里衲的是甚么鞋?”玉楼道:“是昨日你看我开的那双玄色缎子鞋。”金莲道:“你好汉!又早衲出一只来了。”玉楼道:“那只昨日就衲好了,这一只又衲了好些了。”金莲接过看了一回,说:“你这个,到明日使甚么云头子?”玉楼道:“我比不得你每小后生,花花黎黎。我老人家了,使羊皮金缉的云头子罢,周围拿纱绿线锁,好不好?”金莲道:“也罢。你快收拾,咱去来,李瓶儿那里等着哩。”玉楼道:“你坐着吃了茶去。”金莲道:“不吃罢,拿了茶,那里去吃来。”玉楼分咐兰香顿下茶送去。两个妇人手拉着手儿,袖着鞋扇,迳往外走。吴月娘在上房穿廊下坐,便问:“你每那去?”金莲道:“李大姐使我替他叫孟三儿去,与他描鞋。”说着,一直来到花园内。

三人一处坐下,拿起鞋扇,你瞧我的,我瞧你的,都瞧了一遍。玉楼便道:“六姐,你平白又做平底子红鞋做甚么?不如高低好看。你若嫌木底子响脚,也似我用毡底子,却不好?”金莲道:“不是穿的鞋,是睡鞋。他爹因我那只睡鞋,被小奴才儿偷去弄油了,分咐教我从新又做这双鞋。”玉楼道:“又说鞋哩,这个也不是舌头,李大姐在这里听着。昨日因你不见了这只鞋,他爹打了小铁棍儿一顿,说把他打的躺在地下,死了半日。惹的一丈青好不在后边海骂,骂那个淫妇王八羔子学舌,打了他恁一顿,早是活了,若死了,淫妇、王八羔子也不得清洁!俺再不知骂的是谁。落后小铁棍儿进来,大姐姐问他:‘你爹为甚么打你?’小厮才说:‘因在花园里耍子,拾了一只鞋,问姑夫换圈儿来。不知是甚么人对俺爹说了,教爹打我一顿。我如今寻姑夫,问他要圈儿去也。’说毕,一直往前跑了。原来骂的‘王八羔子’是陈姐夫。早是只李娇儿在旁边坐着,大姐没在跟前,若听见时,又是一场儿。”金莲道:“大姐姐没说甚么?”玉楼道:“你还说哩,大姐姐好不说你哩!说:‘如今这一家子乱世为王,九条尾狐狸精出世了,把昏君祸乱的贬子休妻,想着去了的来旺儿小厮,好好的从南边来了,东一帐西一帐,说他老婆养着主子,又说他怎的拿刀弄杖,生生儿祸弄的打发他出去了,把个媳妇又逼的吊死了。如今为一只鞋子,又这等惊天动地反乱。你的鞋好好穿在脚上,怎的教小厮拾了?想必吃醉了,在花园里和汉子不知怎的饧成一块,才掉了鞋。如今没的摭羞,拿小厮顶缸,又不曾为甚么大事。’”金莲听了,道:“没的扯屄淡!甚么是‘大事’?杀了人是大事了,奴才拿刀要杀主子!”向玉楼道:“孟三姐,早是瞒不了你,咱两个听见来兴儿说了一声,唬的甚么样儿的!你是他的大老婆,倒说这个话!你也不管,我也不管,教奴才杀了汉子才好。他老婆成日在你后边使唤,你纵容着他不管,教他欺大灭小,和这个合气,和那个合气。各人冤有头,债有主,你揭条我,我揭条你,吊死了,你还瞒着汉子不说。早是苦了钱,好人情说下来了,不然怎了?你这等推干净,说面子话儿,左右是,左右我调唆汉子!也罢,若不教他把奴才老婆、汉子一条提撵的离门离户也不算!恒数人挟不到我井里头!”玉楼见金莲粉面通红,恼了,又劝道:“六姐,你我姐妹都是一个人,我听见的话儿,有个不对你说?说了,只放在你心里,休要使出来。”金莲不依他。到晚等的西门庆进入他房来,一五一十告西门庆说:“来昭媳妇子一丈青怎的在后边指骂,说你打了他孩子,要逻揸儿和人嚷。”这西门庆不听便罢,听了记在心里。到次日,要撵来昭三口子出门。多亏月娘再三拦劝下,不容他在家,打发他往狮子街房子里看守,替了平安儿来家守大门。后次月娘知道,甚恼金莲,不在话下。

西门庆一日正在前厅坐,忽平安儿来报:“守备府周爷差人送了一位相面先生,名唤吴神仙,在门首伺候见爹。”西门庆唤来人进见,递上守备帖儿,然后道:“有请。”须臾,那吴神仙头戴青布道巾,身穿布袍草履,腰系黄丝双穗绦,手执龟壳扇子,自外飘然进来。年约四十之上,生得神清如长江皓月,貌古似太华乔松。原来神仙有四般古怪:身如松,声如钟,坐如弓,走如风。但见他:

\[
能通风鉴,善究子平。观乾象,能识阴阳;察龙经,明知风水。五星深讲,三命秘谈。审格局,决一世之荣枯;观气色,定行年之休咎。若非华岳修真客,定是成都卖卜人。
\]
西门庆见神仙进来,忙降阶迎接,接至厅上。神仙见西门庆,长揖稽首就坐。须臾茶罢。西门庆动问神仙:“高名雅号,仙乡何处,因何与周大人相识?”那吴神仙欠身道:“贫道姓吴名奭,道号守真。本贯浙江仙游人。自幼从师天台山紫虚观出家。云游上国,因往岱宗访道,道经贵处。周老总兵相约,看他老夫人目疾,特送来府上观相。”西门庆道:“老仙长会那几家阴阳?道那几家相法?”神仙道:“贫道粗知十三家子平,善晓麻衣相法,又晓六壬神课。常施药救人,不爱世财,随时住世。”西门庆听言,益加敬重,夸道:“真乃谓之神仙也。”一面令左右放桌儿,摆斋管待。神仙道:“贫道未道观相,岂可先要赐斋。”西门庆笑道:“仙长远来,一定未用早斋。待用过,看命未迟。”于是陪着神仙吃了些斋食素馔,抬过桌席,拂抹干净,讨笔砚来。

神仙道:“请先观贵造,然后观相尊容。”西门庆便说与八字:“属虎的,二十九岁了,七月二十八日午时生。”这神仙暗暗十指寻纹,良久说道:“官人贵造:戊寅年,辛酉月,壬午日,丙午时。七月廿三日白戊,已交八月算命。月令提刚辛酉,理取伤官格。子平云:伤官伤尽复生财,财旺生官福转来。立命申宫,七岁行运辛酉,十七行壬戌,二十七癸亥,三十七甲子,四十七乙丑。官人贵造,依贫道所讲,元命贵旺,八字清奇,非贵则荣之造。但戊土伤官,生在七八月,身忒旺了。幸得壬午日干,丑中有癸水,水火相济,乃成大器。丙午时,丙合辛生,后来定掌威权之职。一生盛旺,快乐安然,发福迁官,主生贵子。为人一生耿直,干事无二,喜则合气春风,怒则迅雷烈火。一生多得妻财,不少纱帽戴。临死有二子送老。今岁丁未流年,丁壬相合,目下丁火来克,克我者为官为鬼,必主平地登云之喜,添官进禄之荣。大运见行癸亥,戊土得癸水滋润,定见发生。目下透出红鸾天喜,定有熊罴之兆。又命宫驿马临申,不过七月必见矣。”西门庆问道:“我后来运限如何?”神仙道:“官人休怪我说,但八字中不宜阴水太多,后到甲子运中,将壬午冲破了,又有流星打搅,不出六六之年,主有呕血流浓之灾,骨瘦形衰之病。”西门庆问道:“目下如何?”神仙道:“目今流年,日逢破败五鬼在家吵闹,些小气恼,不足为灾,都被喜气神临门冲散了。”西门庆道:“命中还有败否?”神仙道:“年赶着月,月赶着日,实难矣。”

西门庆听了,满心欢喜,便道:“先生,你相我面如何?”神仙道:“请尊容转正。”西门庆把座儿掇了一掇。神仙相道:“夫相者,有心无相,相逐心生;有相无心,相随心往。吾观官人:头圆项短,定为享福之人;体健筋强,决是英豪之辈;天庭高耸,一生衣禄无亏;地阁方圆,晚岁荣华定取。此几椿儿好处。还有几椿不足之处,贫道不敢说。”西门庆道:“仙长但说无妨。”神仙道:“请官人走两步看。”西门庆真个走了几步。神仙道:“你行如摆柳,必主伤妻;若无刑克,必损其身。妻宫克过方好。”西门庆道:“已刑过了。”神仙道:“请出手来看一看。”西门庆舒手来与神仙看。神仙道:“智慧生于皮毛,苦乐观于手足。细软丰润,必享福禄之人也。两目雌雄,必主富而多诈;眉生二尾,一生常自足欢娱;根有三纹,中岁必然多耗散;奸门红紫,一生广得妻财;黄气发于高旷,旬日内必定加官;红色起于三阳,今岁间必生贵子。又有一件不敢说,泪堂丰厚,亦主贪花;且喜得鼻乃财星,验中年之造化;承浆地阁,管来世之荣枯。

\[
承浆地阁要丰隆,准乃财星居正中。
生平造化皆由命,相法玄机定不容。
\]

神仙相毕,西门庆道:“请仙长相相房下众人。”一面令小厮:“后边请你大娘出来。”于是李娇儿、孟玉楼、潘金莲、李瓶儿、孙雪娥等众人都跟出来,在软屏后潜听。神仙见月娘出来,连忙道了稽首,也不敢坐,就立在旁边观相。端详了一回,说:“娘子面如满月,家道兴隆;唇若红莲,衣食丰足,必得贵而生子;声响神清,必益夫而发福。请出手来。”月娘从袖中露出十指春葱来。神仙道:“干姜之手,女人必善持家,照人之鬓,坤道定须秀气。这几椿好处。还有些不足之处,休怪贫道直说。”西门庆道:“仙长但说无妨。”“泪堂黑痣,若无宿疾,必刑夫;眼下皴纹,亦主六亲若冰炭。

\[
女人端正好容仪,缓步轻如出水龟。
行不动尘言有节,无肩定作贵人妻。
\]

相毕,月娘退后。西门庆道:“还有小妾辈,请看看。”于是李娇儿过来。神仙观看良久:“此位娘子,额尖鼻小,非侧室,必三嫁其夫;肉重身肥,广有衣食而荣华安享;肩耸声泣,不贱则孤;鼻梁若低,非贫即夭。请步几步我看。”李娇儿走了几步。神仙道:

\[
额尖露背并蛇行,早年必定落风尘。
假饶不是娼门女,也是屏风后立人。
\]

相毕,李娇儿下去。吴月娘叫:“孟三姐,你也过来相一相。”神仙观道:“这位娘子,三停平等,一生衣禄无亏;六府丰隆,晚岁荣华定取。平生少疾,皆因月孛光辉;到老无灾,大抵年宫润秀。请娘子走两步。”玉楼走了两步,神仙道:

\[
口如四字神清澈,温厚堪同掌上珠。
威命兼全财禄有,终主刑夫两有余。
\]

玉楼相毕,叫潘金莲过来。那潘金莲只顾嘻笑,不肯过来。月娘催之再三,方才出见。神仙抬头观看这个妇人,沉吟半日,方才说道:“此位娘子,发浓鬓重,光斜视以多淫;脸媚眉弯,身不摇而自颤。面上黑痣,必主刑夫;唇中短促,终须寿夭。

\[
举止轻浮唯好淫,眼如点漆坏人伦。
月下星前长不足,虽居大厦少安心。
\]

相毕金莲,西门庆又叫李瓶儿上来,教神仙相一相。神仙观看这个女人:“皮肤香细,乃富室之女娘;容貌端庄,乃素门之德妇。只是多了眼光如醉,主桑中之约;眉眉靥生,月下之期难定。观卧蚕明润而紫色,必产贵儿;体白肩圆,必受夫之宠爱。常遭疾厄,只因根上昏沉;频遇喜祥,盖谓福星明润。此几椿好处。还有几椿不足处,娘子可当戒之:山根青黑,三九前后定见哭声;法令细繵,鸡犬之年焉可过?慎之!慎之!

\[
花月仪容惜羽翰,平生良友凤和鸾。
朱门财禄堪依倚,莫把凡禽一样看。
\]

相毕,李瓶儿下去。月娘令孙雪娥出来相一相。神仙看了,说道:“这位娘子,体矮声高,额尖鼻小,虽然出谷迁乔,但一生冷笑无情,作事机深内重。只是吃了这四反的亏,后来必主凶亡。夫四反者:唇反无棱,耳反无轮,眼反无神,鼻反不正故也。

\[
燕体蜂腰是贱人,眼如流水不廉真。
常时斜倚门儿立,不为婢妾必风尘。
\]

雪娥下去,月娘教大姐上来相一相。神仙道:“这位女娘,鼻梁低露,破祖刑家;声若破锣,家私消散。面皮太急,虽沟洫长而寿亦夭;行如雀跃,处家室而衣食缺乏。不过三九,当受折磨。

\[
唯夫反目性通灵,父母衣食仅养身。
状貌有拘难显达,不遭恶死也艰辛。
\]

大姐相毕,教春梅也上来教神仙相相。神仙睁眼儿见了春梅,年约不上二九,头戴银丝云髻儿,白线挑衫儿,桃红裙子,蓝纱比甲儿,缠手缠脚出来,道了万福。神仙观看良久,相道:“此位小姐五官端正,骨格清奇。发细眉浓,禀性要强;神急眼圆,为人急燥。山根不断,必得贵夫而生子;两额朝拱,主早年必戴珠冠。行步若飞仙,声响神清,必益夫而得禄,三九定然封赠。但吃了这左眼大,早年克父;右眼小,周岁克娘。左口角下这一点黑痣,主常沾啾唧之灾;右腮一点黑痣,一生受夫敬爱。

\[
天庭端正五官平,口若涂砂行步轻。
仓库丰盈财禄厚,一生常得贵人怜。
\]

神仙相毕,众妇女皆咬指以为神相。西门庆封白银五两与神仙,又赏守备府来人银五钱,拿拜帖回谢。吴神仙再三辞却,说道:“贫道云游四方,风餐露宿,要这财何用?决不敢受。”西门庆不得已,拿出一匹大布:“送仙长一件大衣如何?”神仙方才受之,令小童接了,稽首拜谢。西门庆送出大门,飘然而去。正是:

\[
柱杖两头挑日月,葫芦一个隐山川。
\]

西门庆回到后厅,问月娘:“众人所相何如?”月娘道:“相的也都好,只是三个人相不着。”西门庆道:“那三个相不着?”月娘道:“相李大姐有实疾,到明日生贵子,他见今怀着身孕,这个也罢了。相咱家大姐到明日受磨折,不知怎的磨折?相春梅后来也生贵子,或者你用好他,各人子孙也看不见。我只不信,说他后来戴珠冠,有夫人之分。端的咱家又没官,那讨珠冠来?就有珠冠,也轮不到他头上。”西门庆笑道:“他相我目下有平地登云之喜,加官进禄之荣,我那得官来?他见春梅和你俱站在一处,又打扮不同,戴着银丝云髻儿,只当是你我亲生女儿一般,或后来匹配名门,招个贵婿,故说有珠冠之分。自古算的着命,算不着好,相逐心生,相随心灭。周大人送来,咱不好嚣了他的,教他相相除疑罢了。”说毕,月娘房中摆下饭,打发吃了饭。

西门庆手拿芭蕉扇儿,信步闲游。来花园大卷棚聚景堂内,周围放下帘栊,四下花木掩映。正值日午,只闻绿阴深处一派蝉声,忽然风送花香,袭人扑鼻。有诗为证:

\[
绿树荫浓夏日长,楼台倒影入池塘。
水晶帘动微风起,一架蔷薇满院香。
\]
西门庆坐于椅上以扇摇凉。只见来安儿、画童儿两个小厮来井上打水。西门庆道:“教一个来。”来安儿忙走向前,西门庆分咐:“到后边对你春梅姐说,有梅汤提一壶来我吃。”来安儿应诺去了。半日,只见春梅家常戴着银丝云髻儿,手提一壶蜜煎梅汤,笑嘻嘻走来,问道:“你吃了饭了?”西门庆道:“我在后边吃了。”春梅说:“嗔道不进房里来。说你要梅汤吃,等我放在冰里湃一湃你吃。”西门庆点头儿。春梅湃上梅汤,走来扶着椅儿,取过西门庆手中芭蕉扇儿替他打扇,问道:“头里大娘和你说甚么?”西门庆道:“说吴神仙相面一节。”春梅道:“那道士平白说戴珠冠,教大娘说‘有珠冠,只怕轮不到他头上’。常言道凡人不可貌相,海水不可斗量,从来旋的不圆,砍的圆,各人裙带上衣食,怎么料得定?莫不长远只在你家做奴才罢!”西门庆笑道:“小油嘴儿,你若到明日有了娃儿,就替你上了头。”于是把他搂到怀里,手扯着手儿顽耍,问道:“你娘在那里?怎的不见?”春梅道:“娘在屋里,教秋菊热下水要洗浴。等不的,就在床上睡了。”西门庆道:“等我吃了梅汤,鬼混他一混去。”于是春梅向冰盆内倒了一瓯儿梅汤,与西门庆呷了一口,湃骨之凉,透心沁齿,如甘露洒心一般。

须臾吃毕,搭伏着春梅肩膀儿,转过角门来到金莲房中。看见妇人睡在正面一张新买的螺钿床上。原是因李瓶儿房中安着一张螺钿敞厅床,妇人旋教西门庆使了六十两银子,替他也买了这一张螺钿有栏干的床。两边槅扇都是螺钿攒造花草翎毛,挂着紫纱帐幔,锦带银钩。妇人赤露玉体,止着红绡抹胸儿,盖着红纱衾,枕着鸳鸯枕,在凉席之上,睡思正浓。西门庆一见,不觉淫心顿起,令春梅带上门出去,悄悄脱了衣裤,上的床来,掀开纱被,见他玉体相互掩映,戏将两股轻开,按麈柄徐徐插入牝中,比及星眼惊欠之际,已抽拽数十度矣。妇人睁开眼,笑道:“怪强盗,三不知多咱进来?奴睡着了,就不知道。奴睡的甜甜的,掴混死了我!”西门庆道:“我便罢了,若是个生汉子进来,你也推不知道罢?”妇人道:“我不好骂的,谁人七个头八个胆,敢进我这房里来!只许你恁没大没小的罢了。”原来妇人因前日西门庆在翡翠轩夸奖李瓶儿身上白净,就暗暗将茉莉花蕊儿搅酥油定粉,把身上都搽遍了,搽的白腻光滑,异香可爱,欲夺其宠。西门庆见他身体雪白,穿着新做的两只大红睡鞋。一面蹲踞在上,两手兜其股,极力而提之,垂首观其出入之势。妇人道:“怪货,只顾端详甚么?奴的身上黑,不似李瓶儿的身上白就是了。他怀着孩子,你便轻怜痛惜,俺每是拾的,由着这等掇弄。”西门庆问道:“说你等着我洗澡来?”妇人问道:“你怎得知道来?”西门庆道:“是春梅说的。”妇人道:“你洗,我叫春梅掇水来。”不一时把浴盆掇到房中,注了汤。二人下床来,同浴兰汤,共效鱼水之欢。洗浴了一回,西门庆乘兴把妇人仰卧在浴板之上,两手执其双足跨而提之,掀腾\textShan 干,何止二三百回,其声如泥中螃蟹一般响之不绝。妇人恐怕香云拖坠,一手扶着云鬓,一手扳着盆沿,口中燕语莺声,百般难述。怎见这场交战?但见:

\[
华池荡漾波纹乱,翠帏高卷秋云暗。才郎情动逞风流,美女心欢显手段。叭叭嗒嗒弄声响,砰砰啪啪成一片。滑滑\textShuiChu \textShuiChu 怎停住,拦拦济济难存站。一个逆水撑船,将玉股摇;一个艄公把舵,将金莲揝。拖泥带水两情痴,殢雨尤云都不辩。任他锦帐凤鸾交,不似兰汤鱼水战。
\]

二人水中战斗了一回,西门庆精泄而止。拭抹身体干净,撤去浴盆。止着薄纩短襦上床,安放炕桌果酌饮酒。教秋菊:“取白酒来与你爹吃。”又拿果馅饼与西门庆吃,恐怕他肚中饥饿。只见秋菊半日拿上一银注子酒来。妇人才斟了一锺,摸了摸冰凉的,就照着秋菊脸上只一泼,泼了一头一脸,骂道:“好贼少死的奴才!我分咐教你烫了来,如何拿冷酒与爹吃?你不知安排些甚么心儿?”叫春梅:“与我把这奴才采到院子里跪着去。”春梅道:“我替娘后边卷裹脚去来,一些儿没在跟前,你就弄下碜儿了。”那秋菊把嘴谷都着,口里喃喃呐呐说道:“每日爹娘还吃冰湃的酒儿,谁知今日又改了腔儿。”妇人听见骂道:“好贼奴才,你说甚么?与我采过来!”叫春梅每边脸上打与他十个嘴巴。春梅道:“皮脸,没的打污浊了我手。娘只教他顶着石头跪着罢。”于是不由分说,拉到院子里,教他顶着块大石头跪着,不在话下。妇人从新叫春梅暖了酒来,陪西门庆吃了几锺,掇去酒桌,放下纱帐子来,分咐拽上房门,两个抱头交股,体倦而寝。正是:

\[
若非群玉山头见,多是阳台梦里寻。
\]

\newpage
%# -*- coding:utf-8 -*-
%%%%%%%%%%%%%%%%%%%%%%%%%%%%%%%%%%%%%%%%%%%%%%%%%%%%%%%%%%%%%%%%%%%%%%%%%%%%%%%%%%%%%


\chapter{蔡太师擅恩锡爵\KG 西门庆生子加官}


词曰:

\[
十千日日索花奴,白马骄驼冯子都。今年新拜执金吾。侵幕露桃初结子,妒花娇鸟忽嗛雏。闺中姊妹半愁娱。
\]

话说西门庆与潘金莲两个洗毕澡,就睡在房中。春梅坐在穿廊下一张凉椅儿上纳鞋,只见琴童儿在角门首探头舒脑的观看。春梅问道:“你有甚话说?”那琴童见秋菊顶着石头跪在院内,只顾用手往来指。春梅骂道:“怪囚根子!有甚话,说就是了,指手画脚怎的?”那琴童笑了半日,方才说:“看坟的张安,在外边等爹说话哩。”春梅道:“贼囚根子!张安就是了,何必大惊小怪,见鬼也似!悄悄儿的,爹和娘睡着了。惊醒他,你就是死。你且叫张安在外边等等儿。”琴童儿走出来外边,约等勾半日,又走来角门首踅探,问道:“爹起来了不曾?”春梅道:“怪囚!失张冒势,唬我一跳,有要没紧,两头游魂哩!”琴童道:“张安等爹说了话,还要赶出门去,怕天晚了。”春梅道:“爹娘正睡的甜甜儿的,谁敢搅扰他,你教张安且等着去,十分晚了,教他明日去罢。”

正说着,不想西门庆在房里听见,便叫春梅进房,问谁说话。春梅道:“琴童说坟上张安儿在外边,见爹说话哩。”西门庆道:“拿衣我穿,等我起去。”春梅一面打发西门庆穿衣裳,金莲便问:“张安来说甚么话?”西门庆道:“张安前日来说,咱家坟隔壁赵寡妇家庄子儿连地要卖,价银三百两。我只还他二百五十两银子,教张安和他讲去。里面一眼井,四个井圈打水。若买成这庄子,展开合为一处,里面盖三间卷棚,三间厅房,叠山子花园、井亭、射箭厅、打毬场,耍子去处,破使几两银子收拾也罢。”妇人道:“也罢,咱买了罢。明日你娘每上坟,到那里好游玩耍子。”说毕,西门庆往前边和张安说话去了。

金莲起来,向镜台前重匀粉脸,再整云鬟。出来院内要打秋菊。那春梅旋去外边叫了琴童儿来吊板子。金莲问道:“叫你拿酒,你怎的拿冷酒与爹吃?原来你家没大了,说着,你还钉嘴铁舌儿的!”喝声:“叫琴童儿与我老实打与这奴才二十板子!”那琴童才打到十板子上,多亏了李瓶儿笑嘻嘻走过来劝住了,饶了他十板。金莲教与李瓶儿磕了头,放他起来,厨下去了。李瓶儿道:“老潘领了个十五岁的丫头,后边二姐姐买了房里使唤,要七两五钱银子。请你过去瞧瞧。”金莲遂与李瓶儿一同后边去了。李娇儿果问西门庆用七两银子买了,改名夏花儿,房中使唤,不在话下。

单表来保同吴主管押送生辰担,正值炎蒸天气,路上十分难行,免不得饥餐渴饮。有日到了东京万寿门外,寻客店安下。到次日,赍台驮箱礼物,迳到天汉桥蔡太师府门前伺候。来保教吴主管押着礼物,他穿上青衣,迳向守门官吏唱了个喏。那守门官吏问道:“你是那里来的?”来保道:“我是山东清河县西门员外家人,来与老爷进献生辰礼物。”官吏骂道:“贼少死野囚军!你那里便兴你东门员外、西门员外?俺老爷当今一人之下,万人之上,不论三台八位,不论公子王孙,谁敢在老爷府前这等称呼?趁早靠后!”内中有认的来保的,便安抚来保说道:“此是新参的守门官吏,才不多几日,他不认的你,休怪。你要禀见老爷,等我请出翟大叔来。”这来保便向袖中取出一包银子,重一两,递与那人。那人道:“我到不消。你再添一分,与那两个官吏,休和他一般见识。”来保连忙拿出三包银子来,每人一两,都打发了。那官吏才有些笑容儿,说道:“你既是清河县来的,且略等候,等我领你先见翟管家。老爷才从上清宝霄宫进了香回来,书房内睡。”良久,请将翟管家出来,穿着凉鞋净袜,青丝绢道袍。来保见了,忙磕下头去。翟管家答礼相还,说道:“前者累你。你来与老爷进生辰担礼来了?”来保先递上一封揭帖,脚下人捧着一对南京尺头,三十两白金,说道:“家主西门庆,多上覆翟爹,无物表情,这些薄礼,与翟爹赏人。前者盐客王四之事,多蒙翟爹费心。”翟谦道:“此礼我不当受。罢,罢,我且收下。”来保又递上太师寿礼帖儿,看了,还付与来保,分咐把礼抬进来,到二门里首伺候。原来二门西首有三间倒座,来往杂人都在那里待茶。须臾,一个小童拿了两盏茶来,与来保、吴主管吃了。

少顷,太师出厅。翟谦先禀知太师,然后令来保、吴主管进见,跪于阶下。翟谦先把寿礼揭帖呈递与太师观看,来保、吴主管各抬献礼物。但见:

\[
黄烘烘金壶玉盏,白晃晃减靸仙人。锦绣蟒衣,五彩夺目;南京纻缎,金碧交辉。汤羊美酒,尽贴封皮;异果时新,高堆盘盒。
\]
如何不喜,便道:“这礼物决不好受的,你还将回去。”慌的来保等在下叩头,说道:“小的主人西门庆,没甚孝意,些小微物,进献老爷赏人。”太师道:“既是如此,令左右收了。”旁边祗应人等,把礼物尽行收下去。太师又道:“前日那沧州客人王四等之事,我已差人下书,与你巡抚侯爷说了。可见了分上不曾?”来保道:“蒙老爷天恩,书到,众盐客就都放出来了。”太师又向来保说道:“累次承你主人费心,无物可伸,如何是好?你主人身上可有甚官役?”来保道:“小人的主人一介乡民,有何官役?”太师道:“既无官役,昨日朝廷钦赐了我几张空名告身扎付,我安你主人在你那山东提刑所,做个理刑副千户,顶补千户贺金的员缺,好不好?”来保慌的叩头谢道:“蒙老爷莫大之恩,小的家主举家粉首碎身,莫能报答!”于是唤堂候官抬书案过来,即时签押了一道空名告身扎付,把西门庆名字填注上面,列衔金吾卫衣左所副千户、山东等处提刑所理刑。又向来保道:“你二人替我进献生辰礼物,多有辛苦。”因问:“后边跪的是你甚么人?”来保才待说是伙计,那吴主管向前道:“小的是西门庆舅子,名唤吴典恩。”太师道:“你既是西门庆舅子,我观你倒好个仪表。”唤堂候官取过一张扎付:“我安你在本处清河县做个驿丞,倒也去的。”那吴典恩慌的磕头如捣蒜。又取过一张扎付来,把来保名字填写山东郓王府,做了一名校尉。俱磕头谢了,领了扎付。分咐明日早晨,吏、兵二部挂号,讨勘合,限日上任应役。又分咐翟谦西厢房管待酒饭,讨十两银子与他二人做路费,不在话下。

看官听说:那时徽宗,天下失政,奸臣当道,谗佞盈朝,高、杨、童、蔡四个奸党,在朝中卖官鬻狱,贿赂公行,悬秤升官,指方补价。夤缘钻刺者,骤升美任;贤能廉直者,经岁不除。以致风俗颓败,赃官污吏遍满天下,役烦赋兴,民穷盗起,天下骚然。不因奸臣居台辅,合是中原血染人。

当下翟谦把来保、吴主管邀到厢房管待,大盘大碗饱餐了一顿。翟谦向来保说:“我有一件事,央及你爹替我处处,未知你爹肯应承否?”来保道:“翟爹说那里话!蒙你老人家这等老爷前扶持看顾,不拣甚事,但肯分咐,无不奉命。”翟谦道:“不瞒你说,我答应老爷,每日止贱荆一人。我年将四十,常有疾病,身边通无所出。央及你爹,你那贵处有好人才女子,不拘十五六上下,替我寻一个送来。该多少财礼,我一一奉过去。”说毕,随将一封人事并回书付与来保,又送二人五两盘缠。来保再三不肯受,说道:“刚才老爷上已赏过了。翟爹还收回去。”翟谦道:“那是老爷的,此是我的,不必推辞。”当下吃毕酒饭,翟谦道:“如今我这里替你差个办事官,同你到下处,明早好往吏、兵二部挂号,就领了勘合,好起身。省的你明日又费往返了。我分咐了去,部里不敢迟滞你文书。”一面唤了个办事官,名唤李中友:“你与二位明日同到部里挂了号,讨勘合来回我话。”那员官与来保、吴典恩作辞,出的府门,来到天汉桥街上白酒店内会话。来保管待酒饭,又与了李中友三两银子,约定明日绝早先到吏部,然后到兵部,都挂号讨了勘合。闻得是太师老爷府里,谁敢迟滞,颠倒奉行。金吾卫太尉朱勔,即时使印,签了票帖,行下头司,把来保填注在本处山东郓王府当差。又拿了个拜帖,回翟管家。不消两日,把事情干得完备。有日雇头口起身,星夜回清河县来报喜。正是:

\[
富贵必因奸巧得,功名全仗邓通成。
\]

且说一日三伏天气,西门庆在家中聚景堂上大卷棚内,赏玩荷花,避暑饮酒。吴月娘与西门庆俱上坐,诸妾与大姐都两边列坐,春梅、迎春、玉箫、兰香,一般儿四个家乐在旁弹唱。怎见的当日酒席?但见:

\[
盆栽绿草,瓶插红花。水晶帘卷虾须,云母屏开孔雀。盘堆麟脯,佳人笑捧紫霞觞;盆浸冰桃,美女高擎碧玉斝。食烹异品,果献时新。弦管讴歌,奏一派声清韵美;绮罗珠翠,摆两行舞女歌儿。当筵象板撒红牙,遍体舞裙铺锦绣。消遣壶中闲日月,遨游身外醉乾坤。
\]

妻妾正饮酒中间,坐间不见了李瓶儿。月娘向绣春说道:“你娘往屋里做甚么哩?”绣春道:“我娘害肚里疼,\textuni{22C49}着哩。”月娘道:“还不快对他说去,休要\textuni{22C49}着,来这里听一回唱罢。”西门庆便问月娘:“怎的?”月娘道:“李大姐忽然害肚里疼,房里躺着哩。我使小丫头请他去了。”因向玉楼道:“李大姐七八临月,只怕搅撒了。”潘金莲道:“大姐姐,他那里是这个月?约他是八月里孩子,还早哩!”西门庆道:“既是早哩,使丫头请你六娘来听唱。”不一时,只见李瓶儿来到。月娘道:“只怕你掉了风冷气,你吃上锺热酒,管情就好了。”不一时,各人面前斟满了酒。西门庆分咐春梅:“你每唱个‘人皆畏夏日’我听。”那春梅等四个方才筝排雁柱,阮跨鲛绡,启朱唇,露皓齿,唱“人皆畏夏日”。那李瓶儿在酒席上,只是把眉头忔\textuni{3918}着,也没等的唱完,就回房中去了。月娘听了词曲,耽着心,使小玉房中瞧去。回来报说:“六娘害肚里疼,在炕上打滚哩。”慌了月娘道:“我说是时候,这六姐还强说早哩。还不唤小厮快请老娘去!”西门庆即令平安儿:“风跑!快请蔡老娘去!”于是连酒也吃不成,都来李瓶儿房中问他。

月娘问道:“李大姐,你心里觉的怎的?”李瓶儿回道:“大娘,我只心口连小肚子,往下鳖坠着疼。”月娘道:“你起来,休要睡着,只怕滚坏了胎。老娘请去了,便来也。”少顷,渐渐李瓶儿疼的紧了。月娘又问:“使了谁请老娘去了?这咱还不见来?”玳安道:“爹使来安去了。”月娘骂道:“这囚根子,你还不快迎迎去!平白没算计,使那小奴才去,有紧没慢的。”西门庆叫玳安快骑了骡子赶去。月娘道:“一个风火事,还象寻常慢条斯礼儿的。”那潘金莲见李瓶儿待养孩子,心中未免有几分气。在房里看了一回,把孟玉楼拉出来,两个站在西梢间檐柱儿底下那里歇凉,一处说话。说道:“耶嚛嚛!紧着热剌剌的挤了一屋子的人,也不是养孩子,都看着下象胆哩。”良久,只见蔡老娘进门,望众人道:“那位是主家奶奶?”李娇儿指着月娘道:“这位大娘哩。”那蔡老娘倒身磕头。月娘道:“姥姥,生受你。怎的这咱才来?请看这位娘子,敢待生养也?”蔡老娘向床前摸了摸李瓶儿身上,说道:“是时候了。”问:“大娘预备下绷接、草纸不曾?”月娘道:“有。”便叫小玉:“往我房中快取去!”

且说玉楼见老娘进门,便向金莲说:“蔡老娘来了,咱不往屋里看看去?”那金莲一面不是一面,说道:“你要看,你去。我是不看他。他是有孩子的姐姐,又有时运,人怎的不看他?头里我自不是,说了句话儿‘只怕是八月里的’,叫大姐姐白抢白相。我想起来好没来由,倒恼了我这半日。”玉楼道:“我也只说他是六月里孩子。”金莲道:“这回连你也韶刀了!我和你恁算:他从去年八月来,又不是黄花女儿,当年怀,入门养。一个婚后老婆,汉子不知见过了多少,也一两个月才生胎,就认做是咱家孩子?我说差了?若是八月里孩儿,还有咱家些影儿;若是六月的,踩小板凳儿糊险神道——还差着一帽头子哩!失迷了家乡,那里寻犊儿去?”正说着,只见小玉抱着草纸、绷接并小褥子儿来。孟玉楼道:“此是大姐姐自预备下他早晚用的,今日且借来应急儿。”金莲道:“一个是大老婆,一个是小老婆,明日两个对养,十分养不出来,零碎出来也罢。俺每是买了个母鸡不下蛋,莫不吃了我不成!”又道:“仰着合着,没的狗咬尿胞虚欢喜?”玉楼道:“五姐是甚么话!”以后见他说话不防头脑,只低着头弄裙带子,并不作声应答他。少顷,只见孙雪娥听见李瓶儿养孩子,从后边慌慌张张走来观看,不防黑影里被台基险些不曾绊了一交。金莲看见,教玉楼:“你看献勤的小妇奴才!你慢慢走,慌怎的?抢命哩!黑影子绊倒了,磕了牙也是钱!养下孩子来,明日赏你这小妇奴才一个纱帽戴!”良久,只听房里“呱”的一声养下来了。蔡老娘道:“对当家的老爹说,讨喜钱,分娩了一位哥儿。”吴月娘报与西门庆。西门庆慌忙洗手,天地祖先位下满炉降香,告许一百二十分清醮,要祈母子平安,临盆有庆,坐草无虞。这潘金莲听见生下孩子来了,合家欢喜,乱成一块,越发怒气,迳自去到房里,自闭门户,向床上哭去了。时宣和四年戊申六月念三日也。正是:

\[
不如意事常八九,可与人言无二三。
\]

蔡老娘收拾孩子,咬去脐带,埋毕衣胞,熬了些定心汤,打发李瓶儿吃了,安顿孩儿停当。月娘让老娘后边管待酒饭。临去,西门庆与了他五两一锭银子,许洗三朝来,还与他一匹缎子。这蔡老娘千恩万谢出门。

当日,西门庆进房去,见一个满抱的孩子,生的甚是白净,心中十分欢喜。合家无不欢悦。晚夕,就在李瓶儿房中歇了,不住来看孩儿。次日,巴天不明起来,拿十副方盒,使小厮各亲戚邻友处,分投送喜面。应伯爵、谢希大听见西门庆生了子,送喜面来,慌的两步做一步走来贺喜。西门庆留他卷棚内吃面。刚打发去了,正要使小厮叫媒人来寻养娘,忽有薛嫂儿领了个奶子来。原是小人家媳妇儿,年三十岁,新近丢了孩儿,不上一个月。男子汉当军,过不的,恐出征去无人养赡,只要六两银子卖他。月娘见他生的干净,对西门庆说,兑了六两银子留下,取名如意儿,教他早晚看奶哥儿。又把老冯叫来暗房中使唤,每月与他五钱银子,管顾他衣服。

正热闹一日,忽有平安报:“来保、吴主管在东京回还,见在门首下头口。”不一时,二人进来,见了西门庆报喜。西门庆问:“喜从何来?”二人悉把到东京见蔡太师进礼一节,从头至尾说道:“老爷见了礼物甚喜,说道:‘我累次受你主人之礼,无可补报。’朝廷钦赏了他几张空名诰身扎付,就与了爹一张,把爹名姓填注在金吾卫副千户之职,就委差在本处提刑所理刑,顶补贺老爷员缺。把小的做了铁铃卫校尉,填注郓王府当差。吴主管升做本县驿丞。”于是把一样三张印信扎付,并吏、兵二部勘合,并诰身都取出来,放在桌上与西门庆观看。西门庆看见上面衔着许多印信,朝廷钦依事例,果然他是副千户之职,不觉欢从额角眉尖出,喜向腮边笑脸生。便把朝廷明降,拿到后边与吴月娘众人观看,说:“太师老爷抬举我,升我做金吾卫副千户,居五品大夫之职。你顶受五花官诰,做了夫人。又把吴主管携带做了驿丞,来保做了郓王府校尉。吴神仙相我不少纱帽戴,有平地登云之喜,今日果然。不上半月,两椿喜事都应验了。”又对月娘说:“李大姐养的这孩子甚是脚硬,到三日洗了三,就起名叫做官哥儿罢。”来保进来,与月娘众人磕头,说了回话。分咐明日早把文书下到提刑所衙门里,与夏提刑知会了。吴主管明日早下文书到本县,作辞西门庆回家去了。

到次日,洗三毕,众亲邻朋友一概都知西门庆第六个娘子新添了娃儿,未过三日,就有如此美事,官禄临门,平地做了千户之职。谁人不来趋附?送礼庆贺,人来人去,一日不断头。常言:时来谁不来?时不来谁来!正是:

\[
时来顽铁有光辉,运退真金无颜色。
\]

\newpage
%# -*- coding:utf-8 -*-
%%%%%%%%%%%%%%%%%%%%%%%%%%%%%%%%%%%%%%%%%%%%%%%%%%%%%%%%%%%%%%%%%%%%%%%%%%%%%%%%%%%%%


\chapter{琴童儿藏壶构衅\KG 西门庆开宴为欢}


诗曰:

\[
幽情怜独夜,花事复相催。欲使春心醉,先教玉友来。
浓香犹带腻,红晕渐分腮。莫醒沉酣恨,朝云逐梦回。
\]

话说西门庆,次日使来保提刑所下文书。一面使人做官帽,又唤赵裁裁剪尺头,攒造圆领,又叫许多匠人,钉了七八条带。不说西门庆家中热乱,且说吴典恩那日走到应伯爵家,把做驿丞之事,再三央及伯爵,要问西门庆错银子,上下使用,许伯爵十两银子相谢,说着跪在地下。慌的伯爵拉起,说道:“此是成人之美,大官人携带你得此前程,也不是寻常小可。”因问:“你如今所用多少勾了?”吴典恩道:“不瞒老兄说,我家活人家,一文钱也没有。到明日上任参官贽见之礼,连摆酒,并治衣类鞍马,少说也得七八十两银子。如今我写了一纸文书此,也没敢下数儿。望老兄好歹扶持小人,事成恩有重报。”伯爵看了文书,因说:“吴二哥,你借出这七八十两银子来也不勾使。依我,取笔来写上一百两。恒是看我面,不要你利钱,你且得手使了。到明日做了官,慢慢陆续还他也不迟。俗语说得好:借米下得锅,讨米下不得锅。哄了一日是两晌。”吴典恩听了,谢了又谢。于是把文书上填写了一百两之数。

两个吃了茶,一同起身,来到西门庆门首。平安儿通报了,二人进入里面,见有许多裁缝匠人七手八脚做生活。西门庆和陈敬济在穿廊下,看着写见官手本揭帖,见二人,作揖让坐。伯爵问道:“哥的手本扎付,下了不曾?”西门庆道:“今早使小价往提刑府下扎付去了。还有东平府并本县手本,如今正要叫贲四去下。”说毕,画童儿拿上茶来。吃毕茶,那应伯爵并不提吴主管之事,走下来且看匠人钉带。西门庆见他拿起带来看,就卖弄说道:“你看我寻的这几条带如何?”伯爵极口称赞夸奖道:“亏哥那里寻的,都是一条赛一条的好带,难得这般宽大。别的倒也罢了,自这条犀角带并鹤顶红,就是满京城拿着银子也寻不出来。不是面奖,就是东京卫主老爷,玉带金带空有,也没这条犀角带。这是水犀角,不是旱犀角。旱犀角不值钱。水犀角号作通天犀。你不信,取一碗水,把犀角放在水内,分水为两处,此为无价之宝。”因问:“哥,你使了多少银子寻的?”西门庆道:“你们试估估价值。”伯爵道:“这个有甚行款,我每怎么估得出来!”西门庆道:“我对你说了罢,此带是大街上王昭宣府里的带。昨日一个人听见我这里要,巴巴来对我说。我着贲四拿了七十两银子,再三回了来。他家还张致不肯,定要一百两。”伯爵道:“难得这等宽样好看。哥,你明日系出去,甚是霍绰。就是你同僚间,见了也爱。”夸美了一回,坐下。西门庆便向吴主管问道:“你的文书下了不曾?”伯爵道:“吴二哥正要下文书,今日巴巴的央我来激烦你。蒙你照顾他往东京押生辰担,虽是太师与了他这个前程,就是你抬举他一般,也是他各人造化。说不的,一品至九品都是朝廷臣子。但他告我说,如今上任,见官摆酒,并治衣服之类,共要许多银子使,那处活变去?一客不烦二主,没奈何,哥看我面,有银子借与他几两,率性周济了这些事儿。他到明日做上官,就衔环结草也不敢忘了哥大恩!休说他旧在哥门下出入,就是外京外府官吏,哥也不知拔济了多少。不然,你教他那里区处去?”因说道:“吴二哥,你拿出那符儿来,与你大官人瞧。”这吴典恩连忙向怀中取出,递与西门庆观看。见上面借一百两银子,中人就是应伯爵,每月利行五分。西门庆取笔把利钱抹了,说道:“既是应二哥作保,你明日只还我一百两本钱就是了。我料你上下也得这些银子搅缠。”于是把文书收了。才待后边取银子去,忽有夏提刑拿帖儿差了一名写字的,拿手本三班送了二十名排军来答应,就问讨上任日期,讨问字号,衙门同僚具公礼来贺。西门庆教阴阳徐先生择定七月初二日辰时到任,拿帖儿回夏提刑,赏了写字的五钱银子。正打发出门去了,只见陈敬济拿着一百两银子出来,教与吴主管,说:“吴二哥,你明日只还我本钱便了。”那吴典恩拿着银子,欢喜出门。看官听说:后来西门庆死了,家中时败势衰,吴月娘守寡,被平安儿偷盗出解当库头面,在南瓦子里宿娼,被吴驿丞拿住,教他指攀吴月娘与玳安有奸,要罗织月娘出官,恩将仇报。此系后事,表过不题。正是:

\[
不结子花休要种,无义之人不可交。
\]

那时贲四往东平府并本县下了手本来回话,西门庆留他和应伯爵,陪阴阳徐先生摆饭。正吃着饭,只见吴大舅来拜望,徐先生就起身。良久,应伯爵也作辞出门,来到吴主管家。吴典恩早封下十两保头钱,双手递与伯爵,磕下头去。伯爵道:“若不是我那等取巧说着,会胜不肯与借与你。”吴典恩酬谢了伯爵,治办官带衣类,择日见官上任不题。

那时本县正堂李知县,会了四衙同僚,差人送羊酒贺礼来,又拿帖儿送了一名小郎来答应。年方一十八岁,本贯苏州府常熟县人,唤名小张松。原是县中门子出身,生得清俊,面如傅粉,齿白唇红;又识字会写,善能歌唱南曲;穿着青绡直缀,凉鞋净袜。西门庆一见小郎伶俐,满心欢喜,就拿拜帖回覆李知县,留下他在家答应,改唤了名字叫作书童儿。与他做了一身衣服,新鞋新帽,不教他跟马,教他专管书房,收礼帖,拿花园门钥匙。祝实念又举保了一个十四岁小厮来答应,亦改名棋童,每日派定和琴童儿两个背书袋、夹拜帖匣跟马。

到了上任日期,在衙门中摆大酒席桌面,出票拘集三院乐工承应吹打弹唱。此时李铭也夹在中间来了,后堂饮酒,日暮时分散归。每日骑着大白马,头戴乌纱,身穿五彩洒线揉头狮子补子员领,四指大宽萌金茄楠香带,粉底皂靴,排军喝道,张打着大黑扇,前呼后拥,何止十数人跟随,在街上摇摆。上任回来,先拜本府县帅府都监,并清河左右卫同僚官,然后新朋邻舍,何等荣耀施为!家中收礼接帖子,一日不断。正是:

\[
白马红缨色色新,不来亲者强来亲。
时来顽铁生光彩,运去良金不发明。
\]

西门庆自从到任以来,每日坐提刑院衙门中,升厅画卯,问理公事。光阴迅速,不觉李瓶儿坐褥一月将满。吴大妗子、二妗子、杨姑娘、潘姥姥、吴大姨、乔大户娘子,许多亲邻堂客女眷,都送礼来,与官哥儿做弥月。院中李桂姐、吴银儿见西门庆做了提刑所千户,家中又生了子,亦送大礼,坐轿子来庆贺。西门庆那日在前边大厅上摆设筵席,请堂客饮酒。春梅、迎春、玉箫、兰香都打扮起来,在席前斟酒执壶。

原来西门庆每日从衙门中来,只到外边厅上就脱了衣服,教书童叠了,安在书房中,止带着冠帽进后边去。到次日起来,旋使丫鬟来书房中取。新近收拾大厅西厢房一间做书房,内安床几、桌椅、屏帏、笔砚、琴书之类。书童儿晚夕只在床脚踏板上铺着铺睡。西门庆或在那房里歇,早晨就使出那房里丫鬟来前边取衣服。取来取去,不想这小郎本是门子出身,生的伶俐清俊,与各房丫头打牙犯嘴惯熟,于是暗和上房里玉箫两个嘲戏上了。那日也是合当有事,这小郎正起来,在窗户台上搁着镜儿梳头,拿红绳扎头发。不料玉箫推开门进来,看见说道:“好贼囚,你这咱还描眉画眼的,爹吃了粥便出来。”书童也不理,只顾扎包髻儿。玉箫道:“爹的衣服叠了,在那里放着哩?”书童道:“在床南头安放着哩。”玉箫道:“他今日不穿这一套。分咐我教问你要那件玄色\textuni{20DE8}金补子、丝布员领、玉色衬衣穿。”书童道:“那衣服在厨柜里。我昨日才收了,今日又要穿他。姐,你自开门取了去。”那玉箫且不拿衣服,走来跟前看着他扎头,戏道:“怪贼囚,也象老婆般拿红绳扎着头儿,梳的鬓虚笼笼的!”因见他白滚纱漂白布汗褂儿上系着一个银红纱香袋儿,一个绿纱香袋儿,就说道:“你与我这个银红的罢!”书童道:“人家个爱物儿,你就要。”玉箫道:“你小厮家带不的这银红的,只好我带。”书童道:“早是这个罢了,倘是个汉子儿,你也爱他罢?”被玉箫故意向他肩膀上拧了一把,说道:“贼囚,你夹道卖门神——看出来的好画儿。”不由分说,把两个香袋子等不的解,都揪断系儿,放在袖子内。书童道:“你子不尊贵,把人的带子也揪断。”被玉箫发讪,一拳一把,戏打在身上。打的书童急了,说:“姐,你休鬼混我,待我扎上这头发着!”玉箫道:“我且问你,没听见爹今日往那去?”书童道:“爹今日与县中华主簿老爹送行,在皇庄薛公公那里摆酒,来家只怕要下午时分,又听见会下应二叔,今日兑银子,要买对门乔大户家房子,那里吃酒罢了。”玉箫道:“等住回,你休往那去了,我来和你说话。”书童道:“我知道。”玉箫于是与他约会下,才拿衣服往后边去了。

少顷,西门庆出来,就叫书童,分咐:“在家,别往那去了,先写十二个请帖儿,都用大红纸封套,二十八日请官客吃庆官哥儿酒;教来兴儿买办东西,添厨役茶酒,预备桌面齐整;玳安和两名排军送帖儿,叫唱的;留下琴童儿在堂客面前管酒。”分咐毕,西门庆上马送行去了。吴月娘众姊妹,请堂客到齐了,先在卷棚摆茶,然后大厅上屏开孔雀,褥隐芙蓉,上坐。席间叫了四个妓女弹唱。果然西门庆到午后时分来家,家中安排一食盒酒菜,邀了应伯爵和陈敬济,兑了七百两银子,往对门乔大户家成房子去了。

堂客正饮酒中间,只见玉箫拿下一银执壶酒并四个梨、一个柑子,迳来厢房中送与书童儿吃。推开门,不想书童儿不在里面,恐人看见,连壶放下,就出来了。可霎作怪,琴童儿正在上边看酒,冷眼睃见玉箫进书房里去,半日出来,只知有书童儿在里边,三不知叉进去瞧。不想书童儿外边去,不曾进来,一壶热酒和果子还放在床底下。这琴童连忙把果子藏在袖里,将那一壶酒,影着身子,一直提到李瓶儿房里。只见奶子如意儿和绣春在屋里看哥儿。琴童进门就问:“姐在那里?”绣春道:“他在上边与娘斟酒哩。你问他怎的?”琴童儿道:“我有个好的儿,教他替我收着。”绣春问他甚么,他又不拿出来。正说着,迎春从上边拿下一盘子烧鹅肉、一碟玉米面玫瑰果馅蒸饼儿与奶子吃,看见便道:“贼囚,你在这里笑甚么,不在上边看酒?”那琴童方才把壶从衣裳底下拿出来,教迎春:“姐,你与我收了。”迎春道:“此是上边筛酒的执壶,你平白拿来做甚么?”琴童道:“姐,你休管他。此是上房里玉箫,和书童儿小厮,七个八个,偷了这壶酒和些柑子、梨,送到书房中与他吃。我赶眼不见,戏了他的来。你只与我好生收着,随问甚么人来抓寻,休拿出来。我且拾了白财儿着!”因把梨和柑子掏出来与迎春瞧,迎春道:“等住回抓寻壶反乱,你就承当?”琴童道:“我又没偷他的壶。各人当场者乱,隔壁心宽,管我腿事!”说毕,扬长去了。迎春把壶藏放在里间桌子上,不题。

至晚,酒席上人散,查收家火,少了一把壶。玉箫往书房中寻,那里得来!问书童,说:“我外边有事去,不知道。”那玉箫就慌了,一口推在小玉身上。小玉骂道:“\textuni{34B2}昏了你这淫妇!我后边看茶,你抱着执壶,在席间与娘斟酒。这回不见了壶儿,你来赖我!”向各处都抓寻不着。良久,李瓶儿到房来,迎春如此这般告诉:“琴童儿拿了一把进来,教我替他收着。”李瓶儿道:“这囚根子,他做甚么拿进来?后边为这把壶好不反乱,玉箫推小玉,小玉推玉箫,急得那大丫头赌身发咒,只是哭。你趁早还不快送进去哩,迟回管情就赖在你这小淫妇儿身上。”那迎春方才取出壶,送入后边来。后边玉箫和小玉两个,正嚷到月娘面前。月娘道:“贼臭肉,还敢嚷些甚么?你每管着那一门儿?把壶不见了!”玉箫道:“我在上边跟着娘送酒,他守着银器家火。不见了,如今赖我。”小玉道:“大妗子要茶,我不往后边替他取茶去?你抱着执壶儿,怎的不见了?敢屁股大——吊了心也怎的?”月娘道:“今日席上再无闲杂人,怎的不见了东西?等住回你主子来,没这壶,管情一家一顿。”

正乱着,只见西门庆自外来,问:“因甚嚷乱?”月娘把不见壶一节说了一遍。西门庆道:“慢慢寻就是了,平白嚷的是些甚么?”潘金莲道:“若是吃一遭酒,不见了一把,不嚷乱,你家是王十万!头醋不酸,到底儿薄。”看官听说:金莲此话,讥讽李瓶儿首先生孩子,满月就不见了壶,也是不吉利。西门庆明听见,只不做声。只见迎春送壶进来。玉箫便道:“这不是壶有了。”月娘问迎春:“这壶端的往那里来?”迎春悉把琴童从外边拿到我娘屋里收着,不知在那里来。月娘因问:“琴童儿那奴才,如今在那里?”玳安道:“他今日该狮子街房子里上宿去了。”金莲在旁不觉鼻子里笑了一声。西门庆便问:“你笑怎的?”金莲道:“琴童儿是他家人,放壶他屋里,想必要瞒昧这把壶的意思。要叫我,使小厮如今叫将那奴才来,老实打着,问他个下落。不然,头里就赖着他那两个,正是走杀金刚坐杀佛!”西门庆听了,心中大怒,睁眼看着金莲,说道:“依着你恁说起来,莫不李大姐他爱这把壶?既有了,丢开手就是了,只管乱甚么!”那金莲把脸羞的飞红了,便道:“谁说姐姐手里没钱。”说毕,走过一边使性儿去了。

西门庆就有陈敬济进来说话。金莲和孟玉楼站在一处,骂道:“恁不逢好死,三等九做贼强盗!这两日作死也怎的?自从养了这种子,恰似生了太子一般,见了俺每如同生刹神一般,越发通没句好话儿说了,行动就睁着两个\textuni{23B48}窟窿吆喝人。谁不知姐姐有钱,明日惯的他每小厮丫头养汉做贼,把人说遍了,也休要管他!”说着,只见西门庆与陈敬济说了一回话,就往前边去了。孟玉楼道:“你还不去,他管情往你屋里去了。”金莲道:“可是他说的,有孩子屋里热闹,俺每没孩子的屋里冷清。”正说着,只见春梅从外走来。玉楼道:“我说他往你屋里去了,你还不信,这不是春梅叫你来了。”一面叫过春梅来问。春梅道:“我来问玉箫要汗巾子来。”玉楼问道:“你爹在那里?”春梅道:“爹往六娘房里去了。”这金莲听了,心上如撺上把火相似,骂道:“贼强人,到明日永世千年,就跌折脚,也别要进我那屋里!踹踹门槛儿,教那牢拉的囚根子把踝子骨\textuni{22C49}折了!”玉楼道:“六姐,你今日怎的下恁毒口咒他?”金莲道:“不是这等说,贼三寸货强盗,那鼠腹鸡肠的心儿,只好有三寸大一般。都是你老婆,无故只是多有了这点尿胞种子罢了,难道怎么样儿的!做甚么恁抬一个灭一个,把人躧到泥里!”正是:

\[
大风刮倒梧桐树,自有旁人说短长。
\]

这里金莲使性儿不题。且说西门庆走到前边,薛大监差了家人,送了一坛内酒、一牵羊、两匹金缎、一盘寿桃、一盘寿面、四样嘉肴,一者祝寿,二者来贺。西门庆厚赏来人,打发去了。到后边,有李桂姐、吴银儿两个拜辞要家去。西门庆道:“你每两个再住一日儿,到二十八日,我请许多官客,有院中杂耍扮戏的,教你二位只管递酒。”桂姐道:“既留下俺每,我教人家去回妈声,放心些。”于是把两人轿子都打发去了,不在话下。

次日,西门庆在大厅上锦屏罗列,绮席铺陈,请官客饮酒。因前日在皇庄见管砖厂刘公公,故与薛内相都送了礼来。西门庆这里发柬请他,又邀了应伯爵、谢希大两个相陪。从饭时,二人衣帽齐整,又早先到了。西门庆让他卷棚内待茶。伯爵因问:“今日,哥席间请那几客?”西门庆道:“有刘、薛二内相,帅府周大人,都监荆南江,敝同僚夏提刑,团练张总兵,卫上范千户,吴大哥,吴二哥。乔老便今日使人来回了不来。连二位通只数客。”说毕,适有吴大舅、二舅到,作了揖,同坐下,左右放桌儿摆饭。吃毕,应伯爵因问:“哥儿满月抱出来不曾?”西门庆道:“也是因众堂客要看,房下说且休教孩儿出来,恐风试着他,他奶子说不妨事。教奶子用被裹出来,他大妈屋里走了遭,应了个日子儿,就进屋去了。”伯爵道:“那日嫂子这里请去,房下也要来走走,百忙里旧疾又举发了,起不得炕儿,心中急的要不的。如今趁人未到,哥倒好说声,抱哥儿出来,俺每同看一看。”西门庆一面分咐后边:“慢慢抱哥儿出来,休要唬着他。对你娘说,大舅、二舅在这里,和应二爹、谢爹要看一看。”月娘教奶子如意儿用红绫小被儿裹的紧紧的,送到卷棚角门首,玳安儿接抱到卷棚内。众人观看,官哥儿穿着大红缎毛衫儿,生的面白唇红,甚是富态,都夸奖不已。吴大舅、二舅与希大每人袖中掏出一方锦缎兜肚,上带着一个小银坠儿;惟应伯爵是一柳五色线,上穿着十数文长命钱。教与玳安儿好生抱回房去,休要惊唬哥儿,说道:“相貌端正,天生的就是个戴纱帽胚胞儿。”西门庆大喜,作揖谢了。

说话中间,忽报刘公公、薛公公来了。慌的西门庆穿上衣,仪门迎接。二位内相坐四人轿,穿过肩蟒,缨枪排队,喝道而至。西门庆先让至大厅上拜见,叙礼接茶。落后周守备、荆都监、夏提刑等众武官都是锦绣服,藤棍大扇,军牢喝道。须臾都到了门首,黑压压的许多伺候。里面鼓乐喧天,笙歌迭奏。西门庆迎入,与刘、薛二内相相见。厅正面设十二张桌席。西门庆就把盏让坐。刘、薛二内再三让逊道:“还有列位。”只见周守备道:“二位老太监齿德俱尊。常言:三岁内宦,居冠王公之上。这个自然首坐,何消泛讲。”彼此让逊了一回。薛内相道:“刘哥,既是列位不肯,难为东家,咱坐了罢。”于是罗圈唱了个喏,打了恭,刘内相居左,薛内相居右,每人膝下放一条手巾,两个小厮在旁打扇,就坐下了。其次者才是周守备、荆都监众人。须臾阶下一派箫韶,动起乐来。当日这筵席,说不尽食烹异品,果献时新。须臾酒过五巡,汤陈三献,教坊司俳官簇拥一段笑乐院本上来。正是:

\[
百宝妆腰带,珍珠络臂鞲。
笑时能近眼,舞罢锦缠头。
\]

笑院本扮完下去,就是李铭、吴惠两个小优儿上来弹唱。一个\textuni{22E88}筝,一个琵琶。周守备先举手让两位内相,说:“老太监分咐,赏他二人唱那套词儿?”刘太监道:“列位请先。”周守备道:“老太监,自然之理,不必过谦。”刘太监道:“两个子弟唱个‘叹浮生有如一梦里’。”周守备道:“老太监,此是归隐叹世之辞,今日西门庆大人喜事,又是华诞,唱不的。”刘太监又道:“你会唱‘虽不是八位中紫绶臣,管领的六宫中金钗女’?”周守备道:“此是《陈琳抱妆盒》杂记,今日庆贺,唱不的。”薛太监道:“你叫他二人上来,等我分咐他。你记的《普天乐》‘想人生最苦是离别’?”夏提刑大笑道:“老太监,此是离别之词,越发使不的。”薛太监道:“俺每内官的营生,只晓的答应万岁爷,不晓得词曲中滋味,凭他每唱罢。”夏年刑终是金吾执事人员,倚仗他刑名官,遂分咐:“你唱套《三十腔》。今日是你西门老爹加官进禄,又是好日子,又是弄璋之喜,宜该唱这套。”薛内相问:“怎的是弄璋之喜?”周守备道:“二位老太监,此日又是西门大人公子弥月之辰,俺每同僚都有薄礼庆贺。”薛内相道:“这等——”因向刘太监道:“刘家,咱每明日都补礼来庆贺。”西门庆谢道:“学生生一豚犬,不足为贺,到不必老太监费心。”说毕,唤玳安里边叫出吴银儿、李桂姐,席前递酒。两个唱的打扮出来,花枝招展,望上插烛也似磕了四个头儿,起来执壶斟酒,逐一敬奉。两个乐工,又唱一套新词,歌喉宛转,真有绕梁之声。当夜前歌后舞,锦簇花攒,直饮至更余时分,薛内相方才起身,说道:“生等一者过蒙盛情,二者又值喜庆,不觉留连畅饮,十分扰极,学生告辞。”西门庆道:“杯茗相邀,得蒙光降,顿使蓬荜增辉,幸再宽坐片时,以毕余兴。”众人俱出位说道:“生等深扰,酒力不胜。”各躬身施礼相谢。西门庆再三款留不住,只得同吴大舅、二舅等,一齐送至大门。一派鼓乐喧天,两边灯火灿烂,前遮后拥,喝道而去。正是,得多少:

\[
歌舞欢娱嫌日短,故烧高烛照红妆。
\]

\newpage
%# -*- coding:utf-8 -*-
%%%%%%%%%%%%%%%%%%%%%%%%%%%%%%%%%%%%%%%%%%%%%%%%%%%%%%%%%%%%%%%%%%%%%%%%%%%%%%%%%%%%%


\chapter{李桂姐趋炎认女\KG 潘金莲怀妒惊儿}


诗曰:

\[
牛马鸣上风,声应在同类。小人非一流,要呼各相比。
吹彼埙与篪,翕翕骋志意。愿游广漠乡,举手谢时辈。
\]

话说当日众官饮酒席散,西门庆还留吴大舅、二舅、应伯爵、谢希大后坐。打发乐工等酒饭吃了,分咐:“你每明日还来答应一日,我请县中四宅老爹吃酒,俱要齐备些。临了一总赏你每罢。”众乐工道:“小的每无不用心,明日都是官样新衣服来答应。”吃了酒饭,磕头去了。良久,李桂姐、吴银儿搭着头出来,笑嘻嘻道:“爹,晚了,轿子来了,俺每去罢。”应伯爵道:“我儿,你倒且是自在。二位老爹在这里,不说唱个曲儿与老爹听,就要去罢?”桂姐道:“你不说这一声儿,不当哑狗卖。俺每两日没往家去,妈不知怎么盼哩。”伯爵道:“盼怎的?玉黄李子儿,掐了一块儿去了?”西门庆道:“也罢,教他两个去罢,本等连日辛苦了。咱叫李铭、吴惠唱罢。”问道:“你吃了饭了?”桂姐道:“刚才大娘留俺每吃了。”于是齐磕头下去。西门庆道:“你二位后日还来走走,再替我叫两个,不拘郑爱香儿也罢,韩金钏儿也罢,我请亲朋吃酒。”伯爵道:“造化了小淫妇儿,教他叫,又讨提钱使。”桂姐道:“你又不是架儿,你怎晓得恁切?”说毕,笑的去了。伯爵因问:“哥,后日请谁?”西门庆道:“那日请乔老、二位老舅、花大哥、沈姨夫,并会中列位兄弟,欢乐一日。”伯爵道:“说不得,俺每打搅得哥忒多了。到后日,俺两个还该早来,与哥做副东。”西门庆道:“此是二位下顾了。”说毕话,李铭、吴惠拿乐器上来,唱了一套。吴大舅等众人方一齐起身。一宿晚景不题。

到次日,西门庆请本县四宅官员。那日薛内相来的早,西门庆请至卷棚内待茶。薛内相因问:“刘家没送礼来?”西门庆道:“刘老太监送过礼了。”良久,薛内相要请出哥儿来看一看:“我与他添寿。”西门庆推却不得,只得教玳安后边说去,抱哥儿出来。不一时,养娘抱官哥送出到角门首,玳安接到上面。薛内相看见,只顾喝采:“好个哥儿!”便叫:“小厮在那里?”须臾,两个青衣家人,戢金方盒拿了两盒礼物:\textHuoShan 红官缎一匹,福寿康宁镀金银钱四个,追金沥粉彩画寿星博郎鼓儿一个,银八宝贰两。说道:“穷内相没什么,这些微礼儿与哥儿耍子。”西门庆作揖谢道:“多蒙老公公费心。”看毕,抱哥儿回房不题。西门庆陪着吃了茶,就先摆饭。刚才吃罢,忽报:“四宅老爹到了。”西门庆忙整衣冠,出二门迎接。乃是知县李达天,并县丞钱成、主簿任廷贵、典史夏恭基。各先投拜帖,然后厅上叙礼。请薛内相出见,众官让薛内相坐首席。席间又有尚举人相陪。分宾坐定,普坐递了一巡茶。少顷,阶下鼓乐响动,笙歌拥奏,递酒上坐。教坊呈上揭帖。薛内相拣了四摺《韩湘子升仙记》,又队舞数回,十分齐整。薛内相心中大喜,唤左右拿两吊钱出来,赏赐乐工。

不说当日众官饮酒至晚方散,且说李桂姐到家,见西门庆做了提刑官,与虔婆铺谋定计。次日,买了四色礼,做了一双女鞋,教保儿挑着盒担,绝早坐轿子先来,要拜月娘做干娘。进来先向月娘笑嘻嘻拜了四双八拜,然后才与他姑娘和西门庆磕头。把月娘哄的满心欢喜,说道:“前日受了你妈的重礼,今日又教你费心,买这许多礼来。”桂姐笑道:“妈说,爹如今做了官,比不得那咱常往里边走。我情愿只做干女儿罢,图亲戚来往,宅里好走动。”月娘忙教他脱衣服坐的,因问:“吴银姐和那两个怎的还不来?”桂姐道:“吴银儿,我昨日会下他,不知怎的还不见来。前日爹分咐教我叫了郑爱香儿和韩金钏儿,我来时他轿子都在门首,怕不也待来。”言未了,只见银儿和爱香儿,又与一个穿大红纱衫年小的粉头,提着衣裳包儿进来,先望月娘磕了头。吴银儿看见李桂姐脱了衣裳,坐在炕上,说道:“桂姐,你好人儿!不等俺每等儿,就先来了。”桂姐道:“我等你来,妈见我的轿子在门首,说道:‘只怕银姐先去了,你快去罢。’谁知你每来的迟。”月娘笑道:“也不迟。”因问:“这位姐儿上姓?”吴银儿道:“他是韩金钏儿的妹子玉钏儿。”不一时,小玉放桌儿,摆了八碟茶食,两碟点心,打发四个唱的吃了。那李桂姐卖弄他是月娘干女儿,坐在月娘炕上,和玉箫两个剥果仁儿、装果盒。吴银儿三个在下边杌儿上,一条边坐的。那桂姐一径抖搜精神,一回叫:“玉箫姐,累你,有茶倒一瓯子来我吃。”一回又叫:“小玉姐,你有水盛些来,我洗这手。”那小玉真个拿锡盆舀了水,与他洗手。吴银儿众人都看的睁睁的,不敢言语。桂姐又道:“银姐,你三个拿乐器来唱个曲儿与娘听。我先唱过了。”月娘和李娇儿对面坐着。吴银儿见他这般说,只得取过乐器来。当下郑爱香儿弹筝,吴银儿琵琶,韩玉钏儿在旁随唱,唱了一套《八声甘州》“花遮翠楼”。

须臾唱毕,放下乐器。吴银儿先问月娘:“爹今日请那几位官客吃酒?”月娘道:“你爹今日请的都是亲朋。”桂姐道:“今日没有请那两位公公?”月娘道:“今日没有,昨日也只薛内相一位。那姓刘的没来。”桂姐道:“刘公公还好,那薛公公惯顽,把人掐拧的魂也没了。”月娘道:“左右是个内官家,又没什么,随他摆弄一回子就是了。”桂姐道:“娘且是说的好,乞他奈何的人慌。”正说着,只见玳安儿进来取果盒,见他四个在屋里坐着,说道:“客已到了一半,七八待上坐,你每还不快收拾上去?”月娘便问:“前边有谁来了?”玳安道:“乔大爹、花大爹、大舅、二舅、谢爹都来了这一日了。”桂姐问道:“今日有应二花子和祝麻子二人没有?”玳安道:“会中十位,一个儿也不少。应二爹从辰时就来了,爹使他有勾当去了,便道就来也。”桂姐道:“爷嚛!遭遭儿有这起攮刀子的,又不知缠到多早晚。我今日不出去,宁可在屋里唱与娘听罢。”玳安道:“你倒且是自在性儿。”拿出果盒去了。桂姐道:“娘还不知道,这祝麻子在酒席上,两片子嘴不住,只听见他说话,饶人那等骂着,他还不理。他和孙寡嘴两个好不涎脸。”郑爱香儿道:“常和应二走的那祝麻子,他前日和张小二官儿到俺那里,拿着十两银子,要请俺家妹子爱月儿。俺妈说:‘他才教南人梳弄了,还不上一个月,南人还没起身,我怎么好留你?’说着他再三不肯。缠的妈急了,把门倒插了,不出来见他。那张二官儿好不有钱,骑着大白马,四五个小厮跟随,坐在俺每堂屋里只顾不去。急的祝麻了直撅儿跪在天井内,说道:‘好歹请出妈来,收了这银子。只教月姐儿一见,待一杯茶儿,俺每就去。’把俺每笑的要不的。只象告水灾的,好个涎脸的行货子!”吴银儿道:“张小二官儿先包着董猫儿来。”郑爱香儿道:“因把猫儿的虎口内火烧了两醮,和他丁八着好一向了,这日才散走了。”因望着桂姐道:“昨日我在门外会见周肖儿,多上覆你,说前日同聂钺儿到你家,你不在。”桂姐使了个眼色,说道:“我到爹宅里来,他请了俺姐姐桂卿了。”郑爱香儿道:“你和他没点儿相交,如何却打热?”桂姐道:“好\textuni{34B2}的刘九儿,把他当个孤老,甚么行货子,可不砢磪杀我罢了。他为了事出来,逢人至人说了来,嗔我不看他。妈说:‘你只在俺家,俺倒买些什么看看你不打紧。你和别人家打热,俺傻的不匀了。’真是硝子石望着南儿——丁口心!”说着都一齐笑了。月娘坐在炕上听着他说,道:“你每说了这一日,我不懂,不知说的是那家话!”按下这里不题。

却说前边各客都到齐了,西门庆冠冕着递酒。众人让乔大户为首,先与西门庆把盏。只见他三个唱的从后边出来,都头上珠冠蹀躞,身边兰麝浓香。应伯爵一见,戏道:“怎的三个零布在那里来?拦住,休放他进来!”因问:“东家,李家桂儿怎不来?”西门庆道:“我不知道。”初是郑爱香儿弹筝,吴银儿琵琶,韩金钏儿拨板。启朱唇,露皓齿,先唱《水仙子》“马蹄金铸就虎头牌”一套。良久,递酒毕,乔大户坐首席,其次者吴大舅、二舅、花大哥、沈姨夫、应伯爵、谢希大、孙寡嘴、祝实念、常峙节、白赉光、傅自新、贲第传,共十四人上席,八张桌儿。西门庆下席主位。说不尽歌喉宛转,舞态蹁跹,酒若流波,肴如山叠。到了那酒过数巡,歌吟三套之间,应伯爵就在席上开口说道:“东家,也不消教他每唱了,翻来吊过去,左右只是这两套狗挝门的,谁待听!你教大官儿拿三个座儿来,教他与列位递酒,倒还强似唱。”西门庆道:“且教他孝顺众尊亲两套词儿着。你这狗才,就这等摇席破座的。”郑爱香儿道:“应花子,你门背后放花儿——等不到晚了!”伯爵亲自走下席来骂道:“怪小淫妇儿,什么晚不晚?你娘那毴!”教玳安:“过来,你替他把刑法多拿了。”一手拉着一个,都拉到席上,教他递酒。郑爱香儿道:“怪行货子,拉的人手脚儿不着地。”伯爵道:“我实和你说,小淫妇儿,时光有限了,不久青刀马过,递了酒罢,我等不的了。”谢希大便问:“怎么是青刀马?”伯爵道:“寒鸦儿过了,就是青刀马。”众人都笑了。

当下吴银儿递乔大户,郑爱香儿递吴大舅,韩玉钏儿递吴二舅,两分头挨次递将来。落后吴银儿递到应伯爵跟前,伯爵因问:“李家桂儿怎的不来?”吴银儿道:“你老人家还不知道,李桂姐如今与大娘认义做干女儿。我告诉二爹,只放在心里。却说人弄心,前日在爹宅里散了,都一答儿家去了,都会下了明日早来。我在家里收拾了,只顾等他。谁知他安心早买了礼,就先来了,倒教我等到这咱晚。使丫头往他家瞧去,说他来了,好不教妈说我。你就拜认与爹娘做干女儿,对我说了便怎的?莫不搀了你什么分儿?瞒着人干事。嗔道他头里坐在大娘炕上,就卖弄显出他是娘的干女儿,剥果仁儿,定果盒,拿东拿西,把俺每往下躧。我还不知道,倒是里边六娘刚才悄悄对我说,他替大娘做了一双鞋,买了一盒果馅饼儿,两只鸭子,一大副膀蹄,两瓶酒,老早坐了轿子来。”从头至尾告诉一遍。伯爵听了道:“他如今在这里不出来,不打紧,我务要奈何那贼小淫妇儿出来。我对你说罢,他想必和他鸨子计较了,见你大爹做了官,又掌着刑名,一者惧怕他势要,二者恐进去稀了,假着认干女儿往来,断绝不了这门儿亲。我猜的是不是?我教与你个法儿,他认大娘做干女,你到明日也买些礼来,却认与六娘做干女儿就是了。你和他都还是过世你花爹一条路上的人,各进其道就是了。我说的是不是?你也不消恼他。”吴银儿道:“二爹说的是,我到家就对妈说。”说毕,递过酒去,就是韩玉钏儿,挨着来递酒。伯爵道:“韩玉姐起动起动,不消行礼罢。你姐姐家里做什么哩?”玉钏儿道:“俺姐姐家中有人包着哩,好些时没出来供唱。”伯爵道:“我记的五月里在你那里打搅了,再没见你姐姐。”韩玉钏道:“那日二爹怎的不肯深坐,老早就去了?”伯爵道:“不是那日我还坐,坐中有两个人不合节,又是你大老爹这里相招,我就先走了。”韩玉钏儿见他吃过一杯,又斟出一杯。伯爵道:“罢罢,少斟些,我吃不得了!”玉钏道:“二爹你慢慢上,上过待我唱曲儿你听。”伯爵道:“我的姐姐,谁对你说来?正可着我心坎儿。常言道:养儿不要屙金溺银,只要见景生情。倒还是丽春院娃娃,到明日不愁没饭吃,强如郑家那贼小淫妇,\textShouWai 剌骨儿,只躲滑儿,再不肯唱。”郑爱香儿道:“应二花子,汗邪了你,好骂!”西门庆道:“你这狗才,头里嗔他唱,这回又索落他。”伯爵道:“这是头里帐,如今递酒,不教他唱个儿?我有三钱银子,使的那小淫妇鬼推磨。”韩玉钏儿不免取过琵琶来,席上唱了个小曲儿。

伯爵因问主人:“今日李桂姐儿怎的不教他出来?”西门庆道:“他今日没来。”伯爵道:“我才听见后边唱。就替他说谎!”因使玳安:“好歹后边快叫他出来。”那玳安儿不肯动,说:“这应二爹错听了,后边是女先生郁大姐弹唱与娘每听来。”伯爵道:“贼小油嘴还哄我!等我自家后边去叫。”祝实念便向西门庆道:“哥,也罢,只请李桂姐来,与列位老亲递杯酒来,不教他唱也罢。我晓得,他今日人情来了。”西门庆被这起人缠不过,只得使玳安往后边请李桂姐去。那李桂姐正在月娘上房弹着琵琶,唱与大妗子、杨姑娘、潘姥姥众人听,见玳安进来叫他,便问:“谁使你来?”玳安道:“爹教我来,请桂姨上去递一巡酒。”桂姐道:“娘,你看爹韶刀,头里我说不出去,又来叫我!”玳安道:“爹被众人缠不过,才使进我来。”月娘道:“也罢,你出去递巡酒儿,快下来就了。”桂姐又问玳安:“真个是你爹叫,我便出去;若是应二花子,随问他怎的叫,我一世也不出去。”于是向月娘镜台前,重新装点打扮出来。众人看见他头戴银丝\textuni{4BFC}髻,周围金累丝钗梳,珠翠堆满,上着藕丝衣裳,下着翠绫裙,尖尖趫趫一对红鸳,粉面贴着三个翠面花儿。一阵异香喷鼻,朝上席不端不正只磕了一个头。就用洒金扇儿掩面,佯羞整翠,立在西门庆面前。西门庆分咐玳安,放锦杌儿在上席,教他与乔大户上酒。乔大户倒忙欠身道:“倒不消劳动,还有列位尊亲。”西门庆道:“先从你乔大爹起。”这桂姐于是轻摇罗袖,高捧金樽,递乔大户酒。伯爵在旁说道:“乔上尊,你请坐,交他侍立。丽春院粉头供唱递酒是他的职分,休要惯了他。”乔大户道:“二老,此位姐儿乃是大官府令翠,在下怎敢起动,使我坐起不安。”伯爵道:“你老人家放心,他如今不做婊子了,见大人做了官,情愿认做干女儿了。”那桂姐便脸红了,说道:“汗邪了你,谁恁胡言!”谢希大道:“真个有这等事,俺每不晓的。趁今日众位老爹在此,一个也不少,每人五分银子人情,都送到哥这里来,与哥庆庆干女儿。”伯爵接过来道:“还是哥做了官好。自古不怕官,只怕管,这回子连干女儿也有了。到明日洒上些水扭出汁儿来。”被西门庆骂道:“你这贼狗才,单管这闲事胡说。”伯爵道:“胡铁?倒打把好刀儿哩。”郑爱香正递沈姨夫酒,插口道:“应二花子,李桂姐便做了干女儿,你到明日与大爹做个干儿子罢,吊过来就是个儿干子。”伯爵骂道:“贼小淫妇儿,你又少使得,我不缠你念佛。”李桂姐道:“香姐,你替我骂这花子两句。”郑爱香儿道:“不要理这望江南、巴山虎儿、汗东山、斜纹布。”伯爵道:“你这小淫妇,道你调子曰儿骂我,我没的说,只是一味白鬼,把你妈那裤带子也扯断了。由他到明日不与你个功德,你也不怕不把将军为神道。”桂姐道:“咱休惹他,哥儿拿出急来了。”郑爱香笑道:“这应二花子,今日鬼酉上车儿——推丑,东瓜花儿——丑的没时了。他原来是个王姑来子。”伯爵道:“这小\textShouWai 剌骨儿,诸人不要,只我将就罢了。”桂姐骂道:“怪攮刀子,好干净嘴儿,摆人的牙花已磕了。爹,你还不打与他两下子哩,你看他恁发讪。”西门庆骂道:“怪狗才东西!教他递酒,你斗他怎的!”走向席上打了他一下。伯爵道:“贼小淫妇儿!你说你倚着汉子势儿,我怕你?你看他叫的‘爹’那甜!”又道:“且休教他递酒,倒便益了他。拿过刑法来,且教他唱一套与俺每听着。他后边躲了这会滑儿也勾了。”韩玉钏儿道:“二爹,曹州兵备,管的事儿宽。”这里前厅花攒锦簇,饮酒顽耍不题。

单表潘金莲自从李瓶儿生了孩子,见西门庆常在他房里宿歇,于是常怀嫉妒之心,每蓄不平之意。知西门庆前厅摆酒,在镜台前巧画双蛾,重扶蝉鬓,轻点朱唇,整衣出房。听见李瓶儿房中孩儿啼哭,便走入来问道:“他怎这般哭?”奶子如意儿道:“娘往后边去了。哥哥寻娘,这等哭。”那潘金莲笑嘻嘻的向前戏弄那孩儿,说道:“你这多少时初生的小人芽儿,就知道你妈妈。等我抱到后边寻你妈妈去!”奶子如意儿说道:“五娘休抱哥哥,只怕一时撒了尿在五娘身上。”金莲道:“怪臭肉,怕怎的!拿衬儿托着他,不妨事。”一面接过官哥来抱在怀里,一直往后去了。走到仪门首,一迳把那孩儿举的高高的。不想吴月娘正在上房穿廊下,看着家人媳妇定添换菜碟儿,那潘金莲笑嘻嘻看孩子说道:“‘大妈妈,你做什么哩?’你说:‘小大官儿来寻俺妈妈来了。’”月娘忽抬头看见,说道:“五姐,你说的什么话?早是他妈妈没在跟前,这咱晚平白抱出他来做甚么?举的恁高,只怕唬着他。他妈妈在屋里忙着手哩。”便叫道:“李大姐你出来,你家儿子寻你来了。”那李瓶儿慌走出来,看见金莲抱着,说道:“小大官儿好好儿在屋里,奶子抱着,平白寻我怎的?看溺了你五妈身上尿。”金莲道:“他在屋里,好不哭着寻你,我抱出他来走走。”这李瓶儿忙解开怀接过来。月娘引逗了一回,分咐:“好好抱进房里去罢,休要唬着他!”李瓶儿到前边,便悄悄说奶子:“他哭,你慢慢哄着他,等我来,如何教五娘抱到后边寻我?”如意儿道:“我说来,五娘再三要抱了去。”那李瓶儿慢慢看着他喂了奶,就安顿他睡了。谁知睡下不多时,那孩子就有些睡梦中惊哭,半夜发寒潮热起来。奶子喂他奶也不吃,只是哭。李瓶儿慌了。

且说西门庆前边席散,打发四个唱的出门。月娘与了李桂姐一套重绡绒金衣服,二两银子,不必细说。西门庆晚夕到李瓶儿房里看孩儿,因见孩儿只顾哭,便问:“怎么的?”李瓶儿亦不题起金莲抱他后边去一节,只说道:“不知怎的,睡了起来这等哭,奶也不吃。”西门庆道:“你好好拍他睡。”因骂如意儿:“不好生看哥儿,管何事?唬了他!”走过后边对月娘说。月娘就知金莲抱出来唬了他,就一字没对西门庆说,只说:“我明日叫刘婆子看他看。”西门庆道:“休教那老淫妇来胡针乱灸的,另请小儿科太医来看孩儿。”月娘不依他,说道:“一个刚满月的孩子,什么小儿科太医。”到次日,打发西门庆早往衙门中去了,使小厮请了刘婆来看了,说是着了惊。与了他三钱银子。灌了他些药儿,那孩儿方才得睡稳,不洋奶了。李瓶儿一块石头方落地。正是:

\[
满怀心腹事,尽在不言中。
\]

\newpage
%# -*- coding:utf-8 -*-
%%%%%%%%%%%%%%%%%%%%%%%%%%%%%%%%%%%%%%%%%%%%%%%%%%%%%%%%%%%%%%%%%%%%%%%%%%%%%%%%%%%%%


\chapter{陈敬济失钥罚唱\KG 韩道国纵妇争锋}


词曰:

\[
衣染莺黄,爱停板驻拍,劝酒持觞。低鬟蝉影动,私语口脂香。檐滴露、竹风凉,拚剧饮琳琅。夜渐深笼灯就月,仔细端相。
\]

话说西门庆衙门中来家,进门就问月娘:“哥儿好些?使小厮请太医去。”月娘道:“我已叫刘婆子来了。吃了他药,孩子如今不洋奶,稳稳睡了这半日,觉好些了。”西门庆道:“信那老淫妇胡针乱灸,还请小儿科太医看才好。既好些了,罢。若不好,拿到衙门里去拶与老淫妇一拶子。”月娘道:“你恁的枉口拔舌骂人。你家孩儿现吃了他药好了,还恁舒着嘴子骂人!”说毕,丫鬟摆上饭来。西门庆刚才吃了饭,只见玳安儿来报:“应二爹来了。”西门庆教小厮:“拿茶出去,请应二爹卷棚内坐。”向月娘道:“把刚才我吃饭的菜蔬休动,教小厮拿饭出去,教姐夫陪他吃,说我就来。”月娘便问:“你昨日早晨使他往那里去?那咱才来。”西门庆便告说:“应二哥认的一个湖州客人何官儿,门外店里堆着五百两丝线,急等着要起身家去,来对我说要折些发脱。我只许他四百五十两银子。昨日使他同来保拿了两锭大银子作样银,已是成了来了,约下今日兑银子去。我想来,狮子街房子空闲,打开门面两间,倒好收拾开个绒线铺子,搭个伙计。况来保已是郓王府认纳官钱,教他与伙计在那里,又看了房儿,又做了买卖。”月娘道:“少不得又寻伙计。”西门庆道:“应二哥说他有一相识,姓韩,原是绒线行,如今没本钱,闲在家里,说写算皆精,行止端正,再三保举。改日领他来见我,写立合同。”说毕,西门庆在房中兑了四百五十两银子,教来保拿出来。陈敬济已陪应伯爵在卷棚内吃完饭,等的心里火发。见银子出来,心中欢喜,与西门庆唱了喏,说道:“昨日打搅哥,到家晚了,今日再扒不起来。”西门庆道:“这银子我兑了四百五十两,教来保取搭连眼同装了。今日好日子,便雇车辆搬了货来,锁在那边房子里就是了。”伯爵道:“哥主张的有理。只怕蛮子停留长智,推进货来就完了帐。”于是同来保骑头口,打着银子,迳到门外店中成交易去。谁知伯爵背地里与何官儿砸杀了,只四百二十两银子,打了三十两背工。对着来保,当面只拿出九两用银来,二人均分了。雇了车脚,即日推货进城,堆在狮子街空房内,锁了门,来回西门庆话。西门庆教应伯爵,择吉日领韩伙计来见。其人五短身材,三十年纪,言谈滚滚,满面春风。西门庆即日与他写立合同。同来保领本钱雇人染丝,在狮子街开张铺面,发卖各色绒丝。一日也卖数十两银子,不在话下。

光阴迅速,日月如梭,不觉八月十五日,月娘生辰来到,请堂客摆酒。留下吴大妗子、潘姥姥、杨姑娘并两个姑子住两日,晚夕宣唱佛曲儿,常坐到二三更才歇。那日,西门庆因上房有吴大妗子在这里,不方便,走到前边李瓶儿房中看官哥儿,心里要在李瓶儿房里睡。李瓶儿道:“孩子才好些儿,我心里不耐烦,往他五妈妈房里睡一夜罢。”西门庆笑道:“我不惹你。”于是走过金莲这边来。那金莲听见汉子进他房来,如同拾了金宝一般,连忙打发他潘姥姥过李瓶儿这边宿歇。他便房中高点银灯,款伸锦被,薰香澡牝,夜间陪西门庆同寝。枕畔之情,百般难述,无非只要牢宠汉子心,使他不往别人房里去。正是:鼓鬣游蜂,嫩蕊半匀春荡漾;餐香粉蝶,花房深宿夜风流。

李瓶儿见潘姥姥过来,连忙让在炕上坐的。教迎春安排酒菜果饼,晚夕说话,坐半夜才睡。到次日,与了潘姥姥一件葱白绫袄儿,两双缎子鞋面,二百文钱。把婆子欢喜的眉欢眼笑,过这边来,拿与金莲瞧,说:“这是那边姐姐与我的。”金莲见了,反说他娘:“好恁小眼薄皮的,什么好的,拿了他的来!”潘姥姥道:“好姐姐,人倒可怜见与我,你却说这个话。你肯与我一件儿穿?”金莲道:“我比不得他有钱的姐姐。我穿的还没有哩,拿什么与你!你平白吃了人家的来,等住回可整理几碟子来,筛上壶酒,拿过去还了他就是了。到明日少不的教人\textShiDian 言试语,我是听不上。”一面分咐春梅,定八碟菜蔬,四盒果子,一锡瓶酒。打听西门庆不在家,教秋菊用方盒拿到李瓶儿房里,说:“娘和姥姥过来,无事和六娘吃杯酒。”李瓶儿道:“又教你娘费心。”少顷,金莲和潘姥姥来,三人坐定,把酒来斟。春梅侍立斟酒。

娘儿每说话间,只见秋菊来叫春梅,说:“姐夫在那边寻衣裳,教你去开外边楼门哩。”金莲分咐:“叫你姐夫寻了衣裳来这里喝瓯子酒去。”不一时,敬济寻了几家衣服,就往外走。春梅进来回说:“他不来。”金莲道:“好歹拉了他来。”又使出绣春去把敬济请来。潘姥姥在炕上坐,小桌儿摆着果盒儿,金莲、李瓶儿陪着吃酒。连忙唱了喏。金莲说:“我好意教你来吃酒儿,你怎的张致不来?就吊了造化了?呶了个嘴儿,教春梅:“拿宽杯儿来,筛与你姐夫吃。”敬济把寻的衣服放在炕上,坐下。春梅做定科范,取了个茶瓯子,流沿边斟上,递与他。慌的敬济说道:“五娘赐我,宁可吃两小锺儿罢。外边铺子里许多人等着要衣裳。”金莲道:“教他等着去,我偏教你吃这一大锺,那小锺子刁刁的不耐烦。”潘姥姥道:“只教哥哥吃这一锺罢,只怕他买卖事忙。”金莲道:“你信他!有什么忙!吃好少酒儿,金漆桶子吃到第二道箍上。”那敬济笑着拿酒来,刚呷了两口。潘姥姥叫春梅:“姐姐,你拿箸儿与哥哥。教他吃寡酒?”春梅也不拿箸,故意殴他,向攒盒内取了两个核桃递与他。那敬济接过来道:“你敢笑话我就禁不开他?”于是放在牙上只一磕,咬碎了下酒。潘姥姥道:“还是小后生家,好口牙。相老身,东西儿硬些就吃不得。”敬济道:“儿子世上有两椿儿——鹅卵石、牛犄角——吃不得罢了。”金莲见他吃了那锺酒,教春梅再斟上一锺儿,说:“头一锺是我的了。你姥姥和六娘不是人么?也不教你吃多,只吃三瓯子,饶了你罢。”敬济道:“五娘可怜见儿子来,真吃不得了。此这一锺,恐怕脸红,惹爹见怪。”金莲道:“你也怕你爹?我说你不怕他。你爹今日往那里吃酒去了?”敬济道:“后晌往吴驿丞家吃酒,如今在对门乔大户房子里看收拾哩。”金莲问:“乔大户家昨日搬了去,咱今日怎不与他送茶?”敬济道:“今早送茶去了。”李瓶儿问:“他家搬到那里住去了?”敬济道:“他在东大街上使了一千二百银子,买了所好不大的房子,与咱家房子差不多儿,门面七间,到底五层。”说话之间,敬济捏着鼻子又挨了一锺,趁金莲眼错,得手拿着衣服往外一溜烟跑了。迎春道:“娘你看,姐夫忘记钥匙去了。”那金莲取过来坐在身底下,向李瓶儿道:“等他来寻,你每且不要说,等我奈何他一回儿才与他。”潘姥姥道:“姐姐与他罢了,又奈何他怎的。”

那敬济走到铺子里,袖内摸摸,不见钥匙,一直走到李瓶儿房里寻。金莲道:“谁见你什么钥匙,你管着什么来?放在那里,就不知道?”春梅道:“只怕你锁在楼上了。”敬济道:“我记的带出来。”金莲道:“小孩儿家屁股大,敢吊了心!又不知家里外头什么人扯落的你恁有魂没识,心不在肝上。”敬济道:“有人来赎衣裳,可怎的样?趁爹不过来,免不得叫个小炉匠来开楼门,才知有没。”那李瓶儿忍不住,只顾笑。敬济道:“六娘拾了,与了我罢。”金莲道:“也没见这李大姐,不知和他笑什么,恰似我每拿了他的一般。”急得敬济只是牛回磨转,转眼看见金莲身底下露出钥匙带儿来,说道:“这不是钥匙!”才待用手去取,被金莲褪在袖内,不与他,说道:“你的钥匙儿,怎落在我手里?”急得那小伙儿只是杀鸡扯膝。金莲道:“只说你会唱的好曲儿,倒在外边铺子里唱与小厮听,怎的不唱个儿我听?今日趁着你姥姥和六娘在这里,只拣眼生好的唱个儿,我就与你这钥匙。不然,随你就跳上白塔,我也没有。”敬济道:“这五娘,就勒掯出人痞来。谁对你老人家说我会唱?”金莲道:“你还捣鬼?南京沈万三,北京枯树弯——人的名儿,树的影儿。”那小伙儿吃他奈何不过,说道:“死不了人,等我唱。我肚子里撑心柱肝,要一百个也有!”金莲骂道:“说嘴的短命!”自把各人面前酒斟上。金莲道:“你再吃一杯,盖着脸儿好唱。”敬济道:“我唱了慢慢吃。我唱个果子名《山坡羊》你听:

\[
初相交,在桃园儿里结义。相交下来,把你当玉黄李子儿抬举。人人说你在青翠花家饮酒,气的我把频波脸儿挝的粉粉的碎。我把你贼,你学了虎刺宾了,外实里虚,气的我李子眼儿珠泪垂。我使的一对桃奴儿寻你,见你在软枣儿树下就和我别离了去。气的我鹤顶红剪一柳青丝儿来呵,你海东红反说我理亏。骂了句生心红的强贼,逼的我急了,我在吊枝干儿上寻个无常,到三秋,我看你倚靠着谁?
\]

唱毕,就问金莲要钥匙,说道:“五娘快与了我罢!伙计铺子里不知怎的等着我哩。只怕一时爹过来。”金莲道:“你倒自在性儿,说的且是轻巧。等你爹问,我就说你不知在那里吃了酒,把钥匙不见了,走来俺屋里寻。”敬济道:“爷嚛!五娘就是弄人的刽子手。”李瓶儿和潘姥姥再三旁边说道:“姐姐与他去罢。”金莲道:“若不是姥姥和你六娘劝我,定罚教你唱到天晚。头里骗嘴说一百个,才唱一个曲儿就要腾翅子?我手里放你不过。”敬济道:“我还有一个儿看家的,是银名《山坡羊》,亦发孝顺你老人家罢。”于是顿开喉音唱道:

\[
冤家你不来,白闷我一月,闪的人反拍着外膛儿细丝谅不彻。我使狮子头定儿小厮拿着黄票儿请你,你在兵部洼儿里元宝儿家欢娱过夜。我陪铜磬儿家私为焦心一旦儿弃舍,我把如同印箝儿印在心里愁无求解。叫着你把那挺脸儿高扬着不理,空教我拨着双火筒儿顿着罐子等到你更深半夜。气的奴花银竹叶脸儿咬定银牙来呵,唤官银顶上了我房门,随那泼脸儿冤家轻敲儿不理。骂了句煎彻了的三倾儿捣槽斜贼,空把奴一腔子暖汁儿真心倒与你,只当做热血。
\]

敬济唱毕,金莲才待叫春梅斟酒与他,忽有月娘从后边来,见奶子如意儿抱着官哥儿在房门首石基上坐,便说道:“孩子才好些,你这狗肉又抱他在风里,还不抱进去!”金莲问:“是谁说话?”绣春回道:“大娘来了。”敬济慌的拿钥匙往外走不迭。众人都下来迎接月娘。月娘便问:“陈姐夫在这里做什么来?”金莲道:“李大姐整治些菜,请俺娘坐坐。陈姐夫寻衣服,叫他进来吃一杯。姐姐,你请坐,好甜酒儿,你吃一杯。”月娘道:“我不吃。后边他大妗子和杨姑娘要家去,我又记挂着这孩子,迳来看看。李大姐,你也不管,又教奶子抱他在风里坐的。前日刘婆子说他是惊寒,人还不好生看他!”李瓶儿道:“俺陪着姥姥吃酒,谁知贼臭肉三不知抱他出去了。”月娘坐了半歇,回后边去了。一回,使小玉来,请姥姥和五娘、六娘后边坐。那潘金莲和李瓶儿匀了脸,同潘姥姥往后边来,陪大妗子、杨姑娘吃酒。到日落时分,与月娘送出大门,上轿去了。都在门里站立,先是孟玉楼说道:“大姐姐,今日他爹不在,往吴驿丞家吃酒去了,咱到好往对门乔大户家房里瞧瞧。”月娘问看门的平安儿:“谁拿着那边钥匙哩?”平安道:“娘每要过去瞧,开着门哩。来兴哥看着两个坌工的在那里做活。”月娘分咐:“你教他躲开,等俺每瞧瞧去。”平安儿道:“娘每只顾瞧,不妨事。他每都在第四层大空房拨灰筛土,叫出来就是了。”

当下月娘、李娇儿、孟玉楼、潘金莲、李瓶儿,都用轿子短搬抬过房子内。进了仪门,就是三间厅。第二层是楼。月娘要上楼去,可是作怪,刚上到楼梯中间,不料梯磴陡趄,只闻月娘哎了一声,滑下一只脚来,早是月娘攀住楼梯两边栏杆。慌了玉楼,便道:“姐姐怎的?”连忙搊住他一只胳膊,不曾跌下来。月娘吃了一惊,就不上去。众人扶了下来,唬的脸蜡查儿黄了。玉楼便问:“姐姐,怎么上来滑了脚,不曾扭着那里?”月娘道:“跌倒不曾跌着,只是扭了腰子,唬的我心跳在口里。楼梯子趄,我只当咱家里楼上来,滑了脚。早是攀住栏杆,不然怎了!”李娇儿道:“你又身上不方便,早知不上楼也罢了。”于是众姊妹相伴月娘回家。刚到家,叫的应就肚中疼痛。月娘忍不过,趁西门庆不在家,使小厮叫了刘婆子来看。婆子道:“你已是去经事来着伤,多是成不的了。”月娘道:“便了五个多月了,上楼着了扭。”婆子道:“你吃了我这药,安不住,下来罢了。”月娘道:“下来罢!”婆子于是留了两服大黑丸子药,教月娘用艾酒吃。那消半夜,吊下来了,在马桶里。点灯拨看,原来是个男胎,已成形了。正是:

\[
胚胎未能成性命,真灵先到杳冥天。
\]
幸得那日西门庆在玉楼房中歇了。

到次日,玉楼早晨到上房,问月娘:“身子如何?”月娘告诉:“半夜果然疼不住,落下来了,倒是小厮儿。”玉楼道:“可惜了!他爹不知道?”月娘道:“他爹吃酒来家,到我屋里才待脱衣裳,我说你往他们屋里去罢,我心里不自在。他才往你这边来了。我没对他说。我如今肚里还有些隐隐的疼。”玉楼道:“只怕还有些余血未尽,筛酒吃些锅脐灰儿就好了。”又道:“姐姐,你还计较两日儿,且在屋里不可出去。小产比大产还难调理,只怕掉了风寒,难为你的身子。”月娘道:“你没的说,倒没的唱扬的一地里知道,平白噪剌剌的抱什么空窝,惹的人动那唇齿。”以此就没教西门庆知道。此事表过不题。

且说西门庆新搭的开绒线铺伙计,也不是守本分的人,姓韩名道国,字希尧,乃是破落户韩光头的儿子。如今跌落下来,替了大爷的差使,亦在郓王府做校尉,见在县东街牛皮小巷居住。其人性本虚飘,言过其实,巧于词色,善于言谈。许人钱,如捉影捕风;骗人财,如探囊取物。自从西门庆家做了买卖,手里财帛从容,新做了几件虼蚤皮,在街上掇着肩膊儿就摇摆起来。人见了不叫他个韩希尧,只叫他做“韩一摇”。他浑家乃是宰牲口王屠妹子,排行六儿,生的长跳身材,瓜子面皮,紫膛色,约二十八九年纪。身边有个女孩儿,嫡亲三口儿度日。他兄弟韩二,名二捣鬼,是个耍钱的捣子,在外边另住。旧与这妇人有奸,赶韩道国不在家,铺中上宿,他便时常走来与妇人吃酒,到晚夕刮涎就不去了。不想街坊有几个浮浪子弟,见妇人搽脂抹粉,打扮的乔模乔样,常在门首站立睃人,人略斗他斗儿,又臭又硬,就张致骂人。因此街坊这些小伙子儿,心中有几分不愤,暗暗三两成群,背地讲论,看他背地与什么人有首尾。那消半个月,打听出与他小叔韩二这件事来。原来韩道国这间屋门面三间,房里两边都是邻舍,后门逆水塘。这伙人,单看韩二进去,或夜晚扒在墙上看觑,或白日里暗使小猴子在后塘推道捉蛾儿,单等捉奸。不想那日二捣鬼打听他哥不在,大白日装酒和妇人吃,醉了,倒插了门,在房里干事。不防众人睃见踪迹,小猴子扒过来,把后门开了,众人一齐进去,掇开房门。韩二夺门就走,被一少年一拳打倒拿住。老婆还在炕上,慌穿衣不迭。一人进去,先把裤子挝在手里,都一条绳子拴出来。须臾,围了一门首人,跟到牛皮街厢铺里,就哄动了那一条街巷。这一个来问,那一个来瞧,内中一老者见男妇二人拴做一处,便问左右看的人:“此是为什么事的?”旁边有多口的道:“你老人家不知,此是小叔奸嫂子的。”那老都点了点头儿说道:“可伤,原来小叔儿要嫂子的,到官,叔嫂通奸,两个都是绞罪。”那旁边多口的,认的他有名叫做陶扒灰,一连娶三个媳妇,都吃他扒了,因此插口说道:“你老人家深通条律,相这小叔养嫂子的便是绞罪,若是公公养媳妇的却论什么罪?”那老者见不是话,低着头一声儿没言语走了。正是:各人自扫檐前雪,莫管他人屋上霜。这里二捣鬼与妇人被捉不题。

单表那日,韩道国铺子里不该上宿,来家早,八月中旬天气,身上穿着一套儿轻纱软绢衣服,新盔的一顶帽儿,在街上阔行大步摇摆。但遇着人,或坐或立,口惹悬河,滔滔不绝。就是一回,内中遇着他两个相熟的人,一个是开纸铺的张二哥,一个是开银铺的白四哥,慌作揖举手。张好问便道:“韩老兄连日少见,闻得恭喜在西门大官府上,开宝铺做买卖,我等缺礼失贺,休怪休怪!”一面让他坐下。那韩道国坐在凳上,把脸儿扬着,手中摇着扇儿,说道:“学生不才,仗赖列位余光,与我恩主西门大官人做伙计,三七分钱。掌巨万之财,督数处之铺,甚蒙敬重,比他人不同。”白汝晃道:“闻老兄在他门下只做线铺生意。”韩道国笑道:“二兄不知,线铺生意只是名目而已。他府上大小买卖,出入资本,那些儿不是学生算帐!言听计从,祸福共知,通没我一时儿也成不得。大官人每日衙门中来家摆饭,常请去陪侍,没我便吃不下饭去。俺两个在他小书房里,闲中吃果子说话儿,常坐半夜他方进后边去。昨日他家大夫人生日,房下坐轿子行人情,他夫人留饮至二更方回。彼此通家,再无忌惮。不可对兄说,就是背地他房中话儿,也常和学生计较。学生先一个行止端庄,立心不苟,与财主兴利除害,拯溺救焚。凡百财上分明,取之有道。就是傅自新也怕我几分。不是我自己夸奖,大官人正喜我这一件儿。”刚说在热闹处,忽见一人慌慌张张走向前叫道:“韩大哥,你还在这里说什么,教我铺子里寻你不着。”拉到僻静处告他说:“你家中如此这般,大嫂和二哥被街坊众人撮弄了,拴到铺里,明早要解县见官去。你还不早寻人情理会此事?”这韩道国听了,大惊失色。口中只咂嘴,下边顿足,就要翅趫走。被张好问叫道:“韩老兄,你话还未尽,如何就去了?”这韩道国举手道:“大官人有要紧事,寻我商议,不及奉陪。”慌忙而去。正是:

\[
谁人挽得西江水,难洗今朝一面羞。
\]

\newpage
%# -*- coding:utf-8 -*-
%%%%%%%%%%%%%%%%%%%%%%%%%%%%%%%%%%%%%%%%%%%%%%%%%%%%%%%%%%%%%%%%%%%%%%%%%%%%%%%%%%%%%


\chapter{献芳樽内室乞恩\KG 受私贿后庭说事}


词曰:

\[
成吴越,怎禁他巧言相斗谍。平白地送暖偷寒,平白地送暖偷寒,猛可的搬唇弄舌。水晶丸不住撇,蘸刚锹一味撅。
\]

话说韩道国走到家门首打听,见浑家和兄弟韩二拴在铺中去了,急急走到铺子内,和来保计议。来保说:“你还早央应二叔来,对当家的说了,拿个帖儿对县中李老爹一说,不论多大事情都了了。”这韩道国竟到应怕爵家。他娘子儿使丫头出来回:“没人在家,不知往那里去了。只怕在西门大老爹家。”韩道国道:“没在他宅里。”问应宝,也跟出去了。韩道国慌了,往勾栏院里抓寻。原来伯爵被湖州何蛮子的兄弟何二蛮子——号叫何两峰,请在四条巷内何金蝉儿家吃酒。被韩道国抓着了,请出来。伯爵吃的脸红红的,帽檐上插着剔牙杖儿。韩道国唱了喏,拉到僻静处,如此这般告他说。伯爵道:“既有此事,我少不得陪你去。”于是辞了何两峰,与道国先同到家,问了端的。道国央及道:“此事明日只怕要解到县里去,只望二叔往大官府宅里说说,讨个帖儿,转与李老爹,求他只不教你侄妇见官。事毕重谢二叔。”说着跪在地下。伯爵用手拉起来,说道:“贤契,这些事儿,我不替你处?你快写个说帖,把一切闲话都丢开,只说你常不在家,被街坊这伙光棍时常打砖掠瓦,欺负娘子。你兄弟韩二气忿不过,和他嚷乱,反被这伙人群住,揪采踢打,同拴在铺里。望大官府发个帖儿,对李老爹说,只不教你令正出官,管情见个分上就是了。”那韩道国取笔砚,连忙写了说帖,安放袖中。

伯爵领他迳到西门庆门首,问守门的平安儿:“爹在家?”平安道:“爹在花园书房里。二爹和韩大叔请进去。”那应伯爵狗也不咬,走熟了的,同韩道国进入仪门,转过大厅,由鹿顶钻山进去,就是花园角门。抹过木香棚,三间小卷棚,名唤翡翠轩,乃西门庆夏月纳凉之所。前后帘拢掩映,四面花竹阴森,里面一明两暗书房。有画童儿小厮在那里扫地,说:“应二爹和韩大叔来了!”二人掀开帘子。进入明间内,书童看见便道:“请坐。俺爹刚才进后边去了。”一面使画童儿请去。画童儿走到后边金莲房内,问:“春梅姐,爹在这里?”春梅骂道:“贼见鬼小奴才儿!爹在间壁六娘房里不是,巴巴的跑来这里问!”画童便走过这边,只见绣春在石台基上坐的,悄悄问:“爹在房里?应二爹和韩大叔来了,在书房里等爹说话。”绣春道:“爹在房里,看着娘与哥裁衣服哩。”原来西门庆拿出口匹尺头来,一匹大红纻丝,一匹鹦哥绿潞绸,教李瓶儿替官哥裁毛衫、披袄、背心、护顶之类。在炕上正铺着大红毡条。奶子抱着哥儿,迎春执着熨斗。只见绣春进来,悄悄拉迎春一把,迎春道:“你拉我怎么的?拉撇了这火落在毡条上。”李瓶儿便问:“你平白拉他怎的?”绣春道:“画童说应二爹来了,请爹说话。”李瓶儿道:“小奴才儿,应二爹来,你进来说就是了,巴巴的扯他!”

西门庆分咐画童:“请二爹坐坐,我就来。”于是看裁完了衣服,便衣出来,书房内见伯爵二人,作揖坐下,韩道国打横。吃了茶,伯爵就开言说道:“韩大哥,你有甚话,对你大官府说。”西门庆道:“你有甚话说来。”韩道国才待说“街坊有伙不知姓名棍徒……”,被应伯爵拦住便道:“贤侄,你不是这等说了。噙着骨秃露着肉,也不是事。对着你家大官府在这里,越发打开后门说了罢:韩大哥常在铺子里上宿,家下没人,止是他娘子儿一人,还有个孩儿。左右街坊,有几个不三不四的人,见无人在家,时常打砖掠瓦鬼混。欺负的急了,他令弟韩二哥看不过,来家骂了几句,被这起光棍不由分说,群住了打个臭死。如今部拴在铺里,明早要解了往本县李大人那里去。他哭哭啼啼,央烦我来对哥说,讨个帖儿,对李大人说说,青目一二。有了他令弟也是一般,只不要他令正出官就是了。”因说:“你把那说帖儿拿出来与你大官人瞧,好差人替你去。”韩道国便向袖中取出,连忙双膝跪下,说道:“小人忝在老爹门下,万乞老爹看应二叔分上,俯就一二,举家没齿难忘。”西门庆一把手拉起,说道:“你请起来。”于是观看帖儿,上面写着:“犯妇王氏,乞青目免提。”西门庆道:“这帖子不是这等写了!只有你令弟韩二一人就是了。”向伯爵道:“比时我拿帖对县里说,不如只分咐地方改了报单,明日带来我衙门里来发落就是了。”伯爵教:“韩大哥,你还与恩老爹下个礼儿。这等亦发好了!”那韩道国又倒身磕头下去。西门庆教玳安:“你外边快叫个答应的班头来。”不一时,叫了个穿青衣的节级来,在旁边伺候。西门庆叫近前,分咐:“你去牛皮街韩伙计住处,问是那牌那铺地方,对那保甲说,就称是我的钧语,分咐把王氏即时与我放了。查出那几个光棍名字来,改了报帖,明日早解提刑院,我衙门里听审。”那节级应诺,领了言语出门。伯爵道:“韩大哥,你即一同跟了他,干你的事去罢,我还和大官人说话哩。”那韩道国千恩万谢出门,与节级同往牛皮街干事去了。

西门庆陪伯爵在翡翠轩坐下,因令玳安放桌儿:“你去对你大娘说,昨日砖厂刘公公送的木樨荷花酒,打开筛了来,我和应二叔吃,就把糟鲥鱼蒸了来。”伯爵举手道:“我还没谢的哥,昨日蒙哥送了那两尾好鲫鱼与我。送了一尾与家兄去,剩下一尾,对房下说,拿刀儿劈开,送了一段与小女,余者打成窄窄的块儿,拿他原旧红糟儿培着,再搅些香油,安放在一个磁罐内,留着我一早一晚吃饭儿,或遇有个人客儿来,蒸恁一碟儿上去,也不枉辜负了哥的盛情。”西门庆告诉:“刘太监的兄弟刘百户,因在河下管芦苇场,赚了几两银子,新买了一所庄子在五里店,拿皇木盖房,近日被我衙门里办事官缉听着,首了。依着夏龙溪,饶受他一百两银子,还要动本参送,申行省院。刘太监慌了,亲自拿着一百两银子到我这里,再三央及,只要事了。不瞒你说,咱家做着些薄生意,料也过了日子,那里希罕他这样钱!况刘太监平日与我相交,时常受他些礼,今日因这些事情,就又薄了面皮?教我丝毫没受他的,只教他将房屋连夜拆了。到衙门里,只打了他家人刘三二十,就发落开了。事毕,刘太监感情不过,宰了一口猪,送我一坛自造荷花酒,两包糟鲥鱼,重四十斤,又两匹妆花织金缎子,亲自来谢。彼此有光,见个情分。”伯爵道:“哥,你是希罕这个钱的?夏大人他出身行伍,起根立地上没有,他不挝些儿,拿甚过日?哥,你自从到任以来,也和他问了几桩事儿?”西门庆道:“大小也问了几件公事。别的到也罢了,只吃了他贪滥蹋婪,有事不论青红皂白,得了钱在手里就放了,成甚么道理!我便再三扭着不肯,‘你我虽是个武职官儿,掌着这刑条,还放些体面才好。’”说未了,酒菜齐至。西门庆将小金菊花杯斟荷花酒,陪伯爵吃。

不说两个说话儿,坐更余方散。且说那伙人,见青衣节级下地方,把妇人王氏放回家去,又拘总甲,查了各人名字,明早解提刑院问理,都各人口面相觑。就知韩道国是西门庆家伙计,寻的本家攊子,只落下韩二一人在铺里。都说这事弄的不好了。这韩道国又送了节级五钱银子,登时间保甲查写那几个名字,送到西门庆宅内,单等次日早解。

过一日,西门庆与夏提刑两位官,到衙门里坐厅。该地方保甲带上人去,头一起就是韩二,跪在头里。夏提刑先看报单:“牛皮街一牌四铺总甲萧成,为地方喧闹事……”第一个就叫韩二,第二个车淡,第三个管世宽,第四个游守,第三个郝贤。都叫过花名去。然后问韩二:“为什么起来?”那韩二先告道:“小的哥是买卖人,常不在家住的,小男幼女,被街坊这几个光棍,要便弹打胡博词儿,坐在门首,胡歌野调,夜晚打砖,百般欺负。小的在外另住,来哥家看视,含忍不过,骂了几句。被这伙棍徒,不由分说,揪倒在地,乱行踢打,获在老爷案下。望老爷查情。”夏提刑便问:“你怎么说?”那伙人一齐告道:“老爷休信他巧对!他是耍钱的捣鬼。他哥不在家,和他嫂子王氏有奸。王氏平日倚逞刁泼毁驾街坊。昨日被小的们捉住,见有底衣为证。”夏提刑因问保甲萧成:“那王氏怎的不见?”萧成怎的好回节级放了?只说:“王氏脚小,路上走不动,便来。”那韩二在下边,两只眼只看着西门庆。良久,西门庆欠身望夏提刑道:“长官也不消要这王氏。想必王氏有些姿色,这光棍来调戏他不遂,捏成这个圈套。”因叫那为首的车淡上去,问道:“你在那里捉住那韩二来?”众人道:“昨日在他屋里捉来。”又问韩二:“王氏是你甚么人?”保甲道:“是他嫂子儿。”又问保甲:“这伙人打那里进他屋里?”保甲道:“越墙进去。”西门庆大怒,骂道:“我把你这起光棍!他既是小叔,王氏也是有服之亲,莫不不许上门行走?相你这起光棍,你是他什么人,如何敢越墙进去?况他家男子不在,又有幼女在房中,非奸即盗了。”喝令左右拿夹棍来,每人一夹、二十大棍,打的皮开肉绽,鲜血迸流。况四五个都是少年子弟,出娘胞胎未经刑杖,一个个打的号哭动天,呻吟满地。这西门庆也不等夏提刑开口,分咐:“韩二出去听候。把四个都与我收监,不日取供送问。”四人到监中都互相抱怨,个个都怀鬼胎。监中人都吓恐他:“你四个若送问,都是徒罪。到了外府州县,皆是死数。”这些人慌了,等的家下人来送饭,捎信出去,教各人父兄使钱,上下寻人情。内中有拿人情央及夏提刑,夏提刑说:“这王氏的丈夫是你西门老爹门下的伙计。他在中间扭着要送问,同僚上,我又不好处得。你须还寻人情和他说去。”也有央吴大舅出来说的。人都知西门庆家有钱,不敢来打点。

四家父兄都慌了,会在一处。内中一个说道:“也不消再央吴千户,他也不依。我闻得人说,东街上住的开绸绢铺应大哥兄弟应二,和他契厚。咱不如凑了几十两银子,封与应二,教他替咱们说说,管情极好。”于是车淡的父亲开酒店的车老儿为首,每人拿十两银子来,共凑了四十两银子,齐到应伯爵家,央他对西门庆说。伯爵收下,打发众人去了。他娘子儿便说:“你既替韩伙计出力,摆布这起人,如何又揽下这银子,反替他说方便,不惹韩伙计怪?”伯爵道:“我可知不好说的。我别自有处。”因把银子兑了十五两,包放袖中,早到西门庆家。西门庆还未回来。伯爵进厅上,只见书童正从西厢房书房内出来,头带瓦楞帽儿,撇着金头莲瓣簪子,身上穿着苏州绢直掇,玉色纱\textuni{2773D}儿,凉鞋净袜。说道:“二爹请客位内坐。”交画童儿后边拿茶去,说道:“小厮,我使你拿茶与应二爹,你不动,且耍子儿。等爹来家,看我说不说!”那小厮就拿茶去了。伯爵便问:“你爹衙门里还没来家?”书童道:“刚才答应的来,说爹衙门散了,和夏老爹门外拜客去了。二爹有甚话说?”伯爵道:“没甚话。”书童道:“二爹前日说的韩伙计那事,爹昨日到衙门里,把那伙人都打了收监,明日做文书还要送问他。”伯爵拉他到僻静处,和他说:“如今又一件,那伙人家属如此这般,听见要送问,都害怕了。昨日晚夕,到我家哭哭啼啼,再三跪着央及我,教对你爹说。我想我已是替韩伙计说在先,怎又好管他的,惹的韩伙计不怪?没奈何,教他四家处了这十五两银子,看你取巧对你爹说,看怎么将就饶他放了罢。”因向袖中取出银子来递与书童。书童打开看了,大小四锭零四块。说道:“既是应二爹分上,交他再拿五两来,待小的替他说,还不知爹肯不肯。昨日吴大舅亲自来和爹说了,爹不依。小的虼蚤脸儿——好大面皮!实对二爹说,小的这银子,不独自一个使,还破些钞儿,转达知俺生哥的六娘,绕个弯儿替他说,才了他此事。”伯爵道:“既如此,等我和他说。你好歹替他上心些,他后晌些来讨回话。”书童道:“爹不知多早来家,你教他明日早来罢。”说毕,伯爵去了。

这书童把银子拿到铺子,镏下一两五钱来,教人买了一坛金华酒,两只烧鸭,两只鸡,一钱银子鲜鱼,一肘蹄子,二钱顶皮酥果馅饼儿,一钱银子的搽穰卷儿,送到来兴儿屋里,央及他媳妇惠秀替他整理,安排端正。那一日,潘金莲不在家,从早间就坐轿子往门外潘姥姥家做生日去了。书童使画童儿用方盒把下饭先拿在李瓶儿房中,然后又提了一坛金华酒进去。李瓶儿便问:“是那里的?”画童道:“是书童哥送来孝顺娘的。”李瓶儿笑道:“贼囚!他怎的孝顺我?”良久,书童儿进来,见瓶儿在描金炕床上,引着玳瑁猫儿和哥儿耍子。因说道:“贼囚!你送了这些东西来与谁吃,”那书童只是笑。李瓶儿道:“你不言语,笑是怎的说?”书童道:“小的不孝顺娘,再孝顺谁!”李瓶儿道:“贼囚!你平白好好的,怎么孝顺我?你不说明白,我也不吃。”那书童把酒打开,菜蔬都摆在小桌上,教迎春取了把银素筛了来,倾酒在锺内,双手递上去,跪下说道:“娘吃过,等小的对娘说。”李瓶儿道:“你有甚事,说了我才吃。不说,你就跪一百年,我也是不吃。”又道:“你起来说。”那书童于是把应伯爵所央四人之事,从头诉说一遍:“他先替韩伙计说了,不好来说得,央及小的先来禀过娘。等爹问,休说是小的说,只假做花大舅那头使人来说。小的写下个帖儿在前边书房内,只说是娘递与小的,教与爹看。娘再加一美言。况昨日衙门里爹已是打过他,爹胡乱做个处断,放了他罢,也是老大的阴骘。”李瓶儿笑道:“原来也是这个事!不打紧,等你爹来家,我和他说就是了。你平白整治这些东西来做什么?”又道:“贼囚!你想必问他起发些东西了,”书童道:“不瞒娘说,他送了小的五两银子。”李瓶儿道:“贼囚!你倒且是会排铺赚钱!”于是不吃小锺,旋教迎春取了个大银衢花杯来,先吃了两锺,然后也回斟一杯与书童吃。书童道:“小的不敢吃,吃了快脸红,只怕爹来看见。”李瓶儿道:“我赏你吃,怕怎的!”于是磕了头起来,一吸而饮之。李瓶儿把各样嘎饭拣在一个碟儿里,教他吃。那小厮一连陪他吃了两大杯,怕脸红就不敢吃,就出来了。到了前边铺子里,还剩了一半点心嘎饭,摆在柜上,又打了两提坛酒,请了傅伙计、贲四、陈敬济、来兴儿、玳安儿。众人都一阵风卷残云,吃了个净光。就忘了教平安儿吃。

那平安儿坐在大门首,把嘴谷都着。不想西门庆约后晌从门外拜了客来家,平安看见也不说。那书童听见喝道之声,慌的收拾不迭,两三步叉到厅上,与西门庆接衣服。西门庆便问:“今日没人来?”书童道:“没人。”西门庆脱了衣服,摘去冠帽,带上巾帻,走到书房内坐下。书童儿取了一盏茶来递上,西门庆呷了一口放下。因见他面带红色,便问:“你那里吃酒来?”这书童就向桌上砚台下取出一纸柬帖与西门庆瞧,说道:“此是后边六娘叫小的到房里,与小的的,说是花大舅那里送来,说车淡等事。六娘教小的收着与爹瞧。因赏了小的一盏酒吃,不想脸就红了。”西门庆把帖观看,上写道:“犯人车淡四名,乞青目。”看了,递与书童,分咐:“放在我书箧内,教答应的明日衙门里禀我。”书童一面接了放在书箧内,又走在旁边侍立。西门庆见他吃了酒,脸上透出红白来,红馥馥唇儿,露着一口糯米牙儿,如何不爱。于是淫心辄起,搂在怀里,两个亲嘴咂舌头。那小郎口噙香茶桂花饼,身上薰的喷鼻香。西门庆用手撩起他衣服,褪了花裤儿,摸弄他屁股。因嘱咐他:“少要吃酒,只怕糟了脸。”书童道:“爹分咐,小的知道。”两个在屋里正做一处。忽一个青衣人,骑了一匹马,走到大门首,跳下马来,向守门的平安作揖,问道:“这里是问刑的西门庆老爹家?”那平安儿因书童不请他吃东道,把嘴头子撅着,正没好气,半日不答应。那人只顾立着,说道:“我是帅府周老爷差来,送转帖与西门老爹看。明日与新平寨坐营须老爹送行,在永福寺摆酒。也有荆都监老爹,掌刑夏老爹,营里张老爹,每位分资一两。迳来报知,累门上哥禀禀进去,小人还等回话。”那平安方拿了他的转帖入后边,打听西门庆在花园书房内,走到里面,转过松墙,只见画童儿在窗外台基上坐的,见了平安摆手儿。那平安就知西门庆与书童干那不急的事,悄悄走在窗下听觑。半日,听见里边气呼呼,跐的地平一片声响。西门庆叫道:“我的儿,把身子调正着,休要动。”就半日没听见动静。只见书童出来,与西门庆舀水洗手,看见平安儿、画童儿在窗子下站立,把脸飞红了,往后边拿去了。平安拿转帖进去,西门庆看了,取笔画了知,分咐:“后边问你二娘讨一两银子,教你姐夫封了,付与他去。”平安儿应诺去了。

书童拿了水来,西门庆洗毕手,回到李瓶儿房中。李瓶儿便问:“你吃酒?教丫头筛酒你吃。”西门庆看见桌子底下放着一坛金华酒,便问:“是那里的?”李瓶儿不好说是书童儿买进来的,只说:“我一时要想些酒儿吃,旋使小厮街上买了这坛酒来。打开只吃了两锺儿,就懒待吃了。”西门庆道:“阿呀,前头放着酒,你又拿银子买!前日我赊了丁蛮子四十坛河清酒,丢在西厢房内。你要吃时,教小厮拿钥匙取去。”李瓶儿还有头里吃的一碟烧鸭子、一碟鸡肉、一碟鲜鱼没动,教迎春安排了四碟小菜,切了一碟火薰肉,放下桌儿,在房中陪西门庆吃酒。西门庆更不问这嘎饭是那里,可见平日家中受用,这样东西无日不吃。西门庆饮酒中间想起,问李瓶儿:“头里书童拿的那帖儿是你与他的?”李瓶儿道:“是门外花大舅那里来说,教你饶了那伙人罢。”西门庆道:“前日吴大舅来说,我没依。若不是,我定要送问这起光棍。既是他那里分上,我明日到衙门里,每人打他一顿放了罢。”李瓶儿道:“又打他怎的?打的那雌牙露嘴。甚么模样!”西门庆道:“衙门是这等衙门,我管他雌牙不雌牙。还有比他娇贵的。”李瓶儿道:“我的哥哥,你做这刑名官,早晚公门中与人行些方便儿,也是你个阴骘,别的不打紧,只积你这点孩儿罢。”西门庆道:“可说什么哩!”李瓶儿道:“你到明日,也要少拶打人,得将就将就些儿,那里不是积福处。”西门庆道:“公事可惜不的情儿。”

两个正饮酒中间,只见春梅掀帘子进来。见西门庆正和李瓶儿腿压着腿儿吃酒,说道:“你每自在吃的好酒儿!这咱晚就不想使个小厮接接娘去?只有来安儿一个跟着轿子,隔门隔户,只怕来晚了,你倒放心!”西门庆见他花冠不整,云鬓蓬松,便满脸堆笑道:“小油嘴儿,我猜你睡来。”李瓶儿道:“你头上挑线汗巾儿跳上去了,还不往下拉拉!”因让他:“好甜金华酒,你吃锺儿。”西门庆道:“你吃,我使小厮接你娘去。”那春梅一手按着桌儿且兜鞋,因说道:“我才睡起来,心里恶拉拉,懒待吃。”西门庆道:“你看不出来,小油嘴吃好少酒儿!”李瓶儿道:“左右今日你娘不在,你吃上一锺儿怕怎的?”春梅道:“六娘,你老人家自饮,我心里本不待吃,俺娘在家不在家便怎的?就是娘在家,遇着我心不耐烦,他让我,我也不吃。”西门庆道:“你不吃,喝口茶儿罢。我使迎春前头叫个小厮,接你娘去。”因把手中吃的那盏木樨芝麻薰笋泡茶递与他。那春梅似有如无,接在手里,只呷了一口,就放下了。说道:“你不要教迎春叫去。我已叫了平安儿在这里,他还大些。”西门庆隔窗就叫平安儿。那小厮应道:“小的在这里伺候。”西门庆道:“你去了,谁看大门?”平安道:“小的委付棋童儿在门上。”西门庆道:“既如此,你快拿个灯笼接去罢。”

平安儿于是迳拿了灯笼来迎接潘金莲。迎到半路,只见来安儿跟着轿子从南来了。原来两个是熟抬轿的,一个叫张川儿,一个叫魏聪儿。走向前一把手拉住轿扛子,说道:“小的来接娘来了。”金莲就叫平安儿问道:“是你爹使你来接我?谁使你来?”平安道:“是爹使我来倒少!是姐使了小的接娘来了。”金莲道:“你爹想必衙门里没来家。”平安道:“没来家?门外拜了人,从后晌就来家了。在六娘房里,吃的好酒儿。若不是姐旋叫了小的进去,催逼着拿灯笼来接娘,还早哩!小的见来安一个跟着轿子,又小,只怕来晚了,路上不方便,须得个大的儿来接才好,小的才来了。”金莲又问:“你来时,你爹在那里?”平安道:“小的来时,爹还在六娘房里吃酒哩。姐禀问了爹,才打发了小的来了。”金莲听了,在轿子内半日没言语,冷笑骂道:“贼强人,把我只当亡故了的一般。一发在那淫妇屋里睡了长觉罢了。到明日,只交长远倚逞那尿胞种,只休要晌午错了。张川儿在这里听着,也没别人。你脚踏千家门、万家户,那里一个才尿出来的孩子,拿整绫缎尺头裁衣裳与他穿?你家就是王十万,使的使不的?”张川儿接过来道:“你老人家不说,小的也不敢说,这个可是使不的。不说可惜,倒只恐折了他,花麻痘疹还没见,好容易就能养活的大?去年东门外一个大庄屯人家,老儿六十岁,见居着祖父的前程,手里无碑记的银子,可是说的牛马成群,米粮无数,丫鬟侍妾成群,穿袍儿的身边也有十七八个。要个儿子花看样儿也没有。东庙里打斋,西寺里修供,舍经施像,那里没求到?不想他第七个房里,生了个儿子,喜欢的了不得。也像咱当家的一般,成日如同掌儿上看擎,锦绣窝儿里抱大。糊了三间雪洞儿的房,买了四五个养娘扶持。成日见了风也怎的,那消三岁,因出痘疹丢了。休怪小的说,倒是泼丢泼养的还好。”金莲道:“泼丢泼养?恨不得成日金子儿裹着他哩!”平安道:“小的还有桩事对娘说。小的若不说,到明日娘打听出来,又说小的不是了。便是韩伙计说的那伙人,爹衙门里都夹打了,收在监里,要送问他。今早应二爹来和书童儿说话,想必受了几两银子,大包子拿到铺子里,就便凿了二三两使了。买了许多东西嘎饭,在来兴屋里,教他媳妇子整治了,掇到六娘屋里,又买了两瓶金华酒,先和六娘吃了。又走到前边铺子里,和傅二叔、贲四、姐夫、玳安、来兴众人打伙儿,直吃到爹来家时分才散了。”金莲道:“他就不让你吃些?”平安道:“他让小的?好不大胆的蛮奴才!把娘每还不放在心上。不该小的说,还是爹惯了他,爹先不先和他在书房里干的龌龊营生。况他在县里当过门子,什么事儿不知道?爹若不早把那蛮奴才打发了,到明日咱这一家子吃他弄的坏了。”金莲问道:“在你六娘屋里吃酒,吃的多大回?”平安儿道:“吃了好一日儿。小的看见他吃的脸儿通红才出来。”金莲道:“你爹来家,就不说一句儿?”平安道:“爹也打牙粘住了,说什么!”金莲骂道:“恁贼没廉耻的昏君强盗!卖了儿子招女婿,彼此腾倒着做。”嘱付平安:“等他再和那蛮奴才在那里干这龌龊营生,你就来告我说。”平安道:“娘分咐,小的知道。娘也只放在心里,休要题出小的一字儿来。”于是跟着轿子,直说到家门首。

潘金莲下了轿,先进到后边拜见月娘。月娘道:“你住一夜,慌的就来了?”金莲道:“俺娘要留我住。他又招了俺姨那里一个十二岁的女孩儿在家过活,都挤在一个炕上,谁住他!又恐怕隔门隔户的,教我就来了。俺娘多多上复姐姐:多谢重礼。”于是拜毕月娘,又到李娇儿、孟玉楼众人房里,都拜了。回到前边,打听西门庆在李瓶儿屋里说话,迳来拜李瓶儿。李瓶儿见他进来,连忙起身,笑着迎接进房里来,说道:“姐姐来家早,请坐,吃锺酒儿。”教迎春:“快拿座儿与你五娘坐。”金莲道:“今日我偏了杯,重复吃了双席儿,不坐了。”说着,扬长抽身就去了。西门庆道:“好奴才,恁大胆,来家就不拜我拜儿?”那金莲接过来道:“我拜你?还没修福来哩。奴才不大胆,什么人大胆!”看官听说:潘金莲这几句话,分明讥讽李瓶儿,说他先和书童儿吃酒,然后又陪西门庆,岂不是双席儿,那西门庆怎晓得就理。正是:

\[
情知语是针和丝,就地引起是非来。
\]

\newpage
%# -*- coding:utf-8 -*-
%%%%%%%%%%%%%%%%%%%%%%%%%%%%%%%%%%%%%%%%%%%%%%%%%%%%%%%%%%%%%%%%%%%%%%%%%%%%%%%%%%%%%


\chapter{西门庆为男宠报仇\KG 书童儿作女妆媚客}


诗曰:

\[
娟娟游冶童,结束类妖姬。扬歌倚筝瑟,艳舞逞媚姿。
贵人一蛊惑,飞骑争相追。婉娈邀恩宠,百态随所施。
\]

话说西门庆早到衙门,先退厅与夏提刑说:“车淡四人再三寻人情来说,交将就他。”夏提刑道:“也有人到学生那边,不好对长官说。既是这等,如今提出来,戒饬他一番,放了罢。”西门庆道:“长官见得有理。”即升厅,令左右提出车淡等犯人跪下。生怕又打,只顾磕头。西门庆也不等夏提刑开言,就道:“我把你这起光棍,如何寻这许多人情来说!本当都送问,且饶你这遭,若再犯了我手里,都活监死。出去罢!”连韩二都喝出来了,往外金命水命,走投无命。这里处断公事不题。

且说应伯爵拿着五两银子,寻书童儿问他讨话,悄悄递与他银子。书童接的袖了。那平安儿在门首拿眼儿睃着他。书童于是如此这般:“昨日我替爹说了,今日往衙门里发落去了。”伯爵道:“他四个父兄再三说,恐怕又责罚他。”书童道:“你老人家只顾放心去,管情儿一下不打他。”那怕爵得了这消息,急急走去,回他们话去了。到早饭时分,四家人都到家,个个扑着父兄家属放声大哭。每人去了百十两银子,落了两腿疮,再不敢妄生事了。正是:

\[
祸患每从勉强得,烦恼皆因不忍生。
\]

却说那日西门庆未来家时,书童儿在书房内,叫来安儿扫地,向食盒内,把人家送的桌面上响糖与他吃。那小厮千不合万不合,叫:“书童哥,我有句话儿告你说。昨日俺平安哥接五娘轿子,在路上好不学舌,说哥的过犯。”书童问道:“他说我甚么来?”来安儿道:“他说哥揽的人家几两银子,大胆买了酒肉,送在六娘房里,吃了半日出来。又在前边铺子里吃,不与他吃。又说你在书房里,和爹干什么营生。”这书童听了,暗记在心,也不题起。到次日,西门庆早晨约会了,不往衙门里去,都往门外永福寺,置酒与须坐营送行去了。直到下午才来家,下马就分咐平安:“但有人来,只说还没来家。”说毕,进到厅上,书童儿接了衣裳。西门庆因问:“今日没人来?”书童道:“没有。管屯的徐老爹送了两包螃蟹、十斤鲜鱼。小的拿回帖打发去了,与了来人一钱银子。又有吴大舅送了六个帖儿,明日请娘们吃三日。”原来吴大舅子吴舜臣,娶了乔大户娘子侄女儿郑三姐做媳妇儿,西门庆送了茶去,他那里来请。

西门庆到后边,月娘拿了帖儿与他瞧,西门庆说道:“明日你们都收拾了去。”说毕,出来到书房里坐下。书童连忙拿炭火炉内烧甜香饼儿,双手递茶上去。西门庆擎茶在手。他慢慢挨近站立在桌边。良久,西门庆努了个嘴儿,使他把门关上,用手搂在怀里,一手捧着他的脸儿。西门庆吐舌头,那小郎口里噙着凤香饼儿递与他,下边又替他弄玉茎。西门庆问道:“我儿,外边没人欺负你?”那小厮乘机就说:“小的有桩事,不是爹问,小的不敢说。”西门庆道:“你说不妨。”书童就把平安一节告说一遍:“前日爹叫小的在屋里,他和画童在窗外听觑,小的出来舀水与爹洗手,亲自看见。他又在外边对着人骂小的蛮奴才,百般欺负小的。”西门庆听了,心中大怒,说道:“我若不把奴才腿卸下来也不算!”这里书房中说话不题。

且说平安儿专一打听这件事,三不知走去报与金莲。金莲使春梅前边来请西门庆说话。刚转过松墙,只见画童儿在那里弄松虎儿,便道:“姐来做什么?爹在书房里。”被春梅头上凿了一下。西门庆在里面听见裙子响,就知有人来,连忙推开小厮,走在床上睡着。那书童在桌上弄笔砚,春梅推门进来,见了西门庆,咂嘴儿说道:“你们悄悄的在屋里,把门儿关着,敢守亲哩!娘请你说话。”西门庆仰睡在枕头上,便道:“小油嘴儿,他请我说什么话?你先行,等我略倘倘儿就去!”那春梅那里容他,说道:“你不去,我就拉起你来!”西门庆怎禁他死拉活拉,拉到金莲房中。金莲问:“他在前头做什么?”春梅道:“他和小厮两个在书房里,把门儿插着,捏杀蝇儿子是的,知道干的甚么茧儿,恰是守亲的一般。我进去,小厮在桌子跟前推写字,他便倘剌在床上,拉着再不肯来。”潘金莲道:“他进来我这屋里,只怕有锅镬吃了他是的。贼没廉耻的货,你想,有个廉耻,大白日和那奴才平白关着门做什么来?左右是奴才臭屁股门子,钻了,到晚夕还进屋里,和俺每沾身睡,好干净儿!”西门庆道:“你信小油嘴儿胡说,我那里有此勾当!我看着他写礼帖儿来,我便\textuni{22C49}在床上。”金莲道:“巴巴的关着门儿写礼帖?什么机密谣言,什么三只腿的金刚、两个\textuni{27900}角的象,怕人瞧见?明日吴大妗子家做三日,掠了个帖子儿来,不长不短的,也寻件甚么子与我做拜钱。你不与,莫不教我和野汉子要!大姐姐是一套衣裳、五钱银子,别人也有簪子的,也有花的。只我没有,我就不去了!”西门庆道:“前边厨柜内拿一匹红纱来,与你做拜钱罢。”金莲道,“我就去不成,也不要那嚣纱片子,拿出去倒没的教人笑话!”西门庆道:“你休乱,等我往那边楼上,寻一件什么与他便了。如今往东京送贺礼,也要几匹尺头,一答儿寻下来罢。”于是走到李瓶儿那边楼上,寻了两匹玄色织金麒麟补子尺头、两个南京色缎、一匹大红斗牛纻丝、一匹翠蓝云缎。因对李瓶儿说:“要寻一件云绢衫与金莲做拜钱,如无,拿帖缎子铺讨去罢。”李瓶儿道:“你不要铺子里取去,我有一件织金云绢衣服哩!大红衫儿、蓝裙,留下一件也不中用,俺两个都做了拜钱罢。”一面向箱中取出来。李瓶儿亲自拿与金莲瞧:“随姐姐拣,衫儿也得,裙儿也得,咱两个一事包了做拜钱倒好,省得又取去。”金莲道:“你的,我怎好要?”李瓶儿道:“好姐姐,怎生恁说话!”推了半日,金莲方才肯了。又出去教陈敬济换了腰封,写了二人名字在上,不题。

且说平安儿正在大门首,只见白赉光走来问道:“大官人在家么?”平安儿道:“俺爹不在家了。”那白赉光不信,迳入里面厅上,见槅子关着,说道:“果然不在家。往那里去了?”平安道:“今日门外送行去了,还没来。”白赉光道:“既是送行,这咱晚也该来家了。”平安道:“白大叔有甚话说下,待爹来家,小的禀就是了。”白赉光道:“没什么活,只是许多时没见,闲来望望。既不在,我等等罢。”平安道:“只怕来晚了,你老人家等不得。”白赉光不依,把槅子推开,进入厅内,在椅子上就坐了。众小厮也不理他,由他坐去。不想天假其便,西门庆教迎春抱着尺头,从后边走来,刚转过软壁,顶头就撞见白赉光在厅上坐着。迎春儿丢下缎子,往后走不迭。白赉光道:“这不是哥在家!”一面走下来唱喏。西门庆见了,推辞不得,须索让坐。睃见白赉光头戴着一顶出洗覆盔过的、恰如太山游到岭的旧罗帽儿,身穿着一件坏领磨襟救火的硬浆白布衫,脚下靸着一双乍板唱曲儿前后弯绝户绽的皂靴,里边插着一双一碌子蝇子打不到、黄丝转香马凳袜子。坐下,也不叫茶,见琴童在旁伺候,就分咐:“把尺头抱到客房里,教你姐夫封去。”那琴童应诺,抱尺头往厢房里去了。白赉光举手道:“一向欠情,没来望的哥。”西门庆道:“多谢挂意。我也常不在家,日逐衙门中有事。”白赉光道:“哥这衙门中也日日去么?”西门庆道:“日日去两次,每日坐厅问事。到朔望日子,还要拜牌,画公座,大发放,地方保甲番役打卯。归家便有许多穷冗,无片时闲暇。今日门外去,因须南溪新升了新平寨坐营,众人和他送行,只刚到家。明日管皇庄薛公公家请吃酒,路远去不成。后日又要打听接新巡按。又是东京太师老爷四公子又选了驸马,童太尉侄男童天\textuni{38E7}新选上大堂,升指挥使佥书管事。两三层都要贺礼。这连日通辛苦的了不得。”说了半日语,来安儿才拿上茶来。白贲光才拿在手里呷了一口,只见玳安拿着大红帖儿往里飞跑,报道:“掌刑的夏老爹来了!外边下马了。”西门庆就往后边穿衣服去了。白贲光躲在西厢房内,打帘里望外张看。

良久,夏提刑进到厅上,西门庆冠带从后边迎将来。两个叙礼毕,分宾主坐下。不一时,棋童儿拿了两盏茶来吃了。夏提刑道:“昨日所言接大巡的事,今日学生差人打听,姓曾,乙未进士,牌已行到东昌地方。他列位每都明日起身远接。你我虽是武官,系领敕衙门提点刑狱,比军卫有司不同。咱后日起身,离城十里寻个去所,预备一顿饭,那里接见罢!”西门庆道:“长官所言甚妙,也不消长官费心,学生这里着人寻个庵观寺院,或是人家庄园亦好,教个厨役早去整理。”夏提刑谢道:“这等又教长官费心。”说毕,又吃了一道茶,夏提刑起身去了。

西门庆送了进来,宽去衣裳。那白贲光还不去,走到厅上又坐下了。对西门庆说:“自从哥这两个月没往会里去,把会来就散了。老孙虽年纪大,主不得事。应二哥又不管。昨日七月内,玉皇庙打中元醮,连我只三四个人到,没个人拿出钱来,都打撒手儿。难为吴道官,晚夕谢将,又叫了个说书的,甚是破费他。他虽故不言语,各人心上不安。不如那咱哥做会首时,还有个张主。不久还要请哥上会去。”西门庆道:“你没的说散便散了罢,那里得工夫干此事?遇闲时,在吴先生那里一年打上个醮,答报答报天地就是了。随你们会不会,不消来对我说。”几句话抢白的白赉光没言语了。又坐了一回,西门庆见他不去,只得唤琴童儿厢房内放桌儿,拿了四碟小菜,牵荤连素,一碟煎面筋、一碟烧肉。西门庆陪他吃了饭。筛酒上来,西门庆又讨副银镶大锺来,斟与他。吃了几锺,白赉光才起身。西门庆送到二门首,说道:“你休怪我不送你,我戴着小帽,不好出去得。”那白赉光告辞去了。

西门庆回到厅上,拉了把椅子坐下,就一片声叫平安儿。那平安儿走到跟前,西门庆骂道:“贼奴才,还站着?”叫答应的,就是三四个排军在旁伺候。那平安不知甚么缘故,唬的脸蜡查黄,跪下了。西门庆道:“我进门就分咐你,但有人来,答应不在。你如何不听?”平安道:“白大叔来时,小的回说爹往门外送行去了,没来家。他不信,强着进来了。小的就跟进来问他:‘有话说下,待爹来家,小的禀就是了。’他又不言语,自家推开厅上槅子坐下。落后,不想出来就撞见了。”西门庆骂道:“你这奴才,不要说嘴!你好小胆子儿?人进来,你在那里耍钱吃酒去来,不在大门首守着!”令左右:“你闻他口里。”那排军闻了一闻,禀道:“没酒气。”西门庆分咐:“叫两个会动刑的上来,与我着实拶这奴才!”当下两个伏侍一个,套上拶指,只顾擎起来。拶的平安疼痛难忍,叫道:“小的委实回爹不在,他强着进来。”那排军拶上,把绳子绾住,跪下禀道:“拶上了。”西门庆道:“再与我敲五十敲。”旁边数着,敲到五十上住了手。西门庆分咐:“打二十棍!”须臾打了二十,打的皮开肉绽,满腿血淋。西门庆喝令:“与我放了。”两个排军向前解了拶子,解的直声呼唤。西门庆骂道:“我把你这贼奴才!你说你在大门首,想说要人家钱儿,在外边坏我的事,休吹到我耳朵内,把你这奴才腿卸下来!”那平安磕了头起来,提着裤子往外去了。西门庆看见画童儿在旁边,说道:“把这小奴才拿下去,也拶他一拶子。”一面拶的小厮杀猪儿似怪叫。这里西门庆在前厅拶人不题。

单说潘金莲从房里出来往后走,刚走到大厅后仪门首,只见孟玉楼独自一个在软壁后听觑。金莲便问:“你在此听甚么儿哩?”玉楼道:“我在这里听他爹打平安儿,连画童小奴才也拶了一拶子,不知为什么。”一回棋童儿过来,玉楼叫住问他:“为什么打平安儿?”棋童道:“爹嗔他放进白赉光来了。”金莲接过来道:“也不是为放进白赉光来,敢是为他打了象牙来,不是打了象牙,平白为什么打得小厮这样的!贼没廉耻的货,亦发脸做了主了。想有些廉耻儿也怎的!”那棋童就走了。玉楼便问金莲:“怎的打了象牙?”金莲道:“我要告诉你,还没告诉你。我前日去俺妈家做生日去了,不在家,蛮秫秫小厮揽了人家说事几两银子,买两盒嘎饭,又是一坛金华酒,掇到李瓶儿房里,和小厮吃了半日酒,小厮才出来。没廉耻货来家,也不言语,还和小厮在花园书房里,插着门儿,两个不知干着什么营生。平安这小厮拿着人家帖子进去,见门关着,就在窗下站着了。蛮小厮开门看见了,想是学与贼没廉耻的货,今日挟仇打这小厮,打的膫子成。那怕蛮奴才到明日把一家子都收拾了,管人吊脚儿事!”玉楼笑道:“好说,虽是一家子,有贤有愚,莫不都心邪了罢?”金莲道:“不是这般说,等我告诉你。如今这家中,他心肝肐蒂儿偏欢喜的只两个人,一个在里,一个在外,成日把魂恰似落在他身上一般,见了说也有,笑也有。俺们是没时运的,行动就是乌眼鸡一般。贼不逢好死变心的强盗!通把心狐迷住了,更变的如今相他哩!三姐你听着,到明日弄出什么八怪七喇出来!今日为拜钱,又和他合了回气。但来家,就在书房里。今日我使春梅叫他来,谁知大白日里和贼蛮奴才关着门儿哩!春梅推门入去,唬的一个个眼张失道的。到屋里,教我尽力数骂了几句。他只顾左遮右掩的。先拿一匹红纱与我做拜钱,我不要。落后往李瓶儿那边楼上寻去。贼人胆儿虚,自知理亏,拿了他箱内一套织金衣服来,亲自来尽我,我只是不要。他慌了,说:‘姐姐,怎的这般计较!姐姐拣衫儿也得,裙儿也得。看了,好拿到前边,教陈姐夫封写去。’尽了半日,我才吐了口儿。他让我要了衫子。”玉楼道:“这也罢了,也是他的尽让之情。”金莲道:“你不知道,不要让了他。如今年世,只怕睁着眼儿的金刚,不怕闭着眼儿的佛!老婆汉子,你若放些松儿与他,王兵马的皂隶——还把你不当\textuni{34B2}的。”玉楼戏道,“六丫头,你是属面筋的,倒且是有靳道。”说着,两个笑了。只见小玉来请:“三娘、五娘,后边吃螃蟹哩!我去请六娘和大姑娘去。”

两个手拉着手儿进来,月娘和李娇儿正在上房穿廊下坐,说道:“你两个笑什么?”金莲道:“我笑他爹打平安儿。”月娘道:“嗔他恁乱蝍\textuni{45EB}叫喊的,只道打什么人?原来打他。为什么来,”金莲道:“为他打折了象牙了。”月娘老实,便问“象牙放在那里来,怎的教他打折了?”那潘金莲和孟玉楼两个嘻嘻哈哈,只顾笑成一块。月娘道:“不知你每笑什么,不对我说。”玉楼道:“姐姐你不知道,爹打平安为放进白赉光来了。”月娘道:“放进白赉光便罢了,怎么说道打了象牙?也没见这般没稍干的人,在家闭着膫子坐,平白有要没紧来人家撞些什么!”来安道:“他来望爹来了。”月娘道:“那个吊下炕来了?望,没的扯臊淡,不说来抹嘴吃罢了。”良久,李瓶儿和大姐来到,众人围绕吃螃蟹。月娘分咐小玉:“屋里还有些葡萄酒,筛来与你娘每吃。”金莲快嘴,说道:“吃螃蟹得些金华酒吃才好!”又道:“只刚一味螃蟹就着酒吃,得只烧鸭儿撕了来下酒。”月娘道:“这咱晚那里买烧鸭子去!”李瓶儿听了,把脸飞红了。正是:话头儿包含着深意,题目儿哩暗蓄着留心。那月娘是个诚实的人,怎晓的话中之话。这里吃螃蟹不题。

且说平安儿被责,来到外边,贲四、来兴众人都乱来问平安儿:“爹为甚么打你?”平安哭道:“我知为甚么!”来兴儿道:“爹嗔他放进白赉光来了。”平安道,“早是头里你看着,我那等拦他,他只强着进去了。不想爹从后边出来撞见了,又没甚话,吃了茶,再不起身。只见夏老爹来了,我说他去了,他还躲在厢房里又不去。直等拿酒来吃了才去。倒惹的打我这一顿,你说我不造化低!我没拦他?又说我没拦他。他强自进来,管我腿事!打我!教那个贼天杀男盗女娼的狗骨秃,吃了俺家这东西,打背梁脊下过!”来兴儿道:“烂折脊梁骨,倒好了他往下撞!”平安道:“教他生噎食病,把颡根轴子烂吊了。天下有没廉耻皮脸的,不相这狗骨秃没廉耻,来我家闯的狗也不咬。贼雌饭吃花子\textuni{34B2}的,再不烂了贼忘八的屁股门子!”来兴笑道:“烂了屁股门子,人不知道,只说是臊的。”众人都笑了。平安道:“想必是家里没晚米做饭,老婆不知饿的怎么样的。闲的没的干,来人家抹嘴吃。图家里省了一顿,也不是常法儿。不如教老婆养汉,做了忘八倒硬朗些,不教下人唾骂。”玳安在铺子里篦头,篦了,打发那人钱去了,走出来说:“平安儿,我不言语,憋的我慌。亏你还答应主子,当家的性格,你还不知道?你怎怪人?常言养儿不要屙金溺银,只要见景生情。比不的应二叔和谢叔来,答应在家不在家,他彼此都是心甜厚间便罢了。以下的人,他又分咐你答应不在家,你怎的放人来?不打你却打谁!”贲四戏道:“平安儿从新做了小孩儿,才学闲闲,他又会顽,成日只踢毬儿耍子。”众人又笑了一回。贲四道:“他便为放人进来,这画童儿却为什么,也陪拶了一拶子?是甚好吃的果子,陪吃个儿?吃酒吃肉也有个陪客,十个指头套在拶子上,也有个陪的来?”那画童儿揉着手,只是哭。玳安戏道:“我儿少哭,你娘养的你忒娇,把馓子儿拿绳儿拴在你手儿上,你还不吃?”这里前边小厮热乱不题。

西门庆在厢房中,看着陈敬济封了礼物尺头,写了揭帖,次日早打发人上东京,送蔡驸马、童堂上礼,不在话下。到次日,西门庆往衙门里去了。吴月娘与众房,共五顶轿子,头戴珠翠,身穿锦绣,来兴媳妇一顶小轿跟随,往吴大妗家做三日去了。止留下孙雪娥在家中,和西门大姐看家。早间韩道国送礼相谢:一坛金华酒,一只水晶鹅,一副蹄子,四只烧鸭,四尾鲥鱼。帖子上写着“晚生韩道国顿首拜”。书童因没人在家,不敢收,连盒担留下,待的西门庆衙门回来,拿与西门庆瞧。西门庆使琴童儿铺子里旋叫了韩伙计来,甚是说他:“没分晓,又买这礼来做甚么!我决然不受!”那韩道国拜说:“小人蒙老爹莫大之恩,可怜见与小人出了气,小人举家感激不尽。无甚微物,表一点穷心。望乞老爹好歹笑纳。”西门庆道:“这个使不得。你是我门下伙计,如同一家,我如何受你的礼!即令原人与我抬回去。”韩道国慌了,央说了半日。西门庆分咐左右,只受了鹅酒,别的礼都令抬回去了。教小厮拿帖儿,请应二爹和谢爹去,对韩道国说:“你后晌叫来保看着铺子,你来坐坐。”韩道国说:“礼物不受,又教老爹费心。”应诺去了。

西门庆又添买了许多菜蔬,后晌时分,在翡翠轩卷棚内,放下一张八仙桌儿。应伯爵、谢希大先到了。西门庆告他说:“韩伙计费心,买礼来谢我,我再三不受他,他只顾死活央告,只留了他鹅酒。我怎好独享,请你二位陪他坐坐。”伯爵道:“他和我讨较来,要买礼谢。我说你大官府那里稀罕你的,休要费心,你就送去,他决然不受。如何?我恰似打你肚子里钻过一遭的,果然不受他的。”说毕,吃了茶,两个打双陆。不一时,韩道国到了,二人叙礼毕坐下。应伯爵、谢希大居上,西门庆关席,韩道国打横。登时四盘四碗拿来,桌上摆了许多下饭,把金华酒分咐来安儿就在旁边打开,用铜甑儿筛热了拿来,教书童斟酒。伯爵分咐书童儿:“后边对你大娘房里说,怎的不拿出螃蟹来与应二爹吃?你去说我要螃蟹吃哩。”西门庆道:“傻狗才,那里有一个螃蟹!实和你说,管屯的徐大人送了我两包螃蟹,到如今娘们都吃了,剩下腌了几个。”分咐小厮:“把腌螃蟹\textuni{22D5E}几个来。今日娘们都往吴妗子家做三日去了。”不一时,画童拿了两盘子腌蟹上来。那应伯爵和谢希大两个抢着,吃的净光。因见书童儿斟酒,说道:“你应二爹一生不吃哑酒,自夸你会唱的南曲,我不曾听见。今日你好歹唱个儿,我才吃这锺酒。”那书童才待拍着手唱,伯爵道:“这等唱一万个也不算。你装龙似龙,装虎似虎,下边搽画装扮起来,相个旦儿的模样才好。”那书童在席上,把眼只看西门庆的声色儿。西门庆笑骂伯爵:“你这狗才,专一歪厮缠人!”因向书童道:“既是他索落你,教玳安儿前边问你姐要了衣服,下边妆扮了来。”玳安先走到前边金莲房里问春梅要,春梅不与。旋往后问上房玉萧要了四根银簪子,一个梳背儿,面前一件仙子儿,一双金镶假青石头坠子,大红对衿绢衫儿,绿重绢裙子,紫销金箍儿。要了些脂粉,在书房里搽抹起来,俨然就如个女子,打扮的甚是娇娜。走在席边,双手先递上一杯与应伯爵,顿开喉音,在旁唱《玉芙蓉》道:

\[
残红水上飘,梅子枝头小。这些时,眉儿淡了谁描?因春带得愁来到,春去缘何愁未消?人别后,山遥水遥。我为你数归期,画损了掠儿稍。
\]

伯爵听了,夸奖不已,说道:“相这大官儿,不在了与他碗饭吃。你看他这喉音,就是一管萧。说那院里小娘儿便怎的,那些唱都听熟了。怎生如他这等滋润!哥,不是俺们面奖,似你这般的人儿在你身边,你不喜欢!”西门庆笑了。怕爵道:“哥,你怎的笑?我到说的正经话。你休亏这孩子,凡事衣类儿上,另着个眼儿看他。难为李大人送了他来,也是他的盛情。”西门庆道:“正是。如今我不在家,书房中一应大小事,都是他和小婿。小婿又要铺子里兼看看。”应伯爵饮过,又斟双杯。伯爵道:“你替我吃些儿。”书童道:“小的不敢吃,不会吃。”伯爵道:“你不吃,我就恼了。我赏你待怎的?”书童只顾把眼看西门庆。西门庆道:“也罢,应二爹赏你,你吃了。”那小厮打了个佥儿,慢慢低垂粉颈,呷了一口。余下半锺残酒,用手擎着,与伯爵吃了。方才转过身来,递谢希大酒,又唱了个曲儿。谢希大问西门庆道:“哥,书官儿青春多少?”西门庆道:“他今年才交十六岁。”问道:“你也会多少南曲?”书童道:“小的也记不多几个曲子,胡乱答应爹们罢了。”希大道:“好个乖觉孩子!”亦照前递了酒。下来递韩道国。道国道:“老爹在上,小的怎敢欺心。”西门庆道:“今日你是客。”韩道国道:“那有此理!还是从老爹上来,次后才是小人吃酒。”书童下席来递西门庆酒,又唱了一个曲儿。西门庆吃毕,到韩道国跟前。韩道国慌忙立起身来接酒。伯爵道:“你坐着,教他好唱。”韩道国方才坐下。书童又唱了个曲儿。韩道国未等词终,连忙一饮而尽。

正饮酒中间,只见玳安来说:“贲四叔来了,请爹说话。”西门庆道:“你叫他来这里说罢。”不一时,贲四进来,向前作了揖,旁边安顿坐了。玳安又取一双锺箸放下。西门庆令玳安后边取菜蔬。西门庆因问他:“庄子上收拾怎的样了?”贲四道:“前一层才盖瓦,后边卷棚昨日才打的基,还有两边厢房与后一层住房的料,都没有。客位与卷棚漫地尺二方砖,还得五百,那旧的都使不得。砌墙的大城角也没了。垫地脚带山子上土,也添勾了百多车子。灰还得二十两银子的。”西门庆道:“那灰不打紧,我明日衙门里分咐灰户,教他送去。昨日你砖厂刘公公说送我些砖儿。你开个数儿,封几两银子送与他,须是一半人情儿回去。只少这木植。”贲四道:“昨日老爹分咐,门外看那庄子,今早同张安儿去看,原来是向皇亲家庄子。大皇亲没了,如今向五要卖神路明堂。咱们不要他的,讲过只拆他三间厅、六间厢房、一层群房就勾了。他口气要五百两。到跟前拿银子和他讲,三百五十两上,也该拆他的。休说木料,光砖瓦连土也值一二百两银子。”应伯爵道:“我道是谁来!是向五的那庄子。向五被人争地土,告在屯田兵备道,打官司使了好多银子。又在院里包着罗存儿。如今手里弄的没钱了。你若要,与他三百两银子,他也罢了。冷手挝不着热馒头。”西门庆分咐贲四:“你明日拿两锭大银子,同张安儿和他讲去,若三百两银子肯,拆了来罢。”贲四道:“小人理会。”良久,后边拿了一碗汤、一盘蒸饼上来,贲四吃了。斟上,陪众人吃酒。书童唱了一遍,下去了。

应伯爵道:“这等吃的酒没趣。取个骰盆儿,俺们行个令儿吃才好。”西门庆令玳安:“就在前边六娘屋里取个骰盆来。”不一时,玳安取了来,放在伯爵跟前,悄悄走到西门庆耳边说:“六娘房里哥哭哩。迎春姐叫爹着个人儿接接六娘去。”西门庆道:“你放下壶,快叫个小厮拿灯笼接去!”因问:“那两个小厮在那里?”玳安道:“琴童与棋童儿先拿两个灯笼接去了。”伯爵见盆内放着六个骰儿,即用手拈着一个,说:“我掷着点儿,各人要骨牌名一句儿,见合着点数儿,如说不过来,罚一大杯酒。下家唱曲儿,不会唱曲儿说笑话儿,两桩儿不会,定罚一大杯。”西门庆道:“怪狗才,忒韶刀了!”伯爵道:“令官放个屁,也钦此钦遵。你管我怎的!”叫来安:“你且先斟一杯,罚了爹,然后好行令。”西门庆笑而饮之。伯爵道:“众人听着,我起令了!说差了也罚一杯。”说道:“张生醉倒在西厢。吃了多少酒?一大壶,两小壶,”果然是个么。西门庆叫书童儿上来斟酒,该下家谢希大唱。希大拍着手儿道:“我唱个《折桂令》儿你听罢。”唱道:

\[
可人心二八娇娃,百件风流,所事撑达。眉蹙春山,眼横秋水,鬓绾着乌鸦。干相思,撇不下一时半霎;咫尺间,如隔着海角天涯。瘦也因他,病也因他。谁与做个成就了姻缘,便是那救苦难的菩萨。
\]

伯爵吃了酒,过盆与谢希大掷,轮着西门庆唱。谢希大拿过骰儿来说:“多谢红儿扶上床。甚么时候?三更四点。”可是作怪,掷出个四来。伯爵道:“谢子纯该吃四杯。”希大道:“折两杯罢,我吃不得。”书童儿满斟了两杯,先吃了头一杯,等他唱。席上伯爵二人把一碟子荸荠都吃了。西门庆道:“我不会唱,说个笑话儿罢。”说道:“一个人到果子铺问:“可有榧子么?”那人说有。取来看,那买果子的不住的往口里放。卖果子的说:‘你不买,如何只顾吃?’那人道:‘我图他润肺。’那卖的说:‘你便润了肺,我却心疼。’”众人都笑了。伯爵道:“你若心疼,再拿两碟子来。我媒人婆拾马粪——越发越晒。”谢希大吃了。第三该西门庆掷。说:“留下金钗与表记。多少重?五六七钱。”西门庆拈起骰儿来,掷了个五。书童儿也只斟上两锺半酒。谢希大道:“哥大量,也吃两杯儿,没这个理。哥吃四锺罢,只当俺一家孝顺一锺儿。”该韩伙计唱。韩道国让:“贲四哥年长。”贲四道:“我不会唱,说个笑话儿罢。”西门庆吃过两锺,贲四说道:“一官问奸情事。问:‘你当初如何奸他来?’那男子说:‘头朝东,脚也朝东奸来。’官云:‘胡说!那里有个缺着行房的道理!’旁边一个人走来跪下,说道:‘告禀,若缺刑房,待小的补了罢!’”应伯爵道:“好贲四哥,你便益不失当家!你大官府又不老,别的还可说,你怎么一个行房,你也补他的?”贲四听见此言,唬的把脸通红了,说道:“二叔,什么话!小人出于无心。”伯爵道:“什么话?檀木靶,没了刀儿,只有刀鞘儿了。”那贲四在席上终是坐不住,去又不好去,如坐针毡相似。西门庆饮毕四锺酒,就轮该贲四掷。贲四才待拿起骰子来,只见来安儿来请:“贲四叔,外边有人寻你。我问他,说是窑上人。”这贲四巴不得要去,听见这一声,一个金蝉脱壳走了。西门庆道:“他去了,韩伙计你掷罢。”韩道国举起骰儿道:“小人遵令了。”说道:“夫人将棒打红娘。打多少?八九十下。”伯爵道:“该我唱,我不唱罢,我也说个笑话儿。教书童合席都筛上酒,连你爹也筛上。听我这个笑话:一个道士,师徒二人往人家送疏。行到施主门首,徒弟把绦儿松了些,垂下来。师父说:‘你看那样!倒相没屁股的。’徒弟回头答道:‘我没屁股,师父你一日也成不得。’”西门庆骂道:“你这歪狗才,狗口里吐出什么象牙来!”这里饮酒不题。

且说玳安先到前边,又叫了画童,拿着灯笼,来吴大妗子家接李瓶儿。瓶儿听见说家里孩子哭,也等不得上拜,留下拜钱,就要告辞来家。吴大妗、二妗子那里肯放:“好歹等他两口儿上了拜儿!”月娘道:“大妗子,你不知道,倒教他家去罢。家里没人,孩子好不寻他哭哩!俺多坐回儿不妨事。”那吴大妗子才放了李瓶儿出门。玳安丢下画童,和琴童儿两个随轿子先来家了。落后,上了拜,堂客散时,月娘等四乘轿子,只打着一个灯笼,况是八月二十四日,月黑时分。月娘问:“别的灯笼在那里,如何只一个?”棋童道:“小的原拿了两个来。玳安要了一个,和琴童先跟六娘家去了。”月娘便不问,就罢了。潘金莲有心,便问棋童:“你们头里拿几个来?”棋童道:“小的和琴童拿了两个来,落后玳安与画童又要了一个去,把画童换下,和琴童先跟了六娘去了。”金莲道:“玳安那囚根子,他没拿灯笼来?”画童道:“我和他又拿了一个灯笼来了。”金莲道:“既是有一个就罢了,怎的又问你要这个?”棋童道:“我那等说,他强着夺了去。”金莲便叫吴月娘:“姐姐,你看玳安恁贼献勤的奴才!等到家和他答话。”月娘道:“奈烦,孩子家里紧等着,叫他打了去罢了。”金莲道:“姐姐,不是这等说。俺便罢了,你是个大娘子,没些家法儿,晴天还好,这等月黑,四顶轿子只点着一个灯笼,顾那些儿的是?”

说着轿子到了门首。月娘、李娇儿便往后边去了。金莲和孟玉楼一答儿下轿,进门就问,“玳安儿在那里?”平安道:“在后边伺候哩!”刚说着,玳安出来,被金莲骂了几句:“我把你献勤的囚根子!明日你只认清了,单拣着有时运的跟,只休要把脚儿踢踢儿。有一个灯笼打着罢了,信那斜汗世界一般又夺了个来。又把小厮也换了来。他一顶轿子,倒占了两个灯笼,俺们四顶轿子,反打着一个灯笼,俺们不是爹的老婆?”玳安道:“娘错怪小的了。爹见哥儿哭,教小的:‘快打灯笼接你六娘先来家罢,恐怕哭坏了哥儿。’莫不爹不使我,我好干着接去来!”金莲道:“你这囚根子,不要说嘴!他教你接去,没教你把灯笼都拿了来。哥哥,你的雀儿只拣旺处飞,休要认差了,冷灶上着一把儿、热灶上着一把儿才好。俺们天生就是没时运的来?”玳安道:“娘说的什么话!小的但有这心,骑马把脯子骨撞折了!”金莲道:“你这欺心的囚根子!不要慌,我洗净眼儿看着你哩!”说着,和玉楼往后边去了。那玳安对着众人说:“我精晦气的营生,平自爹使我接去,却被五娘骂了恁一顿。”

玉楼、金莲二人到仪门首,撞见来安儿,问:“你爹在那里哩?”来安道:“爹和应二爹、谢爹、韩大叔还在卷棚内吃酒。书童哥装了个唱的,在那里唱哩,娘每瞧瞧去。”二人间走到卷棚槅子外,往里观看。只见应伯爵在上坐着,把帽儿歪挺着,醉的只相线儿提的。谢希大醉的把眼儿通睁不开。书童便妆扮在旁边斟酒唱南曲。西门庆悄悄使琴童儿抹了伯爵一脸粉,又拿草圈儿从后边悄悄儿弄在他头上作戏。把金莲和玉楼在外边忍不住只是笑,骂:“贼囚根子,到明日死了也没罪了,把丑都出尽了!”西门庆听见外边笑,使小厮出来问是谁,二人才往后边去了。散时,已一更天气了。西门庆那日往李瓶儿房里睡去了。金莲归房,因问春梅:“李瓶儿来家说甚么话来?”春梅道:“没说甚么。”金莲又问:“那没廉耻货,进他屋里去来没有?”春梅道:“六娘来家,爹往他房里还走了两遭。”金莲道:“真个是因孩子哭接他来?”春梅道:“孩子后晌好不怪哭的,抱着也哭,放下也哭,再没法处。前边对爹说了,才使小厮接去。”金莲道:“若是这等也罢了。我说又是没廉耻的货,三等儿九般使了接去。”又问:“书童那奴才,穿的是谁的衣服?”春梅道:“先来问我要,教我骂了玳安出去。落后,和玉箫借了。”金莲道:“再要来,休要与秫秫奴才穿。”说毕,见西门庆不来,使性儿关门睡了。

且说应伯爵见贲四管工,在庄子上赚钱,明日又拿银子买向五皇亲房子,少说也有几两银子背。正行令之间,可可见贲四不防头,说出这个笑话儿来。伯爵因此错他这一错,使他知道。贲四果然害怕,次日封了三两银子,亲到伯爵家磕头。伯爵反打张惊儿,说道:“我没曾在你面上尽得心,何故行此事?”贲四道:“小人一向缺礼,早晚只望二叔在老爹面前扶持一二,足感不尽!”伯爵于是把银子收了,待了一锺茶,打发贲四出门。拿银子到房中,与他娘子儿说:“老儿不发狠,婆儿没布裙。贲四这狗啃的,我举保他一场,他得了买卖,扒自饭碗儿,就不用着我了。大官人教他在庄子上管工,明日又托他拿银子成向五家庄子,一向赚的钱也勾了。我昨日在酒席上,拿言语错了他错儿,他慌了,不怕他今日不来求我。送了我三两银子,我且买几匹布,勾孩子们冬衣了。”正是:

\[
只恨闲愁成懊恼,岂知伶俐不如痴。
\]

\newpage
%# -*- coding:utf-8 -*-
%%%%%%%%%%%%%%%%%%%%%%%%%%%%%%%%%%%%%%%%%%%%%%%%%%%%%%%%%%%%%%%%%%%%%%%%%%%%%%%%%%%%%


\chapter{翟管家寄书寻女子\KG 蔡状元留饮借盘缠}


诗曰:

\[
既伤千里目,还惊远去魂。岂不惮跋涉?深怀国士恩。
季布无一诺,侯嬴重一言。人生感意气,黄金何足论。
\]

话说次日,西门庆早与夏提刑接了新巡按,又到庄上犒劳做活的匠人。至晚来家,平安进门就禀:“今日有东昌府下文书快手,往京里顺便捎了一封书帕来,说是太师爷府里翟大爹寄来与爹的。小的接了,交进大娘房里去了。那人明日午后来讨回书。”西门庆听了,走到上房,取书拆开观看,上面写着:

\[
京都侍生翟谦顿首书拜即擢大锦堂西门大人门下:久仰山斗,未接丰标,屡辱厚情,感愧何尽!前蒙驰谕,生铭刻在心。凡百于老爷左右,无不尽力扶持。所有小事,曾托盛价烦渎,想已为我处之矣。今日鸿便,薄具帖金十两奉贺,兼候起居。伏望俯赐回音,生不胜感激之至。外新状元蔡一泉,乃老爷之假子,奉敕回籍省视,道经贵处,仍望留之一饭,彼亦不敢有忘也。至祝至祝!秋后一日信。
\]

西门庆看毕,只顾咨嗟不已,说道:“快叫小厮叫媒人去。我什么营生,就忘死了。”吴月娘问:“甚么勾当?”西门庆道:“东京太师老爷府里翟管家,前日有书来,说无子,央及我这里替他寻个女子。不拘贫富,不限财礼,只要好的,他要图生长。妆奁财礼,该使多少,教我开了去,他一一还我,往后他在老爷面前,一力扶持我做官。我一向乱着上任,七事八事,就把这事忘死了。来保又日逐往铺子里去了,又不题我。今日他老远的教人捎书来,问寻的亲事怎样了。又寄了十两折礼银子贺我。明日差人就来讨回书,你教我怎样回答他?教他就怪死了!叫了媒人,你分咐他,好歹上紧替他寻着,不拘大小人家,只要好女儿,或十五六、十七八的也罢,该多少财礼,我这里与他。再不,把李大姐房里绣春,倒好模样儿,与他去罢。”月娘道:“我说你是个火燎腿行货子!这两三个月,你早做什么来?人家央你一场,替他看个真正女子去也好。那丫头你又收过他,怎好打发去的!你替他当个事干,他到明日也替你用的力。如今急水发,怎么下得浆?比不得买什么儿,拿了银子到市上就买的来了。一个人家闺门女子,好歹不同,也等着媒人慢慢踏看将来。你倒说的好自在话儿!”西门庆道:“明日他来要回书,怎么回答他?”月娘道:“亏你还断事!这些勾当儿,便不会打发人?等那人明日来,你多与他些盘缠,写书回复他,只说女子寻下了,只是衣服妆奁未办,还待几时完毕,这里差人送去。打发去了,你这里教人替他寻也不迟。此一举两得其便,才干出好事来,也是人家托你一场。”西门庆笑道:“说的有理!”一面叫将陈敬济来,隔夜修了回书。

次日,下书人来到,西门庆亲自出来,问了备细。又问蔡状元几时船到,好预备接他。那人道:“小人来时蔡老爹才辞朝,京中起身。翟爹说:只怕蔡老爹回乡,一时缺少盘缠,烦老爹这里多少只顾借与他。写书去,翟老爹那里如数补还。”西门庆道:“你多上复翟爹,随他要多少,我这里无不奉命。”说毕,命陈敬济让去厢房内管待酒饭。临去交割回书,又与了他五两路费。那人拜谢,欢喜出门,长行去了。看官听说:当初安忱取中头甲,被言官论他是先朝宰相安惇之弟,系党人子孙,不可以魁多士。徽宗不得已,把蔡蕴擢为第一,做了状元。投在蔡京门下,做了假子。升秘书省正事,给假省亲。且说月娘家中使小厮叫了老冯、薛嫂儿并别的媒人来,分咐各处打听人家有好女子,拿帖儿来说,不在话下。

一日,西门庆使来保往新河口,打听蔡状元船只,原来就和同榜进士安忱同船。这安进士亦因家贫未续亲,东也不成,西也不就,辞朝还家续亲,因此二人同船来到新河口。来保拿着西门庆拜帖来到船上见,就送了一分下程,酒面、鸡鹅、下饭、盐酱之类。蔡状元在东京,翟谦已预先和他说了:“清河县有老爷门下一个西门千户,乃是大巨家,富而好礼。亦是老爷抬举,见做理刑官。你到那里,他必然厚待。”这蔡状元牢记在心,见面门庆差人远来迎接,又馈送如此大礼,心中甚喜。次日就同安进士进城来拜。西门庆已是预备下酒席。因在李知县衙内吃酒,看见有一起苏州戏子唱的好,旋叫了四个来答应。蔡状元那日封了一端绢帕、一部书、一双云履。安进士亦是书帕二事、四袋芽茶、四柄杭扇。各具宫袍乌纱,先投拜帖进去。西门庆冠冕迎接至厅上,叙礼交拜。献毕贽仪,然后分宾主而坐。先是蔡状元举手欠身说道:“京师翟云峰,甚是称道贤公阀阅名家,清河巨族。久仰德望,未能识荆,今得晋拜堂下,为幸多矣!”西门庆答道:“不敢!昨日云峰书来,具道二位老先生华辀下临,理当迎接,奈公事所羁,望乞宽恕。”因问:“二位老先生仙乡、尊号?”蔡状元道:“学生本贯滁州之匡庐人也。贱号一泉,侥幸状元,官拜秘书正字,给假省亲。”安进士道:“学生乃浙江钱塘县人氏。贱号凤山。见除工部观政,亦给假还乡续亲。敢问贤公尊号?”西门庆道:“在下卑官武职,何得号称。”询之再三,方言:“贱号四泉,累蒙蔡老爷抬举,云峰扶持,袭锦衣千户之职。见任理刑,实为不称。”蔡状元道:“贤公抱负不凡,雅望素著,休得自谦。”叙毕礼话,请去花园卷棚内宽衣。蔡状元辞道:“学生归心匆匆,行舟在岸,就要回去。既见尊颜,又不遽舍,奈何奈何!”西门庆道:“蒙二公不弃蜗居,伏乞暂住文旆,少留一饭,以尽芹献之情。”蔡状元道:“既是雅情,学生领命。”一面脱去衣服,二人坐下。左右又换了一道茶上来。蔡状元以目瞻顾因池台馆,花木深秀,一望无际,心中大喜,极口称羡道:“诚乃蓬瀛也!”于是抬过棋桌来下棋。西门庆道:“今日有两个戏子在此伺候,以供宴赏。”安进士道:“在那里?何不令来一见?”不一时,四个戏子跪下磕头。蔡状元问道:“那两个是生旦?叫甚名字?”内中一个答道:“小的妆生,叫苟子孝。那一个装旦的叫周顺。一个贴旦叫袁琰。那一个装小生的叫胡慥。”安进士问:“你们是那里子弟?”苟子孝道:“小的都是苏州人。”安进士道:“你等先妆扮了来,唱个我们听。”四个戏子下边妆扮去了。西门庆令后边取女衣钗梳与他,教书童也妆扮起来。共三个旦、两个生,在席上先唱《香囊记》。大厅正面设两席,蔡状元、安进士居上,西门庆下边主位相陪。饮酒中间,唱了一折下来,安进士看见书童儿装小旦,便道:“这个戏子是那里的?”西门庆道:“此是小价书童。”安进士叫上去,赏他酒吃,说道:“此子绝妙而无以加矣!”蔡状元又叫别的生旦过来,亦赏酒与他吃。因分咐:“你唱个《朝元歌》‘花边柳边’。”苟子孝答应,在旁拍手道:

\[
花边柳边,檐外晴丝卷。山前水前,马上东风软。自叹行踪,有如蓬转,盼望家乡留恋。雁杳鱼沉,离愁满怀谁与传?日短北堂萱,空劳魂梦牵。洛阳遥远,几时得上九重金殿?
\]
唱完了,安进士问书童道:“你们可记的《玉环记》‘恩德浩无边’?”书童答道:“此是《画眉序》,小的记得。”随唱道:

\[
恩德浩无边,父母重逢感非浅。幸终身托与,又与姻缘。风云会异日飞腾,鸾凤配今谐缱绻。料应夫妇非今世,前生种玉蓝田。
\]

原来安进士杭州人,喜尚男风,见书童儿唱的好,拉着他手儿,两个一递一口吃酒。良久,酒阑上来,西门庆陪他复游花园,向卷棚内下棋。令小厮拿两个桌盒,三十样都是细巧果菜、鲜物下酒。蔡状元道:“学生们初会,不当深扰潭府,天色晚了,告辞罢。”西门庆道:“岂有此理。”因问:“二公此回去,还到船上?”蔡状元道:“暂借门外永福寺寄居。”西门庆道:“如今就门外去也晚了。不如老先生把手下从者止留一二人答应,其余都分咐回去,明日来接,庶可两尽其情。”蔡状元道:“贤公虽是爱客之意,其如过扰何!”当下二人一面分咐手下,都回门外寺里歇去,明日早拿马来接。众人应诺去了,不在话下。

二人在卷棚内下了两盘棋,子弟唱了两折,恐天晚,西门庆与了赏钱,打发去了。止是书童一人,席前递酒伏侍。看看吃至掌灯,二人出来更衣,蔡状元拉西门庆说话:“学生此去回乡省亲,路费缺少。”西门庆道:“不劳老先生分咐。云峰尊命,一定谨领。”良久,让二人到花园:“还有一处小亭请看。”把二人一引,转过粉墙,来到藏春坞雪洞内。里面暖腾腾掌着灯烛,小琴桌上早已陈设果酌之类,床榻依然,琴书潇洒。从新复饮,书童在旁歌唱。蔡状元问道:“大官,你会唱‘红入仙桃’?”书童道:“此是《锦堂月》,小的记得。”于是把酒都斟,拿住南腔,拍手唱了一个。安进士听了,喜之下胜,向西门庆道:“此子可爱。”将杯中之酒一吸而饮之。那书童在席间穿着翠袖红裙,勒着销金箍儿,高擎玉斝,捧上酒,又唱了一个。当日直饮至夜分,方才歇息。西门庆藏春坞、翡翠轩两处俱设床帐,铺陈绩锦被褥,就派书童、玳安两个小厮答应。西门庆道了安置,方回后边去了。

到次日,蔡状元、安进士跟从人夫轿马来接。西门庆厅上摆酒伺候,馔饮下饭与脚下人吃。教两个小厮,方盒捧出礼物。蔡状元是金缎一端,领绢二端,合香五百,白金一百两。安进士是色缎一端,领绢一端,合香三百,白金三十两。蔡状元固辞再三,说道:“但假十数金足矣,何劳如此太多,又蒙厚腆!”安进士道:“蔡年兄领受,学生不当。”西门庆笑道:“些须微赆,表情而已。老先生荣归续亲,在下少助一茶之需。”于是两人俱出席谢道:“此情此德,何日忘之!”一面令家人各收下去,一面与西门庆相别,说道:“生辈此去,暂违台教。不日旋京,倘得寸进,自当图报。”安进士道:“今日相别,何年再得奉接尊颜?”西门庆道:“学生蜗居屈尊,多有亵慢,幸惟情恕!本当远送,奈官守在身,先此告过。”送二人到门首,看着上马而去。正是:

\[
博得锦衣归故里,功名方信是男儿。
\]

\newpage
%# -*- coding:utf-8 -*-
%%%%%%%%%%%%%%%%%%%%%%%%%%%%%%%%%%%%%%%%%%%%%%%%%%%%%%%%%%%%%%%%%%%%%%%%%%%%%%%%%%%%%


\chapter{冯妈妈说嫁韩爱姐\KG 西门庆包占王六儿}


词曰:

\[
淡妆多态,更的的频回眄睐。便认得琴心先许,与绾合欢双带。记华堂风月逢迎,轻嚬浅笑嫣无奈。向睡鸭炉边,翔鸾屏里,暗把香罗偷解。
\]

话说西门庆打发蔡状元、安进士去了。一日,骑马带眼纱在街上喝道而过,撞见冯妈妈,便叫小厮叫住,到面前问他:“你寻的那女子怎样了?如何也不来回话?”婆子说道:“这几日,虽是看了几个,都是卖肉的挑担儿的,怎好回你老人家话?不想天使其便,眼跟前一个人家女儿,就想不起来。十分人材,属马的,交新年十五岁。若不是昨日打他门首过,他娘请我进去吃茶,我还不得看见他哩。才吊起头儿,戴着云髻儿。好不笔管儿般直缕的身子儿,缠得两只脚儿一些些,搽的浓浓的脸儿,又一点小小嘴儿,鬼精灵儿是的。他娘说,他是五月端午日养的,小名叫做爱姐。休说俺们爱,就是你老人家见了,也爱的不知怎么样的哩!”西门庆道:“你看这风妈妈子,我平白要他做甚么?家里放着好少儿。实对你说了罢,此是东京蔡太师老爷府里大管家翟爹,要做二房,图生长,托我替他寻。你若与他成了,管情不亏你。”因问道:“是谁家女子?问他讨个庚帖儿来我瞧。”冯妈妈道:“谁家的?我教你老人家知道了罢,远不一千,近只在一砖。不是别人,是你家开绒线韩伙计的女孩儿。你老人家要相看,等我和他老子说,讨了帖儿来,约会下个日子,你只顾去就是了,”西门庆分咐道:“既如此这般,就和他说,他若肯了,讨了帖儿,来宅内回我话。”那婆子应诺去了。

过两日,西门庆正在前厅坐的,忽见冯妈妈来回话,拿了帖儿与西门庆瞧,上写着“韩氏,女命,年十五岁,五月初五日子时生”。便道:“我把你老人家的话对他老子说了,他说:‘既是大爹可怜见,孩儿也是有造化的。但只是家寒,没些备办。’”西门庆道:“你对他说:不费他一丝儿东西,凡一应衣服首饰、妆奁箱柜等件,都是我这里替他办备,还与他二十两财礼。教他家止办女孩儿的鞋脚就是了。临期,还教他老子送他往东京去。比不的与他做房里人,翟管家要图他生长,做娘子。难得他女儿生下一男半女,也不愁个大富贵。”冯妈妈道:“他那里请问,你老人家几时过去相看,好预备。”西门庆道:“既是他应允了,我明日就过去看看罢。他那里要的急。就对他说,休要他预备什么,我只吃锺清茶就起身。”冯妈妈道:“爷嚛,你老人家上门儿怪人家,虽不稀罕他的,也略坐坐儿。伙计家莫不空教你老人家来了!”西门庆道:“你就不是了。你不知我有事。”冯妈妈道:“既是恁的,等我和他说。”一面先到韩道国家,对他浑家王六儿,将西门庆的话一五一十说了一遍:“明日他衙门中散了,就过来相看。教你一些儿休预备,他只吃一锺茶,看了就起身。”王六儿道:“真个?妈妈子休要说谎。”冯妈妈道:“你当家不恁的说,我来哄你不成!他好少事儿,家中人来人去,通不断头的。”妇人听言,安排了酒食与婆子吃了,打发去了,明日早来伺候。到晚,韩道国来家,妇人与他商议已定。早起往高井上叫了一担甜水,买了些好细果仁,放在家中,还往铺子里做买卖去了。丢下老婆在家,艳妆浓抹,打扮的乔模乔样,洗手剔甲,揩抹杯盏干净,剥下果仁,顿下好茶等候,冯妈妈先来撺掇。

西门庆衙门中散了,到家换了便衣靖巾,骑马带眼纱,玳安、琴童两个跟随,迳来韩道国家,下马进去。冯妈妈连忙请入里面坐了,良久,王六儿引着女儿爱姐出来拜见。这西门庆且不看他女儿,不转晴只看妇人。见他上穿着紫绫袄儿玄色缎金比甲,玉色裙子下边显着趫趫的两只脚儿。生的长挑身材,紫膛色瓜子脸,描的水髩长长的。正是:未知就里何如,先看他妆色油样。但见:

\[
淹淹润润,不搽脂粉,自然体态妖烧;袅袅娉娉,懒染铅华,生定精神秀丽。两弯眉画远山,一对眼如秋水。檀口轻开,勾引得蜂狂蝶乱;纤腰拘束,暗带着月意风情。若非偷期崔氏女,定然闻瑟卓文君。
\]
西门庆见了,心摇目荡,不能定止,口中不说,心中暗道:“原来韩道国有这一个妇人在家,怪不的前日那些人鬼混他。”又见他女孩儿生的一表人物,暗道:“他娘母儿生的这般人物,女儿有个不好的?”妇人先拜见了,教他女儿爱姐转过来,望上向西门庆花枝招飐也磕了四个头,起来侍立在旁。老妈连忙拿茶出来,妇人用手抹去盏上水渍,令他递上。西门庆把眼上下观看这个女子:乌云叠鬓、粉黛盈腮,意态幽花秀丽,肌肤嫩玉生香。便令玳安毡包内取出锦帕二方、金戒指四个、白银二十两,教老妈安放在茶盘内。他娘忙将戒指带在女儿手上,朝上拜谢,回房去了。西门庆对妇人说:“迟两日,接你女孩儿往宅里去,与他裁衣服。这些银子,你家中替他做些鞋脚儿。”妇人连忙又磕下头去,谢道:“俺们头顶脚踏都是大爹的,孩子的事又教大爹费心,俺两口儿就杀身也难报大爹。又多谢爹的插带厚礼。”西门庆问道:“韩伙计不在家了?”妇人道:“他早晨说了话,就往铺子里走了。明日教他往宅里与爹磕头去。”西门庆见妇人说话乖觉,一口一声只是爹长爹短,就把心来惑动了,临出门上覆他:“我去罢。”妇人道:“再坐坐。”西门庆道:“不坐了。”于是出门。一直来家,把上项告吴月娘说了。月娘道:“也是千里姻缘着线牵。既是韩伙计这女孩儿好,也是俺们费心一场。”西门庆道:“明日接他来住两日儿,好与他裁衣服。我如今先拿十两银子,替他打半副头面簪环之类。”月娘道:“及紧儹做去,正好后日教他老子送去,咱这里不着人去罢了。”西门庆道,“把铺子关两日也罢,还着来保同去,就府内问声,前日差去节级送蔡驸马的礼到也不曾?”

话休饶舌。过了两日,西门庆果然使小厮接韩家女儿。他娘王氏买了礼,亲送他来,进门与月娘大小众人磕头拜见,说道:“蒙大爹、大娘并众娘每抬举孩儿,这等费心,俺两口儿知感不尽。”先在月娘房摆茶,然后明间内管待。李娇儿、孟玉楼、潘金莲、李瓶儿都陪坐。西门庆与他买了两匹红绿潞绸、两匹绵绸,和他做里衣儿。又叫了赵裁来,替他做两套织金纱缎衣服,一件大红妆花缎子袍儿。他娘王六儿安抚了女儿,晚夕回家去了。西门庆又替他买了半副嫁妆,描金箱笼、鉴妆、镜架、盒罐、铜锡盆、净桶、火架等件。非止一日,都治办完备。写了一封书信,择定九月初十日起身。西门庆问县里讨了四名快手,又拨了两名排军,执袋弓箭随身。来保、韩道国雇了四乘头口,紧紧保定车辆暖轿,送上东京去了,不题。丢的王六儿在家,前出后空,整哭了两三日。

一日,西门庆无事,骑马来狮子街房里观看。冯妈妈来递茶,西门庆与了一两银子,说道:“前日韩伙什孩子的事累你,这一两银子,你买布穿。”婆子连忙磕头谢了。西门庆又问:“你这两日,没到他那边走走?”冯妈妈道:“老身那一日没到他那里做伴儿坐?他自从女儿去了,他家里没人,他娘母靠惯了他,整哭了两三日,这两日才缓下些儿来了。他又说孩子事多累了爹,问我:‘爹曾与你些辛苦钱儿没有?’我便说:‘他老人家事忙,我连日也没曾去,随他老人家多少与我些儿,我敢争?’他也许我等他官儿回来,重重谢我哩!”西门庆道:“他老子回来一定有些东西,少不得谢你。”说了一回话,见左右无人,悄俏在婆子耳边如此这般:“你闲了到他那里,取巧儿和他说,就说我上覆他,闲中我要到他那里坐半日,看他肯也不肯。我明日还来讨回话。”那婆子掩口冷冷笑道:“你老人家坐家的女儿偷皮匠——逢着的就上。一锹撅了个银娃娃,还要寻他的娘母儿哩!夜晚些,等老身慢慢皮着脸对他说。爹,你还不知这妇人,他是咱后街宰牲口王屠的妹子,排行叫六姐,属蛇的,二十九岁了,虽是打扮的乔样,到没见他输身。你老人家明日来,等我问他,讨个话儿回你。”西门庆道:“是了。”说毕,骑马来家。

婆子做饭吃了,锁了房门,慢慢来到妇人家。妇人开门,便让进房里坐,道:“我昨日下了些面,等你来吃,就不来了。”婆子道:“我可要来哩,到人家就有许多事,挂住了腿,动不得身。”妇人造:“刚才做的热饭,炒面筋儿,你吃些。”婆子道:“老身才吃的饭来,呷些茶罢,”那妇人便浓浓点了一盏茶递与他,看着妇人吃了饭,妇人道:“你看我恁苦!有我那冤家,靠定了他。自从他去了,弄的这屋里空落落的,件件的都看了我。弄的我鼻儿乌,嘴儿黑,相个人模样?到不如他死了,扯断肠子罢了。似这般远离家乡去了,你教我这心怎么放的下来?急切要见他见,也不能勾。”说着,眼酸酸的哭了。婆子道:“说不得,自古养儿人家热腾腾,养女人家冷清清,就是长一百岁,少不得也是人家的。你如今这等抱怨,到明日,你家姐姐到府里脚硬,生下一男半女,你两口子受用,就不说我老身了。”妇人道:“大人家的营生,三层大,两层小,知道怎样的?等他长进了,我们不知在那里晒牙渣骨去了。”婆子道:“怎的恁般说!你们姐姐,比那个不聪明伶俐,愁针指女工不会?各人裙带衣食,你替他愁!”两个一递一句说勾良久,看看说得入港,婆子道:“我每说个傻话儿,你家官人不在,前后恁空落落的,你晚夕一个人儿,不言怕么?”妇人道:“你还说哩,都是你弄得我,肯晚夕来和我做做伴儿?”婆子道:“只怕我一时来不成,我举保个人儿来与你做伴儿,肯不肯?”妇人问:“是谁?”婆子掩口笑道:“一客不烦二主,宅里大老爹昨日到那边房子里,如此这般对我说,见孩子去了,丢的你冷落,他要来和你坐半日儿,你怎么说?这里无人,你若与他凹上了,愁没吃的、穿的、使的、用的!走熟了时,到明日房子也替你寻得一所,强如在这僻格剌子里。”妇人听了微笑说道:“他宅里神道相似的几房娘子,他肯要俺这丑货儿?”婆子道:“你怎的这般说?自古道情人眼内出西施,一来也是你缘法凑巧,他好闲人儿,不留心在你时,他昨日巴巴的肯到我房子里说?又与了一两银子,说前日孩子的事累我。落后没人在跟前,就和我说,教我来对你说。你若肯时,他还等我回话去。典田卖地,你两家愿意,我莫非说谎不成!”妇人道:“既是下顾,明日请他过来,奴这里等候。”这婆子见他吐了口儿,坐了一回去了。

到次日,西门庆来到,一五一十把妇人话告诉一遍。西门庆不胜欢喜,忙称了一两银子与冯妈妈,拿去治办酒菜。那妇人听见西门庆来,收拾房中干净,熏香设帐,预备下好茶好水。不一时,婆子拿篮子买了许多嘎饭菜蔬果品,来厨下替他安排。妇人洗手剔甲,又烙了一箸面饼。明间内,揩抹桌椅光鲜。

西门庆约下午时分,便衣小帽,带着眼纱,玳安、棋童两个小厮跟随,迳到门首,下马进去。分咐把马回到狮子街房子里去,晚上来接,止留玳安一人答应。西门庆到明间内坐下。良久,妇人扮的齐齐整整,出来拜见,说道:“前日孩子累爹费心,一言难尽。”西门庆道:“一时不到处,你两口儿休抱怨。”妇人道:“一家儿莫大之恩,岂有抱怨之理。”磕了四个头。冯妈妈拿上茶来,妇人选了茶。见马回去了,玳安把大门关了。妇人陪坐一回,让进房里坐。正面纸窗门儿厢的炕床,挂着四扇各样颜色绫剪帖的张生遇莺莺蜂花香的吊屏儿,上桌鉴妆、镜架、盒罐、锡器家活堆满,地下插着棒儿香。上面设着一张东坡椅儿。西门庆坐下。妇人又浓浓点一盏胡桃夹盐笋泡茶递上去,西门庆吃了。妇人接了盏,在下边炕沿儿上陪坐,问了回家中长短。西门庆见妇人自己拿托盘儿,说道:“你这里还要个孩子使才好。”妇人道:“不瞒爹说,自从俺女儿去了,凡事不方便。少不的奴自己动手。”西门庆道:“这个不打紧,明日教老冯替你看个十三四岁的丫头子,且胡乱替替手脚。”妇人道:“也得俺家的来,少不得东軿西辏的,央冯妈妈寻一个孩子使。”西门庆道:“也不消,该多少银子,等我与他。”那妇人道:“怎好又烦费你老人家,自恁累你老人家还少哩!”西门庆见他会说话,心中甚喜。一面冯妈妈进来安放桌儿,西门庆就对他说寻使女一节。冯妈妈道:“爹既是许了你,拜谢拜谢儿。南首赵嫂儿有个十三岁的孩子,只要四两银子,教爹替你买下罢。”妇人连忙向前道了万福。不一时,摆下案碟菜蔬,筛上酒来。妇人满斟一盏,双手递与西门庆。才待磕下头去,西门庆连忙用手拉起,说:“头里已是见过,不消又下礼了,只拜拜便了。”妇人笑吟吟道了万福,旁边一个小杌儿上坐下。厨下老妈将嘎饭菜果,一一送上。又是两箸软饼,妇人用手拣肉丝细菜儿裹卷了,用小蝶儿托了,递与西门庆吃。两个在房中,杯来盏去,做一处饮酒。玳安在厨房里,老冯陪他另有坐处,打发他吃,不在话下。

彼此饮勾数巡,妇人把座儿挪近西门庆跟前,与他做一处说话,递酒儿。然后西门庆与妇人一递一口儿吃酒,见无人进来,搂过脖子来亲嘴咂舌。妇人便舒手下边,笼攥西门庆玉茎。彼此淫心荡漾,把酒停住不吃了。掩上房门,褪去衣裤。妇人就在里边炕床上伸开被褥。那时已是日色平西时分。西门庆乘着酒兴,顺袋内取出银托子来使上。妇人用手打弄,见奢棱跳脑,紫强光鲜,沉甸甸甚是粗大。一壁坐在西门庆怀里,一面在上,两个且搂着脖子亲嘴。妇人乃跷起一足,以手导那话入牝中,两个挺一回。西门庆摸见妇人肌肤柔腻,牝毛疏秀,先令妇人仰卧于床背,把双手提其双足,置之于腰眼间,肆行抽送。怎见得这场云雨?但见:

\[
威风迷翠榻,杀气琐鸳衾。珊瑚枕上施雄,翡翠帐中斗勇。男儿气急,使枪只去扎心窝;女帅心忙,开口要来吞脑袋。一个使双炮的,往来攻打内裆兵;一个轮傍牌的,上下夹迎脐下将。一个金鸡独立,高跷玉腿弄精神;一个枯树盘根,倒入翎花来刺牝。战良久朦胧星眼,但动些儿麻上来;斗多时款摆纤腰,百战百回挨不去。散毛洞主倒上桥,放水去淹军;乌甲将军虚点枪,侧身逃命走。脐膏落马,须臾蹂踏肉为泥;温紧妆呆,顷刻跌翻深涧底。大披挂七零八断,犹如急雨打残花;锦套头力尽筋输,恰似猛风飘败叶。硫黄元帅,盔歪甲散走无门;银甲将军,守住老营还要命。
\]
正是:

\[
愁云托上九重天,一块败兵连地滚。
\]

原来妇人有一件毛病,但凡交媾,只要教汉子干他后庭花,在下边揉着心子绕过。不然随问怎的不得丢身子。就是韩道国与他相合,倒是后边去的多,前边一月走不的两三遭儿。第二件,积年好咂\textuni{23B20}\textuni{23B36},把\textuni{23B20}\textuni{23B36}常远放在口里,一夜他也无个足处。随问怎的出了\textuni{23B3D},禁不的他吮舔挑弄,登时就起。自这两椿儿,可在西门庆心坎上。当日和他缠到起更才回家。妇人和西门庆说:“爹到明日再来早些,白日里咱破工夫,脱了衣裳好生耍耍。”西门庆大喜。到次日,到了狮子街线铺里,就兑了四两银子与冯妈妈,讨了丫头使唤,改名叫做锦儿。

西门庆想着这个甜头儿,过了两日,又骑马来妇人家行走。原是棋童、玳安两个跟随。到了门首,就分咐棋童把马回到狮子街房里去。那冯妈妈专一替他提壶打酒,街上买东西整理,通小殷勤儿,图些油菜养口。西门庆来一遭,与妇人一二两银子盘缠。白日里来,直到起更时分才家去。瞒的家中铁桶相似。冯妈妈每日在妇人这里打勤劳儿,往宅里也去的少了。李瓶儿使小厮叫了他两三遍,只是不得闲,要便锁着门去了一日。

一日,画童儿撞见婆子,叫了来家。李瓶儿说道:“妈妈子成日影儿不见,干的什么猫儿头差事?叫了一遍,只是不在,通不来这里走走儿,忙的恁样儿的!丢下好些衣裳带孩子被褥,等你来帮着丫头们拆洗拆洗,再不见来了。”婆子道:“我的奶奶,你到说得且是好,写字的拿逃兵,我如今一身故事儿哩!卖盐的做雕銮匠,我是那咸人儿?”李瓶儿道:“妈妈子请着你就是不闲,成日赚的钱,不知在那里。”婆子道:“老身大风刮了颊耳去——嘴也赶不上在这里,赚甚么钱?你恼我,可知心里急急的要来,再转不到这里来,我也不知成日干的什么事儿哩。后边大娘从那时与了银子,教我门外头替他捎个拜佛的蒲甸儿来,我只要忘了。昨日甫能想起来,卖蒲甸的贼蛮奴才又去了,我怎的回他?”李瓶儿道:“你还敢说没有他甸儿,你就信信拖拖跟了和尚去了罢了!他与了你银子,这一向还不替他买将来,你这等妆憨打呆的。”婆子道,“等我也对大娘说去,就交与他这银子去。昨日骑骡子,差些儿没吊了他的。”李瓶儿道:“等你吊了他的,你死也。”这妈妈一直来到后边,未曾入月娘房,先走在厨下打探子儿。只见玉萧和来兴儿媳妇坐在一处,见了说道:“老冯来了!贵人,你在那里来?你六娘要把你肉也嚼下来,说影边儿就不来了。”那婆子走到跟前拜了两拜,说道:“我才到他前头来,吃他咭咶了这一回来了。”玉萧道:“娘问你替他捎的蒲甸儿怎样的?”婆子道:“昨日拿银子到门外,卖蒲甸的卖了家去了,直到明年三月里才来哩。银子我还拿在这里,姐你收了罢!”玉萧笑道:“怪妈妈子,你爹还在屋里兑银子,等出去了,你还亲交与他罢。”又道:“你且坐的。我问你,韩伙计送他女儿去了多少时了?也待回来,这一回来,你就造化了,他还谢你谢儿。”婆子道:“谢不谢,随他了。他连今才去了八日,也得尽头才得来家。”不一时,西门庆兑出银子,与贲四拿了庄子上去,就出去了。

婆子走在上房,见了月娘,也没敢拿出银子来,只说蛮子有几个粗甸子,都卖没了,回家明年捎双料好蒲甸来。月娘是诚实的人,说道:“也罢,银子你还收着。到明年,我只问你要两个就是了。”与婆子儿个茶食吃了。后又到李瓶儿房里来,瓶儿因问:“你大娘没骂你?”婆子道:“被我如此支吾,调的他喜欢了,倒与我些茶吃,赏了我两个饼定出来了。”李瓶儿道:“还是昨日他往乔大户家吃满月的饼定。妈妈子,不亏你这片嘴头子,六月里蚊子——也钉死了!”又道:“你今日与我洗衣服,不去罢了。”婆子道:“你收拾讨下浆,我明日早来罢。后晌时分,还要到一个熟主顾人家干些勾当儿。”李瓶儿道:“你这老货,偏有这些胡枝扯叶的。你明日不来,我和你答话!”那婆子说笑了一回,脱身走了。李瓶儿留他:“你吃了饭去。”婆子道:“还饱着哩,不吃罢。”恐怕西门庆往王六儿家去,两步做一步。正是:

\[
媒人婆地里小鬼,两头来回抹油嘴。
一日走勾千千步,只是苦了两只腿。
\]

\newpage
%# -*- coding:utf-8 -*-
%%%%%%%%%%%%%%%%%%%%%%%%%%%%%%%%%%%%%%%%%%%%%%%%%%%%%%%%%%%%%%%%%%%%%%%%%%%%%%%%%%%%%


\chapter{王六儿棒槌打捣鬼\KG 潘金莲雪夜弄琵琶}


词曰:

\[
银筝宛转,促柱调弦,声绕梁间。巧作秦声独自怜。指轻妍,风回雪旋,缓扬清曲,响夺钧天。说甚么别鹤乌啼,试按《罗敷陌上》篇,休按《罗敷陌上》篇。
\]

话说冯婆子走到前厅角门首,看见玳安在厅槅子前,拿着茶盘儿伺候。玳安望着冯妈努嘴儿:“你老人家先往那里去,俺爹和应二爹说了话就起身。已先使棋童儿送酒去了。”那婆子听见,两步做一步走的去了。原来应伯爵来说:“揽头李智、黄四派了年例三万香蜡等料钱粮下来,该一万两银子,也有许多利息。上完了批,就在东平府见关银子,来和你计较,做不做?”西门庆道:“我那里做他!揽头以假充真,买官让官。我衙门里搭了事件,还要动他。我做他怎的!”伯爵道:“哥若不做,叫他另搭别人。你只借二千两银子与他,每月五分行利,叫他关了银子还你,你心下何如?”西门庆道:“既是你的分上,我挪一千银子与他罢。如今我庄子收拾,还没银子哩。”伯爵见西门庆吐了口儿,说道:“哥若十分没银子,看怎么再拨五百两货物儿,凑个千五儿与他罢,他不敢少下你的。”西门庆道:“他少下我的,我有法儿处。又一件,应二哥,银子便与他,只不叫他打着我的旗儿,在外边东诓西骗。我打听出来,只怕我衙门监里放不下他。”伯爵道:“哥说的什么话,典守者不得辞其责。他若在外边打哥的旗儿,常没事罢了,若坏了事,要我做甚么?哥你只顾放心,但有差池,我就来对哥说。说定了,我明日叫他好写文书。”西门庆道:“明日不教他来,我有勾当。叫他后日来。”说毕,伯爵去了。

西门庆叫玳安伺候马,带上眼纱,问棋童去没有。玳安道:“来了,取挽手儿去了。”不一时,取了挽手儿来,打发西门庆上马,迳往牛皮巷来。不想韩道国兄弟韩二捣鬼,耍钱输了,吃的光睁睁儿的,走来哥家,问王六儿讨酒吃。袖子里掏出一条小肠儿来,说道:“嫂,我哥还没来哩,我和你吃壶烧酒。”那妇人恐怕西门庆来,又见老冯在厨下,不去兜揽他,说道:“我是不吃。你要吃拿过一边吃去,我那里耐烦?你哥不在家,招是招非的,又来做什么?”那韩二捣鬼,把眼儿涎睁着,又不去,看见桌底下一坛白泥头酒,贴着红纸帖儿,问道:“嫂子,是那里酒?打开筛壶来俺每吃。耶嚛!你自受用!”妇人道:“你趁早儿休动,是宅里老爹送来的,你哥还没见哩。等他来家,有便倒一瓯子与你吃。”韩二道:“等什么哥?就是皇帝爷的,我也吃一锺儿!”才待搬泥头,被妇人劈手一推,夺过酒来,提到屋里去了。把二捣鬼仰八叉推了一交,半日扒起来,恼羞变成怒,口里喃喃呐呐骂道:“贼淫妇,我好意带将菜儿来,见你独自一个冷落落,和你吃杯酒。你不理我,倒推我一交。我教你不要慌,你另叙上了有钱的汉子,不理我了,要把我打开,故意儿嚣我,讪我,又趍我。休叫我撞见,我叫你这不值钱的淫妇,白刀子进去红刀子出来!”妇人见他的话不妨头,一点红从耳边起,须臾紫胀了双腮,便取棒槌在手,赶着打出来,骂道:“贼饿不死的杀才!你那里噇醉了,来老娘这里撒野火儿。老娘手里饶你不过!”那二捣鬼口里喇喇哩哩骂淫妇,直骂出门去。不想西门庆正骑马来,见了他,问是谁,妇人道:“情知是谁,是韩二那厮,见他哥不在家,要便耍钱输了,吃了酒来殴我。有他哥在家,常时撞见打一顿。”那二捣鬼看见,一溜烟跑了。西门庆又道:“这少死的花子,等我明日到衙门里与他做功德!”妇人道:“又叫爹惹恼。”西门庆道:“你不知,休要惯了他。”妇人道:“爹说的是。自古良善彼人欺,慈悲生患害。”一面让西门庆明间内坐。西门庆吩咐棋童回马家去,叫玳安儿:“你在门首看,但掉着那光棍的影儿,就与我锁在这里,明日带到衙门里来。”玳安道:“他的魂儿听见爹到,不知走的那里去了。”

西门庆坐下。妇人见毕礼,连忙屋里叫丫鬟锦儿拿了一盏果仁茶出来,与西门庆吃,就叫他磕头。西门庆道:“也罢,到好个孩子,你且将就使着罢。”又道:“老冯在这里,怎的不替你拿茶?”妇人道:“冯妈妈他老人家,我央及他厨下使着手哩。西门庆又道:“头里我使小厮送来的那酒,是个内臣送我的竹叶清。里头有许多药味,甚是峻利。我前日见你这里打的酒,都吃不上口,我所以拿的这坛酒来。”妇人又道了万福,说:“多谢爹的酒,正是这般说,俺每不争气,住在这僻巷子里,又没个好酒店,那里得上样的酒来吃,只往大街上取去。”西门庆道:“等韩伙计来家,你和他计较,等着狮子街那里,替你破几两银子买所房子,等你两口子亦发搬到那里住去罢。铺子里又近,买东西诸事方便。”妇人道:“爹说的是。看你老人家怎的可怜见,离了这块儿也好。就是你老人家行走,也免了许多小人口嘴——咱行的正,也不怕他。爹心里要处自情处,他在家和不在家一个样儿,也少不的打这条路儿来。”说一回,房里放下桌儿,请西门庆进去宽了衣服坐。

须臾,安排酒菜上来,妇人陪定,把酒来斟。不一时,两个并肩叠股而饮。吃的酒浓时,两个脱剥上床交欢,自在玩耍。妇人早已床炕上铺的厚厚的被褥,被里熏的喷鼻香。西门庆见妇人好风月,一径要打动他。家中袖了一个锦包儿来,打开,里面银托子、相思套、硫黄圈、药煮的白绫带子、悬玉环、封脐膏、勉铃,一弄儿淫器。那妇人仰卧枕上,玉腿高跷,囗舌内吐。西门庆先把勉铃教妇人自放牝内,然后将银托束其根,硫黄圈套其首,脐膏贴于脐上。妇人以手导入牝中,两相迎凑,渐入大半。妇人呼道:“达达!我只怕你墩的腿酸,拿过枕头来,你垫着坐,我淫妇自家动罢。”又道:“只怕你不自在,你把淫妇腿吊着\textuni{34B2},你看好不好?”西门庆真个把他脚带解下一条来,拴他一足,吊在床槅子上低着拽,拽的妇人牝中之津如蜗之吐蜒,绵绵不绝,又拽出好些白浆子来。西门庆问道:“你如何流这些白?”才待要抹去,妇人道:“你休抹,等我吮咂了罢。”于是蹲跪在他面前吮吞数次,呜咂有声。咂的西门庆淫心辄起,吊过身子,两个干后庭花。龟头上有硫黄圈,濡研难涩。妇人蹙眉隐忍,半晌仅没其棱。西门庆颇作抽送,而妇人用手摸之,渐入大半,把屁股坐在西门庆怀里,回首流眸,作颤声叫:“达达!慢着些,后越发粗大,教淫妇怎生挨忍。”西门庆且扶起股,观其出入之势,因叫妇人小名:“王六儿,我的儿,你达不知心里怎的只好这一桩儿,不想今日遇你,正可我之意。我和你明日生死难开。”妇人道:“达达,只怕后来耍的絮烦了,把奴不理怎了?”西门庆道:“相交下来,才见我不是这样人。”说话之间,两个干勾一顿饭时。西门庆令妇人没高低淫声浪语叫着才过。妇人在下,一面用手举股承受其精,乐极情浓,一泄如注。已而抽出那话来,带着圈子,妇人还替他吮咂净了,两个方才并头交股而卧。正是:一般滋味美,好耍后庭花。有词为证:

\[
美冤家,一心爱折后庭花。寻常只在门前里走,又被开路先锋把住了他。放在户中难禁受。转丝缰勒回马,亲得胜弄的我身上麻,蹴损了奴的粉脸那丹霞。
\]

西门庆与妇人搂抱到二鼓时分,小厮马来接,方才起身回家。到次日,到衙门里差了两个缉捕,把二捣鬼拿到提刑院,只当做掏摸土贼,不由分说,一夹二十,打的顺腿流血。睡了一个月,险不把命花了。往后吓的影也再不敢上妇人门缠搅了。正是:

\[
恨小非君子,无毒不丈夫。
\]

迟了几日,来保、韩道国一行人东京回来,备将前事对西门庆说:“翟管家见了女子,甚是欢喜,说爹费心。留俺府里住了两日,讨了回书。送了爹一匹青马,封了韩伙计女儿五十两银子礼钱,又与了小的二十两盘缠。”西门庆道:“勾了。”看了回书,书中无非是知感不尽之意。自此两家都下眷生名字,称呼亲家,不在话下。韩道国与西门庆磕头拜谢回家。西门庆道:“韩伙计,你还把你女儿这礼钱收去,也是你两口儿恩养孩儿一场。”韩道国再三不肯收,说道:“蒙老爹厚恩,礼钱是前日有了。这银子小人怎好又受得?从前累的老爹好少哩!”西门庆道:“你不依,我就恼了。你将回家,不要花了,我有个处。”那韩道国就磕头谢了,拜辞回去。

老婆见他汉子来家,满心欢喜,一面接了行李,与他拂了尘上,问他长短:“孩子到那里好么?”这道国把往回一路的话,告诉一遍,说:“好人家,孩子到那里,就与了三间房,两个丫鬟伏侍,衣服头面不消说。第二日,就领了后边见了太太。翟管家甚是欢喜,留俺们住了两日,酒饭连下人都吃不了。又与了五十两礼钱。我再三推辞,大官人又不肯,还叫我拿回来了。”因把银子与妇人收了。妇人一块石头方落地,因和韩道国说:“咱到明日,还得一两银子谢老冯。你不在,亏他常来做作伴儿。大官人那里,也与了他一两。”正说着,只见丫头过来递茶。韩道国道:“这个是那里大姐?”妇人道:“这个是咱新买的丫头,名唤锦儿。过来与你爹磕头!”磕了头,丫头往厨下去了。

老婆如此这般,把西门庆勾搭之事,告诉一遍,“自从你去了,来行走了三四遭,才使四两银子买了这个丫头。但来一遭,带一二两银子来。第二的不知高低,气不愤走来这里放水。被他撞见了,拿到衙门里,打了个臭死,至今再不敢来了。大官人见不方便,许了要替我每大街上买一所房子,叫咱搬到那里住去。”韩道国道:“嗔道他头里不受这银子,教我拿回来休要花了,原来就是这些话了。”妇人道:“这不是有了五十两银子,他到明日,一定与咱多添几两银子,看所好房儿。也是我输了身一场,且落他些好供给穿戴。”韩道国道:“等我明日往铺子里去了,他若来时,你只推我不知道,休要怠慢了他,凡事奉承他些儿。如今好容易赚钱,怎么赶的这个道路!”老婆笑道:“贼强人,倒路死的!你到会吃自在饭儿,你还不知老娘怎样受苦哩!”两个又笑了一回,打发他吃了晚饭,夫妻收拾歇下。到天明,韩道国宅里讨了钥匙,开铺子去了,与了老冯一两银子谢他。俱不必细说。

一日,西门庆同夏提刑衙门回来。夏提刑见西门庆骑着一匹高头点子青马,问道:“长官那匹白马怎的不骑,又换了这匹马?到好一匹马,不知口里如何?”西门庆道:“那马在家歇他两日儿。这马是昨日东京翟云峰亲家送来的,是西夏刘参将送他的。口里才四个牙儿,脚程紧慢都有他的。只是有些毛病儿,快护糟踅蹬。初时骑了路上走,把膘跌了许多,这两日内吃的好些儿。”夏提刑道:“这马甚是会行,但只好骑着躧街道儿罢了,不可走远了他。论起在咱这里,也值七八十两银子。我学生骑的那马,昨日又瘸了。今早来衙门里来,旋拿帖儿问舍亲借了这匹马骑来,甚是不方便。”西门庆道:“不打紧,长官没马,我家中还有一匹黄马,送与长官罢。”夏提刑举手道:“长官下顾,学生奉价过来。”西门庆道:“不须计较。学生到家,就差人送来。”两个走到西街口上,西门庆举手分路来家。到家就使玳安把马送去。夏提刑见了大喜,赏了玳安一两银子,与了回帖儿,说:“多上覆,明日到衙门里面谢。”

过了两月,乃是十月中旬时分。夏提刑家中做了些菊花酒,叫了两名小优儿,请西门庆一叙,以酬送马之情。西门庆家中吃了午饭,理了些事务,往夏提刑家饮酒。原来夏提刑备办一席齐整酒肴,只为西门庆一人而设。见了他来,不胜欢喜,降阶迎接,至厅上叙礼。西门庆道:“如何长官这等费心?”夏提刑道:“今年寒家做了些菊花酒,闲中屈执事一叙,再不敢请他客。”于是见毕礼数,宽去衣服,分宾主而坐。茶罢着棋,就席饮酒叙谈,两个小优儿在旁弹唱。正是得多少:

\[
金尊进酒浮香蚁,象板催筝唱鹧鸪。
\]

不说西门庆在夏提刑家饮酒,单表潘金莲见西门庆许多时不进他房里来,每日翡翠衾寒,芙蓉帐冷。那一日把角门儿着,在房内银灯高点,靠定帏屏,弹弄琵琶。等到二三更,使春梅连瞧数次,不见动静。正是:银筝夜久殷勤弄,寂寞空房不忍弹。取过琵琶,横在膝上,低低弹了个《二犯江儿水》唱道:

\[
闷把帏屏来靠,和衣强睡倒。
\]
猛听得房檐上铁马儿一片声响,只道西门庆敲的门环儿响,连忙使春梅去瞧。春梅回道:“娘,错了,是外边风起,落雪了。”妇人又弹唱道:

\[
听风声嘹亮,雪洒窗寮,任冰花片片飘。
\]
一回儿灯昏香尽,心里欲待去剔,见西门庆不来,又意儿懒的动弹了。唱道:

\[
懒把宝灯挑,慵将香篆烧。捱过今宵,怕到明朝。细寻思,这烦恼何日是了?想起来,今夜里心儿内焦,误了我青春年少!你撇的人,有上稍来没下稍。
\]

且说西门庆约一更时分,从夏提刑家吃了酒归来。一路天气阴晦,空中半雨半雪下来,落在衣服上都化了。不免打马来家,小厮打着灯笼,就不到后边,迳往李瓶儿房来。李瓶儿迎着,一面替他拂去身上雪霰,接了衣服。止穿绫敞衣,坐在床上,就问:“哥儿睡了不曾?”李瓶儿道:“小官儿顽了这回,方睡下了。”迎春拿茶来吃了。李瓶儿问,“今夜吃酒来的早?”西门庆道:“夏龙溪因我前日送了他那匹马,今日为我费心,治了一席酒请我,又叫了两个小优儿。和他坐了这一回,见天气下雪,来家早些。”李瓶儿道:“你吃酒,叫丫头筛酒来你吃。大雪里来家,只怕冷哩。”西门庆道:“还有那葡萄酒,你筛来我吃。今日他家吃的是造的菊花酒,我嫌他香淆气的,我没大好生吃。”于是迎春放下桌儿,就是几碟嗄饭、细巧果菜之类。李瓶儿拿杌儿在旁边坐下。桌下放着一架小火盆儿。

这里两个吃酒,潘金莲在那边屋里冷清清,独自一个儿坐在床上。怀抱着琵琶,桌上灯昏烛暗。待要睡了,又恐怕西门庆一时来;待要不睡,又是那盹困,又是寒冷。不免除去冠儿,乱挽乌云,把帐儿放下半边来,拥衾而坐,正是:

\[
倦倚绣床愁懒睡,低垂锦帐绣衾空。
早知薄幸轻抛弃,辜负奴家一片心。
\]
又唱道:

\[
懊恨薄情轻弃,离愁闲自恼。
\]
又唤春梅过来:“你去外边再瞧瞧,你爹来了没有?快来回我话。”那春梅走去,良久回来,说道:“娘还认爹没来哩,爹来家不耐烦了,在六娘房里吃酒的不是?”这妇人不听罢了,听了如同心上戳上几把刀子一般,骂了几句负心贼,由不得扑簌簌眼中流下泪来。一迳把那琵琶儿放得高高的,口中又唱道:

\[
心痒痛难搔,愁怀闷自焦。让了甜桃,去寻酸枣。奴将你这定盘星儿错认了。想起来,心儿里焦,误了我青春年少。你撇的人,有上稍来没下稍。
\]

西门庆正吃酒,忽听见弹的琵琶声,便问:“是谁弹琵琶?”迎春答道:“是五娘在那边弹琵琶响。”李瓶儿道:“原来你五娘还没睡哩。绣春,你快去请你五娘来吃酒。你说俺娘请哩。”那绣春去了。李瓶儿忙吩咐迎春:“安下个坐儿,放个锺箸在面前。”良久,绣春走来说:“五娘摘了头,不来哩。”李瓶儿道:“迎春,你再去请五娘去。你说,娘和爹请五娘哩。”不多时,迎春来说:“五娘把角门儿关了,说吹了灯,睡下了。”西门庆道:“休要信那小淫妇儿,等我和你两个拉他去,务要把他拉了来。咱和他下盘棋耍子。”于是和李瓶儿同来打他角门。打了半日,春梅把角门子开了。西门庆拉着李瓶儿进入他房中,只见妇人坐在帐中,琵琶放在旁边。西门庆道:“怪小淫妇儿,怎的两三转请着你不去!”金莲坐在床上,纹丝儿不动,把脸儿沉着,半日说道:“那没时运的人儿,丢在这冷屋里,随我自生自活的,又来瞅采我怎的?没的空费了你这个心,留着别处使。”西门庆道:“怪奴才!八十岁妈妈没牙——有那些唇说的?李大姐那边请你和他下盘棋儿,只顾等你不去了。”李瓶儿道:“姐姐,可不怎的。我那屋里摆下棋子了,咱们闲着下一盘儿,赌杯酒吃。”金莲道:“李大姐,你们自去,我不去。你不知我心里不耐烦,我如今睡也,比不的你们心宽闲散。我这两日只有口游气儿,黄汤淡水谁尝着来?我成日睁着脸儿过日子哩!”西门庆道:“怪奴才,你好好儿的,怎的不好?你若心内不自在,早对我说,我好请太医来看你。”金莲道:“你不信,叫春梅拿过我的镜子来,等我瞧。这两日,瘦的相个人模样哩!”春梅把镜子真个递在妇人手里,灯下观看。正是:

\[
羞对菱花拭粉妆,为郎憔瘦减容光。
闭门不管闲风月,任你梅花自主张。
\]

西门庆拿过镜子也照了照,说道:“我怎么不瘦?”金莲道:“拿甚么比你!你每日碗酒块肉,吃的肥胖胖的,专一只奈何人。”被西门庆不由分说,一屁股挨着他坐在床上,搂过脖子来就亲了个嘴,舒手被里,摸见他还没脱衣裳,两只手齐插在他腰里去,说道:“我的儿,是个瘦了些。”金莲道:“怪行货子,好冷手,冰的人慌!莫不我哄了你不成?我的苦恼,谁人知道,眼泪打肚里流罢了。”乱了一回,西门庆还把他强死强活拉到李瓶儿房内,下了一盘棋,吃了一回酒。临起身,李瓶儿见他这等脸酸,把西门庆撺掇过他这边歇了。正是得多少:

\[
腰瘦故知闲事恼,泪痕只为别情浓。
\]

\newpage
%# -*- coding:utf-8 -*-
%%%%%%%%%%%%%%%%%%%%%%%%%%%%%%%%%%%%%%%%%%%%%%%%%%%%%%%%%%%%%%%%%%%%%%%%%%%%%%%%%%%%%


\chapter{寄法名官哥穿道服\KG 散生日敬济拜冤家}


诗曰:

\[
汉武清斋夜筑坛,自斟明水醮仙官。
殿前玉女移香案,云际金人捧露盘。
绛节几时还入梦?碧桃何处更骖鸾?
茂陵烟雨埋弓剑,石马无声蔓草寒。
\]

话说当日西门庆在潘金莲房中歇了一夜。那妇人恨不的钻入他腹中,在枕畔千般贴恋,万种牢笼,泪揾鲛鮹,语言温顺,实指望买住汉子心。不料西门庆外边又刮剌上了王六儿,替他狮子街石桥东边,使了一百二十两银子,买了一所房屋居住。门面两间,到底四层,一层做客位,一层供养佛像祖先,一层做住房,一层做厨房。自从搬过来,那街坊邻舍知他是西门庆伙计,不敢怠慢,都送茶盒与他,又出人情庆贺。那中等人家称他做韩大哥、韩大嫂。以下者赶着以叔婶称之。西门庆但来他家,韩道国就在铺子里上宿,教老婆陪他自在顽耍。朝来暮往,街坊人家也都知道这件事,惧怕西门庆有钱有势,谁敢惹他!见一月之间,西门庆也来行走三四次,与王六儿打的一似火炭般热。

看看腊月时分,西门庆在家乱着送东京并府县、军卫、本卫衙门中节礼。有玉皇庙吴道官使徒弟送了四盒礼物,并天地疏、新春符、谢灶诰。西门庆正在上房吃饭,玳安儿拿进帖来,上写着:“王皇庙小道吴宗哲顿首拜。”西门庆看了说道:“出家人,又教他费心。”吩咐玳安,叫书童儿封一两银子拿回帖与他。月娘在旁,因话题起道:“一个出家人,你要便年头节尾受他的礼物,到把前日你为李大姐生孩儿许的愿醮,就叫他打了罢。”西门庆道:“早是你题起来,我许下一百二十分醮,我就忘死了。”月娘道:“原来你是个大诌答子货!谁家愿心是忘记的?你便有口无心许下,神明都记着。嗔道孩儿成日恁啾啾唧唧的,想就是这愿心未还压的他。”西门庆道:“既恁说,正月里就把这醮愿,在吴道官庙里还了罢。”月娘道:“昨日李大姐说,这孩子有些病痛儿的,要问那里讨个外名。”西门庆道:“又往那里讨外名?就寄名在吴道官庙里就是了。”因问玳安:“他庙里有谁在这里?”玳安道:“是他第二个徒弟应春跟礼来的。”西门庆一面走出外边来,那应春连忙磕头说道:“家师父多拜上老爹,没什么孝顺,使小徒弟来送这天地疏并些微礼儿,与老爹赏人。”西门庆止还了半礼,说道:“多谢你师父厚礼。”一面让他坐。应春道:“小道怎么敢坐!”西门庆道:“你坐了,我有话和你说。”那道士头戴小帽,身穿青布直裰,谦逊数次,方才把椅儿挪到旁边坐下,问道:“老爹有甚钧语吩咐?”西门庆道:“正月里,我有些醮愿,要烦你师父替我还还儿,就要送小儿寄名,不知你师父闲不闲?”徒弟连忙立起身来说道:“老爹吩咐,随问有甚经事,不敢应承。请问老爹,订在正月几时?”西门庆道:“就订在初九,爷旦日罢。”徒弟道:“此日正是天诞。又《玉匣记》上我请律爷交庆,五福骈臻,修斋建醮甚好。请问老爹多少醮款?”西门庆道:“今岁七月,为生小儿许了一百二十分清醮。”徒弟又问:“那日延请多少道众?”西门庆道:“请十六众罢。”说毕,左右放桌儿待茶。先封十五两经钱,另外又是一两酬答他的节礼,又说:“道众的衬施,你师父不消备办,我这里连阡张香烛一事带去。”喜欢的道士屁滚尿流,临出门谢了又谢,磕了头儿又磕。

到正月初八日,先使玳安儿送了一石白米、一担阡张、十斤官烛、五斤沉檀马牙香、十六匹生眼布做衬施,又送了一对京段、两坛南酒、四只鲜鹅、四只鲜鸡、一对豚蹄、一脚羊肉、十两银子,与官哥儿寄名之礼。西门庆预先发帖儿,请下吴大舅、花大舅、应伯爵、谢希大四位相陪。陈敬济骑头口,先到庙中替西门庆瞻拜。到初九日,西门庆也没往衙门中去,绝早冠带,骑大白马,仆从跟随,前呼后拥,竟出东门往玉皇庙来。远远望见结彩宝幡,过街榜棚。须臾至山门前下马,睁眼观看,果然好座庙宇。但见:

\[
青松郁郁,翠柏森森。金钉朱户,玉桥低影轩官;碧瓦雕檐,绣幕高悬宝槛。七间大殿,中悬敕额金书;两庑长廊,彩画天神帅将。三天门外,离娄与师旷狰狞,左右阶前,自虎与青龙猛勇。八宝殿前,侍立是长生玉女,九龙床上,坐着个不坏金身。金钟撞处,三千世界尽皈依;玉磬鸣时,万象森罗皆拱极。朝天阁上,天风吹下步虚声;演法坛中,夜月常闻仙佩响。自此便为真紫府,更于何处觅蓬莱?
\]
西门庆由正门而入,见头一座流星门上,七尺高朱红牌架,列着两行门对,大书:

\[
黄道天开,祥启九天之阊阖,迓金舆翠盖以延恩;
玄坛日丽,光临万圣之幡幢,诵宝笈瑶章而阐化。
\]
到了宝殿上,悬着二十四字斋题,大书着:“灵宝答天谢地,报国酬恩,九转玉枢,酬盟寄名,吉祥普满斋坛。”两边一联:

\[
先天立极,仰大道之巍巍,庸申至悃;
昊帝尊居,鉴清修之翼翼,上报洪恩。
\]
西门庆进入坛中香案前,旁边一小童捧盆中盥手毕,铺排跪请上香。西门庆行礼叩坛毕,只见吴道官头戴玉环九阳雷巾,身披天青二十八宿大袖鹤氅,腰系丝带,忙下经筵来,与西门庆稽首道:“小道蒙老爹错爱,迭受重礼,使小道却之不恭,受之有愧。就是哥儿寄名,小道礼当叩祝,增延寿命,何以有叨老爹厚赏,诚有愧赧。经衬又且过厚,令小道愈不安。”西门庆道:“厚劳费心辛苦,无物可酬,薄礼表情而已。”叙礼毕,两边道众齐来稽首。一面请去外方丈,三间厂厅名曰松鹤轩,那里待茶。西门庆刚坐下,就令棋童儿:“拿马接你应二爹去。只怕他没马,如何这咱还没来?”玳安道:“有姐夫骑的驴子还在这里。”西门庆道:“也罢,快骑接去。”棋童应诺去了。吴道官诵毕经,下来递茶,陪西门庆坐,叙话:“老爹敬神一点诚心,小道都从四更就起来,到坛讽诵诸品仙经,今日三朝九转玉枢法事,都是整做。又将官哥儿的生日八字,另具一文书,奏名于三宝面前,起名叫做吴应元。永保富贵遐昌。小道这里,又添了二十四分答谢天地,十二分庆赞上帝,二十四分荐亡,共列一百八十分醮款。”西门庆道:“多有费心.”不一时,打动法鼓,请西门庆到坛看文书。西门庆从新换了大红五彩狮补吉服,腰系蒙金犀角带,到坛,有绛衣表白在旁,先宣念斋意:

\[
大宋国山东清河县县牌坊居住,奉道祈恩,酬醮保安,信官西门庆,本命丙寅年七月廿八日子时建生,同妻吴氏,本命戊辰年八月十五日子时建生。
\]
表白道:“还有宝眷,小道未曾添上。”西门庆道:“你只添上个李氏,辛未年正月十五日卯时建生,同男官哥儿,丙申年七月廿三日申时建生罢。”表白文宣过一遍,接念道:

\[
领家眷等,即日投诚,拜干洪造。伏念庆一介微生,三才未品。出入起居,每感龙天之护佑;迭迁寒暑,常蒙神圣以匡扶。职列武班,叨承禁卫,沐恩光之宠渥,享符禄之丰盈。是以修设清醮,共二十四分位,答报天地之洪恩,酬祝皇王之巨泽。又修清醮十二分位,兹逢天诞,庆赞帝真。介五福以遐昌,迓诸天而下迈。庆又于去岁七月二十三日,因为侧室李氏生男官哥儿,要祈坐蓐无虞,临盆有庆。又愿将男官哥儿寄于三宝殿下,赐名吴应元,告许清醮一百二十分位,续箕裘之\textuni{38E7}嗣,保寿命之延长。附荐西门氏门中三代宗亲等魂:祖西门京良,祖妣李氏;先考西门达,妣夏氏;故室人陈氏,及前亡后化,升坠罔知。是以修设清醮十二分位,恩资道力,均证生方。共列仙醮一百八十分位,仰干化单,俯赐勾销。谨以宣和三年正月初九日天诞良辰,特就大慈玉皇殿,仗延官道,修建灵宝,答天谢地,报国酬盟,庆神保安,寄名转经,吉祥普满大斋一昼夜。延三境之司尊,迓万天之帝驾。一门长叨均安,四序公和迪吉。统资道力,介福方来。谨意。
\]
宣毕斋意,铺设下许多文书符命、表白,一一请看,共有一百八九十道,甚是齐整详细。又是官哥儿三宝荫下寄名许多文书、符索、牒札,不暇细览。西门庆见吴道官十分费心,于是向案前炷了香,画了文书,叫左右捧一匹尺头,与吴道官画字。吴道官固辞再三,方令小童收了。然后一个道士向殿角头咕碌碌擂动法鼓,有若春雷相似。合堂道众,一派音乐响起。吴道官身披大红五彩法氅,脚穿朱履,手执牙笏,关发文书,登坛召将。两边鸣起钟来。铺排引西门庆进坛里,向三宝案左右两边上香。西门庆睁眼观看,果然铺设斋坛齐整。但见:

\[
位按五方,坛分八级。上供三请四御,旁分八极九霄,中列山川岳渎,下设幽府冥官。香腾瑞霭,千枝画烛流光;花簇锦筵,百盏银灯散彩。天地亭,高张羽盖;玉帝堂,密布幢幡。金钟撞处,高功蹑步奏虚皇;玉佩鸣时,都讲登坛朝玉帝。绛绡衣,星辰灿烂;美蒙冠,金碧交加。监坛神将狰狞,直日功曹猛勇。青龙隐隐来黄道,白鹤翩翩下紫宸。
\]

西门庆刚绕坛拈香下来,被左右就请到松鹤轩阁儿里,地铺锦毯,炉焚兽炭,那里坐去了。不一时,应伯爵、谢希大来到。唱毕喏,每人封了一星折茶银子,说道:“实告要送些茶儿来,路远。这些微意,权为一茶之需。”西门庆也不接,说道:“奈烦!自恁请你来陪我坐坐,又干这营生做什么?吴亲家这里点茶,我一总都有了。”应伯爵连忙又唱喏,说:“哥,真个?俺每还收了罢。”因望着谢希大说道:“都是你干这营生!我说哥不受,拿出来,倒惹他讪两句好的。”良久,吴大舅、花子由都到了。每人两盒细茶食来点茶,西门庆都令吴道官收了。吃毕茶,一同摆斋,咸食斋馔,点心汤饭,甚是丰洁。西门庆同吃了早斋。原来吴道官叫了个说书的,说西汉评话《鸿门会》。吴道官发了文书,走来陪坐,问:“哥儿今日来不来?”西门庆道,“正是,小顽还小哩,房下恐怕路远唬着他,来不的。到午间,拿他穿的衣服来,三宝面前,摄受过就是一般。”吴道官道:“小道也是这般计较,最好。”西门庆道:“别的倒也罢了,他只是有些小胆儿。家里三四个丫鬟连养娘轮流看视,只是害怕。猫狗都不敢到他跟前。”吴大舅道:“孩儿们好容易养活大——”正说着,只见玳安进来说:“里边桂姨、银姨使了李铭、吴惠送茶来了。”西门庆道:“叫他进来。”李铭、吴惠两个拿着两个盒子跪下,揭开都是顶皮饼、松花饼、白糖万寿糕、玫瑰搽穰卷儿。西门庆俱令吴道官收了,因问李铭:“你每怎得知道?”李铭道:“小的早晨路见陈姑夫骑头口,问来,才知道爹今日在此做好事。归家告诉桂姐、三妈说,旋约了吴银姐,才来了。多上复爹,本当亲来,不好来得,这粗茶儿与爹赏人罢了。”西门庆吩咐:“你两个等着吃斋。”吴道官一面让他二人下去,自有坐处,连手下人都饱食一顿。

话休饶舌。到了午朝,拜表毕,吴道官预备了一张大插桌,又是一坛金华酒,又是哥儿的一顶青缎子绡金道髻,一件玄色纻丝道衣,一件绿云缎小衬衣,一双白绫小袜,一双青潞绸衲脸小履鞋,一根黄绒线绦,一道三宝位下的黄线索,一道子孙娘娘面前紫线索,一付银项圈条脱,刻着“金玉满堂,长命富贵”,一道朱书辟非黄绫符,上书着“太乙司命,桃延合康”八字,就扎在黄线索上,都用方盘盛着,又是四盘羹果,摆在桌上。差小童经袱内包着宛红纸经疏,将三朝做过法事,一一开载节次,请西门庆过了目,方才装入盒担内。共约八抬,送到西门庆家。西门庆甚是欢喜,快使棋童儿家去,叫赏道童两方手帕、一两银子。

且说那日是潘金莲生日,有吴大妗子、潘姥姥、杨姑娘、郁大姐,都在月娘上房坐的。见庙里送了斋来,又是许多羹果插卓礼物,摆了四张桌子,还摆不下,都乱出来观看。金莲便道:“李大姐,你还不快出来看哩!你家儿子师父庙里送礼来了,又有他的小道冠髻,道衣儿。噫,你看,又是小履鞋儿!”孟玉楼走向前,拿起来手中看,说道:“大姐姐,你看道士家也恁精细,这小履鞋,白绫底儿,都是倒扣针儿方胜儿,锁的这云儿又且是好。我说他敢有老婆!不然,怎的扣捺的恁好针脚儿?”吴月娘道:“没的说。他出家人,那里有老婆!想必是雇人做的。”潘金莲接过来说:“道士有老婆,相王师父和大师父会挑的好汗巾儿,莫不是也有汉子?”王姑子道:“道士家,掩上个帽子,那里不去了!似俺这僧家,行动就认出来。”金莲说道:“我听得说,你住的观音寺背后就是玄明观。常言道:男僧寺对着女僧寺,没事也有事。”月娘道:“这六姐,好恁罗说白道的!”金莲道:“这个是他师父与他娘娘寄名的紫线锁。又是这个银脖项符牌儿,上面银打的八个字,带着且是好看。背面坠着他名字,吴什么元?”棋童道:“此是他师父起的法名吴应元。”金莲道:“这是个‘应’字。”叫道:“大姐姐,道士无礼,怎的把孩子改了他的姓?”月娘道:“你看不知礼!”因使李瓶儿:“你去抱了你儿子来,穿上这道衣,俺每瞧瞧好不好?”李瓶儿道:“他才睡下,又抱他出来?”金莲道:“不妨事,你揉醒他。”那李瓶儿真个去了。

这潘金莲识字,取过红纸袋儿,扯出送来的经疏,看见上面西门庆底下同室人吴氏,旁边只有李氏,再没别人,心中就有几分不忿,拿与众人瞧:“你说贼三等儿九格的强人!你说他偏心不偏心?这上头只写着生孩子的,把俺每都是不在数的,都打到赘字号里去了。”孟玉楼问道:“可有大姐姐没有?”金莲道:“没有大姐姐倒好笑。”月娘道:“也罢了,有了一个,也就是一般。莫不你家有一队伍人,也都写上,惹的道士不笑话么?”金莲道:“俺每都是刘湛儿鬼儿么?比那个不出材的,那个不是十个月养的哩!”正说着,李瓶儿从前边抱了官哥儿来。孟玉楼道:“拿过衣服来,等我替哥哥穿。”李瓶儿抱着,孟玉楼替他戴上道髻儿,套上项牌和两道索,唬的那孩子只把眼儿闭着,半日不敢出气儿。玉楼把道衣替他穿上。吴月娘吩咐李瓶儿:“你把这经疏,拿个阡张头儿,亲往后边佛堂中,自家烧了罢。”那李瓶儿去了。玉楼抱弄孩子说道:“穿着这衣服,就是个小道士儿。”金莲接过来说道:“什么小道士儿,倒好相个小太乙儿!”被月娘正色说了两句道:“六姐,你这个什么话,孩儿们面上,快休恁的。”那金莲讪讪的不言了。一回,那孩子穿着衣服害怕,就哭起来。李瓶儿走来,连忙接过来,替他脱衣裳时,就拉了一抱裙奶屎。孟玉楼笑道:“好个吴应元,原来拉屎也有一托盘。”月娘连忙叫小玉拿草纸替他抹。不一时,那孩子就磕伏在李瓶儿怀里睡着了。李瓶儿道:“小大哥原来困了,妈妈送你到前边睡去罢。”吴月娘一面把桌面都散了,请大妗子、杨娘、潘姥姥众人出来吃斋。

看看晚来。原来初八日西门庆因打醮,不用荤酒。潘金莲晚夕就没曾上的寿,直等到今晚来家与他递酒,来到大门站立。不想等到日落时分,只陈敬济和玳安自骑头口来家。潘金莲问:“你爹来了?”敬济道:“爹怕来不成了,我来时,醮事还未了,才拜忏,怕不弄到起更!道士有个轻饶素放的,还要谢将吃酒。”金莲听了,一声儿没言语,使性子回到上房里,对月娘说:“贾瞎子传操——干起了个五更!隔墙掠肝肠——死心塌地,兜肚断了带子——没得绊了!刚才在门首站了一回,见陈姐夫骑头口来了,说爹不来了,醮事还没了,先打发他来家。”月娘道:“他不来罢,咱每自在,晚夕听大师父、王师父说因果、唱佛曲儿。”正说着,只见陈敬济掀帘进来,已带半酣儿,说:“我来与五娘磕头。”问大姐:“有锺儿,寻个儿筛酒,与五娘递一锺儿。”大姐道:“那里寻锺儿去?只恁与五娘磕个头儿。到住回,等我递罢。你看他醉的腔儿,恰好今日打醮,只好了你,吃的恁憨憨的来家。”月娘便问道:“你爹真个不来了?玳安那奴才没来?”陈敬济道:“爹见醮事还没了,恐怕家里没人,先打发我来了,留下玳安在那里答应哩。吴道士再三不肯放我,强死强活拉着吃了两三大锺酒,才了。”月娘问:“今日有那几个在那里?”敬济道:“今日有大舅和门外花大舅、应三叔、谢三叔,又有李铭、吴惠两个小优儿。不知缠到多咱晚。只吴大舅来了。门外花大舅叫爹留住了,也是过夜的数。”金莲没见李瓶儿在跟前,便道:“陈姐夫,你也叫起花大舅来?是那门儿亲,死了的知道罢了。你叫他李大舅才是。”敬济道:“五娘,你老人家乡里姐姐嫁郑恩——睁着个眼儿,闭着个眼儿罢了。”大姐道:“贼囚根子,快磕了头,趁早与我外头挺去!又口里恁汗邪胡说了!”敬济于是请金莲转上,踉踉跄跄磕了四个头,往前边去了。

不一时,掌上灯烛,放桌儿,摆上菜儿,请潘姥姥、杨姑娘、大妗子与众人来。金莲递了酒,打发坐下,吃了面。吃到酒阑,收了家活,抬了桌出去。月娘吩咐小玉把仪门关了,炕上放下小桌儿,众人围定两个姑子,正在中间焚下香,秉着一对蜡烛,听着他说因果。先是大师父讲说,讲说的乃是西天第三十二祖下界降生东土,传佛心印的佛法因果,直从张员外家豪大富说起,漫漫一程一节,直说到员外感悟佛法难闻,弃了家园富贵,竟到黄梅寺修行去。说了一回,王姑子又接念偈言。

念了一回,吴月娘道:“师父饿了,且把经请过,吃些甚么。”一面令小玉安排了四碟儿素菜咸食,又四碟薄脆、蒸酥糕饼,请大妗子、杨姑娘、潘姥姥陪二位师父吃。大妗子说:“俺每都刚吃的饱了,教杨姑娘陪个儿罢,他老人家又吃着个斋。”月娘连忙用小描金碟儿,每样拣了点心,放在碟儿里,先递与两位师父,然后递与杨姑娘,说道:“你老人家陪二位请些儿。”婆子道:“我的佛爷,老身吃的勾了。”又道:“这碟儿里是烧骨朵,姐姐你拿过去,只怕错拣到口里。”把众人笑的了不得。月娘道:“奶奶,这个是庙上送来托荤咸食。你老人家只顾用,不妨事。”杨姑娘道:“既是素的,等老身吃。老身干净眼花了,只当做荤的来。”正吃着,只见来兴儿媳妇子惠香走来。月娘道:“贼臭肉,你也来什么?”惠香道:“我也来听唱曲儿。”月娘道:“仪门关着,你打那里进来了?”玉箫道:“他厨房封火来。”月娘道:“嗔道恁鼻儿乌嘴儿黑的,成精鼓捣,来听什么经!”

当下众丫鬟妇女围定两个姑子,吃了茶食,收过家活去,搽抹经桌干净。月娘从新剔起灯烛来,炷了香。两个姑子打动击子儿,又高念起来。从张员外在黄梅山寺中修行,白日长跪听经,夜夜参禅打坐。四祖禅师见他不凡,收留做了徒弟,与了他三桩宝贝,教他往浊河边投胎夺舍,直说到千金小姐在浊河边洗濯衣裳,见一僧人借房儿住,不合答了他一声,那老人就跳下河去了。潘金莲熬的磕困上来,就往房里睡去了。少顷,李瓶儿房中绣春来叫,说官哥儿醒了,也去了。只剩下李娇儿、孟玉楼、潘姥姥、孙雪娥、杨姑娘、大妗子守着。又听到河中漂过一个大鳞桃来,小姐不合吃了,归家有孕,怀胎十月。王姑子又接唱了一个《耍孩儿》。唱完,大师父又念了四偈言:

\[
五祖一佛性,投胎在腹中,
权住十个月,转凡度众生。
\]
念到此处,月娘见大姐也睡去了,大妗子\textuni{22C49}在月娘里间床上睡着了,杨姑娘也打起欠呵来,桌上蜡烛也点尽了两根,问小玉:“这天有多少晚了?”小玉道:“已是四更天气,鸡叫了。”月娘方令两位师父收拾经卷。杨姑娘便往玉楼房里去了。郁大姐在后边雪娥房里宿歇。月娘打发大师父和李娇儿一处睡去了。王姑子和月娘在炕上睡。两个还等着小玉顿了一瓶子茶,吃了才睡。大妗子在里间床上和玉箫睡。月娘因问王姑子:“后来这五祖长大了,怎生成正果?”王姑子复从爹娘怎的把千金小姐赶出,小姐怎的逃生,来到仙人庄;又怎的降生五祖,落后五祖养活到六岁;又怎的一直走到浊河边,取了三桩宝贝,迳往黄梅寺听四祖说法;又怎的遂成正果,后来还度脱母亲生天;直说完了才罢。月娘听了,越发好信佛法了。有诗为证:

\[
听法闻经怕无常,红莲舌上放毫光。
何人留下禅空话?留取尼僧化饭粮!
\]

\newpage
%# -*- coding:utf-8 -*-
%%%%%%%%%%%%%%%%%%%%%%%%%%%%%%%%%%%%%%%%%%%%%%%%%%%%%%%%%%%%%%%%%%%%%%%%%%%%%%%%%%%%%


\chapter{抱孩童瓶儿希宠\KG 妆丫鬟金莲市爱}


词曰:

\[
种就蓝田玉一株,看来的的可人娱。多方珍重好支持,掌中珠。傞俹漫惊新态变,妖娆偏与旧时殊。相逢一见笑成痴,少人知。
\]

话说当夜月娘和王姑子一炕睡。王姑子因问月娘:“你老人家怎的就没见点喜事儿?”月娘道:“又说喜事哩!前日八月里,因买了对过乔大户房子,平白俺每都过去看。上他那楼梯,一脚蹑滑了,把个六七个月身扭吊了。至今再谁见甚么喜儿来!”王姑子道:“我的奶奶,有七个月也成形了!”月娘道:“半夜里吊下杩子里,我和丫头点灯拨着瞧,倒是个小厮儿。”王姑子道:“我的奶奶,可惜了!怎么来扭着了?还是胎气坐的不牢。你老人家养出个儿来,强如别人。你看前边六娘,进门多少时儿,倒生了个儿子,何等的好!”月娘道:“他各人的儿女,随天罢了。”王姑子道:“也不打紧,俺每同行一个薛师父,一纸好符水药。前年陈郎中娘子,也是中年无子,常时小产了几胎,白不存,也是吃了薛师父符药,如今生了好不好一个满抱的小厮儿!一家儿欢喜的要不得。只是用着一件物件儿难寻。”月娘问道:“什么物件儿?”王姑子道:“用着头生孩子的衣胞,拿酒洗了,烧成灰儿,伴着符药,拣壬子日,人不知,鬼不觉,空心用黄酒吃了。算定日子儿不错,至一个月就坐胎气,好不准!”月娘道:“这师父是男僧女僧?在那里住?”王姑子道:“他也是俺女僧,也有五十多岁。原在地藏庵儿住来,如今搬在南首法华庵儿做首座,好不有道行!他好少经典儿!又会讲说《金刚科仪》各样因果宝卷,成月说不了。专在大人家行走,要便接了去,十朝半月不放出来。”月娘道:“你到明日请他来走走,”王姑子道:“我知道。等我替你老人家讨了这符药来着。止是这一件儿难寻,这里没寻处。恁般如此,你不如把前头这孩子的房儿,借情跑出来使了罢。”月娘道:“缘何损别人安自己。我与你银子,你替我慢慢另寻便了。”王姑子道:“这个到只是问老娘寻,他才有。我替你整治这符水,你老人家吃了管情就有。难得你明日另养出来,随他多少,十个明星当不的月!”月娘吩咐:“你却休对人说。”王姑子道:“好奶奶,傻了我?肯对人说!”说了一回,方睡了。一宿晚景题过。

到次日,西门庆打庙里来家,月娘才起来梳头。玉箫接了衣服,坐下。月娘因说:“昨日家里六姐等你来上寿,怎的就不来了?”西门庆悉把醮事未了,吴亲家晚夕费心,摆了许多桌席——“吴大舅先来了,留住我和花大哥、应二哥、谢希大。两个小优儿弹唱着,俺每吃了一夜酒。今早我便先进城来了,应二哥他三个还吃酒哩。”告诉了一回。玉箫递茶吃了。也没往衙门里去,走到前边书房里,\textuni{22C49}着床上就睡着了。落后潘金莲、李瓶儿梳了头,抱着孩子出来,都到上房,陪着吃茶。月娘向李瓶儿道:“他爹来了这一日,在前头哩,我叫他吃茶食,他不吃。如今有了饭了。你把你家小道士替他穿上衣裳,抱到前头与他爹瞧瞧去。”潘金莲道:“我也去。等我替道士儿穿衣服。”于是戴上销金道髻儿,穿上道衣,带了顶牌符索,套上小鞋袜儿,金莲就要夺过去。月娘道:“叫他妈妈抱罢。你这蜜褐色桃绣裙子不耐污,撒上点子臜到了不成。”于李瓶儿抱定官哥儿,潘金莲便跟着,来到前边西厢房内。书童见他二人掀帘,连忙就躲出来了。金莲见西门庆脸朝里睡,就指着孩子说:“老花子,你好睡!小道士儿自家来请你来了。大妈妈房里摆下饭,叫你吃去,你还不快起来,还推睡儿!”那西门庆吃了一夜酒的人,丢倒头,那顾天高地下,鼾睡如雷。

金莲与李瓶儿一边一个坐在床上,把孩子放在他面前,怎禁的鬼混,不一时把西门弄醒了。睁开眼看见官哥儿在面前,穿着道士衣服,喜欢的眉开眼笑。连忙接过来,抱到怀里,与他亲个嘴儿。金莲道:“好干净嘴头子,就来亲孩儿!小道士儿吴应元,你哕他一口,你说昨日在那里使牛耕地来,今日乏困的这样的,大白日困觉?昨日叫五妈只顾等着你。你恁大胆,不来与五妈磕头。”西门庆道:“昨日醮事散得晚。晚夕谢将,整吃了一夜。今日到这咱还一头酒,在这里睡回,还要往尚举人家吃酒去。”金莲道:“你不吃酒去罢了。”西门庆道:“他家从昨日送了帖儿来,不去惹人家不怪!”金莲道:“你去,晚夕早些儿来家,我等着你哩。”李瓶儿道:“他大妈妈摆下饭了,又做了些酸笋汤,请你吃饭去哩。”西门庆道:“我心里还不待吃,等我去喝些汤罢。”于是起来往后边去了。

这潘金莲见他去了,一屁股就坐在床上正中间,脚蹬着地炉子说道:“这原来是个套炕子。”伸手摸了摸褥子里,说道:“到且是烧的滚热的炕儿。”瞧了瞧旁边桌上,放着个烘砚瓦的铜丝火炉儿,随手取过来,叫:“李大姐,那边香几儿上牙盒里盛的甜香饼儿,你取些来与我。”一面揭开了,拿几个在火炕内,一面夹在裆里,拿裙子裹的沿沿的,且薰热身上。坐了一回,李瓶儿说道:“咱进去罢,只怕他爹吃了饭出来。”金莲道:“他出来不是?怕他么!”于是二人抱着官哥,进入后边来。良久,西门庆吃了饭,吩咐排军备马,午后往尚举人家吃酒去了。潘姥姥先去了。

且说晚夕王姑子要家去。月娘悄悄与了他一两银子,叫他休对大师姑说,好歹请薛姑子带了符药来。王姑子接了银子,和月娘说:“我这一去,只过十六日才来。就替你寻了那件东西儿来。”月娘道:“也罢,你只替我干的停当,我还谢你。”于是作辞去了。看官听说:但凡大人家,似这等尼僧牙婆,决不可抬举。在深宫大院,相伴着妇女,俱以谈经说典为由,背地里送暖偷寒,甚么事儿不干出来?有诗为证:

\[
最有缁流不可言,深宫大院哄婵娟。
此辈若皆成佛道,西方依旧黑漫漫。
\]

却说金莲晚夕走到镜台前,把\textuni{4BFC}髻摘了,打了个盘头楂髻,把脸搽的雪白,抹的嘴唇儿鲜红,戴着两个金灯笼坠子,贴着三个面花儿,带着紫销金箍儿,寻了一套红织金祆儿,下着翠蓝缎子裙:要妆丫头,哄月娘众人耍子。叫将李瓶儿来与他瞧。把李瓶儿笑的前仰后合,说道:“姐姐,你妆扮起来,活象个丫头。我那屋里有红布手巾,替你盖着头。等我往后边去,对他们只说他爹又寻了个丫头,唬他们唬,管定就信了。”春梅打着灯笼在头里走,走到仪门首,撞见陈敬济,笑道:“我道是谁来,这个就是五娘干的营生!”李瓶儿叫道:“姐夫,你过来,等我和你说了,着你先进去见他们,只如此这般。”敬济道:“我有法儿哄他。”于是先走到上房里。众人都在炕上坐着吃茶,敬济道:“娘,你看爹平白里叫薛嫂儿使了十六两银子,买了人家一个二十五岁,会弹唱的姐儿,刚才拿轿子送将来了。”月娘道:“真个?薛嫂儿怎不先来对我说?”敬济道:“他怕你老人家骂他,送轿子到大门首,就去了。丫头便叫他们领进来了。”大妗子还不言语,杨姑娘道:“官人有这几房姐姐勾了,又要他来做什么?”月娘道:“好奶奶,你禁的!有钱就买一百个有什么多?俺们都是老婆当军——充数儿罢了!”玉箫道:“等我瞧瞧去。”只见月亮地里,原是春梅打灯笼,落后叫了来安儿打着,和李瓶儿后边跟着,搭着盖头,穿着红衣服进来。慌的孟玉楼、李娇儿都出来看。良久,进入房里。玉箫挨在月娘边说道:“这个是主子,还不磕头哩!”一面揭了盖头。那潘金莲插烛也似磕下头去,忍不住扑矻的笑了。玉楼道:“好丫头,不与你主子磕头,且笑!”月娘笑了,说道:“这六姐成精死了罢!把俺每哄的信了。”玉楼道:“我不信。”杨姑娘道:“姐姐,你怎的见出来不信?”玉楼道:“俺六姐平昔磕头,也学的那等磕了头起来,倒退两步才拜。”杨姑娘道:“还是姐姐看的出来,要着老身就信了。”李娇儿道:“我也就信了。刚才不是揭盖头,他自家笑,还认不出来。”正说着,只见琴童儿抱进毡包来,说:“爹来家了。”孟玉楼道:“你且藏在明间里。等他进来,等我哄他哄。”

不一时,西门庆来到,杨姑娘、大妗子出去了,进入房内椅子上坐下。月娘在旁不言语。玉楼道:“今日薛嫂儿轿子送人家一个二十岁丫头来,说是你叫他送来要他的,你恁大年纪,前程也在身上,还干这勾当?”西门庆笑道:“我那里叫他买丫头来?信那老淫妇哄你哩!”玉楼道:“你问大姐姐不是?丫头也领在这里,我不哄你。你不信,我叫出来你瞧。”于是叫玉箫:“你拉进那新丫头来,见你爹。”那玉箫掩着嘴儿笑,又不敢去拉,前边走了走儿,又回来了,说道:“他不肯来。”玉楼道:“等我去拉,恁大胆的奴才,头儿没动,就扭主子,也是个不听指教的!”一面走到明间内。只听说道:“怪行货子,我不好骂的!人不进去,只顾拉人,拉的手脚儿不着。”玉楼笑道:“好奴才,谁家使的你恁没规矩,不进来见你主子磕头。”一面拉进来。西门庆灯影下睁眼观看,却是潘金莲打着揸髻装丫头,笑的眼没缝儿。那金莲就坐在旁边椅子上。玉楼道:“好大胆丫头!新来乍到,就恁少条失教的,大剌剌对着主子坐着!”月娘笑道,“你趁着你主子来家,与他磕个头儿罢。”那金莲也不动,走到月娘里间屋里,一顿把簪子拔了,戴上\textuni{4BFC}髻出来。月娘道:“好淫妇,讨了谁上头话,就戴上\textuni{4BFC}髻了!”众人又笑了一回。月娘告诉西门庆说:“今日乔亲家那里,使乔通送了六个帖儿来,请俺们十二日吃看灯酒。咱到明日,不先送些礼儿去?”西门庆道:“明早叫来兴儿,买四盘肴品、一坛南酒送去就是了。到明日,咱家发柬,十四日也请他娘子,并周守备娘子、荆都监娘子、夏大人娘子、张亲家母。大妗子也不必家去了。教贲四叫将花儿匠来,做几架烟火。王皇亲家一起扮戏的小厮,叫他来扮《西厢记》。往院中再把吴银儿、李桂姐接了来。你们在家看灯吃酒,我和应二哥、谢子纯往狮子街楼上吃酒去。”说毕,不一时放下桌儿,安排酒上来。

潘金莲递酒,众姊妹相陪吃了一回。西门庆因见金莲装扮丫头,灯下艳妆浓抹,不觉淫心漾漾,不住把眼色递与他。金莲就知其意,就到前面房里,去了冠儿,挽着杭州缵,重匀粉面,复点朱唇。早在房中预备下一桌齐整酒菜等候。不一时,西门庆果然来到,见妇人还挽起云髻来,心中甚喜,搂着他坐在椅子上,两个说笑。不一时,春梅收拾上酒菜来。妇人从新与他递酒。西门庆道:“小油嘴儿,头里已是递过罢了,又教你费心。”金莲笑道:“那个大伙里酒儿不算,这个是奴家业儿,与你递锺酒儿,年年累你破费,你休抱怨。”把西门庆笑的没眼缝儿,连忙接了他酒,搂在怀里膝盖上坐的。春梅斟酒,秋菊拿菜儿。金莲道:“我问你,十二日乔家请,俺每都去?只教大姐姐去?”西门庆道:“他即下帖儿都请,你每如何不去?到明日,叫奶子抱了哥儿也去走走,省得家里寻他娘哭。”金莲道:“大姐姐他们都有衣裳穿,我老道只有数的那几件子,没件好当眼的。你把南边新治来那衣裳,一家分散几件子,裁与俺们穿了罢!只顾放着,敢生小的儿也怎的?到明日咱家摆酒,请众官娘子,俺们也好见他,不惹人笑话。我长是说着,你把脸儿憨着。”西门庆笑道:“既是恁的,明日叫了赵裁来,与你们裁了罢,”金莲道:“及至明日叫裁缝做,只差两日儿,做着还迟了哩。”西门庆道:“对赵裁说,多带几个人来,替你们攒造两三件出来就勾了。剩下别的慢慢再做也不迟。”金莲道:“我早对你说过,好歹拣两套上色儿的与我,我难比他们都有,我身上你没与我做什么大衣裳。”西门庆笑道:“贼小油嘴儿,去处掐个尖儿。”两个说话饮酒,到一更时分方上床。两个如被底鸳鸯,帐中鸾凤,整狂了半夜。

到次日,西门庆衙门中回来,开了箱柜,拿出南边织造的罗缎尺头来。每人做件妆花通袖袍儿,一套遍地锦衣服,一套妆花衣服。惟月娘是两套大红通袖遍地锦袍儿,四套妆花衣服。在卷棚内,一面使琴童儿叫将赵裁来。赵裁见西门庆,连忙磕了头。桌上铺着毡条,取出剪尺来,先裁月娘的:一件大红遍地锦五彩妆花通袖袄,兽朝麒麟补子缎袍儿;一件玄色五彩金遍边葫芦样鸾凤穿花罗袍;一套大红缎子遍地金通麒麟补子袄儿,翠蓝宽拖遍地金裙;一套沉香色妆花补子遍地锦罗祆儿,大红金枝绿叶百花拖泥裙。其余李娇儿、孟玉楼、潘金莲、李瓶儿四个都裁了一件大红五彩通袖妆花锦鸡缎子袍儿,两套妆花罗缎衣服。孙雪娥只是两套,就没与他袍儿。须臾共裁剪三十件衣服。兑了五两银子,与赵裁做工钱。一面叫了十来个裁缝在家攒造,不在话下。正是:

\[
金铃玉坠妆闺女,锦绮珠翘饰美娃。
\]

\newpage
%# -*- coding:utf-8 -*-
%%%%%%%%%%%%%%%%%%%%%%%%%%%%%%%%%%%%%%%%%%%%%%%%%%%%%%%%%%%%%%%%%%%%%%%%%%%%%%%%%%%%%


\chapter{两孩儿联姻共笑嬉\KG 二佳人愤深同气苦}


词曰:

\[
潇洒佳人,风流才子,天然吩咐成双。兰堂绮席,烛影耀荧煌。数幅红罗锦绣,宝妆篆、金鸭焚香。分明是,芙蕖浪里,一对鸳鸯。
\]

话说西门庆在家中,裁缝攒造衣服,那消两日就完了。到十二日,乔家使人邀请。早晨,西门庆先送了礼去。那日,月娘并众姊妹、大妗子,六顶轿子一搭儿起身。留下孙雪娥看家。奶子如意儿抱着官哥,又令来兴媳妇蕙秀伏侍叠衣服,又是两顶小轿。

西门庆在家,看着贲四叫了花儿匠来扎缚烟火,在大厅、卷棚内挂灯,使小厮拿帖儿往王皇亲宅内定下戏子,俱不必细说。后晌时分,走到金莲房中。金莲不在家,春梅在旁伏侍茶饭,放桌儿吃酒。西门庆因对春梅说:“十四日请众官娘子,你们四个都打扮出去,与你娘跟着递酒,也是好处。”春梅听了,斜靠着桌儿说道:“你若叫,只叫他三个出去,我是不出去。”西门庆道:“你怎的不出去?”春梅道:“娘们都新做了衣裳,陪侍众官户娘子便好看。俺们一个一个只像烧煳了卷子一般,平白出去惹人家笑话。”西门庆道:“你们都有各人的衣服首饰、珠翠花朵。”春梅道:“头上将就戴着罢了,身上有数那两件旧片子,怎么好穿出去见人的!到没的羞剌剌的。”西门庆笑道:“我晓的你这小油嘴儿,见你娘们做了衣裳,却使性儿起来。不打紧,叫赵裁来,连大姐带你四个,每人都裁三件:一套缎子衣裳、一件遍地锦比甲。”春梅道:“我不比与他。我还问你要件白绫袄儿,搭衬着大红遍地锦比甲儿穿。”西门庆道:“你要不打紧,少不的也与你大姐裁一件。”春梅道:“大姑娘有一件罢了,我却没有,他也说不的。”西门庆于是拿钥匙开楼门,拣了五套缎子衣服、两套遍地锦比甲儿,一匹白绫裁了两件白绫对衿袄儿。惟大姐和春梅是大红遍地锦比甲儿,迎春、玉箫、兰香,都是蓝绿颜色;衣服都是大红缎子织金对衿袄,翠蓝边拖裙,共十七件。一面叫了赵裁来,都裁剪停当。又要一匹黄纱做裙腰,贴里一色都是杭州绢儿。春梅方才喜欢了,陪侍西门庆在屋里吃了一日酒,说笑顽耍不题。

且说吴月娘众妹妹到了乔大户家。原来乔大户娘子那日请了尚举人娘子,并左邻朱台官娘子、崔亲家母,并两个外甥侄女儿——段大姐及吴舜臣媳妇儿郑三姐。叫了两个妓女,席前弹唱。听见月娘众姊妹和吴大妗子到了,连忙出仪门首迎接,后厅叙礼。赶着月娘呼姑娘,李娇儿众人都排行叫二姑娘、三姑娘……,俱依吴大妗子那边称呼之礼。又与尚举人、朱台官娘子叙礼毕,段大姐、郑三姐向前拜见了。各依次坐下。丫环递过了茶,乔大户出来拜见,谢了礼。他娘子让进众人房中去宽衣服,就放桌儿摆茶,请众堂客坐下吃茶。奶子如意儿和蕙秀在房中看官哥儿,另自管待。须臾,吃了茶到厅,屏开孔雀,褥隐芙蓉,正面设四张桌席。让月娘坐了首位,其次就是尚举人娘子、吴大妗子、朱台官娘子、李娇儿、孟玉楼、潘金莲、李瓶儿,乔大户娘子,关席坐位,旁边放一桌,是段大姐、郑三姐,共十一位。两个妓女在旁边唱。上了汤饭,厨役上来献了头一道水晶鹅,月娘赏了二钱银子;第二道是顿烂\textHuoKua 蹄儿,月娘又赏了一钱银子;第三道献烧鸭,月娘又赏了一钱银子。乔大户娘子下来递酒,递了月娘过去,又递尚举人娘子。月娘就下来往后房换衣服、匀脸去了。

孟玉楼也跟下来,到了乔大户娘子卧房中,只见奶子如意儿看守着官哥儿,在炕上铺着小褥子儿躺着。他家新生的长姐,也在旁边卧着。两个你打我下儿,我打你下儿顽耍。把月娘、玉楼见了,喜欢的要不得,说道:“他两个倒好相两口儿。”只见吴大妗子进来,说道:“大妗子,你来瞧瞧,两个倒相小两口儿。”大妗子笑道:“正是。孩儿每在炕上,张手蹬脚儿的,你打我,我打你,小姻缘一对儿耍子。”乔大户娘子和众堂客都进房到。吴大妗子如此这般说。乔大户娘子道:“列位亲家听着,小家儿人家,怎敢攀的我这大姑娘府上?”月娘道:“亲家好说,我家嫂子是何人?郑三姐是何人?我与你爱亲做亲,就是我家小儿也玷辱不了你家小姐,如何却说此话?”玉楼推着李瓶儿说道:“李大姐,你怎的说?”那李瓶儿只是笑。吴妗子道:“乔亲家不依,我就恼了。”尚举人娘子和朱台官娘子皆说道:“难为吴亲家厚情,乔亲家你休谦辞了。”因问:“你家长姐去年十一月生的?”月娘道:“我家小儿六月廿三日生的,原大五个月,正是两口儿。”众人不由分说,把乔大户娘子和月娘、李瓶儿拉到前厅,两个就割了衫襟。两个妓女弹唱着。旋对乔大户说了,拿出果盒、三段红来递酒。月娘一面吩咐玳安、琴童快往家中对西门庆说。旋抬了两坛酒、三匹缎子、红绿板儿绒金丝花、四个螺甸大果盒。两家席前,挂红吃酒。一面堂中画烛高擎,花灯灿烂,麝香叆叆,喜笑匆匆。两个妓女,启朱唇,露皓齿,轻拨玉阮,斜抱琵琶唱着。

众堂客与吴月娘、乔大户娘子、李瓶儿三人都簪了花,挂了红,递了酒,各人都拜了。从新复安席坐人饮酒。厨子上了一道裹馅寿字雪花糕、喜重重满池娇并头莲汤。月娘坐在上席,满心欢喜,叫玳安过来,赏一匹大红与厨役。两个妓女每人都是一匹。俱磕头谢了。乔大户娘子不放起身,还在后堂留坐,摆了许多劝碟,细果攒盒。约吃到一更时分,月娘等方才拜辞回来,说道:“亲家,明日好歹下降寒舍那里坐坐。”乔大户娘子道:“亲家盛情,家老儿说来,只怕席间不好坐的,改日望亲家去罢。”月娘道:“好亲家,再没人。亲家只是见外。”因留了大妗子:“你今日不去,明日同乔亲家一搭儿里来罢。”大妗子道:“乔亲家,别的日子你不去罢,到十五日,你正亲家生日,你莫不也不去?”乔大户娘子道:“亲家十五日好日子,我怎敢不去!”月娘道:“亲家若不去,大妗子,我交付与你,只在你身上。”于是,生死把大妗子留下了,然后作辞上轿。

头里两个排军,打着两个大红灯笼;后边又是两个小厮,打着两个灯笼。吴月娘在头里,李娇儿、孟玉楼、潘金莲、李瓶儿一字在中间,如意儿和蕙秀随后。奶子轿子里用红绫小被把官哥儿裹得沿沿的,恐怕冷,脚下还蹬着铜火炉儿。两边小厮圜随。到了家门首下轿,西门庆正在上房吃酒,月娘等众人进来,道了万福,坐下。众丫鬟都来磕了头。月娘先把今日酒席上结亲之话,告诉了一遍。西门庆听了道:“今日酒席上有那几位堂客?”月娘道:“有尚举人娘子、朱序班娘子、崔亲家母、两个侄女。”西门庆说:“做亲也罢了,只是有些不搬陪。”月娘道:“倒是俺嫂子,见他家新养的长姐和咱孩子在床炕上睡着,都盖着那被窝儿,你打我一下儿,我打你一下儿,恰是小两口儿一般,才叫了俺们去,说将起来,酒席上就不因不由做了这门亲。我方才使小厮来对你说,抬送了花红果盒去。”西门庆道:“既做亲也罢了,只是有些不搬陪些。乔家虽有这个家事,他只是个县中大户白衣人。你我如今见居着这官,又在衙门中管着事,到明日会亲酒席间,他戴着小帽,与俺这官户怎相处?甚不雅相。就是前日,荆南冈央及营里张亲家,再三赶着和我做亲,说他家小姐今才五个月儿,也和咱家孩子同岁。我嫌他没娘母子,是房里生的,所以没曾应承他。不想到与他家做了亲。”潘金莲在旁接过来道:“嫌人家是房里养的,谁家是房外养的?就是乔家这孩子,也是房里生的。正是险道神撞着寿星老儿——你也休说我长,我也休嫌你短。”西门庆听了此言,心中大怒,骂道:“贼淫妇,还不过去!人这里说话,也插嘴插舌的。有你甚么说处!”金莲把脸羞的通红了,抽身走出来,说道:“谁说这里有我说处?可知我没说处哩!”

看官听说:今日潘金莲在酒席上,见月娘与乔大户家做了亲,李瓶儿都披红簪花递酒,心中甚是气不愤,来家又被西门庆骂了这两句,越发急了,走到月娘这边屋里哭去了。西门庆因问:“大妗子怎的不来?”月娘道:“乔亲家母明日见有众官娘子,说不得来。我留下他在那里,教明日同他一搭儿里来。”西门庆道:“我说只这席间坐次上不好相处,到明日怎么厮会?”说了回话,只见孟玉楼也走到这边屋里来,见金莲哭泣,说道:“你只顾恼怎的?随他说几句罢了。”金莲道:“早是你在旁边听着,我说他什么歹话来?他说别家是房里养的,我说乔家是房外养的?也是房里生的。那个纸包儿包着,瞒得过人?贼不逢好死的强人,就睁着眼骂起我来。骂的人那绝情绝义。怎的没我说处?改变了心,教他明日现报在我的眼里!多大的孩子,一个怀抱的尿泡种子,平白扳亲家,有钱没处施展的,争破卧单——没的盖,狗咬尿胞——空欢喜!如今做湿亲家还好,到明日休要做了干亲家才难。吹杀灯挤眼儿——后来的事看不见。做亲时人家好,过三年五载方了的才一个儿!”玉楼道:“如今人也贼了,不干这个营生。论起来也还早哩。才养的孩子,割甚么衫襟?无过只是图往来扳陪着耍子儿罢了。”金莲道:“你便浪\textuni{22D5E}着图扳亲家耍子,平白教贼不合钮的强人骂我。”玉楼道:“谁教你说话不着个头项儿就说出来?他不骂你骂狗?”金莲道:“我不好说的,他不是房里,是大老婆?就是乔家孩子,是房里生的,还有乔老头子的些气儿。你家失迷家乡,还不知是谁家的种儿哩!”玉楼听了,一声儿没言语。坐了一回,金莲归房去了。

李瓶儿见西门庆出来了,从新花枝招飐与月娘磕头,说道:“今日孩子的事,累姐姐费心。”那月娘笑嘻嘻,也倒身还下礼去,说道:“你喜呀?”李瓶儿道:“与姐姐同喜。”磕毕头起来,与月娘、李娇儿坐着说话。只见孙雪娥、大姐来与月娘磕头,与李娇儿、李瓶儿道了万福。小玉拿茶来,正吃茶,只见李瓶儿房里丫鬟绣春来请,说:“哥儿屋里寻哩,爹使我请娘来了。”李瓶儿道:“奶子慌的三不知就抱的屋里去了。一搭儿去也罢了,只怕孩子没个灯儿。”月娘道:“头里进门,到是我叫他抱的房里去。恐怕晚了。”小玉道:“头里如意儿抱着他,来安儿打着灯笼送他来。”李瓶儿道:“这等也罢了。”于是,作辞月娘,回房中来。只见西门庆在屋里,官哥儿在奶子怀里睡着了。因说:“你如何不对我说就抱了他来?”如意儿道:“大娘见来安儿打着灯笼,就趁着灯儿来了。哥哥哭了一口,才拍着他睡着了。”西门庆道:“他寻了这一回,才睡了。”李瓶儿说毕,望着他笑嘻嘻说道:“今日与孩儿定了亲,累你,我替你磕个头儿。”于是,插烛也似磕下去。喜欢的西门庆满面堆笑,连忙拉起来,做一处坐的。一面令迎春摆下酒儿,两个吃酒。

且说潘金莲到房中使性子,没好气,明知道西门庆在李瓶儿这边,因秋菊开的门迟了,进门就打了两个耳刮子,高声骂道:“贼淫妇奴才!怎的叫了恁一日不开?你做甚么来?我且不和你答话。”于是走到屋里坐下。春梅走来磕头递茶。妇人他:“贼奴才他在屋里做什么来?”春梅道:“在院子里坐着来。我这等催他,还不理。”妇人道:“我知道他和我两个怄气。党太尉吃匾食,他也学人照样儿欺负我。”待要打他,又恐西门庆听见;不言语,心中又气。一面卸了浓妆,春梅与他搭了铺,上床就睡了。

到次日,西门庆衙门中去了。妇人把秋菊叫他顶着大块柱石,跪在院子里。跪的他梳了头,叫春梅扯了他裤子,拿大板子要打他。春梅道:“好干净的奴才,叫我扯裤子,到没的污浊了我的手!”走到前边,旋叫了画童儿扯去秋菊的衣。妇人打着他骂道:“贼奴才淫妇,你从几时就恁大来?别人兴你,我却不兴你。姐姐,你知我见的,将就脓着些儿罢了。平白撑着头儿,逞什么强?姐姐,你休要倚着,我到明日洗着两个眼儿看着你哩!”一面骂着又打,打了又骂,打的秋菊杀猪也似叫。李瓶儿那边才起来,正看着奶子打发官哥儿睡着了,又唬醒了。明明白白听见金莲这边打丫鬟,骂的言语儿有因,一声儿不言语,唬的只把官哥儿耳朵握着。一面使绣春:“去对你五娘说休打秋菊罢。哥儿才吃了些奶睡着了。”金莲听了,越发打的秋菊狠了,骂道:“贼奴才,你身上打着一万把刀子,这等叫饶。我是恁性儿,你越叫,我越打。莫不为你拉断了路行人?人家打丫头,也来看着你。好姐姐,对汉子说,把我别变了罢!”李瓶儿这边分明听见指骂的是他,把两只手气的冰冷,忍气吞声,敢怒而不敢言。早晨茶水也没吃,搂着官哥儿在炕上就睡着了。

等到西门庆衙门中回家,入房来看官哥儿,见李瓶儿哭的眼红红的,睡在炕上,问道:“你怎的这咱还不梳头?上房请你说话。你怎揉的眼恁红红的?”李瓶儿也不题金莲指骂之事,只说:“我心中不自在。”西门庆告说:“乔亲家那里,送你的生日礼来了。一匹尺头、两坛南酒、一盘寿桃、一盘寿面、四样下饭。又是哥儿送节的两盘元宵、四盘蜜食、四盘细果、两挂珠子吊灯、两座羊皮屏风灯、两匹大红官缎、一顶青缎\textuni{3A5F}的金八吉祥帽儿、两双男鞋、六双女鞋。咱家倒还没往他那里去,他又早与咱孩儿送节来了。如今上房的请你计较去。他那里使了个孔嫂儿和乔通押了礼来。大妗子先来了,说明日乔亲家母不得来,直到后日才来。他家有一门子做皇亲的乔五太太听见和咱们做亲,好不喜欢!到十五日,也要来走走,咱少不得补个帖儿请去。”李瓶儿听了,方慢慢起来梳头,走了后边,拜了大妗子。孔嫂儿正在月娘房里待茶,礼物摆在明间内,都看了。一面打发回盒起身,与了孔嫂儿、乔通每人两方手帕、五钱银子,写了回帖去了。正是:但将钟鼓悦和爱,好把犬羊为国羞。有诗为证:

\[
西门独富太骄矜,襁褓孩儿结做亲。
不独资财如粪上,也应嗟叹后来人。
\]

\newpage
%# -*- coding:utf-8 -*-
%%%%%%%%%%%%%%%%%%%%%%%%%%%%%%%%%%%%%%%%%%%%%%%%%%%%%%%%%%%%%%%%%%%%%%%%%%%%%%%%%%%%%


\chapter{逞豪华门前放烟火\KG 赏元宵楼上醉花灯}


诗曰:

\[
星月当空万烛烧,人间天上两元宵。
乐和春奏声偏好,人蹈衣归马亦娇。
易老韶光休浪度,最公白发不相饶。
千金博得斯须刻,吩咐谯更仔细敲。
\]

话说西门庆打发乔家去了,走来上房,和月娘、大妗子、李瓶儿商议。月娘道:“他家既先来与咱孩子送节,咱少不得也买礼过去,与他家长姐送节。就权为插定一般,庶不差了礼数。”大妗子道:“咱这里,少不的立上个媒人,往来方便些。”月娘道:“他家是孔嫂儿,咱家安上谁好?”西门庆道:“一客不烦二主,就安上老冯罢。”于是,连忙写了请帖八个,就叫了老冯来,同玳安拿请帖盒儿,十五日请乔老亲家母、乔五太太并尚举人娘子、朱序班娘子、崔亲家母、段大姐、郑三姐来赴席,与李瓶儿做生日,并吃看灯酒。一面吩咐来兴儿,拿银子早定下蒸酥点心并羹果食物。又是两套遍地锦罗缎衣服,一件大红小袍儿、一顶金丝绉纱冠儿、两盏云南羊角珠灯、一盒衣翠、一对小金手镯、四个金宝石戒指儿。十四日早装盒担,叫女婿陈敬济和贲四穿青衣服押送过去。乔大户那边,酒筵管待,重加答贺。回盒中,又回了许多生活鞋脚,俱不必细说。正乱着,应伯爵来讲李智、黄四官银子事,看见,问其所以。西门庆告诉与乔大户结亲之事:“十五日好歹请令正来陪亲家坐坐。”伯爵道:“嫂子呼唤,房下必定来。”西门庆道:“今日请众堂官娘子吃酒,咱每往狮子街房子内看灯去罢。”伯爵应诺去了,不题。

且说那日院中吴银儿先送了四盒礼来,又是两方销金汗巾,一双女鞋,送与李瓶儿上寿,就拜干女儿。月娘收了礼物,打发轿子回去。李桂姐只到次日才来,见吴银儿在这里,便悄悄问月娘:“他多咱来的?”月娘如此这般告他说:“昨日送了礼来,拜认你六娘做干女儿了。”李桂姐听了,一声儿没言语。一日只和吴银儿使性子,两个不说话。

却说前厅王皇亲家二十名小厮,两个师父领着,挑了箱子来,先与西门庆磕头。西门庆吩咐西厢房做戏房,管待酒饭。不一时,周守备娘子、荆都监母亲荆太太与张团练娘子,都先到了。俱是大轿,排军喝道,家人媳妇跟随。月娘与众姊妹,都穿着袍出来迎接,至后厅叙礼。与众亲相见毕,让坐递茶,等着夏提刑娘子到才摆茶。不料等到日中,还不见来。小厮邀了两三遍,约午后才喝了道来,抬着衣匣,家人媳妇跟随,许多仆从拥护。鼓乐接进后厅,与众堂客见毕礼数,依次序坐下。先在卷棚内摆茶,然后大厅上坐。春梅、玉箫、迎春、兰香,都是齐整妆束,席上捧茶斟酒。那日扮的是《西厢记》。

不说画堂深处,珠围翠绕,歌舞吹弹饮酒。单表西门庆打发堂客上了茶,就骑马约下应伯爵、谢希大,往狮子街房里去了。吩咐四架烟火,拿一架那里去。晚夕,堂客跟前放两架。旋叫了个厨子,家下抬了两食盒下饭菜蔬,两坛金华酒去。又叫了两个唱的——董娇儿、韩玉钏儿。原来西门庆已先使玳安雇轿子,请王六儿同往狮子街房里去。玳安见妇人道:“爹说请韩大婶,那里晚夕看放烟火。”妇人笑道:“我羞剌剌,怎么好去的,你韩大叔知道不嗔?”玳安道:“爹对韩大叔说了,教你老人家快收拾哩。因叫了两个唱的,没人陪他。”那妇人听了,还不动身。一回,只见韩道国来家。玳安道:“这不是韩大叔来了。韩大婶这里,不信我说哩。”妇人向他汉子说,“真个叫我去?”韩道国道:“老爹再三说,两个唱的没人陪他,请你过去,晚夕就看放烟火。你还不收拾哩!刚才教我把铺子也收了,就晚夕一搭儿里坐坐。保官儿也往家去了,晚夕该他上宿哩。”妇人道:“不知多咱才散,你到那里坐回就来罢,家里没人,你又不该上宿。”说毕,打扮穿了衣服,玳安跟随,迳到狮子街房里来。来昭妻一丈青早在房里收拾下床炕、帐幔、褥被,安息沉香薰的喷鼻香。房里吊着一对纱灯,笼着一盆炭火。妇人走到里面炕上坐下。一丈青走出来,道了万福,拿茶吃了。西门庆与应伯爵看了回灯,才到房子里。两个在楼上打双陆。楼上除了六扇窗户,挂着帘子,下边就是灯市,十分闹热。打了回双陆,收拾摆饭吃了,二人在帘里观看灯市。但见:

\[
万井人烟锦绣围,香车宝马闹如雷。
鳌山耸出青云上,何处游人不看来?
\]

二人看了一回,西门庆忽见人丛里谢希大、祝实念,同一个戴方巾的在灯棚下看灯,指与伯爵瞧。因问:“那戴方巾的,你可认的他?”伯爵道:“此人眼熟,不认的他。”西门庆便叫玳安:“你去下边,悄悄请了谢爹来。休教祝麻子和那人看见。”玳安小厮贼,一直走下楼来,挨到人闹里,待祝实念和那人先过去了,从旁边出来,把谢希大拉了一把。慌的希大回身观看,却是玳安。玳安道:“爹和应二爹在这楼上,请谢爹说话。”希大道:“你去,我知道了。等我陪他两个到粘梅花处,就来见你爹。”玳安便一道烟去了。希大到了粘梅花处,向人闹处,就叉过一边,由着祝实念和那一个人只顾寻。他便走来楼上,见西门庆、应伯爵两个作揖,因说道:“哥来此看灯,早晨就不呼唤兄弟一声?”西门庆道:“我早晨对众人,不好邀你每的。已托应二哥到你家请你去,说你不在家。刚才,祝麻子没看见么?”因问:“那戴方巾的是谁?”希大道:“那戴方巾的,是王昭宣府里王三官儿。今日和祝麻子到我家,要问许不与先生那里借三百两银子。央我和老孙、祝麻子作保。要干前程,入武学肄业。我那里管他这闲帐!刚才陪他灯市里走了走,听见哥呼唤,我只伴他到粘梅花处,交我乘人乱,就叉开了走来见哥。”因问伯爵:“你来多大回了?”伯爵道:“哥使我先到你家,你不在,我就来了,和哥在这里打了这回双陆。”西门庆问道:“你吃了饭不曾?”谢希大道:“早晨从哥那里出来,和他两个搭了这一日,谁吃饭来!”西门庆吩咐玳安:“厨下安排饭来,与你谢爹吃。”不一时,就是春盘小菜、两碗稀烂下饭、一碗\textHuoChuan 肉粉汤、两碗白米饭。希大独自一个,吃的里外干净,剩下些汁汤儿,还泡了碗吃了。玳安收下家活去。希大在旁看着两个打双陆。

只见两个唱的门首下了轿子,抬轿的提着衣裳包儿,笑进来。伯爵在窗里看见,说道:“两个小淫妇儿,这咱才来。”吩咐玳安:“且别教他往后边去,先叫他楼上来见我。”希大道:“今日叫的是那两个?”玳安道:“是董娇儿、韩玉钏儿。”忙下楼说道:“应二爹叫你说话。”两个那里肯来,一直往后走了。见了一丈青,拜了,引他入房中。看见王六儿头上戴着时样扭心\textuni{4BFC}髻儿,身上穿紫潞绸袄儿,玄色披袄儿、白挑线绢裙子,下边露两只金莲,拖的水鬓长长的,紫膛色,不十分搽铅粉,学个中人打扮,耳边带着丁香儿。进门只望着他拜了一拜,都在炕边头坐了。小铁棍拿茶来,王六儿陪着吃了。两个唱的,上上下下把眼只看他身上。看一回,两个笑一回,更不知是什么人。落后,玳安进来,两个悄悄问他道:“房里那一位是谁?”玳安没的回答,只说是:“俺爹大姨人家,接来看灯的。”两个听的,从新到房中说道:“俺每头里不知是大姨,没曾见的礼,休怪。”于是插烛磕了两个头。慌的王六儿连忙还下半礼。落后,摆上汤饭来,陪着同吃。两个拿乐器,又唱与王六儿听。

伯爵打了双陆,下楼来小解净手,听见后边唱,点手儿叫玳安,问道:“你告我说,两个唱的在后边唱与谁听?”玳安只是笑,不做声,说道:“你老人家曹州兵备——管事宽。唱不唱,管他怎的?”伯爵道:“好贼小油嘴,你不说,愁我不知道?”玳安笑道:“你老人家知道罢了,又问怎的?”说毕,一直往后走了。伯爵上的楼来,西门庆又与谢希大打了三贴双陆。只见李铭、吴惠两个蓦地上楼来磕头。伯爵道:“好呀!你两个来的正好,怎知道俺每在这里?”李铭跪下说道:“小的和吴惠先到宅里来,宅里说爹在这边摆酒。特来伏侍爹每。”西门庆道:“也罢,你起来伺候。玳安,快往对门请你韩大叔去。”不一时,韩道国到了,作了揖,坐下。一面放桌儿,摆上春盘案酒来,琴童在旁边筛酒。伯爵与希大居上,西门庆主位,韩道国打横,坐下把酒来筛;一面使玳安后边请唱的去。

少顷,韩玉钏儿、董娇儿两个,慢条斯礼上楼来。望上不当不正磕下头去。伯爵骂道:“我道是谁来,原来是这两个小淫妇儿。头里我叫着,怎的不先来见我?这等大胆!到明日,不与你个功德,你也不怕。”董娇儿笑道:“哥儿那里隔墙掠个鬼脸儿,可不把我唬杀!”韩玉钏儿道:“你知道,爱奴儿掇着兽头城往里掠——好个丢丑儿的孩儿!”伯爵道:“哥,你今日忒多余了。有了李铭、吴惠在这里唱罢了,又要这两个小淫妇做什么?还不趁早打发他去。大节夜,还赶几个钱儿,等住回晚了,越发没人要了。”韩玉钏儿道:“哥儿,你怎么没羞?大爹叫了俺每来答应,又不伏侍你,你怎的闲出气?”伯爵道:“傻小歪剌骨儿,你见在这里,不伏侍我,你说伏侍谁?”韩玉钏道:“唐胖子吊在醋缸里——把你撅酸了。”伯爵道:“贼小淫妇儿,是撅酸了我。等住回散了家去时,我和你答话。我左右有两个法儿,你原出得我手!”董娇儿问道:“哥儿,那两个法儿?说来我听。”伯爵道:“我头一个,是对巡捕说了,拿你犯夜,教他拿了去,拶你一顿好拶子。十分不巧,只消三分银子烧酒,把抬轿的灌醉了,随你这小淫妇儿去,天晚到家没钱,不怕鸨子不打。”韩玉钏道:“十分晚了,俺每不去,在爹这房子里睡。再不,叫爹差人送俺每,王妈妈支钱一百文,不在于你。好淡嘴女又十撇儿。”伯爵道:“我是奴才,如今年程反了,拿三道三。”说笑回,两个唱的在旁弹唱春景之词。

众人才拿起汤饭来吃,只见玳安儿走来,报道:“祝爹来了。”众人都不言语。不一时,祝实念上的楼来,看见伯爵和谢希大在上面,说道:“你两个好吃,可成个人。”因说:“谢子纯,哥这里请你,也对我说一声儿,三不知就走的来了,叫我只顾在粘梅花处寻你。”希大道:“我也是误行,才撞见哥在楼上和应二哥打双陆。走上来作揖,被哥留住了。”西门庆因令玳安儿:“拿椅儿来,我和祝兄弟在下边坐罢。”于是安放锺箸,在下席坐了。厨下拿了汤饭上来,一齐同吃。西门庆只吃了一个包儿,呷了一口汤,因见李铭在旁,都递与李铭下去吃了。那应伯爵、谢希大、祝实念、韩道国,每人吃一大深碗八宝攒汤,三个大包子,还零四个桃花烧卖,只留了一个包儿压碟儿。左右收下汤碗去,斟上酒来饮酒。希大因问祝实念道:“你陪他到那里才拆开了?怎知道我在这里?”祝实念如此这般告说:“我因寻了你一回寻不着,就同王三官到老孙家会了,往许不与先生那里,借三百两银子去,吃孙寡嘴老油嘴把借契写差了。”希大道:“你每休写上我,我不管。左右是你与老孙作保,讨保头钱使。”因问:“怎的写差了?”祝实念道:“我那等吩咐他,文书写滑着些,立与他三限才还。他不依我,教我从新把文书又改了。”希大道:“你立的是那三限?”祝实念道:“头一限,风吹辘轴打孤雁;第二限,水底鱼儿跳上岸;第三限,水里石头泡得烂。这三限交还他。”谢希大道:“你这等写着,还说不滑哩。”祝实念道:“你到说的好,倘或一朝天旱水浅,朝廷挑河,把石头吃做工的两三镢头砍得稀烂,怎了?那时少不的还他银子。”众人说笑了一回。

看看天晚,西门庆吩咐楼上点灯,又楼檐前一边一盏羊角玲灯,甚是奇巧。家中,月娘又使棋童儿和排军,抬送了四个攒盒,都是美口糖食、细巧果品。西门庆叫棋童儿问道:“家中众奶奶们散了不曾?谁使你送来?”棋童道:“大娘使小的来,与爹这边下酒。众奶奶们还未散哩。戏文扮了四折,大娘留在大门首吃酒,看放烟火哩。”西门庆问:“有人看没有?”棋童道:“挤围着满街人看。”西门庆道:“我吩咐留下四名青衣排军,拿杆栏拦人伺候,休放闲杂人挨挤。”棋童道:“小的与平安儿两个,同排军都看放了烟火,并没闲杂人搅扰。”西门庆听了,吩咐把桌上饮馔都搬下去,将攒盒摆上,厨下又拿上一道果馅元宵来。两个唱的在席前递酒。西门庆吩咐棋童回家看去。一面重筛美酒,再设珍羞,叫李铭、吴惠席前弹唱了一套灯词。唱毕,吃了元宵,韩道国先往家去了。少顷,西门庆吩咐来昭将楼下开下两间,吊挂上帘子,把烟火架抬出去。西门庆与众人在楼上看,教王六儿陪两个粉头和一丈青在楼下观看。玳安和来昭将烟火安放在街心里。须臾,点着。那两边围看的,挨肩擦膀,不知其数。都说西门大官府在此放烟火,谁人不来观看?果然扎得停当好烟火。但见:

\[
一丈五高花桩,四周下山棚热闹。最高处一只仙鹤,口里衔着一封丹书,乃是一枝起火,一道寒光,直钻透斗牛边。然后,正当中一个西瓜炮迸开,四下里人物皆着,觱剥剥万个轰雷皆燎彻。彩莲舫,赛月明,一个赶一个,犹如金灯冲散碧天星;紫葡萄,万架千株,好似骊珠倒挂水晶帘。霸玉鞭,到处响亮;地老鼠,串绕人衣。琼盏玉台,端的旋转得好看;银蛾金弹,施逞巧妙难移。八仙捧寿,名显中通;七圣降妖,通身是火。黄烟儿,绿烟儿,氤氲笼罩万堆霞;紧吐莲,慢吐莲,灿烂争开十段锦。一丈菊与烟兰相对,火梨花共落地桃争春。楼台殿阁,顷刻不见巍峨之势;村坊社鼓,仿佛难闻欢闹之声。货郎担儿,上下光焰齐明;鲍老车儿,首尾迸得粉碎。五鬼闹判,焦头烂额见狰狞;十面埋伏,马到人驰无胜负。总然费却万般心,只落得火灭烟消成煨烬。
\]

应伯爵见西门庆有酒了,刚看罢烟火下楼来,因见王六儿在这里,推小净手,拉着谢希大、祝实念,也不辞西门庆就走了。玳安便道:“二爹那里去?”伯爵向他耳边说道:“傻孩子,我头里说的那本帐,我若不起身,别人也只顾坐着,显的就不趣了。等你爹问,你只说俺每都跑了。”落后,西门庆见烟火放了,问伯爵等那里去了,玳安道:“应二爹和谢爹都一路去了。小的拦不回来,多上覆爹。”西门庆就不再问了。因叫过李铭、吴惠来,每人赏了一大巨杯酒与他吃。吩咐:“我且不与你唱钱,你两个到十六日早来答应。还是应二爹三个并众伙计当家儿,晚夕在门首吃酒。”李铭跪下道:“小的告禀爹:十六日和吴惠、左顺、郑奉三个,都往东平府,新升的胡爷那里到任,官身去,只到后晌才得来。”西门庆道:“左右俺每晚夕才吃酒哩。你只休误了就是了。”二人道:“小的并不敢误。”两个唱的也就来拜辞出门。西门庆吩咐:“明日,家中堂客摆酒,李桂姐、吴银姐都在这里,你两个好歹来走一走。”二人应诺了,一同出门,不在话下。西门庆吩咐来昭、玳安、琴童收家活。灭息了灯烛,就往后边房里去了。

且说来昭儿子小铁棍儿,正在外边看放了烟火,见西门庆进去了,就来楼上。见他爹老子收了一盘子杂合的肉菜、一瓯子酒和些元宵,拿到屋里,就问他娘一丈青讨,被他娘打了两下。不防他走在后边院子里顽耍,只听正面房子里笑声,只说唱的还没去哩,见房门关着,就在门缝里张看,见房里掌着灯烛。原来西门庆和王六儿两个,在床沿子上行房。西门庆已有酒的人,把老婆倒按在床沿上,褪去小衣,那话上使着托子干后庭花。一进一退往来\textShan 打,何止数百回,\textShan 打的连声响亮,其喘息之声,往来之势,犹赛折床一般,无处不听见。这小孩子正在那里张看,不防他娘一丈青走来看见,揪着头角儿拖到前边,凿了两个栗爆,骂道:“贼祸根子,小奴才儿,你还少第二遭死?又往那里张他去!”于是,与了他几个元宵吃了,不放他出来,就唬住他上炕睡了。西门庆和老婆足干捣有两顿饭时才了事。玳安打发抬轿的酒饭吃了,跟送他到家,然后才来同琴童两个打着灯儿跟西门庆家去。正是:

\[
不愁明月尽,自有夜珠来。
\]

\newpage
%# -*- coding:utf-8 -*-
%%%%%%%%%%%%%%%%%%%%%%%%%%%%%%%%%%%%%%%%%%%%%%%%%%%%%%%%%%%%%%%%%%%%%%%%%%%%%%%%%%%%%


\chapter{争宠爱金莲惹气\KG 卖富贵吴月攀亲}


词曰:

\[
情怀增怅望,新欢易失,往事难猜。问篱边黄菊,知为谁开?谩道愁须滞酒,酒未醒、愁已先回。凭栏久,金波渐转,白露点苍苔。
\]

话说西门庆归家,已有三更时分,吴月娘还未睡,正和吴大妗子众人说话,李瓶儿还伺候着与他递酒。大妗子见西门庆来家,就过那边去了。月娘见他有酒了,打发他脱了衣裳。只教李瓶儿与他磕了头,同坐下,问了回今日酒席上话。玉箫点茶来吃。因有大妗子在,就往孟玉楼房中歇了。

到次日,厨役早来收拾酒席。西门庆先到衙门中拜牌,大发放。夏提刑见了,致谢日昨房下厚扰之意。西门庆道:“日昨甚是简慢。恕罪,恕罪!”来家早有乔大户家使孔嫂儿引了乔五太太家人送礼来了。西门庆收了,家人管待酒饭。孔嫂儿进月娘房里坐的。吴舜臣媳妇儿郑三姐轿子也先来了,拜了月娘众人,都坐着吃茶。

正值李智、黄四关了一千两香蜡银子,贲四从东平府押了来家。应伯爵打听得知,亦走来帮扶交纳。西门庆令陈敬济拿天平在厅上兑明白,收了。黄四又拿出四锭金镯儿来,重三十两,算一百五十两利息之数,还欠五百两,就要捣换了合同。西门庆吩咐二人:“你等过灯节再来计较。我连日家中有事。”那李智、黄四,老爷长,老爷短,千恩万谢出门。应伯爵因记挂着二人许了他些业障儿,趁此机会好问他要,正要跟随同去,又被西门庆叫住说话。因问:“昨日你每三个,怎的三不知就走了?”伯爵道:“昨日甚是深扰哥,本等酒多了。我见哥也有酒了,今日嫂子家中摆酒,一定还等哥说话。俺每不走了,还只顾缠到多咱?我猜哥今日也没往衙门里去,本等连日辛苦。”西门庆道:“我昨日来家,已有三更天气。今日还早到衙门拜了牌,坐厅大发放,理了回公事。如今家中治料堂客之事。今日观里打上元醮,拈了香回来,还赶往周菊轩家吃酒去,不知到多咱才得到家。”伯爵道:“亏哥好神思,你的大福。不是面奖,若是第二个也成不的。”两个说了一回,西门庆要留伯爵吃饭,伯爵道:“我不吃饭,去罢。”西门庆又问:“嫂子怎的不来?”伯爵道:“房下轿子已叫下了,便来也。”举手作辞出门,一直赶黄四、李智去了。正是:

\[
假饶驾雾腾云术,取火钻冰只要钱。
\]

西门庆打发伯爵去了,手中拿着黄烘烘四锭金镯儿,心中甚是可爱,口中不言,心里暗道:“李大姐生的这孩子,甚是脚硬,一养下来,我平地就得些官。我今日与乔家结亲,又进这许多财。”于是用袖儿抱着那四锭金镯儿,也不到后边,径往李瓶儿房里来。正走到潘金莲角门首,只见金莲出来看见,叫他问道:“你手里托的是什么东西儿?过来我瞧瞧。”那西门庆道:“等我回来与你瞧。”托着一直往李瓶儿那边去了。金莲见叫不回他来,心中就有几分羞讪,说道:“什么罕稀货,忙的这等唬人子剌剌的!不与我瞧罢,贼跌折腿的三寸货强盗,进他门去,一齐的把那两条腿\textuni{22C49}折了,才现报了我的眼。”

却说西门庆拿着金子,走入李瓶儿房里,见李瓶儿才梳了头,奶子正抱着孩子顽耍。西门庆一径把四个金镯儿抱着,教他手儿挝弄。李瓶儿道:“是那里的?只怕冰了他手。”西门庆道:“是李智、黄四今日还银子准折利钱的。”李瓶儿生怕冰着他,取了一方通花汗巾儿,与他裹着耍子。只见玳安走来说道:“云伙计骑了两匹马来,在外边请爹出去瞧。”西门庆问道:“云伙计他是那里的马?”玳安道:“他说是他哥云参将边上捎来的。”正说着,只见后边李娇儿、孟玉楼陪着大妗子并他媳妇郑三姐,都来李瓶儿房里看官哥儿。西门庆丢了那四锭金子,就往外边看马去了。

李瓶儿见众人来到,只顾与众人见礼让坐,也就忘记了孩子拿着这金子,弄来弄去,少了一锭。只见奶子如意儿问李瓶儿道:“娘没曾收哥哥儿耍的那锭金子?怎只三锭,少了一锭了?”李瓶儿道:“我没曾收,我把汗个子替他裹着哩。”如意儿道:“汗巾子也落在地下了。那里得那锭金子?”屋里就乱起来。奶子问迎春,迎春就问老冯。老冯道:“耶嚛,耶嚛!我老身就瞎了眼,也没看见。老身在这里恁几年,莫说折针断线我不敢动,娘他老人家知道我,就是金子,我老身也不爱。你每守着哥儿,怎的冤枉起我来了!”李瓶儿笑道:“你看这妈妈子说混话,这里不见的,不是金子却是什么?”又骂迎春:“贼臭肉!平白乱的是些甚么?等你爹进来,等我问他,只怕是你爹收了。怎的只收一锭儿?”孟玉楼问道:“是那里金子?”李瓶儿道:“是他爹拿来的,与孩子耍。谁知道是那里的。”

且说西门庆在门首看马,众伙计家人都在跟前,叫小厮来回溜了两趟。西门庆道:“虽是东路来的马,鬃尾丑,不十分会行,论小行也罢了。”因问云伙计道:“此马你令兄那里要多少银子?”云离守道:“两匹只要七十两。”西门庆道:“也不多。只是不会行,你还牵了去,另有好马骑来,倒不说银子。”说毕,西门庆进来,只见琴童来说:“六娘房里请爹哩。”于是走入李瓶儿房里来。李瓶儿问他:“金子你收了一锭去了?如何只三锭在这里?”西门庆道:“我丢下,就外边去看马,谁收来!”李瓶儿道:“你没收,却往那里去了?寻了这一日没有。奶子推老冯,急的那老冯赌身罚咒,只是哭。”西门庆道:“端的是谁拿了,由他慢慢儿寻罢。”李瓶儿道:“头里因大妗子女儿两个来,乱着就忘记了。我只说你收了出去,谁知你也没收,就两耽了。才寻起来,唬的他们都走了。”于是把那三锭,还交与西门庆收了。正值贲四倾了一百两银子来交,西门庆就往后边收兑银子去了。

且说潘金莲听见李瓶儿这边嚷,不见了孩子耍的一锭金镯子,得不的风儿就是雨儿,就先走来房里,告月娘说:“姐姐,你看三寸货干的营生!随你家怎的有钱,也不该拿金子与孩子耍。”月娘道:“刚才他每告我说,他房里不见了金镯子,端的不知是那里的?”金莲道:“谁知他是那里的!你还没见,他头里从外边拿进来,用袄子袖儿裹着,恰似八蛮进宝的一般。我问他是什么,拿过来我瞧瞧。头儿也不回,一直奔命往屋里去了。迟了一回,反乱起来,说不见了一锭金子。干净就是他学三寸货,说不见了,由他慢慢儿寻罢。你家就是王十万也使不的。一锭金子,至少重十到两,也值五六十两银子,平白就罢了?瓮里走了鳖——左右是他家一窝子。再有谁进他屋里去?”正说着,只见西门庆进来,兑收贲四倾的银子,把剩的那三锭金子交与月娘收了。因告诉月娘:“此是李智、黄四还的四锭金子,拿了与孩子耍了耍,就不见了一锭。”吩咐月娘:“你与我把各房里丫头叫出来审问审问。我使小厮街上买狼筋去了,早拿出来便罢,不然,我就叫狼筋抽起来。”月娘道:“论起来,这金子也不该拿与孩子,沉甸甸冰着他,一时砸了他手脚怎了!”潘金莲在旁接过来说道:“不该拿与孩子耍?只恨拿不到他屋里。头里叫着,想回头也怎的,恰似红眼军抢将来的,不教一个人儿知道。这回不见了金子,亏你怎么有脸儿来对大姐姐说!叫大姐姐替你查考各房里丫头,叫各房里丫头口里不笑,\textuni{23B48}眼里也笑!”

几句说的西门庆急了,走向前把金莲按在月娘炕上,提起拳来,骂道:“狠杀我罢了!不看世界面上,把你这小\textuni{22C49}剌骨儿,就一顿拳头打死了!单管嘴尖舌快的,不管你事也来插一脚。”那潘金莲就假做乔妆,哭将起来,说道:“我晓的你倚官仗势,倚财为主,把心来横了,只欺负的是我,你说你这般威势,把一个半个人命儿打死了,不放在意里。那个拦着你手儿哩不成?你打不是的!我随你怎么打,难得只打得有这口气儿在着,若没了,愁我家那病妈妈子不问你要人!随你家怎么有钱有势,和你家一递一状。你说你是衙门里千户便怎的?无故只是个破纱帽债壳子——穷官罢了,能禁的几个人命?就不是教皇帝敢杀下人也怎么!”几句说的西门庆反呵呵笑了,说道:“你看这小\textuni{22C49}剌骨儿,这等刁嘴!我是破纱帽穷官?教丫头取我的纱帽来,我这纱帽那块儿破?这清河县问声,我少谁家银子?你说我是债壳子!”金莲道:“你怎的叫我是\textuni{22C49}剌骨来!”因跷起一只脚来,“你看老娘这脚,那些儿放着歪?你怎骂我是\textuni{22C49}剌骨?”月娘在旁笑道:“你两个铜盆撞了铁刷帚。常言:恶人自有恶人磨,见了恶人没奈何!自古嘴强的争一步。六姐,也亏你这个嘴头子,不然,嘴钝些儿也成不的。”

那西门庆见奈何不过他,穿了衣裳往外去了。迎见玳安来说:“周爷家差人邀来了。请问爹先往打醮处去,往周爷家去?”西门庆吩咐:“打醮处,教你姐夫去罢。伺候马,我往你周爷家吃酒去就是了。”只见王皇亲家扮戏两个师父率众过来,与西门庆叩头,西门庆教书童看饭与他吃,说:“今日你等用心伏侍众奶奶,我自有重赏,休要上边打箱去!”那师父跪下说道:“小的每若不用心答应,岂敢讨赏!”西门庆因吩咐书童:“他唱了两日,连赏赐封下五两银子赏他。”书童应诺。西门庆就上马往周守备家吃酒去了。

单表潘金莲在上房坐的,吴月娘便说:“你还不往屋里匀匀那脸去!揉的恁红红的。等住回人来看着甚么张致!谁叫你惹他来?我倒替你捏两把汗。若不是我在跟前劝着,绑着鬼,是也有几下子打在身上。汉子家脸上有狗毛,不知好歹,只顾下死手的和他缠起来了。不见了金子,随他不见去,寻不寻不在你,又不在你屋里不见了,平白扯着脖子和他强怎么!你也丢了这口气儿罢!”几句说的金莲闭口无言,往屋里匀脸去了。

不一时,李瓶儿和吴银儿都打扮出来,到月娘房里。月娘问他:“金子怎的不见了?刚才惹他爹和六姐两个,在这里好不辨了这回嘴,差些儿没曾辨恼了打起来!吃我劝开了。他爹就往人家吃酒去了。吩咐小厮买狼筋去了。等他晚上来家,要把各房丫头抽起来。你屋里丫头老婆管着那一门儿来?看着孩子耍,便不见了他一锭金子。是一个半个钱的东西儿也怎的?”李瓶儿道:“平白他爹拿进四锭金子来与孩子耍,我乱着陪大妗子和郑三姐并他二娘坐着说话,谁知就不见了一锭。如今丫头推奶子,奶子推老冯。急的冯妈妈哭哭啼啼,只要寻死。无眼难明勾当,如今冤谁的是?”吴银儿道:“天么,天么!每常我还和哥儿耍子,早是今日我在这边屋里梳头,没曾过去。不然怎了?虽然爹娘不言语,你我心上何安!谁人不爱钱?俺里边人家,最忌叫这个名声儿,传出去丑听!”

正说着,只见韩玉钏儿、董娇儿两个提着衣包儿进来,笑嘻嘻先向月娘、大妗子、李瓶儿磕了头,起来望着吴银儿拜了一拜,说道:“银姐昨日没家去?”吴银儿道:“你怎的晓得?”董娇儿道:“昨日,俺两个都在灯市街房子里唱来,大爹对俺们说,教俺今日来伏侍奶奶。”一面月娘让他两个坐下。须臾,小玉拿了两盏茶来。那韩玉钏儿、董娇儿连忙立起身来接茶,还望小玉拜了一拜。吴银儿因问:“你两个昨日唱多咱散了?”韩玉钏道:“俺们到家,也有二更多了,同你兄弟吴惠都一路去的。”说了一回话,月娘吩咐玉箫:“早些打发他们吃了茶罢。等住回只怕那边人来忙了。”一面放下桌儿,两方春槅、四盒茶食。月娘使小玉:“你二娘房里,请了桂姐来同吃了茶罢。”不一时,和他姑娘来到,两个各道了礼数坐下,同吃了茶,收过家活去。

忽见迎春打扮着,抱了官哥儿来,头上戴了金梁缎子八吉祥帽儿,身穿大红氅衣儿,下边白绫袜儿、缎子鞋儿,胸前项牌符索,手上小金镯儿。李瓶儿看见说道:“小大官儿,没人请你,来做什么?”一面接过来,放在膝盖上。看见一屋里人,把眼不住的看了这个,又看那个。桂姐坐在月娘炕上,笑引逗他耍子,道:“哥子只看着这里,想必要我抱他。”于是用手引了他引儿,那孩子就扑到怀里教他抱。吴大妗子笑道:“恁点小孩儿,他也晓的爱好!”月娘接过来说:“他老子是谁!到明日大了,管情也是小嫖头儿。”孟玉楼道:“若做了小嫖头儿,叫大妈妈就打死了。”李瓶儿道:“小厮,你姐姐抱,只休溺了你姐姐衣服,我就打死了!”桂姐道:“耶嚛!怕怎么?溺了也罢,不妨事。我心里要抱哥儿耍耍儿。”于是与他两个嘴揾嘴儿耍子。董娇儿、韩玉钏儿说道:“俺两个来了这一日,还没曾唱个儿与娘每听。”因取乐器,韩玉钏儿琵琶,董娇儿弹筝,吴银儿也在旁边陪唱。唱了一套“繁华满月开”《金索挂梧桐》。唱出一句来,端的有落尘绕梁之声,裂石流云之响,把官哥儿唬的在桂姐怀里只磕倒着,再不敢抬头出气儿。月娘看见,便叫:“李大姐,你接过孩子来,教迎春抱到屋里去罢。好个不长进的小厮,你看唬的那脸儿!”这李瓶儿连忙接过来,叫迎春掩着他耳朵,抱的往那边房里去了。

四个唱的正唱着,只见玳安进来,说道:“小的到乔亲家娘那边邀来,朱奶奶、尚举人娘子,都过乔亲家来了,只等着乔五太太到了就来了。大门前边、大厅上,都有鼓乐迎接。娘每都收拾伺候就是了。”月娘又吩咐后厅明间铺下锦毯,安放坐位。卷起帘来,金钩双控,兰麝香飘。春梅、迎春、玉箫、兰香,都打扮起来。家人媳妇都插金戴银,披红垂绿,准备迎接新亲。只见应伯爵娘子应二嫂先到了,应保跟着轿子。月娘等迎接进来。见了礼数,明间内坐下,向月娘拜了又拜,说:“俺家的常时打搅,多蒙看顾!”月娘道:“二娘,好说!常时累你二爹。”良久,只闻喝道之声渐近,前厅鼓乐响动。平安儿先进来报道:“乔太太轿子到了!”须臾,黑压压一群人,跟着五顶大轿落在门首。惟乔五太太轿子在头里,轿上是垂珠银顶、天青重沿、绡金走水轿衣,使藤棍喝路。后面家人媳妇坐小轿跟随,四名校尉抬衣箱、火炉,两个青衣家人骑着小马,后面随从。其余就是乔大户娘子、朱台官娘子、尚举人娘子、崔大官媳妇、段大姐,并乔通媳妇也坐着一顶小轿,跟来收叠衣裳。

吴月娘与李娇儿、孟玉楼、潘金莲、李瓶儿、孙雪娥,一个个打扮的似粉妆玉琢,锦绣耀目,都出二门迎接。众堂客簇拥着乔五太太进来。生的五短身材,约七旬年纪,戴着叠翠宝珠冠,身穿大红宫绣袍儿,近面视之,鬓发皆白。正是:眉分八道雪,髻绾一窝丝,眼如秋水微浑,鬓似楚山云淡。接入后厅,先与吴大妗子叙毕礼数,然后与月娘等厮见。月娘再三请太太受礼,太太不肯,让了半日,受了半礼。次与乔大户娘子,又叙其新亲家之礼,彼此道及款曲,谢其厚仪。已毕,然后向锦屏正面设放一张锦裀座位,坐了乔五太太,其次就让乔大户娘子。乔大户娘子再三辞说:“侄妇不敢与五太太上僭。”让朱台官、尚举人娘子,两个又不肯。彼此让了半日,乔五太太坐了首座,其余客东主西,两分头坐了。当中大方炉火厢笼起火来,堂中气暖如春。春梅、迎春、玉箫、兰香,一般儿四个丫头,都打扮起来,在跟前递茶。

良久,乔五太太对月娘说:“请西门大人出来拜见,叙叙亲情之礼。”月娘道:“拙夫今日衙门中去了,还未来家哩!”乔五太太道:“大人居于何官?”月娘道:“乃一介乡民,蒙朝廷恩例,实授千户之职,见掌刑名。寒家与亲家那边结亲,实是有玷。”乔五太太道:“娘子说那里话,似大人这等峥嵘也彀了。昨日老身听得舍侄妇与府上做亲,心中甚喜。今日我来会会,到明日好厮见。”月娘道:“只是有玷老太太名目。”乔五太太道:“娘子是甚说话,想朝廷不与庶民做亲哩!老身说起来话长,如今当今东宫贵妃娘娘,系老身亲侄女儿。他父母都没了,止有老身。老头儿在时,曾做世袭指挥使,不幸五十岁故了。身边又无儿孙,轮着别门侄另替了,手里没钱,如今倒是做了大户。我这个侄儿,虽是差役立身,颇得过的日子,庶不玷污了门户。”说了一回,吴大妗子对月娘说:“抱孩子出来与老太太看看,讨讨寿。”李瓶儿慌吩咐奶子,抱了官哥来与太太磕头。乔太太看了夸道:“好个端正的哥哥!”即叫过左右,连忙把毡包内打开,捧过一端宫中紫闪黄锦缎,并一副镀金手镯,与哥儿戴。月娘连忙下来拜谢了。请去房中换了衣裳。须臾,前边卷棚内安放四张桌席摆茶,每桌四十碟,都是各样茶果、细巧油酥之类。吃了茶,月娘就引去后边山子花园中,游玩了一回下来。

那时,陈敬济打醮去,吃了午斋回来了。和书童儿、玳安儿,又早在前厅摆放桌席齐整,请众奶奶每递酒上席。端的好筵席,但见:

\[
屏开孔雀,褥隐芙蓉。盘堆异果奇珍,瓶插金花翠叶。炉焚兽炭,香袅龙涎。白玉碟高堆麟脯,紫金壶满贮琼浆。梨园子弟,簇捧着凤管鸾箫;内院歌姬,紧按定银筝象板。进酒佳人双洛浦,分香侍女两姮娥。正是:两行珠翠列阶前,一派笙歌临坐上。
\]

吴月娘与李瓶儿同递酒,阶下戏子鼓乐响动。乔太太与众亲戚,又亲与李瓶儿把盏祝寿,方入席坐下。李桂姐、吴银儿、韩玉钏儿、董娇儿四个唱的,在席前唱了一套“寿比南山”。戏子呈上戏文手本,乔五太太吩咐下来,教做《王月英元夜留鞋记》。厨役上来献小割烧鹅,赏了五钱银子。比及割凡五道,汤陈三献,戏文四折下来,天色已晚。堂中画烛流光,各样花灯都点起来,锦带飘飘,彩绳低转。一轮明月从东而起,照射堂中灯光掩映。乐人又在阶下,琵琶筝\textuni{25C67},笙箫笛管,吹打了一套灯词《画眉序》“花月满香城”。吹打毕,乔太太和乔大户娘子叫上戏子,赏了两包一两银子,四个唱的,每人二钱。月娘又在后边明间内,摆设下许多果碟儿,留后坐。四张桌子都堆满了。唱的唱,弹的弹,又吃了一回酒。乔太太再三说晚了,要起身。月娘众人款留不住,送在大门首,又拦门递酒,看放烟火。两边街上,看的人鳞次蜂排一般。平安儿同众排军执棍拦挡再三,还涌挤上来。须臾,放了一架烟火,两边人散了。乔太大和众娘子方才拜辞月娘等,起身上轿去了。那时也有三更天气,然后又送应二嫂起身。月娘众姐妹归到后边来,吩咐陈敬济、来兴、书童、玳安儿,看着厅上收拾家活,管待戏子并两个师范酒饭,与了五两银子唱钱,打发去了。

月娘吩咐出来,剩攒下一桌肴馔、半罐酒,请傅伙计、贲四、陈姐夫,说:“他每管事辛苦,大家吃锺酒。就在大厅上安放一张桌儿,你爹不知多咱才回。”于是还有残灯未尽,当下傅伙计、贲四、敬济、来保上坐,来兴、书童、玳安、平安打横,把酒来斟。来保叫平安儿:“你还委个人大门首,怕一时爹回,没人看门。”平安道:“我叫画童看着哩,不妨事。”于是八个人猜枚饮酒。敬济道:“你每休猜枚,大惊小怪的,惹后边听见。咱不如悄悄行令儿耍子。每人要一句,说的出免罚,说不出罚一大杯。”该傅伙计先说:“堪笑元宵草物。”贲四道:“人生欢乐有数。”敬济道:“趁此月色灯光。”来保道:“咱且休要辜负。”来兴道:“才约娇儿不在。”书童道:“又学大娘吩咐。”玳安道:“虽然剩酒残灯。”平安道:“也是春风一度。”众人念毕,呵呵笑了。正是:

\[
饮罢酒阑人散后,不知明月转花梢。
\]

\newpage
%# -*- coding:utf-8 -*-
%%%%%%%%%%%%%%%%%%%%%%%%%%%%%%%%%%%%%%%%%%%%%%%%%%%%%%%%%%%%%%%%%%%%%%%%%%%%%%%%%%%%%


\chapter{避马房侍女偷金\KG 下象棋佳人消夜}


词曰:

\[
昼日移阴,揽衣起、春帏睡足。临宝鉴、绿鬟缭乱,未敛装束。蝶粉蜂黄浑褪了,枕痕一线红生玉。背画阑、脉脉悄无言,寻棋局。
\]

话说敬济众人,同傅伙计前边吃酒,吴大妗子轿子来了,收拾要家去。月娘款留再三,说道:“嫂子再住一夜儿,明日去罢。”吴大妗子道:“我连在乔亲家那里,就是三四日了。家里没人,你哥衙里又有事,不得在家,我去罢。明日请姑娘众位,好歹往我那里坐坐,晚夕走百病儿家来。”月娘道:“俺们明日,只是晚上些去罢了。”吴大妗子道:“姑娘早些坐轿子去,晚夕同走了来家就是了。”说毕,装了一盒子元宵,一盒子馒头,叫来安儿送大妗子到家。李桂姐等四个都磕了头,拜辞月娘,也要家去。月娘道:“你们慌怎的?也就要去,还等你爹来家。他吩咐我留下你们,只怕他还有话和你们说,我是不敢放你去。”桂姐道:“爹去吃酒,到多咱晚来家?俺们怎等的他!娘先教我和吴银姐去罢。他两个今日才来,俺们来了两日,妈在家还不知怎么盼望!”月娘道:“可可的就是你妈盼望,这一夜儿等不的?”李桂姐道:“娘且是说的好,我家里没人,俺姐姐又被人包住了。宁可拿乐器来,唱个与娘听,娘放了奴去罢。”正说着,只见陈敬济走进来,交剩下的赏赐,说道:“乔家并各家贴轿赏一钱,共使了十包,重三两。还剩下十包在此。”月娘收了。桂姐便道:“我央及姑夫,你看外边俺们的轿子来了不曾?”敬济道:“只有他两个的轿子。你和银姐的轿子没来。从头里不知谁回了去了。”桂姐道:“姑夫,你真个回了?你哄我哩!”那陈敬济道:“你不信,瞧去不是!我不哄你。”刚言未罢,只见琴童抱进毡包来,说:“爹家来了!”月娘道:“早是你们不曾去,这不你爹来了。”

不一时,西门庆进来,已带七八分酒了。走入房中,正面坐下,董娇儿、韩玉钏儿二人向前磕头。西门庆问月娘道:“人都散了,怎的不教他唱?”月娘道:“他们在这里求着我,要家去哩。”西门庆向桂姐说:“你和银儿亦发过了节儿去。且打发他两个去罢。”月娘道:“如何?我说你们不信,恰象我哄你一般。”那桂姐把脸儿苦低着,不言语。西门庆问玳安:“他两个轿子在这里不曾?”玳安道:“只有董娇儿、韩玉钏儿两顶轿子伺候着哩。”西门庆道:“我也不吃酒了。你们拿乐器来,唱《十段锦儿》我听。打发他两个先去罢。”当下四个唱的,李桂姐弹琵琶,吴银儿弹筝,韩玉钏儿拨阮,董娇儿打着紧急鼓子,一递一个唱《十段锦》“二十八半截儿”。吴月娘、李娇儿、孟玉楼、潘金莲、李瓶儿都在屋里坐的听唱。

唱毕,西门庆与了韩玉钏、董娇儿两个唱钱,拜辞出门。“留李桂姐、吴银儿两个,这里歇罢。”忽听前边玳安儿和琴童儿两个嚷乱,簇拥定李娇儿房里夏花儿进来,禀西门庆说道:“小的刚送两个唱的出去,打灯笼往马房里拌草,牵马上槽,只见二娘房里夏花儿,躲在马槽底下,唬了小的一跳。不知甚么缘故,小的每问着他,又不说。”西门庆听见,就出外边明间穿廊下椅子上坐着,一面叫琴童儿把那丫头揪着跪下。西门庆问他:“往前边做甚么去?那丫头不言语。李娇儿在旁边说道:“我又不使你,平白往马房里做甚么去?”见他慌做一团,西门庆只说丫头要走之情,即令小厮搜他身上。琴童把他拉倒在地,只听滑浪一声,从腰里掉下一件东西来。西门庆问:“是甚么?”玳安递上去,可霎作怪,却是一锭金子。西门庆灯下看了,道:“是头里不见了的那锭金子。原来是你这奴才偷了。”他说:“是拾的。”西门庆问:“是那里拾的?”他又不言语。西门庆心中大怒,令琴童往前边取拶子来,把丫头拶起来,拶的杀猪也似叫。拶了半日,又敲二十敲。月娘见他有酒了,又不敢劝。那丫头挨忍不过,方说:“我在六娘房里地下拾的。”西门庆方命放了拶子,又吩咐与李娇儿领到屋里去:“明日叫媒人即时与我卖了这奴才,还留着做甚么!”李娇儿没的话说,便道:“恁贼奴才,谁叫你往前头去来?三不知就出去了。你就拾了他屋里金子,也对我说一声儿!”那夏花儿只是哭。李娇儿道:“拶死你这奴才才好哩,你还哭!”西门庆道罢,把金子交与月娘收了,就往前边李瓶儿房里去了。

月娘令小玉关上仪门,因叫玉箫问:“头里这丫头也往前边去来么?”小玉道:“二娘、三娘陪大妗子娘儿两个,往六娘那边去,他也跟了去来。谁知他三不知就偷了这锭金子在手里。头里听见娘说,爹使小厮买狼筋去了,唬的他要不的,在厨房里问我:‘狼筋是甚么?’教俺每众人笑道:‘狼筋敢是狼身上的筋,若是那个偷了东西,不拿出来,把狼筋抽将出来,就缠在那人身上,抽攒的手脚儿都在一处!’他见咱说,想必慌了,到晚夕赶唱的出去,就要走的情,见大门首有人,才藏入马坊里。不想被小厮又看见了。”月娘道:“那里看人去!恁小丫头原来这等贼头鼠脑的,就不是个台孩的。”

且说李娇儿领夏花儿到房里,李桂姐甚是说夏花儿:“你原来是个傻孩子!你恁十五六岁,也知道些人事儿,还这等懵懂!要着俺里边,才使不的。这里没人,你就拾了些东西,来屋里悄悄交与你娘。就弄出来,他在旁边也好救你。你怎的不望他题一字儿?刚才这等拶打着好么?干净傻丫头!常言道:穿青衣,抱黑柱。你不是他这屋里人,就不管你。刚才这等掠掣着你,你娘脸上有光没光?”又说他姑娘:“你也忒不长俊,要是我,怎教他把我房里丫头对众拶恁一顿拶子!有不是,拉到房里来,等我打。前边几房里丫头怎的不拶,只拶你房里丫头!你是好欺负的,就鼻子口里没些气儿?等不到明日,真个教他拉出这丫头去罢,你也就没句话儿说?你不说,等我说。休教他领出去,教别人笑话。你看看孟家的和潘家的,两个就是狐狸一般,你怎斗的他过!”因叫夏花儿过来,问他:“你出去不出去?”那丫头道:“我不出去。”桂姐道:“你不出去,今后要贴你娘的心。凡事要你和他一心一计。不拘拿了甚么,交付与他。也似元宵一般抬举你。”那夏花儿说:“姐吩咐,我知道了。”按下这里教唆夏花儿不题。

且说西门庆走到前边李瓶儿房里,只见李瓶儿和吴银儿炕上做一处坐的,心中就要脱衣去睡。李瓶儿道:“银姐在这里,没地方儿安插你,且过一家儿罢。”西门庆道:“怎的没地方儿?你娘儿两个在两边,等我在当中睡就是。”李瓶儿便瞅他一眼儿道:“你就说下道儿去了。”西门庆道:“我如今在那里睡?”李瓶儿道:“你过六姐那边去睡一夜罢。”西门庆坐了一回,起身说道:“也罢,也罢!省的我打搅你娘儿们,我过那边屋里睡去罢。”于是一直走过金莲这边来。金莲听见西门庆进房来,天上落下来一般,向前与他接衣解带,铺陈床铺,展放鲛绡,吃了茶,两个上床歇宿不题。

李瓶儿这里打发西门庆出来,和吴银儿两个灯下放炕桌儿,摆下棋子,对坐下象棋儿。吩咐迎春:“拿个果盒儿,把甜金华酒筛下一壶儿来,我和银姐吃。”因问:“银姐,你吃饭?教他盛饭来你吃。”吴银儿道:“娘,我不饿,休叫姐盛来。”李瓶儿道:“也罢。银姐不吃饭,你拿个盒盖儿,我拣妆里有果馅饼儿,拾四个儿来与银姐吃罢。”须臾,迎春都拿了,放在旁边。李瓶儿与吴银儿下了三盘棋,筛上酒来,拿银锺儿两个共饮。吴银儿叫迎春:“姐,你递过琵琶来,我唱个曲儿与娘听。”李瓶儿道:“姐姐不唱罢,小大官儿睡着了,他爹那边又听着,教他说。咱掷骰子耍耍罢。”于是教迎春递过色盆来,两个掷骰儿赌酒为乐。掷了一回,吴银儿因叫迎春:“姐,你那边屋里请过奶妈儿来,教他吃锺酒儿。”迎春道:“他搂着哥儿在那边炕上睡哩。”李瓶儿道:“教他搂着孩子睡罢。拿一瓯子酒,送与他吃就是了。你不知俺这小大官好不伶俐,人只离开他就醒了。有一日儿,在我这边炕上睡,他爹这里略动一动儿,就睁开眼醒了,恰似知道的一般。教奶子抱了去那边屋里,只是哭,只要我搂着他。”吴银儿笑道:“娘有了哥儿,和爹自在觉儿也不得睡一个儿。爹几日来这屋里走一遭儿?”李瓶儿道:“他也不论,遇着一遭也不可知,两遭也不可知。常进屋里,为这孩子,来看不打紧,教人把肚子也气破了。将他爹和这孩子背地咒的白湛湛的。我是不消说的,只与人家垫舌根。谁和他有甚么大闲事?宁可他不来我这里还好。第二日教人眉儿眼儿,只说俺们把拦汉子。象刚才到这屋里,我就撺掇他出去。银姐你不知,俺家人多舌头多,今日为不见了这锭金子,早是你看着,就有人气不愤,在后边调白你大娘,说拿金子进我屋里来,怎的不见了。落后,不想是你二娘屋里丫头偷了,才显出个青红皂白来。不然,绑着鬼只是俺屋里丫头和奶子、老冯。冯妈妈急的那哭,只要寻死,说道:‘若没有这金子,我也不家去。’落后见有了金子,那咱才打了灯家去了。”吴银儿道:“娘,也罢。你看爹的面上,你守着哥儿慢慢过,到那里是那里!论起后边大娘没甚言语,也罢了。倒只是别人见娘生了哥儿,未免都有些儿气。爹他老人家有些主就好。”李瓶儿道:“若不是你爹和你大娘看觑,这孩子也活不到如今。说话之间,你一锺我一盏,不觉坐到三更天气,方才宿歇。正是:

\[
得意客来情不厌,知心人到话相投。
\]

\newpage
%# -*- coding:utf-8 -*-
%%%%%%%%%%%%%%%%%%%%%%%%%%%%%%%%%%%%%%%%%%%%%%%%%%%%%%%%%%%%%%%%%%%%%%%%%%%%%%%%%%%%%


\chapter{应伯爵劝当铜锣\KG 李瓶儿解衣银姐}


词曰:

\[
徘徊。相期酒会,三千朱履,十二金钗。雅俗熙熙,下车成宴尽春台。好雍容、东山妓女,堪笑傲、北海樽垒。且追陪。凤池归去,那更重来!
\]

话说西门庆因放假没往衙门里去,早晨起来,前厅看着,差玳安送两张桌面与乔家去。一张与乔五太太,一张与乔大户娘子,俱有高顶方糖、时鲜树果之类。乔五太太赏了两方手帕、三钱银子,乔大户娘子是一匹青绢,俱不必细说。

原来应伯爵自从与西门庆作别,赶到黄四家。黄四又早夥中封下十两银子谢他:“大官人吩咐教俺过节去,口气只是捣那五百两银子文书的情。你我钱粮拿甚么支持?”应伯爵道:“你如今还得多少才够?”黄四道:“李三哥他不知道,只要靠着问那内臣借,一般也是五分行利。不如这里借着衙门中势力儿,就是上下使用也省些。如今我算再借出五十个银子来,把一千两合用,就是每月也好认利钱。”应伯爵听了,低了低头儿,说道:“不打紧。假若我替你说成了,你夥计六人怎生谢我?”黄四道:“我对李三说,夥中再送五两银子与你。”伯爵道:“休说五两的话。要我手段,五两银子要不了你的,我只消一言,替你每巧一巧儿,就在里头了。今日俺房下往他家吃酒,我且不去。明日他请俺们晚夕赏灯,你两个明日绝早买四样好下饭,再着上一坛金华酒。不要叫唱的,他家里有李桂儿、吴银儿,还没去哩!你院里叫上六个吹打的,等我领着送了去。他就要请你两个坐,我在旁边,只消一言半句,管情就替你说成了。找出五百两银子来,共捣一千两文书,一个月满破认他三十两银子,那里不去了,只当你包了一个月老婆了。常言道:秀才无假漆无真。进钱粮之时,香里头多放些木头,蜡里头多掺些柏油,那里查帐去?不图打鱼,只图混水,借着他这名声儿,才好行事。”于是计议己定。到次日,李三、黄四果然买了酒礼,伯爵领着两个小厮,抬送到西门庆家来。

西门庆正在前厅打发桌面,只见伯爵来到,作了揖,道及:“昨日房下在这里打搅,回家晚了。”西门庆道:“我昨日周南轩那里吃酒,回家也有一更天气,也不曾见的新亲戚,老早就去了。今早衙门中放假,也没去。”说毕坐下,伯爵就唤李锦:“你把礼抬进来。”不一时,两个抬进仪门里放下。伯爵道:“李三哥、黄四哥再三对我说,受你大恩,节间没甚么,买了些微礼来,孝顺你赏人。”只见两个小厮向前磕头。西门庆道:“你们又送这礼来做甚么?我也不好受的,还教他抬回去。”伯爵道:“哥,你不受他的,这一抬出去,就丑死了。他还要叫唱的来伏侍,是我阻住他了,只叫了六名吹打的在外边伺候。”西门庆向伯爵道:“他既叫将来了,莫不又打发他?不如请他两个来坐坐罢。”伯爵得不的一声儿,即叫过李锦来,吩咐:“到家对你爹说:老爹收了礼了,这里不着人请去了,叫你爹同黄四爹早来这里坐坐。”那李锦应诺下去。须臾,收进礼去。令玳安封二钱银子赏他,磕头去了。六名吹打的下边伺候。

少顷,棋童儿拿茶来,西门庆陪伯爵吃了茶,就让伯爵西厢房里坐。因问伯爵:“你今日没会谢子纯?”伯爵道:“我早晨起来时,李三就到我那里,看着打发了礼来,谁得闲去会他?”西门庆即使棋童儿:“快请你谢爹去!”不一时,书童儿放桌儿摆饭,两个同吃了饭,收了家伙去。西门庆就与伯爵两个赌酒儿打双陆。伯爵趁谢希大未来,乘先问西门庆道:“哥,明日找与李智、黄四多少银子?”西门庆道:“把旧文书收了,另捣五百两银子文书就是了。”伯爵道:“这等也罢了。哥,你不如找足了一千两,到明日也好认利钱。我又一句话,那金子你用不着,还算一百五十两与他,再找不多儿了。”西门庆听罢,道:“你也说的是。我明日再找三百五十两与他罢,改一千两银子文书就是了,省的金子放在家,也只是闲着。”

两个正打双陆,忽见玳安儿来说道:“贲四拿了一座大螺钿大理石屏凤、两架铜锣铜鼓连铛儿,说是白皇亲家的,要当三十两银子,爹当与他不当?”西门庆道:“你教贲四拿进来我瞧。”不一时,贲四与两个人抬进去,放在厅堂上。西门庆与伯爵丢下双陆,走出来看,原来是三尺阔五尺高可桌放的螺钿描金大理石屏凤,端的黑白分明。伯爵观了一回,悄与西门庆道:“哥,你仔细瞧,恰好似蹲着个镇宅狮子一般。两架铜锣铜鼓,都是彩画金妆,雕刻云头,十分齐整。”在旁一力撺掇,说道:“哥,该当下他的。休说两架铜鼓,只一架屏凤,五十两银子还没处寻去。”西门庆道:“不知他明日赎不赎。”伯爵道:“没的说,赎甚么?下坡车儿营生,及到三年过来,七本八利相等。”西门庆道:“也罢,教你姐夫前边铺子里兑三十两与他罢。”刚打发去了,西门庆把屏凤拂抹干净,安在大厅正面,左右看视,金碧彩霞交辉。因问:“吹打乐工吃了饭不曾?”琴童道:“在下边吃饭哩。”西门庆道:“叫他吃了饭来吹打一回我听。”于是厅内抬出大鼓来,穿廊下边一带安放铜锣铜鼓,吹打起来,端的声震云霄,韵惊鱼鸟。正吹打着,只见棋童儿请谢希大到了。进来与二人唱了喏,西门庆道:“谢子纯,你过来估估这座屏风儿,值多少价?”谢希大近前观看了半日,口里只顾夸奖不已,说道:“哥,你这屏风,买得巧也得一百两银子,少也他不肯。”伯爵道:“你看,连这外边两架铜锣铜鼓,带铛铛儿,通共用了三十两银子。”那谢希大拍着手儿叫道:“我的南无耶,那里寻本儿利儿!休说屏风,三十两银子还搅给不起这两架铜锣铜鼓来。你看这两座架子,做的这工夫,朱红彩漆,都照依官司里的样范,少说也有四十斤响铜,该值多少银子?怪不的一物一主,那里有哥这等大福,偏有这样巧价儿来寻你的。”

说了一回,西门庆请入书房里坐的。不一时,李智、黄四也到了。西门庆说道:“你两个如何又费心送礼来?我又不好受你的。”那李智、黄四慌的说道:“小人惶恐,微物胡乱与老爹赏人罢了。蒙老爹呼唤,不敢不来。”于是搬过座儿来,打横坐了。须臾,小厮画童儿拿了五盏茶上来,众人吃了。少顷,玳安走上来请问:“爹,在那里放桌儿?”西门庆道:“就在这里坐罢。”于是玳安与画童两个抬了一张八仙桌儿,骑着火盆安放。伯爵、希大居上,西门庆主位,李智、黄四两边打横坐了。须臾,拿上春檠按酒,大盘大碗汤饭点心、各样下饭。酒泛羊羔,汤浮桃浪。乐工都在窗外吹打。西门庆叫了吴银儿席上递酒,这里前边饮酒不题。

却说李桂姐家保儿,吴银儿家丫头蜡梅,都叫了轿子来接。那桂姐听见保儿来,慌的走到门外,和保儿两个悄悄说了半日话,回到上房告辞要回家去。月娘再三留他道:“俺每如今便都往吴大妗子家去,连你每也带了去。你越发晚了从他那里起身,也不用轿子,伴俺每走百病儿,就往家去便了。”桂姐道:“娘不知,我家里无人,俺姐姐又不在家,有我五姨妈那里又请了许多人来做盒子会,不知怎么盼我。昨日等了我一日,他不急时,不使将保儿来接我。若是闲常日子,随娘留我几日我也住了。”月娘见他不肯,一面教玉箫将他那原来的盒子,装了一盒元宵、一盒白糖薄脆,交与保儿掇着,又与桂姐一两银子,打发他回去。这桂姐先辞月娘众人,然后他姑娘送他到前边,叫画童替他抱了毡包,竟来书房门首,教玳安请出西门庆来说话。这玳安慢慢掀帘子进入书房,向西门庆请道:“桂姐家去,请爹说话。”应伯爵道:“李桂儿这小淫妇儿,原来还没去哩。”西门庆道:“他今日才家去。”一面走出前边来。李姐与西门庆磕了四个头,就道:“打搅爹娘这里。”西门庆道:“你明日家去罢。”桂姐道:“家里无人,妈使保儿拿轿子来接了。”又道:“我还有一件事对爹说:俺姑娘房里那孩子,休要领出去罢。俺姑娘昨日晚夕又打了他几下。说起来还小哩,也不知道甚么,吃我说了他几句,从今改了,他说再不敢了。不争打发他出去,大节间,俺姑娘房中没个人使,他心里不急么?自古木杓火杖儿短,强如手拨剌,爹好歹看我分上,留下这丫头罢。”西门庆道:“既是你恁说,留下这奴才罢。”就吩咐玳安:“你去后边对你大娘说,休要叫媒人去了。”玳安见画童儿抱着桂姐毡包,说道:“拿桂姨毡包等我抱着,教画童儿后边说去罢。”那画童应诺,一直往后边去了。桂姐与西门庆说毕,又到窗子前叫道:“应花子,我不拜你了,你娘家去。”伯爵道:“拉回贼小淫妇儿来,休放他去了,叫他且唱一套儿与我听听着。”桂姐道:“等你娘闲了唱与你听。”伯爵道:“恁大白日就家去了,便益了贼小淫妇儿了,投到黑还接好几个汉子。”桂姐道:“汗邪了你这花子!”一面笑了出去。玳安跟着,打发他上轿去了。

西门庆与桂姐说了话,就后边更衣去了。应伯爵向谢希大说:“李家桂儿这小淫妇儿,就是个真脱牢的强盗,越发贼的疼人子!恁个大节,他肯只顾在人家住着?鸨子来叫他,又不知家里有甚么人儿等着他哩。”谢希大道:“你好猜。”悄悄向伯爵耳边,如此这般。说未数句,伯爵道:“悄悄儿说,哥正不知道哩。”不一时,西门庆走的脚步儿响,两个就不言语了。这应伯爵就把吴银儿搂在怀里,和他一递一口儿吃酒,说道:“是我这干女儿又温柔,又软款,强如李家狗不要的小淫妇儿一百倍了。”吴银儿笑道:“二爹好骂。说一个就一个,百个就百个,一般一方之地也有贤有愚,可可儿一个就比一个来?俺桂姐没恼着你老人家!”西门庆道:“你问贼狗才,单管只六说白道的!”伯爵道:“你休管他,等我守着我这干女儿过日子。干女儿过来,拿琵琶且先唱个儿我听。”这吴银儿不忙不慌,轻舒玉指,款跨鲛绡,把琵琶横于膝上,低低唱了一回《柳摇金》。伯爵吃过酒,又递谢希大,吴银儿又唱了一套。这里吴银儿递酒弹唱不题。

且说画童儿走到后边,月娘正和孟玉楼、李瓶儿、大姐、雪娥并大师父,都在上房里坐的,只见画童儿进来。月娘才待使他叫老冯来,领夏花儿出去,画童便道:“爹使小的对大娘说,教且不要领他出去罢了。”月娘道:“你爹教卖他,怎的又不卖他了?你实说,是谁对你爹说,教休要领他出去?”画童儿道:“刚才小的抱着桂姨毡包,桂姨临去对爹说,央及留下了将就使罢。爹使玳安进来对娘说,玳安不进来,使小的进来,他就夺过毡包送桂姨去了。”这月娘听了,就有几分恼在心中,骂玳安道:“恁贼两头献勤欺主的奴才,嗔道头里使他叫媒人,他就说道爹叫领出去,原来都是他弄鬼。如今又干办着送他去了,住回等他进后来,和他答话。”正说着,只见吴银儿前边唱了进来。月娘对他说:“你家蜡梅接你来了。李家桂儿家去了,你莫不也要家去了罢?”吴银儿道:“娘既留我,我又家去,显的不识敬重了。”因问蜡梅:“你来做甚么?”蜡梅道:“妈使我来瞧瞧你。”吴银儿问道:“家里没甚勾当?”蜡梅道:“没甚事。”吴银儿道:“既没事,你来接我怎的?你家去罢。娘留下我,晚夕还同众娘们往妗奶奶家走百病儿去。我那里回来,才往家去哩。”说毕,蜡梅就要走。月娘道:“你叫他回来,打发他吃些甚么儿。”吴银儿道:“你大奶奶赏你东西吃哩。等着就把衣裳包了带了家去,对妈妈说,休教轿子来,晚夕我走了家去。”因问:“吴惠怎的不来?”蜡梅道:“他在家里害眼哩。”月娘吩咐玉箫领蜡梅到后边,拿下两碗肉,一盘子馒头,一瓯子酒,打发他吃。又拿他原来的盒子,装了一盒元宵、一盒细茶食,回与他拿去。

原来吴银儿的衣裳包儿放在李瓶儿房里,李瓶儿早寻下一套上色织金缎子衣服、两方销金汗巾儿、一两银子,安放在他毡包内与他。那吴银儿喜孜孜辞道:“娘,我不要这衣服罢。”又笑嘻嘻道:“实和娘说,我没个白袄儿穿,娘收了这缎子衣服,不拘娘的甚么旧白绫袄儿,与我一件儿穿罢。”李瓶儿道:“我的白袄儿宽大,你怎的穿?”叫迎春:“拿钥匙,大橱柜里拿一匹整白绫来与银姐。”“对你妈说,教裁缝替你裁两件好袄儿。”因问:“你要花的,要素的?”吴银儿道:“娘,我要素的罢,图衬着比甲儿好穿。”笑嘻嘻向迎春说道:“又起动姐往楼上走一遭,明日我没甚么孝顺,只是唱曲儿与姐姐听罢了。”

须臾,迎春从楼上取了一匹松江阔机尖素白绫,下号儿写着“重三十八两”,递与吴银儿。银儿连忙与李瓶儿磕了四个头,起来又深深拜了迎春八拜。李瓶儿道:“银姐,你把这缎子衣服还包了去,早晚做酒衣儿穿。”吴银儿道:“娘赏了白绫做袄儿,怎好又包了这衣服去?”于是又磕头谢了。

不一时,蜡梅吃了东西,交与他都拿回家去了。月娘便说:“银姐,你这等我才喜欢。休学李桂儿那等乔张致,昨日和今早,只象卧不住虎子一般,留不住的,只要家去。可可儿家里就忙的恁样儿?连唱也不用心唱了。见他家人来接,饭也不吃就去了。银姐,你快休学他。”吴银儿道:“好娘,这里一个爹娘宅里,是那个去处?就有虚篢放着别处使,敢在这里使?桂姐年幼,他不知事,俺娘休要恼他。”正说着,只见吴大妗子家使了小厮来定儿来请,说道:“俺娘上覆三姑娘,好歹同众位娘并桂姐、银姐,请早些过去罢。又请雪姑娘也走走。”月娘道:“你到家对你娘说,俺们如今便收拾去。二娘害腿疼不去,他在家看家了。你姑夫今日前边有人吃酒,家里没人,后边姐也不去。李桂姐家去了。连大姐、银姐和我们六位去。你家少费心整治甚么,俺们坐一回,晚上就来。”因问来定儿:“你家叫了谁在那里唱?”来定儿道:“是郁大姐。”说毕,来定儿先去了。月娘一面同玉楼、金莲、李瓶儿、大姐并吴银儿,对西门庆说了,吩咐奶子在家看哥儿,都穿戴收拾,共六顶轿子起身。派定玳安儿、棋童儿、来安儿三个小厮,四个排军跟轿,往吴大妗子家来。正是:

\[
万井风光春落落,千门灯火夜沉沉。
\]

\newpage
%# -*- coding:utf-8 -*-
%%%%%%%%%%%%%%%%%%%%%%%%%%%%%%%%%%%%%%%%%%%%%%%%%%%%%%%%%%%%%%%%%%%%%%%%%%%%%%%%%%%%%


\chapter{元夜游行遇雪雨\KG 妻妾戏笑卜龟儿}


词曰:

\[
小市东门欲雪天,众中依约见神仙。蕊黄香细贴金蝉。饮散黄昏人草草,醉容无语立门前。马嘶尘哄一街烟。
\]

话说西门庆那日,打发吴月娘众人往吴大妗子家吃酒去了。李智、黄四约坐到黄昏时分,就告辞起身。伯爵赶送出去,如此这般告诉:“我已替二公说了,准在明日还找五百两银子。”那李智、黄四向伯爵打了恭又打恭,去了。伯爵复到厢房中,和谢希大陪西门庆饮酒,只见李铭掀帘子进来。伯爵看见,便道:“李日新来了。”李铭扒在地下磕头。西门庆问道:“吴惠怎的不来?”李铭道:“吴惠今日东平府官身也没去,在家里害眼。小的叫了王柱来了。”便叫王柱:“进来,与爹磕头。”那王柱掀帘进入房里,朝上磕了头,与李铭站立在旁。伯爵道:“你家桂姐刚才家去了,你不知道?”李铭道:“小的官身到家,洗了洗脸就来了,并不知道。”伯爵向西门庆说:“他两个怕不的还没吃饭哩,哥吩咐拿饭与他两个吃。”书童在旁说:“二爹,叫他等一等,亦发和吹打的一答里吃罢,敢也拿饭去了。”伯爵令书童取过一个托盘来,桌上掉了两碟下饭,一盘烧羊肉,递与李铭:“等拿了饭来,你每拿两碗在这明间吃罢。”说书童儿:“我那傻孩子,常言道:方以类聚,物以群分。你不知,他这行人故虽是当院出身,小优儿比乐工不同,一概看待也罢了,显的说你我不帮衬了。”被西门庆向伯爵头上打了一下,笑骂道:“怪不的你这狗才,行计中人只护行计中人,又知这当差的甘苦。”伯爵道:“傻孩儿,你知道甚么!你空做子弟一场,连‘惜玉怜香’四个字你还不晓的。粉头、小优儿如同鲜花一般,你惜怜他,越发有精神。你但折剉他,敢就《八声甘州》恹恹瘦损,难以存活。”西门庆笑道:“还是我的儿晓的道理。”

那李铭、王柱须臾吃了饭,应伯爵叫过来吩咐:“你两个会唱‘雪月风花共裁剪’不会?”李铭道:“此是黄钟,小的每记的。”于是,王柱弹琵琶,李铭\textuni{22E88}筝,顿开喉音唱了一套。唱完了,看看晚来,正是:

\[
金乌渐渐落西山,玉兔看看上画阑;
佳人款款来传报,月透纱窗衾枕寒。
\]

西门庆命收了家火,使人请傅伙计、韩道国、云主管、贲四、陈敬济,大门首用一架围屏安放两张桌席,悬挂两盏羊角灯,摆设酒筵,堆集许多春檠果盒,各样肴馔。西门庆与伯爵、希大都一带上面坐了,伙计、主管两旁打横。大门首两边,一边十二盏金莲灯。还有一座小烟火,西门庆吩咐等堂客来家时放。先是六个乐工,抬铜锣铜鼓在大门首吹打。吹打了一回,又请吹细乐上来。李铭、王柱两个小优儿筝、琵琶上来,弹唱灯词。那街上来往围看的人,莫敢仰视。西门庆带忠靖冠,丝绒鹤氅,白绫袄子。玳安与平安两个,一递一桶放花儿。两名排军执揽杆拦挡闲人,不许向前拥挤。不一时,碧天云静,一轮皓月东升之时,街上游人十分热闹,但见:

\[
户户鸣锣击鼓,家家品竹弹丝。游人队队踏歌声,士女翩翩垂舞调。鳌山结彩,巍峨百尺矗晴云;凤禁褥香,缥缈千层笼绮队。闲庭内外,溶溶宝月光辉;画阁高低,灿灿花灯照耀。三市六街人闹热,凤城佳节赏元宵。
\]

且说春梅、迎春、玉箫、兰香、小玉众人,见月娘不在,听见大门首吹打铜鼓弹唱,又放烟火,都打扮着走来,在围屏后扒着望外瞧。书童儿和画童儿两个,在围屏后火盆上筛酒。原来玉箫和书童旧有私情,两个常时戏狎。两个因按在一处夺瓜子儿嗑,不防火盆上坐着一锡瓶酒,推倒了,那火烘烘望上腾起来,漰了一地灰起去。那王箫还只顾嘻笑,被西门庆听见,使下玳安儿来问:“是谁笑?怎的这等灰起?”那日春梅穿着新白绫袄子,大红遍地金比甲,正坐在一张椅儿上,看见他两个推倒了酒,就扬声骂玉箫道:“好个怪浪的淫妇!见了汉子,就邪的不知怎么样儿的了,只当两个把酒推倒了才罢了。都还嘻嘻哈哈,不知笑的是甚么!把火也漰死了,平白落人恁一头灰。”玉箫见他骂起来,唬的不敢言语,往后走了。慌的书童儿走上去,回说:“小的火盆上筛酒来,扒倒了锡瓶里酒了。”西门庆听了,便不问其长短,就罢了。

先是那日,贲四娘子打听月娘不在,平昔知道春梅、玉箫、迎春、兰香四个是西门庆贴身答应得宠的姐儿,大节下安排了许多菜蔬果品,使了他女孩儿长儿来,要请他四个去他家里坐坐。众人领了来见李娇儿。李娇儿说:“我灯草拐杖——做不得主。你还请问你爹去。”问雪娥,雪娥亦发不敢承揽。看看挨到掌灯以后,贲四娘子又使了长儿来邀四人。兰香推玉箫,玉箫推迎春,迎春推春梅,要会齐了转央李娇儿和西门庆说,放他去。那春梅坐着,纹丝儿也不动,反骂玉箫等:“都是那没见食面的行货子,从没见酒席,也闻些气儿来!我就去不成,也不到央及他家去。一个个鬼撺攥的也似,不知忙些甚么,教我半个眼儿看的上!”那迎春、玉箫、兰香都穿上衣裳,打扮的齐齐整整出来,又不敢去,这春梅又只顾坐着不动身。书童见贲四嫂又使了长儿来邀,说道:“我拚着爹骂两句也罢,等我上去替姐每禀禀去。”一直走到西门庆身边,附耳说道:“贲四嫂家大节间要请姐每坐坐,姐教我来禀问爹,去不去?”西门庆听了,吩咐:“教你姐每收拾去,早些来,家里没人。”这书童连忙走下来,说道:“还亏我到上头,一言就准了。教你姐快收拾去,早些来。”那春梅才慢慢往房里匀施脂粉去了。

不一时,四个都一答儿里出门。书童扯围屏掩过半边来,遮着过去。到了贲四家,贲四娘子见了,如同天上落下来的一般,迎接进屋里。顶槅上点着绣球纱灯,一张桌儿上整齐肴菜。赶着春梅叫大姑,迎春叫二姑,玉箫是三姑,兰香是四姑,都见过礼。又请过韩回子娘子来相陪。春梅、迎春上坐,玉箫、兰香对席,贲四嫂与韩回子娘子打横,长儿往来烫酒拿菜。按下这里不题。

西门庆因叫过乐工来吩咐:“你每吹一套‘东风料悄’《好事近》与我听。”正值后边拿上玫瑰元宵来,众人拿起来同吃,端的香甜美味,入口而化,甚应佳节。李铭、王柱席前拿乐器,接着弹唱此词,端的声韵悠扬,疾徐合节。这里弹唱饮酒不题。

且说玳安与陈敬济袖着许多花炮,又叫两个排军拿着两个灯笼,竟往吴大妗于家来接月娘。众人正在明间饮酒,见了陈敬济来:“教二舅和姐夫房里坐,你大舅今日不在家,卫里看着造册哩。”一面放桌儿,拿春盛点心酒菜上来,陪敬济。玳安走到上边,对月娘说:“爹使小的来接娘每来了,请娘早些家去,恐晚夕人乱,和姐夫一答儿来了。”月娘因头里恼他,就一声儿没言语答他。吴大妗子便叫来定儿:“拿些儿甚么与玳安儿吃。”来定儿道:“酒肉汤饭,都前头摆下了。”吴月娘道:“忙怎的?那里才来乍到就与他吃!教他前边站着,我每就起身。”吴大妗子道:“三姑娘慌怎的?上门儿怪人家?大节下,姊妹间,众位开怀大坐坐儿。左右家里有他二娘和他姐在家里,怕怎的?老早就要家去!是别人家又是一说。”因叫郁大姐:“你唱个好曲儿,伏侍他众位娘。”孟玉楼道:“他六娘好不恼他哩,说你不与他做生日。”郁大姐连忙下席来,与李瓶儿磕了四个头,说道:“自从与五娘做了生日,家去就不好起来。昨日妗奶奶这里接我,教我才收拾\textuni{499B}\textuni{49B7}了来。若好时,怎的不与你老人家磕头?”金莲道:“郁大姐,你六娘不自在哩,你唱个好的与他听,他就不恼你了。”那李瓶儿在旁只是笑,不做声。郁大姐道:“不打紧,拿琵琶过来,等我唱。”大妗子叫吴舜臣媳妇郑三姐:“你把你二位姑娘和众位娘的酒儿斟上。这一日还没上过钟酒儿。”那郁大姐接琵琶在手,用心用意唱了一个《一江风》。

正唱着,月娘便道:“怎的这一回子恁凉凄凄的起来?”来安儿在旁说道:“外边天寒下雪哩。”孟玉楼道:“姐姐,你身上穿的不单薄?我倒带了个绵披袄子来了。咱这一回,夜深不冷么?”月娘道:“既是下雪,叫个小厮家里取皮袄来咱每穿。”那来安连忙走下来,对玳安说:“娘吩咐,叫人家去取娘们皮袄哩。”那玳安便叫琴童儿:“你取去罢,等我在这里伺候。”那琴童也不问,一直家去了。少顷,月娘想起金莲没皮袄,因问来安儿:“谁取皮袄去了?”来安道:“琴童取去了。”月娘道:“也不问我,就去了。”玉楼道:“刚才短了一句话,不该教他拿俺每的,他五娘没皮袄,只取姐姐的来罢。”月娘道:“怎的没有?还有当的人家一件皮袄,取来与六姐穿就是了。”因问:“玳安那奴才怎的不去,却使这奴才去了?你叫他来!”一面把玳安叫到跟前,吃月娘尽力骂了几句道:“好奴才!使你怎的不动?又坐坛遣将儿,使了那个奴才去了。也不问我声儿,三不知就去了。怪不的你做大官儿,恐怕打动你展翅儿,就只遣他去!”玳安道:“娘错怪了小的。头里娘吩咐若是叫小的去,小的敢不去?来安下来,只说叫一个家里去。”月娘道:“那来安小奴才敢吩咐你?俺每恁大老婆,还不敢使你哩!如今惯的你这奴才们有些摺儿也怎的?一来主子烟薰的佛像——挂在墙上,有恁施主,有恁和尚。你说你恁行动两头戳舌,献勤出尖儿,外合里应,好懒食馋,背地瞒官作弊,干的那茧儿我不知道哩!头里你家主子没使你送李桂儿家去,你怎的送他?人拿着毡包,你还匹手夺过去了。留丫头不留丫头不在你,使你进来说,你怎的不进来?你便送他,图嘴吃去了,却使别人进来。须知我若骂只骂那个人了。你还说你不久惯牢成!”玳安道:“这个也没人,就是画童儿过的舌。爹见他抱着毡包,教我:‘你送送你桂姨去罢’,使了他进来的。娘说留丫头不留丫头不在于小的,小的管他怎的!”月娘大怒,骂道:“贼奴才,还要说嘴哩!我可不这里闲着和你犯牙儿哩。你这奴才,脱脖倒坳过颺了。我使着不动,耍嘴儿,我就不信到明日不对他说,把这欺心奴才打与你个烂羊头也不算。”吴大妗子道:“玳安儿,还不快替你娘每取皮袄去。”又道:“姐姐,你吩咐他拿那里皮袄与他五娘穿?”潘金莲接过来说道:“姐姐,不要取去,我不穿皮袄,教他家里捎了我的披袄子来罢。人家当的,好也歹也,黄狗皮也似的,穿在身上,教人笑话,也不长久,后还赎的去了。”月娘道:“这皮袄倒不是当的,是李智少十六两银子准折的。当的王招宣府里那件皮袄,与李娇儿穿了。”因吩咐玳安:“皮袄在大橱里,叫玉箫寻与你,就把大姐的皮袄也带了来。”

玳安把嘴谷都,走出来,陈敬济问道:“你到那去?”玳安道:“精是攮气的营生,一遍生活两遍做,这咱晚又往家里跑一遭。”迳走到家。西门庆还在大门首吃酒,傅伙计、云主管都去了,还有应伯爵、谢希大、韩道国、贲四众人吃酒未去,便问玳安:“你娘们来了?”玳安道:“没来,使小的取皮袄来了。”说毕,便往后走。先是琴童到家,上房里寻玉箫要皮袄。小玉坐在炕上正没好气,说道:“四个淫妇今日都在贲四老婆家吃酒哩。我不知道皮袄放在那里,往他家问他要去。”这琴童一直走到贲四家,且不叫,在窗外悄悄觑听。只见贲四嫂说道:“大姑和三姑,怎的这半日酒也不上,菜儿也不拣一箸儿?嫌俺小家儿人家,整治的不好吃也怎的?”春梅道:“四嫂,俺每酒够了。”贲四嫂道:“耶嚛!没的说。怎的这等上门儿怪人家!”又叫韩回子老婆:“你是我的切邻,就如副东一样,三姑、四姑跟前酒,你也替我劝劝儿,怎的单板着,象客一般?”又叫长姐:“筛酒来,斟与三姑吃,你四姑钟儿浅斟些儿罢。”兰香道:“我自来吃不的。”贲四嫂道:“你姐儿们今日受饿,没甚么可口的菜儿管待,休要笑话。今日要叫了先生来,唱与姑娘们下酒,又恐怕爹那里听着。浅房浅屋,说不的俺小家儿人家的苦。”说着,琴童儿敲了敲门,众人都不言语了。长儿问:“是谁?”琴童道:“是我,寻姐说话。”一面开了门,那琴童入来。玉箫便问:“娘来了?”那琴童看着待笑,半日不言语。玉箫道:“怪雌牙的,谁与你雌牙?问着不言语。”琴童道:“娘每还在妗子家吃酒哩,见天阴下雪,使我来家取皮袄来,都教包了去哩。”玉箫道:“皮袄在描金箱子里不是,叫小玉拿与你。”琴童道:“小玉说教我来问你要。”玉箫道:“你信那小淫妇儿,他不知道怎的!”春梅道:“你每有皮袄的,都打发与他。俺娘没皮袄,只我不动身。”兰香对琴童:“你三娘皮袄,问小鸾要。”迎春便向腰里拿钥匙与琴童儿:“教绣春开里间门拿与你。”

琴童儿走到后边,上房小玉和玉楼房中小鸾,都包了皮袄交与他。正拿着往外走,遇见玳安,问道:“你来家做甚么?”玳安道:“你还说哩!为你来了,平白教大娘骂了我一顿好的。又使我来取五娘的皮袄来。”琴童道:“我如今取六娘的皮袄去也。”玳安道:“你取了,还在这里等着我,一答儿里去。你先去了不打紧,又惹的大娘骂我。”说毕,玳安来到上房。小玉正在炕上笼着炉台烤火,口中嗑瓜子儿,见了玳安,问道:“你也来了?”玳安道:“你又说哩,受了一肚子气在这里。娘说我遣将儿。因为五娘没皮袄,又教我来,说大橱里有李三准折的一领皮袄,教拿去哩。”小玉道:“玉箫拿了里间门上钥匙,都在贲四家吃酒哩,教他来拿。”玳安道:“琴童往六娘房里去取皮袄,便来也,教他叫去,我且歇歇腿儿,烤烤火儿着。”那小玉便让炕头儿与他,并肩相挨着向火。小玉道:“壶里有酒,筛盏子你吃?”玳安道:“可知好哩,看你下顾。”小玉下来,把壶坐在火上,抽开抽屉,拿了一碟子腊鹅肉,筛酒与他。无人处两个就搂着咂舌亲嘴。

正吃着酒,只见琴童儿进来。玳安让他吃了一盏子,便使他:“叫玉箫姐来,拿皮袄与五娘穿。”那琴童抱毡包放下,走到贲四家叫玉箫。玉箫骂道:“贼囚根子,又来做甚么?”又不来。递与钥匙,教小玉开门。那小玉开了里间房门,取了一把钥匙,通了半日,白通不开。琴童儿又往贲四家问去。那玉箫道:“不是那个钥匙。娘橱里钥匙在床褥子座下哩。”小玉又骂道:“那淫妇丁子钉在人家不来,两头来回,只教使我。”及开了,橱里又没皮袄。琴童儿来回走的抱怨道:“就死也死三日三夜,又撞着恁瘟死鬼小奶奶儿们,把人魂也走出了。”向玳安道:“你说此回去,又惹的娘骂。不说屋里,只怪俺们。”走去又对玉箫说:“里间娘橱里寻,没有皮袄。”玉箫想了想,笑道:“我也忘记,在外间大橱里。”到后边,又被小玉骂道:“淫妇吃那野汉子捣昏了,皮袄在这里,却到处寻。”一面取出来,将皮袄包了,连大姐皮袄都交付与玳安、琴童。

两个拿到吴大妗子家,月娘又骂道:“贼奴才,你说同了都不来罢了。”那玳安不敢言语,琴童道:“娘的皮袄都有了,等着姐又寻这件青镶皮袄。”于是打开取出来。吴大妗子灯下观看,说道:“好一件皮袄。五娘,你怎的说他不好,说是黄狗皮。那里有恁黄狗皮,与我一件穿也罢了。”月娘道:“新新的皮袄儿,只是面前歇胸旧了些儿。到明日,从新换两个遍地金歇胸,就好了。孟玉楼拿过来,与金莲戏道:“我儿,你过来,你穿上这黄狗皮,娘与你试试看好不好。”金莲道:“有本事到明日问汉子要一件穿,也不枉的。平白拾人家旧皮袄披在身上做甚么!”玉楼戏道:“好个不认业的,人家有这一件皮袄,穿在身上念佛。”于是替他穿上。见宽宽大大,金莲才不言语。

当下月娘与玉楼、瓶儿俱是貂鼠皮袄,都穿在身上,拜辞吴大妗子、二妗子起身。月娘与了郁大姐一包二钱银子。吴银儿道:“我这里就辞了妗子、列位娘,磕了头罢。”当下吴大妗子与了一对银花儿,月娘与李瓶儿每人袖中拿出一两银子与他,磕头谢了。吴大妗子同二妗子、郑三姐都还要送月娘众人,因见天气落雪,月娘阻回去了。琴童道:“头里下的还是雪,这回沾在身上都是水珠儿,只怕湿了娘们的衣服,问妗子这里讨把伞打了家去。”吴二舅连忙取了伞来,琴童儿打着,头里两个排军打灯笼,引着一簇男女,走几条小巷,到大街上。陈敬济沿路放了许多花炮,因叫:“银姐,你家不远了,俺每送你到家。”月娘便问:“他家在那里?”敬济道:“这条胡同内一直进去,中间一座大门楼,就是他家。”吴银儿道:“我这里就辞了娘每家去。”月娘道:“地下湿,银姐家去罢,头里已是见过礼了。我还着小厮送你到家。”因叫过玳安:“你送送银家去。”敬济道:“娘,我与玳安两个去罢。”月娘道:“也罢,你与他两个同送他送。”那敬济得不的一声,同玳安一路送去了。

吴月娘众人便回家来。潘金莲路上说:“大姐姐,你原说咱每送他家去,怎的又不去了?”月娘笑道:“你也只是个小孩儿,哄你说耍子儿,你就信了。丽春院是那里,你我送去?”金莲道:“像人家汉子在院里嫖了来,家里老婆没曾往那里寻去?寻出没曾打成一锅粥?”月娘道:“你等他爹到明日往院里去,你寻他寻试试。倒没的教人家汉子当粉头拉了去,看你——”两个口里说着,看看走到东街上,将近乔大户门首。只见乔大户娘子和他外甥媳妇段大姐,在门首站立。远远见月娘一簇男女过来,就要拉请进去。月娘再三说道:“多谢亲家盛情,天晚了,不进去罢。”那乔大户娘子那里肯放,说道:“好亲家,怎的上门儿怪人家?”强把月娘众人拉进去了。客位内挂着灯,摆设酒果,有两个女儿弹唱饮酒,不题。

却说西门庆,在门首与伯爵众人饮酒将阑。伯爵与希大整吃了一日,顶颡吃不下去,见西门庆在椅子上打盹,赶眼错把果碟儿都倒在袖子里,和韩道国就走了。只落下贲四,陪西门庆打发了乐工赏钱。吩咐小厮收家火,熄灯烛,归后边去了。只见平安走来,贲四家叫道:“你们还不起身,爹进去了。”玉箫听见,和迎春、兰香慌的辞也不辞,都一溜烟跑了。只落下春梅,拜谢了贲四嫂,才慢慢走回来。看见兰香在后边脱了鞋赶不上,因骂道:“你们都抢棺材奔命哩!把鞋都跑脱了,穿不上,象甚腔儿!”到后边,打听西门庆在李娇儿房里,都来磕头。大师父见西门庆进入李娇儿房中,都躲到上房,和小玉在一处。玉箫进来,道了万福,那小玉就说玉箫:“娘那里使小厮来要皮袄,你就不来管管儿,只教我拿。我又不知那根钥匙开橱门,及自开了又没有,落后却在外边大橱拒里寻出来。你放在里头,怎昏抢了不知道?姐姐每都吃勾来了罢,几曾见长出块儿来!”玉箫吃的脸红红的,道:“怪小淫妇儿,如何狗挝了脸似的?人家不请你,怎的和俺们使性儿!”小玉道:“我稀罕那淫妇请!”大师父在旁劝道:“姐姐每义让一句儿罢,你爹在屋里听着。只怕你娘们来家,顿下些茶儿伺候。”正说着,只见琴童抱进毡包来。玉箫便问:“娘来了?”琴童道:“娘每来了,又被乔亲家娘在门首让进去吃酒哩,也将好起身。”两个才不言语了。

不一时,月娘等从乔大户娘子家出来。到家门首,贲四娘子走出来厮见。陈敬济和贲四一面取出一架小烟火来,在门首又看放了一回烟火,方才进来,与李娇儿、大师父道了万福。雪娥走来,向月娘磕了头,与玉楼等三人见了礼。月娘因问:“他爹在那里?”李娇儿道:“刚才在我那屋里,我打发他睡了。”月娘一声儿没言语。只见春梅、迎春、玉箫、兰香进来磕头。李娇儿便说:“今日前边贲四嫂请了四个去,坐了回儿就来了。”月娘听了,半日没言语。骂道:“恁成精狗肉们,平白去做甚么!谁教他去来?”李娇儿道:“问过他爹才去来。”月娘道:“问他?好有张主的货!你家初一十五开的庙门早了,放出些小鬼来了。”大师父道:“我的奶奶,恁四个上画儿的姐姐,还说是小鬼。”月娘道:“上画儿只画的半边儿,平白放出去做甚么?与人家喂眼!”孟玉楼见月娘说来的不好,就先走了。落后金莲见玉楼起身,和李瓶儿、大姐也走了。止落下大师父,和月娘同在一处睡了。那雪霰直下到四更方止。正是:

\[
香消烛冷楼台夜,挑菜烧灯扫雪天。
\]

一宿晚景题过。到次日,西门庆往衙门中去了。月娘约饭时前后,与孟玉楼、李瓶儿三个同送大师父家去。因在大门里首站立,见一个乡里卜龟儿卦儿的老婆子,穿着水合袄、蓝布裙子,勒黑包头,背着褡裢,正从街上走来。月娘使小厮叫进来,在二门里铺下卦帖,安下灵龟,说道:“你卜卜俺每。”那老婆扒在地下磕了四个头:“请问奶奶多大年纪?”月娘道:“你卜个属龙的女命。”那老婆道:“若是大龙,四十二岁,小龙儿三十岁。”月娘道:“是三十岁了,八月十五日子时生。”那老婆把灵龟一掷,转了一遭儿住了。揭起头一张卦帖儿。上面画着一个官人和一位娘子在上面坐,其余都是侍从人,也有坐的,也有立的,守着一库金银财宝。老婆道:“这位当家的奶奶是戊辰生,戊辰己巳大林木。为人一生有仁义,性格宽洪,心慈好善,看经布施,广行方便。一生操持,把家做活,替人顶缸受气,还不道是。喜怒有常,主下人不足。正是:喜乐起来笑嘻嘻,恼将起来闹哄哄。别人睡到日头半天还未起,你老早在堂前转了。梅香洗铫铛,虽是一时风火性,转眼却无心。和人说也有,笑也有,只是这疾厄宫上着刑星,常沾些啾唧。亏你这心好,济过来了,往后有七十岁活哩。”孟玉楼道:“你看这位奶奶命中有子没有?”婆子道:“休怪婆子说,儿女宫上有些不实,往后只好招个出家的儿子送老罢了。随你多少也存不的。”玉楼向李瓶儿笑道:“就是你家吴应元,见做道士家名哩。”月娘指着玉楼:“你也叫他卜卜。”玉楼道:“你卜个三十四岁的女命,十一月二十七日寅时生。”那婆子从新撇了卦帖,把灵龟一卜,转到命宫上住了。揭起第二张卦帖来,上面画着一个女人,配着三个男人:头一个小帽商旅打扮;第二个穿红官人;第三个是个秀才。也守着一库金银,左右侍从伏侍。婆子道:“这位奶奶是甲子年生。甲子乙丑海中金。命犯三刑六害,夫主克过方可。”玉楼道:“已克过了。”婆子道:“你为人温柔和气,好个性儿。你恼那个人也不知,喜欢那个人也不知,显不出来。一生上人见喜下钦敬,为夫主宠爱。只一件,你饶与人为了美,多不得人心。命中一生替人顶缸受气,小人驳杂,饶吃了还不道你是。你心地好了,虽有小人也拱不动你。”玉楼笑道:“刚才为小厮讨银子和他乱了,这回说是顶缸受气。”月娘道:“你看这位奶奶往后有子没有?”婆子道:“济得好,见个女儿罢了。子上不敢许,若说寿,倒尽有。”月娘道:“你卜卜这位奶奶。李大姐,你与他八字儿。”李瓶儿笑道:“我是属羊的。”婆子道:“若属小羊的,今年念七岁,辛未年生的。生几月?”李瓶儿道:“正月十五日午时。”那婆子卜转龟儿,到命宫上矻磴住了。揭起卦帖来,上面画着一个娘子,三个官人:头一个官人穿红,第二个官人穿绿,第三个穿青。怀着个孩儿,守着一库金银财宝,旁边立着个青脸獠牙红发的鬼。婆子道:“这位奶奶,庚午辛未路旁土。一生荣华富贵,吃也有,穿也有,所招的夫主都是贵人。为人心地有仁义,金银财帛不计较,人吃了转了他的,他喜欢;不吃他,不转他,到恼。只是吃了比肩不和的亏,凡事恩将仇报。正是:比肩刑害乱扰扰,转眼无情就放刁;宁逢虎摘三生路,休遇人前两面刀。奶奶,你休怪我说:你尽好匹红罗,只可惜尺头短了些。气恼上要忍耐些,就是子上也难为。”李瓶儿道:“今已是寄名做了道士。”婆子道:“既出了家,无妨了。又一件,你老人家今年计都星照命,主有血光之灾,仔细七八月不见哭声才好。”说毕,李瓶儿袖中掏出五分一块银子,月娘和玉楼每人与钱五十文。

刚打发卜龟卦婆子去了,只见潘金莲和大姐从后边出来,笑道:“我说后边不见,原来你每都往前头来了。”月娘道:“俺们刚才送大师父出来,卜了这回龟儿卦。你早来一步,也教他与你卜卜儿。”金莲摇头儿道:“我是不卜他。常言:算的着命,算不着行。想前日道士说我短命哩,怎的哩?说的人心里影影的。随他明日街死街埋,路死路埋,倒在洋沟里就是棺材。”说毕,和月娘同归后边去了。正是:

\[
万事不由人算计,一生都是命安排。
\]

\newpage
%# -*- coding:utf-8 -*-
%%%%%%%%%%%%%%%%%%%%%%%%%%%%%%%%%%%%%%%%%%%%%%%%%%%%%%%%%%%%%%%%%%%%%%%%%%%%%%%%%%%%%


\chapter{苗青贪财害主\KG 西门枉法受赃}


诗曰:

\[
怀璧身堪罪,偿金迹未明。龙蛇一失路,虎豹屡相惊。
暂遣虞罗急,终知汉法平。须凭鲁连箭,为汝谢聊成。
\]

话说江南扬州广陵城内,有一苗员外,名唤苗天秀。家有万贯资财,颇好诗礼。年四十岁,身边无子,止有一女尚未出嫁。其妻李氏,身染痼疾在床,家事尽托与宠妾刁氏,名唤刁七儿。原是娼妓出身,天秀用银三百两娶来家,纳为侧室,宠嬖无比。忽一日,有一老僧在门首化缘,自称是东京报恩寺僧,因为堂中缺少一尊镀金铜罗汉,故云游在此,访善纪录。天秀问之,不吝,即施银五十两与那僧人。僧人道:“不消许多,一半足矣。”天秀道:“吾师休嫌少,除完佛像,余剩可作斋供。”那僧人问讯致谢,临行向天秀说道:“员外左眼眶下有一道死气,主不出此年当有大灾。你有如此善缘与我,贫僧焉敢不预先说知。今后随有甚事,切勿出境。戒之戒之。”言毕,作辞而去。

那消半月,天秀偶游后园,见其家人苗青正与刁氏亭侧私语,不意天秀卒至看见,不由分说,将苗青痛打一顿,誓欲逐之。苗青恐惧,转央亲邻再三劝留得免,终是切恨在心。不期有天秀表兄黄美,原是扬州人氏,乃举人出身,在东京开封府做通判,亦是博学广识之人。一日,寄一封书来与天秀,要请天秀上东京,一则游玩,二者为谋其前程。苗天秀得书大喜,因向其妻妾说道:“东京乃辇毂之地,景物繁华,吾心久欲游览,无由得便。今不期表兄书来相招,实慰平生之意。”其妻李氏便说:“前日僧人相你面上有灾厄,嘱咐不可出门。此去京都甚远,况你家私沉重,抛下幼女病妻在家,未审此去前程如何,不如勿往为善。”天秀不听,反加怒叱,说道:“大丈夫生于天地之间,桑弧蓬矢,不能邀游天下,观国之光,徒老死牖下,无益矣。况吾胸中有物,囊有余资,何愁功名不到手?此去表兄必有美事于我,切勿多言!”于是吩咐家人苗青,收拾行李衣装,多打点两箱金银,载一船货物,带了个安童并苗青,上东京。嘱咐妻妾守家,择日起行。

正值秋末冬初之时,从扬州码头上船,行了数日,到徐州洪。但见一派水光,十分阴恶。但见:

\[
万里长洪水似倾,东流海岛若雷鸣,
滔滔雪浪令人怕,客旅逢之谁不惊?
\]
前过地名陕湾,苗员外看见天晚,命舟人泊住船只。也是天数将尽,合当有事,不料搭的船只却是贼船。两个艄子皆是不善之徒:一个名唤陈三,一个乃是翁八。常言道:不着家人,弄不得家鬼。这苗青深恨家主,日前被责之仇一向要报无由,口中不言,心内暗道:“不如我如此这般,与两个艄子做一路,将家主害了性命,推在水内,尽分其财物。我回去再把病妇谋死,这分家私连刁氏,都是我情受的。”正是:

\[
花枝叶下犹藏刺,人心怎保不怀毒。
\]
这苗青于是与两个艄子密密商量,说道:“我家主皮箱中还有一千两金银,二千两缎匹,衣服之类极广。汝二人若能谋之,愿将此物均分。”陈三、翁八笑道:“汝若不言,我等亦有此意久矣。”

是夜天气阴黑,苗天秀与安童在中舱里睡,苗青在橹后。将近三鼓时分,那苗青故意连叫有贼。苗天秀梦中惊醒,便探头出舱外观看,被陈三手持利刀,一下刺中脖下,推在洪波荡里。那安童正要走时,吃翁八一闷棍打落水中。三人一面在船舱内打开箱笼,取出一应财帛金银,并其缎货衣服,点数均分。二艄便说:“我若留此货物,必然有犯。你是他手下家人,载此货物到于市店上发卖,没人相疑。”因此二艄尽把皮箱中一千两金银,并苗员外衣服之类分讫,依前撑船回去了。这苗青另搭了船只,载至临清码头上,钞关上过了,装到清河县城外官店内卸下,见了扬州故旧商家,只说:“家主在后船,便来也。”这个苗青在店发卖货物,不题。

常言:人便如此如此,天理未然未然。可怜苗员外平昔良善,一旦遭其仆人之害,不得好死,虽是不纳忠言之劝,其亦大数难逃。不想安童被一棍打昏,虽落水中,幸得不死,浮没芦港。忽有一只渔船撑将下来,船上坐着个老翁,头顶箬笠,身披短蓑,听得啼哭之声。移船看时,却是一个十七八岁小厮,慌忙救了。问其始末情由,却是扬州苗员外家安童,在洪上被劫之事。这渔翁带下船,取衣服与他换了,给以饮食,因问他:“你要回去,却是同我在此过活?”安童哭道:“主人遭难,不见下落,如何回得家去?愿随公公在此。”渔翁道:“也罢,你且随我在此,等我慢慢替你访此贼人是谁,再作理会。”安童拜谢公公,遂在此翁家过活。

一日,也是合当有事。年除岁末,渔翁忽带安童正出河口卖鱼,正撞见陈三、翁八在船上饮酒,穿着他主人衣服,上岸来买鱼。安童认得,即密与渔翁说道:“主人之冤当雪矣。”渔翁道:“何不具状官司处告理?”安童将情具告到巡河周守备府内。守备见没赃证,不接状子。又告到提刑院。夏提刑见是强盗劫杀人命等事,把状批行了。从正月十四日差缉捕公人,押安童下来拿人。前至新河口,只把陈三、翁八获住到案,责问了口词。二艄见安童在旁执证,也没得动刑,一一招了。供称:“下手之时,还有他家人苗青,同谋杀其家主,分赃而去。”这里把三人监下,又差人访拿苗青,一起定罪。因节间放假,提刑官吏一连两日没来衙门中问事,早有衙门透信的人,悄悄把这件事儿报与苗青。苗青慌了,把店门锁了,暗暗躲在经纪乐三家。

这乐三就住在狮子街韩道国家隔壁,他浑家乐三嫂,与王六儿所交极厚,常过王六儿这边来做伴儿。王六儿无事,也常往他家行走,彼此打的热闹。这乐三见苗青面带忧容,问其所以,说道:“不打紧,间壁韩家就是提刑西门老爹的外室,又是他家伙计,和俺家交往的甚好,几事百依百随,若要保得你无事,破多少东西,教俺家过去和他家说说。”这苗青听了,连忙下跪,说道:“但得我身上没事,恩有重报,不敢有忘。”于是写了说帖,封下五十两银子,两套妆花缎子衣服,乐三教他老婆拿过去,如此这般对王六儿说。王六儿喜欢的要不的,把衣服银子并说帖都收下,单等西门庆,不见来。

到十七日日西时分,只见玳安夹着毡包,骑着头口,从街心里来。王六儿在门首,叫下来问道:“你往那里去来?”玳安道:“我跟爹走了个远差,往东平府送礼去来。”王六儿道:“你爹如今来了不曾?”玳安道:“爹和贲四两个先往家去了。”王六儿便叫进去,和他如此这般说话,拿帖儿与他瞧,玳安道:“韩大婶,管他这事!休要把事轻看了,如今衙门里监着那两个船家,供着只要他哩。拿过几两银子来,也不够打发脚下人哩。我不管别的帐,韩大婶和他说,只与我二十两银子罢。等我请将俺爹来,随你老人家与俺爹说就是了。”王六儿笑道:“怪油嘴儿,要饭吃休要恶了火头。事成了,你的事甚么打紧?宁可我们不要,也少不得你的。”玳安道:“韩大婶,不是这等说。常言:君子不羞当面。先断过,后商量。”王六儿当下备几样菜,留玳安吃酒。玳安道:“吃的红头红脸,怕家去爹问,却怎的回爹?”王六儿道:“怕怎的?你就说在我这里来。”玳安只吃了一瓯子,就走了。王六儿道:“好歹累你,说是我这里等着哩。”

玳安一直来家,交进毡包。等的西门庆睡了一觉出来,在厢房中坐的。这玳安慢慢走到跟前,说:“小的回来,韩大婶叫住小的,要请爹快些过去,有句要紧话和爹说。”西门庆说:“甚么话?我知道了。”说毕,正值刘学官来借银子。打发刘学官去了,西门庆骑马,带着眼纱、小帽,便叫玳安、琴童两个跟随,来到王六儿家。下马进去,到明间坐下,王六儿出来拜见了。那日,韩道国铺子里上宿,没来家。老婆买了许多东西,叫老冯厨下整治。见西门庆来了,慌忙递茶。西门庆吩咐琴童:“把马送到对门房子里去,把大门关上。”妇人且不敢就题此事,先只说:“爹家中连日摆酒辛苦。我闻得说哥儿定了亲事,你老人家喜呀!”西门庆道:“只因舍亲吴大妗那里说起,和乔家做了这门亲事。他家也只这一个女孩儿,论起来也还不般配,胡乱亲上做亲罢了。”王六儿道:“就是和他做亲也好,只是爹如今居着恁大官,会在一处,不好意思的。”西门庆道:“说甚么哩!”说了一回,老婆道:“只怕爹寒冷,往房里坐去罢。”一面让至房中,一面安着一张椅儿,笼着火盆,西门庆坐下。妇人慢慢先把苗青揭帖拿与西门庆看,说:“他央了间壁经纪乐三娘子过来对我说:这苗青是他店里客人,如此这般,被两个船家拽扯,只望除豁了他这名字,免提他。他备了些礼儿在此谢我。好歹望老爹怎的将就他罢。”西门庆看了帖子,因问:“他拿了多少礼物谢你?”王六儿向箱中取出五十两银子来与西门庆瞧,说道:“明日事成,还许两套衣裳。”西门庆看了,笑道:“这些东西儿,平白你要他做甚么?你不知道,这苗青乃扬州苗员外家人,因为在船上与两个船家杀害家主,撺在河里,图财谋命。如今见打捞不着尸首,他原跟来的一个小厮安童与两个船家,当官三口执证着要他。这一拿去,稳定是个凌迟罪名。那两个都是真犯斩罪。两个船家见供他有二千两银货在身上。拿这些银子来做甚么?还不快送与他去!”这王六儿一面到厨下,使了丫头锦儿把乐三娘子儿叫了来,将原礼交付与他,如此这般对他说了去。

那苗青不听便罢,听他说了,犹如一桶水顶门上直灌到脚底下。正是:

\[
惊开六叶连肝肺,唬坏三魂七魄心。
\]
即请乐三一处商议道:“宁可把二千货银都使了,只要救得性命家去。”乐三道:“如今老爹上边既发此言,一些半些恒属打不动。两位官府,须得凑一千货物与他。其余节级、原解、缉捕,再得一半,才得够用。”苗青道:“况我货物未卖,那讨银子来?”因使过乐三嫂来,和王六儿说:“老爹就要货物,发一千两银子货与老爹。如不要,伏望老爹再宽限两三日,等我倒下价钱,将货物卖了,亲往老爹宅里进礼去。”王六儿拿礼帖复到房里与西门庆瞧。西门庆道:“既是恁般,我吩咐原解且宽限他几日,教他即便进礼来。”当下乐三子得此口词,回报苗青,苗青满心欢喜。西门庆见间壁有人,也不敢久坐,吃了几钟酒,与老婆坐了回,见马来接,就起身家去了。

次日,到衙门早发放,也不题问这件事。这苗青就托经纪乐三,连夜替他会了人,撺掇货物出去。那消三日,都发尽了,共卖了一千七百两银子。把原与王六儿的不动,又另加上五十两银子、四套上色衣服。到十九日,苗青打点一千两银子,装在四个酒坛内,又宰一口猪。约掌灯以后,抬送到西门庆门首。手下人都是知道的,玳安、平安、书童、琴童四个家人,与了十两银子才罢。玳安在王六儿这边,梯已又要十两银子。须臾,西门庆出来,卷棚内坐的,也不掌灯,月色朦胧才上来,抬至当面。苗青穿青衣,望西门庆只顾磕头,说道:“小人蒙老爹超拔之恩,粉身碎骨难报。”西门庆道:“你这件事情,我也还没好审问哩。那两个船家甚是攀你,你若出官,也有老大一个罪名。既是人说,我饶了你一死。此礼我若不受你的,你也不放心。我还把一半送你掌刑夏老爹,同做分上。你不可久住,即便星夜回去。”因问:“你在扬州那里?”苗青磕头道:“小的在扬州城内住。”西门庆吩咐后边拿了茶来,那苗青在松树下立着吃了,磕头告辞回去。又叫回来问:“下边原解的,你都与他说了不曾?”苗青道:“小的外边已说停当了。”西门庆吩咐:“既是说了,你即回家。”那苗青出门,走到乐三家收拾行李,还剩一百五十两银子。苗青拿出五十两来,并余下几匹缎子,都谢了乐三夫妇。五更替他雇长行牲口,起身往扬州去了。正是:

\[
忙忙如丧家之狗,急急似漏网之鱼。
\]

不说苗青逃出性命去了。单表次日,西门庆、夏提刑从衙门中散了出来,并马而行。走到大街口上,夏提刑要作辞分路,西门庆在马上举着马鞭儿说道:“长官不弃,到舍下一叙。”把夏提刑邀到家来。进到厅上叙礼,请入卷棚里,宽了衣服,左右拿茶吃了。书童、玳安就安放桌席。夏提刑道:“不当闲来打搅长官。”西门庆道:“岂有此理。”须臾,两个小厮用方盒摆下各样鸡、蹄、鹅、鸭、鲜鱼下饭。先吃了饭,收了家伙去,就是吃酒的各样菜蔬出来。小金钟儿,银台盘儿,慢慢斟劝。饮酒中间,西门庆方题起苗青的事来,道:“这厮昨日央及了个士夫,再三来对学生说,又馈送了些礼在此。学生不敢自专,今日请长官来,与长官计议。”于是,把礼帖递与夏提刑。夏提刑看了,便道:“恁凭长官尊意裁处。”西门庆道:“依着学生,明日只把那个贼人、真赃送过去罢,也不消要这苗青。那个原告小厮安童,便收领在外,待有了苗天秀尸首,归结未迟。礼还送到长官处。”夏提刑道:“长官,这就不是了。长官见得极是,此是长官费心一番,何得见让于我?决然使不得。”彼此推辞了半日,西门庆不得已,还把礼物两家平分了,装了五百两在食盒内。夏提刑下席来,作揖谢道:“既是长官见爱,我学生再辞,显的迂阔了。盛情感激不尽,实为多愧。”又领了几杯酒,方才告辞起身。西门庆随即差玳安拿食盒,还当酒抬送到夏提刑家。夏提刑亲在门上收了,拿回帖,又赏了玳安二两银子,两名排军四钱,俱不在话下。

常言道:火到猪头烂,钱到公事办。西门庆、夏提刑已是会定了。次日到衙门里升厅,那提控、节级并缉捕、观察,都被乐三上下打点停当。摆设下刑具,监中提出陈三、翁八审问情由,只是供称:“跟伊家人苗青同谋。”西门庆大怒,喝令左右:“与我用起刑来!你两个贼人,专一积年在江河中,假以舟楫装载为名,实是劫帮凿漏,邀截客旅,图财致命。见有这个小厮供称,是你等持刀戮死苗天秀波中,又将棍打伤他落水,见有他主人衣服存证,你如何抵赖别人!”因把安童提上来,问道:“是谁刺死你主人?是谁推你在水中?”安童道:“某日三更时分,先是苗青叫有贼,小的主人出舱观看,被陈三一刀戮死,推下水去。小的便被翁八一棍打落水中,才得逃出性命。苗青并不知下落。”西门庆道:“据这小厮所言,就是实话,汝等如何展转得过?”于是每人两夹棍,三十榔头,打的胫骨皆碎,杀猪也似喊叫。一千两赃货已追出大半,余者花费无存。这里提刑做了文书,并赃货申详东平府。府尹胡师文又与西门庆相交,照原行文书叠成案卷,将陈三、翁八问成强盗杀人斩罪。

安童保领在外听候。有日走到东京,投到开封府黄通判衙内,具诉:“苗青夺了主人家事,使钱提刑衙门,除了他名字出来。主人冤仇,何时得报?”通判听了,连夜修书,并他诉状封在一处,与他盘费,就着他往巡按山东察院里投下。这一来,管教苗青之祸从头上起,西门庆往时做过事,今朝没兴一齐来。有诗为证:

\[
善恶从来报有因,吉凶祸福并肩行。
平生不作亏心事,夜半敲门不吃惊。
\]

\newpage
%# -*- coding:utf-8 -*-
%%%%%%%%%%%%%%%%%%%%%%%%%%%%%%%%%%%%%%%%%%%%%%%%%%%%%%%%%%%%%%%%%%%%%%%%%%%%%%%%%%%%%


\chapter{弄私情戏赠一枝桃\KG 走捷径探归七件事}


词曰:

\[
碧桃花下,紫箫吹罢。蓦然一点心惊,却把那人牵挂,向东风泪洒。东风泪洒,不觉暗沾罗帕,恨如天大。那冤家既是无情去,回头看怎么!
\]

话说安童领着书信,辞了黄通判,径往山东大道而来。打听巡按御史在东昌府住扎,姓曾,双名孝序,乃都御史曾布之子,新中乙未科进士,极是个清廉正气的官。这安童自思:“我若说下书的,门上人决不肯放。不如等放告牌出来,我跪门进去,连状带书呈上。老爹见了,必然有个决断。”于是早把状子写下,揣在怀里,在察院门首等候多时。只听里面打的云板响,开了大门,曾御史坐厅。头面牌出来,大书告亲王、皇亲、驸马、势豪之家;第二面牌出来,告都、布、按并军卫有司官吏;第三面牌出来,才是百姓户婚田土词讼之事。这安童就随状牌进去,待把一应事情发放净了,方走到丹墀上跪下。两边左右问是做甚么的,这安童方才把书双手举得高高的呈上。只听公座上曾御史叫:“接上来!”慌的左右吏典下来把书接上去,安放于书案上。曾公拆开观看,端的上面写着甚言词?书曰:

\[
寓都下年教生黄端肃书奉大柱史少亭曾年兄先生大人门下:违越光仪,倏忽一载。知己难逢,胜游易散。此心耿耿,常在左右。去秋忽报瑶章,开轴启函,捧诵之间而神游恍惚,俨然长安对面时也。未几,年兄省亲南旋,复闻德音,知年兄按巡齐鲁,不胜欣慰。叩贺,叩贺。惟年兄忠孝大节,风霜贞操,砥砺其心,耿耿在廊庙,历历在士论。今兹出巡,正当摘发官邪,以正风纪之日。区区爱念,尤所不能忘者矣。窃谓年兄平日抱可为之器,当有为之年,值圣明有道之世,老翁在家康健之时,当乘此大展才猷,以振扬法纪,勿使舞文之吏以挠其法,而奸顽之徒以逞其欺。胡乃如东平一府,而有挠大法如苗青者,抱大冤如苗天秀者乎?生不意圣明之世而有此魍魉。年兄巡历此方,正当分理冤滞,振刷为之一清可也。去伴安童,持状告诉,幸垂察,不宣。\named{仲春望后一日具}
\]

这曾御史览书已毕,便问:“有状没有?”左右慌忙下来问道:“老爷问你有状没有。”这安童向怀中取状递上。曾公看了,取笔批:“仰东平府府官,从公查明,验相尸首,连卷详报。”喝令安童东平府伺候。这安童连忙磕头起来,从便门放出。

这里曾公将批词连状装在封套内,钤了关防,差人赍送东平府来。府尹胡师文见了上司批下来,慌得手脚无措,即调委阳谷县县丞狄斯彬——本贯河南舞阳人氏,为人刚方不要钱,问事糊突,人都号他做狄混。先是这狄县丞往清河县城西河边过,忽见马头前起一阵旋风,团团不散,只随着狄公马走。狄县丞道:“怪哉!”便勒住马,令左右公人:“你随此旋风,务要跟寻个下落。”那公人真个跟定旋风而来,七八将近新河口而止,走来回覆了狄公话。狄公即拘集里老,用锹掘开岸上数尺,见一死尸,宛然颈上有一刀痕。命仵作检视明白,问其前面是那里。公人禀道:“离此不远就是慈惠寺。”县丞即拘寺中僧行问之,皆言:“去冬十月中,本寺因放水灯儿,见一死尸从上流而来,漂入港里。长老慈悲,故收而埋之。不知为何而死。”县丞道:“分明是汝众僧谋杀此人,埋于此处。想必身上有财帛,故不肯实说。”于是不由分说,先把长老一箍两拶,一夹一百敲,余者众僧都是二十板,俱令收入狱中。报与曾公,再行查看。各僧皆称冤不服。曾公寻思道:“既是此僧谋死,尸必弃于河中,岂反埋于岸上?又说干碍人众,此有可疑。”因令将众僧收监。将近两月,不想安童来告此状。即令委官押安童前至尸所,令其认视。安童见尸大哭道:“正是我的主人,被贼人所伤,刀痕尚在。”于是检验明白,回报曾公,即把众僧放回。一面查刷卷宗,复提出陈三、翁八审问,俱执称苗青主谋之情。曾公大怒,差人行牌,星夜往扬州提苗青去了。一面写本参劾提刑院两员问官受赃卖法。正是:

\[
污吏赃官滥国刑,曾公判刷雪冤情。
虽然号令风霆肃,梦里输赢总未真。
\]

话分两头,却表王六儿自从得了苗青干事的那一百两银子、四套衣服,与他汉子韩道国就白日不闲,一夜没的睡,计较着要打头面,治簪环,唤裁缝来裁衣服,从新抽银丝\textuni{4BFC}髻。用十六两银子,又买了个丫头——名唤春香——使唤,早晚教韩道国收用不题。

一日,西门庆到韩道国家,王六儿接着。里面吃茶毕,西门庆往后边净手去,看见隔壁月台,问道:“是谁家的?”王六儿道:“是隔壁乐三家月台。”西门庆吩咐王六儿:“如何教他遮住了这边风水?你对他说,若不与我即便拆了,我教地方吩咐他。”这王六儿与韩道国说:“邻舍家,怎好与他说的。”韩道国道:“咱不如瞒着老爹,买几根木植来,咱这边也搭起个月台来。上面晒酱,下边不拘做马坊,做个东净,也是好处。”老婆道:“呸!贼没算计的。比时搭月台,不如买些砖瓦来,盖上两间厦子却不好?”韩道国道:“盖两间厦子,不如盖一层两间小房罢。”于是使了三十两银子,又盖两间平房起来。西门庆差玳安儿抬了许多酒、肉、烧饼来,与他家犒赏匠人。那条街上谁人不知。

夏提刑得了几百两银子在家,把儿子夏承恩——年十八岁——干入武学肄业,做了生员。每日邀结师友,习学弓马。西门庆约会刘薛二内相、周守备、荆都监、张团练、合卫官员,出人情与他挂轴文庆贺,俱不必细说。

西门庆因坟上新盖了山子卷棚房屋,自从生了官哥,并做了千户,还没往坟上祭祖。叫阴阳徐先生看了,从新立了一座坟门,砌的明堂神路,门首栽桃柳,周围种松柏,两边叠成坡峰。清明日上坟,要更换锦衣牌匾,宰猪羊,定桌面。三月初六日清明,预先发柬,请了许多人,搬运了东西、酒米、下饭、菜蔬,叫的乐工、杂耍、扮戏的。小优儿是李铭、吴惠、王柱、郑奉;唱的是李桂姐、吴银儿、韩金钏,董娇儿。官客请了张团练、乔大户、吴大舅、吴二舅、花大舅、沈姨夫、应伯爵、谢希大、傅伙计、韩道国、云理守、贲第传并女婿陈敬济等,约二十余人。堂客请了张团练娘子、张亲家母、乔大户娘子、朱台官娘子、尚举人娘子、吴大妗子、二妗子、杨姑娘、潘姥姥、花大妗子、吴大姨、孟大姨、吴舜臣媳妇郑三姐、崔本妻段大姐,并家中吴月娘、李娇儿,孟玉楼、潘金莲、李瓶儿、孙雪娥、西门大姐、春梅、迎春、玉箫、兰香、奶子如意儿抱着官哥儿,里外也有二十四五顶轿子。先是月娘对西门庆说:“孩子且不消教他往坟上去罢。一来还不曾过一周,二者刘婆子说这孩子\textuni{29544}门还未长满,胆儿小。这一到坟上路远,只怕唬着他。依着我不教他去,留下奶子和老冯在家和他做伴儿,只教他娘母子一个去罢。”西门庆不听,便道:“此来为何?他娘儿两个不到坟前与祖宗磕个头儿去!你信那婆子老淫妇胡说,可可就是孩子\textuni{29544}门未长满,教奶子用被儿裹着,在轿子里按的孩儿牢牢的,怕怎的?”那月娘便道:“你不听人说,随你。”从清早晨,堂客都从家里取齐,起身上了轿子,无辞。

出南门,到五里外祖坟上,远远望见青松郁郁,翠柏森森,新盖的坟门,两边坡峰上去,周围石墙,当中甬道,明堂、神台、香炉、烛台都是白玉石凿的。坟门上新安的牌匾,大书“锦衣武略将军西门氏先茔”。坟内正面土山环抱,林树交枝。西门庆穿大红冠带,摆设猪羊祭品桌席祭奠。官客祭毕,堂客才祭。响器锣鼓,一齐打起来。那官哥儿唬的在奶子怀里磕伏着,只倒咽气,不敢动一动儿。月娘便叫:“李大姐,你还不教奶子抱了孩子往后边去哩,你看唬的那腔儿!我说且不教孩儿来罢,恁强的货,只管教抱了他来。你看唬的那孩儿这模样!”李瓶儿连忙下来,吩咐玳安:“且叫把锣鼓住了。”连忙撺掇掩着孩儿耳朵,快抱了后边去了。

须臾,祭毕,徐先生念了祭文,烧了纸。西门庆邀请官客在前客位。月娘邀请堂客在后边卷棚内,由花园进去,两边松墙竹径,周围花草,一望无际。正是:

\[
桃红柳绿莺梭织,都是东君造化成。
\]

当下,扮戏的在卷棚内扮与堂客们瞧,四个小优儿在前厅官客席前弹唱。四个唱的,轮番递酒。春梅、玉箫、兰香、迎春四个,都在堂客上边执壶斟酒,就立在大姐桌头,同吃汤饭点心。

吃了一回,潘金莲与玉楼、大姐、李桂姐、吴银儿同往花园里打了回秋千。原来卷棚后边,西门庆收拾了一明两暗三间房儿。里边铺陈床帐,摆放桌椅、梳笼、抿镜、妆台之类,预备堂客来上坟,在此梳妆歇息,糊的犹如雪洞般干净,悬挂的书画,琴棋潇洒。奶子如意儿看守官哥儿,正在那洒金床炕上铺着小褥子儿睡,迎春也在旁和他顽耍。只见潘金莲独自从花园蓦地走来,手中拈着一枝桃花儿,看见迎春便道:“你原来这一日没在上边伺候。”迎春道:“有春梅、兰香、玉箫在上边哩,俺娘叫我下边来看哥儿,就拿了两碟下饭点心与如意儿吃。”奶子见金莲来,就抱起官哥儿来。金莲便戏他说道:“小油嘴儿,头里见打起锣鼓来,唬的不则声,原来这等小胆儿。”于是一面解开藕丝罗袄儿,接过孩儿抱在怀里,与他两个嘴对嘴亲嘴儿。忽有陈敬济掀帘子走入来,看见金莲逗孩子顽耍,便也逗那孩子。金莲道:“小道士儿,你也与姐夫亲个嘴儿。”可霎作怪,那官哥儿便嘻嘻望着他笑。敬济不由分说,把孩子就搂过来,一连亲了几个嘴。金莲骂道:“怪短命,谁家亲孩子,把人的鬓都抓乱了!”敬济笑戏道:“你还说,早时我没错亲了哩。”金莲听了,恐怕奶子瞧科,便戏发讪,将手中拿的扇子倒过柄子来,向他身上打了一下,打的敬济鲫鱼般跳。骂道:“怪短命,谁和你那等调嘴调舌的!”敬济道:“不是,你老人家摸量惜些情儿。人身上穿着恁单衣裳,就打恁一下!”金莲道:“我平自惜甚情儿?今后惹着我,只是一味打。”如意儿见他顽的讪,连忙把官哥儿接过来抱着,金莲与敬济两个还戏谑做一处。金莲将那一枝桃花儿做了一个圈儿,悄悄套在敬济帽子上。走出去,正值孟玉楼和大姐、桂姐三个从那边来。大姐看见,便问:“是谁干的营生?”敬济取下来去了,一声儿也没言语。堂客前戏文扮了四大折。但见:

\[
窗外日光弹指过,席前花影座间移。
\]

看看天色晚来,西门庆吩咐贲四,先把抬轿子的每人一碗酒、四个烧饼、一盘子熟肉,分散停当,然后,才把堂客轿子起身。官家起马在后,来兴儿与厨役慢慢的抬食盒煞后。玳安、来安、画童、棋童儿跟月娘众人轿子,琴童并四名排军跟西门庆马。奶子如意儿独自坐一顶小轿,怀中抱着哥儿,用被裹得紧紧的进城。月娘还不放心,又使回画童儿来,叫他跟定着奶子轿子,恐怕进城人乱。

且说月娘轿子进了城,就与乔家那边众堂客轿子分路,来家先下轿进去,半日西门庆、陈敬济才到家下马。只见平安儿迎门就禀说:“今日掌刑夏老爹,亲自下马到厅,问了一遍去了。落后又差人问了两遍。不知有甚勾当。”西门庆听了,心中犹豫。到于厅上,只见书童儿在旁接衣服。西门庆因问:“今日你夏老爹来,留下甚么话来?”书童道:“他也没说出来,只问爹往那去了:‘使人请去,我有句要紧话儿说。’小的便道:‘今日都往坟上烧纸去了,至晚才来。’夏老爹说:‘我到午上还来。’落后又差人来问了两遭,小的说:‘还未来哩!’”西门庆心下转道:“却是甚么?”

正疑惑之间,只见平安来报:“夏老爹来了。”那时已有黄昏时分,只见夏提刑便衣坡巾,两个伴当跟随。下马到于厅上叙礼,说道:“长官今日往宝庄去来?”西门庆道:“今日先茔祭扫,不知长官下降,失迎,恕罪,恕罪!”夏提刑道:“有一事敢来报与长官知道。”因说:“咱们往那边客位内坐去罢。”西门庆令书童开卷棚门,请往那里说话,左右都令下去。夏提刑道:“今朝县中李大人到学生那里,如此这般,说大巡新近有参本上东京,长官与学生俱在参例。学生令人抄了个底本在此,与长官看。”西门庆听了,大惊失色,急接过邸报来灯下观看,端的上面写着甚言词?

\[
巡按山东监察御史曾孝序一本,参劾贪肆不职武官,乞赐罢黜,以正法纪事:臣闻巡搜四方,省察风俗,乃天子巡狩之事也;弹压官邪,振扬法纪,乃御史纠政之职也。昔《春秋》载天王巡狩,而万邦怀保,民风协矣,王道彰矣,四民顺矣,圣治明矣。臣自去年奉命巡按山东齐鲁之邦,一年将满,历访方面有司文武官员贤否,颇得其实。兹当差满之期,敢不循例甄别,为我皇上陈之!除参劾有司方面官员,另具疏上请。参照山东提刑所掌刑金吾卫正千户夏延龄,\textuni{26D91}茸之材,贪鄙之行,久于物议,有玷班行。昔者典牧皇畿,大肆科扰,被属官阴发其私。今省理山东刑狱,复著狼贪,为同僚之箝制。纵子承恩冒籍武举,倩人代考,而士风扫地矣。信家人夏寿监索班钱,被军腾詈而政事不可知乎!接物则奴颜婢膝,时人有丫头之称;问事则依违两可,群下有木偶之诮。理刑副千户西门庆,本系市井棍徒,夤缘升职,滥冒武功,菽麦不知,一丁不识。纵妻妾嬉游街巷而帷薄为之不清;携乐妇而酣饮市楼,官箴为之有玷。至于包养韩氏之妇,恣其欢淫,而行检不修;受苗青夜赂之金,曲为掩饰,而赃迹显著。此二臣者,皆贪鄙不职,久乖清议,一刻不可居任者也。伏望圣明垂听,敕下该部,再加详查。如果臣言不谬,将延龄等亟赐罢斥,则官常有赖而俾圣德永光矣。
\]

西门庆看了一遍,唬的面面相觑,默默不言。夏提刑道:“长官,似此如何计较?”西门庆道:“常言:兵来将挡,水来土掩。事到其间,道在人为。少不的你我打点礼物,早差人上东京央及老爷那里去。”于是,夏提刑急急作辞,到家拿了二百两银子、两把银壶。西门庆这里是金镶玉宝石闹妆一条、三百两银子。夏家差了家人夏寿,西门庆这里是来保,将礼物打包端正,西门庆写了一封书与翟管家,两个早雇了头口,星夜往东京干事去了,不题。

且表官哥儿自从坟上来家,夜间只是惊哭,不肯吃奶。但吃下奶去就吐了。慌的李瓶儿走来告诉月娘,月娘道:“我那等说,还未到一周的孩子,且休带他出城门去。浊漒货他生死不依,只说:‘今日坟上祭祖为甚么来?不教他娘儿两个走走!’只象那里搀了分儿一般,睁着眼和我两个叫。如今却怎么好?”李瓶儿正没法儿摆布。况西门庆又因巡按参了,和夏提刑在前边说话,往东京打点干事,心上不遂,家中孩子又不好。月娘使小厮叫婆子来看,又请小儿科太医,开门阖户,乱了一夜。刘婆子看了说:“哥儿着了些惊气入肚,又路上撞见五道将军。不打紧,买些纸儿退送退送就好了。”又留了两服朱砂丸药儿,用薄荷灯心汤送下去,那孩儿方才宁贴睡了一觉,不惊哭吐奶了。只是身上热还未退,李瓶儿连忙拿出一两银子,教刘婆子备纸去。后又带了他老公,还和一个师婆来,在卷棚内与哥儿烧纸跳神。那西门庆早五更打发来保、夏寿起身,就乱着和夏提刑往东平府胡知府那里打听提苗青消息去了。吴月娘听见刘婆说孩子路上着了惊气,甚是抱怨如意儿,说他:“不用心看孩儿,想必路上轿子里唬了他了。不然,怎的就不好起来?”如意儿道:“我在轿子里,将被儿包得紧紧的,又没\textShiDian 着他。娘叫画童儿来跟着轿子,他还好好的,我按着他睡。只进城七八到家门首,我只觉他打了个冷战,到家就不吃奶,哭起来了。”

按下这里家中烧纸,与孩子下神。且说来保、夏寿一路攒行,只六日就赶到东京城内。到太师府内见了翟管家,将两家礼物交割明白。翟谦看了西门庆书信,说道:“曾御史参本还未到哩,你且住两日。如今老爷新近条陈了七件事,旨意还未曾下来。待行下这个本去,曾御史本到,等我对老爷说,交老爷阁中只批与他‘该部知道’。我这里差人再拿帖儿吩咐兵部余尚书,把他的本只不覆上来。交你老爹只顾放心,管情一些事儿没有。”于是把二人管待了酒饭,还归到客店安歇,等听消息。

一日蔡太师条陈本,圣旨准下来了。来保央府中门吏暗暗抄了个邸报,带回家与西门庆瞧,不在话下。一日等的翟管家写了回书,与了五两盘缠,与夏寿取路回山东清河县。来到家中,西门庆正在家耽心不下,那夏提刑一日一遍来问信。听见来保二人到了,叫至后边问他端的。来保对西门庆悉把上项事情诉说一遍,道:“翟爹看了爹的书,便说:‘此事不打紧,教你爹放心。见今巡按也满了,另点新巡按下来了。况他的参本还未到,等他本上时,等我对老爷说了,随他本上参的怎么重,只批该部知道,老爷这里再拿帖儿吩咐兵部余尚书,只把他的本立了案不覆上去,随他有拨天关本事也无妨。’”西门庆听了,方才心中放下。因问:“他的本怎还不到?”来保道:“俺们一去时,昼夜马上行去,只五日就赶到京中,可知在他头里。俺每回来,见路上一簇响铃驿马,背着黄色袱,插着两根雉尾、两面牙旗,怕不就是巡按衙门进送实封才到了。”西门庆道:“得他的本上的迟,事情就停当了。我只怕去迟了。”来保道:“爹放心,管情没事。小的不但干了这件事,又打听得两桩好事来,报爹知道。”西门庆问道:“端的何事?”来保道:“太师老爷新近条陈了七件事,旨意已是准行。如今老爷亲家户部侍郎韩爷题准事例:在陕西等三边开引种盐,各府州郡县设立义仓,官粜粮米。令民间上上之户赴仓上米,讨仓钞,派给盐引支盐。旧仓钞七分,新仓钞三分。咱旧时和乔亲家爹,高阳关上纳的那三万粮仓钞,派三万盐引,户部坐派。如今蔡状元又点了两淮巡盐,不日离京,倒有好些利息。”西门庆听言问道:“真个有此事?”来保道:“爹不信,小的抄了个邸报在此。”向书箧中取出来与西门庆观看。因见上面许多字样,前边叫了陈敬济来念与他听。陈敬济念到中间,只要结住了,还有几个眼生字不认的。旋叫了书童儿来念。那书童倒还是门子出身,荡荡如流水不差,直念到底。端的上面奏着那七件事?

\[
崇政殿大学士吏部尚书鲁国公蔡京一本,为陈愚见,竭愚衷,收人才,臻实效,足财用,便民情,以隆圣治事:
第一曰罢科举,取士悉由学校升贡。窃谓教化凌夷,风俗颓败,皆由取士不得真才,而教化无以仰赖。《书》曰:“天生斯民,作之君,作之师。”汉举孝廉,唐兴学校,我国家始制考贡之法,各执偏陋,以致此辈无真才,而民之司牧何以赖焉?今皇上寤寐求才,宵旰图治。治在于养贤,养贤莫如学校。今后取士,悉遵古由学校升贡。其州县发解礼闱,一切罢之。每岁考试上舍则差知贡举,亦如礼闱之式。仍立八行取士之科。八行者,谓孝、友、睦、姻、任、恤、忠、和也。士有此者,即免试,率相补太学上舍。
二曰罢讲议财利司。窃惟国初定制,都堂置讲议财利司。盖谓人君节浮费,惜民财也。今陛下即位以来,不宝远物,不劳逸民,躬行节俭以自奉。盖天下亦无不可返之俗,亦无不可节之财。惟当事者以俗化为心,以禁令为信,不忽其初,不弛其后,治隆俗美,丰亨豫大,又何讲议之为哉?悉罢。
三曰更盐钞法。窃惟盐钞,乃国家之课以供边备者也。今合无遵复祖宗之制盐法者。诏云中、陕西、山西三边,上纳粮草,关领旧盐钞,易东南淮浙新盐钞。每钞折派三分,旧钞搭派七分。今商人照所派产盐之地下场支盐。亦如茶法,赴官秤验,纳息请批引,限日行盐之处贩卖。如遇过限,并行拘收;别买新引增贩者,俱属私盐。如此则国课日增,而边储不乏矣。
四曰制钱法。窃谓钱货,乃国家之血脉,贵乎流通而不可淹滞。如有厄阻淹滞不行者,则小民何以变通,而国课何以仰赖矣?自晋末鹅眼钱之后,至国初琐屑不堪,甚至杂以铅铁夹锡。边人贩于虏,因而铸兵器,为害不小,合无一切通行禁之也。以陛下新铸大钱崇宁、大观通宝,一以当十,庶小民通行,物价不致于踊贵矣。
五曰行结粜俵籴之法。窃惟官籴之法,乃赈恤之义也。近年水旱相仍,民间就食,上始下赈恤之诏。近有户部侍郎韩侣题覆钦依:将境内所属州县各立社会,行结粜俵籴之法。保之于党,党之于里,里之于乡,倡之结也。每乡编为三户,按上上、中中、下下。上户者纳粮,中户者减半,下户者退派粮数关支,谓之俵粜。如此则敛散便民之法得以施行,而皇上可广不费之仁矣。惟责守令核切举行,其关系盖匪细矣。
六曰诏天下州郡纳免夫钱。窃惟我国初寇乱未定,悉令天下军徭丁壮集于京师,以供运馈,以壮国势。今承平日久,民各安业,合颁诏行天下州郡,每岁上纳免夫钱,每名折钱三十贯,解赴京师,以资边饷之用。庶两得其便,而民力少苏矣。
七曰置提举御前人船所。窃惟陛下自即位以来,无声色犬马之奉。所尚花石,皆山林间物,乃人之所弃者。但有司奉行之过因而致扰,有伤圣治。陛下节其浮滥,仍请作御前提举人船所。凡有用悉出内帑,差官取之,庶无扰于州郡。伏乞圣裁。
奉旨曰:“卿言深切时艰,朕心嘉悦,足见忠猷,都依拟行。”该部知道。
\]

西门庆听了,又看了翟管家书信,已知礼物交得明白。蔡状元见朝,又点了两淮巡盐,不日往此经过,心中不胜欢喜。一面打发夏寿回家:“报与你老爹知道。”一面赏了来保五两银子、两瓶酒、一方肉,回房歇息,不在话下。正是:树大招风风损树,人为名高名丧身。有诗为证:

\[
得失荣枯命里该,皆因年月日时栽。
胸中有志终须至,囊内无财莫论才。
\]

\newpage
%# -*- coding:utf-8 -*-
%%%%%%%%%%%%%%%%%%%%%%%%%%%%%%%%%%%%%%%%%%%%%%%%%%%%%%%%%%%%%%%%%%%%%%%%%%%%%%%%%%%%%


\chapter{请巡按屈体求荣\KG 遇胡僧现身施药}


诗曰:

\[
雅集无兼客,高情洽二难。一尊倾智海,八斗擅吟坛。
话到如生旭,霜来恐不寒。为行王舍乞,玄屑带云餐。
\]

话说夏寿到家回复了话,夏提刑随即就来拜谢西门庆,说道:“长官活命之恩,不是托赖长官余光这等大力量,如何了得!”西门庆笑道:“长官放心。料着你我没曾过为,随他说去,老爷那里自有个明见。”一面在厅上放桌儿留饭,谈笑至晚,方才作辞回家。到次日,依旧入衙门里理事,不在话下。

却表巡按曾公见本上去不行,就知道二官打点了,心中忿怒。因蔡太师所陈七事,内多舛讹,皆损下益上之事,即赴京见朝覆命,上了一道表章。极言:“天下之财贵于通流,取民膏以聚京师,恐非太平之治。民间结粜俵籴之法不可行,当十大钱不可用,盐钞法不可屡更。臣闻民力殚矣,谁与守邦?”蔡京大怒,奏上徽宗天子,说他大肆倡言,阻挠国事。将曾公付吏部考察,黜为陕西庆州知州。陕西巡按御史宋盘,就是学士蔡攸之妇兄也。太师阴令盘就劾其私事,逮其家人,锻炼成狱,将孝序除名,窜于岭表,以报其仇。此系后事,表过不题。

再说西门庆在家,一面使韩道国与乔大户外甥崔本,拿仓钞早往高阳关户部韩爷那里赶着挂号。留下来保家中定下果品,预备大桌面酒席,打听蔡御史船到。一日,来保打听得他与巡按宋御史船一同京中起身,都行至东昌府地方,使人来家通报。这里西门庆就会夏提刑起身。来保从东昌府船上就先见了蔡御史,送了下程。然后,西门庆与夏提刑出郊五十里迎接到新河口——地名百家村。先到蔡御史船上拜见了,备言邀请宋公之事。蔡御史道:“我知道,一定同他到府。”那时,东平胡知府,及合属州县方面有司军卫官员、吏典生员、僧道阴阳,都具连名手本,伺候迎接。帅府周守备、荆都监、张团练,都领人马披执跟随,清跸传道,鸡犬皆隐迹。鼓吹迎接宋巡按进东平府察院,各处官员都见毕,呈递了文书,安歇一夜。

到次日,只见门吏来报:“巡盐蔡爷来拜。”宋御史连忙出迎。叙毕礼数,分宾主坐下。献茶已毕,宋御史便问:“年兄几时方行?”蔡御史道:“学生还待一二日。”因告说:“清河县有一相识西门千兵,乃本处巨族,为人清慎,富而好礼,亦是蔡老先生门下,与学生有一面之交。蒙他远接,学生正要到他府上拜他拜。”宋御史问道:“是那个西门千兵?”蔡御史道:“他如今见是本处提刑千户,昨日已参见过年兄了。”宋御史令左右取手本来看,见西门庆与夏提刑名字,说道:“此莫非与翟云峰有亲者?”蔡御史道:“就是他。如今见在外面伺候,要央学生奉陪年兄到他家一饭。未审年兄尊意若何?”宋御史道:“学生初到此处,只怕不好去得。”蔡御史道:“年兄怕怎的?既是云峰分上,你我走走何害?”于是吩咐看轿,就一同起行,一面传将出来。

西门庆知了此消息,与来保、贲四骑快马先奔来家,预备酒席。门首搭照山彩棚,两院乐人奏乐,叫海盐戏并杂耍承应。原来宋御史将各项伺候人马都令散了,只用几个蓝旗清道官吏跟随,与蔡御史坐两顶大轿,打着双檐伞,同往西门庆家来。当时哄动了东平府,大闹了清河县,都说:“巡按老爷也认的西门大官人,来他家吃酒来了。”慌的周守备、荆都监、张团练,各领本哨人马把住左右街口伺候。西门庆青衣冠带,远远迎接。两边鼓乐吹打,到大门首下了轿进去。宋御史与蔡御史都穿着大红獬豸绣服,乌纱皂履,鹤顶红带,从人执着两把大扇。只见五间厅上湘帘高卷,锦屏罗列。正面摆两张吃看桌席,高顶方糖,定胜簇盘,十分齐整。二官揖让进厅,与西门庆叙礼。蔡御史令家人具贽见之礼:两端湖绸、一部文集、四袋芽茶、一方端溪砚。宋御史只投了个宛红单拜帖,上书“侍生宋乔年拜”。向西门庆道:“久闻芳誉。学生初临此地,尚未尽情,不当取扰。若不是蔡年兄邀来进拜,何以幸接尊颜?”慌的西门庆倒身下拜,说道:“仆乃一介武官,属于按临之下。今日幸蒙清顾,蓬荜生光。”于是鞠恭展拜,礼容甚谦。宋御史亦答礼相还,叙了礼数。当下蔡御史让宋御史居左,他自在右,西门庆垂首相陪。茶汤献罢,阶下箫韶盈耳,鼓乐喧阗,动起乐来。西门庆递酒安席已毕,下边呈献割道。说不尽肴列珍羞,汤陈桃浪,端的歌舞声容,食前方丈。两位轿上跟从人,每位五十瓶酒、五百点心、一百斤熟肉,都领下去。家人、吏书、门子人等,另在厢房中管待,不必细说。当日西门庆这席酒,也费够千两金银。

那宋御史又系江西南昌人,为人浮躁,只坐了没多大回,听了一折戏文就起来。慌的西门庆再三固留。蔡御史在旁便说:“年兄无事,再消坐一时,何遽回之太速耶!”宋御史道:“年兄还坐坐,学生还欲到察院中处分些公事。”西门庆早令手下,把两张桌席连金银器,已都装在食盒内,共有二十抬,叫下人夫伺候。宋御史的一张大桌席、两坛酒、两牵羊、两封金丝花、两匹段红、一副金台盘、两把银执壶、十个银酒杯、两个银折盂、一双牙箸。蔡御史的也是一般的。都递上揭帖。宋御史再三辞道:“这个,我学生怎么敢领?”因看着蔡御史。蔡御史道:“年兄贵治所临,自然之道,我学生岂敢当之!”西门庆道:“些须微仪,不过侑觞而已,何为见外?”比及二官推让之次,而桌席已抬送出门矣。宋御史不得已,方令左右收了揭帖,向西门庆致谢说道:“今日初来识荆,既扰盛席,又承厚贶,何以克当?余容图报不忘也。”因向蔡御史道:“年兄还坐坐,学生告别。”于是作辞起身。西门庆还要远送,宋御史不肯,急令请回,举手上轿而去。

西门庆回来,陪侍蔡御史,解去冠带,请去卷棚内后坐。因吩咐把乐人都打发散去,只留下戏子。西门庆令左右重新安放桌席,摆设珍羞果品上来,二人饮酒。蔡御史道:“今日陪我这宋年兄坐便僭了,又叨盛筵并许多酒器,何以克当?”西门庆笑道:“微物惶恐,表意而已!”因问道:“宋公祖尊号?”蔡御史道:“号松原。松树之松,原泉之原。”又说起:“头里他再三不来,被学生因称道四泉盛德,与老先生那边相熟,他才来了。他也知府上与云峰有亲。”西门庆道:“想必翟亲家有一言于彼。我观宋公为人有些蹊跷。”蔡御史道:“他虽故是江西人,倒也没甚蹊跷处。只是今日初会,怎不做些模样!”说毕笑了。门庆便道:“今日晚了,老先生不回船上去罢了。”蔡御史道:“我明早就要开船长行。“西门庆道:“请不弃在舍留宿一宵,明日学生长亭送饯。”蔡御史道:“过蒙爱厚。”因吩咐手下人:“都回门外去罢,明早来接。”众人都应诺去了,只留下两个家人伺候。

西门庆见手下人都去了,走下席来,叫玳安儿附耳低言,如此这般:“即去院里坐名叫了董娇儿、韩金钏儿两个,打后门里用轿子抬了来,休交一人知道。”那玳安一面应诺去了。西门庆复上席,陪蔡御史吃酒。海盐子弟在旁歌唱。西门庆因问:“老先生到家多少时就来了?令堂老夫人起居康健么?”蔡御史道:“老母到也安。学生在家,不觉荏苒半载,回来见朝,不想被曹禾论劾,将学生敝同年一十四人之在史馆者,一时皆黜授外职。学生便选在西台,新点两淮巡盐。宋年兄便在贵处巡按,也是蔡老先生门下。”西门庆问道:“如今安老先生在那里?”蔡御史道:“安凤山他已升了工部主事,往荆州催攒皇木去了。也待好来也。”说毕,西门庆教海盐子弟上来递酒。蔡御史吩咐:“你唱个《渔家傲》我听。”子弟排手在旁正唱着,只见玳安走来请西门庆下边说话。玳安道:“叫了董娇儿、韩金钏打后门来了,在娘房里坐着哩。”西门庆道:“你吩咐把轿子抬过一边才好。”玳安道:“抬过一边了。”

这西门庆走至上房,两个唱的向前磕头。西门庆道:“今日请你两个来,晚夕在山子下扶侍你蔡老爹。他如今见做巡按御史,你不可怠慢,用心扶侍他,我另酬答你。”韩金钏儿笑道:“爹不消吩咐,俺每知道。”西门庆因戏道:“他南人的营生,好的是南风,你每休要扭手扭脚的。”董娇儿道:“娘在这里听着,爹你老人家羊角葱靠南墙——越发老辣了。王府门首磕了头,俺们不吃这井里水了?”

西门庆笑的往前边来。走到仪门首,只见来保和陈敬济拿着揭帖走来,与西门庆看,说道:“刚才乔亲家爹说,趁着蔡老爹这回闲,爹倒把这件事对蔡老爹说了罢,只怕明日起身忙了。教姐夫写了俺两个名字在此。”西门庆道:“你跟了来。”来保跟到卷棚槅子外边站着。西门庆饮酒中间因题起:“有一事在此,不敢干渎。”蔡御史道:“四泉,有甚事只顾吩咐,学生无不领命。”西门庆道:“去岁因舍亲在边上纳过些粮草,坐派了些盐引,正派在贵治扬州支盐。望乞到那里青目青目,早些支放就是爱厚。”因把揭帖递上去,蔡御史看了。上面写着:“商人来保、崔本,旧派淮盐三万引,乞到日早掣。”蔡御史看了笑道:“这个甚么打紧。”一面把来保叫至跟前跪下,吩咐:“与你蔡爷磕头。”蔡御史道:“我到扬州,你等径来察院见我。我比别的商人早掣一个月。”西门庆道:“老先生下顾,早放十日就够了。”蔡御史把原帖就袖在袖内。一面书童旁边斟上酒,子弟又唱。

唱毕,已有掌灯时分,蔡御史便说:“深扰一日,酒告止了罢。”因起身出席,左右便欲掌灯,西门庆道:“且休掌烛,请老先生后边更衣。”于是从花园里游玩了一回,让至翡翠轩,那里又早湘帘低簇,银烛荧煌,设下酒席。海盐戏子,西门庆已命打发去了。书童把卷棚内家活收了,关上角门,只见两个唱的盛妆打扮,立于阶下,向前插烛也似磕了四个头。但见:

\[
绰约容颜金缕衣,香尘不动下阶墀。
时来水溅罗裙湿,好似巫山行雨归。
\]

蔡御史看见,欲进不能,欲退不舍。便说道:“四泉,你如何这等爱厚?恐使不得。”西门庆笑道:“与昔日东山之游,又何异乎?”蔡御史道:“恐我不如安石之才,而君有王右军之高致矣。”于是月下与二妓携手,恍若刘阮之入天台。因进入轩内,见文物依然,因索纸笔就欲留题相赠。西门庆即令书童连忙将端溪砚研的墨浓浓的,拂下锦笺。这蔡御史终是状元之才,拈笔在手,文不加点,字走龙蛇,灯下一挥而就,作诗一首。诗曰:

\[
不到君家半载余,轩中文物尚依稀。
雨过书童开药圃,风回仙子步花台。
饮将醉处钟何急,诗到成时漏更催。
此去又添新怅望,不知何日是重来。
\]
写毕,教书童粘于壁上,以为后日之遗焉。因问二妓:“你们叫甚名字?”一个道:“小的姓董,名唤娇儿。他叫韩金钏儿。”蔡御史又道:“你二人有号没有?”董娇儿道:“小的无名娼妓,那讨号来?”蔡御史道:“你等休要太谦。”问至再三,韩金钏方说:“小的号玉卿。”董娇儿道:“小的贱号薇仙。”蔡御史一闻“薇仙”二字,心中甚喜,遂留意在怀。令书童取棋桌来,摆下棋子,蔡御史与董娇儿两个着棋。西门庆陪侍,韩金钏儿把金樽在旁边递酒,书童歌唱。蔡御史赢了一盘棋,董娇儿吃过,又回奉蔡御史一杯。韩金钏这里也递与西门庆一杯陪饮。饮了酒,两人又下。董娇儿赢了,连忙递酒一杯与蔡御史,西门庆在旁又陪饮一杯。饮毕,蔡御史道:“四泉,夜深了,不胜酒力,”于是走出外边来,站立在花下。

那时正是四月半头,月色才上。西门庆道:“老先生,天色还早哩。还有韩金钏,不曾赏他一杯酒。”蔡御史道:“正是。你唤他来,我就此花下立饮一杯。”于是韩金钏拿大金桃杯,满斟一杯,用纤手捧递上去。董娇儿在旁捧果,蔡御史吃过,又斟了一杯,赏与韩金钏儿。因告辞道:“四泉,今日酒大多了,令盛价收过去罢。”于是与西门庆握手相语,说道:“贤公盛情盛德,此心悬悬。非斯文骨肉,何以至此?向日所贷,学生耿耿在心,在京已与云峰表过。倘我后日有一步寸进,断不敢有辜盛德。”西门庆道:“老先生何出此言?到不消介意。”

韩金钏见他一手拉着董娇儿,知局,就往后边去了。到了上房里,月娘问道:“你怎的不陪他睡,来了?”韩金钏笑道:“他留下董娇儿了,我不来,只管在那里做甚么?”良久,西门庆亦告了安置进来,叫了来兴儿吩咐:“明日早五更,打发食盒酒米点心下饭,叫了厨役,跟了往门外永福寺去,与你蔡老爹送行。叫两个小优儿答应。休要误了。”来兴儿道:“家里二娘上寿,没有人看。”西门庆道:“留下棋童儿买东西,叫厨子后边大灶上做罢。”

不一时,书童、玳安收下家活来,又讨了一壶好茶,往花园里去与蔡老爹漱口。翡翠轩书房床上,铺陈衾枕俱各完备。蔡御史见董娇儿手中拿着一把湘妃竹泥金面扇儿,上面水墨画着一种湘兰平溪流水。董娇儿道:“敢烦老爹赏我一首诗在上面。”蔡御史道:“无可为题,就指着你这薇仙号。”于是灯下拈起笔来,写了四句在上:

\[
小院闲庭寂不哗,一池月上浸窗纱。
邂逅相逢天未晚,紫薇郎对紫薇花。
\]
写毕,那董娇儿连忙拜谢了。两个收拾上床就寝。书童、玳安与他家人在明间里睡。一宿晚景不题。

次日早晨,蔡御史与了董娇儿一两银子,用红纸大包封着,到于后边,拿与西门庆瞧。西门庆笑说道:“文职的营生,他那里有大钱与你!这个就是上上签了。”因交月娘每人又与了他五钱银子,从后门打发去了。书童舀洗面水,打发他梳洗穿衣。西门庆出来,在厅上陪他吃了粥。手下又早伺候轿马来接,与西门庆作辞,谢了又谢。西门庆又道:“学生日昨所言之事,老先生到彼处,学生这里书去,千万留神一二,足仞不浅。”蔡御史道:“休说贤公华扎下临,只盛价有片纸到,学生无不奉行。”说毕,二人同上马,左右跟随。出城外,到于永福寺,借长老方丈摆酒饯行。来兴儿与厨役早已安排桌席停当。李铭、吴惠两个小优弹唱。

数杯之后,坐不移时,蔡御史起身,夫马、坐轿在于三门外伺候。临行,西门庆说起苗青之事:“乃学生相知,因诖误在旧大巡曾公案下,行牌往扬州案候捉他。此事情已问结了。倘见宋公,望乞借重一言,彼此感激。”蔡御史道:“这个不妨,我见宋年兄说,设使就提来,放了他去就是了。”西门庆又作揖谢了。看官听说:后来宋御史往济南去,河道中又与蔡御史会在那船上。公人扬州提了苗青来,蔡御史说道:“此系曾公手里案外的,你管他怎的?”遂放回去了。倒下详去东平府,还只把两个船家,决不待时,安童便放了。正是:

\[
公道人情两是非,人情公道最难为。
若依公道人情失,顺了人情公道亏。
\]

当日西门庆要送至船上,蔡御史不肯,说道:“贤公不消远送,只此告别。”西门庆道:“万惟保重,容差小价问安。”说毕,蔡御史上轿而去。

西门庆回到方丈坐下,长老走来合掌问讯,递茶,西门庆答礼相还。见他雪眉交白,便问:“长老多大年纪?”长老道:“小僧七十有四。”西门庆道:“到还这等康健。”因问法号,长老道:“小僧法名道坚。”又问:“有几位徒弟?”长老道:“止有两个小徒。本寺也有三十余僧行。”西门庆道:“这寺院也宽大,只是欠修整。”长老道:“不满老爹说,这座寺原是周秀老爹盖造,长住里没钱粮修理,丢得坏了。”西门庆道:“原来就是你守备府周爷的香火院。我见他家庄子不远。不打紧处,你禀了你周爷,写个缘簿,别处也再化些,我也资助你些布施。”道坚连忙又合掌问讯谢了。西门庆吩咐玳安儿:“取一两银子谢长老。今日打搅。”道坚道:“小僧不知老爹来,不曾预备斋供。”西门庆道:“我要往后边更更衣去。”道坚连忙叫小沙弥开门。西门庆更了衣,因见方丈后面五间大禅堂,有许多云游和尚在那里敲着木鱼看经。西门庆不因不由,信步走入里面观看。见一个和尚形骨古怪,相貌搊搜,生的豹头凹眼,色若紫肝,戴了鸡蜡箍儿,穿一领肉红直裰。颏下髭须乱拃,头上有一溜光檐,就是个形容古怪真罗汉,未除火性独眼龙。在禅床上旋定过去了,垂着头,把脖子缩到腔子里,鼻孔中流下玉箸来。西门庆口中不言,心中暗道:“此僧必然是个有手段的高僧。不然,如何因此异相?等我叫醒他,问他个端的。”于是高声叫:“那位僧人,你是那里人氏,何处高僧?”叫了头一声不答应;第二声也不言语;第三声,只见这个僧人在禅床上把身子打了个挺,伸了伸腰,睁开一只眼,跳将起来,向西门庆点了点头儿,麄声应道:“你问我怎的?贫僧行不更名,坐不改姓,乃西域天竺国密松林齐腰峰寒庭寺下来的胡僧,云游至此,施药济人。官人,你叫我有甚话说?”西门庆道:“你既是施药济人,我问你求些滋补的药儿,你有也没有?”胡僧道:“我有,我有。”又道:“我如今请你到家,你去不去?”胡僧道:“我去,我去。”西门庆道:“你说去,即此就行。”那胡僧直竖起身来,向床头取过他的铁柱杖来拄着,背上他的皮褡裢——褡裢内盛了两个药葫芦儿。下的禅堂,就往外走。西门庆吩咐玳安:“叫了两个驴子,同师父先往家去等着,我就来。”胡僧道:“官人不消如此,你骑马只顾先行。贫僧也不骑头口,管情比你先到。”西门庆道:“一定是个有手段的高僧。不然如何开这等朗言。”恐怕他走了,吩咐玳安:“好歹跟着他同行。”于是作辞长老上马,仆从跟随,迳直进城来家。

那日四月十七日,不想是王六儿生日,家中又是李娇儿上寿,有堂客吃酒。后晌时分,只见王六儿家没人使,使了他兄弟王经来请西门庆。吩咐他宅门首只寻玳安儿说话,不见玳安在门首,只顾立。立了约一个时辰,正值月娘与李娇儿送院里李妈妈出来上轿,看见一个十五六岁扎包髻儿小厮,问是那里的。那小厮三不知走到跟前,与月娘磕了个头,说道:“我是韩家,寻安哥说话。”月娘问:“那安哥?”平安在旁边,恐怕他知道是王六儿那里来的,恐怕他说岔了话,向前把他拉过一边,对月娘说:“他是韩伙计家使了来寻玳安儿,问韩伙计几时来。”以此哄过。月娘不言语,回后边去了。

不一时玳安与胡僧先到门首,走的两腿皆酸,浑身是汗,抱怨的要不的。那胡僧体貌从容,气也不喘。平安把王六儿那边使了王经来请爹,寻他说话一节,对玳安儿说了一遍,道:“不想大娘看见,早是我在旁边替他摭拾过了。不然就要露出马脚来了。等住回娘若问,你也是这般说。”那玳安走的睁睁的,只顾\textShan 扇子:“今日造化低也怎的?平白爹交我领了这贼秃囚来。好近路儿!从门外寺里直走到家,路上通没歇脚儿,走的我上气儿接不着下气儿。爹交雇驴子与他骑,他又不骑。他便走着没事,难为我这两条腿了!把鞋底子也磨透了,脚也踏破了。攘气的营生!”平安道:“爹请他来家做甚么?”玳安道:“谁知道!他说问他讨甚么药哩。”正说着,只闻喝道之声。西门庆到家,看见胡僧在门首,说道:“吾师真乃人中神也。果然先到。”一面让至里面大厅上坐。西门庆叫书童接了衣裳,换了小帽,陪他坐的。吃了茶,那胡僧睁眼观见厅堂高远,院字深沉,门上挂的是龟背纹虾须织抹绿珠帘,地下铺狮子滚绣球绒毛线毯。正当中放一张蜻蜓腿、螳螂肚、肥皂色起楞的桌子,桌子上安着绦环样须弥座大理石屏风。周围摆的都是泥鳅头、楠木靶肿筋的交倚,两壁挂的画都是紫竹杆儿绫边、玛瑙轴头。正是:

\[
鼍皮画鼓振庭堂,乌木春台盛酒器。
\]

胡僧看毕,西门庆问道:“吾师用酒不用?”胡僧道:“贫僧酒肉齐行。”西门庆一面吩咐小厮:“后边不消看素馔,拿酒饭来。”那时正是李娇儿生日,厨下肴馔下饭都有。安放桌儿,只顾拿上来。先绰边儿放了四碟果子、四碟小菜,又是四碟案酒:一碟头鱼、一碟糟鸭、一碟乌皮鸡、一碟舞鲈公。又拿上四样下饭来:一碟羊角葱\textHuoChuan 炒的核桃肉、一碟细切的\textShiJie\textShiHe 样子肉、一碟肥肥的羊贯肠、一碟光溜溜的滑鳅。次又拿了一道汤饭出来:一个碗内两个肉圆子,夹着一条花肠滚子肉,名唤一龙戏二珠汤;一大盘裂破头高装肉包子。西门庆让胡僧吃了,教琴童拿过团靶钩头鸡脖壶来,打开腰州精制的红泥头,一股一股邈出滋阴摔白酒来,倾在那倒垂莲蓬高脚钟内,递与胡僧。那胡僧接放口内,一吸而饮之。随即又是两样添换上来:一碟寸扎的骑马肠儿、一碟子腌腊鹅脖子。又是两样艳物与胡僧下酒:一碟子癞葡萄、一碟子流心红李子。落后又是一大碗鳝鱼面与菜卷儿,一齐拿上来与胡僧打散。登时把胡僧吃的楞子眼儿,便道:“贫僧酒醉饭饱,足以够了。”

西门庆叫左右拿过酒桌去,因问他求房术的药儿。胡僧道:“我有一枝药,乃老君炼就,王母传方。非人不度,非人不传,专度有缘。既是官人厚待于我,我与你几丸罢。”于是向褡裢内取出葫芦来,倾出百十丸,吩咐:“每次只一粒,不可多了,用烧酒送下。”又将那一个葫儿捏了,取二钱一块粉红膏儿,吩咐:“每次只许用二厘,不可多用。若是胀的慌,用手捏着,两边腿上只顾摔打,百十下方得通。你可樽节用之,不可轻泄于人。”西门庆双手接了,说道:“我且问你,这药有何功效?”胡僧说:

\[
形如鸡卵,色似鹅黄。三次老君炮炼,王母亲手传方。外视轻如粪土,内觑贵乎玕琅。比金金岂换,比玉玉何偿!任你腰金衣紫,任你大厦高堂,任你轻裘肥马,任你才俊栋梁,此药用托掌内,飘然身人洞房。洞中春不老,物外景长芳;玉山无颓败,丹田夜有光。一战精神爽,再战气血刚。不拘娇艳宠,十二美红妆,交接从吾好,彻夜硬如枪。服久宽脾胃,滋肾又扶阳。百日须发黑,千朝体自强。固齿能明目,阳生姤始藏。恐君如不信,拌饭与猫尝:三日淫无度,四日热难当;白猫变为黑,尿粪俱停亡;夏月当风卧,冬天水里藏。若还不解泄,毛脱尽精光。每服一厘半,阳兴愈健强。一夜歇十女,其精永不伤。老妇颦眉蹙,淫娼不可当。有时心倦怠,收兵罢战场。冷水吞一口,阳回精不伤。快美终宵乐,春色满兰房。赠与知音客,永作保身方。
\]
西门庆听了,要问他求方儿,说道:“请医须请良,传药须传方。吾师不传于我方儿,倘或我久后用没了,那里寻师父去?随师父要多少东西,我与师父。”因令玳安:“后边快取二十两白金来。”递与胡僧,要问他求这一枝药方。那胡僧笑道:“贫僧乃出家之人,云游四方,要这资财何用?官人趁早收拾回去。”一面就要起身。西门庆见他不肯传方,便道:“师父,你不受资财,我有一匹五丈长大布,与师父做件衣服罢。”即令左右取来,双手递与胡僧。胡僧方才打问讯谢了。临出门又吩咐:“不可多用,戒之!戒之!”言毕,背上褡裢,拴定拐杖,出门扬长而去。正是:

\[
柱杖挑擎双日月,芒鞋踏遍九军州。
\]

\newpage
%# -*- coding:utf-8 -*-
%%%%%%%%%%%%%%%%%%%%%%%%%%%%%%%%%%%%%%%%%%%%%%%%%%%%%%%%%%%%%%%%%%%%%%%%%%%%%%%%%%%%%


\chapter{琴童潜听燕莺欢\KG 玳安嬉游蝴蝶巷}


词曰:

\[
欲掩香帏论缱绻,先敛双蛾愁夜短。催促少年郎,先去睡,鸳衾图暖。须臾整顿蝶蜂情,脱罗裳、恣情无限。留着帐前灯,时时看伊娇面。
\]

话说那日李娇儿上寿,观音庵王姑子请了莲花庵薛姑子来,又带了他两个徒弟妙凤、妙趣。月娘知道他是个有道行的姑子,连忙出来迎接。见他戴着清净僧帽,披着茶褐袈裟,剃的青旋旋头儿,生得魁肥胖大,沼口豚腮。进来与月娘众人合掌问讯,慌的月娘众人连忙行礼。见他铺眉苫眼,拿班做势,口里咬文嚼字,一口一声只称呼他“薛爷”。他便叫月娘是“在家菩萨”,或称“官人娘子”。月娘甚是敬重他。那日大妗子、杨姑娘都在这里,月娘摆茶与他吃,菜蔬点心摆了一大桌子,比寻常分外不同。两个小姑子妙趣、妙凤才十四五岁,生的甚是清俊,就在他旁边桌头吃东西。吃了茶,都在上房内坐的。听着他讲道说话。只见书童儿前边收下家活来,月娘便问道:“前边那吃酒肉的和尚去了?”书童道:“刚才起身,爹送出他去了。”吴大妗子因问:“是那里请来的僧人?”月娘道:“是他爹今日与蔡御史送行,门外寺里带来的一个和尚,酒肉都吃的。他求甚么药方,与他银子也不要,钱也不受,谁知他干的甚么营生!”那薛姑子听见,便说道:“茹荤、饮酒这两件事也难断。倒是俺这比丘尼还有些戒行,他汉僧们那里管!《大藏经》上不说的,如你吃他一口,到转世过来须还他一口。”吴大妗子听了,道:“象俺们终日吃肉,却不知转世有多少罪业!”薛姑子道:“似老菩萨,都是前生修来的福,享荣华,受富贵。譬如五谷,你春天不种下,到那有秋之时,怎望收成?”这里说话不题。

且说西门庆送了胡僧进来,只见玳安悄悄说道:“头里韩大婶使了他兄弟来请爹,说今日是他生日,请爹好歹过去坐坐。”西门庆得了胡僧药,心里正要去和妇人试验,不想来请,正中下怀,即吩咐玳安备马,使琴童先送一坛酒去。于是迳走到金莲房里取了淫器包儿,便衣小帽,带着眼纱,玳安跟随,径往王六儿家来。下马到里面,就吩咐:“留琴童儿伺候,玳安回了马家去。等家里问,就说我在狮子街房子里算帐哩。”玳安应诺,骑马回家去了。王六儿出来与西门庆磕了头,在旁边陪坐,说道:“无事,请爹过来散心坐坐。又多谢爹送酒来。”西门庆道:“我忘了你生日。今日往门外送行去,才来家。”因向袖中取出一根簪儿,递与他道:“今日与你上寿。”妇人接过来观看,却是一对金寿字簪儿,说道:“到好样儿。”连忙道了万福。西门庆又递与他五钱银子,吩咐:“你称五分,交小厮有南烧酒买一瓶来我吃。”王六儿笑道:“爹老人家别的酒吃厌了,想起来又要吃南烧酒了。”连忙称了五分银子,使琴童儿拿瓶买去。一面替西门庆脱了衣裳,请入房里坐的。亲自顿好茶与西门庆吃,又放小桌儿看牌耍子。看了一回,才收拾吃酒不题。

单表玳安回马到家,因跟和尚走的乏困了,一觉直睡到掌灯时便才醒了。揉揉眼儿,见天晚了,走到后边要灯笼接爹去,只顾立着。月娘因问他:“头里你爹打发和尚去了,也不进来换衣裳,三不知就去了。端的在谁家吃酒?”玳安道:“爹没往人家去,在狮子街房里算帐哩。”月娘道:“算帐?没的算恁一日!”玳安道:“算了帐,爹自家吃酒哩。”月娘道:“又没人陪他,莫不平白的自家吃酒?眼见的就是两样话。头里韩道国的小厮来寻你做甚么?”玳安道:“他来问韩大叔几时来。”月娘骂道:“贼囚根子,你又不知弄甚么鬼!”玳安不敢多言。月娘交小玉拿了灯笼与他,吩咐:“你说家中你二娘等着上寿哩。”

玳安应诺,走到前边铺子里,只见书童儿和傅伙计坐着,水柜上放着一瓶酒、几个碗碟、一盘牛肚子,平安儿从外拿了两瓶鲊来,正饮酒。玳安看见,把灯笼掠下,说道:“好呀!我赶着了。”因向书童儿戏道:“好淫妇,我那里没寻你,你原来躲在这里吃酒儿。”书童道:“你寻我做甚么?想是要与我做半日孙子儿!”玳安骂道:“秫秫小厮,你也回嘴!我寻你,要\textuni{34B2}你的屁股。”于是走向前按在椅子上就亲嘴。那书童用手推开,说道:“怪行货子,我不好骂出来的。把人牙花都磕破了,帽子都抓落了人的。”傅伙计见他帽子在地下,说道:“新一盏灯帽儿。”交平安儿:“你替他拾起来,只怕躧了。”被书童拿过,往炕上只一摔,把脸通红了。玳安道:“好淫妇,我逗你逗儿,你就恼了?”不由分说,掀起腿把他按在炕上,尽力往他口里吐了一口唾沫,把酒推翻了,流在水柜上。傅伙计恐怕湿了帐簿,连忙取手巾来抹了,说道:“管情住回两个顽恼了。”玳安道:“好淫妇,你今日讨了谁口里话,这等扭手扭脚?”书童把头发都揉乱了,说道:“耍便耍,笑便笑,臜剌剌的\textuni{379E}水子吐了人恁一口!”玳安道:“贼村秫秫,你今日才吃\textuni{379E}?你从前已后把\textuni{379E}不知吃了多少!”平安筛了一瓯子酒递与玳安,说道:“你快吃了接爹去罢,有话回来和他说。”玳安道:“等我接了爹回来,和他答话。我不把秫秫小厮不摆布的见神见鬼的,他也不怕。我使一些唾沫也不是人养的,我只一味干粘。”

于是吃了酒,门班房内叫了个小伴当拿着灯笼,他便骑着马,到了王六儿家。叫开门,问琴童儿:“爹在那里?”琴童道:“爹在屋里睡哩。”于是关上门,两个走到后边厨下。老冯便道:“安官儿,你韩大婶只顾等你不见来,替你留下分儿了。”就向厨柜里拿了一盘驴肉、一碟腊烧鸡、两碗寿面、一素子酒。玳安吃了一回,又让琴童道:“你过来,这酒我吃不了,咱两个噤了罢。”琴童道:“留与你的,你自吃罢。”玳安道:“我刚才吃了瓯子来了。”于是二人吃毕,玳安便叫道:“冯奶奶,我有句话儿说,你休恼我。想着你老人家在六娘那里,替俺六娘当家,如今在韩大婶这里,又与韩大婶当家。到家看我对六娘说也不说!”那老冯便向他身上拍了一下,说道:“怪倒路死猴儿!休要是言不是语到家里说出来,就交他恼我一生,我也不敢见他去。”

这里玳安儿和老冯说话,不想琴童走到卧房窗子底下,悄悄听觑。原来西门庆用烧酒把胡僧药吃了一粒下去,脱了衣裳,坐在床沿上。打开淫器包儿,先把银托束其根下,龟头上使了硫黄圈子,又把胡僧与他的粉红膏子药儿,盛在个小银盒儿内,捏了有一厘半儿,安放在马眼内。登时药性发作,那话暴怒起来,露棱跳脑,凹眼圆睁,横筋皆见,色若紫肝,约有六七寸长,比寻常分外粗大。西门庆心中暗喜:果然此药有些意思。妇人脱得光赤条条,坐在他怀里,一面用手笼攥。说道:“怪道你要烧酒吃,原来干这营生!”因问:“你是那里讨来的药?”西门庆把胡僧与他的药告诉一遍。先令妇人仰卧床上,背靠双枕,手拿那话往里放。龟头昂大,濡研半晌,方才进入些须。妇人淫津流溢,少顷滑落,已而仅没龟棱。西门庆酒兴发作,浅抽深送,觉翕翕然畅美不可言。妇人则淫心如醉,酥瘫于枕上,口内呻吟不止。口口声声只叫:“大\textMaoJi \textMaoBa 达达,淫妇今日可死也!”又道:“我央及你,好歹留些功夫在后边耍耍。”西门庆于是把老婆倒蹶在床上,那话顶入户中,扶其股而极力\textShan 磞,\textShan 磞的连声响亮。老婆道:“达达,你好生\textShan 打着淫妇,休要住了。再不,你自家拿过灯来照着顽耍。”西门庆于是移灯近前,令妇人在下直舒双足,他便骑在上面,兜其股蹲踞而提之;老婆在下一手揉着花心,扳其股而就之,颤声不已。西门庆因对老婆说:“等你家的来,我打发他和来保、崔本扬州支盐去。支出盐来卖了,就交他往湖州织了丝绸来,好不好?”老婆道:“好达达,随你交他那里,只顾去,留着王八在家里做甚么?”因问:“铺子却交谁管?”西门庆道:“我交贲四且替他卖着。”王六儿道:“也罢,且交贲四看着罢。”

这里二人行房,不想都被琴童儿窗外听了。玳安从后边来,见他听觑,向身上拍了一下,说道:“平白听他怎的?趁他未起来,咱们去来。”琴童跟他到外边。玳安道:“这后面小胡同子里,新来了两个小丫头子。我头里骑马打这里过,看见在鲁长腿屋里。一个叫金儿,一个叫赛儿,都不上十七八岁。交小伴当在这里看着,咱们混一回子去。”一面吩咐小伴当:“你在此听着门,俺们净净手去。等里边寻,你往小胡同口儿上来叫俺们。”吩咐了,两个月亮地里走到小巷内。原来这条巷唤做蝴蝶巷,里边有十数家,都是开坊子吃衣饭的。玳安已有酒了,叫门叫了半日才开。原来王八正和虔婆鲁长腿在灯下拿黄杆大等子称银子,见两个凶神也似撞进来,连忙把里间屋里灯一口悄灭。王八认的玳安是提刑所西门老爹家管家,便让坐。玳安道:“叫出他姐儿两个,唱个曲儿俺们听就去。”王八道:“管家,你来的迟了一步儿,两个刚才都有人了。”玳安不由分说,两步就撞进里面。只见灯也不点,月影中,看见炕上有两个戴白毡帽的酒太公——一个炕上睡下,那一个才脱裹脚,便问道:“是甚么人进屋里来?”玳安道:“我\textuni{34B2}你娘的眼!”飕的只一拳去,打的那酒保叫声:“阿嚛!”裹脚袜子也穿不上,往外飞跑。那一个在炕上爬起来,一步一跌也走了。玳安叫掌起灯来,骂道:“贼野蛮流民,他倒问我是那里人!刚才把毛搞净了他的才好,平白放他去了。好不好拿到衙门里去,交他且试试新夹棍着!”鲁长腿向前掌上灯,拜了又拜,说:“二位管家哥哥息怒,他外京人不知道,休要和他一般见识。”因令:“金儿、赛儿出来,唱与二位叔叔听。”只见两个都是一窝丝盘髻,穿着洗白衫儿,红绿罗裙儿,向前道:“今日不知叔叔来,夜晚了,没曾做得准备。”一面放了四碟干菜,其余几碟都是鸭蛋、虾米、熟鲊、咸鱼、猪头肉、干板肠儿之类。玳安便搂着赛儿,琴童便拥着金儿。玳安看见赛儿带着银红纱香袋儿,就拿袖中汗巾儿,两个换了。少顷筛酒上来,赛儿拿钟儿斟酒,递与玳安。先是金儿取过琵琶来,奉酒与琴童,唱个《山坡羊》道:

\[
烟花寨,委实的难过。白不得清凉到坐。逐日家迎宾待客,一家儿吃穿全靠着奴身一个。到晚来印子房钱逼的是我。老虔婆他不管我死活。在门前站到那更深儿夜晚,到晚来有那个问声我那饱饿?烟花寨再住上五载三年来,奴活命的少来死命的多。不由人眼泪如梭。有铁树上开花,那是我收圆结果。”
\]

金儿唱毕,赛儿又斟一杯酒递与玳安儿,接过琵琶来才待要唱,忽见小伴当来叫,二人连忙起身。玳安向赛儿说:“俺们改日再来望你。”说毕出门,来到王六儿家。西门庆才起来,老婆陪着吃酒哩。两个进入厨房内,问老冯:“爹寻我每来?”老冯道:“你爹没寻,只问马来了,我回说来了。再没言语。”两个坐在厨下问老冯要茶吃,每人喝了一瓯子茶,交小伴当点上灯笼牵出马去。西门庆临起身,老婆道:“爹,好暖酒儿,你再吃上一钟儿。你到家莫不又吃酒?”西门庆道:“到家不吃了。”于是拿起酒来又吃了一钟。老婆便道:“你这一去,几时来走走?”西门庆道:“等打发了他每起身,我才来哩。”说毕,丫头点茶来漱了口。王六儿送到门首,西门庆方上马归家。

却表金莲同众人在月娘房内,听薛姑子徒弟——两个小姑子唱佛曲儿。忽想起头里月娘骂玳安:“说两样话,……不知弄的甚么鬼!”因回房向床上摸那淫器包儿,又没了。叫春梅问,春梅说:“头里爹进屋里来,向床背阁抽屉内翻了一回去了。谁知道那包子放在那里。”金莲道:“他多咱进来,我怎就不知道?”春梅道:“娘正往后边瞧薛姑子去了。爹戴着小帽儿进屋里来,我问着,他又不言语。”金莲道:“一定拿了这行货,往院中那淫妇家去了。等他来家,我好生问他!”因又往后边去了。不想西门庆来家,见夜深,也没往后边去,琴童打着灯笼,送到花园角门首,就往李瓶儿屋里去了。琴童儿把灯一交送到后边,小玉收了。月娘看见,便问道:“你爹来了?”琴童道:“爹来了,往前边六娘房里去了。”月娘道:“你看是有个槽道的?这里人等着,就不进来了。”李瓶儿慌的走到前边,对面门庆说道:“他二娘在后边等着你上寿,你怎的平白进我这屋里来了?”西门庆笑道:“我醉了,明日罢。”李瓶儿道:“就是你醉了,到后边也接个钟儿。你不去,惹他二娘不恼么!”一力撺掇西门庆进后边来。李娇儿递了酒,月娘问道:“你今日独自一个,在那边房子里坐到这早晚?”西门庆道:“我和应二哥吃酒来。”月娘道,“可又来。我说没个人儿,自家怎么吃!”说过就罢了。

西门庆坐不移时,提起脚儿还踅到李瓶儿房里来。原来是王六儿那里,因吃了胡僧药,被药性把住了,与老婆弄耸了一日,恰好没曾丢身子。那话越发坚硬,形如铁杵。进房交迎春脱了衣裳,就要和李瓶儿睡。李瓶儿只说他不来,和官哥在床上已睡下了。回过头来见是他,便道:“你在后边睡罢了,又来做甚么?孩子才睡的甜甜儿的。我这里不奈烦,又身上来了,不方便。你往别人屋里睡去不是,只来这里缠!”被西门庆搂过脖子来就亲了个嘴,说道:“这奴才,你达心里要和你睡睡儿。”因把那话露出来与李瓶儿瞧,唬的李瓶儿要不的。说道:“耶嚛!你怎么弄的他这等大?”西门庆笑着告他说吃了胡僧药一节:“你若不和我睡,我就急死了。”李瓶儿道:“可怎么样的?身上才来了两日,还没去,亦发等去了,我和你睡罢。你今日且往他五娘屋里歇一夜儿,也是一般。”西门庆道:“我今日不知怎的,一心只要和你睡。我如今拉个鸡儿央及你央及儿,再不你交丫头掇些水来洗洗,和我睡睡也罢。”李瓶儿道:“我到好笑起来——你今日那里吃的恁醉醉儿的,来家歪斯缠我?就是洗了也不干净。一个老婆的月经沾污在男子汉身上臜剌剌的,也晦气。我到明日死了,你也只寻我?”于是吃逼勒不过,交迎春掇了水,下来澡牝干净,方上床与西门庆交会。可霎作怪,李瓶儿慢慢拍哄的官哥儿睡下,只刚爬过这头来,那孩子就醒了。一连醒了三次。李瓶儿交迎春拿博浪鼓儿哄着他,抱与奶子那边屋里去了,这里二人方才自在顽耍。西门庆坐在帐子里,李瓶儿便马爬在他身上,西门庆倒插那话入牝中。已而灯下窥见他雪白的屁股儿,用手抱着,且细观其出入。那话已被吞进小截,兴不可遏。李瓶儿怕带出血来,不住取巾帕抹之。西门庆抽拽了一个时辰,两手抱定他屁股,只顾揉搓,那话尽入至根,不容毛发,脐下毳毛皆刺其股,觉翕翕然畅美不可言。瓶儿道:“达达,慢着些,顶的奴里边好不疼!”西门庆道:“你既害疼,我丢了罢。”于是向桌上取过冷茶来呷了一口,登时精来,一泄如注。正是:四体无非畅美,一团都是阳春。西门庆方知胡僧有如此之妙药。睡下时已三更天气。

且说潘金莲见西门庆在李瓶儿屋里歇了,只道他偷去淫器包儿和他顽耍,更不体察外边勾当。是夜暗咬银牙,关门睡了。月娘和薛姑子、王姑子在上房宿睡。王姑子把整治的头男衣胞并薛姑子的药,悄悄递与月娘。薛姑子叫月娘:“拣个壬子日,用酒吃下,晚夕与官人同床一次,就是胎气。不可交一人知道。”月娘连忙将药收了,拜谢了两个姑子。又向王姑子道:“我正月里好不等着,你就不来了。”王姑子道:“你老人家倒说的好,这件物儿好不难寻!亏了薛师父。——也是个人家媳妇儿养头次娃儿,可可薛爷在那里,悄悄与了个熟老娘三钱银子,才得了。替你老人家熬矾水打磨干净,两盒鸳鸯新瓦,泡炼如法,用重罗筛过,搅在符药一处才拿来了。”月娘道:“只是多累薛爷和王师父。”于是每人拿出二两银子来相谢。说道:“明日若坐了胎气,还与薛爷一匹黄褐缎子做袈裟穿。”那薛姑子合掌道了问讯:“多承菩萨好心!”常言:十日卖一担针卖不得,一日卖三担甲倒卖了。正是:

\[
若教此辈成佛道,天下僧尼似水流。
\]

\newpage
%# -*- coding:utf-8 -*-
%%%%%%%%%%%%%%%%%%%%%%%%%%%%%%%%%%%%%%%%%%%%%%%%%%%%%%%%%%%%%%%%%%%%%%%%%%%%%%%%%%%%%


\chapter{打猫儿金莲品玉\KG 斗叶子敬济输金}


诗曰:

\[
羞看鸾镜惜朱颜,手托香腮懒去眠。
瘦损纤腰宽翠带,泪流粉面落金钿。
薄幸恼人愁切切,芳心缭乱恨绵绵。
何时借得东风便,刮得檀郎到枕边。
\]

话说潘金莲见西门庆拿了淫器包儿,与李瓶儿歇了,足恼了一夜没睡,怀恨在心。到第二日,打听西门庆往衙门里去了,老早走到后边对月娘说:“李瓶儿背地好不说姐姐哩!说姐姐会那等虔婆势,乔坐衙,别人生日,又要来管。‘你汉子吃醉了进我屋里来,我又不曾在前边,平白对着人羞我,望着我丢脸儿。交我恼了,走到前边,把他爹赶到后边来。落后他怎的也不在后边,还到我房里来了?我两个黑夜说了一夜梯己话儿,只有心肠五脏没曾倒与我罢了。’”这月娘听了,如何不恼!因向大妗子、孟玉楼说:“你们昨日都在跟前看着,我又没曾说他甚么。小厮交灯笼进来,我只问了一声:‘你爹怎的不进来?’小厮倒说:‘往六娘屋里去了。’我便说:‘你二娘这里等着,恁没槽道,却不进来!’论起来也不伤他,怎的说我虔婆势,乔坐衙?我还把他当好人看成,原来知人知面不知心,那里看人去?干净是个绵里针、肉里刺的货,还不知背地在汉子跟前架甚么舌儿哩!怪道他昨日决烈的就往前走了。傻姐姐,那怕汉子成日在你屋里不出门,不想我这心动一动儿。一个汉子丢与你们,随你们去,守寡的不过。想着一娶来之时,贼强人和我门里门外不相逢,那等怎的过来?”大妗子在旁劝道:“姑娘罢么,看孩儿的分上罢!自古宰相肚里好行船。当家人是个恶水缸儿,好的也放在心里,歹的也放在心里。”月娘道:“不拘几时,我也要对这两句话。等我问他,我怎么虔婆势,乔做衙?”金莲慌的没口子说道:“姐姐宽恕他罢。常言大人不责小人过,那个小人没罪过?他在背地挑唆汉子,俺们这几个谁没吃他排说过?我和他紧隔着壁儿,要与他一般见识起来,倒了不成!行动只倚着孩儿降人,他还说的好话儿哩!说他的孩儿到明日长大了,有恩报恩,有仇报仇,俺们都是饿死的数儿——你还不知道哩!”吴大妗子道:“我的奶奶,那里有此话说?”月娘一声儿也没言语。

常言:路见不平,也有向灯向火。不想西门大姐平日与李瓶儿最好,常没针线鞋面,李瓶儿不拘好绫罗缎帛就与他,好汗巾手帕两三方背地与大姐,银钱不消说。当日听了此话,如何不告诉他。李瓶儿正在屋里与孩子做端午戴的绒线符牌,及各色纱小粽子并解毒艾虎儿。只见大姐走来,李瓶儿让他坐,又交迎春:“拿茶与你大姑娘吃。”大姐道:“头里请你吃茶,你怎的不来?”李瓶儿道:“打发他爹出门,我赶早凉与孩子做这戴的碎生活儿来。”大姐道:“有桩事儿,我也不是舌头,敢来告你说:你没曾恼着五娘?他对着俺娘,如此这般说了你一篇是非——说你说俺娘虔婆势,乔做衙。如今俺娘要和你对话哩!你别要说我对你说,交他怪我。你须预备些话儿打发他。”这李瓶儿不听便罢,听了此言,手中拿着那针儿通拿不起来,两只胳膊都软了,半日说不出话来,对着大姐掉眼泪,说道:“大姑娘,我那里有一字儿?昨晚我在后边,听见小厮说他爹往我这边来了,我就来到前边,催他往后边去了。再谁说一句话儿来?你娘恁觑我一场,莫不我恁不识好歹,敢说这个话?设使我就说,对着谁说来?也有个下落。”大姐道:“他听见俺娘说不拘几时要对这话,他也就慌了。要是我,你两个当面锣对面鼓的对不是!”李瓶儿道:“我对的过他那嘴头子?只凭天罢了。他左右昼夜算计的只是俺娘儿两个,到明日终久吃他算计了一个去,才是了当。”说毕哭了。大姐坐着劝了一回,只见小玉来请六娘、大姑娘吃饭。李瓶儿丢下针指,同大姐到后边,也不曾吃饭,回来房中,倒在床上就睡着了。

西门庆衙门中来家,见他睡,问迎春。迎春道:“俺娘一日饭也还没吃哩。”慌的西门庆向前问道:“你怎的不吃饭?你对我说。”又见他哭的眼红红的,只顾问:“你心里怎么的?对我说。”李瓶儿连忙起来,揉了揉眼说道:“我害眼疼,不怎的。今日心里懒待吃饭。”并不题出一字儿来。正是:满怀心腹事,尽在不言中。有诗为证:

\[
莫道佳人总是痴,惺惺伶俐没便宜。
只因会尽人间事,惹得闲愁满肚皮。
\]

大姐在后边对月娘说:“才五娘说的话,我问六娘来。他好不赌身发咒,望着我哭,说娘这般看顾他,他肯说此话!”吴大妗子道:“我就不信。李大姐好个人儿,他怎肯说这等话!”月娘道:“想必两个有些小节不足,哄不动汉子,走来后边,没的拿我垫舌根。我这里还多着个影儿哩!”大妗子道:“大姑娘,今后你也别要亏了人。不是我背地说,潘五姐一百个不及他。为人心地儿又好,来了咱家恁二三年,要一些歪样儿也没有。”

正说着,只见琴童儿背进个蓝布大包袱来。月娘问是甚么,琴童道:“是三万盐引。韩伙计和崔本才从关上挂了号来,爹说打发饭与他二人吃,如今兑银子打包。后日二十,是个好日子,起身,打发他三个往扬州去。”吴大妗子道:“只怕姐夫进来。我和二位师父往他二娘房里坐去罢。”刚说未毕,只见西门庆掀帘子进来,慌的吴妗子和薛姑子、王姑子往李娇儿房里走不迭。早被西门庆看见,问月娘:“那个是薛姑子?贼胖秃淫妇,来我这里做甚么!”月娘道:“你好恁枉口拨舌,不当家化化的,骂他怎的?他惹着你来?你怎的知道他姓薛?”西门庆道:“你还不知他弄的乾坤儿哩!他把陈参政的小姐吊在地藏庵儿里和一个小伙偷奸,他知情,受了三两银子。事发,拿到衙门里,被我褪衣打了二十板,交他嫁汉子还俗。他怎的还不还俗?好不好,拿来衙门里再与他几拶子。”月娘道:“你有要没紧,恁毁僧傍佛的。他一个佛家弟子,想必善根还在,他平白还甚么俗?你还不知他好不有道行!”西门庆道:“你问他有道行一夜接几个汉子?”月娘道:“你就休汗邪!又讨我那没好口的骂你。”因问:“几时打发他三个起身?”西门庆道:“我刚才使来保会乔亲家去了,他那里出五百两,我这里出五百两。二十是个好日子,打发他每起身去罢了。”月娘道:“线铺子却交谁开?”西门庆道:“且交贲四替他开着罢。”说毕,月娘开箱子拿银子,一面兑了出来,交付与三人,在卷棚内看着打包。每人又兑五两银子,交他家中收拾衣装行李。

只见应伯爵走到卷棚里,看见便问:“哥打包做甚么?”西门庆因把二十日打发来保等往扬州支盐去一节告诉一遍。伯爵举手道:“哥,恭喜!此去回来必得大利。”西门庆一面让坐,唤茶来吃。因问:“李三、黄四银子几时关?”应伯爵道:“也只在这个月里就关出来了。他昨日对我说,如今东平府又派下二万香来了,还要问你挪五百两银子,接济他这一时之急。如今关出这批银子,一分也不动,都抬过这边来。”西门庆道:“到是你看见,我打发扬州去还没银子,问乔亲家借了五百两在里头,那讨银子来?”伯爵道:“他再三央及我对你说,一客不烦二主,你不接济他这一步儿,交他又问那里借去?”西门庆道:“门外街东徐四铺少我银子,我那里挪五百两银子与他罢。”伯爵道:“可知好哩。”正说着,只见平安儿拿进帖儿来,说:“夏老爹家差了夏寿,说请爹明日坐坐。”西门庆看了柬帖,道:“晓得了。”伯爵道:“我有桩事儿来报与哥:你知道李桂儿的勾当么?他没来?”西门庆道:“他从正月去了,再几时来?我并不知道甚么勾当。”伯爵因说道:“王招宣府里第三的,原来是东京六黄太尉侄女儿女婿。从正月往东京拜年,老公公赏了一千两银子,与他两口儿过节。你还不知六黄太尉这侄女儿生的怎么标致,上画儿只画半边儿,也没恁俊俏相的。你只守着你家里的罢了,每日被老孙、祝麻子、小张闲三四个摽着在院里撞,把二条巷齐家那小丫头子齐香儿梳笼了,又在李桂儿家走。把他娘子儿的头面都拿出来当了。气的他娘子儿家里上吊。不想前日老公公生日,他娘子儿到东京只一说,老公公恼了,将这几个人的名字送与朱太尉,朱太尉批行东平府,着落本县拿人。昨日把老孙、祝麻子与小张闲都从李桂儿家拿的去了。李桂儿便躲在隔壁朱毛头家过了一夜。今日说来央及你来了。”西门庆道:“我说正月里都摽着他走,这里谁人家这银子,那里谁人家银子。那祝麻子还对着我捣生鬼。”说毕,伯爵道:“我去罢。等住回只怕李桂儿来,你管他不管他,他又说我来串作你。”西门庆道:“我还和你说,李三,你且别要许他,等我门外讨了银子来,再和你说话。”伯爵道:“我晓的。”刚走出大门首,只见李桂姐轿子在门首,又早下轿进去了。伯爵去了。

西门庆正分咐陈敬济,交他往门外徐四家催银子去,只见琴童儿走来道:“大娘后边请,李桂姨来了。”西门庆走到后边,只见李桂姐身穿茶色衣裳,也不搽脸,用白挑线汗巾子搭着头,云鬟不整,花容淹淡,与西门庆磕着头哭起来,说道:“爹可怎么样儿的,恁造化低的营生,正是关着门儿家里坐,祸从天上来。一个王三官儿,俺每又不认的他。平白的祝麻子、孙寡嘴领了来俺家讨茶吃。俺姐姐又不在家,依着我说别要招惹他,那些儿不是,俺这妈越发老的韶刀了。就是来宅里与俺姑娘做生日的这一日,你上轿来了就是了,见祝麻子打旋磨儿跟着,从新又回去,对我说:‘姐姐你不出去待他锺茶儿,却不难为嚣了人?’他便往爹这里来了。交我把门插了不出来,谁想从外边撞了一伙人来,把他三个不由分说都拿的去了。王三官儿便夺门走了,我便走在隔壁人家躲了。家里有个人牙儿!才使来保儿来这里接的他家去。到家把妈唬的魂都没了,只要寻死。今日县里皂隶,又拿着票喝罗了一清早起去了。如今坐名儿只要我往东京回话去。爹,你老人家不可怜见救救儿,却怎么样儿的?娘也替我说说儿。”西门庆笑道:“你起来。”因问票上还有谁的名字。桂姐道:“还有齐香儿的名字。他梳笼了齐香儿,在他家使钱,他便该当。俺家若见了他一个钱儿,就把眼睛珠子吊了;若是沾他沾身子儿,一个毛孔儿里生一个天疱疮。”月娘对西门庆道:“也罢,省的他恁说誓剌剌的,你替他说说罢。”西门庆道:“如今齐香儿拿了不曾?”桂姐道:“齐香儿他在王皇亲宅里躲着哩。”西门庆道:“既是恁的,你且在我这里住两日。我就差人往县里替你说去。”就叫书童儿:“你快写个帖儿,往县里见你李老爹,就说桂姐常在我这里答应,看怎的免提他罢。”书童应诺,穿青绢衣服去了。不一时,拿了李知县回贴儿来。书童道:“李老爹说:‘多上覆你老爹,别的事无不领命,这个却是东京上司行下来批文,委本县拿人,县里只拘的人到。既是你老爹分上,我这里且宽限他两日。要免提,还往东京上司说去。’”西门庆听了,只顾沉吟,说道:“如今来保一两日起身,东京没人去。”月娘道:“也罢,你打发他两个先去,存下来保,替桂姐往东京说了这勾当,交他随后边赶了去罢。你看唬的他那腔儿。”那桂姐连忙与月娘、西门庆磕头。

西门庆随使人叫将来保来,分咐:“二十日你且不去罢。教他两个先去。你明日且往东京替桂姐说说这勾当来。见你翟爹,如此这般,好歹差人往卫里说说。”桂姐连忙就与来保下礼。慌的来保顶头相还,说道:“桂姨,我就去。”西门庆一面教书童儿写就一封书,致谢翟管家前日曾巡按之事甚是费心,又封了二十两折节礼银子,连书交与来保。桂姐便欢喜了,拿出五两银子来与来保做盘缠,说道:“回来俺妈还重谢保哥。”西门庆不肯,还了桂姐,教月娘另拿五两银子与来保盘缠。桂姐道:“也没这个道理,我央及爹这里说人情,又教爹出盘缠。”西门庆道:“你笑话我没这五两银子盘缠了,要你的银子!”那桂姐方才收了,向来保拜了又拜,说道:“累保哥,好歹明早起身罢,只怕迟了。”来保道:“我明日早五更就走道儿了。”

于是领了书信,又走到狮子街韩道国家。王六儿正在屋里缝小衣儿哩,打窗眼看见是来保,忙道:“你有甚说话,请房里坐。他不在家,往裁缝那里讨衣裳去了,便来也。”便叫锦儿:“还不往对过徐裁家叫你爹去!你说保大爷在这里。”来保道:“我来说声,我明日还去不成,又有桩业障钻出来,当家的留下,教我往东京替院里李桂姐说人情去哩。他刚才在爹跟前,再三磕头礼拜央及我。明早就起身了。且教韩伙计和崔大官儿先去,我回来就赶了来。”因问:“嫂子,你做的是甚么?”王六儿道:“是他的小衣裳儿。”来保道:“你教他少带衣裳。到那去处是出纱罗缎绢的窝儿里,愁没衣裳穿!”正说着,韩道国来了。两个唱了喏,因把前事说了一遍,因说:“我到明日,扬州那里寻你每?”韩道国道:“老爹分咐,教俺每马头上投经纪王伯儒店里下。说过世老爹曾和他父亲相交,他店内房屋宽广,下的客商多,放财物不耽心。你只往那里寻俺每就是了。”来保又说:“嫂子,我明日东京去,你没甚鞋脚东西捎进府里,与你大姐去?”王六儿道道:“没甚么,只有他爹替他打的两对簪儿,并他两双鞋,起动保叔捎捎进去与他。”于是将手帕包袱停当,递与来保。一面教春香看菜儿筛酒。妇人连忙丢下生活就放桌儿。来保道:“嫂子,你休费心,我不坐。我到家还要收拾褡裢,明日早起身。”王六儿笑嘻嘻道:“耶嚛,你怎的上门怪人家!伙计家,自恁与你饯行,也该吃锺儿。”因说韩道国:“你好老实!桌儿不稳,你也撒撒儿,让保叔坐。只相没事的人儿一般。”于是拿上菜儿来,斟酒递与来保,王六儿也陪在旁边,三人坐定吃酒。来保吃了几锺,说道:“我家去罢。晚了,只怕家里关门早。”韩道国问道:“你头口雇下了不曾?”来保道:“明日早雇罢了。铺子里钥匙并帐簿都交与贲四罢了,省的你又上宿去。家里歇息歇息,好走路儿。”韩道国道:“伙计说的是,我明日就交与他。”王六儿又斟了一瓯子,说道:“保叔,你只吃这一锺,我也不敢留你了。”来保道:“嫂子,你既要我吃,再筛热着些。”那王六儿连忙归到壶里,教锦儿炮热了,倾在盏内,双手递与来保,说道:“没甚好菜儿与保叔下酒。”来保道:“嫂子好说,家无常礼。”拿起酒来与妇人对饮,一吸同干,方才作辞起身。王六儿便把女儿鞋脚递与他,说道:“累保叔,好歹到府里问声孩子好不好,我放心些。”两口儿齐送出门来。

不说来保到家收拾行李,第二日起身东京去了。单表这吴大舅前来对西门庆说:“有东平府行下文书来,派俺本卫两所掌印千户管工修理社仓,题准旨意,限六月工完,升一级。违限,听巡按御史查参。姐夫有银子借得几两,工上使用。待关出工价来,一一奉还。”西门庆道:“大舅用多少,只顾拿去。”吴大舅道:“姐夫下顾,与二十两罢。”一面同进后边,见月娘说了话,教月娘拿二十两出来,交与大舅,又吃了茶。因后边有堂客,就出来了。月娘教西门庆留大舅大厅上吃酒。正饮酒中间,只见陈敬济走来,与吴大舅作了揖,就回说:“门外徐四家,禀上爹,还要再让两日儿。”西门庆道:“胡说!我这里等银子使,照旧还去骂那狗弟子孩儿。”敬济应诺。吴大舅就让他打横坐下,陪着吃酒不题。

且说后边大妗子、杨姑娘、李娇儿、孟玉楼、潘金莲、李瓶儿、大姐,都伴桂姐在月娘房里吃酒。先是郁大姐数了一回“张生游宝塔”,放下琵琶。孟玉楼在旁斟酒递菜儿与他吃,说道:“贼瞎转磨的唱了这一日,又说我不疼你。”潘金莲又大箸子夹块肉放在他鼻子上,戏弄他顽耍。桂姐因叫玉箫姐:“你递过郁大姐琵琶来,等我唱个曲儿与姑奶奶和大妗子听。”月娘道:“桂姐,你心里热剌剌的,不唱罢。”桂姐道:“不妨事。见爹娘替我说人情去了,我这回不焦了。”孟玉楼笑道:“李桂姐倒还是院中人家娃娃,做脸儿快。头里一来时,把眉头忔\textuni{3918}着,焦的茶儿也吃不下去。这回说也有,笑也有。”当下桂姐轻舒玉指,顿拨冰弦,唱了一回。

正唱着,只见琴童儿收进家活来。月娘便问道:“你大舅去了?”琴童儿道:“大舅去了。”吴大妗子道:“只怕姐夫进来,我每活变活变儿。”琴童道:“爹往五娘房里去了。”这潘金莲听见,就坐不住,趋趄着脚儿只要走,又不好走的。月娘也不等他动身,就说道:“他往你屋里去了,你去罢。省的你欠肚儿亲家是的。”那潘金莲嚷:“可可儿的——”起来,口儿里硬着,那脚步儿且是去的快。

来到房里,西门庆已是吃了胡僧药,教春梅脱了裳,在床上帐子里坐着哩。金莲看见笑道:“我的儿!今日好呀,不等你娘来就上床了。俺每在后边吃酒,被李桂姐唱着,灌了我几锺好的。独自一个儿,黑影子里,一步高一步低,不知怎的走来了。”叫春梅:“你有茶倒瓯子我吃。”那春梅真个点了茶来。金莲吃了,努了个嘴与春梅,那春梅就知其意。那边屋里早已替他热下水,妇人抖些檀香白矾在里面,洗了牝。就灯下摘了头,止撇着一根金簪子,拿过镜子来,从新把嘴唇抹了脂胭,口中噙着香茶,走过这边来。春梅床头上取过睡鞋来与他换了,带上房门出去。这妇人便将灯台挪近旁边桌上放着,一手放下半边纱帐子来,褪去红裤,露出玉体。西门庆坐在枕头上,那话带着两个托子,一霎弄的大大的与他瞧。妇人灯下看见,唬了一跳——一手攥不过来,紫巍巍,沉甸甸——便昵瞅了西门庆一眼,说道:“我猜你没别的话,一定吃了那和尚药,弄耸的恁般大,一味要来奈何老娘。好酒好肉,王里长吃的去。你在谁人跟前试了新,这回剩了些残军败将,才来我这屋里来了。俺每是雌剩\textMaoJi \textMaoBa \textuni{34B2}的?你还说不偏心哩!嗔道那一日我不在屋里,三不知把那行货包子偷的往他屋里去了。原来晚夕和他干这个营生,他还对着人撇清捣鬼哩。你这行货子,干净是个没挽回的三寸货。想起来,一百年不理你才好。”西门庆笑道:“小淫妇儿,你过来。你若有本事,把他咂过了,我输一两银子与你。”妇人道:“汗邪了你了。你吃了甚么行货子,我禁的过他!”于是把身子斜軃在衽席之上,双手执定那话,用朱唇吞裹。说道:“好大行货子,把人的口也撑的生疼的。”说毕,出入鸣咂。或舌尖挑弄蛙口,舐其龟弦;或用口噙着,往来哺摔;或在粉脸上擂晃,百般抟弄,那话越发坚硬\textShouZao 掘起来。

西门庆垂首窥见妇人香肌掩映于纱帐之内,纤手捧定毛都鲁那话,往口里吞放,灯下一往一来。不想旁边蹲着一个白狮子猫儿,看见动弹,不知当做甚物件儿,扑向前,用爪儿来挝。这西门庆在上,又将手中拿的洒金老鸦扇儿,只顾引逗他耍子。被妇人夺过扇子来,把猫尽力打了一扇靶子,打出帐子外去了。昵向西门庆道:“怪发讪的冤家!紧着这扎扎的不得人意,又引逗他恁上头上脸的,一时间挝了人脸却怎的?好不好我就不干这营生了。”西门庆道:“怪小淫妇儿,会张致死了!”妇人道:“你怎不叫李瓶儿替你咂来?我这屋里尽着教你掇弄。不知吃了甚么行货子,咂了这一日,益发咂的没些事儿。”西门庆于是向汗巾上小银盒儿里,用挑牙挑了些粉红膏子药儿,抹在马口内,仰卧于上,教妇人骑在身上。妇人道:“等我\textShan 着,你往里放。”龟头昂大,濡研半晌,仅没龟棱。妇人在上,将身左右捱擦,似有不胜隐忍之态。因叫道:“亲达达,里边紧涩住了,好不难捱。”一面用手摸之,窥见麈柄已被牝户吞进半截,撑的两边皆满。妇人用唾津涂抹牝户两边,已而稍宽滑落,颇作往来,一举一坐,渐没至根。妇人因向西门庆说:“你每常使的颤声娇,在里头只是一味热痒不可当,怎如和尚这药,使进去,从子宫冷森森直掣到心上,这一回把浑身上下都酥麻了。我晓的今日死在你手里了。好难捱忍也!”西门庆笑道:“五儿,我有个笑话儿说与你听——是应二哥说的:一个人死了,阎王就拿驴皮披在身上,教他变驴。落后判官查簿籍,还有他十三年阳寿,又放回来了。他老婆看见浑身都变过来了,只有阳物还是驴的,未变过来,那人道:‘我往阴间换去。’他老婆慌了,说道:‘我的哥哥,你这一去,只怕不放你回来怎了?等我慢慢儿的挨罢。’”妇人听了,笑将扇把子打了一下子,说道:“怪不的应花子的老婆挨惯了驴的行货。硶说嘴的贼,我不看世界,这一下打的你……”

两个足缠了一个更次,西门庆精还不过。他在下面合着眼,由着妇人蹲踞在上极力抽提,提的龟头刮答刮答怪响。提勾良久,又吊过身子去,朝向西门庆。西门庆双手举其股,没棱露脑而提之,往来甚急。西门庆虽身接目视,而犹如无物。良久,妇人情急,转过身子来,两手搂定西门庆脖项,合伏在身上,舒舌头在他口里,那话直抵牝中,只顾揉搓,没口子叫:“亲达达,罢了,五儿\textuni{34B2}死了!”须臾,一阵昏迷,舌尖冰冷。泄讫一度,西门庆觉牝中一股热气直透丹田,心中翕翕然,美快不可言也。已而,淫津溢出,妇人以帕抹之。两个相搂相抱,交头叠股,鸣咂其舌,那话通不拽出来。睡的没半个时辰,妇人淫情未定,爬上身去,两个又干起来。妇人一连丢了两遭身子,亦觉稍倦。西门庆只是佯佯不采,暗想胡僧药神通。看看窗外鸡鸣,东方渐白,妇人道:“我的心肝,你不过却怎样的?到晚夕你再来,等我好歹替你咂过了罢。”西门庆道:“就咂也不得过。管情只一桩事儿就过了。”妇人道:“告我说是那一桩儿?”西门庆道:“法不传六耳,等我晚夕来对你说。”

早晨起来梳洗,春梅打发穿上衣裳。韩道国、崔本又早外边伺候。西门庆出来烧了纸,打发起身。交付二人两封书:“一封到扬州马头上,投王伯儒店里下;这一封就往扬州城内抓寻苗青,问他的事情下落,快来回报我。如银子不勾,我后边再教来保捎去。”崔本道:“还有蔡老爹书没有?”西门庆道:“你蔡老爹书还不曾写,教来保后边稍了去罢。”二人拜辞,上头口去了,不在话下。

西门庆冠带了,就往衙门中来与夏提刑相会,道及昨承见招之意。夏提刑道:“今日奉屈长官一叙,再无他客。”发放已毕,各分散来家。只见一个穿青衣皂隶,骑着快马,夹着毡包,走的满面汗流。到大门首,问平安:“此是提刑西门老爹家?”平安道:“你是那里来的?”那人即便下马作揖,说:“我是督催皇木的安老爹差来,送礼与老爹。俺老爹与管砖厂黄老爹,如今都往东平府胡老爹那里吃酒,顺便先来拜老爹,看老爹在家不在。”平安道:“有帖儿没有?”那人向毡包内取出,连礼物都递与平安。平安拿进去与西门庆看,见礼帖上写着浙绸二端,湖绵四斤,香带一束,古镜一圆。分咐:“包五钱银子,拿回帖打发来人,就说在家拱候老爹。”那人急急去了。

西门庆一面预备酒菜,等至日中,二位官员喝道而至,乘轿张盖甚盛。先令人投拜帖,一个是“侍生安忱拜”,一个是“侍生黄葆光拜”。都是青云白鹇补子,乌纱皂履,下轿揖让而入。西门庆出大门迎接,至厅上叙礼,各道契阔之情,分宾主坐下:黄主事居左,安主事居右,西门庆主位相陪。先是黄主事举手道:“久仰贤名芳誉,学生迟拜。”西门庆道:“不敢!辱承老先生先施枉驾,当容踵叩。敢问尊号?”安主事道:“黄年兄号泰宇,取‘履泰定而发天光’之意。”黄主事道:“敢问尊号?”西门庆道:“学生贱号四泉,——因小庄有四眼井之说。”安主事道:“昨日会见蔡年兄,说他与宋松原都在尊府打搅。”西门庆道:“因承云峰尊命,又是敝邑公祖,敢不奉迎!小价在京已知凤翁荣选,未得躬贺。”又问:“几时起身府上来?”安主事道:“自去岁尊府别后,到家续了亲,过了年,正月就来京了。选在工部,备员主事。钦差督运皇木,前往荆州,道经此处,敢不奉谒!”西门庆又说:“盛仪感谢不尽。”说毕,因请宽衣,令左右安放桌席。黄主事就要起身,安主事道:“实告:我与黄年兄,如今还往东平胡太府那里赴席,因打尊府过,敢不奉谒。容日再来取扰。”西门庆道:“就是往胡公处,去路尚远,纵二公不饿,其如从者何?学生敢不具酌,只备一饭在此,以犒从者。”于是先打发轿上攒盘。厅上安放桌席。珍羞异品,极时之盛,就是汤饭点心、海鲜美味,一齐上来。西门庆将小金锺,每人只奉了三杯,连桌儿抬下去,管待亲随家人吏典。少倾,两位官人拜辞起身,安主事因向西门庆道:“生辈明日有一小东,奉屈贤公到我这黄年兄同僚刘老太监庄上一叙,未审肯命驾否?”西门庆道:“既蒙宠招,敢不趋命!”说毕,送出大门,上轿而去。

只见夏提刑差人来邀。西门庆说道:“我就去。”一面分咐备马,走到后边换了冠带衣服,出来上马。玳安、琴童跟随,排军喝道,迳往夏提刑家来。到厅上叙礼,说道:“适有工部督催皇木安主政和砖厂黄主政来拜,留坐了半日,方才去了。不然,也来的早。”说毕,让至大厅,上面设放两张桌席,让西门庆居左,其次就是西宾倪秀才。座间因叙话问道:“老先生尊号?”倪秀才道:“学生贱名倪鹏,字时远,号桂岩,见在府庠备数,在我这东主夏老先生门下,设馆教习贤郎大先生举业。友道之间,实有多愧。”说话间,两个小优儿上来磕头,弹唱饮酒不题。

且说潘金莲从打发西门庆出来,直睡到晌午才爬起来。甫能起来,又懒待梳头。恐怕后边人说他,月娘请他吃饭也不吃,只推不好。大后晌才出房门,来到后边。月娘因西门庆不在,要听薛姑子讲说佛法,演颂金刚科仪。在明间内安放一张经桌儿,焚下香。薛姑子与王姑子两个对坐,妙趣、妙凤两个徒弟立在两边,接念佛号。大妗子、杨姑娘、吴月娘、李娇儿、孟玉楼、潘金莲、李瓶儿、孙雪娥和李桂姐众人,一个不少,都在跟前围着他坐的,听他演诵。先是,薛姑子道:

\[
盖闻电光易灭,石火难消。落花无返树之期,逝水绝归源之路。画堂绣阁,命尽有若长空;极品高官,禄绝犹如作梦。黄金白玉,空为祸患之资;红粉轻衣,总是尘劳之费。妻孥无百载之欢,黑暗有千重之苦。一朝枕上,命掩黄泉。青史扬虚假之名,黄土埋不坚之骨。田园百顷,其中被儿女争夺;绫锦千箱,死后无寸丝之分。青春未半,而白发来侵;贺者才闻,而吊者随至。苦,苦,苦!气化清风尘归土。点点轮回唤不回,改头换面无遍数。南无尽虚空遍法界,过去未来佛法僧三宝。无上甚深微妙法,百千万劫难遭遇。我今见闻得受持,愿解如来真实义。
\]

王姑子道:“当时释迦牟尼佛,乃诸佛之祖,释教之主,如何出家?愿听演说。”薛姑子便唱《五供养》:

\[
释迦佛,梵王子,舍了江山雪山去,割肉喂鹰鹊巢顶。只修的九龙吐水混金身,才成南无大乘大觉释迦尊。
\]
王姑子又道:“释迦佛既听演说,当日观音菩萨如何修行,才有庄严百化化身,有大道力?愿听其说——”

薛姑子正待又唱,只见平安儿慌慌张张走来说道:“巡按宋爷差了两个快手、一个门子送礼来。”月娘慌了,说道:“你爹往夏家吃酒去了,谁人打发他?”正说着,只见玳安儿回马来家,放进毡包来,说道:“不打紧,等我拿帖儿对爹说去。教姐夫且请那门子进来,管待他些酒饭儿着。”这玳安交下毡包,拿着帖子,骑马云飞般走到夏提刑家,如此这般,说巡按宋老爷送礼来。西门庆看了帖子,上写着“鲜猪一口,金酒二尊,公纸四刀,小书一部”,下书“侍生宋乔年拜”。连忙分咐:“到家交书童快拿我的官衔双摺手本回去,门子答赏他三两银子、两方手帕,抬盒的每人与他五钱。”玳安来家,到处寻书童儿,那里得来?急的只牛回磨转。陈敬济又不在,交傅伙计陪着人吃酒,玳安旋打后边讨了手帕、银子出来,又没人封,自家在柜上弥封停当,叫傅伙计写了,大小三包。因向平安儿道:“你就不知往那去了?”平安道:“头里姐夫在家时,他还在家来。落后姐夫往门外讨银子去了,他也不见了。”玳安道:“别要题,一定秫秫小厮在外边胡行乱走的,养老婆去了。”正在急唣之间,只见陈敬济与书童两个,叠骑骡子才来,被玳安骂了几句,教他写了官衔手本,打发送礼人去了。玳安道:“贼秫秫小厮,仰\textShan 着挣了合蓬着去。爹不在,家里不看,跟着人养老婆儿去了。爹又没使你和姐夫门外讨银子,你平白跟了去做甚么!看我对爹说不说!”书童道:“你说不是,我怕你?你不说就是我的儿。”玳安道:“贼狗攮的秫秫小厮,你赌几个真个?”走向前,一个泼脚撇翻倒,两个就磆碌成一块了。那玳安得手,吐了他一口唾沫才罢了。说道:“我接爹去,等我来家和淫妇算帐。”骑马一直去了。

月娘在后边,打发两个姑子吃了些茶食,又听他唱佛曲儿,宣念偈子。那潘金莲不住在旁先拉玉楼不动,又扯李瓶儿,又怕月娘说。月娘便道:“李大姐,他叫你,你和他去不是。省的急的他在这里恁有\textuni{34E6}划没是处的。”那李瓶儿方才同他出来。被月娘瞅了一眼,说道:“拔了萝卜地皮宽。交他去了,省的他在这里跑兔子一般。原不是听佛法的人。”

这潘金莲拉着李瓶儿走出仪门,因说道:“大姐姐好干这营生,你家又不死人,平白交姑子家中宣起卷来了。都在那里围着他怎的?咱们出来走走,就看看大姐在屋里做甚么哩。”于是一直走出大厅来。只见厢房内点着灯,大姐和敬济正在里面絮聒,说不见了银子。被金莲向窗棂上打了一下,说道:“后面不去听佛曲儿,两口子且在房里拌的甚么嘴儿?”陈敬济出来,看见二人,说道:“早是我没曾骂出来,原是五娘、六娘来了。请进来坐。”金莲道:“你好胆子,骂不是!”进来见大姐正在灯下纳鞋,说道:“这咱晚,热剌剌的,还纳鞋?”因问:“你两口子嚷的是些甚么?”陈敬济道:“你问他。爹使我门外讨银子去,他与了我三钱银子,就教我替他捎销金汗巾子来。不想到那里,袖子里摸银子没了,不曾捎得来。来家他说我那里养老婆,和我嚷骂了这一日,急的我赌身发咒。不想丫头扫地,地下拾起来。他把银子收了不与,还教我明日买汗巾子来。你二位老人家说,却是谁的不是?”那大姐便骂道:“贼囚根子,别要说嘴。你不养老婆,平白带了书童儿去做甚么?刚才教玳安甚么不骂出来!想必两个打伙儿养老婆去来。去到这咱晚才来,你讨的银子在那里?”金莲问道:“有了银子不曾?”大姐道:“刚才丫头扫地,拾起来,我拿着哩。”金莲道:“不打紧处。我与你些银子,明日也替我带两方销金汗巾子来。”李瓶儿便问:“姐夫,门外有,也捎几方儿与我。”敬济道:“门外手帕巷有名王家,专一发卖各色改样销金点翠手帕汗巾儿,随你要多少也有。你老人家要甚么颜色,销甚花样,早说与我,明日都替你一齐带的来了。”李瓶儿道:“我要一方老黄销金点翠穿花凤的。”敬济道:“六娘,老金黄销上金不现。”李瓶儿道:“你别要管我。我还要一方银红绫销江牙海水嵌八宝儿的,又是一方闪色芝麻花销金的。”敬济便道:“五娘,你老人家要甚花样?”金莲道:“我没银子,只要两方儿勾了。要一方玉色绫琐子地儿销金的。”敬济道:“你又不是老人家,白剌剌的,要他做甚么?”金莲道:“你管他怎的!戴不的,等我往后有孝戴。”敬济道:“那一方要甚颜色?”金莲道:“那一方,我要娇滴滴紫葡萄颜色四川绫汗巾儿。上销金间点翠,十样锦,同心结,方胜地儿——一个方胜儿里面一对儿喜相逢,两边栏子儿,都是缨络珍珠碎八宝儿。”敬济听了,说道:“耶嚛,耶嚛!再没了?卖瓜子儿打开箱子打嚏喷——琐碎一大堆。”金莲道:“怪短命,有钱买了称心货,随各人心里所好,你管他怎的!”李瓶儿便向荷包里拿出一块银子儿,递与敬济,说:“连你五娘的都在里头了。”金莲摇着头儿说道:“等我与他罢。”李瓶儿道:“都一答交姐夫捎了来,那又起个窖儿!”敬济道:“就是连五娘的,这银子还多着哩。”一面取等子称称,一两九钱。李瓶儿道:“剩下的就与大姑娘捎两方来。”大姐连忙道了万福。金莲道:“你六娘替大姐买了汗巾儿,把那三钱银子拿出来,你两口儿斗叶儿,赌了东道罢。少,便叫你六娘贴些儿出来,明日等你爹不在,买烧鸭子、白酒咱每吃。”敬济道:“既是五娘说,拿出来。”大姐递与金莲,金莲交付与李瓶儿收着。拿出纸牌来,灯下大姐与敬济斗。金莲又在旁替大姐指点,登时赢了敬济三掉。忽听前边打门,西门庆来家,金莲与李瓶儿才回房去了。

敬济出来迎接西门庆回了话,说徐四家银子,后日先送二百五十两来,余者出月交还。西门庆骂了几句,酒带半酣,也不到后边,迳往金莲房里来。正是:

\[
自有内事迎郎意,何怕明朝花不开。
\]

\newpage
%# -*- coding:utf-8 -*-
%%%%%%%%%%%%%%%%%%%%%%%%%%%%%%%%%%%%%%%%%%%%%%%%%%%%%%%%%%%%%%%%%%%%%%%%%%%%%%%%%%%%%


\chapter{应伯爵山洞戏春娇\KG 潘金莲花园调爱婿}


诗曰:

\[
春楼晓日珠帘映,红粉春妆宝镜催。
已厌交欢怜旧枕,相将游戏绕池台。
坐时衣带萦纤草,行处裙裾扫落梅。
更道明朝不当作,相期共斗管弦来。
\]

话说那日西门庆在夏提刑家吃酒,见宋巡按送礼,他心中十分欢喜。夏提刑亦敬重不同往日,拦门劝酒,吃至三更天气才放回家。潘金莲又早向灯下除去冠儿,设放衾枕,薰香澡牝等候。西门庆进门,接着,见他酒带半酣,连忙替他脱衣裳。春梅点茶吃了,打发上床歇息。见妇人脱得光赤条身子,坐在床沿,低垂着头,将那白生生腿儿横抱膝上缠脚,换了双大红平底睡鞋儿。西门庆一见,淫心辄起,麈柄挺然而兴。因问妇人要淫器包儿,妇人忙向褥子底下摸出来递与他。西门庆把两个托子都带上,一手搂过妇人在怀里,因说:“你达今日要和你干个‘后庭花儿’,你肯不肯?”那妇人瞅了一眼,说道:“好个没廉耻冤家,你成日和书童儿小厮干的不值了,又缠起我来了,你和那奴才干去不是!”西门庆笑道:“怪小油嘴,罢么!你若依了我,又稀罕小厮做甚么?你不知你达心里好的是这桩儿,管情放到里头去就过了。”妇人被他再三缠不不过,说道:“奴只怕挨不得你这大行货。你把头子上圈去了,我和你耍一遭试试。”西门庆真个除去硫磺圈,根下只束着银托子,令妇人马爬在床上,屁股高蹶,将唾津涂抹在龟头上,往来濡研顶入。龟头昂健,半晌仅没其棱。妇人在下蹙眉隐忍,口中咬汗巾子难捱,叫道:“达达慢着些。这个比不的前头,撑得里头热炙火燎的疼起来。”这西门庆叫道:“好心肝,你叫着达达,不妨事。到明日买一套好颜色妆花纱衣服与你穿。”妇人道:“那衣服倒也有在,我昨日见李桂姐穿的那玉色线掐羊皮挑的金油鹅黄银条纱裙子,倒好看,说是里边买的。他每都有,只我没这裙子。倒不知多少银子,你倒买一条我穿罢了。”西门庆道:“不打紧,我到明日替你买。”一壁说着,在上颇作抽拽,只顾没棱露脑,浅抽深送不已。妇人回首流眸叫道:“好达达,这里紧着人疼的要不的,如何只顾这般动作起来了?我央及你,好歹快些丢了罢!”这西门庆不听,且扶其股,玩其出入之势。一面口中呼道:“潘五儿,小淫妇儿,你好生浪浪的叫着达达,哄出你达达\textuni{379E}儿出来罢。”那妇人真个在下星眼朦胧,莺声款掉,柳腰款摆,香肌半就,口中艳声柔语,百般难述。良久,西门庆觉精来,两手扳其股,极力而\textuni{22D5E}之,扣股之声响之不绝。那妇人在下边呻吟成一块,不能禁止。临过之时,西门庆把妇人屁股只一扳,麈柄尽没至根,直抵于深异处,其美不可当。于是怡然感之,一泄如注。妇人承受其精,二体偎贴。良久拽出麈柄,但见猩红染茎,蛙口流涎,妇人以帕抹之,方才就寝。一宿晚景题过。

次日,西门庆早晨到衙门中回来,有安主事、黄主事那里差人来下请书,二十二日在砖厂刘太监庄上设席,请早去。西门庆打发来人去了,从上房吃了粥,正出厅来,只见篦头的小周儿扒倒地下磕头。西门庆道:“你来的正好,我正要篦篦头哩。”于是走到翡翠轩小卷棚内,坐在一张凉椅儿上,除了巾帻,打开头发。小周儿铺下梳篦家活,与他篦头栉发。观其泥垢,辨其风雪,跪下讨赏钱,说:“老爹今岁必有大迁转,发上气色甚旺。”西门庆大喜。篦了头,又叫他取耳,掐捏身上。他有滚身上一弄儿家活,到处与西门庆滚捏过,又行导引之法,把西门庆弄的浑身通泰。赏了他五钱银子,教他吃了饭,伺候着哥儿剃头。西门庆就在书房内,倒在大理石床上就睡着了。

那日杨姑娘起身,王姑子与薛姑子要家去。吴月娘将他原来的盒子都装了些蒸酥茶食,打发起身。两个姑子,每人都是五钱银子,两个小姑子,与了他两匹小布儿,管待出门。薛姑子又嘱咐月娘:“到了壬子日把那药吃了,管情就有喜事。”月娘道:“薛爷,你这一去,八月里到我生日,好来走走,我这里盼你哩。”薛姑子合掌问讯道:“打搅。菩萨这里,我到那日一定来。”于是作辞。月娘众人都送到大门首。月娘与大妗子回后边去了。只有玉楼、金莲、瓶儿、西门大姐、李桂姐抱着官哥儿,来到花园里游玩。李瓶儿道:“桂姐,你递过来,等我抱罢。”桂姐道:“六娘,不妨事,我心里要抱抱哥子。”玉楼道:“桂姐,你还没到你爹新收拾书房里瞧瞧哩。”到花园内,金莲见紫薇花开得烂熳,摘了两朵与桂姐戴。于是顺着松墙儿到翡翠轩,见里面摆设的床帐屏几、书画琴棋,极其潇洒。床上绡帐银钩,冰簟珊枕。西门庆倒在床上,睡思正浓。旁边流金小篆,焚着一缕龙涎。绿窗半掩,窗外芭蕉低映。潘金莲且在桌上掀弄他的香盒儿,玉楼和李瓶儿都坐在椅儿上,西门庆忽翻过身来,看刚见众妇人都在屋里,便道:“你每来做甚么?”金莲道:“桂姐要看看你的书房,俺每引他来瞧瞧。”那西门庆见他抱着官哥儿,又引逗了一回。忽见画童来说:“应二爹来了。”众妇人都乱走不迭,往李瓶儿那边去了。应伯爵走到松墙边,看见桂姐抱着官哥儿,便道:“好呀!李桂姐在这里。”故意问道:“你几时来?”那桂姐走了,说道:“罢么,怪花子!又不关你事,问怎的?”伯爵道:“好小淫妇儿,不关我事也罢,你且与我个嘴着。”于是搂过来就要亲嘴。被桂姐用手只一推,骂道:“贼不得人意怪攮刀子,若不是怕唬了哥子,我这一扇把子打的你……”西门庆走出来看见,说道:“怪狗才,看唬了孩儿!”因教书童:“你抱哥儿送与你六娘去。”那书童连忙接过来。奶子如意儿正在松墙拐角边等候,接的去了。伯爵和桂姐两个站着说话,问:“你的事怎样了?”桂姐道:“多亏爹这里可怜见,差保哥替我往东京说去了。”伯爵道:“好,好,也罢了。如此你放心些。”说毕,桂姐就往后边去了。伯爵道:“怪小淫妇儿,你过来,我还和你说话。”桂姐道:“我走走就来。”于是也往李瓶儿这边来了。

伯爵与西门庆才唱喏坐的。西门庆道:“昨日我在夏龙溪家吃酒,大巡宋道长那里差人送礼,送了一口鲜猪。我恐怕放不的,今早旋叫厨子来卸开,用椒料连猪头烧了。你休去,如今请谢子纯来,咱每打双陆,同享了罢。”一面使琴童儿:“快请你谢爹去。你说应二爹在这里。”琴童儿应诺去了。伯爵因问:“徐家银子讨来了不曾?”西门庆道:“贼没行止的狗骨秃,明日才先与二百五十两。你教他两个后日来,少的,我家里凑与他罢。”伯爵道:“这等又好了。怕不得他今日也买些鲜物儿来孝顺你。”西门庆道:“倒不消教他费心。”说了一回,西门庆问道:“老孙、祝麻子两个都起身去了不曾?”伯爵道:“自从李桂儿家拿出来,在县里监了一夜,第二日,三个一条铁索,都解上东京去了。到那里,没个清洁来家的!你只说成日图饮酒吃肉,好容易吃的果子儿!似这等苦儿,也是他受。路上这等大热天,着铁索扛着,又没盘缠,有甚么要紧。”西门庆笑道:“怪狗才,充军摆战的不过!谁教他成日跟着王家小厮只胡撞来!他寻的苦儿他受。”伯爵道:“哥说的有理。苍蝇不钻没缝的鸡蛋,他怎的不寻我和谢子纯?清的只是清,浑的只是浑。”

正说着,谢希大到了。唱毕喏坐下,只顾扇扇子。西门庆问道:“你怎的走恁一脸汗?”希大道:“哥别题起。今日平白惹了一肚子气。大清早晨,老孙妈妈子走到我那里,说我弄了他去。恁不合理的老淫妇!你家汉子成日摽着人在院里大酒大肉吃,大把挝了银子钱家去,你过阴去来?谁不知道!你讨保头钱,分与那个一分儿使也怎的?交我扛了两句走出来。不想哥这里呼唤。”伯爵道:“我刚才和哥不说,新酒放在两下里,清自清,浑自浑。当初咱每怎么说来?我说跟着王家小厮,到明日有一失。今日如何?撞到这网里,怨怅不的人!”西门庆道:“王家那小厮,有甚大气概?脑子还未变全,养老婆!还不勾俺每那咱撒下的,羞死鬼罢了!”伯爵道:“他曾见过甚么大头面目,比哥那咱的勾当,题起来把他唬杀罢了。”说毕,小厮拿茶上来吃了。西门庆道:“你两个打双陆。后边做着水面,等我叫小厮拿来咱每吃。”不一时,琴童来放桌儿。画童儿用方盒拿上四个小菜儿,又是三碟儿蒜汁、一大碗猪肉卤,一张银汤匙、三双牙箸。摆放停当,三人坐下,然后拿上三碗面来,各人自取浇卤,倾上蒜醋。那应伯爵与谢希大拿起箸来,只三扒两咽就是一碗。两人登时狠了七碗。西门庆两碗还吃不了,说道:“我的儿,你两个吃这些!”伯爵道:“哥,今日这面是那位姐儿下的?又好吃又爽口。”谢希大道:“本等卤打的停当,我只是刚才吃了饭了,不然我还禁一碗。”两个吃的热上来,把衣服脱了。见琴童儿收家活,便道:“大官儿,到后边取些水来,俺每漱漱口。”谢希大道:“温茶儿又好,热的烫的死蒜臭。”少顷,画童儿拿茶至。三人吃了茶,出来外边松墙外各花台边走了一道。只见黄四家送了四盒子礼来。平安儿掇进来与西门庆瞧:一盒鲜乌菱、一盒鲜荸荠、四尾冰湃的大鲥鱼、一盒枇杷果。伯爵看见说道:“好东西儿!他不知那里剜的送来,我且尝个儿着。”一手挝了好几个,递了两个与谢希大,说道:“还有活到老死,还不知此是甚么东西儿哩。”西门庆道:“怪狗才,还没供养佛,就先挝了吃?”伯爵道:“甚么没供佛,我且入口无赃着。”西门庆分咐:“交到后边收了。问你三娘讨三钱银子赏他。”伯爵问:“是李锦送来,是黄宁儿?”平安道:“是黄宁儿。”伯爵道:“今日造化了这狗骨秃了,又赏他三钱银子。”这里西门庆看着他两个打双陆不题。

且说月娘和桂姐、李娇儿、孟玉楼、潘金莲、李瓶儿、大姐,都在后边吃了饭,在穿廊下坐的。只见小周儿在影壁前探头舒脑的,李瓶儿道:“小周儿,你来的好。且进来与小大官儿剃剃头,他头发都长长了。”小周儿连忙向前都磕了头,说:“刚才老爹分咐,交小的进来与哥儿剃头。”月娘道:“六姐,你拿历头看看,好日子,歹日子,就与孩子剃头?”金莲便交小玉取了历头来,揭开看了一回,说道:“今日是四月廿一日,是个庚戌日,金定娄金狗当直,宜祭祀、官带、出行、裁衣、沐浴、剃头、修造、动土,宜用午时。——好日期。”月娘道:“既是好日子,叫丫头热水,你替孩儿洗头,教小周儿慢慢哄着他剃。”小玉在旁替他用汗巾儿接着头发,才剃得几刀,这官哥儿呱的怪哭起来。那小周连忙赶着他哭只顾剃,不想把孩子哭的那口气憋下去,不做声了,脸便胀的红了。李瓶儿唬慌手脚,连忙说:“不剃罢,不剃罢!”那小周儿唬的收不迭家活,往外没脚的跑。月娘道:“我说这孩予有些不长俊,护头。自家替他剪剪罢。平白教进来剃,剃的好么!”天假其便,那孩子憋了半日气,才放出声来。李瓶儿方才放心,只顾拍哄他,说道:“好小周儿,恁大胆!平白进来把哥哥头来剃了去了。剃的恁半落不合的,欺负我的哥哥。还不拿回来,等我打与哥哥出气。”于是抱到月娘跟前。月娘道:“不长俊的小花子儿,剃头耍了你了,这等哭?剩下这些,到明日做剪毛贼。”引逗了一回,李瓶儿交与奶子。月娘分咐:“且休与他奶吃,等他睡一回儿与他吃。”奶子抱的前边去了。只见来安儿进来取小周儿的家活,说唬的小周儿脸焦黄的。月娘问道:“他吃了饭不曾?”来安道:“他吃了饭。爹赏他五钱银子。”月娘教来安:“你拿一瓯子酒出去与他。唬着人家,好容易讨这几个钱!”小玉连忙筛了一盏,拿了一碟腊肉,教来安与他吃了去了。

吴月娘因教金莲:“你看看历头,几时是壬子日?”金莲看了,说道:“二十三日是壬子日,交芒种五月节。”便道:“姐姐你问他怎的?”月娘道:“我不怎的,问一声儿。”李桂姐接过历头来看了,说道:“这二十四日,苦恼是俺娘的生日!我不得在家。”月娘道:“前月初十日,是你姐姐生日,过了。这二十四日,可可儿又是你妈的生日了。原来你院中人家一日害两样病,做三个生日:日里害思钱病,黑夜思汉子的病。早晨是妈妈的生日,晌午是姐姐生日,晚夕是自家生日。——怎的都挤在一块儿?趁着姐夫有钱,撺掇着都生日了罢!”桂姐只是笑,不做声。只见西门庆使了画童儿来请,桂姐方向月娘房中妆点匀了脸,往花园中来。

卷棚内,又早放下八仙桌儿,桌上摆设两大盘烧猪肉并许多肴馔。众人吃了一回,桂姐在旁拿锺儿递酒,伯爵道:“你爹听着说,不是我索落你,人情儿已是停当了。你爹又替你县中说了,不寻你了。亏了谁?还亏了我再三央及你爹,他才肯了。平白他肯替你说人情去?随你心爱的甚么曲儿,你唱个儿我下酒,也是拿勤劳准折。”桂姐笑骂道:“怪硶花子,你虼蚤包网儿——好大面皮!爹他肯信你说话?”伯爵道:“你这贼小淫妇儿!你经还没念,就先打和尚。要吃饭,休恶了火头!你敢笑和尚投丈母,我就单丁摆布不起你这小淫妇儿?你休笑话,我半边俏还动的。”被桂姐把手中扇把子,尽力向他身上打了两下。西门庆笑骂道:“你这狗才,到明日论个男盗女娼,还亏了原问处。”笑了一回,桂姐慢慢才拿起琵琶,横担膝上,启朱唇,露皓齿,唱道:

\[
\cipaim{黄莺儿}谁想有这一种。减香肌,憔瘦损。镜鸾尘锁无心整。脂粉倦匀,花枝又懒簪。空教黛眉蹙破春山恨。
\]
伯爵道:“你两个当初好来,如今就为他耽些惊怕儿,也不该抱怨了。”桂姐道:“汗邪了你,怎的胡说!”——

\[
最难禁,谯楼上画角,吹彻了断肠声。
\]
伯爵道:“肠子倒没断,这一回来提你的断了线,你两个休提了。”被桂姐尽力打了一下,骂道:“贼攘刀的,今日汗邪了你,只鬼混人的。”——
\[
\cipaim{集资宾}幽窗静悄月又明,恨独倚帏屏。蓦听的孤鸿只在楼外鸣,把万愁又还题醒。更长漏永,早不觉灯昏香烬眠未成。他那里睡得安稳!
\]
伯爵道:“傻小淫妇儿,他怎的睡不安稳?又没拿了他去。落的在家里睡觉儿哩。你便在人家躲着,逐日怀着羊皮儿,直等东京人来,一块石头方落地。”桂姐被他说急了,便道:“爹,你看应花子,不知怎的,只发讪缠我。”伯爵道:“你这回才认的爹了?”桂姐不理他,弹着琵琶又唱:

\[
\cipaim{双声叠韵}思量起,思量起,怎不上心?无人处,无人处,泪珠儿暗倾。
\]
伯爵道:“一个人惯溺尿。一日,他娘死了,守孝打铺在灵前睡。晚了,不想又溺下了。人进来看见褥子湿,问怎的来,那人没的回答,只说:‘你不知,我夜间眼泪打肚里流出来了。’——就和你一般,为他声说不的,只好背地哭罢了。”桂姐道:“没羞的孩儿,你看见来?汗邪了你哩!”——

\[
我怨他,我怨他,说他不尽,谁知道这里先走滚。自恨我当初不合他认真。
\]
伯爵道:“傻小淫妇儿,如今年程,三岁小孩儿也哄不动,何况风月中子弟。你和他认真?你且住了,等我唱个南曲儿你听:‘风月事,我说与你听:如今年程,论不得假真。个个人古怪精灵,个个人久惯牢成,倒将计活埋把瞎缸暗顶。老虔婆只要图财,小淫妇儿少不得拽着脖子往前挣。苦似投河,愁如觅并。几时得把业罐子填完,就变驴变马也不干这营生。’”当下把桂姐说的哭起来了。被西门庆向伯爵头上打了一扇子,笑骂道:“你这搊断肠子的狗才!生生儿吃你把人就欧杀了。”因叫桂姐:“你唱,不要理他。”谢希大道:“应二哥,你好没趣!今日左来右去只欺负我这干女儿。你再言语,口上生个大疔疮。”那桂姐半日拿起琵琶,又唱:

\[
\cipaim{簇御林}人都道他志诚。
\]
伯爵才待言语,被希大把口按了,说道:“桂姐你唱,休理他!”桂姐又唱道:

\[
却原来厮勾引。眼睁睁心口不相应。
\]
希大放了手,伯爵又说:“相应倒好了。心口里不相应,如今虎口里倒相应。不多,也只三两炷儿。”桂姐道:“白眉赤眼,你看见来?”伯爵道:“我没看见,在乐星堂儿里不是?”连西门庆众人都笑起来了。桂姐又唱:

\[
山盟海誓,说假道真,险些儿不为他错害了相思病。负人心,看伊家做作,如何教我有前程?
\]
伯爵道:“前程也不敢指望他,到明日,少不了他个招宣袭了罢。”桂姐又唱:

\[
\cipaim{琥珀猫儿坠}日疏日远,何日再相逢?枉了奴痴心宁耐等。想巫山云雨梦难成。薄情,猛拚今生和你凤拆鸾零。
\cipaim{尾声}冤家下得忒薄幸,割舍的将人孤另。那世里的恩情翻成做话柄。
\]

唱毕,谢希大道:“罢,罢。叫画童儿接过琵琶去,等我酬劳桂姐一杯酒儿,消消气罢。”伯爵道:“等我哺菜儿。我本领儿不济事,拿勤劳准折罢了。”桂姐道:“花子过去,谁理你!你大拳打了人,这回拿手来摸挲。”当下,希大一连递了桂姐三杯酒,拉伯爵道:“咱每还有那两盘双陆,打了罢。”于是二人又打双陆。西门庆递了个眼色与桂姐,就往外走。伯爵道:“哥,你往后边左,捎些香茶儿出来。头里吃了些蒜,这回子倒反恶泛泛起来了。”西门庆道:“我那里得香茶来!”伯爵道:“哥,你还哄我哩,杭州刘学官送了你好少儿,你独吃也不好。”西门庆笑的后边去了。桂姐也走出来,在太湖石畔推摘花儿戴,也不见了。伯爵与希大一连打了三盘双陆,等西门庆白不见出来。问画童儿:“你爹在后边做甚么哩?”画童儿道:“爹在后边,就出来了。”伯爵道:“就出来,有些古怪!”因交谢希大:“你这里坐着,等我寻他寻去。”那谢希大且和书童儿两个下象棋。

原来西门庆只走到李瓶儿房里,吃了药就出来了。在木香棚下看见李桂姐,就拉到藏春坞雪洞儿里,把门儿掩着,坐在矮床儿上,把桂姐搂在怀中,腿上坐的,一径露出那话来与他瞧,把桂姐唬了一跳。便问:“怎的就这般大?”西门庆悉把吃胡僧药告诉了一遍。先交他低垂粉颈,款启猩唇,品咂了一回。然后,轻轻搊起他两只小小金莲来,跨在两边胳膊上,抱到一张椅儿上,两个就干起来。不想应伯爵到各亭儿上寻了一遭,寻不着,打滴翠岩小洞儿里穿过去,到了木香棚,抹过葡萄架,到松竹深处,藏春坞边,隐隐听见有人笑声,又不知在何处。这伯爵慢慢蹑足潜踪,掀开帘儿,见两扇洞门儿虚掩,在外面只顾听觑。听见桂姐颤着声儿,将身子只顾迎播着西门庆,叫:“达达,快些了事罢,只怕有人来。”被伯爵猛然大叫一声,推开门进来,看见西门庆把桂姐扛着腿子正干得好。说道:“快取水来,泼泼两个搂心的,搂到一答里了!”李桂姐道:“怪攘刀子,猛的进来,唬了我一跳!”伯爵道:“快些儿了事?好容易!也得值那些数儿是的。怕有人来看见,我就来了。且过来,等我抽个头儿着。”西门庆便道:“怪狗才,快出去罢了,休鬼混!我只怕小厮来看见。”那应伯爵道:“小淫妇儿,你央及我央及儿。不然我就吆喝起来,连后边嫂子每都嚷的知道。你既认做干女儿了,好意教你躲住两日儿,你又偷汉子。教你了不成!”桂姐道:“去罢,应怪花子!”伯爵道:“我去罢?我且亲个嘴着。”于是按着桂姐亲了一个嘴,才走出来。西门庆道:“怪狗才,还不带上门哩。”伯爵一面走来把门带上,说道:“我儿,两个尽着捣,尽着捣,捣吊底也不关我事。”才走到那个松树儿底下,又回来说道:“你头里许我的香茶在那里?”西门庆道:“怪狗才,等住回我与你就是了,又来缠人!”那伯爵方才一直笑的去了。桂姐道:“好个不得人意的攮刀子!”这西门庆和那桂姐两个,在雪洞内足干勾一个时辰,吃了一枚红枣儿,才得了事,雨散云收。有诗为证:

\[
海棠技上莺梭急,绿竹阴中燕语频。
闲来付与丹青手,一段春娇画不成。
\]

少顷,二人整衣出来。桂姐向他袖子内掏出好些香茶来袖了。西门庆使的满身香汗,气喘吁吁,走来马缨花下溺尿。李桂姐腰里摸出镜子来,在月窗上搁着,整云理鬓,往后边去了。

西门庆走到李瓶儿房里,洗洗手出来。伯爵问他要香茶,西门庆道:“怪花子,你害了痞,如何只鬼混人!”每人掐了一撮与他。伯爵道:“只与我这两个儿!由他,由他!等我问李家小淫妇儿要。”正说着,只见李铭走来磕头。伯爵道:“李日新在那里来?你没曾打听得他每的事怎么样儿了?”李铭道:“俺桂姐亏了爹这里。这两日,县里也没人来催,只等京中示下哩。”伯爵道:“齐家那小老婆子出来了?”李铭道:“齐香儿还在王皇亲宅内躲着哩。桂姐在爹这里好,谁人敢来寻?”伯爵道:“要不然也费手,亏我和你谢爹再三央劝你爹:‘你不替他处处儿,教他那里寻头脑去!’”李铭道:“爹这里不管,就了不成。俺三婶老人家,风风势势的,干出甚么事!”伯爵道:“我记的这几时是他生日,俺每会了你爹,与他做做生日。”李铭道:“爹每不消了。到明日事情毕了,三婶和桂姐,愁不请爹每坐坐?”伯爵道:“到其间,俺每补生日就是了。”因叫他近前:“你且替我吃了这锺酒着。我吃了这一日,吃不的了。”那李铭接过银把锺来,跪着一饮而尽。谢希大交琴童又斟了一锺与他。伯爵道:“你敢没吃饭?”桌上还剩了一盘点心,谢希大又拿两盘烧猪头肉和鸭子递与他。李铭双手接的,下边吃去了。伯爵用箸子又拨了半段鲥鱼与他,说道:“我见你今年还没食这个哩,且尝新着。”西门庆道:“怪狗才,都拿与他吃罢了,又留下做甚么?”伯爵道:“等住回吃的酒阑,上来饿了,我不会吃饭儿?你们那里晓得,江南此鱼一年只过一遭儿,吃到牙缝里剔出来都是香的。好容易!公道说,就是朝廷还没吃哩!不是哥这里,谁家有?”正说着,只见画童儿拿出四碟鲜物儿来:一碟乌菱、一碟荸荠、一碟雪藕、一碟枇杷。西门庆还没曾放到口里,被应伯爵连碟子都挝过去,倒的袖了。谢希大道:“你也留两个儿我吃。”也将手挝一碟子乌菱来。只落下藕在桌子上。西门庆掐了一块放在口内,别的与了李铭吃了。分付画童后边再取两个枇杷来赏李铭。李铭接的袖了,才上来拿筝弹唱。唱了一回,伯爵又出题目,叫他唱了一套《花药栏》。三个直吃到掌灯时候,还等后边拿出绿豆白米水饭来吃了,才起身。伯爵道:“哥,我晓得明日安主事请你,不得闲。李四、黄三那事,我后日会他来罢。”西门庆点头儿,二人也不等送,就去了。西门庆教书童看收家伙,就归后边孟玉楼房中歇去了。一宿无话。

到次日早起,也没往衙门中去,吃了粥,冠带骑马,书童、玳安两个跟随,出城南三十里,迳往刘太监庄上来赴席,不在话下。

潘金莲赶西门庆不在家,与李瓶儿计较,将陈敬济输的那三钱银子,又教李瓶儿添出七钱来,教来兴儿买了一只烧鸭、两只鸡、一钱银子下饭、一坛金华酒、一瓶白酒、一钱银子裹馅凉糕,教来兴儿媳妇整理端正。金莲对着月娘说:“大姐那日斗牌,赢了陈姐夫三钱银子,李大姐又添了些,今治了东道儿,请姐姐在花园里吃。”吴月娘就同孟玉楼、李娇儿、孙雪娥、大姐、桂姐众人,先在卷棚内吃了一回,然后拿酒菜儿,在山子上卧云亭下棋,投壶,吃酒耍子。月娘想起问道:“今日主人,怎倒不来坐坐?”大姐道:“爹又使他往门外徐家催银子去了,也好待来也。”

不一时,陈敬济来到,向月娘众人作了揖,就拉过大姐一处坐下。向月娘说:“徐家银子讨了来了,共五封二百五十两,送到房里,玉箫收了。”于是传杯换盏,酒过数巡,各添春色。月娘与李娇儿、桂姐三个下棋,玉楼众人都起身向各处观花玩草耍子。惟金莲独自手摇着白团纱扇儿,往山子后芭蕉深处纳凉。因见墙角草地下一朵野紫花儿可爱,便走去要摘。不想敬济有心,一眼睃见,便悄悄跟来,在背后说道:“五娘,你老人家寻甚么?这草地上滑齑齑的,只怕跌了你,教儿子心疼。”那金莲扭回粉颈,斜睨秋波,带笑带骂道:“好个贼短命的油嘴,跌了我,可是你就心疼哩?谁要你管!你又跟了我来做甚么,也不怕人看着。”因问:“你买的汗巾儿怎了?”敬济笑嘻嘻向袖于中取出,递与他,说道:“六娘的都在这里了。”又道:“汗巾儿买了来,你把甚来谢我?”于是把脸子挨的他身边,被金莲举手只一推。不想李瓶儿抱着官哥儿,并奶子如意儿跟着,从松墙那边走来。见金莲手拿自团扇一动,不知是推敬济,只认做扑蝴蝶,忙叫道:“五妈妈,扑的蝴蝶儿,把官哥儿一个耍子。”慌的敬济赶眼不见,两三步就钻进山子里边去了。金莲恐怕李瓶儿瞧见,故意问道:“陈姐夫与了汗巾不曾?”李瓶儿道:“他还没有与我哩。”金莲道:“他刚才袖着,对着大姐姐不好与咱的,悄悄递与我了。”于是两个坐在芭蕉丛下花台石上,打开分了。两个坐了一回,李瓶儿说道:“这答儿里到且是荫凉。”因使如意儿:“你去叫迎春屋里取孩子的小枕头并凉席儿来,就带了骨牌来,我和五娘在这里抹回骨牌儿。你就在屋里看罢。”如意儿去了。

不一时,迎春取了枕席并骨牌来。李瓶儿铺下席,把官哥儿放在小枕头儿上躺着,教他顽耍,他便和金莲抹牌。抹了一回,交迎春往屋里拿一壶好茶来。不想盂玉楼在卧云亭上看见,点手儿叫李瓶儿说:“大姐姐叫你说句话儿。”李瓶儿撇下孩子,教金莲看着:“我就来。”那金莲记挂敬济在洞儿里,那里又去顾那孩子,赶空儿两三步走入洞门首,教敬济,说:“没人,你出来罢。”敬济便叫妇人进去瞧蘑菇:“里面长出这些大头蘑菇来了。”哄的妇人入到洞里,就折叠腿跪着,要和妇人云雨。两个正接着亲嘴。也是天假其便,李瓶儿走到亭子上,月娘说:“孟三姐和桂姐投壶输了,你来替他投两壶儿。”李瓶儿道:“底下没人看孩子哩。”玉楼道:“左右有六姐在那里,怕怎的。”月娘道:“孟三姐,你去替他看看罢。”李瓶儿道:“三娘累你,亦发抱了他来罢。”教小玉:“你去就抱他的席和小枕头儿来。”那小玉和玉楼走到芭蕉丛下,孩子便躺在席上,蹬手蹬脚的怪哭,并不知金莲在那里。只见旁边一个大黑猫,见人来,一溜烟跑了。玉楼道:“他五娘那里去了?耶嚛,耶嚛!把孩子丢在这里,吃猫唬了他了。”那金莲连忙从雪洞儿里钻出来,说道:“我在这里净了净手,谁往那里去来!那里有猫唬了他?白眉赤眼的!”那玉楼也更不往洞里看,只顾抱了官哥儿,拍哄着他往卧云亭儿上去了。小玉拿着枕席跟的去了。金莲恐怕他学舌,随屁股也跟了来。月娘问:“孩子怎的哭?”玉楼道:“我去时,不知是那里一个大黑猫蹲在孩子头跟前。”月娘说:“干净唬着孩儿。”李瓶儿道,“他五娘看着他哩。”玉楼道:“六姐往洞儿里净手去来。”金莲走上来说:“三姐,你怎的恁白眉赤眼儿的?那里讨个猫来!他想必饿了,要奶吃哭,就赖起人来。”李瓶几见迎春拿上茶来,就使他叫奶子来喂哥儿奶。

陈敬济见无人,从洞儿钻出来,顺着松墙儿转过卷棚,一直往外去了。正是:

\[
两手劈开生死路。一身跳出是非门。
\]


月娘见孩子不吃奶,只是哭,分咐李瓶儿:“你抱他到屋里,好好打发他睡罢。”于是也不吃酒,众人都散了。原来陈敬济也不曾与潘金莲得手,事情不巧,归到前边厢房中,有些咄咄不乐。正是:

\[
无可奈何花落去,似曾相识燕归来。
\]


\newpage
%# -*- coding:utf-8 -*-
%%%%%%%%%%%%%%%%%%%%%%%%%%%%%%%%%%%%%%%%%%%%%%%%%%%%%%%%%%%%%%%%%%%%%%%%%%%%%%%%%%%%%


\chapter{潘金莲惊散幽欢\KG 吴月娘拜求子息}


词曰:

\[
小院闲阶玉砌,墙隈半簇兰芽。一庭萱草石榴花,多子宜男爱插。休使风吹雨打,老天好为藏遮。莫教变作杜鹃花,粉褪红销香罢。
\]

话说陈敬济与金莲不曾得手,怅怏不题。单表西门庆赴黄、安二主事之席。乘着马,跟随着书童、玳安四五人,来到刘太监庄上。早有承局报知,黄、安二主事忙整衣冠,出来迎接。那刘太监是地主,也同来相迎。西门庆下了马,刘太监一手挽了西门庆,笑道:“咱三个等候的好半日了,老丈却才到来。”西门庆答道:“蒙两位老先生见招,本该早来,实为家下有些小事,反劳老公公久待,望乞恕罪。”三个大打恭,进仪门来。让到厅上,西门庆先与黄主事作揖,次与安主事、刘太监都作了揖,四人分宾主而坐。第一位让西门庆坐了,第二就该刘太监坐。刘太监再四不肯,道:“咱忝是房主,还该两位老先生,是远客。”安主事道:“定是老先儿。”西门庆道:“若是序齿,还该刘公公。”刘大监推却不过,向黄、安两主事道:“斗胆占了。”便坐了第二位。黄、安二主事坐了主席。一班小优儿上来磕了头,左右献过茶,当值的就递上酒来。黄、安二主事起身安席坐下。小优儿拿檀板、琵琶、弦索、箫管上来,合定腔调,细细唱了一套《宜春令》“青阳候烟雨淋”。唱毕,刘太监举杯劝众官饮酒。安主事道:“这一套曲儿,做的清丽无比,定是一个绝代才子。况唱的声音嘹亮,响遏行云,却不是个双绝了么!”西门庆道:“那个也不当奇,今日有黄、安二位做了贤主,刘公公做了地主,这才是难得哩!”黄主事笑道:“也不为奇。刘公公是出入紫禁,日觐龙颜,可不是贵臣?西门老丈,堆金积玉,仿佛陶朱,可不是富人?富贵双美,这才是奇哩!”四个人哈哈大笑。当值的斟上酒来,又饮了一回。小优儿又拿碧玉洞箫,吹得悠悠咽咽,和着板眼,唱一套《沽美酒》“桃花溪,杨柳腰”的时曲。唱毕,众客又赞了一番,欢乐饮酒不题。

且说陈敬济因与金莲不曾得手,耐不住满身欲火。见西门庆吃酒到晚还未来家,依旧闪入卷棚后面,探头探脑张看。原来金莲被敬济鬼混了一场,也十分难熬,正在无人处手托香腮,沉吟思想。不料敬济三不知走来,黑影子里看见了,恨不的一碗水咽将下去。就大着胆,悄悄走到背后,将金莲双手抱住,便亲了个嘴,说道:“我前世的娘!起先吃孟三儿那冤儿打开了,几乎把我急杀了。”金莲不提防,吃了一吓。回头看见是敬济,心中又惊又喜,便骂道:“贼短命,闪了我一闪,快放手,有人来撞见怎了!”敬济那里肯放,便用手去解他裤带。金莲犹半推半就,早被敬济一扯扯断了。金莲故意失惊道:“怪贼囚,好大胆!就这等容容易易要奈何小丈母!”敬济再三央求道:“我那前世的亲娘,要敬济的心肝煮汤吃,我也肯割出来。没奈何,只要今番成就成就。”敬济口里说着,腰下那话已是硬帮帮的露出来,朝着金莲单裙只顾乱插。金莲桃颊红潮,情动久了。初还假做不肯,及被敬济累垂敖曹触着,就禁不的把手去摸。敬济便趁势一手掀开金莲裙子,尽力往内一插,不觉没头露脑。原来金莲被缠了一回,臊水湿漉漉的,因此不费力送进了。两个紧傍在红栏干上,任意抽送,敬济还嫌不得到根,教金莲倒在地下:“待我奉承你一个不亦乐乎!”金莲恐散了头发,又怕人来,推道:“今番且将就些,后次再得相聚,凭你便了。”一个“达达”连声,一个“亲亲”不住,厮併了半个时辰。只听得隔墙外籁籁的响,又有人说话,两个一哄而散。

敬济云情未已,金莲雨意方浓。却是书童、玳安拿着冠带拜匣,都醉醺醺的嚷进门来。月娘听见,知道是西门庆来家,忙差小玉出来看。书童、玳安道:“爹随后就到了。我两人怕晚了,先来了。”不多时,西门庆下马进门,已醉了,直奔到月娘房里来。搂住月娘就待上床。月娘因要他明日进房,应二十三壬子日服药行事,便不留他,道:“今日我身子不好,你往别房里去罢。”西门庆笑道:“我知道你嫌我醉了,不留我。也罢,别要惹你嫌。我去了,明晚来罢。”月娘笑道:“我真有些不好,月经还未净。谁嫌你?明晚来罢。”西门庆就往潘金莲房里去了。金莲正与敬济不尽兴回房,眠在炕上,一见西门庆进来,忙起来笑迎道:“今日吃酒,这咱时才来家。”西门庆也不答应,一手搂将过来,连亲了几个嘴,一手就下边一摸,摸着他牝户,道:“怪小淫妇儿,你想着谁来?兀那话湿搭搭的。”金莲自觉心虚,也不做声。只笑推开了西门庆,向后边澡牝去了。当晚与西门庆云情雨意,不消说得。

且表吴月娘次日起身,正是二十三壬子日,梳洗毕,就教小玉摆着香桌,上边放着宝炉,烧起名香,又放上《白衣观音经》一卷。月娘向西皈依礼拜,拈香毕,将经展开,念一遍,拜一拜,念了二十四遍,拜了二十四拜,圆满。然后箱内取出丸药放在桌上,又拜了四拜,祷告道:“我吴氏上靠皇天,下赖薛师父、王师父这药,仰祈保佑,早生子嗣。”告毕,小玉烫的热酒,倾在盏内。月娘接过酒盏,一手取药调匀,西向跪倒,先将丸药咽下,又取末药也服了,喉咙内微觉有些腥气。月娘迸着气一口呷下,又拜了四拜。当日不出房,只在房里坐的。

西门庆在潘金莲房中起身,就叫书童写谢宴贴,往黄、安二主事家谢宴。书童去了,就是应伯爵来到。西门庆出来,应伯爵作了揖,说道:“哥,昨在刘太监家吃酒,几时来家?”西门庆道:“承两公十分相爱,灌了好几杯酒,归路又远,更余来家。已是醉了,这咱才起身。”玳安捧出早饭,西门庆正和伯爵同吃,又报黄主事、安主事来拜。西门庆整衣冠,教收过家活出迎。应伯爵忙回避了。黄、安二主事一齐下轿。进门厮见毕,三人坐下,一面捧出茶来吃了。黄、安二主事道:“夜来有亵,”西门庆道:“多感厚情,正要叩谢两位老先生,如何反劳台驾先施!”安主事道:“昨晚老先生还未尽兴,为何就别了?”西门庆道:“晚生已大醉了。临起身,又被刘公公灌上十数杯葡萄酒,在马上就要呕,耐得到家,睡到今日还有些不醒哩。”笑了一番,又吃过三杯茶,说些闲话,作别去了。应伯爵也推事故家去。西门庆回进后边吃了饭,就坐轿答拜黄、安二主事去。又写两个红礼帖,吩咐玳安备办两副下程,赶到他家面送。当日无话。

西门庆来家,吴月娘打点床帐,等候进房。西门庆进了房,月娘就教小玉整设肴馔,烫酒上来,两人促膝而坐。西门庆道:“我昨夜有了杯酒,你便不肯留我,又假推甚么身子不好,这咱捣鬼!”月娘道,“这不是捣鬼,果然有些不好。难道夫妻之间恁地疑心?”西门庆吃了十数杯酒,又吃了些鲜鱼鸭腊,便不吃了,月娘交收过了。小玉熏的被窝香喷喷的,两个洗澡已毕,脱衣上床。枕上绸缪,被中缱绻,言不可尽。这也是吴月娘该有喜事,恰遇月经转,两下似水如鱼,便得了子了。正是:

\[
花有并头莲并蒂,带宜同挽结同心。
\]

次日,西门庆起身梳洗,月娘备有羊羔美酒、鸡子腰子补肾之物,与他吃了,打发进衙门去。西门庆衙门散了回来,就进李瓶儿房看哥儿。李瓶儿抱着孩子向西门庆道:“前日我有些心愿未曾了。这两日身子有些不好,坐净桶时,常有些血水淋得慌。早晚要酬酬心愿,你又忙碌碌的,不得个闲空。”西门庆道:“你既要了愿时,我叫玳安去接王姑子来,与他商量,做些好事就是了。”便叫玳安,吩咐接王姑子。玳安应诺去了。

书童又报:“常二叔和应二爹来到。”西门庆便出迎厮见。应伯爵道:“前日谢子纯在这里吃酒,我说的黄四、李三的那事,哥应付了他罢。”西门庆道:“我那里有银子?”应伯爵道:“哥前日已是许下了,如何又变了卦?哥不要瞒我,等地财主,说个无银出来?随分凑些与他罢。”西门庆不答应他,只顾呆了脸看常峙节。常峙节道:“连日不曾来,哥,小哥儿长养么?”西门庆道:“生受注念,却才你李家嫂子要酬心愿,只得去请王姑子来家做些好事。”应伯爵道:“但凡人家富贵,专待子孙掌管。养得来时,须要十分保护。譬如种五谷的,初长时也得时时灌溉,才望个秋收。小哥儿万金之躯,是个掌中珠,又比别的不同。小儿郎三岁有关,六岁有厄,九岁有煞,又有出痧出痘等症。哥,不是我口直,论起哥儿,自然该与他做些好事,广种福田。若是嫂子有甚愿心,正宜及早了当,管情交哥儿无灾无害好养。”说话间,只见玳安来回话道:“王姑子不在庵里,到王尚书府中去了。小的又到王尚书府中找寻他,半日才得出来。与他说了,便来了。”西门庆听罢,依旧和伯爵、常峙节说话儿,一处坐地,书童拿些茶来吃了。伯爵因开言道:“小弟蒙哥哥厚爱,一向因寒家房子窄隘,不敢简亵,多有疏失。今日禀明了哥,若明后日得空,望哥同常二哥出门外花园里顽耍一日,少尽兄弟孝顺之心。”常峙节从旁赞道:“应二哥一片献芹之心,哥自然鉴纳,决没有见却的理。”西门庆道:“若论明日,到没事,只不该生受。”伯爵道:“小弟在宅里,筷子也不知吃了多少下去,今日一杯水酒,当的甚么。”西门庆道:“既如此,我便不往别处去了。”伯爵道:“只是还有一件——小优儿,小弟便叫了。但郊外去,必须得两个唱的去,方有兴趣。”西门庆道:“这不打紧,我叫人去叫了吴银儿与韩金钏儿就是了。”伯爵道:“如此可知好哩。只是又要哥费心,不当。”西门庆一面就叫琴童,吩咐去叫吴银儿、韩金钏儿,明日早往门外花园内唱。琴童应诺去了。

不多时,王姑子来到厅上,见西门庆道个问讯:“动问施主,今日见召,不知有何吩咐?老身因王尚书府中有些小事去了,不得便来,方才得脱身。”西门庆道:“因前日养官哥许下些愿心,一向忙碌碌,未曾完得。托赖皇天保护,日渐长大。我第一来要酬报佛恩,第二来要消灾延寿,因此请师父来商议。”王姑子道:“小哥儿万金之躯,全凭佛力保护。老爹不知道,我们佛经上说,人中生有夜叉罗刹,常喜啖人,令人无子,伤胎夺命,皆是诸恶鬼所为。如今小哥儿要做好事,定是看经念佛,其余都不是路了。”西门庆便问做甚功德好,王姑子道:“先拜卷《药师经》,待回向后,再印造两部《陀罗经》,极有功德。”西门庆问道:“不知几时起经?”王姑子道:“明日到是好日,就我庵中完愿罢。”西门庆点着头道:“依你,依你。”

王姑子说毕,就往后边,见吴月娘和六房姊妹都在李瓶儿房里。王姑子各打了问讯。月娘便道:“今日央你做好事保护官哥,你几时起经头?”王姑子道:“来日黄道吉日,就我庵里起经。”小玉拿茶来吃了。李瓶儿因对王姑子道:“师父,我还有句话,一发央及你。”王姑子道:“你老人家有甚话,但说不妨。”李瓶儿道:“自从有了孩子,身子便有些不好。明日疏意里边,带通一句何如?行的去,我另谢你。”王姑子道:“这也何难。且待写疏的时节,一发写上就是了。”正是:

\[
祸因恶积非无种,福自天来定有根。
\]

\newpage
%# -*- coding:utf-8 -*-
%%%%%%%%%%%%%%%%%%%%%%%%%%%%%%%%%%%%%%%%%%%%%%%%%%%%%%%%%%%%%%%%%%%%%%%%%%%%%%%%%%%%%


\chapter{应伯爵隔花戏金钏\KG 任医官垂帐诊瓶儿}


词曰:

\[
美酒斗十千,更对花前。芳樽肯放手中闲?起舞酬花花不语,似解人怜。不醉莫言还,请看枝间。已飘零一片减婵娟。花落明年犹自好,可惜朱颜。
\]

却说王姑子和李瓶儿、吴月娘,商量来日起经头停当,月娘便拿了些应用物件送王姑子去,又教陈敬济来吩咐道:“明日你李家丈母拜经保佑官哥,你早去礼拜礼拜。”敬济推道:“爹明日要去门外花园吃酒,留我店里照管,着别人去罢。”原来敬济听见应伯爵请下了西门庆,便想要乘机和潘金莲弄松,因此推故。月娘见说照顾生意,便不违拗他,放他出去了,便着书童礼拜。调拨已定,单待明日起经。

且说西门庆和应伯爵、常峙节谈笑多时,只见琴童来回话道:“唱的叫了。吴银儿有病去不的,韩金钏儿答应了,明日早去。”西门庆道:“吴银儿既病,再去叫董娇儿罢。”常峙节道:“郊外饮酒,有一个尽够了,不消又去叫。”说毕,各各别去,不在话下。

次日黎明,西门庆起身梳洗毕,月娘安排早饭吃了,便乘轿往观音庵起经。书童、玳安跟随而行。王姑子出大门迎接,西门庆进庵来,北面皈依参拜。但见:

\[
金仙建化,启第一之真乘;玉偈演音,集三千之妙利。宝花座上,装成庄严世界;惠日光中,现出欢喜慈悲。香烟缭绕,直透九霄;仙鹤盘旋,飞来秪树。访问缘由,果然稀罕;但思福果,那惜金钱!正是:办个至诚心,何处皇天难感;愿将大佛事,保祈殇子彭篯。
\]

王姑子宣读疏头,西门庆听了,平身更衣。王姑子捧出茶来,又拿些点心饼馓之物摆在桌上。西门庆不吃,单呷了口清茶,便上轿回来,留书童礼拜。正是:

\[
愿心酬毕喜匆匆,感谢灵神保佑功。
更愿皈依莲座下,却教关煞永亨通。
\]

回来,红日才半竿,应伯爵早同常峙节来请。西门庆笑道:“那里有请吃早饭的?我今日虽无事故,也索下午才好去。”应伯爵道:“原来哥不知,出城二十里,有个内相花园,极是华丽,且又幽深,两三日也游玩不到哩。因此要早去,尽这一日工夫,可不是好。”常峙节道:“今日哥既没甚事故,应哥早邀,便索去休。”西门庆道:“既如此;常二哥和应二哥先行,我乘轿便到了。”应伯爵道:“专待哥来。”说罢,两人出门,叫头口前去,又转到院内,立等了韩金钏儿坐轿子同去。应伯爵先一日已着火家来园内,杀鸡宰鹅,安排筵席,又叫下两个优童随着去了。

西门庆见三人去了多时,便乘轿出门,迤逦渐近。举头一看,但见:

\[
千树浓阴,一湾流水。粉墙藏不谢之花,华屋掩长春之景。武陵桃放,渔人何处识迷津?庾岭梅开,词客此中寻好句。端的是天上蓬莱,人间阆苑。
\]
西门庆赞叹不已道:“好景致!”下轿步人园来。应伯爵和常峙节出来迎接,园亭内坐的。先是韩金钏儿磕了头,才是两个歌童磕头。吃了茶,伯爵就要递上酒来,西门庆道:“且住,你每先陪我去瞧瞧景致来。”一面立起身来,搀着韩金钏手儿同走。伯爵便引着,慢慢的步出回廊,循朱阑转过垂杨边一曲荼蘼架,踅过太湖石、松凤亭,来到奇字亭。亭后是绕屋梅花三十树,中间探梅阁。阁上名人题咏极多,西门庆备细看了。又过牡丹台,台上数十种奇异牡丹。又过北是竹园,园左有听竹馆、凤来亭,匾额都是名公手迹;右是金鱼池,池上乐水亭,凭朱栏俯看金鱼,却象锦被也似一片浮在水面。西门庆正看得有趣,伯爵催促,又登一个大楼,上写“听月楼”。楼上也有名人题诗对联,也是刊板砂绿嵌的。下了楼,往东一座大山,山中八仙洞,深幽广阔。洞中有石棋盘,壁上铁笛铜箫,似仙家一般。出了洞,登山顶一望,满园都是见的。

西门庆走了半日,常峙节道:“恐怕哥劳倦了,且到园亭上坐坐,再走不迟。”西门庆道:“十分走不过一分,却又走不得了。多亏了那些抬轿的,一日赶百来里多路。”大家笑了,让到园亭里,西门庆坐了上位,常峙节坐东,应伯爵坐西,韩金钏儿在西门庆侧边陪坐。大家送过酒来,西门庆道:“今日多有相扰,怎的生受!”伯爵道:“一杯水酒,哥说那里话!”三人吃够数杯,两个歌童上来。西门庆看那歌童生得——

\[
粉块捏成白面,胭脂点就朱唇。绿糁糁披几寸青丝,香馥馥着满身罗绮。秋波一转,凭他铁石心肠。檀板轻敲,遮莫金声玉振。正是但得倾城与倾国,不论南方与北方。
\]

两个歌童上来,拿着鼓板,合唱了一套时曲《字字锦》“群芳绽锦鲜”。唱的娇喉婉转,端的是绕梁之声,西门庆称赞不已。常峙节道:“怪他是男子,若是妇女,便无价了。”西门庆道:“若是妇女,咱也早叫他坐了,决不要他站着唱。”伯爵道:“哥本是在行人,说的话也在行。”众人都笑起来。三人又吃了数杯,伯爵送上令盆,斟一大钟酒,要西门庆行令。西门庆道:“这便不消了。”伯爵定要行令,西门庆道:“我要一个风花雪月,第一是我,第二是常二哥,第三是主人,第四是钏姐。但说的出来,只吃这一杯。若说不出,罚一杯,还要讲十个笑话。讲得好便休;不好,从头再讲。如今先是我了。”拿起令钟,一饮而尽,就道:“云淡风轻近午天。——如今该常二哥了。”常峙节接过酒来吃了,便道:“傍花随柳过前川。——如今该主人家了。”应伯爵吃了酒,呆登登讲不出来。西门庆道:“应二哥请受罚。”伯爵道:“且待我思量。”又迟了一回,被西门庆催逼得紧,便道:“泄漏春光有几分。”西门庆大笑道:“好个说别字的,论起来,讲不出该一杯,说别字又该一杯,共两杯。”伯爵笑道:“我不信,有两个‘雪’字,便受罚了两杯?”众人都笑了,催他讲笑话。伯爵说道:“一秀才上京,泊船在扬子江。到晚,叫艄公:‘泊别处罢,这里有贼。’艄公道:‘怎的便见得有贼?’秀才道:‘兀那碑上写的不是江心贼?’艄公笑道:‘莫不是江心赋,怎便识差了?’秀才道:‘赋便赋,有些贼形。’”西门庆笑道:“难道秀才也识别字?”常峙节道:“应二哥该罚十大杯。”伯爵失惊道:“却怎的便罚十杯?”常峙节道:“你且自家去想。”原来西门庆是山东第一个财主,却被伯爵说了“贼形”,可不骂他了!西门庆先没理会,到被常峙节这句话提醒了。伯爵觉失言,取酒罚了两杯,便求方便。西门庆笑道:“你若不该,一杯也不强你;若该罚时,却饶你不的。”伯爵满面不安。又吃了数杯,瞅着常峙节道:“多嘴!”西门庆道:“再说来!”伯爵道:“如今不敢说了。”西门庆道:“胡乱取笑,顾不的许多,且说来看。”伯爵才安心,又说:“孔夫子西狩得麟,不能够见,在家里日夜啼哭。弟子恐怕哭坏了,寻个牯牛,满身挂了铜钱哄他。那孔子一见便识破,道:‘这分明是有钱的牛,却怎的做得麟!’”说罢,慌忙掩着口跪下道:“小人该死了,实是无心。”西门庆笑着道:“怪狗才,还不起来。”金钏儿在旁笑道:“应花子成年说嘴麻犯人,今日一般也说错了。大爹,别要理他。”说的伯爵急了,走起来把金钏儿头上打了一下,说道:“紧自常二那天杀的韶叨,还禁的你这小淫妇儿来插嘴插舌!”不想这一下打重了,把金钏疼的要不的,又不敢哭,肐\textYueChou 着脸,待要使性儿。西门庆笑骂道:“你这狗才,可成个人?嘲戏了我,反又打人,该得何罪?”伯爵一面笑着,搂了金钏说道:“我的儿,谁养的你恁娇?轻轻荡得一荡儿就待哭,亏你挨那驴大的行货子来!”金钏儿揉着头,瞅了他一眼,骂道:“怪花子,你见来?没的扯淡!敢是你家妈妈子倒挨驴的行货来。”伯爵笑说道:“我怎不见?只大爹他是有名的潘驴邓小闲,不少一件,你怎的赖得过?”又道:“哥,我还有个笑话儿,一发奉承了列位罢:一个小娘,因那话宽了,有人教道他:‘你把生矾一块,塞在里边,敢就紧了。’那小娘真个依了他。不想那矾涩得疼了,不好过,肐\textYueChou 着立在门前。一个走过的人看见了,说道:‘这小淫妇儿,倒象妆霸王哩!’这小娘正没好气,听见了,便骂道:‘怪囚根子,俺樊哙妆不过,谁这里妆霸王哩!’”说毕,一座大笑,连金钏儿也噗嗤的笑了。

少顷,伯爵饮过酒,便送酒与西门庆完令。西门庆道:“该钏姐了。”金钏儿不肯。常峙节道:“自然还是哥。”西门庆取酒饮了,道:“月殿云梯拜洞仙。”令完,西门庆便起身更衣散步。伯爵一面叫摆上添换来,转眼却不见了韩金钏儿。伯爵四下看时,只见他走到山子那边蔷薇架儿底下,正打沙窝儿溺尿。伯爵看见了,连忙折了一枝花枝儿,轻轻走去,蹲在他后面,伸手去挑弄他的花心。韩金钏儿吃了一惊,尿也不曾溺完就立起身来,连裤腰都湿了。不防常峙节从背后又影来,猛力把伯爵一推,扑的向前倒了一交,险些儿不曾溅了一脸子的尿。伯爵爬起来,笑骂着赶了打,西门庆立在那边松阴下看了,笑的要不的。连韩金钏儿也笑的打跌道:“应花子,可见天理近哩!”于是重新入席饮酒。西门庆道:“你这狗才,刚才把俺们都嘲了,如今也要你说个自己的本色。”伯爵连说:“有有有,一财主撒屁,帮闲道:‘不臭。’财主慌的道:‘屁不臭,不好了,快请医人!’帮闲道:‘待我闻闻滋味看。’假意儿把鼻一嗅,口一咂,道:‘回味略有些臭,还不妨。’”说的众人都笑了。常峙节道:“你自得罪哥哥,怎的把我的本色也说出来?”众人又笑了一场。伯爵又要常峙节与西门庆猜枚饮酒。韩金钏儿又弹唱着奉酒。众人欢笑,不在话下。

且说陈敬济探听西门庆出门,便百般打扮的俊俏,一心要和潘金莲弄鬼,又不敢造次,只在雪洞里张看,还想妇人到后园来。等了半日不见来,耐心不过,就一直迳奔到金莲房里来,喜得没有人看见。走到房门首,忽听得金莲娇声低唱了一句道:“莫不你才得些儿便将人忘记。”已知妇人动情,便接口道:“我那敢忘记了你!”抢进来,紧紧抱住道:“亲亲,昨日丈母叫我去观音庵礼拜,我一心放你不下,推事故不去。今日爹去吃酒了,我绝早就在雪洞里张望。望得眼穿,并不见我亲亲的俊影儿。因此,拚着死踅得进来。”金莲道:“硶说嘴的,你且禁声。墙有风,壁有耳,这里说话不当稳便。”说未毕,窗缝里隐隐望见小玉手拿一幅白绢,渐渐走近屋里来,又忽地转去了。金莲忖道:“这怪小丫头,要进房却又跑转去,定是忘记甚东西。”知道他要再来,慌教陈敬济:“你索去休,这事不济了。”敬济没奈何,一溜烟出去了。果然,小玉因月娘教金莲描画副裙拖送人,没曾拿得花样,因此又跑转去。这也是金莲造化,不该出丑。待的小玉拿了花样进门,敬济已跑去久了。金莲接着绢儿,尚兀是手颤哩。

话分两头。再表西门庆和应伯爵、常峙节,三人吃的酩酊,方才起身。伯爵再四留不住,忙跪着告道:“莫不哥还怪我那句话么?可知道留不住哩。”西门庆笑道:“怪狗才,谁记着你话来!”伯爵便取个大瓯儿,满满斟了一瓯递上来,西门庆接过吃了。常峙节又把些细果供上来,西门庆也吃了,便谢伯爵起身。与了金钏儿一两银子,叫玳安又赏了歌童三钱银子,吩咐:“我有酒,也着人叫你。”说毕,上轿便行,两个小厮跟随。伯爵叫人家收过家活,打发了歌童,骑头口同金钏儿轿子进城来,不题。

西门庆到家,已是黄昏时分,就进李瓶儿房里歇了。次日,李瓶儿和西门庆说:“自从养了孩子,身上只是不净。早晨看镜子,兀那脸皮通黄了,饮食也不想,走动却似闪肭了腿的一般。倘或有些山高水低,丢了孩子教谁看管?”西门庆见他掉下泪来,便道:“我去请任医官来,看你脉息,吃些丸药,管就好了。”便叫书童写个帖儿,去请任医官来。书童依命去了。

西门庆自来厅上,只见应伯爵早来谢劳。西门庆谢了相扰,两人一处坐地说话。不多时,书童通报任医官到,西门庆慌忙出迎,和应伯爵厮见,三人依次而坐。书童递上茶来吃了,任医官便动问:“府上是那一位贵恙?”西门庆道:“就是第六个小妾,身子有些不好,劳老先生仔细一看。”任医官道:“莫不就是前日得哥儿的么?”西门庆道:“正是。不知怎么生起病来。”任医官道:“且待学生进去看看。”说毕,西门庆陪任医官进到李瓶儿屋里,就床前坐下。叫丫头把帐儿轻轻揭开一缝,先放出李瓶儿的右手来,用帕儿包着,搁在书上。任医官道:“且待脉息定着。”定了一回,然后把三个指头按在脉上,自家低着头,细玩脉息,多时才放下。李瓶儿在帐缝里慢慢的缩了进去。不一时,又把帕儿包着左手,捧将出来,搁在书上,任医官也如此看了。看完了,便向西门庆道:“老夫人两手脉都看了,却斗胆要瞧瞧气色。”西门道:“通家朋友,但看何妨。”就教揭起帐儿。任医官一看,只见:脸上桃花红绽色,眉尖柳叶翠含颦。那任医官略看了两眼,便对西门庆说:“夫人尊颜,学生已是望见了。大约没有甚事,还要问个病源,才是个望、闻、问、切。”西门庆就唤奶子。只见如意儿打扮的花花哨哨走过来,向任医官道个万福,把李瓶儿那口燥唇干、睡炕不稳的病症,细细说了一遍。那任医官即便起身,打个恭儿道:“老先生,若是这等,学生保的没事。大凡以下人家,他形神粗卤,气血强旺,可以随分下药,就差了些,也不打紧的。如宅上这样大家,夫人这样柔弱的形躯,怎容得一毫儿差池!正是药差指下,延祸四肢。以此望、闻、问、切,一件儿少不得的。前日,王吏部的夫人也有些病症,看来却与夫人相似。学生诊了脉,问了病源,看了气色,心下就明白得紧。到家查了古方,参以己见,把那热者凉之,虚者补之,停停当当,不消三四剂药儿,登时好了。那吏部公也感小弟得紧,不论尺头银两,加礼送来。那夫人又有梯己谢意,吏部公又送学生一个匾儿,鼓乐喧天,送到家下。匾上写着‘儒医神术’四个大字。近日,也有几个朋友来看,说道写的是甚么颜体,一个个飞得起的。况学生幼年曾读几行书,因为家事消乏,就去学那岐黄之术。真正那‘儒医’两字,一发道的着哩!”西门庆道:“既然不妨,极是好了。不满老先生说,家中虽有几房,只是这个房下,极与学生契合。学生偌大年纪,近日得了小儿,全靠他扶养,怎生差池的!全仗老先生神术,与学生用心儿调治他速好,学生恩有重报。纵是咱们武职比不的那吏部公,须索也不敢怠慢。”任医官道:“老先生这样相处,小弟一分也不敢望谢。就是那药本,也不敢领。”西门庆听罢,笑将起来道:“学生也不是吃白药的。近日有个笑话儿讲得好:有一人说道:‘人家猫儿若是犯了癞的病,把乌药买来,喂他吃了就好了。’旁边有一人问:‘若是狗儿有病,还吃甚么药?’那人应声道:‘吃白药,吃白药。’可知道白药是狗吃的哩!”那任医官拍手大笑道:“竟不知那写白方儿的是什么?”又大笑一回。任医官道:“老先生既然这等说,学生也止求一个匾儿罢。谢仪断然不敢,不敢。”又笑了一回,起身,大家打恭到厅上去了。正是:

\[
神方得自蓬莱监,脉诀传从少室君。
凡为采芝骑白鹤,时缘度世访豪门。
\]

\newpage
%# -*- coding:utf-8 -*-
%%%%%%%%%%%%%%%%%%%%%%%%%%%%%%%%%%%%%%%%%%%%%%%%%%%%%%%%%%%%%%%%%%%%%%%%%%%%%%%%%%%%%


\chapter{西门庆两番庆寿旦\KG 苗员外一诺送歌童}


词曰:

\[
师表方眷遇,鱼水君臣,须信从来少。宝运当千,佳辰余五,嵩岳诞生元老。帝遣阜安宗社,人仰雍容廊庙。愿岁岁共祝眉寿,寿比山高。
\]

却说任医官看了脉息,依旧到厅上坐下。西门庆便开言道:“不知这病症端的何如?”任医官道:“夫人这病,原是产后不慎调理,因此得来。目下恶路不净,面带黄色,饮食也没些要紧,走动便觉烦劳。依学生愚见,还该谨慎保重。如今夫人两手脉息虚而不实,按之散大。这病症都只为火炎肝腑,土虚木旺,虚血妄行。若今番不治,后边一发了不的。”说毕,西门庆道:“如今该用甚药才好?”任医官道:“只用些清火止血的药——黄柏、知母为君,其余再加减些,吃下看住,就好了。”西门庆听了,就叫书童封了一两银子,送任医官做药本,任医官作谢去了。不一时,送将药来,李瓶儿屋里煎服,不在话下。

且说西门庆送了任医官去,回来与应伯爵说话。伯爵因说:“今日早晨,李三、黄四走来,说他这宗香银子急的紧,再三央我来求哥。好歹哥看我面,接济他这一步儿罢。”西门庆道:“既是这般急,我也只得依你了。你叫他明日来兑了去罢。”一面让伯爵到小卷棚内,留他吃饭。伯爵因问:“李桂儿还在这里住着哩?东京去的也该来了。”西门庆道:“正是,我紧等着还要打发他往扬州去,敢怕也只在早晚到也。”说毕,吃了饭,伯爵别去。到次日,西门庆衙门中回来,伯爵早已同李智、黄四坐在厅上等。见西门庆回来,都慌忙过来见了。西门庆进去换了衣服,就问月娘取出徐家讨的二百五十两银子,又添兑了二百五十两,叫陈敬济拿了,同到厅上,兑与李三、黄四。因说道:“我没银子,因应二哥再三来说,只得凑与你。——我却是就要的。”李三道:“蒙老爹接济,怎敢迟延!如今关出这批银子,一分也不敢动,就都送了来,”于是兑收明,千恩万谢去了。伯爵也就要去,被西门庆留下。

正坐的说话,只见平安儿进来报说:“来保东京回来了。”伯爵道:“我昨日就说也该来了。”不一时,来保进到厅上,与西门庆磕了头。西门庆便问:“你见翟爹么?李桂姐事情怎样了?”来保道:“小的亲见翟爹。翟爹见了爹的书,随即叫长班拿帖儿与朱太尉去说,小的也跟了去。朱太尉亲吩咐说:‘既是太师府中分上,就该都放了。因是六黄太尉送的,难以回他,如乃未到者,俱免提;已拿到的,且监些时。他内官性儿,有头没尾。等他性儿坦些,也都从轻处就是了。’”伯爵道:“这等说,连齐香儿也免提了?——造化了这小淫妇儿了!”来保道:“就是祝爹他每,也只好打几下罢了。罪,料是没了。”一面取出翟管家书递上。西门庆看了说道:“老孙与祝麻子,做梦也不晓的是我这里人情。”伯爵道:“哥,你也只当积阴骘罢了。”来保又说:“翟爹见小的去,好不欢喜,问爹明日可与老爷去上寿?小的不好回说不去,只得答应:‘敢要来也。’翟爹说:‘来走走也好,我也要与你爹会一会哩。’”西门庆道:“我到也不曾打点自去。既是这等说,只得要去走遭了。”因吩咐来保:“你辛苦了,且到后面吃些酒饭,歇息歇息。迟一两日,还要赶到扬州去哩。”来保应诺去了。西门庆就要进去与李桂姐说知,向伯爵道:“你坐着,我就来。”伯爵也要去寻李三、黄四,乘机说道:“我且去着,再来罢。”一面别去。

西门庆来到月娘房里,李桂姐已知道信了,忙走来与西门庆、月娘磕头,谢道:“难得爹娘费心,救了我这一场大祸。拿甚么补报爹娘!”月娘道:“你既在咱家恁一场,有些事儿,不与你处处,却为着甚么来?”桂姐道:“俺便赖爹娘可怜救了,只造化齐香儿那小淫妇儿,他甚相干?连他都饶了。他家赚钱赚钞,带累俺们受惊怕,俺每倒还只当替他说了个大人情,不该饶他才好!”西门庆笑道:“真造化了这小淫妇儿了。”说了一回,挂姐便要辞了家去,道:“我家妈还不知道这信哩,我家去说声,免得他记挂,再同妈来与爹娘磕头罢。”西门庆道:“也罢,我不留你,你且家去说声着。”月娘道:“桂姐,你吃了饭去。”桂姐道:“娘,我不吃饭了。”一面又拜辞西门庆与月娘众人。临去,西门庆说道:“事便完了,你今后,这王三官儿也少招揽他了。”桂姐道:“爹说的是甚么话,还招揽他哩!再要招揽他,就把身子烂化了。就是前日,也不是我招揽他。”月娘道:“不招揽他就是了,又平白说誓怎的?”一面叫轿子,打发桂姐去了。西门庆因告月娘说要上东京之事。月娘道:“既要去,须要早打点,省得临时促忙促急。”西门庆道:“蟒袍锦绣、金花宝贝,上寿礼物,俱已完备,倒只是我的行李不曾整备。”月娘道:“行李不打紧。”西门庆说毕,就到前边看李瓶儿去了。到次日,坐在卷棚内,叫了陈敬济来,看着写了蔡御史的书,交与来保,又与了他盘缠,叫他明日起早赶往扬州去,不题。

倏忽过了数日,看看与蔡太师寿诞将近,只得择了吉日,吩咐琴童、玳安、书童、画童四个小厮跟随,各各收拾行李。月娘同玉楼、金莲众人,将各色礼物并冠带衣服应用之物,共装了二十余扛。头一日晚夕,妻妾众人摆设酒肴和西门庆送行。吃完酒,就进月娘房里宿歇。次日,把二十扛行李先打发出门,又发了一张通行马牌,仰经过驿递起夫马迎送。各各停当,然后进李瓶儿房里来,看了官哥儿,与李瓶儿说道:“你好好调理。要药,叫人去问任医官讨。我不久便来家看你。”那李瓶儿阁着泪道:“路上小心保重。”直送出厅来,和月娘、玉楼、金莲打伙儿送了出大门。西门庆乘了凉轿,四个小厮骑了头口,望东京进发。迤逦行来,免不得朝登紫陌,夜宿邮亭,一路看了些山明水秀,相遇的无非都是各路文武官员进京庆贺寿诞,生辰扛不计其数。约行了十来日,早到东京。进了万寿城门,那时天色将晚,赶到龙德街牌楼底下,就投翟家屋里去住歇。

那翟管家闻知西门庆到了,忙出来迎接,各叙寒暄。吃了茶,西门庆叫玳安将行李一一交盘进翟家来。翟谦交府干收了,就摆酒和西门庆洗尘。不一时,只见剔犀官桌上,摆上珍羞美味来,只好没有龙肝凤髓罢了,其余般般俱有,便是蔡太师自家受用,也不过如此。当值的拿上酒来,翟谦先滴了天,然后与西门庆把盏。西门庆也回敬了。两人坐下,糖果按酒之物,流水也似递将上来。酒过两巡,西门庆便对翟谦道:“学生此来,单为与老太师庆寿,聊备些微礼孝顺太师,想不见却。只是学生久有一片仰高之心,欲求亲家预先禀过:但得能拜在太师门下做个干生子,便也不枉了人生一世。不知可以启口么?”翟谦道:“这个有何难哉!我们主人虽是朝廷大臣,却也极好奉承。今日见了这般盛礼,不惟拜做干子,定然允从,自然还要升选官爵。”西门庆听说,不胜之喜。饮够多时,西门庆便推不吃酒了。翟管家道:“再请一杯,怎的不吃了?”西门庆道:“明日有正经事,不敢多饮。”再四相劝,只又吃了一杯。

翟管家赏了随从人酒食,就请西门庆到后边书房里安歇。排下暖床绡帐,银钩锦被,香喷喷的。一班小厮扶侍西门庆脱衣上床。独宿——西门庆一生不惯,那一晚好难捱过。巴到天明,正待起身,那翟家门户重重掩着。直挨到巳牌时分,才有个人把钥匙一路开将出来。随后才是小厮拿手巾香汤进书房来。西门庆梳洗完毕,只见翟管家出来和西门庆厮见,坐下。当值的就托出一个朱红盒子来,里边有三十来样美味,一把银壶斟上酒来吃早饭。翟谦道:“请用过早饭,学生先进府去和主翁说知,然后亲家搬礼物进来。”西门庆道:“多劳费心!”酒过数杯,就拿早饭来吃了,收过家活。翟管家道:“且权坐一回,学生进府去便来。”

翟谦去不多时,就忙来家,向西门庆说:“老爷正在书房梳洗,外边满朝文武官员都伺候拜寿,未得厮见哩。学生已对老爷说过了,如今先进去拜贺罢,省的住回人杂。学生先去奉候,亲家就来罢了。”说毕去了。西门庆不胜欢喜。便教跟随人拉同翟家几个伴当,先把那二十扛金银缎匹抬到太师府前,一行人应声去了。西门庆即冠带,乘了轿来。只见乱哄哄,挨肩擦背,都是大小官员来上寿的。西门庆远远望见一个官员,也乘着轿进龙德坊来。西门庆仔细一看,却认的是故人扬州苗员外。不想那苗员外也望见西门庆,两个同下轿作揖,叙说寒温。原来这苗员外也是个财主,他身上也现做着散官之职,向来结交在蔡太师门下,那时也来上寿,恰遇了故人。当下,两个忙匆匆路次话了几句,问了寓处,分手而别。

西门庆来到太师府前,但见:

\[
堂开绿野,阁起凌烟。门前宽绰堪旋马,阀阅嵬峨好竖旗。锦绣丛中,风送到画眉声巧;金银堆里,日映出琪树花香。左右活屏风,一个个夷光红拂;满堂死宝玩,一件件周鼎商彝。室挂明珠十二,黑夜里何用灯油;门迎珠履三千,白日间尽皆名士。九州四海,大小官员,都来庆贺;六部尚书,三边总督,无不低头。正是:除却万年天子贵,只有当朝宰相尊。
\]
西门庆恭身进了大门,翟管家接着,只见中门关着不开,官员都打从角门而入。西门庆便问:“为何今日大事,却不开中门?”翟管家道:“中门曾经官家行幸,因此人不敢走。”西门庆和翟谦进了几重门,门上都是武官把守,一些儿也不混乱。见了翟谦,一个个都欠身问管家:“从何处来?”翟管家答道:“舍亲打山东来拜寿老爷的。”说罢,又走过几座门,转几个弯,无非是画栋雕梁,金张甲第。隐隐听见鼓乐之声,如在天上一般。西门庆又问道:“这里民居隔绝,那里来的鼓乐喧嚷?”翟管家道:“这是老爷教的女乐,一班二十四人,都晓得天魔舞、霓裳舞、观音舞。但凡老爷早膳、中饭、夜宴,都是奏的。如今想是早膳了。”西门庆听言未了,又鼻子里觉得异香馥馥,乐声一发近了。翟管家道:“这里与老爷书房相近了,脚步儿放松些。”

转个回廊,只见一座大厅,如宝殿仙宫。厅前仙鹤、孔雀种种珍禽,又有那琼花、昙花、佛桑花,四时不谢,开的闪闪烁烁,应接不暇。西门庆还未敢闯进,交翟管家先进去了,然后挨挨排排走到堂前。只见堂上虎皮交椅上坐一个大猩红蟒衣的,是太师了。屏风后列有二三十个美女,一个个都是宫样妆束,执巾执扇,捧拥着他。翟管家也站在一边。西门庆朝上拜了四拜,蔡太师也起身,就绒单上回了个礼。——这是初相见了。落后,翟管家走近蔡太师耳边,暗暗说了几句话下来,西门庆理会的是那话了,又朝上拜四拜,蔡太师便不答礼。——这四拜是认干爷,因此受了。西门庆开言便以父子称呼道:“孩儿没恁孝顺爷爷,今日华诞,特备的几件菲仪,聊表千里鹅毛之意。愿老爷寿比南山。”蔡太师道:“这怎的生受!”便请坐下。当值的拿了把椅子上来,西门庆朝上作了个揖道:“告坐了。”就西边坐地吃茶。翟管家慌跑出门来,叫抬礼物的都进来。须臾,二十扛礼物摆列在阶下。揭开了凉箱盖,呈上一个礼目:大红蟒袍一套、官绿龙袍一套、汉锦二十匹、蜀锦二十匹、火浣布二十匹、西洋布二十匹,其余花素尺头共四十匹、狮蛮玉带一围、金镶奇南香带一围、玉杯犀杯各十对、赤金攒花爵杯八只、明珠十颗,又另外黄金二百两,送上蔡太师做贽见礼。蔡太师看了礼目,又瞧见抬上二十来扛,心下十分欢喜,说了声“多谢!”便叫翟管家收进库房去了。一面吩咐摆酒款待。西门庆因见他忙冲冲,就起身辞蔡太师。太师道:“既如此,下午早早来罢。”西门庆又作个揖,起身出来。蔡太师送了几步,便不送了。西门庆依旧和翟管家同出府来。翟管家府内有事,也作别进去。

西门庆竟回到翟家来,脱下冠带,已整下午饭,吃了一顿。回到书房,打了个盹,恰好蔡太师差舍人邀请赴席,西门庆谢了些扇金,着先去了。即便重整冠带,又叫玳安封下许多赏封,做一拜匣盛了,跟随着四个小厮,复乘轿望太师府来。蔡太师那日满朝文武官员来庆贺的,各各请酒。自次日为始,分做三停:第一日是皇亲内相,第二日是尚书显要、衙门官员,第三日是内外大小等职。只有西门庆,一来远客,二来送了许多礼物,蔡太师到十分欢喜,因此就是正日独独请他一个。见西门庆到了,忙走出轩下相迎。西门庆再四谦逊,让:“爷爷先行。”自家屈着背,轻轻跨入槛内,蔡太师道:“远劳驾从,又损隆仪。今日略坐,少表微忱。”西门庆道:“孩儿戴天履地,全赖爷爷洪福,些小敬意,何足挂怀!”两个喁喁笑语,真似父子一般。二十四个美女,一齐奏乐,府干当值的斟上酒来。蔡太师要与西门庆把盏,西门庆力辞不敢,只领的一盏,立饮而尽,随即坐了桌席。西门庆叫书童取过一只黄金桃杯,斟上一杯,满满走到蔡太师席前,双膝跪下道:“愿爷爷千岁!”蔡太师满面欢喜道:“孩儿起来。”接过便饮个完。西门庆才起身,依旧坐下。那时相府华筵,珍奇万状,都不必说。西门庆直饮到黄昏时候,拿赏封赏了诸执役人,才作谢告别道:“爷爷贵冗,孩儿就此叩谢,后日不敢再来求见了。”出了府门,仍到翟家安歇。

次日,要拜苗员外,着玳安跟寻了一日,却在皇城后李太监房中住下。玳安拿着帖子通报了,苗员外来出迎道:“学生正想个知心朋友讲讲,恰好来得凑巧。”就留西门庆筵燕。西门庆推却不过,只得便住了。当下山肴海错不记其数。又有两个歌童,生的眉清目秀,顿开喉音,唱几套曲儿。西门庆指着玳安、琴童向苗员外说道:“这班蠢材,只会吃酒饭,怎地比的那两个!”苗员外笑道:“只怕伏侍不的老先生,若爱时,就送上也何难!”西门庆谦谢不敢夺人之好。饮到更深,别了苗员外,依旧来翟家歇。那几日内相府管事的,各各请酒,留连了八九日。西门庆归心如箭,便叫玳安收拾行李。翟管家苦死留住,只得又吃了一夕酒,重叙姻亲,极其眷恋。次日早起辞别,望山东而行。一路水宿风餐,不在话下。

且说月娘家中,自从西门庆往东京庆寿,姊妹每望眼巴巴,各自在屋里做些针指,通不出来闲耍。只有潘金莲打扮的如花似玉,乔模乔样,在丫鬓伙里,或是猜枚,或是抹牌,说也有,笑也有,狂的通没些成色。嘻嘻哈哈,也不顾人看见,只想着与陈敬济勾搭。每日只在花园雪洞内踅来踅去,指望一时凑巧。敬济也一心想着妇人,不时进来寻撞,撞见无人便调戏,亲嘴咂舌做一处,只恨人多眼多,不能尽情欢会。正是:

\[
虽然未入巫山梦,却得时逢洛水神。
\]

一日,吴月娘、孟玉楼、李瓶儿同一处坐地,只见玳安慌慌跑进门来,见月娘众人磕了头,报道:“爹回来了。”月娘便问:“如今在那里?”玳安道:“小的一路骑头口,拿着马牌先行,因此先到家。爹这时节,也差不上二十里远近了。”月娘道:“你曾吃饭没有?”玳安道:“从早上吃来,却不曾吃中饭。”月娘便吩咐整饭伺候,一面就和六房姊妹同伙儿到厅上迎接。正是:

\[
诗人老去莺莺在,公子归时燕燕忙。
\]

妻妾每在厅上等候多时,西门庆方到门前下轿了,众妻妾一齐相迎进去。西门庆先和月娘厮见毕,然后孟玉楼、李瓶儿、潘金莲依次见了,各叙寒温。落后,书童、琴童、画童也来磕了头,自去厨下吃饭。西门庆把路上辛苦并到翟家住下、感蔡太师厚情请酒并与内相日吃酒事情,备细说了一遍。因问李瓶儿:“孩子这几时好么?你身子吃的任医官药,有些应验么?我虽则往东京,一心只吊不下家里。”李瓶儿道:“孩子也没甚事,我身子吃药后,略觉好些。”月娘一面收好行李及蔡太师送的下程,一面做饭与西门庆吃。到晚又设酒和西门庆接风。西门庆晚夕就在月娘房里歇了。两个是久旱逢甘雨,他乡遇故知。欢爱之情,俱不必说。

次日,陈敬济和大姐也来见了,说了些店里的帐目。应伯爵和常峙节打听的来家,都来探望。西门庆出来相见毕,两个一齐说:“哥一路辛苦。”西门庆便把东京富丽的事情及太师管待情分,备细说了一遍。两人只顾称羡不已。当日,西门庆留二人吃了一日酒。常峙节临起身向西门庆道:“小弟有一事相求,不知哥可照顾么?”说着,只是低了脸,半含半吐。西门庆道:“但说不妨。”常峙节道:“实为住的房子不方便,待要寻间房子安身,却没有银子。因此要求哥周济些儿,日后少不的加些利钱送还哥。”西门庆道:“相处中说甚利钱!只我如今忙忙的,那讨银子?且待韩伙计货船来家,自有个处。”说罢,常峙节、应伯爵作谢去了,不在话下。

且说苗员外自与西门庆相会,在酒席上把两个歌童许下。不想西门庆归心如箭,不曾别的他,竟自归来。苗员外还道西门庆在京,差伴当来翟家问,才晓得西门庆家去了。苗员外自想道:“君子一言,快马一鞭。我既许了他,怎么失信!”于是叫过两个歌童吩咐道:“我前日请山东西门大官人,曾把你两个许下他。我如今就要送你到他家去,你们早收拾行李。”那两个歌童一齐跪告道:“小的每伏侍的员外多年,员外不知费尽多少心力,教的俺每这些南曲,却不留下自家欢乐,怎地到送与别人?”说罢,扑簌簌掉下泪来。那员外也觉惨然不乐,说道:“你也说的是,咱何苦定要送人?只是:‘人而无信,不知其可也。’——那孔圣人说的话怎么违得!如今也由不得你了,待咱修书一封,差人送你去,教他好生看觑你就是了。”两个歌童违拗不过,只得应诺起来。苗员外就叫那门管先生写着一封书信,写那相送歌童之意。又写个礼单儿,把些尺头书帕封了,差家人苗实赍书,护送两个歌童往西门庆家来。两个歌童洒泪辞谢了员外,翻身上马,迤逦同望山东大道而来。有日到了清河县,三人下马访问,一直迳到县牌坊西门庆家府里投下。

却说西门庆自从东京到家,每日忙不迭,送礼的,请酒的,日日三朋四友,以此竟不曾到衙门里去。那日稍闲无事,才到衙门里升堂画卯,把那些解到的人犯,同夏提刑一一审问一番。审问了半日,公事毕,方乘了一乘凉轿,几个牢子喝道,簇拥来家。只见那苗实与两个歌童已是候的久了,就跟着西门庆的轿子,随到前厅,跪下禀说:“小的是扬州苗员外有书拜候老爹。”随将书并礼物呈上。西门庆连忙说道:“请起来。”一面打开副启,细细看了。见是送他歌童,心下喜之不胜,说道:“我与你员外意外相逢,不想就蒙你员外情投意合。酒后一言,就果然相赠,又不惮千里送来。你员外真可谓千金一诺矣。难得,难得!”两个歌童从新走过,又磕了四个头,说道:“员外着小的们伏侍老爹,万求老爹青目!”西门庆道:“你起来,我自然重用。”一面叫摆酒饭,管待苗实并两个歌童;一面整办厚礼——绫罗细软,修书答谢员外;一面就叫两个歌童,在于书房伺候。不想,韩道国老婆王六儿,因见西门庆事忙,要时常通个信儿,没人往来,算计将他兄弟王经——才十五六岁,也生得清秀——送来伏侍西门庆,也是这日进门。西门庆一例收下,也叫在书房中伺候。

西门庆正在厅上分拨,忽伯爵走来。西门庆与他说知苗员外送歌童之事,就叫玳安里面讨出酒菜儿来,留他坐,就叫两个歌童来唱南曲。那两个歌童走近席前,并足而立,手执檀板,唱了一套《新水令》“小园昨夜放江梅”,果然是响遏行云,调成白雪。伯爵听了,欢喜的打跌,赞说道:“哥的大福,偏有这些妙人儿送将来。也难为这苗员外好情。”西门庆道:“我少不得寻重礼答他。”一面又与这歌童起了两个名:一个叫春鸿,一个叫春燕。又叫他唱了几个小词儿,二人吃一回酒,伯爵方才别去。正是:

\[
风花弄影新莺啭,俱是筵前歌舞人。
\]

\newpage
%# -*- coding:utf-8 -*-
%%%%%%%%%%%%%%%%%%%%%%%%%%%%%%%%%%%%%%%%%%%%%%%%%%%%%%%%%%%%%%%%%%%%%%%%%%%%%%%%%%%%%


\chapter{西门庆捐金助朋友\KG 常峙节得钞傲妻儿}


诗曰:

\[
清河豪士天下奇,意气相投山可移。
济人不惜千金诺,狂饮宁辞百夜期。
雕盘绮食会众客,吴歌赵舞香风吹。
堂中亦有三千士,他日酬恩知是谁?
\]

话说西门庆留下两个歌童,随即打发苗家人回书礼物,又赏了些银钱。苗实领书,磕头谢了出门。后来不多些时,春燕死了,止春鸿一人,正是:

\[
千金散尽教歌舞,留与他人乐少年。
\]

却说常峙节自那日求了西门庆的事情,还不得到手,房主又日夜催逼。恰遇西门庆从东京回家,今日也接风,明日也接风,一连过了十来日,只不得个会面。常言道:见面情难尽。一个不见,却告诉谁?每日央了应伯爵,只走到大官人门首问声,说不在,就空回了。回家又被浑家埋怨道:“你也是男子汉大丈夫,房子没间住,吃这般懊恼气。你平日只认的西门大官人,今日求些周济,也做了瓶落水。”说的常峙节有口无言,呆瞪瞪不敢做声。到了明日,早起身寻了应伯爵,来到一个酒店内,便请伯爵吃三杯。伯爵道:“这却不当生受。”常峙节拉了坐下,量酒打上酒来,摆下一盘熏肉、一盘鲜鱼。酒过两巡,常峙节道:“小弟向求哥和西门大官人说的事情,这几日通不能会面,房子又催逼的紧,昨晚被房下聒絮了一夜,耐不的。五更抽身,专求哥趁着大官人还没出门时,慢慢的候他。不知哥意下如何?”应伯爵道:“受人之托,必当终人之事。我今日好歹要大官人助你些就是了。”两个又吃过几杯,应伯爵便推早酒不吃了。常峙节又劝一杯,算还酒钱,一同出门,径奔西门庆家里来。

那时,正是新秋时候,金风荐爽。西门庆连醉了几日,觉精神减了几分。正遇周内相请酒,便推事故不去,自在花园藏春坞,和吴月娘、孟玉楼、潘金莲、李瓶儿五个寻花问柳顽耍,好不快活。常峙节和应伯爵来到厅上,问知大官人在屋里,满心欢喜。坐着等了好半日,却不见出来。只见门外书童和画童两个抬着一只箱子,都是绫绢衣服,气吁吁走进门来,乱嚷道:“等了这半日,还只得一半。”就厅上歇下。应伯爵便问:“你爹在那里?”书童道:“爹在园里顽耍哩。”伯爵道:“劳你说声。”两个依旧抬着进去了。不一时,书童出来道:“爹请应二爹、常二叔少待,便来也。”两人又等了一回,西门庆才走出来。二人作了揖,便请坐的。伯爵道:“连日哥吃酒忙,不得些空,今日却怎的在家里?”西门庆道:“自从那日别后,整日被人家请去饮酒,醉的了不的,通没些精神。今日又有人请酒,我只推有事不去。”伯爵道:“方才那一箱衣服,是那里抬来的?”西门庆道:“目下交了秋,大家都要添些秋衣。方才一箱,是你大嫂子的。还做不完,才勾一半哩。”常峙节伸着舌道:“六房嫂子,就六箱了,好不费事!小户人家,一匹布也难得。哥果是财主哩。”西门庆和应伯爵都笑起来。伯爵道:“这两日,杭州货船怎的还不见到?不知买卖货物何如。这几日,不知李三、黄四的银子,曾在府里头开了些送来与哥么?”西门庆道:“货船不知在那里担搁着,书也没捎封寄来,好生放不下。李三、黄四的,又说在出月才关。”应伯爵挨到身边坐下,乘闲便说:“常二哥那一日在哥席上求的事情,一向哥又没的空,不曾说的。常二哥被房主催逼慌了,每日被嫂子埋怨,二哥只麻作一团,没个理会。如今又是秋凉了,身上皮袄儿又当在典铺里。哥若有好心,常言道:救人须救急时无,省的他嫂子日夜在屋里絮絮叨叨。况且寻的房子住着,也是哥的体面。因此,常二哥央小弟特地来求哥,早些周济他罢。”西门庆道:“我曾许下他来,因为东京去,费的银子多了,本待等韩伙计到家,和他理会。如今又恁的要紧?”伯爵道:“不是常二哥要紧,当不的他嫂子聒絮,只得求哥早些便好。”西门庆踌躇了半晌道:“既这等,也不难。且问你,要多少房子才够住?”伯爵道:“他两口儿,也得一间门面、一间客坐、一间床房、一间厨灶——四间房子,是少不得的。论着价银,也得三四个多银子。哥只早晚凑些,教他成就了这桩事罢。”西门庆道:“今日先把几两碎银与他拿去,买件衣服,办些家活,盘搅过来,待寻下房子,我自兑银与你成交,可好么?”两个一齐谢道:“难得哥好心。”西门庆便叫书童:“去对你大娘说,皮匣内一包碎银取了出来。”书童应诺。不一时,取了一包银子出来,递与西门庆。西门庆对常峙节道:“这一包碎银子,是那日东京太师府赏封剩下的十二两,你拿去好杂用。”打开与常峙节看,都是三五钱一块的零碎纹银。常峙节接过放在衣袖里,就作揖谢了。西门庆道:“我这几日不是要迟你的,你又没曾寻的。只等你寻下,待我有银,一起兑去便了。”常峙节又称谢不迭。三个依旧坐下,伯爵便道:“多少古人轻财好施,到后来子孙高大门闾,把祖宗基业一发增的多了。悭吝的,积下许多金宝,后来子孙不好,连祖宗坟土也不保。可知天道好还哩!”西门庆道:“兀那东西,是好动不喜静的,怎肯埋没在一处!也是天生应人用的,一个人堆积,就有一个人缺少了。因此积下财宝,极有罪的。”

正说着,只见书童托出饭来。三人吃毕,常峙节作谢起身,袖着银子欢喜走到家来。刚刚进门,只见浑家闹吵吵嚷将出来,骂道:“梧桐叶落——满身光棍的行货子!出去一日,把老婆饿在家里,尚兀自千欢万喜到家来,可不害羞哩!房子没的住,受别人许多酸呕气,只教老婆耳朵里受用。”那常二只是不开口,任老婆骂的完了,轻轻把袖里银子摸将出来,放在桌儿上,打开瞧着道:“孔方兄,孔方兄!我瞧你光闪闪、响当当无价之宝,满身通麻了,恨没口水咽你下去。你早些来时,不受这淫妇几场气了。”那妇人明明看见包里十二三两银子一堆,喜的抢近前来,就想要在老公手里夺去。常二道:“你生世要骂汉子,见了银子,就来亲近哩。我明日把银子买些衣服穿,自去别处过活,再不和你鬼混了。”那妇人陪着笑脸道:“我的哥!端的此是那里来的这些银子?”常二也不做声。妇人又问道:“我的哥,难道你便怨了我?我也只是要你成家。今番有了银子,和你商量停当,买房子安身却不好?倒恁地乔张致!我做老婆的,不曾有失花儿,凭你怨我,也是枉了。”常二也不开口。那妇人只顾饶舌,又见常二不揪不采,自家也有几分惭愧,禁不得掉下泪来。常二看了,叹口气道:“妇人家,不耕不织,把老公恁地发作!”那妇人一发掉下泪来。两个人都闭着口,又没个人劝解,闷闷的坐着。常二寻思道:“妇人家也是难做。受了辛苦,埋怨人,也怪他不的。我今日有了银子不采他,人就道我薄情。便大官人知道,也须断我不是。”就对那妇人笑道:“我自耍你,谁怪你来!只你时常聒噪,我只得忍着出门去了,却谁怨你来?我明白和你说:这银子,原是早上耐你不的,特地请了应二哥在酒店里吃了三杯,一同往大官人宅里等候。恰好大官人正在家,没曾去吃酒,亏了应二哥许多婉转,才得这些银子到手。还许我寻下房子,兑银与我成交哩!这十二两,是先教我盘搅过日子的。”那妇人道:“原来正是大官人与你的,如今不要花费开了,寻件衣服过冬,省的耐冷。”常二道:“我正要和你商量,十二两纹银,买几件衣服,办几件家活在家里。等有了新房子,搬进去也好看些。只是感不尽大官人恁好情,后日搬了房子,也索请他坐坐是。”妇人道:“且到那时再作理会。”正是:

\[
惟有感恩并积恨,万年千载不生尘。
\]

常二与妇人说了一回,妇人道:“你吃饭来没有?”常二道:“也是大官人屋里吃来的。你没曾吃饭,就拿银子买了米来。”妇人道:“仔细拴着银子,我等你就来。”常二取栲栳望街上买了米,栲栳上又放着一大块羊肉,拿进门来。妇人迎门接住道:“这块羊肉,又买他做甚?”常二笑道:“刚才说了许多辛苦,不争这一些羊肉,就牛也该宰几个请你。”妇人笑指着常二骂道:“狠心的贼!今日便怀恨在心,看你怎的奈何了我!”常二道:“只怕有一日,叫我一万声:‘亲哥,饶我小淫妇罢!’我也只不饶你哩。试试手段看!”那妇人听说,笑的往井边打水去了。当下妇人做了饭,切了一碗羊肉,摆在桌儿上,便叫:“哥,吃饭。”常二道:“我才吃的饭,不要吃了。你饿的慌,自吃些罢。”那妇人便一个自吃了。收了家活,打发常二去买衣服。常二袖着银子,一直奔到大街上来。看了几家,都不中意。只买了一件青杭绢女袄、一条绿绸裙子、一件月白云绸衫儿、一件红绫袄子、一件白绸裙儿,共五件。自家也对身买了一件鹅黄绫袄子、一件丁香色绸直身,又买几件布草衣服。共用去六两五钱银子。打做一包,背到家中,叫妇人打开看看。妇人看了,便问:“多少银子买的?”常二道:“六两五钱银子。”妇人道:“虽没便宜,却值这些银子。”一面收拾箱笼放好,明日去买家活。当日妇人欢天喜地过了一日,埋怨的话都掉在东洋大海里去了,不在话下。

再表应伯爵和西门庆两个,自打发常峙节出门,依旧在厅上坐的。西门庆因说起:“我虽是个武职,恁的一个门面,京城内外也交结许多官员,近日又拜在太师门下,那些通问的书柬,流水也似往来,我又不得细工夫料理。我一心要寻个先生在屋里,教他替写写,省些力气也好,只没个有才学的人。你看有时,便对我说。”伯爵道:“哥,你若要别样却有,要这个倒难。第一要才学,第二就要人品了。又要好相处,没些说是说非,翻唇弄舌,这就好了。若是平平才学,又做惯捣鬼的,怎用的他!小弟只有一个朋友,他现是本州秀才,应举过几次,只不得中。他胸中才学,果然班马之上,就是人品,也孔孟之流。他和小弟,通家兄弟,极有情分。曾记他十年前,应举两道策,那一科试官极口赞好。不想又有一个赛过他的,便不中了。后来连走了几科,禁不的发白鬓斑。如今虽是飘零书剑,家里也还有一百亩田、三四带房子住着。”西门庆道:“他家几口儿也够用了,却怎的肯来人家坐馆?”应伯爵道:“当先有的田房,都被那些大户人家买去了,如今只剩得双手皮哩。”西门庆道:“原来是卖过的田,算什么数!”伯爵道:“这果是算不的数了。只他一个浑家,年纪只好二十左右,生的十分美貌,又有两个孩子,才三四岁。”西门庆道:“他家有了美貌浑家,那肯出来?”伯爵道:“喜的是两年前,浑家专要偷汉,跟了个人,走上东京去了,两个孩子又出痘死了,如今只存他一口,定然肯出来。”西门庆笑道:“恁他说的他好,都是鬼混。你且说他姓甚么?”伯爵道:“姓水,他才学果然无比,哥若用他时,管情书柬诗词,一件件增上哥的光辉。人看了时,都道西门大官人恁地才学哩!”西门庆道:“你都是吊慌,我却不信。你记的他些书柬儿,念来我听,看好时,我就请他来家,拨间房子住下。只一口儿,也好看承的。”伯爵道:“曾记得他捎书来,要我替他寻个主儿。这一封书,略记的几句,念与哥听:

\[
\cipaim{黄莺儿}书寄应哥前,别来思,不待言。满门儿托赖都康健。舍字在边,傍立着官,有时一定求方便。羡如椽,往来言疏,落笔起云烟。”
\]

西门庆听毕,便大笑将起来,道:“他既要你替他寻个好主子,却怎的不捎书来,到写一只曲儿来?又做的不好。可知道他才学荒疏,人品散荡哩。”伯爵道:“这到不要作准他。只为他与我是三世之交,自小同上学堂。先生曾道:‘应家学生子和水学生子一般的聪明伶俐,后来一定长进。”落后做文字,一样同做,再没些妒忌,极好兄弟。故此不拘形迹,便随意写个曲儿。况且那只曲儿,也倒做的有趣。”西门庆道:“别的罢了,只第五句是甚么说话?”白爵道:“哥不知道,这正是拆白道字,尤人所难。‘舍’字在边,旁立着‘官’字,不是个‘馆’字?——若有馆时,千万要举荐。因此说:‘有时定要求方便。’哥,你看他词里,有一个字儿是闲话么?只这几句,稳稳把心窝里事都写在纸上,可不好哩!”西门庆被伯爵说的他恁地好处,到没的说了。只得对伯爵道:“到不知他人品如何?”伯爵道:”他人品比才学又高。前年,他在一个李侍郎府里坐馆,那李家有几十个丫头,一个个都是美貌俊俏的。又有几个伏侍的小厮,也一个个都标致龙阳的。那水秀才连住了四五年,再不起一些邪念。后来不想被几个坏事的丫头小厮,见他似圣人一般,反去日夜括他。那水秀才又极好慈悲的人,便口软勾搭上了。因此,被主人逐出门来,哄动街坊,人人都说他无行。其实,水秀才原是坐怀不乱的。若哥请他来家,凭你许多丫头、小厮,同眠同宿,你看水秀才乱么?再不乱的。”西门庆笑骂道:“你这狗才,单管说慌吊皮鬼混人。前月敝同僚夏龙溪请的先生倪桂岩,曾说他有个姓温的秀才。且待他来时再处。”正是:

\[
将军不好武,稚子总能文。
\]

\newpage
%# -*- coding:utf-8 -*-
%%%%%%%%%%%%%%%%%%%%%%%%%%%%%%%%%%%%%%%%%%%%%%%%%%%%%%%%%%%%%%%%%%%%%%%%%%%%%%%%%%%%%


\chapter{开缘簿千金喜舍\KG 戏雕栏一笑回嗔}


诗曰:

\[
野寺根石壁,诸龛遍崔巍。前佛不复辨,百身一莓苔。
惟有古殿存,世尊亦尘埃。如闻龙象泣,足令信者哀。
公为领兵徒,咄嗟檀施开。吾知多罗树,却倚莲花台。
诸天必欢喜,鬼物无嫌猜。
\]

话说那山东东平府地方,向来有个永福禅寺,起建自梁武帝普通二年,开山是那万回老祖。怎么叫做万回老祖?因那老祖做孩子的时节,才七八岁,有个哥儿从军边上,音信不通,不知生死。他老娘思想大的孩儿,时常在家啼哭。忽一日,孩子问母亲,说道:“娘,这等清平世界,咱家也尽挨得过,为何时时掉下泪来?娘,你说与咱,咱也好分忧的。”老娘就说:“小孩子,你那里知道。自从你老头儿去世,你大哥儿到边上去做了长官,四五年,信儿也没一个。不知他生死存亡,教我老人家怎生吊的下!”说着,又哭起来。那孩子说:“早是这等,有何难哉!娘,如今哥在那里?咱做弟郎的,早晚间走去抓寻哥儿,讨个信来,回复你老人家,却不是好?”那婆婆一头哭,一头笑起来,说道:“怪呆子,你哥若是一百二百里程途,便可去的,直在那辽东地面,去此一万余里,就是好汉子,也走四五个月才到哩,你孩儿家怎么去的?”那孩子就说:“嗄,若是果在辽东,也终不在个天上,我去寻哥儿就回也。”只见他把靸鞋儿系好了,把直掇儿整一整,望着婆儿拜个揖,一溜烟去了。那婆婆叫之不应,追之不及,愈添愁闷。也有邻舍街坊、婆儿妇女前来解劝,说道:“孩儿小,怎去的远?早晚间自回也。”因此,婆婆收着两眶眼泪,闷闷坐的。看看红日西沉,那婆婆探头探脑向外张望,只见远远黑魆魆影儿里,有一个小的儿来也。那婆婆就说:“靠天靠地,靠日月三光。若的俺小的儿子来了,也不枉了俺修斋吃素的念头。”只见那万回老祖忽地跪到跟前说:“娘,你还未睡哩?咱已到辽东抓寻哥儿,讨的平安家信来也。”婆婆笑道:“孩儿,你不去的正好,免教我老人家挂心。只是不要吊慌哄着老娘。那有一万里路程朝暮往还的?”孩儿道:“娘,你不信么?”一直卸下衣包,取出平安家信,果然是他哥儿手笔。又取出一件汗衫,带回浆洗,也是婆婆亲手缝的,毫厘不差。因此哄动了街坊,叫做“万回”。日后舍俗出家,就叫做“万回长老”。果然道德高妙,神通广大。曾在后赵皇帝石虎跟前,吞下两升铁针,又在梁武皇殿下,在头顶上取出舍利三颗。因此敕建永福禅寺,做万回老祖的香火院,正不知费了多少钱粮。正是:

\[
神僧出世神通大,圣主尊隆圣泽深。
\]

不想岁月如梭,时移事改。那万回老祖归天圆寂,就有些得皮得肉的上人们,一个个多化去了。只有几个惫赖和尚,养老婆,吃烧酒,甚事儿不弄出来!不消几日儿,把袈裟也当了,钟儿、磬儿都典了,殿上椽儿、砖儿、瓦儿换酒吃了。弄的那雨淋风刮,佛像儿倒的,荒荒凉凉,将一片钟鼓道场,忽变作荒烟衰草。三四十年,那一个肯扶衰起废!不想有个道长老,原是西印度国出身,因慕中国清华,打从流沙河、星宿海走了八九个年头,才到中华区处。迤逦来到山东,就卓锡在这个破寺里,面壁九年,不言不语,真个是:

\[
佛法原无文字障,工夫向好定中寻。
\]

忽一日发个念头,说道:“呀,这寺院坍塌的不成模样了,这些蠢狗才攮的秃驴,止会吃酒噇饭,把这古佛道场弄得赤白白地,岂不可惜!到今日,咱不做主,那个做主?咱不出头,那个出头?况山东有个西门大官人,居锦衣之职,他家私巨万,富比王侯,前日饯送蔡御史,曾在咱这里摆设酒席。他见寺宇倾颓,就有个鼎建重新的意思。若得他为主作倡,管情早晚间把咱好事成就也。咱须去走一遭。”当时唤起法子徒孙,打起钟鼓,举集大众,上堂宣扬此意。那长老怎生打扮?但见:

\[
身上禅衣猩血染,双环挂耳是黄金。手中锡杖光如镜,百八明珠耀日明。开觉明路现金绳,提起凡夫梦亦醒。庞眉绀发铜铃眼,道是西天老圣僧。
\]
长老宣扬已毕,就叫行者拿过文房四宝,写了一篇疏文。好长老,真个是古佛菩萨现身。于是辞了大众,着上禅鞋,戴上个斗笠子,一壁厢直奔到西门庆家里来。

且说西门庆辞别了应伯爵,走到吴月娘房内,把应伯爵荐水秀才的事体说了一番,就说道:“咱前日东京去,多得众亲朋与咱把盏,如今少不的也要整酒回答他。今日到空闲,就把这事儿完了罢。”当下就叫了玳安,吩咐买办嗄饭之类。又吩咐小厮,分头去请各位。一面拉着月娘,走到李瓶儿房里来看官哥。李瓶儿笑嘻嘻的接住了,就叫奶子抱出官哥儿来。只见眉目稀疏,就如粉块妆成,笑欣欣,直撺到月娘怀里来。月娘把手接着,抱起道:“我的儿,恁的乖觉,长大来,定是聪明伶俐的。”又向那孩子说:“儿,长大起来,恁地奉养老娘哩!”李瓶儿就说:“娘说那里话。假饶儿子长成,讨的一官半职,也先向上头封赠起,那凤冠霞帔,稳稳儿先到娘哩。”西门庆接口便说:“儿,你长大来还挣个文官。不要学你家老子做个西班出身,——虽有兴头,却没十分尊重。”正说着,不想潘金莲在外边听见,不觉怒从心上起,就骂道:“没廉耻、弄虚脾的臭娼根,偏你会养儿子!也不曾经过三个黄梅、四个夏至,又不曾长成十五六岁,出幼过关,上学堂读书,还是个水泡,与阎罗王合养在这里的,怎见的就做官,就封赠那老夫人?怪贼囚根子,没廉耻的货,怎的就见的要做文官,不要象你!”正在唠唠叨叨,喃喃呐呐,一头骂,一头着恼的时节,只见玳安走将进来,叫声“五娘”,说道:“爹在那里?”潘金莲便骂:“怪尖嘴的贼囚根子,那个晓的你什么爹在那里!怎的到我这屋里来?他自有五花官诰的太奶奶老封婆,八珍五鼎奉养他的在那里,那里问着我讨!”那玳安就晓的不是路了,望六娘房里就走。走到房门前,打个咳嗽,朝着西门庆道:“应二爹在厅上。”西门庆道:“应二爹,才送的他去,又做甚?”玳安道:“爹出去便知。”

西门庆只得撇了月娘、李瓶儿,走到外边。见伯爵,正要问话,只见那募缘的道长老已到西门庆门首了。高声叫:“阿弥陀佛!这是西门老爹门首么?那个掌事的管家与吾传报一声,说道:扶桂子,保兰孙,求福有福,求寿有寿。——东京募缘的长老求见。”原来,西门庆平日原是一个撒漫使钱的汉子,又是新得官哥,心下十分欢喜,也要干些好事,保佑孩儿。小厮们通晓得,并不作难,一壁厢进报西门庆。西门庆就说:“且叫他进来看。”不一时,请那长老进到花厅里面,打了个问讯,说道:“贫僧出身西印度国,行脚到东京汴梁,卓锡在永福禅寺,面壁九年,颇传心印。止为那宇殿倾颓,琳宫倒塌,贫僧想起来,为佛弟子,自应为佛出力,因此上贫僧发了这个念头。前日老檀越饯行各位老爹时,悲怜本寺废坏,也有个良心美腹,要和本寺作主。那时,诸佛菩萨已作证盟。贫僧记的佛经上说得好:如有世间善男子、善女人以金钱喜舍庄严佛像者,主得桂于兰孙,端严美貌,日后早登科甲,荫子封妻之报。故此特叩高门,不拘五百一千,要求老檀那开疏发心,成就善果。”就把锦帕展开,取出那募缘疏簿,双手递上。不想那一席话儿,早已把西门庆的心儿打动了,不觉的欢天喜地接了疏簿,就叫小厮看茶。揭开疏簿,只见写道:

\[
伏以白马驼经开象教,竺腾衍法启宗门。大地众僧,无不皈依佛祖;三千世界,尽皆兰若庄严。看此瓦砾倾颓,成甚名山胜境?若不慈悲喜舍,何称佛子仁人?今有永福禅寺,古佛道场,焚修福地。启建自梁武皇帝,开山是万回祖师。规制恢弘,仿佛那给孤园黄金铺地;雕楼精制,依稀似衹洹舍白玉为阶。高阁摩空,旃檀气直接九霄云表;层基亘地,大雄殿可容千众禅僧。两翼巍峨,尽是琳宫绀宇;廊房洁净,果然精胜洞天。那时钟鼓宣扬,尽道是寰中佛国;只这缁流济楚,却也像尘界人天。那知岁久年深,一瞬时移事换。莽和尚纵酒撒泼,毁坏清规;呆道人懒惰贪眠,不行打扫。渐成寂寞,断绝门徒;以致凄凉,罕稀瞻仰。兼以鸟鼠穿蚀,那堪风雨漂摇。栋宇摧颓,一而二,二而三,支撑靡计;墙垣坍塌,日复日,年复年,振起无人。朱红棂槅,拾来煨酒煨茶;合抱栋梁,拿去换盐换米。风吹罗汉金消尽,雨打弥陀化作尘。吁嗟乎!金碧焜炫,一旦为灌莽荆榛。虽然有成有败,终须否极泰来。幸而有道长老之虔诚,不忍见梵王宫之废败。发大弘愿,遍叩檀那。伏愿咸起慈悲,尽兴恻隐。梁柱椽楹,不拘大小,喜舍到高题姓字;银钱布币,岂论丰赢,投柜入疏簿标名。仰仗着佛祖威灵,福禄寿永永百年千载;倚靠他伽蓝明镜,父子孙个个厚禄高官。瓜瓞绵绵,森挺三槐五桂;门庭奕奕,辉煌金阜钱山。凡所营求,吉祥如意。疏文到日,各破悭心。谨疏。
\]

西门庆看毕,恭恭敬敬放在桌儿上面,对长老说:“实不相瞒,在下虽不成个人家,也有几万产业,忝居武职。不想偌大年纪,未曾生下儿子,有意做些善果。去年第六房贱内生下孩子,咱万事已是足了。偶因饯送俺友,得到上方,因见庙字倾颓,实有个舍财助建的念头。蒙老师下顾,那敢推辞!”拿着兔毫妙笔,正在踌躇之际,应伯爵就说:“哥,你既有这片好心为侄儿发愿,何不一力独成,也是小可的事体。”西门庆拿着笔笑道:“力薄,力薄。”伯爵又道:“极少也助一千。”西门庆又笑道:“力薄,力薄。”那长老就开口说道:“老檀越在上,不是贫僧多口,我们佛家的行径,只要随缘喜舍,终不强人所难,但凭老爹发心便是。此外亲友,更求檀越吹嘘吹嘘。”西门庆说道:“还是老师体量。少也不成,就写上五百两。”搁了兔毫笔,那长老打个问讯谢了。西门庆又说:“我这里内官太监、府县仓巡,一个个都与我相好的,我明日就拿疏簿去要他们写。写的来,就不拘三百二百、一百五十,管情与老师成就这件好事。”当日留了长老素斋,相送出门。正是:

\[
慈悲作善豪家事,保福消灾父母心。
\]

西门庆送了长老,转到厅上,与应伯爵坐地,道:“我正要差人请你,你来的正好。我前日往东京,多谢众亲友们与咱把盏,今日安排小酒与众人回答,要二哥在此相陪,不想遇着这个长老,鬼混了一会儿。”伯爵便说道:“好个长老,想是果然有德行的。他说话中间,连咱也心动起来,做了施主。”西门庆说道:“你又几时做施主来?疏簿又是几时写的?”应伯爵笑道:“哥,你不知道,佛经上第一重的是心施,第二法施,第三才是财施。难道我从旁撺掇的,不当个心施?”西门庆笑道:“二哥,只怕你有口无心哩。”两人拍手大笑,应伯爵就说:“小弟在此等待客来,哥有正事,自与嫂子商议去。”

只见西门庆别了伯爵,转到内院里头,只见那潘金莲唠唠叨叨,没揪没采,不觉的睡魔缠扰,打了几个喷涕,走到房中,倒在象牙床上睡去了。李瓶儿又为孩子啼哭,自与奶子、丫鬟在房中坐地,看官哥。只有吴月娘与孙雪娥两个看着整办嗄饭。西门庆走到面前坐的,就把道长老募缘与自己开疏的事,备细说了一番。又把应伯爵耍笑打觑的话也说了一番。欢天喜地,大家嘻笑了一会。那吴月娘毕竟是个正经的人,不慌不忙说下几句话儿,到是西门庆顶门上针。正是:

\[
妻贤每至鸡鸣警,款语常闻药石言。
\]
月娘说道:“哥,你天大的造化,生下孩儿。你又发起善念。广结良缘,岂不是俺一家儿的福分!只是那善念头怕他不多,那恶念头怕他不尽。哥,你日后那没来回没正经养婆娘、没搭煞贪财好色的事体少干几桩儿,却不攒下些阴功,与那小孩子也好!”西门庆笑道:“你的醋话儿又来了。却不道天地尚有阴阳,男女自然配合。今生偷情的、苟合的,都是前生分定,姻缘簿上注名,今生了还,难道是生剌剌胡搊乱扯歪厮缠做的?咱闻那佛祖西天,也止不过要黄金铺地,阴司十殿,也要些楮镪营求。咱只消尽这家私广为善事,就使强奸了姮娥,和奸了织女,拐了许飞琼,盗了西王母的女儿,也不减我泼天的富贵。”月娘笑道:“狗吃热屎,原道是个香甜的;生血掉在牙儿内,怎生改得!”

正在笑间,只见王姑子同了薛姑子,提了一个盒儿,直闯进来,朝月娘打问讯,又向西门庆拜了拜,说:“老爹,你倒在家里。”月娘一面让坐。看官听说,原来这薛姑子不是从幼出家的,少年间曾嫁丈夫,在广成寺前卖蒸饼儿生理。不料生意浅薄,与寺里的和尚、行童调嘴弄舌,眉来眼去,刮上了四五六个。常有些馒头斋供拿来进奉他,又有那应付钱与他买花,开地狱的布,送与他做裹脚。他丈夫那里晓得!以后,丈夫得病死了,他因佛门情熟,就做了个姑子。专一在士夫人家往来,包揽经忏。又有那些不长进、要偷汉子的妇人,叫他牵引。闻得西门庆家里豪富,侍妾多人,思想拐些用度,因此频频往来。有一只歌儿道得好:

\[
尼姑生来头皮光,拖子和尚夜夜忙。三个光头好象师父师兄并师弟,只是铙钹原何在里床?
\]
薛姑子坐下,就把小盒儿揭开,说道:“咱每没有甚么孝顺,拿得施主人家几个供佛的果子儿,权当献新。”月娘道:“要来竟自来便了,何苦要你费心!”只见潘金莲睡觉,听得外边有人说话,又认是前番光景,便走向前来听看。见李瓶儿在房中弄孩子,因晓得王姑于在此,也要与他商议保佑官哥。因一同走到月娘房中。大家道个万福,各各坐地。西门庆因见李瓶儿来,又把那道长老募缘与自家开疏舍财,替官哥求福的事情,又说一番。不想恼了潘金莲,抽身竟走,喃喃哝哝,竟自去了。那薛姑子听了,就站将起来,合掌叫声:“佛阿!老爹你这等样好心作福,怕不的寿年千岁,五男二女,七子团圆。只是我还有一件说与你老人家——这个因果费不甚多,更自获福无量。咦,老檀越,你若干了这件功德,就是那老瞿昙雪山修道,迦叶尊散发铺地,二祖师投崖饲虎,给孤老满地黄金,也比不得你功德哩!”西门庆笑道:“姑姑且坐下,细说甚么功果,我便依你。”薛姑子就说:“我们佛祖留下一卷《陀罗经》,专一劝人生西方净土。因为那肉眼凡夫不生尊信,故此佛祖演说此经,劝你专心念佛,竟往西方,永永不落轮回。那佛祖说的好,如有人持诵此经,或将此经印刷抄写,转劝一人至千万人持诵,获福无量。况且此经里面又有《护诸童子经》儿,凡有人家生育男女,必要从此发心,方得易长易养,灾去福来。如今这副经板现在,只没人印刷施行。老爹只消破些工料印上几千卷,装钉完成,普施十方。那个功德真是大的紧。”西门庆道:“这也不难,只不知这一卷经要多少纸札,多少装钉,多少印刷,有个细数才好动弹。”薛姑子又道:“老爹,你那里去细细算他,止消先付九两银子,叫经坊里印造几千万卷,装钉完满,以后一搅果算还他就是了。”

正说的热闹,只见陈敬济要与西门庆说话,寻到卷棚底下,刚刚凑巧遇着了潘金莲凭栏独恼。猛抬头儿见了敬济,就是猫儿见了鱼鲜饭一般,不觉把一天愁闷都改做春风和气。两个见没有人来,就执手相偎,剥嘴咂舌头。两个肉麻顽了一回,又恐怕西门庆出来撞见,连算帐的事情也不提了。一双眼又象老鼠儿防猫,左顾右盼,要做事又没个方便,只得一溜烟出去了。

且说西门庆听了薛姑子的话头,不觉又动了一片善心,就叫玳安拿拜匣,取出一封银子,准准三十两,便交付薛姑子与王姑子:“即便同去经坊里,与我印下五千卷经,待完了,我就算帐找他。”正话间,只见书童忙忙来报道:“请的各位客人都到了。”少不的是吴大舅、花大舅、谢希大、常峙节这一班。西门庆忙整衣出外迎接升堂。就叫小厮摆下桌儿,请众人一行儿分班列次,各叙长幼坐的。不一时,大鱼大肉、时新果品,一齐儿捧将出来。只见酒逢知己,形迹都忘。猜枚的、打鼓的、催花的,三拳两谎的,歌的歌,唱的唱,顽不尽少年场光景,说不了醉乡里日月。正是:

\[
秋月春花随处有,赏心乐事此时同。
\]

\newpage
%# -*- coding:utf-8 -*-
%%%%%%%%%%%%%%%%%%%%%%%%%%%%%%%%%%%%%%%%%%%%%%%%%%%%%%%%%%%%%%%%%%%%%%%%%%%%%%%%%%%%%


\chapter{潘金莲打狗伤人\KG 孟玉楼周贫磨镜}


词曰:

\[
愁旋释,还似织;泪暗拭,又偷滴。嗔怒着丫头,强开怀,也只是恨怀千叠。拚则而今已拚了,忘只怎生便忘得!又还倚栏杆,试重听消息。
\]

话说当日西门庆陪亲朋饮酒,吃的酩酊大醉,走入后边孙雪娥房里来。雪娥正顾灶上,看收拾家火,听见西门庆往房里去,慌的两步做一步走。先是郁大姐在他炕上坐的,一面撺掇他往月娘房里和玉箫、小玉一处睡去了。原来孙雪娥也住着一明两暗三间房——一间床房,一间炕房。西门庆也有一年多没进他房中来。听见今日进来,连忙向前替西门庆接衣服,安顿中间椅子上坐的。一面揩抹凉席,收拾铺床,薰香澡牝,走来递茶与西门庆吃了,搀扶上床,脱靴解带,打发安歇。一宿无话。

到次日廿八,乃西门庆正生日。刚烧毕纸,只见韩道国后生胡秀到了门首,下头口。左右禀知西门庆,就叫胡秀到厅上,磕头见了。问他货船在那里,胡秀递上书帐,说道:“韩大叔在杭州置了一万两银子缎绢货物,见今直抵临清钞关,缺少税钞银两,未曾装载进城。”西门庆看了书帐,心内大喜,吩咐棋童看饭与胡秀吃了,教他往乔亲家爹那里见见去。就进来对吴月娘说:“韩伙计货船到了临清,使后生胡秀送书帐上来,如今少不的把对门房子打扫,卸到那里,寻伙计收拾,开铺子发卖。”月娘听了,就说:“你上紧寻着,也不早了。”西门庆道:“如今等应二哥来,我就对他说。”不一时,应伯爵来了。西门庆陪着他在厅上坐,就对他说:“韩伙计杭州货船到了,缺少个伙计发卖。”伯爵就说:“哥,恭喜!今日华诞的日子,货船到,决增十倍之利,喜上加喜。哥若寻卖手,不打紧,我有一相识,却是父交子往的朋友,原是缎子行卖手,连年运拙,闲在家中,今年才四十多岁,眼力看银水是不消说,写算皆精,又会做买卖。此人姓甘,名润,字出身,现在石桥儿巷住,倒是自己房儿。”西门庆道:“若好,你明日叫他见我。”

正说着,只见李铭、吴惠、郑奉三个先来磕头。不一时,杂耍乐工都到了。厢房中打发吃饭。只见答应的节级拿票来回话说:“小的叫唱的,止有郑爱月儿不到。他家鸨子说,收拾了才待来,被王皇亲家人拦往宅里唱去了。小的只叫了齐香儿、董娇儿、洪四儿三个,收拾了便来也。”西门庆听见他不来,便道:“胡说!怎的不来?”便叫过郑奉问:“怎的你妹子我这里叫他不来?果系是被王皇亲家拦了去?”那郑奉跪下便道:“小的另住,不知道。”西门庆道:“他说往王皇亲家唱就罢了?敢量我拿不得来!”便叫玳安儿近前吩咐:“你多带两个排军,就拿我个侍生帖儿,到王皇亲家宅内见你王二老爹,就说我这里请几位客吃酒,郑爱月儿答应下两三日了,好歹放了他来。倘若推辞,连那鸨子都与我锁了,墩在门房儿里。这等可恶!”一面叫郑奉:“你也跟了去。”那郑奉又不敢不去,走出外边来,央及玳安儿说道:“安哥,你进去,我在外边等着罢。一定是王二老爹府里叫,怕不还没去哩。有累安哥,若是没动身,看怎的将就叫他好好的来罢。”玳安道:“若果然往王家去了,等我拿帖儿讨去;若是在家藏着,你进去对他妈说,教他快收拾一答儿来,俺就替他回护两句言语儿,爹就罢了。你每不知道他性格,他从夏老爹宅里定下,你不来,他可知恼了哩。”这郑奉一面先往家中说去,玳安同两个排军、一名节级也随后走来。

且说西门庆打发玳安去了,因向伯爵道:“这个小淫妇儿,这等可恶!在别人家唱,我这里叫他不来。”伯爵道:“小行货子,他晓的甚么?他还不知你的手段哩!”西门庆道:“我倒见他酒席上说话儿伶俐,叫他来唱两日试他,倒这等可恶!”伯爵道:“哥今日拣这四个粉头,都是出类拔萃的尖儿了。”李铭道:“二爹,你还没见爱月儿哩!”伯爵道:“我同你爹在他家吃酒,他还小哩,这几年倒没曾见,不知出落的怎样的了。”李铭道:“这小粉头子,虽故好个身段儿,光是一味妆饰,唱曲也会,怎生赶的上桂姐一半儿。爹这里是那里?叫着敢不来!就是来了,亏了你?还是不知轻重。”正说着,只见胡秀来回话道:“小的到乔爹那边见了来了,伺候老爹示下。”西门庆教陈敬济:“后边讨五十两银子,令书童写一封书,使了印色,差一名节级,明日早起身,一同下去,与你钞关上钱老爹,教他过税之时青目一二。”须臾,陈敬济取了一封银子来交与胡秀,胡秀领了文书并税帖,次日早同起身,不在话下。

忽听喝的道子响,平安来报:“刘公公与薛公公来了。”西门庆忙冠带迎接至大厅,见毕礼数,请至卷棚内,宽去上盖蟒衣,上面设两张交椅坐下。应伯爵在下,与西门庆关席陪坐。薛内相便问:“此位是何人?”西门庆道:“去年老太监会过来,乃是学生故友应二哥。”薛内相道:“却是那快耍笑的应先儿么?”应伯爵欠身道:“老公公还记的,就是在下。”须臾,拿茶上来吃了。只见平安走来禀道:“府里周爷差人拿帖儿来说,今日还有一席,来迟些,叫老爹这里先坐,不须等罢。”西门庆看了帖儿,便说:“我知道了。”薛内相因问:“西门大人,今日谁来迟?”西门庆道:“周南轩那边还有一席,使人来说休要等他,只怕来迟些。”薛内相道:“既来说,咱虚着他席面就是。”

正说话间,王经拿了两个帖儿进来:“两位秀才来了。”西门庆见帖儿上,一个是倪鹏,一个是温必古,就知倪秀才举荐了同窗朋友来了,连忙出来迎接。见都穿着衣巾进来,且不看倪秀才,只见那温必古,年纪不上四旬,生的端庄质朴,落腮胡,仪容谦仰,举止温恭。未知行藏如何,先观动静若是。有几句单道他好:

\[
虽抱不羁之才,惯游非礼之地。功名蹭蹬,豪杰之志已灰;家业凋零,浩然之气先丧。把文章道学,一并送还了孔夫子;将致君泽民的事业及荣身显亲的心念,都撇在东洋大海。和光混俗,惟其利欲是前;随方逐圆,不以廉耻为重。峨其冠,博其带,而眼底旁若无人;阔其论,高其谈,而胸中实无一物。三年叫案,而小考尚难,岂望月桂之高攀;广坐衔杯,遁世无闷,且作岩穴之隐相。
\]
西门庆让至厅上叙礼,每人递书帕二事与西门庆祝寿。交拜毕,分宾主而坐。西门庆道:“久仰温老先生大才,敢问尊号?”温秀才道:“学生贱字日新,号葵轩。”西门庆道:“葵轩老先生。”又问:“贵庠?何经?”温秀才道:“学生不才,府学备数。初学《易经》。一向久仰大名,未敢进拜。昨因我这敝同窗倪桂岩道及老先生盛德,敢来登堂恭谒。”西门庆道:“承老先生先施,学生容日奉拜。只因学生一个武官,粗俗不知文理,往来书柬无人代笔。前者因在敝同僚府上会遇桂岩老先生,甚是称道老先生大才盛德。正欲趋拜请教,不意老先生下降,兼承厚贶,感激不尽。”温秀才道:“学生匪才薄德,谬承过誉。”茶罢,西门庆让至卷棚内,有薛、刘二老太监在座。薛内相道:“请二位老先生宽衣进来。”西门庆一面请宽了青衣,请进里面,各逊让再四,方才一边一位,垂首坐下。

正叙谈间,吴大舅、范千户到了,叙礼坐定。不一时,玳安与同答应的和郑奉都来回话道:“四个唱的都叫来了。”西门庆问:“可是王皇亲那里?”玳安道:“是王皇亲宅内叫,还没起身,小的要拿他鸨子墩锁,他慌了,才上轿,都一答儿来了。”西门庆即出到厅台基上站立。只见四个唱的一齐进来,向西门庆磕下头去。那郑爱月儿穿着紫纱衫儿,白纱挑线裙子。腰肢袅娜,犹如杨柳轻盈;花貌娉婷,好似芙蓉艳丽。正是:

\[
万种风流无处买,千金良夜实难消。
\]
西门庆便向郑爱月儿道:“我叫你,如何不来?这等可恶!敢量我拿不得你来!”那郑爱月儿磕了头起来,一声儿也不言语,笑着同众人一直往后边去了。到后边,与月娘众人都磕了头。看见李桂姐、吴银儿都在跟前,各道了万福,说道:“你二位来的早。”李桂姐道:“我每两日没家去了。”因说:“你四个怎的这咱才来?”董娇儿道:“都是月姐带累的俺们来迟了。收拾下,只顾等着他,白不起身。”郑爱月儿用扇儿遮着脸,只是笑,不做声。月娘便问:“这位大姐是谁家的?”董娇儿道:“娘不知道,他是郑爱香儿的妹子郑爱月儿。才成人,还不上半年光景。”月娘道:“可倒好个身段儿。”说毕,看茶吃了,一面放桌儿,摆茶与众人吃。潘金莲且揭起他裙子,撮弄他的脚看,说道:“你每这里边的样子,只是恁直尖了,不象俺外边的样子趫。俺外边尖底停匀,你里边的后跟子大。”月娘向大妗子道:“偏他恁好胜,问他怎的!”一回又取下他头上金鱼撇杖儿来瞧,因问:“你这样儿是那里打的?”郑爱月儿道:“是俺里边银匠打的。”须臾,摆下茶,月娘便叫:“桂姐、银姐,你陪他四个吃茶。”不一时,六个唱的做一处同吃了茶。李桂姐、吴银儿便向董娇儿四个说:“你每来花园里走走。”董娇儿道:“等我每到后边走走就来。”李桂姐和吴银儿就跟着潘金莲、孟玉楼,出仪门往花园中来。因有人在大卷棚内,就不曾过那边去。只在这边看了回花草,就往李瓶儿房里看官哥儿。官儿心中又有些不自在,睡梦中惊哭,吃不下奶去。李瓶儿在屋里守着不出来。看见李桂姐、吴银儿和孟王楼、潘金莲进来,连忙让坐。桂姐问道:“哥儿睡哩?”李瓶儿道:“他哭了这一日,才睡下了。”玉楼道:“大娘说,请刘婆子来看他看,你怎的不使小厮请去?”李瓶儿道:“今日他爹好日子,明日请他去罢。”

正说话中间,只见四个唱的和西门大姐、小玉走来。大姐道:“原来你每都在这里,却教俺花园内寻你。”玉楼道:“花园内有人,咱们不好去的,瞧了瞧儿就来了。”李桂姐问洪四儿:“你每四个在后边做甚么,这半日才来?”洪四儿道:“俺每在后边四娘房里吃茶来。”潘金莲听了,望着玉楼、李瓶儿笑,问洪四儿:“谁对你说是四娘来?”董娇儿道:“他留俺每在房里吃茶,他每问来:‘还不曾与你老人家磕头,不知娘是几娘?’他便说:‘我是你四娘哩。’”金莲道:“没廉耻的小妇奴才,别人称你便好,谁家自己称是四娘来。这一家大小,谁兴你、谁数你、谁叫你是四娘?汉子在屋里睡了一夜儿,得了些颜色儿,就开起染房来了。若不是大娘房里有他大妗子,他二娘房里有桂姐,你房里有杨姑奶奶,李大姐有银姐在这里,我那屋里有他潘姥姥,且轮不到往你那屋里去哩!”玉楼道:“你还没曾见哩——今日早晨起来,打发他爹往前边去了,在院子里呼张唤李的,便那等花哨起来。”金莲道:“常言道:奴才不可逞,小孩儿不宜哄。”又问小玉:“我听见你爹对你奶奶说,要替他寻丫头。说你爹昨日在他屋里,见他只顾收拾不了,因问他。那小淫妇就趁势儿对你爹说:‘我终日不得个闲收拾屋里,只好晚夕来这屋里睡罢了。’你爹说:‘不打紧,到明日对你娘说,寻一个丫头与你使便了。’——真个有此话?”小玉道:“我不晓的,敢是玉箫听见来?”金莲向桂姐道:“你爹不是俺各房里有人,等闲不往他后边去。莫不俺每背地说他,本等他嘴头子不达时务,惯伤犯人,俺每急切不和他说话。”正说着,绣春拿了茶上来。正吃间,忽听前边鼓乐响动,荆都监众人都到齐了,递酒上座,玳安儿来叫四个唱的,就往前边去了。

那日,乔大户没来。先是杂耍百戏,吹打弹唱。队舞才罢,做了个笑乐院本。割切上来,献头一道汤饭。只见任医官到了,冠带着进来。西门庆迎接至厅上叙礼。任医官令左右,毡包内取出一方寿帕、二星白金来,与西门庆拜寿。说道:“昨日韩明川说,才知老先生华诞。恕学生来迟!”西门庆道:“岂敢动劳车驾,又兼谢盛仪。外日多谢妙药。”彼此拜毕,任医官还要把盏,西门庆辞道:“不消了。”一面脱了大衣,与众人见过,就安在左首第四席,与吴大舅相近而坐。献上汤饭并手下攒盒,任医官谢了,令仆从领下去。四个唱的弹着乐器,在旁唱了一套寿词。西门庆令上席分头递酒。下边乐工呈上揭帖,刘、薛二内相拣了韩湘子度陈半街《升仙会》杂剧。才唱得一折,只见喝道之声渐近。平安进来禀道:“守备府周爷来了。”西门庆慌忙迎接。未曾相见,就先请宽盛服。周守备道:“我来要与四泉把一盏。”薛内相说道:“周大人不消把盏,只见礼儿罢。”于是二人交拜毕,才与众人作揖,左首第三席安下钟箸。下边就是汤饭割切上来,又是马上人两盘点心、两盘熟肉、两瓶酒。周守备谢了,令左右领下去,然后坐下。一面觥筹交错,歌舞吹弹,花攒锦簇饮酒。正是:

\[
舞低杨柳楼头月,歌罢桃花扇底风。
\]

吃至日暮,先是任医官隔门去的早。西门庆送出来,任医官因问:“老夫人贵恙觉好了?”西门庆道:“拙室服了良剂,已觉好些。这两日不知怎的,又有些不自在。明日还望老先生过来看看。”说毕,任医官作辞上马而去。落后又是倪秀才、温秀才起身。西门庆再三款留不住,送出大门,说道:“容日奉拜请教。寒家就在对门收拾一所书院,与老先生居住。连宝眷都搬来,一处方便。学生每月奉上束修,以备菽水之需。”温秀才道:“多承厚爱,感激不尽。”倪秀才道:“此是老先生崇尚斯文之雅意矣。”打发二秀才去了。

西门庆陪客饮酒,吃至更阑方散。四个唱的都归在月娘房内,唱与月娘、大妗子、杨姑娘众人听。西门庆还在前边留下吴大舅、应伯爵,复坐饮酒。看着打发乐工酒饭吃了,先去了。其余席上家火都收了,又吩咐从新后边拿果碟儿上来,教李铭、吴惠、郑奉上来弹唱,拿大杯赏酒与他吃。应伯爵道:“哥今日华诞设席,列位都是喜欢。”李铭道:“今日薛爷和刘爷也费了许多赏赐,落后见桂姐、银姐又出来,每人又递了一包与他。只是薛爷比刘爷年小,快顽些。”不一时,画童儿拿上果碟儿来,应伯爵看见酥油鲍螺,就先拣了一个放在口内,如甘露洒心,入口而化。说道:“倒好吃。”西门庆道:“我的儿,你倒会吃!此是你六娘亲手拣的。”伯爵笑道:“也是我女儿孝顺之心。”说道:“老舅,你也请个儿。”于是拣了一个,放在吴大舅口内。又叫李铭、吴惠、郑奉近前,每人拣了一个赏他。

正饮酒间,伯爵向玳安道:“你去后边,叫那四个小淫妇出来。我便罢了,也叫他唱个儿与老舅听,再迟一回儿,便好去。今日连递酒,他只唱了两套,休要便宜了他。”那玳安不动身,说道:“小的叫了他了,在后边唱与妗子和娘每听哩,便来也。”伯爵道:“贼小油嘴,你几时去来?还哄我。”因叫王经:“你去。”那王经又不动。伯爵道:“我使着你每都不去,等我自去罢。”正说着,只闻一阵香风过,觉有笑声,四个粉头都用汗巾儿答着头出来。伯爵看见道:“我的儿,谁养的你恁乖!搭上头儿,心里要去的情,好自在性儿。不唱个曲儿与俺每听,就指望去?好容易!连轿子钱就是四钱银子,买红梭儿米买一石七八斗,够你家鸨子和你一家大小吃一个月。”董娇儿道:“哥儿,恁便宜衣饭儿,你也入了籍罢了。”洪四儿道:“这咱晚,七八有二更,放了俺每去罢了。”齐香儿道:“俺每明日还要起早,往门外送殡去哩。”伯爵道:“谁家?”齐香儿道:“是房檐底下开门的那家子。”伯爵道:“莫不又是王三官儿家?前日被他连累你那场事,多亏你大爹这里人情,替李桂儿说,连你也饶了。这一遭,雀儿不在那窠儿罢了。”齐香儿笑骂道:“怪老油嘴,汗邪了你,恁胡说。”伯爵道:“你笑话我老?我半边俏!把你这四个小淫妇儿还不够摆布哩。”洪四儿笑道:“哥儿,我看你行头不怎么好,光一味好撇。”伯爵道:“我那儿,到跟前看手段还钱。”又道:“郑家那贼小淫妇儿,吃了糖五老座子儿,白不言语,有些出神的模样,敢记挂着那孤老儿在家里?”董娇儿道:“他刚才听见你说,在这里有些怯床。”伯爵道:“怯床不怯床,拿乐器来,每人唱一套,你每去罢,我也不留你了。”西门庆道:“也罢,你们两个递酒,两个唱一套与他听罢。”齐香儿道:“等我和月姐唱。”当下,郑月儿琵琶,齐香儿弹筝,坐在交床上,歌美韵,放娇声,唱了一套《越调·斗鹌鹑》“夜去明来”。董娇儿递吴大舅酒,洪四儿递应伯爵酒,在席上交杯换盏,倚翠偎红。正是:

\[
舞回明月坠秦楼,歌遏行云迷楚馆。
\]

当下,酒进数巡,歌吟两套,打发四个唱的去了。西门庆还留吴大舅坐,又叫春鸿上来唱了一套南曲,才吩咐棋童备马,拿灯笼送大舅。大舅道:“姐夫不消备马,我同应二哥一路走罢。”西门庆道:“既如此,教棋童打灯笼送到家。”吴大舅与伯爵起身作别。西门庆送至大门首,因和伯爵说:“你明日好歹上心,约会了那甘伙计来见我,批合同。我会了乔亲家,好收拾那边房子卸货。”伯爵道:“哥不消吩咐,我知道。”一面作辞,与吴大舅同行,棋童打着灯笼。吴大舅便问:“刚才姐夫说收拾那里房子?”伯爵道:“韩伙计货船到,他新开个缎子铺,收拾对门房子,叫我替他寻个伙计。”大舅道:“几时开张?咱每亲朋少不的作贺作贺。”须臾,出大街,到了伯爵小胡同口上,吴大舅要棋童:“打灯笼送你应二爹到家。”伯爵不肯,说道:“棋童,你送大舅,我不消灯笼,进巷内就是了。”一面作辞,分路回家。棋童便送大舅去了。

西门庆打发李铭等唱钱去了,回后边月娘房中歇了一夜。到次日,果然伯爵领了甘出身,穿青衣走来拜见,讲说买卖之事。西门庆叫将崔本来会乔大户,那边收拾房子,开张举事。乔大户对崔本说:“将来凡一应大小事,随你亲家爹这边只顾处,不消计较。”当下就和甘伙计批了合同。就立伯爵作保,得利十分为率:西门庆五分,乔大户三分,其余韩道国、甘出身与崔本三分均分。一面修盖土库,装画牌面,待货车到日,堆卸开张。后边又独自收拾一所书院,请将温秀才来作西宾,专修书柬,回答往来士夫。每月三两束修,四时礼物不缺,又拨了画童儿小厮伏侍他。西门庆家中宴客,常请过来陪侍饮酒,俱不必细说。

不觉过了西门庆生辰。第二日早晨,就请了任医官来看李瓶儿,又在对门看着收拾。杨姑娘先家去了,李桂姐、吴银儿还没家去。吴月娘买了三钱银子螃蟹,午间煮了,请大妗子、李桂姐、吴银儿众人围着吃了一回。只见月娘请的刘婆子来看官哥儿,吃了茶,李瓶儿就陪他往前边房里去了。刘婆子说:“哥儿惊了,要住了奶。”又留下几服药。月娘与了他三钱银子,打发去了。孟玉楼、潘金莲和李桂姐、吴银儿、大姐都在花架底下,放小桌儿,铺毡条,同抹骨牌赌酒顽耍。孙雪娥吃众人赢了七八钟酒,不敢久坐,就去了。众人就拿李瓶儿顶缺。金莲又教吴银儿、桂姐唱了一套。当日众姊妹饮酒至晚,月娘装了盒子,相送李桂姐、吴银儿家去了。

潘金莲吃的大醉归房,因见西门庆夜间在李瓶儿房里歇了一夜,早晨又请任医官来看他,恼在心里。知道他孩子不好,进门不想天假其便——黑影中躧了一脚狗屎,到房中叫春梅点灯来看,一双大红缎子鞋,满帮子都展污了。登时柳眉剔竖,星眼圆睁,叫春梅打着灯把角门关了,拿大棍把那狗没高低只顾打,打的怪叫起来。李瓶儿使过迎春来说:“俺娘说,哥儿才吃了老刘的药,睡着了,教五娘这边休打狗罢。”潘金莲坐着,半日不言语。一面把那狗打了一回,开了门放出去,又寻起秋菊的不是来。看着那鞋,左也恼,右也恼,因把秋菊唤至跟前说:“这咱晚,这狗也该打发去了,只顾还放在这屋里做甚么?是你这奴才的野汉子?你不发他出去,教他恁遍地撒屎,把我恁双新鞋儿——连今日才三四日儿——躧了恁一鞋帮子屎。知道我来,你也该点个灯儿出来,你如何恁推聋妆哑装憨儿的?”春梅道:“我头里就对他说,你趁娘不来,早喂他些饭,关到后边院子里去罢。他佯打耳睁的不理我,还拿眼儿瞅着我。”妇人道:“可又来,贼胆大万杀的奴才,我知道你在这屋里成了把头,把这打来不作准。”因叫他到跟前:“瞧,躧的我这鞋上的龌龊!”哄得他低头瞧,提着鞋拽巴,兜脸就是几鞋底子。打的秋菊嘴唇都破了,只顾揾着抹血,忙走开一边。妇人骂道:“好贼奴才,你走了!”教春梅:“与我采过来跪着,取马鞭子来,把他身上衣服与我扯去。好好教我打三十马鞭子便罢,但扭一扭儿,我乱打了不算。”春梅于是扯了他衣裳,妇人教春梅把他手扯住,雨点般鞭子打下来,打的这丫头杀猪也似叫。那边官哥才合上眼儿,又惊醒了。又使了绣春来说:“俺娘上覆五娘,饶了秋菊罢,只怕唬醒了哥哥。”那潘姥姥正\textShouWai 在里间炕上,听见打的秋菊叫,一骨碌子爬起来,在旁边劝解。见金莲不依,落后又见李瓶儿使过绣春来说,又走向前夺他女儿手中鞭子,说道:“姐姐少打他两下儿罢,惹得他那边姐姐说,只怕唬了哥哥。为驴扭棍不打紧,倒没的伤了紫荆树。”金莲紧自心里恼,又听见他娘说了这一句,越发心中撺上把火一般。须臾,紫漒了面皮,把手只一推,险些儿不把潘姥姥推了一交。便道:“怪老货,你与我过一边坐着去!不干你事,来劝甚么?甚么紫荆树、驴扭棍,单管外合里应。”潘姥姥道:“贼作死的短寿命,我怎的外合里应?我来你家讨冷饭吃,教你恁顿摔我?”金莲道:“你明日夹着那老\textMaoBi 走,怕他家拿长锅煮吃了我!”潘姥姥听见女儿这等擦他,走到里边屋里呜呜咽咽哭去了,随着妇人打秋菊。打够二三十马鞭子,然后又盖了十栏杆,打的皮开肉绽,才放出来。又把他脸和腮颊都用尖指甲掐的稀烂。李瓶儿在那边,只是双手握着孩子耳朵,腮边堕泪,敢怒而下敢言。

西门庆在对门房子里,与伯爵、崔本、甘伙计吃了一日酒散了,迳往玉楼房中歇息。到次日,周守备家请吃补生日酒,不在家。李瓶儿见官哥儿吃了刘婆子药不见动静,夜间又着惊唬,一双眼只是往上吊吊的。因那日薛姑子、王姑子家去,走来对月娘说:“我向房中拿出他压被的一对银狮子来,要教薛姑子印造《佛顶心陀罗经》,赶八月十五日岳庙里去舍。”那薛姑子就要拿着走,被孟玉楼在旁说道:“师父你且住,大娘,你还使小厮叫将贲四来,替他兑兑多少分两,就同他往经铺里讲定个数儿来,每一部经多少银子,到几时有,才好。你教薛师父去,他独自一个,怎弄的来?”月娘道:“你也说的是。”一面使来安儿叫了贲四来,向月娘众人作了揖,把那一对银狮子上天平兑了,重四十一两五钱。月娘吩咐,同薛师父往经铺印造经数去了。

潘金莲随即叫孟玉楼:“咱送送两位师父去,就前边看看大姐,他在屋里做鞋哩。”两个携着手儿往前边来。贲四同薛姑子、王姑子去了。金莲与玉楼走出大厅东厢房门首,见大姐正在檐下纳鞋,金莲拿起来看,却是沙绿潞绸鞋面。玉楼道:“大姐,你不要这红锁线子,爽利着蓝头线儿,好不老作些!你明日还要大红提跟子?”大姐道:“我有一双是大红提跟子的。这个,我心里要蓝提跟子,所以使大红线锁口。”金莲瞧了一回,三个都在厅台基上坐的。玉楼问大姐:“你女婿在屋里不在?”大姐道:“他不知那里吃了两盅酒,在屋里睡哩。”孟玉楼便向金莲道:“刚才若不是我在旁边说着,李大姐恁哈帐行货,就要把银子交姑子拿了印经去。经也印不成,没脚蟹行货子藏在那大人家,你那里寻他去?早是我说,叫将贲四来,同他去了。”金莲道:“恁有钱的姐姐,不赚他些儿是傻子,只象牛身上拔一根毛儿。你孩儿若没命,休说舍经,随你把万里江山舍了也成不的。如今这屋里,只许人放火,不许俺每点灯。——大姐听着,也不是别人。偏染的白儿不上色,偏他会那等轻狂使势,大清早晨,刁蹬着汉子请太医看。他乱他的,俺每又不管。每常在人前会那等撇清儿说话:‘我心里不耐烦,他爹要便进我屋里推看孩子,雌着和我睡,谁耐烦!教我就撺掇往别人屋里去了。俺每自恁好罢了,背地还嚼说俺们。’那大姐姐偏听他一面词儿。不是俺每争这个事,怎么昨日汉子不进你屋里去,你使丫头在角门子首叫进屋里?推看孩子,你便吃药,一径把汉子作成和吴银儿睡了一夜,一迳显你那乖觉,叫汉子喜欢你,那大姐姐就没的话说了。昨日晚夕,人进屋里躧了一脚狗屎,打丫头赶狗,也嗔起来,使丫头过来说,唬了他孩子了。俺娘那老货,又不知道,走来劝甚么的驴扭棍伤了紫荆树。我恼他那等轻声浪气,叫我墩了他两句,他今日使性子家去了。——去了罢!教我说,他家有你这样穷亲戚也不多,没你也不少。”玉楼笑道:“你这个没训教的子孙,你一个亲娘母儿,你这等讧他!”金莲道:“不是这等说。——恼人的肠子,单管黄猫黑尾,外合里应,只替人说话。吃人家碗半,被人家使唤。得不的人家一个甜头儿,千也说好,万也说好。——想着迎头儿养了这个孩子,把汉子调唆的生根也似的,把他便扶的正正儿的,把人恨不的躧到泥里头还躧。今日恁的天也有眼,你的孩儿也生出病来了。”

正说着,只见贲四往经铺里交回银子,来回月娘话,看见玉楼、金莲和大姐都在厅台基上坐的,只顾在仪门外立着,不敢进来。来安走来说道:“娘每闪闪儿,贲四来了。”金莲道:“怪囚根子,你叫他进去,不是才乍见他来?”来安儿说了,贲四低着头,一直后边见月娘、李瓶儿,说道:“银子四十一两五钱,眼同两个师父交付与翟经儿家收了。讲定印造绫壳《陀罗》五百部,每部五分;绢壳经一千部,每部三分。共该五十五两银子。除收过四十一两五钱,还找与他十三两五钱。准在十四日早抬经来。”李瓶儿连忙向房里取出一个银香球来,叫贲四上天平兑了,十五两。李瓶儿道:“你拿了去,除找与他,别的你收着,换下些钱,到十五日庙上舍经,与你们做盘缠就是了,省的又来问我要。”贲四于是拿了香球出来,李瓶儿道:“四哥,多累你。”贲四躬着身说道:“小人不敢。”走到前边,金莲、玉楼又叫住问他:“银子交付与经铺了?”贲四道:“已交付明白。共一千五百部经,共该五十五两银子,除收过四十一两五钱,刚才六娘又与了这件银香球。”玉楼、金莲瞧了瞧,没言语,贲四便回家去了。玉楼向金莲说道:“李大姐象这等都枉费了钱。他若是你的儿女,就是榔头也桩不死;他若不是你儿女,莫说舍经造像,随你怎的也留不住他。信着姑子,甚么茧儿干不出来!”

两个说了一回,都立起来。金莲道:“咱每往前边大门首走走去。”因问大姐:“你去不去?”大姐道:“我不去。”潘金莲便拉着玉楼手儿,两个同来到大门里首站立。因问平安儿:“对门房子都收拾了?”平安道:“这咱哩?昨日爹看着就都打扫干净了。后边楼上堆货,昨日教阴阳来破土,楼底下还要装厢房三间,土库搁缎子,门面打开,一溜三间,都教漆匠装新油漆,在出月开张。”玉楼又问:“那写书的温秀才,家小搬过来了不曾?”平安道,“从昨日就过来了。今早爹吩咐,把后边那一张凉床拆了与他,又搬了两张桌子、四张椅子与他坐。”金莲道:“你没见他老婆怎的模样儿?”平安道:“黑影子坐着轿子来,谁看见他来!”

正说着,只见远远一个老头儿,斯琅琅摇着惊闺叶过来。潘金莲便道:“磨镜子的过来了。”教平安儿:“你叫住他,与俺每磨磨镜子。我的镜子这两日都使的昏了,吩咐你这囚根子,看着过来再不叫!俺每出来站了多大回,怎的就有磨镜子的过来了?”那平安一面叫住磨镜老儿,放下担儿,金莲便问玉楼道:“你要磨,都教小厮带出来,一答儿里磨了罢。”于是使来安儿:“你去我屋里,问你春梅姐讨我的照脸大镜子、两面小镜子儿,就把那大四方穿衣镜也带出来,教他好生磨磨。”玉楼吩咐来安:“你到我屋里,教兰香也把我的镜子拿出来。”那来安儿去不多时,两只手提着大小八面镜于,怀里又抱着四方穿衣镜出来。金莲道:“臭小囚儿,你拿不了,做两遭儿拿,如何恁拿出来?一时叮当了我这镜子怎了?”玉楼道:“我没见你这面大镜子,是那里的?”金莲道:“是人家当的,我爱他且是亮,安在屋里,早晚照照。”因问:“我的镜子只三面?”玉楼道:“我大小只两面。”金莲道:“这两面是谁的?”来安道:“这两面是春梅姐的,捎出来也叫磨磨。”金莲道:“贼小肉儿,他放着他的镜子不使,成日只挝着我的镜子照,弄的恁昏昏的。”共大小八面镜于,交付与磨镜老叟,教他磨。当下绊在坐架上,使了水银,那消顿饭之间,都净磨的耀眼争光。妇人拿在手内,对照花容,犹如一汪秋水相似。有诗为证:

\[
莲萼菱花共照临,风吹影动碧沉沉。
一池秋水芙蓉现,好似姮娥傍月阴。
\]

妇人看了,就付与来安儿收进去。玉楼便令平安,问铺子里傅伙计柜上要五十文钱与磨镜的。那老子一手接了钱,只顾立着不去。玉楼教平安问那老子:“你怎的不去?敢嫌钱少?”那老子不觉眼中扑簌簌流下泪来,哭了。平安道:“俺当家的奶奶问你怎的烦恼。”老子道:“不瞒哥哥说,老汉今年痴长六十一岁,在前丢下个儿子,二十二岁尚未娶妻,专一浪游,不干生理。老汉日逐出来挣钱养活他。他又不守本分,常与街上捣子耍钱。昨日惹了祸,同拴到守备府中,当土贼打回二十大棍。归来把妈妈的裙袄都去当了。妈妈便气了一场病,打了寒,睡在炕上半个月。老汉说他两句,他便走出来不往家去,教老汉逐日抓寻他,不着个下落。待要赌气不寻他,老汉恁大年纪,止生他一个儿子,往后无人送老;有他在家,见他不成人,又要惹气。似这等,乃老汉的业障。有这等负屈衔冤,各处告诉,所以泪出痛肠。”玉楼叫平安儿:“你问他,你这后娶婆儿今年多大年纪了?”老子道:“他今年五十五岁了,男女花儿没有,如今打了寒才好些,只是没将养的,心中想块腊肉儿吃。老汉在街上恁问了两三日,白讨不出块腊肉儿来。甚可嗟叹人子。”玉楼道:“不打紧处,我屋里抽屉内有块腊肉儿哩。”即令来安儿:“你去对兰香说,还有两个饼锭,教他拿与你来。”金莲叫:“那老头子,问你家妈妈儿吃小米儿粥不吃?”老汉子道:“怎的不吃!那里有?可知好哩。”金莲也叫过来安儿来:“你对春梅说,把昨日你姥姥捎来的新小米儿量二升,就拿两根酱瓜儿出来,与他妈妈儿吃。”那来安去不多时,拿出半腿腊肉、两个饼锭、二升小米、两个酱瓜儿,叫道:“老头子过来,造化了你!你家妈妈子不是害病想吃,只怕害孩子坐月子,想定心汤吃。”那老子连忙双手接了,安放在担内,望着玉楼、金莲唱了个喏,扬长挑着担儿,摇着惊闺叶去了。平安道:“二位娘不该与他这许多东西,被这老油嘴设智诓的去了。他妈妈子是个媒人,昨日打这街上走过去不是,几时在家不好来?”金莲道:“贼囚,你早不说做甚么来?”平安道:“罢了,也是他造化。可可二位娘出来看见叫住他,照顾了他这些东西去了。”正是:

\[
闲来无事倚门楣,恰见惊闺一老来。
不独纤微能济物,无缘滴水也难为。
\]

\newpage
%# -*- coding:utf-8 -*-
%%%%%%%%%%%%%%%%%%%%%%%%%%%%%%%%%%%%%%%%%%%%%%%%%%%%%%%%%%%%%%%%%%%%%%%%%%%%%%%%%%%%%


\chapter{西门庆露阳惊爱月\KG 李瓶儿睹物哭官哥}


诗曰:

\[
枫叶初丹槲叶黄,河阳愁鬓恰新霜。
鬼门徒忆空回首,泉路凭谁说断肠?
路杳云迷愁漠漠,珠沉玉殒事茫茫。
惟有泪珠能结雨,尽倾东海恨无疆。
\]

话说孟玉楼和潘金莲,在门首打发磨镜叟去了。忽见从东一人,带着大帽眼纱骑着骡子,走得甚急,迳到门首下来,慌的两个妇人往后走不迭。落后揭开眼纱却是韩伙计来家了。平安忙问道:“货车到了不曾?”韩道国道:“货车进城了禀问老爹卸在那里?”平安道:“爹不在家,往周爷府里吃酒去了,教卸在对门楼上哩。你老人家请进里边去。”不一时,陈敬济出来,陪韩道国入后边见了月娘出来厅上,拂去尘土,把行李搭裢教王经送到家去。月娘一面打发出饭来与他吃了。不一时,货车才到。敬济拿钥匙开了那边楼上门,就有卸车的小脚子领筹搬运一箱箱都堆卸在楼上。十大车缎货,直卸到掌灯时分。崔本也来帮扶。完毕,查数锁门,贴上封皮,打发小脚钱出门。早有玳安往守备府报西门庆去了。

西门庆听见家中卸货,吃了几杯酒,约掌灯以后就来家。韩伙计等着见了,在厅上坐的,悉把前后往回事说了一遍。西门庆因问:“钱老爹书下了,也见些分上不曾?”韩道国道:“全是钱老爹这封书,十车货少使了许多税钱。小人把段箱,两箱并一箱,三停只报了两停,都当茶叶、马牙香柜上税过来了。通共十大车货,只纳了三十两五钱钞银子。老爹接了报单,也没差巡拦下来查点,就把车喝过来了。”西门庆听言,满心欢喜,因说:“到明日,少不的重重买一分礼谢他。”于是吩咐陈敬济陪韩伙计、崔大哥坐,后边拿菜出来,留吃了一回酒,方才各散回家。

王六儿听见韩道国来了,吩咐丫头春香、锦儿,伺候下好茶好饭。等的晚上,韩道国到家,拜了家堂,脱了衣裳,净了面目,夫妻二人各诉离情一遍。韩道国悉把买卖得意一节告诉老婆,老婆又见搭裢内沉沉重重许多银两,因问他,替己又带了一二百两货物酒米,卸在门外店里,慢慢发卖了银子来家。老婆满心欢喜道:“我听见王经说,又寻了个甘伙计做卖手,咱每和崔大哥与他同分利钱使,这个又好了。到出月开铺了。”韩道国道:“这里使着了人做卖手,南边还少个人立庄置货老爹一定还裁派我去。”老婆道:“你看货才料,自古能者多劳。你不会做买卖那老爹托你么!常言:不将辛苦意,难得世间财。你外边走上三年,你若懒得去等我对老爹说了,教姓甘的和保官儿打外,你便在家卖货就是了。”韩道国道:“外边走熟了,也罢了。”老婆道:“可又来,你先生迷了路,在家也是闲!”说毕,摆上酒来,夫妇二人饮了几杯阔别之酒,收拾就寝。是夜欢娱无度,不必细说。次日却是八月初一日,韩道国早到房子内,同崔本、甘伙计看着收拾装修土库,不在话下。

却说西门庆见货物卸了,家中无事,忽然心中想起要往郑爱月儿家去。暗暗使玳安儿送了三两银子、一套纱衣服与他。郑家鸨子听见西门老爹来请他家姐儿,如天上落下来的一般,连忙收下礼物,没口子向玳安道:“你多顶上老爹,就说他姐儿两个都在家里伺候老爹,请老爹早些儿下降。”玳安走来家中书房内,回了西门庆话。西门庆约午后时分,吩咐玳安收拾着凉轿,头上戴着披巾,身上穿青纬罗暗补子直身,粉底皂靴,先走在房子看了一回装修土库,然后起身,坐上凉轿,放下斑竹帘来,琴童、玳安跟随,留王经在家,止叫春鸿背着直袋,迳往院中郑爱月儿家。正是:

\[
天仙机上整香罗,入手先拖雪一窝。
不独桃源能问渡,却来月窟伴嫦娥。
\]

却说郑爱香儿打扮的粉面油头,见西门庆到,笑吟吟在半门里首迎接进去。到于明间客位,道了万福。西门庆坐下,就吩咐小厮琴童:“把轿回了家去,晚夕骑马来接。”琴童跟轿家去,止留玳安和春鸿两个伺候。少顷,鸨子出来拜见,说道“外日姐儿在宅内多有打搅,老爹来这里,自恁走走罢了,如何又赐将礼来?又多谢与姐儿的衣服。”西门庆道:“我那日叫他,怎的不去?——只认王皇亲家了!”鸨子道:“俺每如今还怪董娇儿和李桂儿。不知是老爹生日叫唱,他每都有了礼,只俺们姐儿没有。若早知时,决不答应王皇亲家唱,先往老爹宅里去了。落后,老爹那里又差了人来,慌的老身背着王家人,连忙撺掇姐儿打后门上轿去了。”西门庆道:“先日我在他夏老爹家酒席上,就定下他了。他若那日不去,我不消说的就恼了。怎的他那日不言不语,不做喜欢,端的是怎么说?”鸨子道:“小行货子家,自从梳弄了,那里好生出去供唱去!到老爹宅内,见人多,不知唬的怎样的。他从小是恁不出语,娇养惯了。你看,甚时候才起来!老身该催促了几遍,说老爹今日来,你早些起来收拾了罢。他不依,还睡到这咱晚。”

不一时,丫鬟拿茶上来,郑爱香儿向前递了茶吃了。鸨子道:“请老爹到后边坐罢。”郑爱香儿就让西门庆进入郑爱月儿的房外明间内坐下,西门庆看见上面楷书“爱月轩”三字。坐了半日,忽听帘栊响处,郑爱月儿出来,不戴\textuni{4BFC}髻,头上挽着一窝丝杭州缵,梳的黑\textuni{29B79}\textuni{29B79}光油油的乌云,云鬓堆鸦犹若轻烟密雾。上着白藕丝对衿仙裳,下穿紫绡翠纹裙,脚下露红鸳凤嘴鞋,前摇宝玉玲珑,越显那芙蓉粉面。正是:

\[
若非道子观音画,定然延寿美人图。
\]

爱月儿走到下面,望上不端不正与西门庆道了万福,就用洒金扇儿掩着粉脸坐在旁边。西门庆注目停视,比初见时节越发齐整,不觉心摇目荡,不能禁止。不一时,丫鬟又拿一道茶来。这粉头轻摇罗袖,微露春纤,取一钟,双手递与西门庆,然后与爱香各取一钟相陪。吃毕,收下盏托去,请宽衣服房里坐。西门庆叫玳安上来,把上盖青纱衣宽了,搭在椅子上。进入粉头房中,但见瑶窗绣幕,锦褥华裀,异香袭人,极其清雅,真所谓神仙洞府,人迹不可到者也。彼此攀话调笑之际,只见丫鬟进来安放桌儿,摆下许多精制菜蔬。先请吃荷花细饼,郑爱月儿亲手拣攒肉丝,卷就,安放小泥金碟儿内,递与西门庆吃。须臾,吃了饼,收了家火去,就铺茜红毡条,取出牙牌三十二扇,与西门庆抹牌。抹了一回,收过去,摆上酒来。但见盘堆异果,酒泛金波,十分齐整。姊妹二人递了酒,在旁筝排雁柱,款跨绞绡——爱香儿弹筝,爱月儿琵琶,唱了一套“兜的上心来”。端的词出佳人口,有裂石绕梁之声。唱毕,促席而坐,拿骰盆儿与西门庆抢红猜枚。

饮够多时,郑爱香儿推更衣出去了,独有爱月儿陪着西门庆吃酒。先是西门庆向袖中取出白绫汗巾儿,上头束着个金穿心盒儿。郑爱月儿只道是香茶,便要打开西门庆道:“不是香茶,是我逐日吃的补药。我的香茶不放在这里面,只用纸包着。”于是袖中取出一包香茶桂花饼儿递与他。那爱月儿不信,还伸手往他袖子里掏,又掏出个紫绉纱汗巾儿,上拴着一副拣金挑牙儿,拿在手中观看,甚是可爱。说道:“我见桂姐和吴银姐都拿着这样汗巾儿,原来是你与他的。”西门庆道:“是我扬州船上带来的。不是我与他,谁与他的?你若爱,与了你罢。到明日,再送一副与你姐姐。”说毕,西门庆就着钟儿里酒,把穿心盒儿内药吃了一服,把粉头搂在怀中,两个一递一口儿饮酒咂舌,无所不至。西门庆又舒手摸弄他香乳,紧紧就就赛麻圆滑腻。一面扯开衫儿观看,白馥馥犹如莹玉一般。揣摩良久,淫心辄起,腰间那话突然而兴。解开裤带,令他纤手笼攥。粉头见其粗大,唬的吐舌害怕,双手搂定西门庆脖项说道:“我的亲亲,你今日初会,将就我,只放半截儿罢!若都放进去,我就死了。你敢吃药养的这等大,不然,如何天生恁怪剌剌儿的——红赤赤,紫漒漒,好砢碜人子!”西门庆笑道:“我的儿!你下去替我品品。”爱月儿道:“慌怎的,往后日子多如树叶儿。今日初会,人生面不熟,再来等我替你品。”说毕,西门庆欲与他交欢,爱月儿道:“你不吃酒了?”西门庆道:“我不吃了,咱睡罢。”爱月儿便叫丫鬟把酒桌抬过一边,与西门庆脱靴,他便往后边更衣澡牝去了。西门庆脱靴时,还赏了丫头一块银子,打发先上床睡,炷了香,放在薰笼内。良久,妇人进房,问西门庆:“你吃茶不吃?”西门庆道:“我不吃。”一面掩上房门,放下绫绡来,将绢儿安放在褥下,解衣上床。两个枕上鸳鸯,被中鸂鶒。西门庆见粉头肌肤纤细,牝净无毛,犹如白面蒸饼一般,柔嫩可爱。抱了抱腰肢,未盈一掬。诚为软玉温香,千金难买。于是把他两只白生生银条般嫩腿儿夹在两边腰眼间,那话上使了托子,向花心里顶入。龟头昂大,濡搅半晌,方才没棱。那爱月儿把眉头绉在一处,两手攀搁在枕上,隐忍难挨。朦胧着星眼,低声说道:“今日你饶了郑月儿罢!”西门庆听了,愈觉销魂,肆行抽送,不胜欢娱。正是:得多少——

\[
春点桃花红绽蕊,风欺杨柳绿翻腰。
\]

西门庆与郑月儿留恋至三更方才回家。到次日,吴月娘打发他往衙门中去了,和玉楼、金莲、李娇儿都在上房坐的。只见玳安进来上房取尺头匣儿,往夏提刑送生日礼去。月娘因问玳安:“你爹昨日坐轿于往谁家吃酒,吃到那咱晚才回家?想必又在韩道国家,望他那老婆去来。原来贼囚根子成日只瞒着我,背地替他干这等茧儿!”玳安道:“不是。他汉子来家,爹怎好去的!”月娘道:“不是那里,却是谁家?”那玳安又不说,只是笑。取了段匣,送礼去了。潘金莲道:“大姐姐,你问这贼囚根子,他怎肯实说?我听见说蛮小厮昨日也跟了去来,只叫蛮小厮来问就是了。”一面把春鸿叫到跟前。金莲问:“你昨日跟了你爹轿子去,在谁家吃酒来?你实说便罢,不实说,如今你大娘就要打你。”那春鸿跪下便道:“娘休打小的,待小的说就是了。小的和玳安、琴童哥三个,跟俺爹从一座大门楼进去,转了几条街巷,到个人家,只半截门儿,都用锯齿儿镶了。门里立着个娘娘,打扮的花花黎黎的。”金莲听见笑了,说道:“囚根子,一个院里半门子也不认的?赶着粉头叫娘娘起来。”又问道:“那个娘娘怎么模样?你认的他不认的?”春鸿道:“我不认的他,也象娘每头上戴着这个假壳。进入里面,一个白头的阿婆出来,望俺爹拜了一拜。落后请到后边,又是一位年小娘娘出来,不戴假壳,生的瓜子面,搽的嘴唇红红的,陪着俺爹吃酒。”金莲道:“你们都在那里坐来?”春鸿道:“我和玳安、琴童哥便在阿婆房里,陪着俺每吃酒并肉兜子来。”把月娘、玉楼笑的了不得。因问道:“你认的他不认的?”春鸿道:“那一个好似在咱家唱的。”玉楼笑道:“就是李桂姐了。”月娘道:“原来摸到他家去来。”李娇儿道:“俺家没半门子。”金莲道:“只怕你家新安了半门子是的。”问了一回。西门庆来家,就往夏提刑家拜寿去了。

却说潘金莲房中养的一只白狮子猫儿,浑身纯白,只额儿上带龟背一道黑,名唤雪里送炭,又名雪狮子。又善会口衔汗巾子,拾扇儿。西门庆不在房中,妇人晚夕常抱他在被窝里睡,又不撒尿屎在衣服上,呼之即至,挥之即去,妇人常唤他是雪贼。每日不吃牛肝干鱼,只吃生肉,调养的十分肥壮,毛内可藏一鸡蛋。甚是爱惜他,终日在房里用红绢裹肉,令猫扑而挝食。这日也是合当有事,官哥儿心中不自在,连日吃刘婆子药,略觉好些。李瓶儿与他穿上红缎衫儿,安顿在外间炕上顽耍,迎春守着,奶子便在旁吃饭。不料这雪狮子正蹲在护炕上,看见官哥儿在炕上,穿着红衫儿一动动的顽耍,只当平日哄喂他肉食一般,猛然望下一跳,将官哥儿身上皆抓破了。只听那官哥儿“呱”的一声,倒咽了一口气,就不言语了,手脚俱风搐起来。慌的奶子丢下饭碗,搂抱在怀,只顾唾哕与他收惊。那猫还来赶着他要挝,被迎春打出外边去了。如意儿实承望孩子搐过一阵好了,谁想只顾常连,一阵不了一阵搐起来。忙使迎春后边请李瓶儿去,说:“哥儿不好了,风搐着哩,娘快去!”那李瓶儿不听便罢,听了,正是:

\[
惊损六叶连肝肺,唬坏三毛七孔心。
\]

连月娘慌的两步做一步,迳扑到房中。见孩子搐的两只眼直往上吊,通不见黑眼睛珠儿,口中白沫流出,咿咿犹如小鸡叫,手足皆动。一见心中犹如刀割相侵,连忙搂抱起来,脸揾着他嘴儿,大哭道:“我的哥哥,我出去好好儿,怎么就搐起来?”迎春与奶子,悉把被五娘房里猫所唬一节说了。那李瓶儿越发哭起来,说道:“我的哥哥,你紧不可公婆意,今日你只当脱不了打这条路儿去了!”月娘听了,一声儿没言语,一面叫将金莲来,问他说:“是你屋里的猫唬了孩子?”金莲问:“是谁说的?”月娘指着:“是奶子和迎春说来。”金莲道:“你看这老婆子这等张嘴!俺猫在屋里好好儿的卧着不是。你每怎的把孩子唬了,没的赖人起来。爪儿只拣软处捏,俺每这屋里是好缠的!”月娘道:“他的猫怎得来这屋里?”迎春道:“每常也来这边屋里走跳。”金莲接过来道:“早时你说,每常怎的不挝他?可可今日儿就挝起来?你这丫头也跟着他恁张眉瞪眼儿,六说白道的。将就些儿罢了,怎的要把弓儿扯满了?可可儿俺每自恁没时运来。”于是使性子抽身往房里去了。看官听说:潘金莲见李瓶儿有了官哥儿,西门庆百依百随,要一奉十,故行此阴谋之事,驯养此猫,必欲唬死其子,使李瓶儿宠衰,教西门庆复亲于己。就如昔日屠岸贾养神獒害赵盾丞相一般。正是:

\[
花枝叶底犹藏刺,人心怎保不怀毒。
\]

月娘众人见孩子只顾搐起来,一面熬姜汤灌他,一面使来安儿快叫刘婆去。不一时,刘婆子来到,看了脉息,只顾跌脚,说道:“此遭惊唬重了,难得过了。快熬灯心薄荷金银汤。”取出一丸金箔丸来,向钟儿内研化。牙关紧闭,月娘连忙拔下金簪儿来,撬开口,灌下去。刘婆道:“过得来便罢。如过不来,告过主家奶奶,必须要灸几醮才好。”月娘道:“谁敢耽?必须等他爹来问了不敢。灸了,惹他来家吆喝。”李瓶儿道:“大娘救他命罢!若等来家,只恐迟了。若是他爹骂,等我承当就是了。”月娘道:“孩儿是你的孩儿,随你灸,我不敢张主,”当下,刘婆子把官哥儿眉攒、脖根、两手关尺并心口,共灸了五醮,放他睡下。那孩子昏昏沉沉,直睡到日暮时分西门庆来家还不醒。那刘婆见西门庆来家,月娘与了他五钱银子,一溜烟从夹道内出去了。

西门庆归到上房,月娘把孩子风搐不好对西门庆说了,西门庆连忙走到前边来看视,见李瓶儿哭的眼红红的,问:“孩儿怎的风搐起来?”李瓶儿满眼落泪,只是不言语。问丫头、奶子,都不敢说。西门庆又见官哥手上皮儿去了,灸的满身火艾,心中焦燥,又走到后边问月娘。月娘隐瞒不住,只得把金莲房中猫惊唬之事说了:“刘婆子刚才看,说是急惊风,若不针灸,难过得来。若等你来,只恐怕迟了。他娘母子自主张,叫他灸了孩儿身上五醮,才放下他睡了。这半日还未醒。”西门庆不听便罢,听了此言,三尸暴跳,五脏气冲,怒从心上起,恶向胆边生,直走到潘金莲房中,不由分说,寻着雪狮子,提着脚走向穿廊,望石台基轮起来只一摔,只听响亮一声,脑浆迸万朵桃花,满口牙零噙碎玉。正是:

\[
不在阳间擒鼠耗,却归阴府作狸仙。
\]

潘金莲见他拿出猫去摔死了,坐在炕上风纹也不动。待西门庆出了门,口里喃喃呐呐骂道:“贼作死的强盗,把人妆出去杀了才是好汉!一个猫儿碍着你噇屎?亡神也似走的来摔死了。他到阴司里,明日还问你要命,你慌怎的?贼不逢好死变心的强盗!”西门庆走到李瓶儿房里,因说奶子、迎春:“我教你好看着孩儿,怎的教猫唬了他,把他手也挝了!又信刘婆子那老淫妇,平白把孩子灸的恁样的。若好便罢,不好,把这老淫妇拿到衙门里,与他两拶!”李瓶儿道:“你看孩儿紧自不得命,你又是恁样的。孝顺是医家,他也巴不得要好哩。”李瓶儿只指望孩儿好来,不料被艾火把风气反于内,变为慢风,内里抽搐的肠肚儿皆动,尿屎皆出,大便屙出五花颜色,眼目忽睁忽闭,终朝只是昏沉不省,奶也不吃了。李瓶儿慌了,到处求神问卜打卦,皆有凶无吉。月娘瞒着西门庆又请刘婆子来家跳神,又请小儿科太医来看。都用接鼻散试之:若吹在鼻孔内打鼻涕,还看得;若无鼻涕出来,则看阴骘守他罢了。于是吹下去,茫然无知,并无一个喷涕出来。越发昼夜守着哭涕不止,连饮食都减了。

看看到八月十五日将近,月娘因他不好,连自家生日都回了不做,亲戚内眷,就送礼来也不请。家中止有吴大妗子、杨姑娘并大师父来相伴。那薛姑子和王姑子两个,在印经处争分钱不平,又使性儿,彼此互相揭调。十四日,贲四同薛姑子催讨,将经卷挑将米,一千五百卷都完了。李瓶儿又与了一吊钱买纸马香烛。十五日同陈敬济早往岳庙里进香纸,把经看着都散施尽了,走来回李瓶儿话。乔大户家,一日一遍使孔嫂儿来看,又举荐了一个看小儿的鲍太医来看,说道:“这个变成天吊客忤,治不得了。”白与了他五钱银子,打发去了。灌下药去也不受,还吐出了。只是把眼合着,口中咬的牙格支支响。李瓶儿通衣不解带,昼夜抱在怀中,眼泪不干的只是哭。西门庆也不往那里去,每日衙门中来家,就进来看孩儿。

那时正值八月下旬天气,李瓶儿守着官哥儿睡在床上,桌上点着银灯,丫鬟养娘都睡熟了。觑着满窗月色,更漏沉沉,果然愁肠万结,离思千端。正是:

\[
人逢喜事精神爽,闷来愁肠瞌睡多。
\]
但见:

\[
银河耿耿,玉漏迢迢。穿窗皓月耿寒光,透户凉风吹夜气。樵楼禁鼓,一更未尽一更敲;别院寒砧,千捣将残千捣起。画檐前叮当铁马,敲碎思妇情怀;银台上闪烁灯光,偏照佳人长叹。一心只想孩儿好,谁料愁来睡梦多。
\]
当下,李瓶儿卧在床上,似睡不睡,梦见花子虚从前门外来,身穿白衣,恰似活时一般。见了李瓶儿,厉声骂道:“泼贼淫妇,你如何抵盗我财物与西门庆?如今我告你去也。”被李瓶儿一手扯住他衣袖,央及道:“好哥哥,你饶恕我则个!”花子虚一顿,撒手惊觉,却是南柯一梦。醒来,手里扯着却是官哥儿的衣衫袖子。连哕了几口道:“怪哉!怪哉!”听一听更鼓,正打三更三点。李瓶儿唬的浑身冷汗,毛发皆竖。

到次日,西门庆进房来,就把梦中之事告诉一遍。西门庆道:“知道他死到那里去了!此是你梦想旧境。只把心来放正着,休要理他。如今我使小厮拿轿子接了吴银儿来,与你做个伴儿。再把老冯叫来伏侍两日。”玳安打院里接了吴银儿来。那消到日西时分,那官哥儿在奶子怀里只搐气儿了。慌的奶子叫李瓶儿:“娘,你来看哥哥,这黑眼睛珠儿只往上翻,口里气儿只有出来的,没有进去的。”这李瓶儿走来抱到怀中,一面哭起来,叫丫头:“快请你爹去!你说孩子待断气也。”可可常峙节又走来说话,告诉房子儿寻下了,门面两间,二层,大小四间,只要三十五两银子。西门庆听见后边官哥儿重了,就打发常峙节起身,说:“我不送你罢,改日我使人拿银子和你看去。”急急走到李瓶儿房中。月娘众人都在房里瞧着,那孩子在他娘怀里一口口搐气儿。西门庆不忍看他,走到明间椅子上坐着,只长吁短叹。那消半盏茶时,官哥儿呜呼哀哉,断气身亡。时八月廿三日申时也,只活了一年零两个月。合家大小放声号哭。那李瓶儿挝耳挠腮,一头撞在地下,哭的昏过去。半日方才苏省,搂着他大放声哭叫道:“我的没救星儿,心疼杀我了!宁可我同你一答儿里死了罢,我也不久活在世上了。我的抛闪杀人的心肝,撇的我好苦也!”那奶子如意儿和迎春在旁,哭的言不得,动不得。西门庆即令小厮收拾前厅西厢房干净,放下两条宽凳,要把孩子连枕席被褥抬出去那里挺放。那李瓶儿倘在孩儿身上,两手搂抱着,那里肯放!口口声声直叫:“没救星的冤家!娇娇的儿!生揭了我的心肝去了!撇的我枉费辛苦,干生受一场,再不得见你了,我的心肝!……”月娘众人哭了一回,在旁劝他不住。西门庆走来,见他把脸抓破了,滚的宝髻蓬松,乌云散乱,便道:“你看蛮的!他既然不是你我的儿女,干养活他一场,他短命死了,哭两声丢开罢了,如何只顾哭了去!又哭不活他,你的身子也要紧。如今抬出去,好叫小厮请阴阳来看。——这是甚么时候?”月娘道:“这个也有申时前后。”玉楼道:“我头里怎么说来?他管情还等他这个时候才去。——原是申时生,还是申时死。日子又相同,都是二十三日,只是月分差些。圆圆的一年零两个月。”李瓶儿见小厮每伺候两旁要抬他,又哭了,说道:“慌抬他出去怎么的?大妈妈,你伸手摸摸,他身上还热哩!”叫了一声:“我的儿嚛!你教我怎生割舍的你去?坑得我好苦也!……”一头又撞倒在地下,哭了一回。众小厮才把官哥儿抬出,停在西厢房内。

月娘向西门庆计较:“还对亲家那里并他师父庙里说声去。”西门庆道,“他师父庙里,明早去罢。”一面使玳安往乔大户家说了,一面使人请了徐阴阳来批书。又拿出十两银子与贲四,教他快抬了一付平头杉板,令匠人随即攒造了一具小棺椁儿,就要入殓。乔宅那里一闻来报,乔大户娘子随即坐轿子来,进门就哭。月娘众人又陪着大哭了一场,告诉前事一遍。不一时,阴阳徐先生来到,看了,说道:“哥儿还是正申时永逝。”月娘吩咐出来,教与他看看黑书。徐先生将阴阳秘书瞧了一回,说道:“哥儿生于政和丙申六月廿三日申时,卒于政和丁酉八月廿三日申时。月令丁酉,日干壬子,犯天地重丧,本家要忌:忌哭声。亲人不忌。入殓之时,蛇、龙、鼠、兔四生人,避之则吉。又黑书上云:壬子日死者,上应宝瓶宫,下临齐地。他前生曾在兖州蔡家作男子,曾倚力夺人财物,吃酒落魄,不敬天地六亲,横事牵连,遭气寒之疾,久卧床席,秽污而亡。今生为小儿,亦患风痫之疾。十日前被六畜惊去魂魄,又犯土司太岁,先亡摄去魂魄,托生往郑州王家为男子,后作千户,寿六十八岁而终。”须臾,徐先生看了黑书,请问老爹,明日出去或埋或化,西门庆道:“明日如何出得!搁三日,念了经,到五日出去,坟上埋了罢。”徐先生道:“二十七日丙辰,合家本命都不犯,宜正午时掩土。”批毕书,一面就收拾入殓,已有三更天气。李瓶儿哭着往房中,寻出他几件小道衣、道髻、鞋袜之类,替他安放在棺椁内,钉了长命钉,合家大小又哭了一场,打发阴阳去了。

次日,西门庆乱着,也没往衙门中去。夏提刑打听得知,早晨衙门散时,就来吊问。又差人对吴道官庙里说知,到三日,请报恩寺八众僧人在家诵经。吴道官庙里并乔大户家,俱备折卓三牲来祭奠。吴大舅、沈姨夫、门外韩姨夫、花大舅都有三牲祭卓来烧纸。应伯爵、谢希大、温秀才、常峙节、韩道国、甘出身、贲第传、李智、黄四都斗了分资,晚夕来与西门庆伴宿。打发僧人去了,叫了一起提偶的,先在哥儿灵前祭毕,然后,西门庆在大厅上放桌席管待众人。那日院中李桂姐、吴银儿并郑月儿三家,都有人情来上纸。

李瓶儿思想官哥儿,每日黄恹恹,连茶饭儿都懒待吃,题起来只是哭涕,把喉音都哭哑了。西门庆怕他思想孩儿,寻了拙智,白日里吩咐奶子、丫鬟和吴银儿相伴他,不离左右。晚夕,西门庆一连在他房中歇了三夜,枕上百般解劝。薛姑子夜间又替他念《楞严经》、《解冤咒》,劝他:“休要哭了。他不是你的儿女,都是宿世冤家债主。《陀罗经》上不说的好:昔日有一妇人,生产孩儿三遍,俱不过两岁而亡,妇人悲啼不已。抱儿江边,不忍抛弃。感得观世音菩萨化作一僧,谓此妇人曰:‘不用啼哭,此非你儿,是你生前冤家。三度托生,皆欲杀汝。你若不信,我交你看。’将手一指,其儿遂化作一夜叉之形,向水中而立,报言:‘汝曾杀我来,我特来报冤。今因汝常持《佛顶心陀罗经》,善神日夜拥护,所以杀汝个得。我已蒙观世音菩萨受度了,从今永不与汝为冤。’道毕,遂沉水中不见。不该我贫僧说,你这儿子,必是宿世冤家,托来你荫下,化目化财,要恼害你身。为你舍了此《佛顶心陀罗经》一千五百卷,有此功行,他害你不得,故此离身。到明日再生下来,才是你儿女。”李瓶儿听了,终是爱缘不断。但题起来,辄流涕不止。

须臾过了五日,到廿七日早晨,雇了八名青衣白帽小童,大红销金棺与幡幢、雪盖、玉梅、雪柳围随,前首大红铭旌,题着“西门冢男之枢”。吴道官庙里,又差了十二众青衣小道童儿来,绕棺转咒《生神玉章》,动清乐送殡。众亲朋陪西门庆穿素服走至大街东口,将及门上,才上头口。西门庆恐怕李瓶儿到坟上悲痛,不叫他去。只是吴月娘、李娇儿、孟玉楼、潘金莲、大姐,家里五顶轿子,陪乔亲家母、大妗子和李桂儿、郑月儿、吴舜臣媳妇郑三姐往坟头去,留下孙雪娥、吴银儿并两个姑子在家与李瓶儿做伴儿。李瓶儿见不放他去,见棺材起身,送出到大门首,赶着棺材大放声,一口一声只叫:“不来家亏心的儿嚛!”叫的连声气破了。不防一头撞在门底下,把粉额磕伤,金钗坠地,慌的吴银儿与孙雪娥向前搊扶起来,劝归后边去了。到了房中,见炕上空落落的,只有他耍的那寿星博浪鼓儿还挂在床头上,想将起来,拍了桌子,又哭个不了。吴银儿在旁,拉着他手劝说道:“娘少哭了,哥哥已是抛闪你去了,那里再哭得活!你须自解自叹,休要只顾烦恼。”雪娥道:“你又年少青春,愁到明日养不出来也怎的?这里墙有缝,壁有眼,俺每不好说的。他使心用心,反累已身。他将你孩子害了,教他一还一报,问他要命。不知你我被他活埋了几遭了!只要汉子常守着他便好,到人屋里睡一夜儿,他就气生气死。早是前者,你每都知道,汉子等闲不到我后边,才到了一遭儿,你看他就背地里唧喳成一块,对着他姐儿每说我长道我短。俺每也不言语,每日洗眼儿看着他。这个淫妇,到明日还不知怎么死哩!”李瓶儿道:“罢了,我也惹了一身病在这里,不知在今日明日死,和他也争执不得了,随他罢!”

正说着,只见奶子如意儿向前跪下,哭道:“小媳妇有句活,不敢对娘说——今日哥儿死了,乃是小媳妇没造化。只怕往后爹与大娘打发小媳妇出去,小媳妇男子汉又没了,那里投奔?”李瓶儿见他这般说,又心中伤痛起来,便道:“怪老婆,孩子便没了,我还没死哩!总然我到明日死了,你恁在我手下一场,我也不教你出门。往后你大娘生下哥儿小姐来,交你接了奶,就是一般了。你慌乱的是甚么?”那如意儿方才不言语了。李瓶儿良久又悲恸哭起来,雪娥与吴银儿两个又解劝说道:“你肚中吃了些甚么,只顾哭了去!”一面叫绣春后边拿了饭来,摆在桌上,陪他吃。那李瓶儿怎生咽下去!只吃了半瓯儿,就丢下不吃了。

西门庆在坟上,叫徐先生画了穴,把官哥儿就埋在先头陈氏娘怀中,抱孙葬了。那日乔大户井众亲戚都有祭祀,就在新盖卷棚管待饮酒一日。来家,李瓶儿与月娘、乔大户娘子、大妗子磕着头又哭了。向乔大户娘子说道:“亲家,谁似奴养的孩儿不气长,短命死了。既死了,累你家姐姐做了望门寡,劳而无功,亲家休要笑话。”乔大户娘子说道:“亲家怎的这般说话?孩儿每各人寿数,谁人保的后来的事!常言:先亲后不改。亲家每又不老,往后愁没子孙?须要慢慢来。亲家也少要烦恼了。”说毕,作辞回家去了。

西门庆在前厅教徐先生洒扫,各门上都贴辟非黄符。死者煞高三丈,向东北方而去,遇日游神冲回不出,斩之则吉,亲人不忌。西门庆拿出一匹大布、二两银子谢了徐先生,管待出门。晚夕入李瓶儿房中陪他睡。夜间百般言语温存。见官哥儿的戏耍物件都还在跟前,恐怕这瓶儿看见思想烦恼,都令迎春拿到后边去了。正是:

\[
思想娇儿昼夜啼,寸心如割命悬丝。
世间万般哀苦事,除非死别共生离。
\]

\newpage
%# -*- coding:utf-8 -*-
%%%%%%%%%%%%%%%%%%%%%%%%%%%%%%%%%%%%%%%%%%%%%%%%%%%%%%%%%%%%%%%%%%%%%%%%%%%%%%%%%%%%%


\chapter{李瓶儿病缠死孽\KG 西门庆官作生涯}


词曰:

\[
倦睡恹恹生怕起,如痴如醉如慵,半垂半卷旧帘栊。眼穿芳草绿,泪衬落花红。追忆当年魂梦断,为云为雨为风。凄凄楼上数归鸿。悲泪三两阵,哀绪万千重。
\]

话说潘金莲见孩子没了,每日抖擞精神,百般称快,指着丫头骂道:“贼淫妇!我只说你日头常响午,却怎的今日也有错了的时节?你斑鸠跌了蛋——也嘴答谷了。春凳折了靠背儿——没的椅了。王婆子卖了磨——推不的了。老鸨子死了粉头——没指望了。却怎的也和我一般!”李瓶儿这边屋里分明听见,不敢声言,背地里只是掉泪。着了这暗气暗恼,又加之烦恼忧戚,渐渐精神恍乱,梦魂颠倒,每日茶饭都减少了。自从葬了官哥儿第二日,吴银儿就家去了。老冯领了个十三岁的丫头来,五两银子卖与孙雪娥房中使唤,改名翠儿,不在话下。

这李瓶儿一者思念孩儿,二者着了重气,把旧病又发起来,照旧下边经水淋漓不止。西门庆请任医官来看,讨将药来吃下去,如水浇石一般,越吃越旺。那消半月之间,渐渐容颜顿减,肌肤消瘦,而精彩丰标无复昔时之态矣。正是:肌骨大都无一把,如何禁架许多愁!一日,九月初旬,天气凄凉,金风渐渐。李瓶儿夜间独宿房中,银床枕冷,纱窗月浸,不觉思想孩儿,唏嘘长叹,恍恍然恰似有人弹的窗棂响。李瓶儿呼唤丫鬓,都睡熟了不答,乃自下床来,倒靸弓鞋,翻披绣袄,开了房门。出户视之,仿佛见花子虚抱着官哥儿叫他,新寻了房儿,同去居住。李瓶儿还舍不的西门庆,不肯去,双手就抱那孩儿,被花子虚只一推,跌倒在地。撒手惊觉,却是南柯一梦。吓了一身冷汗,呜呜咽咽,只哭到天明。正是:有情岂不等,着相自家迷。有诗为证:

\[
纤纤新月照银屏,人在幽闺欲断魂。
益悔风流多不足,须知恩爱是愁根。
\]

那时,来保南京货船又到了,使了后生王显上来取车税银两。西门庆这里写书,差荣海拿一百两银子,又具羊酒金缎礼物谢主事:“就说此货过税,还望青目一二。”家中收拾铺面完备,又择九月初四日开张,就是那日卸货,连行李共装二十大车。那日,亲朋递果盒挂红者约有三十多人,夏提刑也差人送礼花红来。乔大户叫了十二名吹打的乐工、杂耍撮弄。西门庆这里,李铭、吴惠、郑春三个小优儿弹唱。甘伙计与韩伙计都在柜上发卖,一个看银子,一个讲说价钱,崔本专管收生活。西门庆穿大红,冠带着,烧罢纸,各亲友递果盒把盏毕,后边厅上安放十五张桌席,五果五菜、三汤五割,从新递酒上坐,鼓乐喧天。在坐者有乔大户、吴大舅、吴二舅、花大舅、沈姨夫、韩姨夫、吴道官、倪秀才、温葵轩、应伯爵、谢希大、常峙节,还有李智、黄四、傅自新等众伙计主管并街坊邻舍,都坐满了席面。三个小优儿在席前唱了一套《南吕·红衲袄》“混元初生太极”。须臾,酒过五巡,食割三道,下边乐工吹打弹唱,杂耍百戏过去,席上觥筹交错。应伯爵、谢希大飞起大钟来,杯来盏去。

饮至日落时分,把众人打发散了,西门庆只留下吴大舅、沈姨夫、韩姨夫、温葵轩、应伯爵、谢希大,从新摆上桌席留后坐。那日新开张,伙计攒帐,就卖了五百余两银子。西门庆满心欢喜,晚夕收了铺面,把甘伙计、韩伙计、傅伙计、崔本、贲四连陈敬济都邀来,到席上饮酒。吹打良久,把吹打乐工也打发去了,止留下三个小优儿在席前唱。

应伯爵吃的已醉上来,走出前边解手,叫过李铭问道:“那个扎包髻儿清俊的小优儿,是谁家的?”李铭道:“二爹原来不知道?”因说道:“他是郑奉的兄弟郑春。前日爹在他家吃酒,请了他姐姐爱月儿了。”伯爵道:“真个?怪道前日上纸送殡都有他。”于是归到酒席上,向西门庆道:“哥,你又恭喜,又抬了小舅子了。”西门庆笑道:“怪狗才,休要胡说。”一面叫过王经来:“斟与你应二爹一大杯酒。”伯爵向吴大舅说道:“老舅,你怎么说?这钟罚的我没名。”西门庆道:“我罚你这狗才一个出位妄言。”伯爵低头想了想儿,呵呵笑了,道:“不打紧处,等我吃,我吃死不了人。”又道:“我从来吃不得哑酒,你叫郑春上来唱个儿我听,我才罢了。”当下,三个小优一齐上来弹唱。伯爵令李铭、吴惠下去:“不要你两个。我只要郑春单弹着筝儿,只唱个小小曲儿我下酒罢。”谢希大叫道:“郑春你过来,依着你应二爹唱个罢。”西门庆道:“和花子讲过:有一个曲儿吃一钟酒。”叫玳安取了两个大银钟放在应二面前。那郑春款按银筝,低低唱《清江引》道:

\[
一个姐儿十六七,见一对蝴蝶戏。
香肩靠粉墙,春笋弹珠泪。
唤梅香赶他去别处飞。
\]
郑春唱了请酒,伯爵才饮讫,玳安又连忙斟上。郑春又唱:

\[
转过雕栏正见他,斜倚定荼蘼架;
佯羞整凤衩,不说昨宵话,笑吟吟掐将花片儿打。
\]
伯爵吃过,连忙推与谢希大,说道:“罢,我是成不的,成不的!这两大钟把我就打发了。”谢希大道:“傻花子,你吃不得推与我来,我是你家有\textMaoPi 的蛮子?”伯爵道:“傻花子,我明日就做了堂上官儿,少不的是你替。”西门庆道:“你这狗才,到明日只好做个韶武。”伯爵笑道:“傻孩儿,我做了韶武,把堂上让与你就是了。”西门庆笑令玳安儿:“拿磕瓜来打这贼花子!”谢希大悄悄向他头上打了一个响瓜儿,说道:“你这花子,温老先生在这里,你口里只恁胡说。”伯爵道:“温老先儿他斯文人,不管这闲事。”温秀才道:“二公与我这东君老先生,原来这等厚。酒席中间,诚然不如此也不乐。悦在心,乐主发散在外,自不觉手之舞之,足之蹈之如此。”

沈姨夫向西门庆说:“姨夫,不是这等。请大舅上席,还行个令儿——或掷骰,或猜枚,或看牌,不拘诗词歌赋、顶真续麻、急口令,说不过来吃酒。这个庶几均匀,彼此不乱。”西门庆道:“姨夫说的是。”先斟了一杯,与吴大舅起令。吴大舅拿起骰盆儿来说道:“列位,我行一令:顺着数去,遇点要个花名,花名下要顶真,不拘诗词歌赋说一句。说不来,罚一大杯。我就是一起——

\[
一掷一点红,红梅花对白梅花。”
\]
吴大舅掷了个二,多一杯。饮过酒,该沈姨夫接掷。沈姨夫说道:

\[
“二掷并头莲,莲漪戏彩鸳。”
\]
沈姨夫也掷了个二,饮过两杯,就过盆与韩姨夫行令。韩姨夫说道:

\[
“三掷三春李,李下不整冠。”
\]
韩姨夫掷完,吃了酒,送与温秀才。秀才道:“我学生奉令了——

\[
四掷状元红,红紫不以为亵服。”
\]
温秀才只遇了一杯酒,吃过,该应伯爵行令。伯爵道:“我在下一个字也不识,不会顶真,只说个急口令儿罢:

\[
一个急急脚脚的老小,左手拿着一个黄豆巴斗,右手拿着一条绵花叉口,望前只管跑走。一个黄白花狗,咬着那绵花叉口,那急急脚脚的老小,放下那左手提的那黄豆巴斗,走向前去打那黄白花狗。不知手斗过那狗,狗斗过那手。”
\]
西门庆笑骂道:“你这贼诌断肠子的天杀的,谁家一个手去逗狗来?一口不被那狗咬了?”伯爵道:“谁叫他不拿个棍儿来!我如今抄化子不见了拐棒儿——受狗的气了。”谢希大道:“大官人,你看花子自家倒了架,说他是花子。”西门庆道:“该罚他一钟,不成个令。谢子纯,你行罢!”谢希大道:“我也说一个,比他更妙:

\[
墙上一片破瓦,墙下一匹骡马。落下破瓦,打着骡马。不知是那破瓦打伤骡马,不知是那骡马踏碎了破瓦。”
\]
伯爵道:“你笑话我的令不好,你这破瓦倒好?你家娘子儿刘大姐就是个骡马,我就是个破瓦。——俺两个破磨对瘸驴。”谢希大道:“你家那杜蛮婆老淫妇,撒把黑豆只好喂猪哄狗,也不要他。”两个人斗了回嘴,每人斟了一钟,该韩伙计掷。韩道国道:“老爹在上,小人怎敢占先?”西门庆道:“顺着来,不要逊了。”于是韩道国说道:

\[
“五掷腊梅花,花里遇神仙。”
\]
掷毕,该西门庆掷,西门庆道:“我要掷个六:

\[
六掷满天星,星辰冷落碧潭水。”
\]
果然掷出个六来。应伯爵看见,说道:“哥今年上冬,管情加官进禄,主有庆事。”于是斟了一大杯酒与西门庆。一面李铭等三个上来弹唱,顽耍至更阑方散。西门庆打发小优儿出门,看收了家伙,派定韩道国、甘伙计、崔本、来保四人轮流上宿,吩咐仔细门户,就过那边去了。一宿晚景不题。

次日,应伯爵领了李智、黄四来交银子,说:“此遭只关了一千四百五六十两银子,不够还人,只挪了三百五十两银子与老爹。等下遭关出来再找完,不敢迟了。”伯爵在旁又替他说了两句美言。西门庆教陈敬济来,把银子兑收明白,打发去了。银子还摆在桌上,西门庆因问伯爵道:“常二哥说他房子寻下了,前后四间,只要三十五两银子。他来对我说,正值小儿病重,我心里乱,就打发他去了。不知他对你说来不曾?”伯爵道:“他对我说来,我说,你去的不是了,他乃郎不好,他自乱乱的,有甚么心绪和你说话?你且休回那房主儿,等我见哥,替你题就是了。”西门庆道:“也罢,你吃了饭,拿一封五十两银子,今日是个好日子,替他把房子成了来罢。剩下的,叫常二哥门面开个小铺儿,月间赚几钱银子儿,就够他两口儿盘搅了。”伯爵道:“此是哥下顾他了。”不一时,放桌儿摆上饭来,西门庆陪他吃了饭,道:“我不留你。你拿了这银子去,替他干干这勾当去罢。”伯爵道:“你这里还教个大官和我去。”西门庆道:“没的扯淡,你袖了去就是了。”伯爵道:“不是这等说,今日我还有小事。实和哥说,家表弟杜三哥生日,早晨我送了些礼儿去,他使小厮来请我后晌坐坐。我不得来回你话,教个大官儿跟了去,成了房子,好教他来回你话的。”西门庆道:“若是恁说,叫王经跟你去罢。”一面叫王经跟伯爵来到了常家。

常峙节正在家,见伯爵至,让进里面坐。伯爵拿出银子来与常峙节看,说:“大官人如此如此,教我同你今日成房子去,我又不得闲,杜三哥请我吃酒。我如今了毕你的事,我方才得去。”常峙节连忙叫浑家快看茶来,说道:“哥的盛情,谁肯!”一面吃茶毕,叫了房中人来,同到新市街,兑与卖主银子,写立房契。伯爵吩咐与王经,归家回西门庆话。剩的银子,叫与常峙节收了。他便与常峙节作别,往杜家吃酒去了。西门庆看了文契,还使王经送与常二收了,不在话下。正是:

\[
求人须求大丈夫,济人须济急时无。
一切万般皆下品,谁知恩德是良图。
\]

\newpage
%# -*- coding:utf-8 -*-
%%%%%%%%%%%%%%%%%%%%%%%%%%%%%%%%%%%%%%%%%%%%%%%%%%%%%%%%%%%%%%%%%%%%%%%%%%%%%%%%%%%%%


\chapter{西门庆乘醉烧阴户\KG 李瓶儿带病宴重阳}


词曰:

\[
蛩声泣露惊秋枕,泪湿鸳鸯锦。独卧玉肌凉,残更与恨长。 
阴风翻翠幌,雨涩灯花暗。毕竟不成眠,鸦啼金井寒。
\]

话说一日,韩道国铺中回家,睡到半夜,他老婆王六儿与他商议道:“你我被他照顾,挣了恁些钱,也该摆席酒儿请他来坐坐。况他又丢了孩儿,只当与他释闷,他能吃多少!彼此好看。就是后生小郎看着,到明日南边去,也知财主和你我亲厚,比别人不同。”韩道国道:“我心里也是这等说。明日初五日是月忌,不好。到初六日,安排酒席,叫两个唱的,具个柬帖,等我亲自到宅内,请老爹散闷坐坐。我晚夕便往铺子里睡去。”王六儿道:“平白又叫甚么唱的?只怕他酒后要来这屋里坐坐,不方便。隔壁乐三嫂家,常走的一个女儿申二姐,年纪小小的,且会唱,他又是瞽目的,请将他来唱唱罢。要打发他过去还容易。”韩道国道:“你说的是。”一宿晚景题过。

到次日,韩道国走到铺子里,央及温秀才写了个请柬儿,亲见西门庆,声喏毕,说道:“明日,小人家里治了一杯水酒,无事请老爹贵步下临,散闷坐一日。”因把请柬递上去。西门庆看了,说道:“你如何又费此心。我明日倒没事,衙门中回家就去。”韩道国作辞出门。到次早,拿银子叫后生胡秀买嗄饭菜蔬,一面叫厨子整理,又拿轿子接了申二姐来,王六儿同丫鬟伺候下好茶好水,单等西门庆来到。等到午后,只见琴童儿先送了一坛葡萄酒来,然后西门庆坐着凉轿,玳安、王经跟随,到门首下轿,头戴忠靖冠,身穿青水纬罗直身,粉头皂靴。韩道国迎接入内,见毕礼数,说道:“又多谢老爹赐将酒来。”正面独独安放一张交椅,西门庆坐下。

不一时,王六儿打扮出来,与西门庆磕了四个头,回后边看茶去了。须臾,王经拿出茶来,韩道国先取一盏,举的高高的奉与西门庆,然后自取一盏,旁边相陪。吃毕,王经接了茶盏下去,韩道国便开言说道:“小人承老爹莫大之恩,一向在外,家中小媳妇承老爹看顾,王经又蒙抬举,叫在宅中答应,感恩不浅。前日哥儿没了,虽然小人在那里,媳妇儿因感了些风寒,不曾往宅里吊问的,恐怕老爹恼。今日,一者请老爹解解闷,二者就恕俺两口儿罪。”西门庆道:“无事又教你两口儿费心。”说着,只见王六儿也在旁边坐下。因向韩道国道:“你和老爹说了不?”道国道:“我还不曾说哩。”西门庆问道:“是甚么?”王六儿道:“他今日要内边请两位姐儿来伏侍老爹,我恐怕不方便,故不去请。隔壁乐家常走的一个女儿,叫做申二姐,诸般大小时样曲儿,连数落都会唱。我前日在宅里,见那一位郁大姐唱的也中中的,还不如这申二姐唱的好。教我今日请了他来,唱与爹听。未知你老人家心下何如?若好,到明日叫了宅里去,唱与他娘每听。”西门庆道:“既是有女儿,亦发好了。你请出来我看看。”不一时,韩道国叫玳安上来:“替老爹宽去衣服。”一面安放桌席,胡秀拿果菜案酒上来。王六儿把酒打开,烫热了,在旁执壶,道国把盏,与西门庆安席坐下,然后才叫出申二姐来。西门庆睁眼观看,见他高髻云鬟,插着几枝稀稀花翠,淡淡钗梳,绿袄红裙,显一对金莲趫趫;桃腮粉脸,抽两道细细春山。望上与西门庆磕了四个头。西门庆便道:“请起。你今青春多少?”申二姐道:“小的二十一岁了。”又问:“你记得多少唱?”申二姐道:“大小也记百十套曲子。”西门庆令韩道国旁边安下个坐儿与他坐。申二姐向前行毕礼,方才坐下。先拿筝来唱了一套《秋香亭》,然后吃了汤饭,添换上来,又唱了一套《半万贼兵》。落后酒阑上来,西门庆吩咐:“把筝拿过去,取琵琶与他,等他唱小词儿我听罢。”那申二姐一迳要施逞他能弹会唱。一面轻摇罗袖,款跨鲛绡,顿开喉音,把弦儿放得低低的,弹了个《四不应·山坡羊》。唱完了,韩道国教浑家满斟一盏,递与西门庆。王六儿因说:“申二姐,你还有好《锁南枝》,唱两个与老爹听。”那申二姐就改了调儿,唱《锁南枝》道:

初相会,可意人,年少青春,不上二旬。黑\textSiCan \textSiCan 两朵乌云,红馥馥一点朱唇,脸赛夭桃如嫩笋。若生在画阁兰堂,端的也有个夫人分。可惜在章台,出落做下品。但能够改嫁从良,胜强似弃旧迎新。

初相会,可意娇,月貌花容,风尘中最少。瘦腰肢一捻堪描,俏心肠百事难学,恨只恨和他相逢不早。常则怨席上樽前,浅斟低唱相偎抱。一觑一个真,一看一个饱。虽然是半霎欢娱,权且将闷解愁消。

西门庆听了这两个《锁南枝》,正打着他初请了郑月儿那一节事来,心中甚喜。王六儿满满的又斟上一盏,笑嘻嘻说道:“爹,你慢慢儿的饮,申二姐这个才是零头儿,他还记的好些小令儿哩。到明日闲了,拿轿子接了,唱与他娘每听,管情比郁大姐唱的高。”西门庆因说:“申二姐,我重阳那日,使人来接你,去不去?”申二姐道:“老爹说那里话,但呼唤,怎敢违阻!”西门庆听见他说话伶俐,心中大喜。

不一时,交杯换盏之间,王六儿恐席间说话不方便,叫他唱了几套,悄悄向韩道国说:“教小厮招弟儿,送过乐三嫂家歇去罢。”临去拜辞,西门庆向袖中掏出一包儿三钱银子,赏他买弦。申二姐连忙嗑头谢了。西门庆约下:“我初八日使人请你去。”王六儿道:“爹只使王经来对我说,等我这里教小厮请他去。”说毕,申二姐往隔壁去了。韩道国与老婆说知,也就往铺子里睡去了。只落下老婆在席上,陪西门庆掷骰饮酒。吃了一回,两个看看吃的涎将上来,西门庆推起身更衣,就走入妇人房里,两个顶门顽耍。王经便把灯烛拿出来,在前半间和玳安、琴童儿做一处饮酒。

那后生胡秀,在厨下偷吃了几碗酒,打发厨子去了,走在王六儿隔壁供养佛祖先堂内,地下铺着一领席,就睡着了。睡了一觉起来,忽听见妇人房里声唤,又见板壁缝里透过灯亮来,只道西门庆去了,韩道国在房中宿歇。暗暗用头上簪子刺破板缝中糊的纸,往那边张看。见那边房中亮腾腾点着灯烛,不想西门庆和老婆在屋里正干得好。伶伶俐俐看见,把老婆两只腿,却是用脚带吊在床头上,西门庆上身止着一件绫袄儿,下身赤露,就在床沿上一来一往,一动一静,扇打的连声响亮,老婆口里百般言语都叫将出来。良久,只听老婆说:“我的亲达!你要烧淫妇,随你心里拣着那块只顾烧,淫妇不敢拦你。左右淫妇的身子属了你,怕那些儿了!”西门庆道:“只怕你家里的嗔是的。”老婆道:“那忘八七个头八个胆,他敢嗔!他靠着那里过日子哩?”西门庆道:“你既一心在我身上,等这遭打发他和来保起身,亦发留他长远在南边,做个买手置货罢。”老婆道:“等走过两遭儿,却教他去。省的闲着在家做甚么?他说倒在外边走惯了,一心只要外边去。你若下顾他,可知好哩!等他回来,我房里替他寻下一个,我也不要他,一心扑在你身上,随你把我安插在那里就是了。我若说一句假,把淫妇不值钱身子就烂化了。”西门庆道:“我儿,你快休赌誓!”两个一动一静,都被胡秀听了个不亦乐乎。

韩道国先在家中不见胡秀,只说往铺子里睡去了。走到缎子铺里,问王显、荣海,说他没来。韩道国一面又走回家,叫开门,前后寻胡秀,那里得来,只见王经陪玳安、琴童三个在前边吃酒。胡秀听见他的语音来家,连忙倒在席上,又推睡了。不一时,韩道国点灯寻到佛堂地下,看见他鼻口内打鼾睡,用脚踢醒,骂道:“贼野狗死囚,还不起来!我只说先往铺子里睡去,你原来在这里挺得好觉儿。还不起来跟我去!”那胡秀起来,推揉了揉眼,楞楞睁睁跟道国往铺子里去了。

西门庆弄老婆,直弄够有一个时辰,方才了事。烧了王六儿心口里并\textMaoBi 盖子上、尾亭骨儿上共三处香。老婆起来穿了衣服,教丫头打发舀水净了手,重筛暖酒,再上佳肴,情话攀盘。又吃了几钟,方才起身上马,玳安、王经、琴童三个跟着。到家中已有二更天气,走到李瓶儿房中。李瓶儿睡在床上,见他吃的酣酣儿的进来,说道:“你今日在谁家吃酒来?”西门庆道:“韩道国家请我。见我丢了孩子,与我释闷。他叫了个女先生申二姐来,年纪小小,好不会唱!又不说郁大姐。等到明日重阳,使小厮拿轿子接他来家,唱两日你每听,就与你解解闷。你紧心里不好,休要只顾思想他了。”说着,就要叫迎春来脱衣裳,和李瓶儿睡。李瓶儿道:“你没的说!我下边不住的长流,丫头替我煎药哩。你往别人屋里睡去罢。你看着我成日好模样儿罢了,只有一口游气儿在这里,又来缠我起来。”西门庆道:“我的心肝!我心里舍不的你。只要和你睡,如之奈何?”李瓶儿瞟了他一眼,笑了笑儿:“谁信你那虚嘴掠舌的。我倒明日死了,你也舍不的我罢!”又道:“亦发等我好好儿,你再进来和我睡也不迟。”西门庆坐了一回,说道:“罢,罢。你不留我,等我往潘六儿那边睡去罢。”李瓶儿道:“原来你去,省的屈着你那心肠儿。他那里正等的你火里火发,你不去,却忙惚儿来我这屋里缠。”西门庆道:“你恁说,我又不去了。”李瓶儿微笑道:“我哄你哩,你去罢。”于是打发西门庆过去了。李瓶儿起来,坐在床上,迎春伺候他吃药。拿起那药来,止不住扑簌簌香腮边滚下泪来,长吁了一口气,方才吃了那盏药。正是:

\[
心中无限伤心事,付与黄鹂叫几声。
\]

不说李瓶儿吃药睡了,单表西门庆到于潘金莲房里。金莲才叫春梅罩了灯上床睡下。忽见西门庆推开门进来便道:“我儿,又早睡了?”金莲道:“稀幸!那阵风儿刮你到我这屋里来!”因问:“你今日往谁家吃酒去来?”西门庆道:“韩伙计打南边来,见我没了孩子,一者与我释闷,二者照顾他外边走了这遭,请我坐坐。”金莲道:“他便在外边,你在家又照顾他老婆了。”西门庆道:“伙计家,那里有这道理?”妇人道:“伙计家,有这个道理!齐腰拴着根线儿,只怕\textuni{34B2}过界儿去了。你还捣鬼哄俺每哩,俺每知道的不耐烦了!你生日,贼淫妇他没在这里?你悄悄把李瓶儿寿字簪子,黄猫黑尾偷与他,却叫他戴了来施展。大娘、孟三儿,这一家子那个没看见?吃我问了一句,他把脸儿都红了,他没告诉你?今日又摸到那里去,贼没廉耻的货,一个大摔瓜长淫妇,乔眉乔样,描的那水鬓长长的,搽的那嘴唇鲜红的——倒象人家那血\textMaoBi 。甚么好老婆,一个大紫腔色黑淫妇,我不知你喜欢他那些儿!嗔道把忘八舅子也招惹将来,一早一晚教他好往回传话儿。”西门庆坚执不认,笑道:“怪小奴才儿,单管只胡说,那里有此勾当?今日他男子汉陪我坐,他又没出来。”妇人道:“你拿这个话儿来哄我?谁不知他汉子是个明忘八,又放羊,又拾柴,一径把老婆丢与你,图你家买卖做,要赚你的钱使。你这傻行货子,只好四十里听铳响罢了!”西门庆脱了衣裳,坐在床沿上,妇人探出手来,把裤子扯开,摸见那话软叮当的,托子还带在上面,说道:“可又来,你腊鸭子煮到锅里——身子儿烂了,嘴头儿还硬。见放着不语先生在这里,强盗和那淫妇怎么弄耸,耸到这咱晚才来家?弄的恁个样儿,嘴头儿还强哩!你赌个誓,我叫春梅舀一瓯子凉水,你只吃了,我就算你好胆子。论起来,盐也是这般咸,醋也是这般酸,秃子包网中——饶这一抿子儿也罢了。若是信着你意儿,把天下老婆都耍遍了罢。贼没羞的货,一个大眼里火行货子!你早是个汉子,若是个老婆,就养遍街,\textuni{34B2}遍巷。”几句说的西门庆睁睁的,只是笑。

上的床来,叫春梅筛热了烧酒,把金穿心盒儿内药拈了一粒,放在口里咽下去,仰卧在枕上,令妇人:“我儿,你下去替你达品,品起来是你造化。”那妇人一径做乔张致,便道:“好干净儿!你在那淫妇窟窿子里钻了来,教我替你咂,可不臜杀了我!”西门庆道:“怪小淫妇儿,单管胡说白道的,那里有此勾当?”妇人道:“那里有此勾当?你指着肉身子赌个誓么!”乱了一回,教西门庆下去使水,西门庆不肯下去,妇人旋向袖子里掏出个汗巾来,将那话抹展了一回,方才用朱唇裹没。呜咂半晌,咂弄的那话奢棱跳脑,暴怒起来,乃骑在妇人身上,纵麈柄自后插入牝中,两手兜其股,蹲踞而摆之,肆行扇打,连声响亮。灯光之下,窥玩其出入之势,妇人倒伏在枕畔,举股迎凑者久之。西门庆兴犹不惬,将妇人仰卧朝上,那话上使了粉红药儿,顶入去,执其双足,又举腰没棱露脑掀腾者将二三百度。妇人禁受不的,瞑目颤声,没口子叫:“达达,你这遭儿只当将就我,不使上他也罢了。”西门庆口中呼叫道:“小淫妇儿,你怕我不怕?再敢无礼不敢?”妇人道:“我的达达,罢么,你将就我些儿,我再不敢了!达达慢慢提,看提散了我的头发。”两个颠鸳倒凤,足狂了半夜,方才体倦而寝。

话休饶舌,又早到重阳令节。西门庆对吴月娘说:“韩伙计前日请我,一个唱的申二姐,生的人材又好,又会唱。我使小厮接他来,留他两日,教他唱与你每听。”又吩咐厨下收拾肴馔果酒,在花园大卷棚聚景堂内,安放大八仙桌,合家宅眷,庆赏重阳。

不一时,王经轿子接的申二姐到了。入到后边,与月娘众人磕了头。月娘见他年小,生的好模样儿。问他套数,也会不多,诸般小曲儿倒记的有好些。一面打发他吃了茶食,先教在后边唱了两套,然后花园摆下酒席。那日,西门庆不曾往衙门中去,在家看着栽了菊花。请了月娘、李娇儿、孟玉楼、潘金莲、李瓶儿、孙雪娥并大姐,都在席上坐的。春梅、玉箫、迎春、兰香在旁斟酒伏侍。申二姐先拿琵琶在旁弹唱。那李瓶儿在房中,因身上不方便,请了半日才来。恰似风儿刮倒的一般,强打着精神陪西门庆坐,众人让他酒儿也不大吃。西门庆和月娘见他面带忧容,眉头不展,说道:“李大姐,你把心放开,教申二姐弹唱曲儿你听。”玉楼道:“你说与他,教他唱甚么曲儿,他好唱。”李瓶儿只顾不说。正饮酒中间,忽见王经走来说道:“应二爹、常二叔来了。”西门庆道:“请你应二爹、常二叔在小卷棚内坐,我就来。”王经道:“常二叔教人拿了两个盒子在外头。”西门庆向月娘道:“此是他成了房子,买礼来谢我的意思。”月娘道:“少不的安排些甚么管待他,怎好空了他去!你陪他坐去,我这里吩咐看菜儿。”西门庆临出来,又叫申二姐:“你唱个好曲儿,与你六娘听。”一直往前边去了。金莲道:“也没见这李大姐,随你心里说个甚么曲儿,教申二姐唱就是了,辜负他爹的心!为你叫将他来,你又不言语。”催逼的李瓶儿急了,半日才说出来:“你唱个‘紫陌红尘’罢。”那申二姐道:“这个不打紧,我有。”于是取过筝来,顿开喉音,细细唱了一套。唱毕,吴月娘道:“李大姐,好甜酒儿,你吃上一钟儿。”李瓶儿又不敢违阻,拿起钟儿来咽了一口儿,又放下了。坐不多时,下边一阵热热的来,又往屋里去了,不题。

且说西门庆到于小卷棚翡翠轩,只见应伯爵与常峙节在松墙下正看菊花。原来松墙两边,摆放二十盆,都是七尺高,各样有名的菊花,也有大红袍、状元红、紫袍金带、白粉西、黄粉西、满天星、醉杨妃、玉牡丹、鹅毛菊、鸳鸯花之类。西门庆出来,二人向前作揖。常峙节即唤跟来人,把盒儿掇进来。西门庆一见便问:“又是甚么?”伯爵道:“常二哥蒙哥厚情,成了房子,无可酬答,教他娘子制造了这螃蟹鲜并两只炉烧鸭儿,邀我来和哥坐坐。”西门庆道:“常二哥,你又费这个心做甚么?你令正病才好些,你又禁害他!”伯爵道:“我也是恁说。他说道别的东西儿来,恐怕哥不稀罕。”西门庆令左右打开盒儿观看:四十个大螃蟹,都是剔剥净了的,里边酿着肉,外用椒料姜蒜米儿团粉裹就,香油煠,酱油醋造过,香喷喷,酥脆好食。又是两大只院中炉烧熟鸭。西门庆看了,即令春鸿、王经掇进去,吩咐拿五十文钱赏拿盒人,因向常峙节谢了。

琴童在旁掀帘,请入翡翠轩坐。伯爵只顾夸奖不尽好菊花,问:“哥是那里寻的?”西门庆道:“是管砖厂刘太监送的。这二十盆,就连盆都送与我了。”伯爵道:“花到不打紧,这盆正是官窑双箍邓浆盆,都是用绢罗打,用脚跐过泥,才烧造这个物儿,与苏州邓浆砖一个样儿做法。如今那里寻去!”夸了一回。西门庆唤茶来吃了,因问:“常二哥几时搬过去?”伯爵道:“从兑了银子三日就搬过去了。昨见好日子,买了些杂货儿,门首把铺儿也开了。就是常二嫂兄弟,替他在铺里看银子儿。”西门庆道:“俺每几时买些礼儿,休要人多了,再邀谢子纯你三四位,我家里整理菜儿抬了去——休费烦常二哥一些东西——叫两个妓者,咱每替他暖暖房,耍一日。”常峙节道:“小弟有心也要请哥坐坐,算计来不敢请。地方儿窄狭,只怕亵渎了哥。”西门庆道:“没的扯淡,那里又费你的事起来。如今使小厮请将谢子纯来,和他说说。”即令琴童儿:“快请你谢爹去!”伯爵因问:“哥,你那日叫那两个去?”西门庆笑道:“叫将郑月儿和洪四儿去罢。”伯爵道:“哥,你是个人,你请他就不对我说声,我怎的也知道了?比李挂儿风月如何?”西门庆道:“通色丝子女不可言!”伯爵道:“他怎的前日你生日时,那等不言语,扭扭的,也是个肉佞贼小淫妇儿。”西门庆道:“等我到几时再去着,也携带你走走。你月娘会打的好双陆,你和他打两贴双陆。”伯爵道:“等我去混那小淫妇儿,休要放了他!”西门庆道:“你这歪狗才,不要恶识他便好。”正说着,谢希大到了,声诺毕,坐下。西门庆道:“常二哥如此这般,新有了华居,瞒着俺每,已搬过去了。咱每人随意出些分资,休要费烦他丝毫。我这里整治停当,教小厮抬到他府上,我还叫两个妓者,咱耍一日何如?”谢希大道:“哥吩咐每人出多少分资,俺每都送到哥这里来就是了。还有那几位?”西门庆道:“再没人,只这三四个儿,每人二星银子就够了。”伯爵道:“十分人多了,他那里没地方儿。”

正说着,只见琴童来说:“吴大舅来了。”西门庆道:“请你大舅这里来坐。”不一时,吴大舅进入轩内,先与三人作了揖,然后与西门庆叙礼坐下。小厮拿茶上来,同吃了茶,吴大舅起身说道:“请姐夫到后边说句话儿。”西门庆连忙让大舅到后边月娘房里。月娘还在卷棚内与众姊妹吃酒听唱,听见说:“大舅来了,爹陪着在后边说话哩。”一面走到上房,见大舅道了万福,叫小玉递上茶来。大舅向袖中取出十两银子递与月娘,说道:“昨日府里才领了三锭银子,姐夫且收了这十两,余者待后次再送来。”西门庆道:“大舅,你怎的这般计较?且使着,慌怎的!”大舅道:“我恐怕迟了姐夫的。”西门庆因问:“仓廒修理的也将完了?”大舅道:“还得一个月终完。”西门庆道:“工完之时,一定抚按有些奖励。”大舅道:“今年考选军政在迩,还望姐夫扶持,大巡上替我说说。”西门庆道:“大舅之事,都在于我。”

说毕话,月娘道:“请大舅前边同坐罢。”大舅道:“我去罢,只怕他三位来有甚么话说。”西门庆道:“没甚么话。常二哥新近问我借了几两银子,买下了两间房子,已搬过去了,今日买了些礼儿来谢我,节间留他每坐坐。大舅来的正好。”于是让至前边坐了。月娘连忙叫厨下打发莱儿上去。琴童与王经先安放八仙桌席端正,西门庆旋教开库房,拿出一坛夏提刑家送的菊花酒来。打开碧靛清,喷鼻香,未曾筛,先搀一瓶凉水,以去其蓼辣之性,然后贮于布甑内,筛出来醇厚好吃,又不说葡萄酒。叫王经用小金钟儿斟一杯儿,先与吴大舅尝了,然后,伯爵等每人都尝讫,极口称羡不已。须臾,大盘大碗摆将上来,众人吃了一顿。然后才拿上酿螃蟹并两盘烧鸭子来,伯爵让大舅吃。连谢希大也不知是甚么做的,这般有味,酥脆好吃。西门庆道:“此是常二哥家送我的。”大舅道:“我空痴长了五十二岁,并不知螃蟹这般造作,委的好吃!”伯爵又问道:“后边嫂子都尝了尝儿不曾?”西门庆道:“房下每都有了。”伯爵道:“也难为我这常嫂子,真好手段儿!”常峙节笑道:“贱累还恐整理的不堪口,教列位哥笑话。”

吃毕螃蟹,左右上来斟酒,西门庆令春鸿和书童两个,在旁一递一个歌唱南曲。应伯爵忽听大卷棚内弹筝歌唱之声,便问道:“哥,今日李桂姐在这里?不然,如何这等音乐之声?”西门庆道:。“你再听,看是不是?”伯爵道:“李桂姐不是,就是吴银儿。”西门庆道:“你这花子单管只瞎诌。倒是个女先生。”伯爵道:“不是郁大姐?”西门庆道:“不是他,这个是申二姐。年小哩,好个人材,又会唱。”伯爵道:“真个这等好?哥怎的不牵出来俺每瞧瞧?就唱个儿俺每听。”西门庆道:“今日你众娘每大节间,叫他来赏重阳顽耍,偏你这狗才耳朵尖,听的见!”伯爵道:“我便是千里眼,顺风耳,随他四十里有蜜蜂儿叫,我也听见了。”谢希大道:“你这花子,两耳朵似竹签儿也似,愁听不见!”两个又顽笑了一回,伯爵道:“哥,你好歹叫他出来,俺每见见儿,俺每不打紧,教他只当唱个与老舅听也罢了。休要就古执了。”西门庆吃他逼迫不过,一面使王经领申二姐出来唱与大舅听。不一时,申二姐来,望上磕了头起来,旁边安放交床儿与他坐下。伯爵问申二姐:“青春多少?”申二姐回道:“属牛的,二十一岁了。”又问:“会多少小唱?”申二姐道:“琵琶筝上套数小唱,也会百十来套。”伯爵道:“你会许多唱也够了。”西门庆道:“申二姐,你拿琵琶唱小词儿罢,省的劳动了你。说你会唱‘四梦八空’,你唱与大舅听。”吩咐王经、书童儿,席间斟上酒。那申二姐款跨鲛绡,微开檀口,慢慢唱着,众人饮酒不题。

且说李瓶儿归到房中,坐净桶,下边似尿的一般,只顾流将起来,登时流的眼黑了。起来穿裙子,忽然一阵旋晕,向前一头撞倒在地。饶是迎春在旁搊扶着,还把额角上磕伤了皮。和奶子搊到炕上,半日不省人事。慌了迎春,忙使绣春:“快对大娘说去!”绣春走到席上,报与月娘众人。月娘撇了酒席,与众姐妹慌忙走来看视。见迎春、奶子两个搊扶着他坐在炕上,不省人事。便问:“他好好的进屋里,端的怎么来就不好了?”迎春揭开净桶与月娘瞧,把月娘唬了一跳。说道:“他刚才只怕吃了酒,助赶的他血旺了,流了这些。”玉楼、金莲都说:“他几曾大吃酒来!”一面煎灯心姜汤灌他。半晌苏醒过来,才说出话儿来。月娘问:“李大姐,你怎的来?”李瓶儿道:“我不怎的。坐下桶子起来穿裙子,只见眼儿前黑黑的一块子,就不觉天旋地转起来,由不的身子就倒了。”月娘便要使来安儿:“请你爹进来——对他说,教他请任医官来看你。”李瓶儿又嗔教请去:“休要大惊小怪,打搅了他吃酒。”月娘吩咐迎春:“打铺教你娘睡罢。”月娘于是也就吃不成酒了,吩咐收拾了家伙,都归后边去了。

西门庆陪侍吴大舅众人,至晚归到后边月娘房中。月娘告诉李瓶儿跌倒之事,西门庆慌走到前边来看视。见李瓶儿睡在炕上,面色蜡查黄了,扯着西门庆衣袖哭泣。西门庆问其所以,李瓶儿道:“我到屋里坐杩子,不知怎的,下边只顾似尿也一般流将起来,不觉眼前一块黑黑的。起来穿裙子,天旋地转,就跌倒了。”西门庆见他额上磕伤一道油皮,说道,“丫头都在那里,不看你,怎的跌伤了面貌?”李瓶儿道:“还亏大丫头都在跟前,和奶子搊扶着我,不然,还不知跌的怎样的。”西门庆道:“我明早请任医官来看你。”当夜就在李瓶儿对面床上睡了一夜。

次日早晨,往衙门里去,旋使琴童请任医官去了。直到晌午才来。西门庆先在大厅上陪吃了茶,使小厮说进去。李瓶儿房里收拾干净,熏下香,然后请任医官进房中。诊毕脉,走出外边厅上,对西门庆说:“老夫人脉息,比前番甚加沉重,七情伤肝,肺火太旺,以致木旺土虚,血热妄行,犹如山崩而不能节制。若所下的血紫者,犹可以调理;若鲜红者,乃新血也。学生撮过药来,若稍止,则可有望;不然,难为矣。”西门庆道:“望乞老先生留神加减,学生必当重谢!”任医官道:“是何言语!你我厚间,又是明用情分,学生无不尽心。”西门庆待毕茶,送出门,随即具一匹杭绢、二两白金,使琴童儿讨将药来,名曰“归脾汤”,乘热吃下去,其血越流之不止。西门庆越发慌了,又请大街口胡太医来瞧。胡太医说是气冲血管,热入血室,亦取将药来。吃下去,如石沉大海一般。

月娘见前边乱着请太医,只留申二姐住了一夜,与了他五钱银子、一件云绢比甲儿并花翠,装了个盒于,就打发他坐轿子去了。花子由自从那日开张吃了酒去,听见李瓶儿不好,使了花大嫂,买了两盒礼来看他。见他瘦的黄恹恹儿,不比往时,两个在屋里大哭了一回。月娘后边摆茶请他吃了。韩道国说:“东门外住的一个看妇人科的赵太医,指下明白,极看得好。前岁,小媳妇月经不通,是他看来。老爹请他来看看六娘,管情就好哩。”西门庆听了,就使琴童和王经两个叠骑着头口,往门外请赵太医去了。

西门庆请了应伯爵来,和他商议道:“第六个房下,甚是不好的重,如之奈何?”伯爵失惊道:“这个嫂子贵恙说好些,怎的又不好起来?”西门庆道:“自从小儿没了,着了忧戚,把病又发了。昨日重阳,我接了申二姐,与他散闷顽耍,他又没好生吃酒,谁知走到屋中就晕起来,一交跌倒,把脸都磕破了。请任医官来看,说脉息比前沉重。吃了药,倒越发血盛了。”伯爵道:“你请胡太医来看,怎的说?”西门庆道:“胡大医说,是气冲了血管,吃了他的,也不见动静。今日韩伙计说,门外一个赵太医,名唤赵龙岗,专科看妇女,我使小厮请去了。把我焦愁的了不的。生生为这孩子不好,白日黑夜思虑起这病来了。妇女人家,又不知个回转,劝着他,又不依你,叫我无法可处。”

正说着,平安来报:“乔亲家爹来了。”西门庆一面让进厅上,同伯爵叙礼坐下。乔大户道:“闻得六亲家母有些不安,特来候问。”西门庆道:“便是。一向因小儿没了,着了忧戚,身上原有些不调,又发起来了。蒙亲家挂念。”乔大户道:“也曾请人来看不曾?”西门庆道:“常吃任后溪的药,昨日又请大街胡先生来看,吃药越发转盛。今日又请门外专看妇人科赵龙岗去了。”乔大户道:“咱县门前住的何老人,大小方脉俱精。他儿子何歧轩,见今上了个冠带医士。亲家何不请他来看看亲家母?”西门庆道:“既是好,等赵龙岗来,来过再请他来看看。”乔大户道:“亲家,依我愚见,不如先请了何老人来,再等赵龙岗来,叫他两个细讲一讲,就论出病原来了。然后下药,无有不效之理。”西门庆道:“亲家说的是。”一面使玳安拿拜帖儿和乔通去请。

那消半晌,何老人到来,与西门庆、乔大户等作了揖,让于上面坐下。西门庆举手道:“数年不见你老人家,不觉越发苍髯皓首。”乔大户又问:“令郎先生肄业盛行?”何老人道:“他逐日县中迎送,也不得闲,倒是老拙常出来看病。”伯爵道:“你老人家高寿了,还这等健朗。”何老人道:“老拙今年痴长八十一岁。”叙毕话,看茶上来吃了,小厮说进去。须臾,请至房中,就床看李瓶儿脉息,旋搊扶起来,坐在炕上,形容瘦的十分狼狈了。但见他——

\[
面如金纸,体似银条。看看减褪丰标,渐渐消磨精彩。隐隐耳虚闻磐响,昏昏眼暗觉萤飞。六脉细沉,一灵缥缈,丧门吊客已临身,扁鹊卢医难下手。
\]

何老人看了脉息,出到厅上,向西门庆、乔大户说道:“这位娘子,乃是精冲了血管起,然后着了气恼。气与血相搏,则血如崩。不知当初起病之由是也不是?”西门庆道:“是便是,却如何治疗?”正论间,忽报:“琴童和王经请了赵先生来了。”何老人便问:“是何人?”西门庆道:“也是伙计举来一医者,你老人家只推不知,待他看了脉息,你老人家和他讲一讲,好下药。”不一时,赵大医从外而入,西门庆与他叙礼毕,然后与众人相见。何、乔二老居中,让他在左,伯爵在右,西门庆主位相陪。吃了茶,赵太医便问:“列位尊长贵姓?”乔大户道:“俺二人一姓何,一姓乔。”伯爵道:“在下姓应。老先想就是赵龙岗先生了。”赵太医答道:“龙岗是贱号。在下以医为业,家祖见为太医院院判,家父见充汝府良医,祖传三辈,习学医术。每日攻习王叔和、东垣勿听子《药性赋》、《黄帝素问》、《难经》、《活人书》、《丹溪纂要》、《丹溪心法》、《洁古老脉诀》、《加减十三方》、《千金奇效良方》、《寿域神方》、《海上方》,无书不读。药用胸中活法,脉明指下玄机。六气四时,辨阴阳之标格;七表八里,定关格之沉浮。风虚寒热之症候,一览无余;弦洪芤石之脉理,莫不通晓。小人拙口钝吻,不能细陈。”何老人听了,道:“敢问看病当以何者为先?”赵太医道:“古人云,望闻问切,神圣功巧。学生先问病,后看脉,还要观其气色。就如子平兼五星一般,才看得准,庶乎不差。”何老人道:“既是如此,请先生进去看看。”西门庆即令琴童:“后边说去,又请了赵先生来了。”

不一时,西门庆陪他进入李瓶儿房中。那李瓶儿方才睡下安逸一回,又搊扶起来,靠着枕褥坐着。这赵太医先诊其左手,次诊右手,便教:“老夫人抬起头来,看看气色。”那李瓶儿真个把头儿扬起来。赵太医教西门庆:“老爹,你问声老夫人,我是谁?”西门庆便教李瓶儿:“你看这位是谁?”那李瓶儿抬头看了一眼,便低声说道:“他敢是太医?”赵先生道:“老爹,不妨事,还认的人哩。”西门庆道:“赵先生,你用心看,我重谢你。”一面看视了半日,说道:“老夫人此病,休怪我说,据看其面色,又诊其脉息,非伤寒,只为杂症,不是产后,定然胎前。”西门庆道:“不是此疾。先生你再仔细诊一诊。”赵先生又沉吟了半晌道:“如此面色这等黄,多管是脾虚泄泻,再不然定是经水不调。”西门庆道:“实说与先生,房下如此这般,下边月水淋漓不止,所以身上都瘦弱了。有甚急方妙药,我重重谢你。”赵先生道:“如何?我就说是经水不调。不打紧处,小人有药。”

西门庆一面同他来到前厅,乔大户、何老人问他甚么病源,赵先生道:“依小人讲,只是经水淋漓。”何老人道:“当用何药治之?”赵先生道:“我有一妙方,用着这几味药材,吃下去管情就好。听我说:

\[
甘草甘遂与碙砂,黎芦巴豆与芫花,姜汁调着生半夏,用乌头杏仁天麻。这几味儿齐加,葱蜜和丸只一挝,清晨用烧酒送下。”
\]
何老人听了,便道:“这等药恐怕太狠毒,吃不得。”赵先生道:“自古毒药苦口利于病。怎么吃不得?”西门庆见他满口胡说,因是韩伙计举保来,不好嚣他,称二钱银子,也不送,就打发他去了。因向乔大户说:“此人原来不知甚么。”何老人道:“老拙适才不敢说,此人东门外有名的赵捣鬼,专一在街上卖杖摇铃,哄过往之人,他那里晓的甚脉息病源!”因说:“老夫人此疾,老拙到家撮两帖药来,遇缘,若服毕经水少减,胸口稍开,就好用药。只怕下边不止,就难为矣。”说毕,起身。

西门庆封白金一两,使玳安拿盒儿讨将药来,晚夕与李瓶儿吃了,并不见分毫动静。吴月娘道:“你也省可与他药吃。他饮食先阻住了,肚腹中有甚么儿,只是拿药淘碌他。前者,那吴神仙算他三九上有血光之灾,今年却不整二十七岁了。你还使人寻这吴神仙去,叫替他打算算那禄马数上如何。只怕犯着甚么星辰,替他禳保禳保。”西门庆听了,旋差人拿帖儿往周守备府里问去。那里回说:“吴神仙云游之人,来去不定。但来,只在城南土地庙下。今岁从四月里,往武当山去了。要打数算命,真武庙外有个黄先生打的好数,一数只要三钱银子,不上人家门。”西门庆随即使陈敬济拿三钱银子,迳到北边真武庙门首黄先生家。门上贴着:“抄算先天易数,每命卦金三钱。”陈敬济向前作揖,奉上卦金,说道:“有一命烦先生推算。”写与他八字:女命,年二十七岁,正月十五日午时。这黄先生把算子一打,就说:“这个命,辛未年庚寅月辛卯日甲午时,理取印绥之格,借四岁行运。四岁己未,十四岁戊午,二十四岁丁巳,三十四岁丙辰。今年流年丁酉,比肩用事,岁伤日干,计都星照命,又犯丧门五鬼,灾杀作炒。夫计都者,阴晦之星也。其象犹如乱丝而无头,变异无常。大运逢之,多主暗昧之事,引惹疾病,主正、二、三、七、九月病灾有损,小口凶殃,小人所算,口舌是非,主失财物。或是阴人大为不利。”抄毕数,敬济拿来家。西门庆正和应伯爵、温秀才坐的,见抄了数来,拿到后边,解说与月娘听。见命中多凶少吉,不觉——

\[
眉间搭上三黄锁,腹内包藏一肚愁。
\]

\newpage
%# -*- coding:utf-8 -*-
%%%%%%%%%%%%%%%%%%%%%%%%%%%%%%%%%%%%%%%%%%%%%%%%%%%%%%%%%%%%%%%%%%%%%%%%%%%%%%%%%%%%%


\chapter{潘道士法遣黄巾士\KG 西门庆大哭李瓶儿}


诗曰:

\[
玉钗重合两无缘,鱼在深潭鹤在天。
得意紫鸾休舞镜,传言青鸟罢衔笺。
金盆已覆难收水,玉轸长笼不续弦。
若向蘼芜山下过,遥将红泪洒穷泉。
\]

话说西门庆见李瓶儿服药无效,求神问卜发课,皆有凶无吉,无法可处。初时,李瓶儿还\textuni{28D03}\textuni{499B}着梳头洗脸,下炕来坐净桶,次后渐渐饮食减少,形容消瘦,那消几时,把个花朵般人儿,瘦弱得黄叶相似,也不起炕了,只在床褥上铺垫草纸。恐怕人嫌秽恶,教丫头只烧着香。西门庆见他胳膊儿瘦得银条相似,只守着在房内哭泣,衙门中隔日去走一走。李瓶儿道:“我的哥,你还往衙门中去,只怕误了你公事。我不妨事,只吃下边流的亏,若得止住了,再把口里放开,吃些饮食儿,就好了。你男子汉,常绊在我房中做甚么!”西门庆哭道:“我的姐姐,我见你不好,心中舍不的你。”李瓶儿道:“好傻子,只不死,死将来你拦的住那些!”又道:“我有句话要对你说:我不知怎的,但没人在房里,心中只害怕,恰似影影绰绰有人在跟前一般。夜里要便梦见他,拿刀弄杖,和我厮嚷,孩子也在他怀里。我去夺,反被他推我一交,说他又买了房子,来缠了好几遍,只叫我去。只不好对你说。”西门庆听了说道:“人死如灯灭,这几年知道他往那里去了!此是你病的久,神虚气弱了,那里有甚么邪魔魍魉、家亲外祟!我如今往吴道官庙里,讨两道符来,贴在房门上,看有邪祟没有。”

说毕,走到前边,即差玳安骑头口往玉皇庙讨符去。走到路上,迎见应怕爵和谢希大,忙下头口。伯爵因问:“你往那里去?你爹在家里?”玳安道:“爹在家里,小的往玉皇庙讨符去。”伯爵与谢希大到西门庆家,因说道:“谢子纯听见嫂子不好,唬了一跳,敬来问安。”西门庆道:“这两日身上瘦的通不象模样了,丢的我上不上,下不下,却怎生样的?”伯爵道:“哥,你使玳安往庙里做甚么去?”西门庆悉把李瓶儿害怕之事告诉一遍:“只恐有邪祟,教小厮讨两道符来镇压镇压。”谢希大道:“哥,此是嫂子神气虚弱,那里有甚么邪祟!”伯爵道:“哥若遣邪也不难,门外五岳观潘道士,他受的是天心五雷法,极遣的好邪,有名唤着潘捉鬼,常将符水救人。哥,你差人请他来,看看嫂子房里有甚邪祟,他就知道。你就教他治病,他也治得。”西门庆道:“等讨了吴道官符来看,在那里住?没奈何,你就领小厮骑了头口,请了他来。”伯爵道:“不打紧,等我去。天可怜见嫂子好了,我就头着地也走。”说了一回话,伯爵和希大起身去了。

玳安儿讨了符来,贴在房中。晚间李瓶儿还害怕,对西门庆说:“死了的,他刚才和两个人来拿我,见你进来,躲出去了。”西门庆道:“你休信邪,不妨事。昨日应二哥说,此是你虚极了。他说门外五岳观有个潘道士,好符水治病,又遣的好邪,我明日早教应伯爵去请他来看你,有甚邪祟,教他遣遣。”李瓶儿道:“我的哥哥,你请他早早来,那厮他刚才发恨而去,明日还来拿我哩!你快些使人请去。”西门庆道:“你若害怕,我使小厮拿轿子接了吴银儿,和你做两日伴儿。”李瓶儿摇头儿说:“你不要叫他,只怕误了他家里勾当。”西门庆道:“叫老冯来伏侍你两日儿如何?”李瓶儿点头儿。这西门庆一面使来安,往那边房子里叫冯妈妈,又不在,锁了门出去了。对一丈青说下:“等他来,好歹教他快来宅内,六娘叫他哩。”西门庆一面又差下玳安:“明日早起,你和应二爹往门外五岳观请潘道士去。”俱不在话下。

次日,只见王姑子挎着一盒儿粳米、二十块大乳饼、一小盒儿十香瓜茄来看。李瓶儿见他来,连忙教迎春搊扶起来坐的。王姑子道了问讯,李瓶儿请他坐下,道:“王师父,你自印经时去了,影边儿通不见你。我恁不好,你就不来看我看儿?”王姑子道:“我的奶奶,我通不知你不好,昨日大娘使了大官儿到庵里,我才晓得。又说印经哩,你不知道,我和薛姑子老淫妇合了一场好气。与你老人家印了一场经,只替他赶了网儿。背地里和印经的打了五两银子夹帐,我通没见一个钱儿。你老人家作福,这老淫妇到明日堕阿鼻地狱!为他气的我不好了,把大娘的寿日都误了,没曾来。”李瓶儿道:“他各人作业,随他罢,你休与他争执了。”王姑子道:“谁和他争执甚么。”李瓶儿道:“大娘好不恼你哩,说你把他受生经都误了。”王姑子道:“我的菩萨,我虽不好,敢误了他的经?——在家整诵了一个月,昨日圆满了,今日才来。先到后边见了他,把我这些屈气告诉了他一遍。我说,不知他六娘不好,没甚么,这盒粳米和些十香爪、几块乳饼,与你老人家吃粥儿。大娘才叫小玉姐领我来看你老人家。”小玉打开盒儿,李瓶儿看了说道:“多谢你费心。”王姑子道:“迎春姐,你把这乳饼就蒸两块儿来,我亲看你娘吃些粥儿。”迎春一面收下去了。李瓶儿吩咐迎春:“摆茶来与王师父吃。”王姑子道:“我刚才后边大娘屋里吃了茶,煎些粥来,我看着你吃些。”

不一时,迎春安放桌儿,摆了四样茶食,打发王姑子吃了,然后拿上李瓶儿粥来,一碟十香甜酱瓜茄、一碟蒸的黄霜霜乳饼、两盏粳米粥,一双小牙筷。迎春拿着,奶子如意儿在旁拿着瓯儿,喂了半日,只呷了两三口粥儿,咬了一些乳饼儿,就摇头儿不吃了,教:“拿过去罢。”王姑子道:“人以水食为命,恁煎的好粥儿,你再吃些儿不是?”李瓶儿道:“也得我吃得下去是!”迎春便把吃茶的桌儿掇过去。王姑子揭开被,看李瓶儿身上,肌体都瘦的没了,唬了一跳,说道:“我的奶奶,我去时你好些了,如何又不好了,就瘦的恁样的了?”如意儿道:“可知好了哩!娘原是气恼上起的病,爹请了太医来看,每日服药,已是好到七八分了。只因八月内,哥儿着了惊唬不好,娘昼夜忧戚,那样劳碌,连睡也不得睡,实指望哥儿好了,不想没了。成日哭泣,又着了那暗气,暗恼在心里,就是铁石人也禁不的,怎的不把病又发了!是人家有些气恼儿,对人前分解分解也还好,娘又不出语,着紧问还不说哩。”王姑子道:“那讨气来?你爹又疼他,你大娘又敬他,左右是五六位娘,端的谁气着他?”奶子道:“王爷,你不知道——”因使绣春外边瞧瞧,看关着门不曾:“——俺娘都因为着了那边五娘一口气。——他那边猫挝了哥儿手,生生的唬出风来。爹来家,那等问着,娘只是不说。落后大娘说了,才把那猫来摔杀了。他还不承认,拿我每煞气。八月里,哥儿死了,他每日那边指桑树骂槐树,百般称快。俺娘这屋里分明听见,有个不恼的!左右背地里气,只是出眼泪。因此这样暗气暗恼,才致了这一场病。——天知道罢了!娘可是好性儿,好也在心里,歹也在心里,姊妹之间,自来没有个面红面赤。有件称心的衣裳,不等的别人有了,他还不穿出来。这一家子,那个不叨贴娘些儿?可是说的,饶叨贴了娘的,还背地不道是。”王姑子道:“怎的不道是?”如意儿道:“象五娘那边潘姥姥,来一遭,遇着爹在那边歇,就过来这屋里和娘做伴儿。临去,娘与他鞋面、衣服、银子,甚么不与他?五娘还不道是。”李瓶儿听见,便嗔如意儿:“你这老婆,平白只顾说他怎的?我已是死去的人了,随他罢了。天不言而自高,地不言而自厚。”王姑子道:“我的佛爷,谁如你老人家这等好心!天也有眼,望下看着哩。你老人家往后来还有好处。”李瓶儿道:“王师父,还有甚么好处!一个孩儿也存不住,去了。我如今又不得命,身底下弄这等疾,就是做鬼,走一步也不得个伶俐。我心里还要与王师父些银子儿,望你到明日我死了,你替我在家请几位师父,多诵些《血盆经》,忏忏我这罪业。”王姑子道:“我的菩萨,你老人家忒多虑了。你好心人,龙天自然加护。”正说着,只见琴童儿进来对迎春说:“爹吩咐把房内收拾收拾,花大舅便进来看娘,在前边坐着哩。”王姑子便起身说道:“我且往后边去走走。”李瓶儿道:“王师父,你休要去了,与我做两日伴儿,我还和你说话哩。”王姑子道:“我的奶奶,我不去。”

不一时,西门庆陪花大舅进来看问,见李瓶儿睡在炕上不言语,花子由道:“我不知道,昨日听见这边大官儿去说,才晓的。明日你嫂子来看你。”那李瓶儿只说了一声:“多有起动。”就把面朝里去了。花子由坐了一回,起身到前边,向西门庆说道:“俺过世老公公在广南镇守,带的那三七药,曾吃了不曾?不拘妇女甚崩漏之疾,用酒调五分末儿,吃下去即止。大姐他手里曾收下此药,何不服之?”西门庆道:“这药也吃过了。昨日本县胡大尹来拜,我因说起此疾,他也说了个方儿:棕炭与白鸡冠花煎酒服之。只止了一日,到第二日,流的比常更多了。”花子由道:“这个就难为了。姐夫,你早替他看下副板儿,预备他罢。明日教他嫂子来看他。”说毕,起身去了。

奶子与迎春正与李瓶儿垫草纸在身底下,只见冯妈妈来到,向前道了万福。如意儿道:“冯妈妈贵人,怎的不来看看娘?昨日爹使来安儿叫你去,说你锁着门,往那里去来?”冯婆子道:“说不得我这苦。成日往庙里修法,早晨出去了,是也直到黑,不是也直到黑来家,偏有那些张和尚、李和尚、王和尚。”如意儿道:“你老人家怎的有这些和尚?早时没王师父在这里?”那李瓶儿听了,微笑了一笑儿,说道:“这妈妈子,单管只撒风。”如意儿道:“冯妈妈,叫着你还不来!娘这几日,粥儿也不吃,只是心内不耐烦,你刚才来到,就引的娘笑了一笑儿。你老人家伏侍娘两日,管情娘这病就好了。”冯妈妈道:“我是你娘退灾的博士!”又笑了一回。因向被窝里摸了摸他身上,说道:“我的娘,你好些儿也罢了!”又问:“坐杩子还下的来?”迎春道:“下的来倒好!前两遭,娘还\textuni{28D03}\textuni{499B},俺每搊扶着下来。这两日通只在炕上铺垫草纸,一日两三遍。”

正说着,只见西门庆进来,看见冯妈妈,说道:“老冯,你也常来这边走走,怎的去了就不来?”婆子道:“我的爷,我怎不来?这两日腌菜的时候,挣两个钱儿,腌些菜在屋里,遇着人家领来的业障,好与他吃。不然,我那讨闲钱买菜来与他吃?”西门庆道:“你不对我说,昨日俺庄子上起菜,拨两三畦与你也够了。”婆子道:“又敢缠你老人家。”说毕,过那边屋里去了。

西门庆便坐在炕沿上,迎春在旁熏爇芸香。西门庆便问:“你今日心里觉怎样?”又问迎春:“你娘早晨吃些粥儿不曾?”迎春道:“吃的倒好!王师父送了乳饼,蒸来,娘只咬了一些儿,呷了不上两口粥汤,就丢下了。”西门庆道:“应二哥刚才和小厮门外请那潘道士,又不在了。明日我教来保再请去。”李瓶儿道:“你上紧着人请去,那厮,但合上眼,只在我跟前缠。”西门庆道:“此是你神弱了,只把心放正着,休要疑影他。请他来替你把这邪崇遣遣,再服他些药,管情你就好了。”李瓶儿道:“我的哥哥,奴已是得了这个拙病,那里好甚么!奴指望在你身边团圆几年,也是做夫妻一场,谁知到今二十七岁,先把冤家死了,奴又没造化,这般不得命,抛闪了你去。若得再和你相逢,只除非在鬼门关上罢了。”说着,一把拉着西门庆手,两眼落泪,哽哽咽咽,再哭不出声来。那西门庆又悲恸不胜,哭道:“我的姐姐,你有甚话,只顾说。”两个正在屋里哭,忽见琴童儿进来,说:“答应的禀爹,明日十五,衙门里拜牌,画公座,大发放,爹去不去?班头好伺候。”西门庆道:“我明日不得去,拿帖儿回了夏老爹,自己拜了牌罢。”琴童应诺去了。李瓶儿道:“我的哥哥,你依我还往衙门去,休要误了公事。我知道几时死,还早哩!”西门庆道:“我在家守你两日儿,其心安忍!你把心来放开,不要只管多虑了。刚才花大舅和我说,教我早与你看下副寿木,冲你冲,管情你就好了。”李瓶儿点头儿,便道:“也罢,你休要信着人使那憨钱,将就使十来两银子,买副熟料材儿,把我埋在先头大娘坟旁,只休把我烧化了,就是夫妻之情。早晚我就抢些浆水,也方便些。你偌多人口,往后还要过日子哩!”西门庆不听便罢,听了如刀剜肝胆、剑锉身心相似。哭道:“我的姐姐,你说的是那里话!我西门庆就穷死了,也不肯亏负了你!”

正说着,只见月娘亲自拿着一小盒儿鲜苹菠进来,说道:“李大姐,他大妗子那里送苹菠儿来你吃。”因令迎春:“你洗净了,拿刀儿切块来你娘吃。”李瓶儿道:“又多谢他大妗子挂心。”不一时,迎春旋去皮儿,切了,用瓯儿盛贮,拈了一块,与他放在口内,只嚼了些味儿,还吐出来了。月娘恐怕劳碌他,安顿他面朝里就睡了。

西门庆与月娘都出外边商议。月娘道:“李大姐,我看他有些沉重,你须早早与他看一副材板儿,省得到临时马捉老鼠,又乱不出好板来。”西门庆道:“今日花大哥也是这般说。适才我略与他题了题儿,他吩咐:‘休要使多了钱,将就抬副熟板儿罢。你偌多人口,往后还要过日子。’倒把我伤心了这一会。我说亦发等请潘道士来看了,看板去罢。”月娘道:“你看没分晓,一个人形也脱了,关口都锁住,勺水也不进,还指望好!咱一壁打鼓,一壁磨旗。幸的他好了,把棺材就舍与人,也不值甚么。”西门庆道:“既是恁说……”就出到厅上,叫将贲四来,问他:“谁家有好材板,你和姐夫两个拿银子看一副来。”贲四道:“大街上陈千户家,新到了几副好板。”西门庆道:“既有好板,”即令陈敬济:“你后边问你娘要五锭大银子来,你两个看去。”那陈敬济忙进去取了五锭元宝出来,同贲四去了。直到后晌才来回话,说:“到陈千户家看了几副板,都中等,又价钱不合。回来路上,撞见乔亲家爹,说尚举人家有一副好板——原是尚举人父亲在四川成都府做推官时,带来预备他老夫人的两副桃花洞,他使了一副,只剩下这一副——墙磕、底盖、堵头俱全,共大小五块,定要三百七十两银子。乔亲家爹同俺每过去看了,板是无比的好板。乔亲家与做举人的讲了半日,只退了五十两银子。不是明年上京会试用这几两银子,他也还舍不得卖哩。”西门庆道:“既是你乔亲家爹主张,兑三百二十两抬了来罢,休要只顾摇铃打鼓的。”陈敬济道:“他那里收了咱二百五十两,还找与他七十两银子就是了。”一面问月娘又要出七十两银子,二人去了。

比及黄昏时分,只见几个闲汉,用大红毡条裹着,抬板进门,放在前厅天井内。打开,西门庆观看,果然好板。随即叫匠人来锯开,里面喷香。每块五寸厚,二尺五寸宽,七尺五寸长。看了满心欢喜。又旋寻了伯爵到来看,因说:“这板也看得过了。”伯爵喝采不已,说道,“原说是姻缘板,大抵一物必有一主。嫂子嫁哥一场,今日情受这副材板够了。”吩咐匠人:“你用心只要做的好,你老爹赏你五两银子。”匠人道:“小人知道。”一面在前厅七手八脚,连夜攒造。伯爵嘱来保:“明日早五更去请潘道士,他若来,就同他一答儿来,不可迟滞。”说毕,陪西门庆在前厅看着做材,到一更时分才家去。西门庆道:“明日早些来,只怕潘道士来的早。”伯爵道:“我知道。”作辞出门去了。

却说老冯与王姑子,晚夕都在李瓶儿屋里相伴。只见西门庆前边散了,进来看视,要在屋里睡。李瓶儿不肯,说道:“没的这屋里龌龌龊龊的,他每都在这里,不方便,你往别处睡去罢。”西门庆又见王姑子都在这里,遂过那边金莲房里去了。

李瓶儿教迎春把角门关了,上了拴,教迎春点着灯,打开箱子,取出几件衣服、银首饰来,放在旁边。先叫过王姑子来,与了他五两一锭银子、一匹绸子:“等我死后,你好歹请几位师父,与我诵《血盆经忏》。”王姑子道:“我的奶奶,你忒多虑了。天可怜见,你只怕好了。”李瓶儿道:“你只收着,不要对大娘说我与你银子,只说我与了你这匹绸子做经钱。”王姑子道,“我知道。”于是把银子和绸子收了。又唤过冯妈妈来,向枕头边也拿过四两银子、一件白绫袄、黄绫裙、一根银掠儿,递与他,说道:“老冯,你是个旧人,我从小儿,你跟我到如今。我如今死了去,也没甚么,这一套衣服并这件首饰儿,与你做一念儿。这银子你收着,到明日做个棺材本儿。你放心,那边房子,等我对你爹说,你只顾住着,只当替他看房儿,他莫不就撵你不成!”冯妈妈一手接了银子和衣服,倒身下拜,哭着说道:“老身没造化了。有你老人家在一日,与老身做一日主儿。你老人家若有些好歹,那里归着?”李瓶儿又叫过奶子如意儿,与了他一袭紫绸子袄儿、蓝绸裙、一件旧绫披袄儿、两根金头簪子、一件银满冠儿,说道:“也是你奶哥儿一场。哥儿死了,我原说的,教你休撅上奶去,实指望我在一日,占用你一日,不想我又死去了。我还对你爹和你大娘说,到明日我死了,你大娘生了哥儿,就教接你的奶儿罢。这些衣服,与你做一念儿,你休要抱怨。”那奶子跪在地下,磕着头哭道:“小媳妇实指望伏侍娘到头,娘自来没曾大气儿呵着小媳妇。还是小媳妇没造化,哥儿死了,娘又病的这般不得命。好歹对大娘说,小媳妇男子汉又没了,死活只在爹娘这里答应了,出去投奔那里?”说毕,接了衣服首饰,磕了头起来,立在旁边,只顾揩眼泪。李瓶儿一面叫过迎春、绣春来跪下,嘱咐道:“你两个,也是你从小儿在我手里答应一场,我今死去,也顾不得你每了。你每衣服都是有的,不消与你了。我每人与你这两对金裹头簪儿、两枝金花儿做一念儿。大丫头迎春,已是他爹收用过的,出不去了,我教与你大娘房里拘管。这小丫头绣春,我教你大娘寻家儿人家,你出身去罢。省的观眉说眼,在这屋里教人骂没主子的奴才。我死了,就见出样儿来了。你伏侍别人,还象在我手里那等撤娇撒痴,好也罢,歹也罢了,谁人容的你?”那绣春跪在地下哭道:“我娘,我就死也不出这个门。”李瓶儿道:“你看傻丫头,我死了,你在这屋里伏侍谁?”绣春道:“我守着娘的灵。”李瓶儿道:“就是我的灵,供养不久,也有个烧的日子,你少不的也还出去。”绣春道:“我和迎春都答应大娘。”李瓶儿道:“这个也罢了。”这绣春还不知甚么,那迎春听见李瓶儿嘱咐他,接了首饰,一面哭的言语都说不出来。正是:

\[
流泪眼观流泪眼,断肠人送断肠人。
\]

当夜,李瓶儿都把各人嘱咐了。到天明,西门庆走进房来。李瓶儿问:“买了我的棺材来了没有?”西门庆道:“昨日就抬了板来,在前边做哩。——且冲冲你,你若好了,情愿舍与人罢。”李瓶儿因问:“是多少银子买的?休要使那枉钱。”西门庆道:“没多,只百十两来银子。”李瓶儿道:“也还多了。预备下,与我放着。”西门庆说了回出来,前边看着做材去了。吴月娘和李娇儿先进房来,看见他十分沉重,便问道:“李大姐,你心里却怎样的?”李瓶儿攥着月娘手哭道:“大娘,我好不成了。”月娘亦哭道:“李大姐,你有甚么话儿,二娘也在这里,你和俺两个说。”李瓶儿道:“奴有甚话儿——奴与娘做姊妹这几年,又没曾亏了我,实承望和娘相守到白头,不想我的命苦,先把个冤家没了,如今不幸,我又得了这个拙病死去了。我死之后,房里这两个丫头无人收拘。那大丫头已是他爹收用过的,教他往娘房里伏侍娘。小丫头,娘若要使唤,留下;不然,寻个单夫独妻,与小人家做媳妇儿去罢,省得教人骂没主子的奴才。也是他伏侍奴一场,奴就死,口眼也闭。奶子如意儿,再三不肯出去,大娘也看奴分上,也是他奶孩儿一场,明日娘生下哥儿,就教接他奶儿罢。”月娘说道:“李大姐,你放宽心,都在俺两个身上。说凶得吉,若有些山高水低,迎春教他伏侍我,绣春教他伏侍二娘罢。如今二娘房里丫头不老实做活,早晚要打发出去,教绣春伏侍他罢。奶子如意儿,既是你说他没投奔,咱家那里占用不下他来?就是我有孩子没孩子,到明日配上个小厮,与他做房家人媳妇也罢了。”李娇儿在旁便道:“李大姐,你休只要顾虑,一切事都在俺两个身上。绣春到明日过了你的事,我收拾房内伏侍我,等我抬举他就是了。”李瓶儿一面叫奶子和两个丫头过来,与二人磕头。那月娘由不得眼泪出。

不一时,盂玉楼、潘金莲、孙雪娥都进来看他,李瓶儿都留了几句姊妹仁义之言。落后待的李娇儿、玉楼、金莲众人都出去了,独月娘在屋里守着他,李瓶儿悄悄向月娘哭泣道:“娘到明日好生看养着,与他爹做个根蒂儿,休要似奴粗心,吃人暗算了。”月娘道:“姐姐,我知道。”看官听说:只这一句话,就感触目娘的心来。后次西门庆死了,金莲就在家中住不牢者,就是想着李瓶儿临终这句话。正是:

\[
惟有感恩并积恨,千年万载不生尘。
\]

正说话间,只见琴童吩咐房中收拾焚下香,五岳观请了潘法官来了。月娘一面看着,教丫头收拾房中干净,伺候净茶净水,焚下百合真香。月娘与众妇女都藏在那边床屋里听观。不一时,只见西门庆领了那潘道士进来。怎生形相?但见:

\[
头戴云霞五岳冠,身穿皂布短褐袍,腰系杂色彩丝绦,背插横纹古铜剑。两只脚穿双耳麻鞋,手执五明降鬼扇。八字眉,两个杏子眼;四方口,一道落腮胡。威仪凛凛,相貌堂堂。若非霞外云游客,定是蓬莱玉府人。
\]
潘道士进入角门,刚转过影壁,将走到李瓶儿房穿廊台基下,那道士往后退讫两步,似有呵叱之状,尔语数四,方才左右揭帘进入房中,向病榻而至。运双晴,拿力以慧通神目一视,仗剑手内,掐指步罡,念念有辞,早知其意。走出明间,朝外设下香案。西门庆焚了香,这潘道士焚符,喝道:“值日神将,不来等甚?”噀了一口法水去,忽阶下卷起一阵狂风,仿佛似有神将现于面前一般。潘道士便道:“西门氏门中,有李氏阴人不安,投告于我案下。汝即与我拘当坊土地、本家六神查考,有何邪祟,即与我擒来,毋得迟滞!”良久,只见潘道士瞑目变神,端坐于位上,据案击令牌,恰似问事之状,良久乃止。出来,西门庆让至前边卷棚内,问其所以,潘道士便说:“此位娘子,惜乎为宿世冤愆诉于阴曹,非邪祟也,不可擒之。”西门庆道:“法官可解禳得么?”潘道士道:“冤家债主,须得本人,虽阴官亦不能强。”因见西门庆礼貌虔切,便问:“娘于年命若干?”西门庆道:“属羊的,二十七岁。”潘道士道:“也罢,等我与他祭祭本命星坛,看他命灯如何。”西门庆问:“几时祭?用何香纸祭物?”潘道士道:“就是今晚三更正子时,用白灰界画,建立灯坛,以黄绢围之,镇以生辰坛斗,祭以五谷枣汤,不用酒脯,只用本命灯二十七盏,上浮以华盖之仪,余无他物,官人可斋戒青衣,坛内俯伏行礼,贫道祭之,鸡犬皆关去,不可入来打搅。”西门庆听了,忙吩咐一一备办停当。就不敢进去,只在书房中沐浴斋戒,换了净衣。留应伯爵也不家去了,陪潘道士吃斋馔。

到三更天气,建立灯坛完备,潘道士高坐在上。下面就是灯坛,按青龙、白虎、朱雀、玄武,上建三台华盖;周列十二宫辰,下首才是本命灯,共合二十七盏。先宣念了投词。西门庆穿青衣俯伏阶下,左右尽皆屏去,不许一人在左右。灯烛荧煌,一齐点将起来。那潘道士在法座上披下发来,仗剑,口中念念有词。望天罡,取真气,布步玦,蹑瑶坛。正是:三信焚香三界合,一声令下一声雷。但见晴天月明星灿,忽然地黑天昏,起一阵怪风。正是:

\[
非干虎啸,岂是龙吟?仿佛入户穿帘,定是催花落叶。推云出岫,送雨归川。雁迷失伴作哀鸣,鸥鹭惊群寻树杪。姮娥急把蟾宫闭,列子空中叫救人。
\]
大风所过三次,忽一阵冷气来,把李瓶儿二十七盏本命灯尽皆刮灭。潘道士明明在法座上见一个白衣人领着两个青衣人,从外进来,手里持着一纸文书,呈在法案下。潘道士观看,却是地府勾批,上面有三颗印信,唬的慌忙下法座来,向前唤起西门庆来,如此这般,说道:“官人请起来罢!娘子已是获罪于天,无所祷也!本命灯已灭,岂可复救乎?只在旦夕之间而已。”那西门庆听了,低首无语,满眼落泪,哀告道:“万望法师搭救则个!”潘道士道:“定数难逃,不能搭救了。”就要告辞。西门庆再三款留:“等天明早行罢!”潘道士道:“出家人草行露宿,山栖庙止,自然之道。”西门庆不复强之。因令左右取出布一匹、白金三两作经衬钱。潘道士道:“贫道奉行皇天至道,对天盟誓,不敢贪受世财,取罪不便。”推让再四,只令小童收了布匹,作道袍穿,就作辞而行。嘱咐西门庆:“今晚,官人切忌不可往病人房里去,恐祸及汝身。慎之!慎之!”言毕,送出大门,拂袖而去。

西门庆归到卷棚内,看着收拾灯坛。见没救星,心中甚恸,向伯爵,不觉眼泪出。伯爵道:“此乃各人禀的寿数,到此地位,强求不得。哥也少要烦恼。”因打四更时分,说道:“哥,你也辛苦了,安歇安歇罢。我且家去,明日再来。”西门庆道:“教小厮拿灯笼送你去。”即令来安取了灯送伯爵出去,关上门进来。

那西门庆独自一个坐在书房内,掌着一枝蜡烛,心中哀恸,口里只长吁气,寻思道:“法官教我休往房里去,我怎生忍得!宁可我死了也罢。须厮守着和他说句话儿。”于是进入房中。见李瓶儿面朝里睡,听见西门庆进来,翻过身来便道:“我的哥哥,你怎的就不进来了?”因问:“那道士点得灯怎么说?”西门庆道:“你放心,灯上不妨事。”李瓶儿道:“我的哥哥,你还哄我哩,刚才那厮领着两个人又来,在我跟前闹了一回,说道:‘你请法师来遣我,我已告准在阴司,决不容你!’发恨而去,明日便来拿我也。”西门庆听了,两泪交流,放声大哭道:“我的姐姐,你把心来放正着,休要理他。我实指望和你相伴几日,谁知你又抛闪了我去了。宁教我西门庆口眼闭了,倒也没这等割肚牵肠。”那李瓶儿双手搂抱着西门庆脖子,呜呜咽咽悲哭,半日哭不出声。说道:“我的哥哥,奴承望和你白头相守,谁知奴今日死去也。趁奴不闭眼,我和你说几句话儿:你家事大,孤身无靠,又没帮手,凡事斟酌,休要一冲性儿。大娘等,你也少要亏了他。他身上不方便,早晚替你生下个根绊儿,庶不散了你家事。你又居着个官,今后也少要往那里去吃酒,早些儿来家,你家事要紧。比不的有奴在,还早晚劝你。奴若死了,谁肯苦口说你?”西门庆听了,如刀剜心肝相似,哭道:“我的姐姐,你所言我知道,你休挂虑我了。我西门庆那世里绝缘短幸,今世里与你做夫妻不到头。疼杀我也!天杀我也!”李瓶儿又吩咐迎春、绣春之事:“奴已和他大娘说来,到明日我死,把迎春伏侍他大娘;那小丫头,他二娘已承揽。——他房内无人,便教伏侍二娘罢。”西门庆道:“我的姐姐,你没的说,你死了,谁人敢分散你丫头!奶子也不打发他出去,都教他守你的灵。”李瓶儿道:“甚么灵!回个神主子,过五七烧了罢了。”西门庆道:“我的姐姐,你不要管他,有我西门庆在一日,供养你一日。”两个说话之间,李瓶儿催促道:“你睡去罢,这咱晚了。”西门庆道:“我不睡了,在这屋里守你守儿。”李瓶儿道:“我死还早哩,这屋里秽污,熏的你慌,他每伏侍我不方便。”

西门庆不得已,吩咐丫头:“仔细看守你娘。”往后边上房里,对月娘悉把祭灯不济之事告诉一遍:“刚才我到他房中,我观他说话儿还伶俐。天可怜,只怕还熬出来也不见得。”月娘道:“眼眶儿也塌了,嘴唇儿也干了,耳轮儿也焦了,还好甚么!也只在早晚间了。他这个病是恁伶俐,临断气还说话儿。”西门庆道:“他来了咱家这几年,大大小小,没曾惹了一个人,且是又好个性格儿,又不出语,你教我舍的他那些儿!”题起来又哭了。月娘亦止不住落泪。

不说西门庆与月娘说话,且说李瓶儿唤迎春、奶子:“你扶我面朝里略倒倒儿。”因问道:“有多咱时分了?”奶子道:“鸡还未叫,有四更天了。”叫迎春替他铺垫了身底下草纸,搊他朝里,盖被停当,睡了。众人都熬了一夜没曾睡,老冯与王姑子都已先睡了。迎春与绣春在面前地坪上搭着铺,刚睡倒没半个时辰,正在睡思昏沉之际,梦见李瓶儿下炕来,推了迎春一推,嘱咐:“你每看家,我去也。”忽然惊醒,见桌上灯尚未灭。忙向床上视之,还面朝里,摸了摸,口内已无气矣。不知多咱时分呜呼哀哉,断气身亡。可怜一个美色佳人,都化作一场春梦。正是:

\[
阎王教你三更死,怎敢留人到五更!
\]

迎春慌忙推醒众人,点灯来照,果然没了气儿,身底下流血一洼,慌了手脚,忙走去后边,报知西门庆。西门庆听见李瓶儿死了,和吴月娘两步做一步奔到前边,揭起被,但见面容不改,体尚微温,悠然而逝,身上止着一件红绫抹胸儿。西门庆也不顾甚么身底下血渍,两只手捧着他香腮亲着,口口声声只叫:“我的没救的姐姐,有仁义好性儿的姐姐!你怎的闪了我去了?宁可教我西门庆死了罢。我也不久活于世了,平白活着做甚么!”在房里离地跳的有三尺高,大放声号哭。吴月娘亦揾泪哭涕不止。落后,李娇儿、孟玉楼、潘金莲、孙雪娥、合家大小丫头养娘都哭起来,哀声动地。月娘向众人道:“不知多咱死的,恰好衣服儿也不曾穿一件在身上。”玉楼道:“我摸他身上还温温儿的,也才去了不多回儿。咱趁热脚儿不替他穿上衣裳,还等甚么?”月娘见西门庆磕伏在他身上,挝脸儿那等哭,只叫:“天杀了我西门庆了!姐姐你在我家三年光景,一日好日子没过,都是我坑陷了你了!”月娘听了,心中就有些不耐烦了,说道:“你看韶刀!哭两声儿,丢开手罢了。一个死人身上,也没个忌讳,就脸挝着脸儿哭,倘或口里恶气扑着你是的!他没过好日子,谁过好日子来?各人寿数到了,谁留的住他!那个不打这条路儿来?”因令李娇儿、孟玉楼:“你两个拿钥匙,那边屋里寻他几件衣服出来,咱每眼看着与他穿上。”又叫:“六姐,咱两个把这头来替他整理整理。”西门庆又向月娘说:“多寻出两套他心爱的好衣服,与他穿了去。”月娘吩咐李娇儿、玉楼:“你寻他新裁的大红缎遍地锦袄儿、柳黄遍地锦裙,并他今年乔亲家去那套丁香色云绸妆花衫、翠蓝宽拖子裙,并新做的白绫袄、黄绸子裙出来罢。”

当下迎春拿着灯,孟玉楼拿钥匙,走到那边屋里,开了箱子,寻了半日,寻出三套衣裳来,又寻出一件衬身紫绫小袄儿、一件白绸子裙、一件大红小衣儿并白绫女袜儿、妆花膝裤腿儿。李娇儿抱过这边屋里与月娘瞧。月娘正与金莲灯下替他整理头髻,用四根金簪儿绾一方大鸦青手帕,旋勒停当。李娇儿因问:“寻双甚么颜色鞋,与他穿了去?”潘金莲道:“姐姐,他心爱穿那双大红遍地金高底鞋儿,只穿了没多两遭儿,倒寻出来与他穿去罢。”吴月娘道:“不好,倒没的穿到阴司里,教他跳火坑。你把前日往他嫂子家去穿的那双紫罗遍地金高底鞋,与他装绑了去罢。”李娇儿听了,忙叫迎春寻出来。众人七手八脚,都装绑停当。

西门庆率领众小厮,在大厅上收卷书画,围上帏屏,把李瓶儿用板门抬出,停于正寝。下铺锦褥,上覆纸被,安放几筵香案,点起一盏随身灯来。专委两个小厮在旁侍奉:一个打磐,一个炷纸,一面使玳安:“快请阴阳徐先生来看时批书。”月娘打点出装绑衣服来,就把李瓶儿床房门锁了,只留炕屋里,交付与丫头养娘。冯妈妈见没了主儿,哭的三个鼻头两行眼泪,王姑子且口里喃喃呐呐,替李瓶儿念《密多心经》、《药师经》、《解冤经》、《楞严经》并《大悲中道神咒》,请引路王菩萨与他接引冥途。西门庆在前厅,手拍着胸膛,抚尸大恸,哭了又哭,把声都哭哑了。口口声声只叫:“我的好性儿有仁义的姐姐。”

比及乱着,鸡就叫了。玳安请了徐先生来,向西门庆施礼,说道:“老爹烦恼,奶奶没了在于甚时候?”西门庆道:“因此时候不真:睡下之时,已可四更,房中人都困倦睡熟了,不知多咱时候没了。”徐先生道:“不打紧。”因令左右掌起灯来,揭开纸被观看,手掐丑更,说道:“正当五更二点辙,还属丑时断气。”西门庆即令取笔砚,请徐先生批书。徐先生向灯下问了姓氏并生辰八字,批将下来:“一故锦衣西门夫人李氏之丧。生于元祐辛未正月十五日午时,卒于政和丁酉九月十六日丑时。今日丙子,月令戊戌,犯天地往亡,煞高一丈,本家忌哭声,成服后无妨。入殓之时,忌龙、虎、鸡、蛇四生人,亲人不避。”吴月娘使出玳安来:“叫徐先生看看黑书上,往那方去了。”徐先生一面打开阴阳秘书观看,说道:“今乃丙子日,已丑时,死者上应宝瓶宫,下临齐地。前生曾在滨州王家作男子,打死怀胎母羊,今世为女人,属羊。虽招贵夫,常有疾病,比肩不和,生子夭亡,主生气疾而死。前九日魂去,托生河南汴梁开封府袁家为女,艰难不能度日。后耽阁至二十岁嫁一富家,老少不对,终年享福,寿至四十二岁,得气而终。”看毕黑书,众妇女听了,皆各叹息。西门庆就叫徐先生看破土安葬日期。徐先生请问:“老爹,停放几时?”西门庆哭道:“热突突怎么就打发出去的,须放过五七才好。”徐先生道:“五七内没有安葬日期,倒是四七内,宜择十月初八日丁酉午时破土,十二日辛丑未时安葬,合家六位本命都不犯。”西门庆道:“也罢,到十月十二日发引,再没那移了。”徐先生写了殃榜,盖伏死者身上,向西门庆道:“十九日辰时大殓,一应之物,老爹这里备下。”

刚打发徐先生出了门,天已发晓。西门庆使琴童儿骑头口,往门外请花大舅,然后分班差人各亲眷处报丧。又使人往衙门中给假,又使玳安往狮子街取了二十桶瀼纱漂白、三十桶生眼布来,叫赵裁雇了许多裁缝,在西厢房先造帷幕、帐子、桌围,并入殓衣衾缠带、各房里女人衫裙,外边小厮伴当,每人都是白唐巾,一件白直裰。又兑了一百两银子,教贲四往门外店里买了三十桶魁光麻布、二百匹黄丝孝绢,一面又教搭彩匠,在天井内搭五间大棚。西门庆因思想李瓶儿动止行藏模样,忽然想起忘了与他传神,叫过来保来问:“那里有好画师?寻一个来传神。我就把这件事忘了。”来保道:“旧时与咱家画围屏的韩先儿,他原是宣和殿上的画士,革退来家,他传的好神。”西门庆道:“他在那里住?快与我请来。”来保应诺去了。

西门庆熬了一夜没睡的人,前后又乱了一五更,心中又着了悲恸,神思恍乱,只是没好气,骂丫头、踢小厮,守着李瓶儿尸首,由不的放声哭叫。那玳安在旁,亦哭的言不的语不的。吴月娘正和李娇儿、孟玉楼、潘金莲在帐子后,打伙儿分孝与各房里丫头并家人媳妇,看见西门庆哑着喉咙只顾哭,问他,茶也不吃,只顾没好气。月娘便道:“你看恁劳叨!死也死了,你没的哭的他活?只顾扯长绊儿哭起来了。三两夜没睡,头也没梳,脸也没洗,乱了恁五更,黄汤辣水还没尝着,就是铁人也禁不的。把头梳了,出来吃些甚么,还有个主张。好小身子,一时摔倒了,却怎样儿的!”玉楼道:“原来他还没梳头洗脸哩?”月娘道:“洗了脸倒好!我头里使小厮请他后边洗脸,他把小厮踢进来,谁再问他来!”金莲道:“你还没见,头里我倒好意说,他已死了,你恁般起来,把骨秃肉儿也没了。你在屋里吃些甚么儿,出去再乱也不迟。他倒把眼睁红了的,骂我:‘狗攮的淫妇,管你甚么事!’我如今整日不教狗攮,却教谁攮哩!——恁不合理的行货子。只说人和他合气。”月娘道:“热突突死了,怎么不疼?你就疼,也还放在心里,那里就这般显出来?人也死了,不管那有恶气没恶气,就口挝着口那等叫唤,不知甚么张致。他可可儿来三年没过一日好日子,镇日教他挑水挨磨来?”孟玉楼道:“李大姐倒也罢了,倒吃他爹恁三等九格的。”

正说着,只见陈敬济手里拿着九匹水光绢,说:“爹教娘每剪各房里手帕,剩下的与娘每做裙子。”月娘收了绢,便道:“姐夫,你去请你爹进来扒口子饭。这咱七八晌午,他茶水还没尝着哩。”敬济道:“我是不敢请他。头里小厮请他吃饭,差些没一脚踢杀了,我又惹他做甚么?”月娘道:“你不请他,等我另使人请他来吃饭。”良久,叫过玳安来说道:“你爹还没吃饭,哭这一日了。你拿上饭去,趁温先生在这里,陪他吃些儿。”玳安道:“请应二爹和谢爹去了。等他来时,娘这里使人拿饭上去,消不的他几句言语,管情爹就吃了。”吴月娘说道:“硶嘴的囚根子,你是你爹肚里蛔虫?俺每这几个老婆倒不如你了。你怎的知道他两个来才吃饭?”玳安道:“娘每不知,爹的好朋友,大小酒席儿,那遭少了他两个?爹三钱,他也是三钱;爹二星,他也是二星。爹随问怎的着了恼,只他到,略说两句话儿,爹就眉花眼笑的。”

说了一回,棋童儿请了应伯爵、谢希大二人来到。进门扑倒灵前地下,哭了半日,只哭“我那有仁义的嫂子”,被金莲和玉楼骂道:“贼油嘴的囚根子,俺每都是没仁义的?”二人哭毕,爬起来,西门庆与他回礼,两个又哭了,说道:“哥烦恼,烦恼。”一面让至厢房内,与温秀才叙礼坐下。先是伯爵问道:“嫂子是甚时候殁了?”西门庆道:“正丑时断气。”伯爵道:“我到家已是四更多了,房下问我,我说看阴骘,嫂子这病已在七八了。不想刚睡下就做了一梦,梦见哥使大官儿来请我,说家里吃庆官酒,教我急急来到。见哥穿着一身大红衣服,向袖中取出两根玉簪儿与我瞧,说一根折了。我瞧了半日,对哥说:‘可惜了,这折了是玉的,完全的倒是硝子石。’哥说两根都是玉的。我醒了,就知道此梦做的不好。房下见我只顾咂嘴,便问:‘你和谁说话?’我道:‘你不知,等我到天晓告诉你。’等到天明,只见大官儿到了,戴着白,教我只顾跌脚。果然哥有孝服。”西门庆道:“我昨夜也做了恁个梦,和你这个一样儿。梦见东京翟亲家那里寄送了六根簪儿,内有一根\textShiFou 折了。我说,可惜了。醒来正告诉房下,不想前边断了气。好不睁眼的天,撇的我真好苦!宁可教我西门庆死了,眼不见就罢了。到明日,一时半刻想起来,你教我怎不心疼!平时,我又没曾亏欠了人,天何今日夺吾所爱之甚也!——先是一个孩儿没了,今日他又长伸脚去了。我还活在世上做甚么?虽有钱过北斗,成何大用?”伯爵道:“哥,你这话就不是了。我这嫂子与你是那样夫妻,热突突死了,怎的不心疼?争奈你偌大家事,又居着前程,这一家大小,泰山也似靠着你。你若有好歹,怎么了得!就是这些嫂子,都没主儿。常言:一在三在,一亡三亡。哥,你聪明怜俐人,何消兄弟每说?就是嫂子他青春年少,你疼不过,越不过他的情,成了服,令僧道念几卷经,大发送,葬埋在坟里,哥的心也尽了,也是嫂子一场的事,再还要怎样的?哥,你且把心放开。”当时,被伯爵一席话,说的西门庆心地透彻,茅塞顿开,也不哭了。须臾,拿上茶来吃了,便唤玳安:“后边说去,看饭来,我和你应二爹、温师父、谢爹吃。”伯爵道:“哥原来还未吃饭哩?”西门庆道:“自你去了,乱了一夜,到如今谁尝甚么儿来。”伯爵道:“哥,你还不吃饭,这个就胡突了,常言道:‘宁可折本,休要饥损。’《孝经》上不说的:‘教民无以死伤生,毁不灭性。’死的自死了,存者还要过日子。哥要做个张主。”正是:

\[
数语拨开君子路,片言题醒梦中人。
\]


\newpage
%# -*- coding:utf-8 -*-
%%%%%%%%%%%%%%%%%%%%%%%%%%%%%%%%%%%%%%%%%%%%%%%%%%%%%%%%%%%%%%%%%%%%%%%%%%%%%%%%%%%%%


\chapter{韩画士传真作遗爱\KG 西门庆观戏动深悲}


诗曰:

\[
香杳美人违,遥遥有所思。幽明千里隔,风月两边时。
相对春那剧,相望景偏迟。当由分别久,梦来还自疑。
\]

话说西门庆被应伯爵劝解了一回,拭泪令小厮后边看饭去了。不一时,吴大舅、吴二舅都到了。灵前行礼毕,与西门庆作揖,道及烦恼之意。请至厢房中,与众人同坐。

玳安走至后边,向月娘说:“如何?我说娘每不信,怎的应二爹来了,一席话说的爹就吃饭了。”金莲道:“你这贼,积年久惯的囚根子,镇日在外边替他做牵头,有个拿不住他性儿的!”玳安道:“从小儿答应主子,不知心腹?”月娘问道:“那几个陪他吃饭?”玳安道:“大舅、二舅才来,和温师父,连应二爹、谢爹、韩伙计、姐夫,共爹八个人哩。”月娘道:“请你姐夫来后边吃罢了,也挤在上头!”玳安道:“姐夫坐下了。”月娘吩咐:“你和小厮往厨房里拿饭去。你另拿瓯儿粥与他吃,怕清早晨不吃饭。”玳安道:“再有谁?止我在家,都使出报丧、买东西,王经,又使他往张亲家爹那里借云板去了。”月娘道:“书童那奴才和你拿去是的,怕打了他纱帽展翅儿!”玳安道:“书童和画童两个在灵前,一个打磐,一个伺候焚香烧纸哩。春鸿,爹又使他跟贲四换绢去了——嫌绢不好,要换六钱一匹的破孝。”月娘道:“论起来,五钱的也罢,又巴巴儿换去!”又道:“你叫下画童儿那小奴才,和他快拿去,只顾还挨甚么!”玳安于是和画童两个,大盘大碗拿到前边,安放八仙桌席。众人正吃着饭,只见平安拿进手本来禀:“夏老爹差写字的,送了三班军卫来这里答应。”西门庆看了,吩咐:“讨三钱银子赏他。写期服生帖儿回你夏老爹:多谢了!”

一面吃毕饭,收了家伙。只见来保请的画师韩先生来到。西门庆与他行毕礼,说道:“烦先生揭白传个神子儿。”那韩先生道:“小人理会得。”吴大舅道:“动手迟了些,只怕面容改了。”韩先生道:“也不妨,就是揭白也传得。”正吃茶毕,忽见平安来报:“门外花大舅来了。”西门庆陪花子由灵前哭涕了一回,见毕礼数,与众人一处,因问:“甚么时侯?”西门庆道:“正丑时断气。临死还伶伶俐俐说话儿,刚睡下,丫头起来瞧,就没了气儿。”因见韩先生旁边小童拿着屏插,袖中取出描笔颜色来,花子由道:“姐夫如今要传个神子?”西门庆道:“我心里疼他,少不得留个影像儿,早晚看着,题念他题念儿。”一面吩咐后边堂客躲开,掀起帐子,领韩先生和花大舅众人到跟前。这韩先生揭起千秋幡,打一观看,见李瓶儿勒着鸦青手帕,虽故久病,其颜色如生,姿容不改,黄恹恹的,嘴唇儿红润可爱。那西门庆由不的掩泪而哭。来保与琴童在旁捧着屏插、颜色。韩先生一见就知道了。众人围着他求画,应伯爵便道:“先生,此是病容,平昔好时,还生的面容饱满,姿容秀丽。”韩先生道:“不须尊长吩咐,小人知道。敢问老爹:此位老夫人,前者五月初一日曾在岳庙里烧香,亲见一面,可是否?”西门庆道:“正是。那时还好哩。先生,你用心想着,传画一轴大影、一轴半身,灵前供养,我送先生一匹缎子、十两银子。”韩先生道:“老爹吩咐,小人无不用心。”须臾,描染出个半身来,端的玉貌幽花秀丽,肌肤嫩玉生香。拿与众人瞧,就是一幅美人图儿。西门庆看了,吩咐玳安:“拿与你娘每瞧瞧去,看好不好。有那些儿不是,说来好改。”

玳安拿到后边,向月娘道:“爹说叫娘每瞧瞧,六娘这影画得如何,那些儿不象,说出去教韩先生好改。”月娘道:“成精鼓捣,人也不知死到那里去了,又描起影来了。”潘金莲接说道:“那个是他的儿女?画下影,传下神,好替他磕头礼拜!到明日六个老婆死了,画六个影才好。”孟玉楼和李娇儿接过来观看,说道:“大娘,你来看,李大姐这影,倒象好时模样,打扮的鲜鲜的,只是嘴唇略扁了些。”月娘看了道:“这左边额头略低了些,他的眉角还弯些。亏这汉子,揭白怎的画来!”玳安道:“他在庙上曾见过六娘一面,刚才想着,就画到这等模样。”

少顷,只见王经进来说道:“娘每看了,就教拿出去。乔亲家爹来了,等乔亲家爹瞧哩。”玳安走到前边,向韩先生道:“里边说来,嘴唇略扁了些,左额角稍低些,眉还要略放弯些儿。”韩先生道:“这个不打紧。”随即取描笔改过了,呈与乔大户瞧。乔大户道:“亲家母这幅尊像,真画得好,只少了口气儿。”西门庆满心欢喜,一面递了三钟酒与韩先生,管待了酒饭,又教取出一匹尺头、十两白金与韩先生,教他:“先攒造出半身来,就要挂,大影,不误出殡就是了。俱要用大青大绿,冠袍齐整,绫裱牙轴。”韩先生道:“不必吩咐,小人知道。”领了银子,教小童拿着插屏,拜辞出门。乔大户与众人又看了一回做成的棺木,便道:“亲家母今已小殓罢了?”西门庆道:“如今仵作行人来就小殓。大殓还等到三日。”乔大户吃毕茶,就告辞去了。

不一时,仵作行人来伺候,纸札打卷,铺下衣衾,西门庆要亲与他开光明,强着陈敬济做孝子,与他抿了目,西门庆旋寻出一颗胡珠,安放在他口里。登时小殓停当,照前停放端正,合家大小哭了一场。来兴又早冥衣铺里,做了四座堆金沥粉捧盆巾盥栉毛女儿,一边两座摆下。灵前的彝炉商瓶、烛台香盒,教锡匠打造停当,摆在桌上,耀日争辉。又兑了十两银子,教银匠打了三副银爵盏。又与应伯爵定管丧礼簿籍:先兑了五百两银子、一百吊钱来,委付与韩伙计管帐;贲四与来兴儿管买办,兼管外厨房;应伯爵、谢希大、温秀才、甘伙计轮番陪待吊客;崔本专管付孝帐;来保管外库房;王经管酒房;春鸿与画童专管灵前伺候;平安与四名排军,单管人来打云板、捧香纸;又叫一个写字带领四名排军,在大门首记门簿,值念经日期,打伞挑幡幢。都派委已定,写了告示,贴在影壁上,各遵守去讫。只见皇庄上薛内相差人送了六十根杉条、三十条毛竹、三百领芦席、一百条麻绳,西门庆赏了来人五钱银子,拿期服生回帖儿打发去了。吩咐搭采匠把棚起脊搭大些,留两个门走,把影壁夹在中间,前厨房内还搭三间罩棚,大门首扎七间榜棚,请报恩寺十二众僧人先念倒头经,每日两个茶酒伺候茶水。

花大舅、吴二舅坐了一回,起身去了。西门庆交温秀才写孝帖儿,要刊去,令写“荆妇奄逝”,温秀才悄悄拿与应伯爵看,伯爵道:“这个礼上说不通。见有如今吴家嫂子在正室,如何使得?这一出去,不被人议论!就是吴大哥,心内也不自在。等我慢慢再与他讲,你且休要写着。”陪坐至晚,各散归家去了。

西门庆晚夕也不进后边去,就在李瓶儿灵旁装一张凉床,拿围屏围着,独自宿歇,止春鸿、书童儿近前伏侍。天明便往月娘房里梳洗,穿戴了白唐巾孝冠孝衣、白绒袜、白履鞋,絰带随身。

第二日清晨,夏提刑就来探丧吊问,慰其节哀。西门庆还礼毕,温秀才相陪,待茶而去。到门首,吩咐写字的:“好生答应,查有不到的排军,呈来衙门内惩治。”说毕,骑马去了。西门庆令温秀才发帖儿,差人请各亲眷,三日诵经,早来吃斋。后晌,铺排来收拾道场,悬挂佛像,不必细说。

那日,吴银儿打听得知,坐轿子来灵前哭泣上纸。到后边,月娘相接。吴银儿与月娘磕头,哭道:“六娘没了,我通一字不知,就没个人儿和我说声儿。可怜,伤感人也!”孟玉楼道:“你是他干女儿,他不好了这些时,你就不来看他看儿?”吴银儿道:“好三娘,我但知道,有个不来看的?说句假就死了!委实不知道。”月娘道:“你不来看你娘,他倒还挂牵着你,留下件东西儿,与你做一念儿,我替你收着哩。”因令小玉:“你取出来与银姐看。”小玉走到里面,取出包袱,打开是一套缎子衣服、两根金头簪儿、一技金花。把吴银儿哭的泪如雨点相似,说道:“饿早知他老人家不好,也来伏侍两日儿。”说毕,一面拜谢了月娘。月娘待茶与他吃,留他过了三日去。

到三日,和尚打起磐子,道场诵经,挑出纸钱去。合家大小都披麻带孝。陈敬济穿重孝絰巾,佛前拜礼,街坊邻舍、亲朋长官都来吊问,上纸祭奠者,不论其数。阴阳徐先生早来伺候大殓。祭告已毕,抬尸入棺,西门庆交吴月娘又寻出他四套上色衣服来,装在棺内,四角又安放了四锭小银子儿。花子由说:“姐夫,倒不消安他在里面,金银日久定要出世,倒非久远之计。”西门庆不肯,定要安放。不一时,放下了七星板,搁上紫盖,仵作四面用长命钉一齐钉起来,一家大小放声号哭。西门庆亦哭的呆了,口口声声只叫:“我的年小的姐姐,再不得见你了!”良久哭毕,管待徐先生斋馔,打发去了。阖家伙计都是巾带孝服,行香之时,门首一片皆白。温秀才举荐,北边杜中书来题铭旌。杜中书名子春,号云野,原侍真宗宁和殿,今坐闲在家,西门庆备金帛请来。在卷棚内备果盒,西门庆亲递三杯酒,应伯爵与温秀才相陪。铺大红官紵题旌,西门庆要写“诏封锦衣西门恭人李氏柩”十一字,伯爵再三不肯,说:“见有正室夫人在,如何使得!”杜中书道:“曾生过子,于礼也无碍。”讲了半日,去了“恭”字,改了“室人”。温秀才道:“恭人系命妇,有爵;室人乃室内之人,只是个浑然通常之称。”于是用白粉题毕,“诏封”二字贴了金,悬于灵前。又题了神主。叩谢杜中书,管待酒馔,拜辞而去。

那日,乔大户、吴大舅、花大舅、韩姨夫、沈姨夫各家都是三牲祭桌来烧纸。乔大户娘子并吴大妗子、二妗子、花大妗子,坐轿子来吊丧,祭祀哭泣。月娘等皆孝髻,头须系腰,麻布孝裙,出来回礼举哀,让后边待茶摆斋。惟花大妗子与花大舅便是重孝直身,余者都是轻孝。那日李桂姐打听得知,坐轿子也来上纸,看见吴银儿在这里,说道:“你几时来的?怎的也不会我会儿?好人儿,原来只顾你!”吴银儿道:“我也不知道娘没了,早知也来看看了。”月娘后边管待,俱不必细说。

须臾过了,看看到首七,又是报恩寺十六众上僧,朗僧官为首座,引领做水陆道场,诵《法华经》,拜三昧水忏。亲朋伙计无不毕集。那日,玉皇庙吴道官来上纸吊孝,就揽二七经,西门庆留在卷棚内吃斋。忽见小厮来报:“韩先生送半身影来。”众人观看,但见头戴金翠围冠,双凤珠子挑牌、大红妆花袍儿,白馥馥脸儿,俨然如生。西门庆见了,满心欢喜。悬挂材头,众人无不夸奖:“只少口气儿!”一面让卷棚内吃斋,嘱咐:“大影还要加工夫些。”韩先生道:“小人随笔润色,岂敢粗心!”西门庆厚赏而去。

午间,乔大户来上祭,猪羊祭品、金银山、缎帛彩缯、冥纸炷香共约五十余抬,地吊高撬,锣鼓细乐吹打,缨络喧阗而至。西门庆与陈敬济穿孝衣在灵前还礼。乔大户邀了尚举人、朱堂官、吴大舅、刘学官、花千户、段亲家七八位亲朋,各在灵前上香。三献已毕,俱跪听阴阳生读祝文曰:

\[
维政和七年,岁次丁酉,九月庚申朔,越二十二日辛巳,眷生乔洪等谨以刚鬣柔毛庶羞之奠,致祭于故亲家母西门孺人李氏之灵曰:呜呼!孺人之性,宽裕温良,治家勤俭,御众慈祥,克全妇道,誉动乡邦。闺阃之秀,兰蕙之芳,夙配君子,效聘鸾凰。蓝玉已种,浦珠已光。正期谐琴瑟于有永,享弥寿于无疆。胡为一病,梦断黄粱?善人之殁,孰不哀伤?弱女襁褓,沐爱姻嫱。不期中道,天不从愿,鸳伴失行。恨隔幽冥,莫睹行藏。悠悠情谊,寓此一觞。灵其有知,来格来歆。尚飨。
\]
官客祭毕,回礼毕,让卷棚内桌席管待。然后乔大户娘子、崔亲家母、朱堂官娘子、尚举人娘子、段大姐众堂客女眷祭奠,地吊锣鼓,灵前吊鬼判队舞。吴月娘陪着哭毕,请去后边待茶设席,三汤五割,俱不必细说。

西门庆正在卷棚内陪人吃酒,忽前边打的云板响。答应的慌慌张张进来禀报:“本府胡爷上纸来了,在门首下轿子。”慌的西门庆连忙穿孝衣,灵前伺候。即使温秀才衣巾素服出迎,左右先捧进香纸,然后胡府尹素服金带进来。许多官吏围随,扶衣搊带,到了灵前,春鸿跪着,捧的香高高的,上了香,展拜两礼。西门庆便道:“老先生请起,多有劳动。”连忙下来回礼。胡府尹道,“令夫人几时没了?学生昨日才知。吊迟,吊迟!”西门庆道:“侧室一疾不救,辱承老先生枉吊。”温秀才在旁作揖毕,请到厅上待茶一杯,胡府尹起身,温秀才送出大门,上轿而去。上祭人吃至后晌方散。

第二日,院中郑爱月儿家来上纸。爱月儿进至灵前,烧了纸。月娘见他抬了八盘饼馓、三牲汤饭来祭奠,连忙讨了一匹整绢孝裙与他。吴银儿与李桂姐都是三钱奠仪,告西门庆说。西门庆道:“值甚么,每人都与他一匹整绢就是了。”月娘邀到后边房里,摆茶管待,过夜。

晚夕,亲朋伙计来伴宿,叫了一起海盐子弟搬演戏文。李铭、吴惠、郑奉、郑春都在这里答应。西门庆在大棚内放十五张桌席,为首的就是乔大户、吴大舅、吴二舅、花大舅、沈姨夫、韩姨夫、倪秀才、温秀才、任医官、李智、黄四、应伯爵、谢希大、祝实念、孙寡嘴、白赉光、常峙节、傅日新、韩道国、甘出身、贲第传、吴舜臣、两个外甥,还有街坊六七位人,都是开桌儿。点起十数枝大烛来,堂客便在灵前围着围屏,垂帘放桌席,往外观戏。当时众人祭奠毕,西门庆与敬济回毕礼,安席上坐。下边戏子打动锣鼓,搬演的是韦皋、玉箫女两世姻缘《玉环记》。不一时吊场,生扮韦皋,唱了一回下去。贴旦扮玉箫,又唱了一回下去。厨役上汤饭割鹅。应伯爵便向西门庆说:“我闻的院里姐儿三个在这里,何不请出来,与乔老亲家、老舅席上递杯酒儿。他倒是会看戏文,倒便益了他!”西门庆便使玳安进入说去:“请他姐儿三个出来。”乔大户道:“这个却不当。他来吊丧,如何叫他递起酒来?”伯爵道:“老亲家,你不知,象这样小淫妇儿,别要闲着他。——快与我牵出来!你说应二爹说,六娘没了,只当行孝顺,也该与俺每人递杯酒儿。”玳安进去半日,说:“听见应二爹在坐,都不出来哩。”伯爵道:“既恁说,我去罢。”走了两步,又回坐下。西门庆笑道:“你怎的又回了?”伯爵道:“我有心待要扯那三个小淫妇出来,等我骂两句,出了我气,我才去。”落后又使玳安请了一遍,三个才慢条条出来。都一色穿着白绫对衿袄儿、蓝缎裙子,向席上不端不正拜了拜儿,笑嘻嘻立在旁边。应伯爵道:“俺每在这里,你如何只顾推三阻四,不肯出来?”那三个也不答应,向上边递了回酒,设一席坐着。下边鼓乐响动,关目上来,生扮韦皋,净扮包知木,同到勾栏里玉箫家来。那妈儿出来迎接,包知木道:“你去叫那姐儿出来。”妈云:“包官人,你好不着人,俺女儿等闲不便出来。说不得一个‘请’字儿,你如何说‘叫他出来’?”那李桂姐向席上笑道:“这个姓包的,就和应花子一般,就是个不知趣的蹇味儿!”伯爵道:“小淫妇,我不知趣,你家妈怎喜欢我?”桂姐道:“他喜欢你?过一边儿!”西门庆道:“看戏罢,且说甚么。再言语,罚一大杯酒!”那伯爵才不言语了。那戏子又做了一回,并下。

厅内左边吊帘子看戏的,是吴大妗子、二妗子、杨姑娘、潘姥姥、吴大姨、孟大姨、吴舜臣媳妇郑三姐、段大姐,并本家月娘姊妹;右边吊帘子看戏的,是春梅、玉箫、兰香、迎春、小玉,都挤着观看。那打茶的郑纪,正拿着一盘果仁泡茶从帘下过,被春梅叫住,问道:“拿茶与谁吃?”郑纪道:“那边六妗子娘每要吃。”这春梅取一盏在手。不想小玉听见下边扮戏的旦儿名字也叫玉箫,便把王箫拉着说道:“淫妇,你的孤老汉子来了。鸨子叫你接客哩,你还不出去。”使力往外一推,直推出帘子外,春梅手里拿着茶,推泼一身。骂玉箫:“怪淫妇,不知甚么张致,都顽的这等!把人的茶都推泼了,早是没曾打碎盏儿。”西门庆听得,使下来安儿来问:“谁在里面喧嚷?”春梅坐在椅上道:“你去就说,玉箫浪淫妇,见了汉子这等浪。”那西门庆问了一回,乱着席上递酒,就罢了。月娘便走过那边数落小玉:“你出来这一日,也往屋里瞧瞧去。都在这里,屋里有谁?”小玉道:“大姐刚才后边去的,两位师父也在屋里坐着。”月娘道:“教你们贼狗胎在这里看看,就恁惹是招非的。”春梅见月娘过来,连忙立起身来说道:“娘,你问他。都一个个只象有风病的,狂的通没些成色儿,嘻嘻哈哈,也不顾人看见。”那月娘数落了一回,仍过那边去了。

那时,乔大户与倪秀才先起身去了。沈姨夫与任医官、韩姨夫也要起身,被应伯爵拦住道:“东家,你也说声儿。俺每倒是朋友,不敢散,一个亲家都要去。沈姨夫又不隔门,韩姨夫与任大人、花大舅都在门外。这咱晚三更天气,门也还未开,慌的甚么?都来大坐回儿,左右关目还未了哩。”西门庆又令小厮提四坛麻姑酒,放在面前,说:“列位只了此四坛酒,我也不留了。”因拿大赏钟放在吴大舅面前,说道:“那位离席破坐说起身者,任大舅举罚。”于是众人又复坐下了。西门庆令书童:“催促子弟,快吊关目上来,吩咐拣着热闹处唱罢。”须臾打动鼓板,扮末的上来,请问面门庆:“‘寄真容’那一折可要唱?”西门庆道:“我不管你,只要热闹。”贴旦扮玉箫唱了回。西门庆看唱到“今生难会面,因此上寄丹青”一句,忽想起李瓶儿病时模样,不觉心中感触起来,止不住眼中泪落,袖中不住取汗巾儿搽拭。又早被潘金莲在帘内冷眼看见,指与月娘瞧,说道:“大娘,你看他好个没来头的行货子,如何吃着酒,看见扮戏的哭起来?”盂玉楼道:“你聪明一场,这些儿就不知道了?乐有悲欢离合,想必看见那一段儿触着他心,他睹物思人,见鞍思马,才掉泪来。”金莲道:“我不信。打谈的掉眼泪——替古人耽忧,这些都是虚。他若唱的我泪出来,我才算他好戏子。”月娘道:“六姐,悄悄儿,咱每听罢。”玉楼因向大妗子道:“俺六姐不知怎的,只好快说嘴。”

那戏子又做了一回,约有五更时分,众人齐起身。西门庆拿大杯拦门递酒,款留不住,俱送出门。看收了家伙,留下戏厢:“明日有刘公公、薛公公来祭奠,还做一日。”众戏子答应。管待了酒饭,归下处歇去了。李铭等四个亦归家不题。西门庆见天色已将晓,就归后边歇息去了。正是,得多少——

\[
红日映窗寒色浅,淡烟笼竹曙光微。
\]


\newpage
%# -*- coding:utf-8 -*-
%%%%%%%%%%%%%%%%%%%%%%%%%%%%%%%%%%%%%%%%%%%%%%%%%%%%%%%%%%%%%%%%%%%%%%%%%%%%%%%%%%%%%


\chapter{玉箫跪受三章约\KG 书童私挂一帆风}


诗曰:

\[
玉殒珠沉思悄然,明中流泪暗相怜。
常图蛱蝶花楼下,记效鸳鸯翠幕前。
只有梦魂能结雨,更无心绪学非烟。
朱颜皓齿归黄土,脉脉空寻再世缘。
\]

话说众人散了,已有鸡唱时分,西门庆歇息去了。玳安拿了一大壶酒、几碟下饭,在铺子里还要和傅伙计、陈敬济同吃。傅伙计老头子熬到这咱,已是坐不住,搭下铺就倒在炕上,向玳安道:“你自和平安吃罢,陈姐夫想也不来了。”玳安叫进平安来,两个把那酒你一钟我一盏都吃了。收过家伙,平安便去门房里睡了。玳安一面关上铺子门,上炕和傅伙计两个对厮脚儿睡下。傅伙计因闲话,向玳安说道:“你六娘没了,这等棺椁念经发送,也够他了。”玳安道:“他的福好,只是不长寿。俺爹饶使了这些钱,还使不着俺爹的哩。俺六娘嫁俺爹,瞒不过你老人家,他带了多少带头来!别人不知道,我知道。银子休说,只金珠玩好、玉带、绦环、\textuni{4BFC}髻、值钱的宝石,也不知有多少。为甚俺爹心里疼?不是疼人,是疼钱。若说起六娘的性格儿,一家子都不如他,又谦让又和气,见了人,只是一面儿笑,自来也不曾喝俺每一喝,并没失口骂俺每一句‘奴才’。使俺每买东西,只拈块儿。俺每但说:‘娘,拿等子,你称称。’他便笑道:‘拿去罢,称什么。你不图落图什么来?只要替我买值着。’这一家子,那个不借他银使?只有借出来,没有个还进去的。还也罢,不还也罢。俺大娘和俺三娘使钱也好。只是五娘和二娘,悭吝的紧。他当家,俺每就遭瘟来。会胜买东西,也不与你个足数,绑着鬼,一钱银子,只称九分半,着紧只九分,俺每莫不赔出来!”傅伙计道:“就是你大娘还好些。”玳安道:“虽故俺大娘好,毛司火性儿,一回家好,娘儿每亲亲哒哒说话儿,你只休恼着他,不论谁,他也骂你几句儿。总不如六娘,万人无怨,又常在爹跟前替俺每说方便儿。随问天来大事,俺每央他央儿对爹说,无有个不依。只是五娘,行动就说:‘你看我对爹说不说!’把这打只提在口里。如今春梅姐,又是个合气星。——天生的都在他一屋里。”傅伙计道:“你五娘来这里也好几年了。”玳安道:“你老人家是知道的,想的起他那咱来的光景哩。他一个亲娘也不认的,来一遭,要便抢的哭了家去。如今六娘死了,这前边又是他的世界,明日那个管打扫花园,干净不干净,还吃他骂的狗血喷了头哩!”两个说了一回,那傅伙计在枕上齁齁就睡着了。玳安亦有酒了,合上眼,不知天高地下,直至红日三竿,都还未起来。

原来西门庆每常在前边灵前睡,早晨玉箫出来收叠床铺,西门庆便往后边梳头去。书童蓬着头,要便和他两个在前边打牙犯嘴,互相嘲逗,半日才进后边去。不想这日西门庆归上房歇去,玉箫赶人没起来,暗暗走出来,与书童约了,走在花园书房里干营生去了。不料潘金莲起的早,蓦地走到厅上,只见灵前灯儿也没了,大棚里丢的桌椅横三竖四,没一个人儿,只有画童儿在那里扫地。金莲道:“贼囚根子,干净只你在这里,都往那里去了?”画童道:“他每都还没起来哩。”金莲道:“你且丢下笤帚,到前边对你姐夫说,有白绢拿一匹来,你潘姥姥还少一条孝裙子,再拿一副头须系腰来与他。他今日家去。”画童道:“怕不俺姐夫还睡哩,等我问他去。”良久回来道:“姐夫说不是他的首尾,书童哥与崔本哥管孝帐。娘问书童哥要就是了。”金莲道:“知道那奴才往那去了,你去寻他来。”画童向厢房里瞧了瞧,说道:“才在这里来,敢往花园书房里梳头去了。”金莲说道:“你自扫地,等我自家问这囚根子要去。”因走到花园书房内,忽然听见里面有人笑声。推开门,只见书童和玉箫在床上正干得好哩。便骂道:“好囚根子,你两个干得好事!”唬得两个做手脚不迭,齐跪在地下哀告。金莲道:“贼囚根子,你且拿一匹孝绢、一匹布来,打发你潘姥姥家去着。”书童连忙拿来递上。金莲迳归房来。

那玉箫跟到房中,打旋磨儿跪在地下央及:“五娘,千万休对爹说。”金莲便问:“贼狗肉,你和我实说,从前已往,偷了几遭?一字儿休瞒我,便罢。”那玉箫便把和他偷的缘由说了一遍。金莲道:“既要我饶你,你要依我三件事。”玉箫道:“娘饶了我,随问几件事我也依娘。”金莲道:“第一件,你娘房里,但凡大小事儿,就来告我说。你不说,我打听出来,定不饶你。第二件,我但问你要甚么,你就捎出来与我。第三件,你娘向来没有身孕,如今他怎生便有了?”玉箫道:“不瞒五娘说,俺娘如此这般,吃了薛姑子的衣胞符药,便有了。”潘金莲一一听记在心,才不对西门庆说了。

书童见潘金莲冷笑领进玉箫去了,知此事有几分不谐。向书房厨柜内收拾了许多手帕汗巾、挑牙簪纽,并收的人情,他自己也攒有十来两银子,又到前边柜上诓了傅伙计二十两,只说要买孝绢,迳出城外,雇了长行头口,到码头上,搭在乡里船上,往苏州原籍家去了。正是:

\[
撞碎玉笼飞彩凤,顿开金锁走蛟龙。
\]

那日,李桂姐、吴银儿、郑爱月都要家去了。薛内相、刘内相早晨差人抬三牲桌面来祭奠烧纸。又每人送了一两银子伴宿分资,叫了两个唱道情的来,白日里要和西门庆坐坐。紧等着要打发孝绢,寻书童儿要钥匙,一地里寻不着。傅伙计道:“他早晨问我柜上要了二十两银子买孝绢去了,口称爹吩咐他孝绢不够,敢是向门外买去了?”西门庆道:“我并没吩咐他,如何问你要银子?”一面使人往门外绢铺找寻,那里得来!月娘向西门庆说:“我猜这奴才有些跷蹊,不知弄下甚么硶儿,拐了几两银子走了。你那书房里还大瞧瞧,只怕还拿甚么去了。”西门庆走到两个书房里都瞧了,只见库房里钥匙挂在墙上,大橱柜里不见了许多汗巾手帕,并书礼银子、挑牙纽扣之类,西门庆心中大怒,叫将该地方管役来,吩咐:“各处三街两巷与我访缉。”那里得来!正是:

\[
不独怀家归兴急,五湖烟水正茫茫。
\]

那日,薛内相从晌午就坐轿来了。西门庆请下吴大舅、应伯爵、温秀才相陪。先到灵前上香,打了个问讯,然后与西门庆叙礼,说道:“可伤,可伤!如夫人是甚病儿殁了?”西门庆道:“不幸患崩泻之疾殁了,多谢老公公费心。”薛内相道:“没多儿,将就表意罢了。”因看见挂的影,说道:“好位标致娘子!正好青春享福,只是去世太早些。”温秀才在旁道:“物之不齐,物之情也。穷通寿夭,自有个定数,虽圣人亦不能强。”薛内相扭回头来,见温秀才穿着衣巾,因说道:“此位老先儿是那学里的?”温秀才躬身道:“学生不才,备名府庠。”薛内相道:“我瞧瞧娘子的棺木儿。”西门庆即令左右把两边帐子撩起,薛内相进去观看了一遍,极口称赞道:“好副板儿!请问多少价买的?”西门庆道:“也是舍亲的一副板,学生回了他的来了。”应伯爵道:“请老公公试估估,那里地道,甚么名色?”薛内相仔细看了说:“此板不是建昌,就是副镇远。”伯爵道:“就是镇远,也值不多。”薛内相道:“最高者,必定是杨宣榆。”伯爵道:“杨宣榆单薄短小,怎么看得过!此板还在杨宣榆之上,名唤做桃花洞,在于湖广武陵川中。昔日唐渔父入此洞中,曾见秦时毛女在此避兵,是个人迹罕到之处。此板七尺多长,四寸厚,二尺五宽。还看一半亲家分上,还要了三百七十两银子哩。公公,你不曾看见,解开喷鼻香的,里外俱有花色。”薛内相道:“是娘子这等大福,才享用了这板。俺每内官家,到明日死了,还没有这等发送哩。”吴大舅道:“老公公好说,与朝廷有分的人,享大爵禄,俺们外官焉能赶的上。老公公日近清光,代万岁传宣金口。见今童老爷加封王爵,子孙皆服蟒腰玉,何所不至哉!”薛内相便道:“此位会说话的兄,请问上姓?”西门庆道:“此是妻兄吴大哥,见居本卫千户之职。”薛内相道:“就是此位娘子令兄么?”西门庆道:“不是。乃贱荆之兄。”薛内相复于吴大舅声诺说道:“吴大人,失瞻!”

看了一回,西门庆让至卷棚内,正面安放一把交椅,薛内相坐下,打茶的拿上茶来吃了。薛内相道:“刘公公怎的这咱还不到?叫我答应的迎迎去。”青衣人跪下禀道:“小的邀刘公公去来,刘公公轿已伺候下了,便来也。”薛内相又问道:“那两个唱道情的来了不曾?”西门庆道:“早上就来了。——叫上来!”不一时,走来面前磕头。薛内相道:“你每吃了饭不曾?”那人道:“小的每吃了饭了。”薛内相道:“既吃了饭,你每今日用心答应,我重赏你。”西门庆道:“老公公,学生这里还预备着一起戏子,唱与老公公听。”薛内相问:“是那里戏子?”西门庆道:“是一班海盐戏子。”薛内相道:“那蛮声哈剌,谁晓的他唱的是甚么!那酸子每在寒窗之下,三年受苦,九载遨游,背着琴剑书箱来京应举,得了个官,又无妻小在身边,便希罕他这样人。你我一个光身汉、老内相,要他做甚么?”温秀才在旁边笑说道:“老公公说话,太不近情了。居之齐则齐声,居之楚则楚声。老公公处于高堂广厦,岂无一动其心哉?”这薛内相便拍手笑将起来道:“我就忘了温先儿在这里。你每外官,原来只护外官。”温秀才道:“虽是士大夫,也只是秀才做的。老公公砍一枝损百林,兔死狐悲,物伤其类。”薛内相道:“不然。一方之地,有贤有愚。”

正说着,忽左右来报:“刘公公下轿了。”吴大舅等出去迎接进来,向灵前作了揖。叙礼已毕,薛内相道:“刘公公,你怎的这咱才来?”刘内相道:“北边徐同家来拜望,陪他坐了一回,打发去了。”一面分席坐下,左右递茶上去。因问答应的:“祭奠桌面儿都摆上了不曾?”下边人说:“都排停当了。”刘内相道:“咱每去烧了纸罢。”西门庆道:“老公公不消多礼,头里已是见过礼了。”刘内相道:“此来为何?还当亲祭祭。”当下,左右捧过香来,两个内相上了香,递了三钟酒,拜下去。西门庆道:“老公公请起。”于是拜了两拜起来,西门庆还了礼,复至卷棚内坐下。然后收拾安席,递酒上坐。两位内相分左右坐了,吴大舅、温秀才、应伯爵从次,西门庆下边相陪。子弟鼓板响动,递了关目揭帖。两位内相看了一回,拣了一段《刘智远白兔记》。唱了还未几折,心下不耐烦,一面叫上两个唱道情的去,打起渔鼓,并肩朝上,高声唱了一套“韩文公雪拥蓝关”故事下去。

薛内相便与刘内相两个说说话儿,道:“刘哥,你不知道,昨日这八月初十日,下大雨如注,雷电把内里凝神殿上鸱尾裘碎了,唬死了许多宫人。朝廷大惧,命各官修省,逐日在上清宫宣《精灵疏》建醮。禁屠十日,法司停刑,百官不许奏事。昨日大金遣使臣进表,要割内地三镇,依着蔡京那老贼,就要许他。掣童掌事的兵马,交都御史谭积、黄安十大使节制三边兵马,又不肯,还交多官计议。昨日立冬,万岁出来祭太庙,太常寺一员博士,名唤方轸,早晨打扫,看见太庙砖缝出血,殿东北上地陷了一角,写表奏知万岁。科道官上本,极言童掌事大了,宦官不可封王。如今马上差官,拿金牌去取童掌事回京。”刘内相道:“你我如今出来在外做土官,那朝事也不干咱每。俗语道,咱过了一日是一日。便塌了天,还有四个大汉。到明天,大宋江山管情被这些酸子弄坏了。王十九,咱每只吃酒!”因叫唱道情的上来,吩咐:“你唱个‘李白好贪杯’的故事。”那人立在席前,打动渔鼓,又唱了一回。

直吃至日暮时分,吩咐下人,看轿起身。西门庆款留不住,送出大门,喝道而去。回来,吩咐点起烛来,把桌席休动,留下吴大舅、应伯爵、温秀才坐的,又使小厮请傅伙计、甘伙计、韩道国、贲第传、崔本和陈敬济复坐。叫上子弟来吩咐:“还找着昨日《玉环记》上来。”因向伯爵道:“内相家不晓的南戏滋味。早知他不听,我今日不留他。”伯爵道:“哥,到辜负你的意思。内臣斜局的营生,他只喜《蓝关记》、捣喇小子山歌野调,那里晓的大关目悲欢离合!”于是下边打动鼓板,将昨日《玉环记》做不完的折数,一一紧做慢唱,都搬演出来。西门庆令小厮席上频斟美酒。伯爵与西门庆同桌而坐,便问:“他姐儿三个还没家去,怎的不叫出来递杯酒儿?”西门庆道:“你还想那一梦儿,他每去的不耐烦了!”伯爵道:“他每在这里住了有两三日?”西门庆道:“吴银儿住的久了。”当日,众人坐到三更时分,搬戏已完,方起身各散。西门庆邀下吴大舅,明日早些来陪上祭官员。与了戏子四两银子,打发出门。

到次日,周守备、荆都监、张团练、夏提刑,合卫许多官员,都合了分资,办了一副猪羊吃桌祭奠,有礼生读祝。西门庆预备酒席,李铭等三个小优儿伺候答应。到晌午,只听鼓响,祭礼到了。吴大舅、应伯爵、温秀才在门首迎接,只见后拥前呼,众官员下马,在前厅换衣服。良久,把祭品摆下,众官齐到灵前,西门庆与陈敬济还礼。礼生喝礼,三献毕,跪在旁边读祝,祭毕。西门庆下来谢礼已毕,吴大舅等让众官至卷棚内,宽去素服,待毕茶,就安席上坐,觥筹交错,殷勤劝酒。李铭等三个小优儿,银筝檀板,朝上弹唱。众官欢饮,直到日暮方散。西门庆还要留吴大舅众人坐,吴大舅道:“各人连日打搅,姐夫也辛苦了,各自歇息去罢。”当时告辞回家。正是:

\[
天上碧桃和露种,日边红杏倚云栽。
家中巨富人趋附,手内多时莫论财。
\]

\newpage
%# -*- coding:utf-8 -*-
%%%%%%%%%%%%%%%%%%%%%%%%%%%%%%%%%%%%%%%%%%%%%%%%%%%%%%%%%%%%%%%%%%%%%%%%%%%%%%%%%%%%%


\chapter{愿同穴一时丧礼盛\KG 守孤灵半夜口脂香}


诗曰:

\[
湘皋烟草碧纷纷,泪洒东风忆细君。
见说嫦娥能入月,虚疑神女解为云。
花阴昼坐闲金剪,竹里游春冷翠裙。
留得丹青残锦在,伤心不忍读回文。
\]

话说到十月二十八日,是李瓶儿二七,玉皇庙吴道官受斋,请了十六个道众,在家中扬幡修建斋坛。又有安郎中来下书,西门庆管待来人去了。吴道官庙中抬了三牲祭礼来,又是一匹尺头以为奠仪。道众绕棺传咒,吴道官灵前展拜。西门庆与敬济回礼,谢道:“师父多有破费,何以克当?”吴道官道:“小道甚是惶愧,本该助一经追荐夫人,奈力薄,粗祭表意而已。”西门庆命收了,打发抬盒人回去。那日三朝转经,演生神章,破九幽狱,对灵摄召,整做法事,不必细说。

第二日,先是门外韩姨夫家来上祭。那时孟玉楼兄弟孟锐做买卖来家,见西门庆这边有丧事,跟随韩姨夫那边来上祭,讨了一分孝去,送了许多人事。西门庆叙礼,进入玉楼房中拜见。西门庆亦设席管待,俱不在言表。

那日午间,又是本县知县李拱极、县丞钱斯成、主簿任良贵、典史夏恭基,又有阳谷县知县狄斯朽,共五员官,都斗了分子,穿孝服来上纸帛吊问。西门庆备席在卷棚内管待,请了吴大舅与温秀才相陪,三个小优儿弹唱。

正饮酒到热闹处,忽报:“管砖厂工部黄老爹来吊孝。”慌的西门庆连忙穿孝衣灵前伺侯,温秀才又早迎接至大门外,让至前厅,换了衣裳进来。家人手捧香烛纸匹金段到灵前,黄主事上了香,展拜毕,西门庆同敬济下来还礼。黄主事道:“学生不知尊阃没了,吊迟,恕罪,恕罪!”西门庆道:“学生一向欠恭,今又承老先生赐吊,兼辱厚仪,不胜感激。”叙毕礼,让至卷棚上面坐下。西门庆与温秀才下边相陪,左右捧茶上来吃了。黄主事道:“昨日宋松原多致意先生,他也闻知令夫人作过,也要来吊问,争奈有许多事情羁绊。他如今在济州住扎。先生还不知,朝廷如今营建艮岳,敕令太尉朱勔,往江南湖湘采取花石纲,运船陆续打河道中来。头一运将到淮上。又钦差殿前六黄太尉来迎取卿云万态奇峰——长二丈,阔数尺,都用黄毡盖覆,张打黄旗,费数号船只,由山东河道而来。况河中没水,起八郡民夫牵挽。官吏倒悬,民不聊生。宋道长督率州县,事事皆亲身经历,案牍如山,昼夜劳苦,通不得闲。况黄太尉不久自京而至,宋道长说,必须率三司官员,要接他一接。想此间无可相熟者,委托学生来,敬烦尊府做一东,要请六黄大尉一饭,未审尊意允否?”因唤左右:“叫你宋老爹承差上来。”有二青衣官吏跪下,毡包内捧出一对金段、一根沉香、两根白蜡、一分绵纸。黄主事道:“此乃宋公致赙之仪。那两封,是两司八府官员办酒分资——两司官十二员、府官八员,计二十二分,共一百零六两。”交与西门庆:“有劳盛使一备何如?”西门庆再三辞道:“学生有服在家,奈何,奈何?”因问:“迎接在于何时?”黄主事道:“还早哩,也得到出月半头。黄太监京中还未起身。”西门庆道:“学生十月十二日才发引。既是宋公祖与老先生吩咐,敢不领命!但这分资决不敢收。该多少桌席,只顾吩咐,学生无不毕具。”黄主事道:“四泉此意差矣!松原委托学生来烦渎,此乃山东一省各官公礼,又非松原之己出,何得见却?如其不纳,学生即回松原,再不敢烦渎矣!”西门庆听了此言,说道:“学生权且领下。”因令玳安、王经接下去。问备多少桌席,黄主事道:“六黄备一张吃看大桌面,宋公与两司都是平头桌席,以下府官散席而已。承应乐人,自有差拨伺候,府上不必再叫。”说毕,茶汤两换,作辞起身。西门庆款留,黄主事道:“学生还要到尚柳塘老先生那里拜拜,他昔年曾在学生敝处作县令,然后转成都府推官。如今他令郎两泉,又与学生乡试同年。”西门庆道:“学生不知老先生与尚两泉相厚,两泉亦与学生相交。”黄主事起身,西门庆道:“烦老先生多致意宋公祖,至期寒舍拱候矣。”黄主事道:“临期,松原还差人来通报先生,亦不可太奢。”西门庆道,“学生知道。”送出大门,上马而去。

那县中官员,听见黄主事带领巡按上司人来,唬的都躲在山子下小卷棚内饮酒,吩咐手下把轿马藏过一边。当时,西门庆回到卷棚与众官相见,具说宋巡按率两司八府来,央烦出月迎请六黄太尉之事。众官悉言:“正是州县不胜忧苦。这件事,钦差若来,凡一应衹迎、廪饩、公宴、器用、人夫,无不出于州县,州县必取之于民,公私困极,莫此为甚。我辈还望四泉于上司处美言提拔,足见厚爱。”言讫,都不久坐,告辞起身而去。

话休饶舌。到李瓶儿三七,有门外永福寺道坚长老,领十六众上堂僧来念经,穿云锦袈裟,戴毗卢帽,大钹大鼓,甚是齐整。十月初八日是四七,请西门外宝庆寺赵喇嘛,亦十六众,来念番经,结坛跳沙,洒花米行香,口诵真言。斋供都用牛乳茶酪之类,悬挂都是九丑天魔变相,身披缨络琉璃,项挂髑髅,口咬婴儿,坐跨妖魅,腰缠蛇螭,或四头八臂,或手执戈戟,朱发蓝面,丑恶莫比。午斋以后,就动荤酒。西门庆那日不在家,同阴阳徐先生往坟上破土开圹去了,后晌方回。晚夕,打发喇嘛散了。

次日,推运山头酒米、桌面肴品一应所用之物,又委付主管伙计,庄上前后搭棚,坟内穴边又起三间罩棚。先请附近地邻来,大酒大肉管待。临散,皆肩背项负而归,俱不必细说。

十一日白日,先是歌郎并锣鼓地吊来灵前参灵,吊《五鬼闹判》、《张天师着鬼迷》、《钟馗戏小鬼》、《老子过函关》、《六贼闹弥陀》、《雪里梅》、《庄周梦蝴蝶》、《天王降地水火风》、《洞宾飞剑斩黄龙》、《赵太祖千里送荆娘》,各样百戏吊罢,堂客都在帘内观看。参罢灵去了,内外亲戚都来辞灵烧纸,大哭一场。

到次日发引,先绝早抬出名旌、各项幡亭纸扎,僧道、鼓手、细乐、人役都来伺候。西门庆预先问帅府周守备讨了五十名巡捕军士,都带弓马,全装结束。留十名在家看守,四十名在材边摆马道,分两翼而行。衙门里又是二十名排军打路,照管冥器。坟头又是二十名把门,管收祭祀。那日官员士夫、亲邻朋友来送殡者,车马喧呼,填街塞巷。本家并亲眷轿子也有百十余顶,三院鸨子粉头小轿也有数十。徐阴阳择定辰时起棺,西门庆留下孙雪娥并二女僧看家,平安儿同两名排军把前门。女婿陈敬济跪在柩前摔盆,六十四人上扛,有仵作一员官立于增架上,敲响板,指拨抬材人上肩。先是请了报恩寺僧官来起棺,转过大街口望南走。两边观看的人山人海。那日正值晴明天气,果然好殡。但见:

和风开绮陌,细雨润芳尘,东方晓日初升,北陆残烟乍敛。冬冬咙咙,花丧鼓不住声喧;叮叮当当,地吊锣连宵振作。铭旌招飐,大书九尺红罗;起火轩天,冲散半天黄雾。狰狰狞狞开路鬼,斜担金斧;忽忽洋洋险道神,端秉银戈。逍逍遥遥八洞仙,龟鹤绕定;窈窈窕窕四毛女,虎鹿相随。热热闹闹采莲船,撒科打诨;长长大大高跷汉,贯甲顶盔。清清秀秀小道童一十六众,都是霞衣道髻,动一派之仙音;肥肥胖胖大和尚二十四个,个个都是云锦袈裟,转五方之法事。一十二座大绢亭,亭亭皆绿舞红飞;二十四座小绢亭,座座尽珠围翠绕。左势下,天仓与地库相连;右势下,金山与银山作队。掌醢厨,列八珍之罐;香烛亭,供三献之仪。六座百花亭,现千团锦绣;一乘引魂轿,扎百结黄丝。这边把花与雪柳争辉,那边宝盖与银幢作队。金字幡银字幡,紧护棺舆;白绢繖绿绢繖,同围增架。功布招飐,孝眷声哀。打路排军,执榄杆前后呼拥;迎丧神会,耍武艺左右盘旋。卖解犹如鹰鹞,走马好似猿猴。竖肩桩,打斤斗,隔肚穿钱,金鸡独立,人人喝彩,个个争夸。扶肩挤背,不辨贤愚;挨睹并观,那分贵贱!张三蠢胖,只把气吁;李四矮矬,频将脚跕。白头老叟,尽将拐棒拄髭须;绿鬓佳人,也带儿童来看殡。

吴月娘与李娇儿等本家轿子十余顶,一字儿紧跟材后。西门庆总冠孝服同众亲朋在材后,陈敬济紧扶棺舆,走出东街口。西门庆具礼,请玉皇庙吴道官来悬真。身穿大红五彩鹤氅,头戴九阳雷巾,脚登丹舄,手执牙笏,坐在四人肩舆上,迎殡而来。将李瓶儿大影捧于手内,陈敬济跪在前面,那殡停住了。众人听他在上高声宣念:

\[
恭惟\KG 故锦衣西门恭人李氏之灵,存日阳年二十七岁,元命辛未相,正月十五日午时受生,大限于政和七年九月十七日丑时分身故。伏以尊灵,名家秀质,绮阁娇姝。禀花月之仪容,蕴蕙兰之佳气。郁德柔婉,赋性温和。配我西君,克谐伉俪。处闺门而贤淑,资琴瑟以好和。曾种蓝田,寻嗟楚畹。正宜享福百年,可惜春光三九。呜呼!明月易缺,好物难全。善类无常,修短有数。今日棺舆载道,丹旆迎风,良夫躃踊于柩前,孝眷哀矜于巷陌。离别情深而难已,音容日远以日忘。某等谬忝冠簪,愧领玄教。愧无新垣平之神术,恪遵玄元始之遗风。徒展崔巍镜里之容,难返庄周梦中之蝶。漱甘露而沃琼浆,超知识登于紫府;披百宝而面七真,引净魄出于冥途。一心无挂,四大皆空。苦,苦,苦!气化清风形归土。一灵真性去弗回,改头换面无遍数。众听末后一句:咦!精爽不知何处去,真容留与后人看。
\]
吴道官念毕,端坐轿上,那轿卷坐退下去了。这里鼓乐喧天,哀声动地,殡才起身,迤逦出南门。众亲朋陪西门庆,走至门上方乘马,陈敬济扶柩,到于山头五里原。

原来坐营张团练,带领二百名军,同刘、薛二内相,又早在坟前高阜处搭帐房,吹响器,打铜锣铜鼓,迎接殡到,看着装烧冥器纸扎,烟焰涨天。棺舆到山下扛,徐先生率仵作,依罗经吊向,巳时祭告后土方隅后,才下葬掩土。西门庆易服,备一对尺头礼,请帅府周守备点主。卫中官员并亲朋伙计,皆争拉西门庆递酒,鼓乐喧天,烟火匝地,热闹丰盛,不必细说。

吃毕,后晌回灵,吴月娘坐魂轿,抱神主魂幡,陈敬济扶灵床,鼓手细乐十六众小道童两边吹打。吴大舅并乔大户、吴大舅、花大舅、沈姨夫、孟二舅、应伯爵、谢希大、温秀才、众主管伙计,都陪着西门庆进城,堂客轿子压后,到家门首燎火而入。李瓶儿房中安灵已毕,徐先生前厅祭神洒扫,么门户皆贴辟非黄符。谢徐先生一匹尺头、五两银子出门,各项人役打发散了。又拿出二十吊钱来,五吊赏巡捕军人,五吊与衙门中排军,十吊赏营里人马。拿帖儿回谢周守备、张团练、夏提刑,俱不在话下。西门庆还要留乔大户、吴大舅众人坐,众人都不肯,作辞起身。来保进说:“搭棚在外伺候,明日来拆棚。”西门庆道:“棚且不消拆,亦发过了你宋老爹摆酒日子来拆罢。”打发搭彩匠去了。后边花大娘子与乔大户娘子众堂客,还等着安毕灵,哭了一场,方才去了。

西门庆不忍遽舍,晚夕还来李瓶儿房中,要伴灵宿歇。见灵床安在正面,大影挂在旁边,灵床内安着半身,里面小锦被褥,床几、衣服、妆奁之类,无不毕具,下边放着他的一对小小金莲,桌上香花灯烛、金碟樽俎,般般供养,西门庆大哭不止。令迎春就在对面炕上搭铺,到夜半,对着孤灯,半窗斜月,翻复无寐,长吁短叹,思想佳人。有诗为证:

\[
短叹长吁对锁窗,舞鸾孤影寸心伤。
兰枯楚畹三秋雨,枫落吴江一夜霜。
夙世已违连理愿,此生难觅返魂香。
九泉果有精灵在,地下人间两断肠。
\]

白日间供养茶饭,西门庆俱亲看着丫鬟摆下,他便对面和他同吃。举起箸儿来:“你请些饭儿!”行如在之礼。丫鬟养娘都忍不住掩泪而哭。奶子如意儿,无人处常在跟前递茶递水,挨挨抢抢,掐掐捏捏,插话儿应答,那消三夜两夜。这日,西门庆因请了许多官客堂客,坟上暖墓来家,陪人吃得醉了。进来,迎春打发歇下。到夜间要茶吃,叫迎春不应,如意儿便来递茶。因见被拖下炕来,接过茶盏,用手扶被,西门庆一时兴动,搂过脖子就亲了个嘴,递舌头在他口内。老婆就咂起来,一声儿不言语。西门庆令脱去衣服上炕,两个搂在被窝内,不胜欢娱,云雨一处。老婆说:“既是爹抬举,娘也没了,小媳妇情愿不出爹家门,随爹收用便了。”西门庆便叫:“我儿,你只用心伏侍我,愁养活不过你来!”这老婆听了,枕席之间,无不奉承,颠鸾倒凤,随手而转,把西门庆欢喜的要不的。

次日,老婆早晨起来,与西门庆拿鞋脚,叠被褥,就不靠迎春,极尽殷勤,无所不至。西门庆开门寻出李瓶儿四根簪儿来赏他,老婆磕头谢了。迎春知收用了他,两个打成一路。老婆自恃得宠,脚跟已牢,无复求告于人,就不同往日,打扮乔模乔样,在丫鬟伙内,说也有,笑也有。早被潘金莲看在眼里。

早晨,西门庆正陪应伯爵坐的,忽报宋御史差人来送贺黄太尉一桌金银酒器:两把金壶、两副金台盏、十副小银钟、两副银折盂、四副银赏钟;两匹大红彩蟒、两匹金缎、十坛酒、两牵羊。传报:“太尉船只已到东昌地方,烦老爹这里早备酒席,准在十八日迎请。”西门庆收入明白,与了来人一两银子,用手本打发回去。随即兑银与贲四、来兴儿,定桌面,粘果品,买办整理,不必细说。因向伯爵说:“自从他不好起,到而今,我再没一日儿心闲。刚刚打发丧事出去了,又钻出这等勾当来,教我手忙脚乱。”伯爵道:“这个哥不消抱怨,你又不曾兜揽他,他上门儿来央烦你。虽然你这席酒替他陪几两银子,到明日,休说朝廷一位钦差殿前大太尉来咱家坐一坐,只这山东一省官员,并巡抚巡按、人马散级,也与咱门户添许多光辉。”西门庆道:“不是此说,我承望他到二十已外也罢,不想十八日就迎接,忒促急促忙。这日又是他五七,我已与了吴道官写法银子去了,如何又改!不然,双头火杖都挤在一处,怎乱得过来?”应伯爵道:“这个不打紧,我算来,嫂子是九月十七日没了,此月二十一日正是五七。你十八日摆了酒,二十日与嫂子念经也不迟。”西门庆道:“你说的是,我就使小厮回吴道官改日子去。”伯爵道:“哥,我又一件:东京黄真人,朝廷差他来泰安州进金铃吊挂御香,建七昼夜罗天大醮,如今在庙里住。趁他未起身,倒好教吴道官请他那日来做高功,领行法事。咱图他个名声,也好看。”西门庆道:“都说这黄真人有利益,请他到好,争奈吴道官斋日受他祭礼,出殡又起动他悬真,道童送殡,没的酬谢他,教他念这个经儿,表意而已。今又请黄真人主行,却不难为他?”伯爵道:“斋一般还是他受,只教他请黄真人做高功就是了。哥只多费几两银子,为嫂子,没曾为了别人。”西门庆一面教陈敬济写帖子,又多封了五两银子,教他早请黄真人,改在二十日念经,二十四众道士,水火炼度一昼夜。即令玳安骑头口去了。

西门庆打发伯爵去讫,进入后边。只见吴月娘说:“贲四嫂买了两个盒儿,他女儿长姐定与人家,来磕头。”西门庆便问:“谁家?”贲四娘子领他女儿,穿着大红缎袄儿、黄绸裙子,戴着花翠,插烛向西门庆磕了四个头。月娘在旁说:“咱也不知道,原来这孩子与了夏大人房里抬举,昨日才相定下。这二十四日就娶过门,只得了他三十两银子。论起来,这孩子倒也好身量,不象十五岁,到有十六七岁的。多少时不见,就长的成成的。”西门庆道:“他前日在酒席上和我说,要抬举两个孩子学弹唱,不知你家孩子与了他。”于是教月娘让至房内,摆茶留坐。落后,李娇儿、孟玉楼、潘金莲、孙雪娥、大姐都来见礼陪坐。临去,月娘与了一套重绢衣服、一两银子,李娇儿众人都有与花翠、汗巾、脂粉之类。晚上,玳安回话:“吴道官收了银子,知道了。黄真人还在庙里住,过二十头才回东京去。十九日早来铺设坛场。”

西门庆次日,家中厨役落作治办酒席,务要齐整,大门上扎七级彩山,厅前五级彩山。十七日,宋御史差委两员县官来观看筵席:厅正面,屏开孔雀,地匝氍毹,都是锦绣桌帏,妆花椅甸。黄太尉便是肘件大饭簇盘、定胜方糖,吃看大插桌;观席两张小插桌,是巡抚、巡按陪坐;两边布按三司,有桌席列坐。其余八府官,都在厅外棚内两边,只是五果五菜平头桌席。看毕,西门庆待茶,起身回话去了。

到次日,抚按率领多官人马,早迎到船上,张打黄旗“钦差”二字,捧着敕书在头里走,地方统制、守御、都监、团练,各卫掌印武官,皆戎服甲胄,各领所部人马,围随,仪杖摆数里之远。黄太尉穿大红五彩双挂绣蟒,坐八抬八簇银顶暖轿,张打茶褐伞。后边名下执事人役跟随无数,皆骏骑咆哮,如万花之灿锦,随鼓吹而行。黄土塾道,鸡犬不闻,樵采遁迹。人马过东平府,进清河县,县官黑压压跪于道旁迎接,左右喝叱起去。随路传报,直到西门庆门首。教坊鼓乐,声震云霄,两边执事人役皆青衣排伏,雁翅而列。西门庆青衣冠冕,望尘拱伺。良久,人马过尽,太尉落轿进来,后面抚按率领大小官员,一拥而入。到于厅上,又是筝\textXiaoQin 、方晌、云璈、龙笛、凤管,细乐响动。为首就是山东巡抚都御史侯濛、巡按监察御史宋乔年参见,大尉还依礼答之。其次就是山东左布政龚共、左参政何其高、右布政陈四箴、右参政季侃廷、参议冯廷鹄、右参议汪伯彦、廉使赵讷、采访使韩文光、提学副使陈正汇、兵备副使雷启元等两司官参见,太尉稍加优礼。及至东昌府徐崧、东平府胡师文、兖州府凌云翼、徐州府韩邦奇、济南府张叔夜、青州府王士奇、登州府黄甲、莱州府叶迁等八府官行厅参之礼,太尉答以长揖而已。至于统制、制置、守御、都监、团练等官,太尉则端坐。各官听其发放,外边伺候。然后,西门庆与夏提刑上来拜见献茶,侯巡抚、宋巡按向前把盏,下边动鼓乐,来与太尉簪金花,捧玉\textuni{659D},彼此酬饮。递酒已毕,太尉正席坐下,抚按下边主席,其余官员并西门庆等,各依次第坐了。教坊伶官递上手本奏乐,一应弹唱队舞,各有节次,极尽声容之盛。当筵搬演《裴晋公还带记》,一折下来,厨役割献烧鹿、花猪、百宝攒汤、大饭烧卖。又有四员伶官,筝\textXiaoQin 、琵琶、箜篌,上来清弹小唱。

唱毕,汤未两陈,乐已三奏。下边跟从执事人等,宋御史差两员州官,在西门庆卷棚内自有桌席管待。守御、都监等官,西门庆都安在前边客位,自有坐处。黄太尉令左右拿十两银子来赏赐各项人役,随即看轿起身。众官再三款留不住,即送出大门。鼓乐笙簧迭奏,两街仪卫喧阗,清跸传道,人马森列。多官俱上马远送,太尉悉令免之,举手上轿而去。

宋御史、候巡抚吩咐都监以下军卫有司,直护送至皇船上来回话。桌面器皿,答贺羊酒,具手本差东平府知府胡师文与守御周秀,亲送到船所,交付明白。回至厅上,拜谢西门庆说:“今日负累取扰,深感,深感!分资有所不足,容当奉补。”西门庆慌躬身施礼道:“卑职重承教爱,累辱盛仪,日昨又蒙赙礼,蜗居卑陋,犹恐有不到处,万里公祖谅宥,幸甚!”宋御史谢毕,即令左右看轿,与候巡抚一同起身,两司八府官员皆拜辞而去。各项人役,一哄而散。西门庆回至厅上,将伶官乐人赏以酒食,俱令散了,止留下四名官身小优儿伺候。厅内外各官桌面,自有本官手下人领不题。

西门庆见天色尚早,收拾家伙停当,攒下四张桌席,使人请吴大舅、应伯爵、谢希大、温秀才、傅自新、甘出身、韩道国、贲四、崔本及女婿陈敬济,——从五更起来,各项照管辛苦,坐饮三杯。不一时,众人来到,摆上酒来饮酒。伯爵道:“哥,今日黄太尉坐了多大一回?欢喜不欢喜?”韩道国道:“今日六黄老公公见咱家酒席齐整,无个不欢喜的。巡抚、巡按两位甚是知感不尽,谢了又谢。”伯爵道:“若是第二家摆这席酒也成不的,也没咱家恁大地方,也没府上这些人手。今日少说也有上千人进来,都要管待出去。哥就陪了几两银子,咱山东一省也响出名去了。”温秀才道:“学生宗主提学陈老先生,也在这里预席。”西门庆问其名,温秀才道:“名陈正汇者,乃谏垣陈了翁先生乃郎,本贯河南鄄城县人,十八岁科举,中壬辰进士,今任本处提学副使,极有学问。”西门庆道:“他今年才二十四岁?”正说着,汤饭上来。

众人吃毕,西门庆叫上四个小优儿,问道:“你四人叫甚名字?”答道:“小的叫周采、梁铎、马真、韩毕。”伯爵道:“你不是韩金钏儿一家?”韩毕跪下说道:“金钏儿、玉钏儿是小的妹子。”西门庆因想起李瓶儿来:“今日摆酒,就不见他。”吩咐小优儿:“你们拿乐器过来,唱个‘洛阳花,梁园月’我听。”韩毕与周采一面搊筝拨阮,唱道:

\[
\cipaim{普天乐}洛阳花,梁园月。好花须买,皓月须赊。花倚栏杆看烂熳开,月曾把酒问团圞夜。月有盈亏,花有开谢。想人生最苦离别。花谢了,三春近也;月缺了,中秋到也;人去了,何日来也?
\]
唱毕,应伯爵见西门庆眼里酸酸的,便道:“哥教唱此曲,莫非想起过世嫂子来?”西门庆看见后边上果碟儿,叫:“应二哥,你只嗔我说,有他在,就是他经手整定。从他没了,随着丫鬟撮弄,你看象甚模样?好应口菜也没一根我吃!”温秀才道:“这等盛设,老先生中馈也不谓无人,足可以够了。”伯爵道:“哥休说此话。你心间疼不过,便是这等说,恐一时冷淡了别的嫂子们心。”

这里酒席上说话,不想潘金莲在软壁后听唱,听见西门庆说此话,走到后边,一五一十告诉月娘。月娘道:“随他说去就是了,你如今却怎样的?前日他在时,即许下把绣春教伏侍李娇儿,他到睁着眼与我叫,说:‘死了多少时,就分散他房里丫头!’教我就一声儿再没言语。这两日凭着他那媳妇子和两个丫头,狂的有些样儿?我但开口,就说咱们挤撮他。”金莲道:“这老婆这两日有些别改模样,只怕贼没廉耻货,镇日在那屋里,缠了这老婆也不见的。我听见说,前日与了他两对簪子,老婆戴在头上,拿与这个瞧,拿与那个瞧。”月娘道:“豆芽菜儿——有甚捆儿!”众人背地里都不喜欢。正是:

\[
遗踪堪入时人眼,多买胭脂画牡丹。
\]

\newpage
%# -*- coding:utf-8 -*-
%%%%%%%%%%%%%%%%%%%%%%%%%%%%%%%%%%%%%%%%%%%%%%%%%%%%%%%%%%%%%%%%%%%%%%%%%%%%%%%%%%%%%


\chapter{翟管家寄书致赙\KG 黄真人发牒荐亡}


词曰:

\[
胸中千种愁,挂在斜阳树。绿叶阴阴自得春,草满莺啼处。
不见凌波步,空想如簧语。门外重重叠叠山,遮不断愁来路。
\]

话说西门庆陪吴大舅、应伯爵等饮酒中间,因问韩道国:“客伙中标船几时起身?咱好收拾打包。”韩道国道:“昨日有人来会,也只在二十四日开船。”西门庆道:“过了二十念经,打包便了。”伯爵问道:“这遭起身,那两位去?”西门庆道:“三个人都去。明年先打发崔大哥押一船杭州货来,他与来保还往松江下五处,置买些布货来卖。家中缎货绸绵都还有哩。”伯爵道:“哥主张极妙。常言道:要的般般有,才是买卖。”说毕,已有起更时分,吴大舅起身说:“姐夫连日辛苦,俺每酒已够了,告回,你可歇息歇息。”西门庆不肯,还留住,令小优儿奉酒唱曲,每人吃三钟才放出门。西门庆赏小优四人六钱银子,再三不敢接,说:“宋爷出票叫小的每来,官身如何敢受老爹重赏?”西门庆道:“虽然官差,此是我赏你,怕怎的!”四人方磕头领去。西门庆便归后边歇去了。

次日早起往衙门中去,早有吴道官差了一个徒弟、两名铺排,来大厅上铺设坛场,铺设的齐齐整整。西门庆来家看见,打发徒弟铺排斋食吃了回去。随即令温秀才写帖儿,请乔大户、吴大舅、吴二舅、花大舅、沈姨夫、孟二舅、应伯爵、谢希大、常峙节、吴舜臣许多亲眷并堂客,明日念经。家中厨役落作,治办斋供不题。

次日五更,道众皆来,进入经坛内,明烛焚香,打动响乐,讽诵诸经,铺排大门首挂起长幡,悬吊榜文,两边黄纸门对一联,大书:

\[
东极垂慈仙识乘晨而超登紫府;
南丹赦罪净魄受炼而迳上朱陵。
\]
大厅经坛,悬挂斋题二十字,大书:“青玄救苦、颁符告简、五七转经、水火炼度荐扬斋坛。”即日,黄真人穿大红,坐牙轿,系金带,左右围随,仪从暄喝,日高方到。吴道官率众接至坛所,行礼毕,然后西门庆着素衣絰巾,拜见递茶毕。洞案旁边安设经筵法席,大红销金桌围,妆花椅褥,二道童侍立左右。发文书之时,西门庆备金缎一匹;登坛之时,换了九阳雷巾,大红金云白百鹤法氅。先是表白宣毕斋意,斋官沐手上香。然后黄真人焚香净坛,飞符召将,关发一应文书符命,启奏三天,告盟十地。三献礼毕,打动音乐,化财行香。西门庆与陈敬济执手炉跟随,排军喝路,前后四把销金伞、三对缨络挑搭。行香回来,安请监斋毕,又动音乐,往李瓶儿灵前摄召引魂,朝参玉陛,旁设几筵,闻经悟道。到了午朝,高功冠裳,步罡踏斗,拜进朱表,遣差神将,飞下罗酆。原来黄真人年约三旬,仪表非常,妆束起来,午朝拜表,俨然就是个活神仙。但见:

\[
星冠攒玉叶,鹤氅缕金霞。神清似长江皓月,貌古如太华乔松。踏罡朱履进丹霄,步虚琅函浮瑞气。长髯广颊,修行到无漏之天;皓齿明眸,佩箓掌五雷之令。三更步月鸾声远,万里乘云鹤背高。就是都仙太史临凡世,广惠真人降下方。
\]

拜了表文,吴道官当坛颁生天宝箓神虎玉札。行毕午香,卷棚内摆斋。黄真人前,大桌面定胜;吴道官等,稍加差小;其余散众,俱平头桌席。黄真人、吴道官皆衬缎尺头、四对披花、四匹丝绸,散众各布一匹。桌面俱令人抬送庙中,散众各有手下徒弟收入箱中,不必细说。

吃毕午斋,都往花园内游玩散食去了。一面收下家火,从新摆上斋馔,请吴大舅等众亲朋伙计来吃。正吃之间,忽报:“东京翟爷那里差人下书。”西门庆即出厅上,请来人进来。只见是府前承差干办,青衣窄裤,万字头巾,乾黄靴,全副弓箭,向前施礼。西门庆答礼相还。那人向身边取出书来递上,又是一封折赙仪银十两。问来人上姓,那人道:“小人姓王名玉,蒙翟爷差遣,送此书来。不知老爹这边有丧事,安老爹书到才知。”西门庆问道:“你安老爹书几时到的?”那人说:“十月才到京。因催皇木一年已满,升都水司郎中。如今又奉敕修理河道,直到工完回京。”西门庆问了一遍,即令来保厢房中管待斋饭,吩咐明日来讨回书。那人问:“韩老爹在那里住?宅内捎信在此。小的见了,还要赶往东平府下书去。”西门庆即唤出韩道国来见那人,陪吃斋饭毕,同往家中去了。

西门庆拆看书中之意,于是乘着喜欢,将书拿到卷棚内教温秀才看。说:“你照此修一封回书答他,就捎寄十方绉纱汗巾、十方绫汗巾、十副拣金挑牙、十个乌金酒杯作回奉之礼。他明日就来取回书。”温秀才接过书来观看,其书曰:

\[
寓京都眷生翟谦顿首,书奉即擢大锦堂西门四泉亲家大人门下:自京邸话别之后,未得从容相叙,心甚歉然。其领教之意,生已于家老爷前悉陈之矣。迩者,安凤山书到,方知老亲家有鼓盆之叹,但恨不能一吊为怅,奈何,奈何!伏望以礼节哀可也。外具赙仪,少表微忱,希管纳。又久仰贵任荣修德政,举民有五绔之歌,境内有三留之誉,今岁考绩,必有甄升。昨日神运都功,两次工上,生已对老爷说了,安上亲家名字。工完题奏,必有恩典,亲家必有掌刑之喜。夏大人年终类本,必转京堂指挥列衔矣。谨此预报,伏惟高照,不宣。
\]
附云:

\[
此书可自省览,不可使闻之于渠。谨密,谨密!
\]
又云:

\[
杨老爷前月二十九日卒于狱。\named{冬上浣具}
\]

温秀才看毕,才待袖,早被应伯爵取过来,观看了一遍,还付与温秀才收了。说道:“老先生把回书千万加意做好些。翟公府中人才极多,休要教他笑话。”温秀才道:“貂不足,狗尾续。学生匪才,焉能在班门中弄大斧!不过乎塞责而已。”西门庆道:“温老先他自有个主意,你这狗才晓的甚么!”须臾,吃罢午斋,西门庆吩咐来兴儿打发斋馔,送各亲眷街邻。又使玳安回院中李桂姐、吴银儿、郑爱月儿、韩钏儿、洪四儿、齐香儿六家香仪人情礼去。每家回答一匹大布、一两银子。

后晌,就叫李铭、吴惠、郑奉三个小优儿来伺候。良久,道众升坛发擂,上朝拜忏观灯,解坛送圣。天色渐晚。比及设了醮,就有起更天气。门外花大舅被西门庆留下不去了,乔大户、沈姨夫、孟二舅告辞回家。止有吴大舅、二舅、应伯爵、谢希大、温秀才、常峙节并众伙计在此,晚夕观看水火练度。就在大厅棚内搭高座,扎彩桥,安设水池火沼,放摆斛食。李瓶儿灵位另有几筵帏幕,供献齐整。旁边一首魂幡、一首红幡、一首黄幡,上书“制魔保举,受炼南宫”。先是道众音乐,两边列座,持节捧盂剑,四个道童侍立两边。黄真人头戴黄金降魔冠,身披绛绡云霞衣,登高座,口中念念有词。宣偈云:

\[
太乙慈尊降驾来,夜壑幽关次第开。
童子双双前引导,死魂受炼步云阶。
\]

宣偈毕,又熏沐焚香,念曰:“伏以玄皇阐教,广开度于冥途;正一垂科,俾炼形而升举。恩沾幽爽,泽被饥嘘。谨运真香,志诚上请东极大慈仁者太乙救苦天尊、十方救苦诸真人圣众,仗此真香,来临法会。切以人处尘凡,日萦俗务,不知有死,惟欲贪生。鲜能种于善根,多随入于恶趣,昏迷弗省,恣欲贪嗔。将谓自己长存,岂信无常易到!一朝倾逝,万事皆空。业障缠身,冥司受苦。今奉道伏为亡过室人李氏灵魂,一弃尘缘,久沦长夜。若非荐拔于愆辜,必致难逃于苦报。恭惟天尊秉好生之仁,救寻声之苦。洒甘露而普滋群类,放瑞光而遍烛昏衢。命三官宽考较之条,诏十殿阁推研之笔。开囚释禁,宥过解冤。各随符使,尽出幽关。咸令登火池之沼,悉荡涤黄华之形。凡得更生,俱归道岸。兹焚灵宝炼形真符,谨当宣奏:

\[
太微回黄旗,无英命灵幡,
摄召长夜府,开度受生魂。”
\]

道众先将魂幡安于水池内,焚结灵符,换红幡;次于火沼内焚郁仪符,换黄幡。高功念:“天一生水,地二生火,水火交炼,乃成真形。”炼度毕,请神主冠帔步金桥,朝参玉陛,皈依三宝,朝玉清,众举《五供养》。举毕,高功曰:“既受三皈,当宣九戒。”九戒毕,道众举音乐,宣念符命并《十类孤魂》。炼度已毕,黄真人下高座,道众音乐送至门外,化财焚烧箱库。

回来,斋功圆满,道众都换了冠服,铺排收卷道像。西门庆又早大厅上画烛齐明,酒筵罗列。三个小优弹唱,众亲友都在堂前。西门庆先与黄真人把盏,左右捧着一匹天青云鹤金缎、一匹色缎、十两白银,叩首下拜道:“亡室今日赖我师经功救拔,得遂超生,均感不浅,微礼聊表寸心。”黄真人道:“小道谬忝冠裳,滥膺玄教,有何德以达人天?皆赖大人一诚感格,而尊夫人已驾景朝元矣。此礼若受,实为赧颜。”西门庆道:“此礼甚薄,有亵真人,伏乞笑纳!”黄真人方令小童收了。西门庆递了真人酒,又与吴道官把盏,乃一匹金缎、五两白银,又是十两经资。吴道官只受经资,余者不肯受,说:“小道素蒙厚爱,自恁效劳诵经,追拔夫人往生仙界,以尽其心。受此经资尚为不可,又岂敢当此盛礼乎!”西门庆道:“师父差矣。真人掌坛,其一应文简法事,皆乃师父费心。此礼当与师父酬劳,何为不可?”吴道官不得已,方领下,再三致谢。西门庆与道众递酒已毕,然后吴大舅、应伯爵等上来与西门庆散福递酒。吴大舅把盏,伯爵执壶,谢希大捧菜,一齐跪下。伯爵道:“嫂子今日做此好事,幸请得真人在此,又是吴师父费心,嫂子自得好处。此虽赖真人追荐之力,实是哥的虔心,嫂子的造化。”于是满斟一杯送与西门庆。西门庆道:“多蒙列位连日劳神,言谢不尽。”说毕,一饮而尽。伯爵又斟一盏,说:“哥,吃个双杯,不要吃单杯。”谢希大慌忙递一箸菜来吃了。西门庆回敬众人毕,安席坐下。小优弹唱起来,厨役上割道。当夜在席前猜拳行令,品竹弹丝,直吃到二更时分,西门庆已带半酣,众人方作辞起身而去。西门庆进来赏小优儿三钱银子,往后边去了。正是:

\[
人生有酒须当醉,一滴何曾到九泉。
\]


\newpage
%# -*- coding:utf-8 -*-
%%%%%%%%%%%%%%%%%%%%%%%%%%%%%%%%%%%%%%%%%%%%%%%%%%%%%%%%%%%%%%%%%%%%%%%%%%%%%%%%%%%%%


\chapter{西门庆书房赏雪\KG 李瓶儿梦诉幽情}


词曰:

\[
朔风天,琼瑶地。冻色连波,波上寒烟砌。山隐彤云云接水,衰草无情,想在彤云内。黯香魂,追苦意。夜夜除非,好梦留人睡。残月高楼休独倚,酒入愁肠,化作相思泪。
\]

话说西门庆归后边,辛苦的人,直睡至次日日高还未起来。有来兴儿进来说:“搭彩匠外边伺候,请问拆棚。”西门庆骂了来兴儿几句,说:“拆棚教他拆就是了,只顾问怎的!”搭彩匠一面卸下席绳松条,送到对门房子里堆放不题。玉箫进房说:“天气好不阴的重。”西门庆令他向暖炕上取衣裳穿,要起来。月娘便说:“你昨日辛苦了一夜,天阴,大睡回儿也好。慌的老早爬起去做甚么?就是今日不往衙门里去也罢了。”西门庆道:“我不往衙门里去,只怕翟亲家那人来讨书。”月娘道:“既是恁说,你起去,我去叫丫鬟熬下粥等你吃。”西门庆也不梳头洗面,披着绒衣,戴着毡巾,径走到花园里书房中。

原来自从书童去了,西门庆就委王经管花园书房,春鸿便收拾大厅前书房。冬月间,西门庆只在藏春阁书房中坐。那里烧下地炉暖炕,地平上又放着黄铜火盆,放下油单绢暖帘来。明间内摆着夹枝桃,各色菊花,清清瘦竹,翠翠幽兰,里面笔砚瓶梅,琴书潇洒。西门庆进来,王经连忙向流金小篆炷爇龙涎。西门庆使王经:“你去叫来安儿请你应二爹去。”王经出来吩咐来安儿请去了。只见平安走来对王经说:“小周儿在外边伺候。”王经走入书房对西门庆说了,西门庆叫进小周儿来,磕了头,说道:“你来得好,且与我篦篦头,捏捏身上。”因说:“你怎一向不来?”小周儿道:“小的见六娘没了,忙,没曾来。”西门庆于是坐在一张醉翁椅上,打开头发教他整理梳篦。只见来安儿请的应伯爵来了,头戴毡帽,身穿绿绒袄子,脚穿一双旧皂靴棕套,掀帘子进来唱喏。西门庆正篦头,说道:“不消声喏,请坐。”伯爵拉过一张椅子来,就着火盆坐下。西门庆道:“你今日如何这般打扮?”伯爵道:“你不知,外边飘雪花儿哩,好不寒冷。昨日家去,鸡也叫了,今日白爬不起来。不是大官儿去叫,我还睡哩。哥,你好汉,还起的早。若是我,成不的。”西门庆道:“早是你看着,我怎得个心闲!自从发送他出去了,又乱着接黄太尉,念经,直到如今。今日房下说:‘你辛苦了,大睡回起去。’我又记挂着翟亲家人来讨回书,又看着拆棚,二十四日又要打发韩伙计和小价起身。丧事费劳了人家,亲朋罢了,士大夫官员,你不上门谢谢孝,礼也过不去。”伯爵道:“正是,我愁着哥谢孝这一节。少不的只摘拨谢几家要紧的,胡乱也罢了。其余相厚的,若会见,告过就是了。谁不知你府上事多,彼此心照罢。”

正说着,只见画童儿拿了两盏酥油白糖熬的牛奶子。伯爵取过一盏,拿在手内,见白潋潋鹅脂一般酥油飘浮在盏内,说道:“好东西,滚热!”呷在口里,香甜美味,那消气力,几口就喝没了。西门庆直待篦了头,又教小周儿替他取耳,把奶子放在桌上,只顾不吃。伯爵道:“哥且吃些不是?可惜放冷了。象你清晨吃恁一盏儿,倒也滋补身子。”西门庆道:“我且不吃,你吃了,停会我吃粥罢。”那伯爵得不的一声,拿在手中,又一吸而尽。西门庆取毕耳,又叫小周儿拿木滚子滚身上,行按摩导引之术。伯爵问道:“哥滚着身子,也通泰自在么?”西门庆道:“不瞒你说,象我晚夕身上常发酸起来,腰背疼痛,不着这般按捏,通了不得!”伯爵道:“你这胖大身子,日逐吃了这等厚味,岂无痰火!”西门庆道:“任后溪常说:‘老先生虽故身体魁伟,而虚之太极。’送了我一罐儿百补延龄丹,说是林真人合与圣上吃的,教我用人乳常清晨服。我这两日心上乱,也还不曾吃。你们只说我身边人多,终日有此事,自从他死了,谁有甚么心绪理论此事!”

正说着,只见韩道国进来,作揖坐下,说:“刚才各家都来会了,船已雇下,准在二十四日起身。”西门庆吩咐:“甘伙计攒下帐目,兑了银子,明日打包。”因问:“两边铺子里卖下多少银两?”韩道国说:“共凑六千余两。”西门庆道:“兑二千两一包,着崔本往湖州买绸子去。那四千两,你与来保往松江贩布,过年赶头水船来。你每人先拿五两银子,家中收拾行李去。”韩道国道:“又一件:小人身从郓王府,要正身上直,不纳官钱如何处?”西门庆道:“怎的不纳官钱?象来保一般也是郓王差事,他每月只纳三钱银子。”韩道国道:“保官儿那个,亏了太师老爷那边文书上注过去,便不敢缠扰。小人乃是祖役,还要勾当余丁。”西门庆道:“既是如此,你写个揭帖,我央任后溪到府中替你和王奉承说,把你名字注销,常远纳官钱罢。你每月只委人打米就是了。”韩伙计作揖谢了。伯爵道:“哥,你替他处了这件事,他就去也放心。”少顷,小周滚毕身上,西门庆往后边梳头去了,吩咐打发小周儿吃点心。

良久,西门庆出来,头戴白绒忠靖冠,身披绒氅,赏了小周三钱银子。又使王经:“请你温师父来。”不一时,温秀才峨冠博带而至。叙礼已毕,左右放桌儿,拿粥来,伯爵与温秀才上坐,西门庆关席,韩道国打横。西门庆吩咐来安儿:“再取一盏粥、一双筷儿,请姐夫来吃粥。”不一时,陈敬济来到,头戴孝巾,身穿白绸道袍,与伯爵等作揖,打横坐下。须臾吃了粥,收下家火去,韩道国起身去了。西门庆因问温秀才:“书写了不曾?”温秀才道:“学生已写稿在此,与老先生看过,方可誊真。”一面袖中取出,递与西门庆观看。其书曰:

\[
寓清河眷生西门庆端肃书复大硕德柱国云峰老亲丈大人先生台下:自从京邸邂逅,不觉违越光仪,倏忽半载。生不幸闺人不禄,特蒙亲家远致赙仪,兼领悔教,足见为我之深且厚也。感刻无任,而终身不能忘矣。但恐一时官守责成有所疏陋之处,企仰门墙有负荐拔耳,又赖在老爷钧前常为锦覆。则生始终蒙恩之处,皆亲家所赐也。今因便鸿谨候起居,不胜驰恋,伏惟照亮,不宣。外具扬州绉纱汗巾十方、色绫汗巾十方、拣金挑牙二十付、乌金酒钟十个,少将远意,希笑纳。
\]

西门庆看毕,即令陈敬济书房内取出人事来,同温秀才封了,将书誊写锦笺,弥封停当,印了图书。另外又封五两白银与下书人王玉,不在话下。

一回见雪下的大了,西门庆留下温秀才在书房中赏雪。揩抹桌儿,拿上案酒来。只见有人在暖帘外探头儿,西门庆问是谁,王经说:“是郑春。”西门庆叫他进来。那郑春手内拿着两个盒儿,举的高高的,跪在当面,上头又搁着个小描金方盒儿,西门庆问是甚么,郑春道:“小的姐姐月姐,知道昨日爹与六娘念经辛苦了,没甚么,送这两盒儿茶食儿来,与爹赏人。”揭开,一盒果馅顶皮酥、一盒酥油泡螺儿。郑春道:“此是月姐亲手拣的。知道爹好吃此物,敬来孝顺爹。”西门庆道:“昨日多谢你家送茶,今日你月姐费心又送这个来。”伯爵道:“好呀!拿过来,我正要尝尝!死了我一个女儿会拣泡螺儿,如今又是一个女儿会拣了。”先捏了一个放在口内,又拈了一个递与温秀才,说道:“老先儿,你也尝尝。吃了牙老重生,抽胎换骨。眼见希奇物,胜活十年人。”温秀才呷在口内,入口而化,说道:“此物出于西域,非人间可有。沃肺融心,实上方之佳味。”西门庆又问:“那小盒儿内是甚么?”郑春悄悄跪在西门庆跟前,递上盒儿,说:“此是月姐捎与爹的物事。”西门庆把盒子放在膝盖儿上,揭开才待观看,早被伯爵一手挝过去,打开是一方回纹锦同心方胜桃红绫汗巾儿,里面裹着一包亲口嗑的瓜仁儿。伯爵把汗巾儿掠与西门庆,将瓜仁两把喃在口里都吃了。比及西门庆用手夺时,只剩下没多些儿,便骂道:“怪狗才,你害馋痨馋痞!留些儿与我见见儿,也是人心。”伯爵道:“我女儿送来,不孝顺我,再孝顺谁?我儿,你寻常吃的够了。”西门庆道:“温先儿在此,我不好骂出来,你这狗才,忒不象模样!”一面把汗巾收入袖中,吩咐王经把盒儿掇到后边去。

不一时,杯盘罗列,筛上酒来。才吃了一巡酒,玳安儿来说:“李智、黄四关了银子,送银子来了。”西门庆问多少,玳安道:“他说一千两,余者再一限送来。”伯爵道:“你看这两个天杀的,他连我也瞒了不对我说。嗔道他昨日你这里念经他也不来,原来往东平府关银子去了。你今收了,也少要发银子出去了。这两个光棍,他揽的人家债多了,只怕往后后手不接。昨日,北边徐内相发恨,要亲往东平府自家抬银子去。只怕他老牛箍嘴箍了去,却不难为哥的本钱!”西门庆道:“我不怕他。我不管甚么徐内相李内相,好不好把他小厮提在监里坐着,不怕他不与我银子。”一面教陈敬济:“你拿天平出去收兑了他的就是了。我不出去罢。”

良久,陈敬济走来回话说:“银子已兑足一千两,交入后边,大娘收了。黄四说,还要请爹出去说句话儿。”西门庆道:“你只说我陪着人坐着哩。左右他只要捣合同,教他过了二十四日来罢。”敬济道:“不是。他说有桩事儿要央烦爹。”西门庆道:“甚么事?等我出去。”一面走到厅上,那黄四磕头起来,说:“银子一千两,姐夫收了。余者下单我还。小人有一桩事儿央烦老爹。”说着磕在地下哭了。西门庆拉起来道:“端的有甚么事,你说来。”黄四道:“小的外父孙清,搭了个伙计冯二,在东昌府贩绵花。不想冯二有个儿子冯淮,不守本分,要便锁了门出去宿娼。那日把绵花不见了两大包,被小人丈人说了两句,冯二将他儿子打了两下。他儿子就和俺小舅子孙文相厮打起来,把孙文相牙打落了一个,他亦把头磕伤。被客伙中解劝开了。不想他儿子到家,迟了半月,破伤风身死。他丈人是河西有名土豪白五,绰号白千金,专一与强盗做窝主,教唆冯二,具状在巡按衙门朦胧告下来,批雷兵备老爹问。雷老爹又伺候皇船,不得闲,转委本府童推官问。白家在童推官处使了钱,教邻见人供状,说小人丈人在旁喝声来。如今童推官行牌来提俺丈人。望乞老爹千万垂怜,讨封书对雷老爹说,宁可监几日,抽上文书去,还见雷老爹问,就有生路了。他两人厮打,委的不管小人丈人事,又系歇后身死,出于保辜限外。先是他父冯二打来,何必独赖孙文相一人身上?”西门庆看了说帖,写着:“东昌府见监犯人孙清、孙文相,乞青目。”因说:“雷兵备前日在我这里吃酒,我只会了一面,又不甚相熟,我怎好写书与他?”黄四就跪下哭哭啼啼哀告说:“老爹若不可怜见,小的丈人子父两个就都是死数了。如今随孙文相出去罢了,只是分豁小人外父出来,就是老爹莫大之恩。小人外父今年六十岁,家下无人,冬寒时月再放在监里,就死罢了。”西门庆沉吟良久,说:“也罢,我转央钞关钱老爹和他说说去——与他是同年,都是壬辰进士。”黄四又磕下头去,向袖中取出“一百石白米”帖儿递与西门庆,腰里就解两封银子来。西门庆不接,说道:“我那里要你这行钱!”黄四道:“老爹不稀罕,谢钱老爹也是一般。”西门庆道:“不打紧,事成我买礼谢他。”

正说着,只见应伯爵从角门首出来,说:“哥,休替黄四哥说人情。他闲时不烧香,忙时抱佛腿。昨日哥这里念经,连茶儿也不送,也不来走走儿,今日还来说人情!”那黄四便与伯爵唱喏,说道:“好二叔,你老人家杀人哩!我因这件事,整走了这半月,谁得闲来?昨日又去府里领这银子,今日一来交银子,就央说此事,救俺丈人。老爹再三不肯收这礼物,还是不下顾小人。”伯爵看见一百两雪花官银放在面前,因问:“哥,你替他去说不说?”西门庆道:“我与雷兵备不熟,如今要转央钞关钱主政替他说去。到明日,我买分礼谢老钱就是了,又收他礼做甚么?”伯爵道:“哥,你这等就不是了。难道他来说人情,哥你倒陪出礼去谢人?也无此道理。你不收,恰似嫌少的一般。你依我收下。虽你不稀罕,明日谢钱公也是一般。黄四哥在这里听着:看你外父和你小舅子造化,这一回求了书去,难得两个都没事出来。你老爹他恒是不稀罕你钱,你在院里老实大大摆一席酒,请俺们耍一日就是了。”黄四道:“二叔,你老人家费心,小人摆酒不消说,还叫俺丈人买礼来,磕头酬谢你老人家。不瞒说,我为他爷儿两个这一场事,昼夜替他走跳,还寻不出个门路来。老爹再不可怜怎了!”伯爵道:“傻瓜,你搂着他女儿,你不替他上紧谁上紧?”黄四道:“房下在家只是哭。”西门庆被伯爵说着,把礼帖收了,说礼物还令他拿回去。黄四道:“你老人家没见好大事,这般多计较!”就往外走。伯爵道:“你过来,我和你说:你书几时要?”黄四道:“如今紧等着救命,望老爹今日写了书,差下人,明早我使小儿同去走遭。不知差那位大官儿去,我会他会。”西门庆道:“我就替你写书。”因叫过玳安来吩咐:“你明日就同黄大官一路去。”

那黄四见了玳安,辞西门庆出门。走到门首,问玳安要盛银子的褡裢。玳安进入后边,月娘房里正与玉箫、小玉裁衣裳,见玳安站着等褡裢,玉箫道:“使着手,不得闲誊。教他明日来与他就是了。”玳安道:“黄四等紧着明日早起身东昌府去,不得来了,你誊誊与他罢。”月娘便说:“你拿与他就是了,只教人家等着。”玉箫道:“银子还在床地平上掠着不是?”走到里间,把银子往床上只一倒,掠出褡裢来,说:“拿了去!怪囚根子,那个吃了他这条褡裢,只顾立叮蚂蝗的要!”玳安道:“人家不要,那个好来取的!”于是拿了出去,走到仪门首,还抖出三两一块麻姑头银子来。原来纸包破了,怎禁玉箫使性子那一倒,漏下一块在褡裢底内。玳安道:“且喜得我拾个白财。”于是褪入袖中。到前边递与黄四,约会下明早起身。

且说西门庆回到书房中,即时教温秀才修了书,付与玳安不题。一面觑那门外下雪,纷纷扬扬,犹如风飘柳絮,乱舞梨花相似。西门庆另打开一坛双料麻姑酒,教春鸿用布甑筛上来,郑春在旁弹筝低唱,西门庆令他唱一套“柳底风微”。正唱着,只见琴童进来说:“韩大叔教小的拿了这个帖儿与爹瞧。”西门庆看了,吩咐:“你就拿往门外任医官家,替他说说去。央他明日到府中承奉处替他说说,注销差事。”琴童道:“今日晚了,小的明早去罢。”西门庆道:“明早去也罢。”不一时,来安儿用方盒拿了八碗下饭,又是两大盘玫瑰鹅油烫面蒸饼,连陈敬济共四人吃了。西门庆教王经盒盘儿拿两碗下饭、一盘点心与郑春吃,又赏了他两大钟酒。郑春跪禀:“小的吃不的。”伯爵道:“傻孩子,冷呵呵的,你爹赏你不吃。你哥他怎的吃来?”郑春道:“小的哥吃的,小的本吃不的。”伯爵道:“你只吃一钟罢,那一钟我教王经替你吃罢。”王经说道:“二爹,小的也吃不的。”伯爵道:“你这傻孩儿,你就替他吃些儿也罢。休说一个大分上,自古长者赐,少者不敢辞。”一面站起来说:“我好歹教你吃这一杯。”那王经捏着鼻子,一吸而饮。西门庆道:“怪狗才,小行货子他吃不的,只恁奈何他!”还剩下半盏,应伯爵教春鸿替他吃了,就要令他上来唱南曲。西门庆道:“咱每和温老先儿行个令,饮酒之时教他唱便有趣。”于是教王经取过骰盆儿,“就是温老先儿先起。”温秀才道:“学生岂敢僭,还从应老翁来。”因问:“老翁尊号?”伯爵道:“在下号南坡。”西门庆戏道:“老先生你不知,他孤老多,到晚夕桶子掇出来,不敢在左近倒,恐怕街坊人骂,教丫头直掇到大南首县仓墙底下那里泼去,因起号叫做‘南泼’。”温秀才笑道:“此‘坡’字不同。那‘泼’字乃点水边之‘发’,这‘坡’字却是‘土’字旁边着个‘皮’字。”西门庆道:“老先儿倒猜得着,他娘子镇日着皮子缠着哩。”温秀才笑道:“岂有此说?”伯爵道:“葵轩,你不知道,他自来有些快伤叔人家。”温秀才道:“自古言不亵不笑。”伯爵道:“老先儿,误了咱每行令,只顾和他说甚么,他快屎口伤人!你就在手,不劳谦逊。”温秀才道:“掷出几点,不拘诗词歌赋,要个‘雪’字,就照依点数儿上。说过来,饮一小杯;说不过来,吃一大盏。”温秀才掷了个幺点,说道:“学生有了:雪残鸂鶒亦多时。”推过去,该应伯爵行,掷出个五点来。伯爵想了半日,想不起来,说:“逼我老人家命也!”良久,说道:“可怎的也有了。”说道:“雪里梅花雪里开。——好不好?”温秀才道:“南老说差了,犯了两个‘雪’字,头上多了一个‘雪’字。”伯爵道:“头上只小雪,后来下大雪来了。”西门庆道:“这狗才,单管胡说。”教王经斟上大钟,春鸿拍手唱南曲《驻马听》:

\[
寒夜无茶,走向前村觅店家。这雪轻飘僧舍,密洒歌楼,遥阻归槎。江边乘兴探梅花,庭中欢赏烧银蜡。一望无涯,有似灞桥柳絮满天飞下。
\]

伯爵才待拿起酒来吃,只见来安儿后边拿了几碟果食,内有一碟酥油泡螺,又一碟黑黑的团儿,用桔叶裹着。伯爵拈将起来,闻着喷鼻香,吃到口犹如饴蜜,细甜美味,不知甚物。西门庆道:“你猜?”伯爵道:“莫非是糖肥皂?”西门庆笑道:“糖肥皂那有这等好吃。”伯爵道:“待要说是梅酥丸,里面又有核儿。”西门庆道:“狗才过来,我说与你罢,你做梦也梦不着。是昨日小价杭州船上捎来,名唤做衣梅。都是各样药料和蜜炼制过,滚在杨梅上,外用薄荷、桔叶包裹,才有这般美味。每日清晨噙一枚在口内,生津补肺,去恶味,煞痰火,解酒克食,比梅酥丸更妙。”伯爵道:“你不说,我怎的晓得。”因说:“温老先儿,咱再吃个儿。”教王经:“拿张纸儿来,我包两丸儿,到家捎与你二娘吃。”又拿起泡螺儿来问郑春:“这泡螺儿果然是你家月姐亲手拣的?”郑春跪下说:“二爹,莫不小的敢说谎?不知月姐费了多少心,只拣了这几个儿来孝顺爹。”伯爵道:“可也亏他,上头纹溜,就象螺蛳儿一般,粉红、纯白两样儿。”西门庆道:“我儿,此物不免使我伤心。惟有死了的六娘他会拣,他没了,如今家中谁会弄他!”伯爵道:“我头里不说的,我愁甚么?死了一个女儿会拣泡螺儿孝顺我,如今又钻出个女儿会拣了。偏你也会寻,寻的都是妙人儿。”西门庆笑的两眼没缝儿,赶着伯爵打,说:“你这狗才,单管只胡说。”温秀才道:“二位老先生可谓厚之至极。”伯爵道:“老先儿你不知,他是你小侄人家。”西门庆道:“我是他家二十年旧孤老。”陈敬济见二人犯言,就起身走了。那温秀才只是掩口而笑。

须臾,伯爵饮过大钟,次该西门庆掷骰儿。于是掷出个七点来,想了半日说:“我说《香罗带》上一句唱:‘东君去意切,梨花似雪。’”伯爵道:“你说差了,此在第九个字上了,且吃一大钟。”于是流沿儿斟了一银衢花钟,放在西门庆面前,教春鸿唱,说道:“我的儿,你肚子里裹枣核解板儿——能有几句!”春鸿又拍手唱了一个。看看饮酒至昏,掌烛上来。西门庆饮过,伯爵道:“姐夫不在,温老先生你还该完令。”温秀才拿起骰儿,掷出个幺点,想了想,见壁上挂着一幅吊屏,泥金书一联:“风飘弱柳平桥晚;雪点寒梅小院春。”就说了末后一句。伯爵道:“不算,不算,不是你心上发出来的。该吃一大钟。”春鸿斟上,那温秀才不胜酒力,坐在椅上只顾打盹,起来告辞。伯爵还要留他,西门庆道:“罢罢!老先儿他斯文人,吃不的。”令画童儿:“你好好送你温师父那边歇去。”温秀才得不的一声,作别去了。伯爵道:“今日葵轩不济,吃了多少酒儿?就醉了。”于是又饮够多时,伯爵起身说:“地下滑,我也酒够了。”因说:“哥,明日你早教玳安替他下书去。”西门庆道:“你不见我交与他书,明日早去了。”伯爵掀开帘子,见天阴地下滑,旋要了个灯笼,和郑春一路去。西门庆又与了郑春五钱银子,盒内回了一罐衣梅,捎与他姐姐郑月儿吃。临出门,西门庆因戏伯爵:“你哥儿两个好好去。”伯爵道:“你多说话。父子上山,各人努力。好不好,我如今就和郑月儿那小淫妇儿答话去。”说着,琴童送出门去了。

西门庆看收了家伙,扶着来安儿,打灯笼入角门,从潘金莲门首过,见角门关着,悄悄就往李瓶儿房里来。弹了弹门,绣春开了门,来安就出去了。西门庆进入明间,见李瓶儿影,就问:“供养了羹饭不曾?”如意儿就出来应道:“刚才我和姐供养了。”西门庆椅上坐了,迎春拿茶来吃了。西门庆令他解衣带,如意儿就知他在这房里歇,连忙收拾床铺,用汤婆熨的被窝暖洞洞的,打发他歇下。绣春把角门关了,都在明间地平上支着板凳,打铺睡下。西门庆要茶吃,两个已知科范,连忙撺掇奶子进去和他睡。老婆脱衣服钻入被窝内,西门庆乘酒兴服了药,那话上使了托子,老婆仰卧炕上,架起腿来,极力鼓捣,没高低扇磞,扇磞的老婆舌尖冰冷,淫水溢下,口中呼“达达”不绝。夜静时分,其声远聆数室。西门庆见老婆身上如绵瓜子相似,用一双胳膊搂着他,令他蹲下身子,在被窝内咂\textMaoJi \textMaoBa ,老婆无不曲体承奉。西门庆说:“我儿,你原来身体皮肉也和你娘一般白净,我搂着你,就如和他睡一般。你须用心伏侍我,我看顾你。”老婆道:“爹没的说,将天比地,折杀奴婢!奴婢男子汉已没了,爹不嫌丑陋,早晚只看奴婢一眼儿就够了。”西门庆便问:“你年纪多少?”老婆道:“我今年属免的,三十一岁了。”西门庆道:“你原来小我一岁。”见他会说话儿,枕上又好风月,心下甚喜。早晨起来,老婆伏侍拿鞋袜,打发梳洗,极尽殷勤,把迎春、绣春打靠后。又问西门庆讨葱白绸子:“做披袄子,与娘穿孝。”西门庆一一许他。就教小厮铺子里拿三匹葱白绸来:“你每一家裁一件。”瞒着月娘,背地银钱、衣服、首饰,甚么不与他!

次日,潘金莲就打听得知,走到后边对月娘说:“大姐姐,你不说他几句!贼没廉耻货,昨日悄悄钻到那边房里,与老婆歇了一夜。饿眼见瓜皮,甚么行货子,好的歹的揽搭下。不明不暗,到明日弄出个孩子来算谁的?又象来旺儿媳妇子,往后教他上头上脸,甚么张致!”月娘道:“你们只要栽派教我说,他要了死了的媳妇子,你每背地都做好人儿,只把我合在缸底下。我如今又做傻子哩!你每说只顾和他说,我是不管你这闲帐。”金莲见月娘这般说,一声儿不言语,走回房去了。

西门庆早起见天晴了,打发玳安往钱主事家下书去了。往衙门回来,平安儿来禀:“翟爹人来讨书。”西门庆打发书与他,因问那人:“你怎的昨日不来取?”那人说:“小的又往巡抚侯爷那里下书来,耽搁了两日。”说毕,领书出门。西门庆吃了饭就过对门房子里,看着兑银、打包、写书帐。二十四日烧纸,打发韩伙计、崔本并后生荣海、胡秀五人起身往南边去。写了一封书捎与苗小湖,就谢他重礼。

看看过了二十五六,西门庆谢毕孝,一日早晨,在上房吃了饭坐的。月娘便说:“这出月初一日,是乔亲家长姐生日,咱也还买份礼儿送了去。常言先亲后不改,莫非咱家孩儿没了,就断礼不送了?”西门庆道:“怎的不送!”于是吩咐来兴买四盒礼,又是一套妆花缎子衣服、两方销金汗巾、一盒花翠。写帖儿,叫王经送了去。这西门庆吩咐毕,就往花园藏春阁书房中坐的。只见玳安下了书回来回话,说:“钱老爹见了爹的帖子,随即写书差了一吏,同小的和黄四儿子到东昌府兵备道下与雷老爹。雷老爹旋行牌问童推官催文书,连犯人提上去从新问理。连他家儿子孙文相都开出来,只追了十两烧埋钱,问了个不应罪名,杖七十,罚赎。复又到钞关上回了钱老爹话,讨了回帖,才来了。”西门庆见玳安中用,心中大喜。拆开回帖观看,原来雷兵备回钱主事帖子都在里面。上写道:

\[
来谕悉已处分,但冯二已曾责子在先,何况与孙文相忿殴,彼此俱伤,歇后身死,又在保辜限外,问之抵命,难以平允。量追烧埋钱十两给与冯二,相应发落。谨此回覆。
\]
下书:“年侍生雷启元再拜。”

西门庆看了欢喜,因问:“黄四舅子在那里?”玳安道:“他出来都往家去了。明日同黄四来与爹磕头。黄四丈人与了小的一两银子。”西门庆吩咐置鞋脚穿,玳安磕头而出。西门庆就\textShouWai 在床炕上眠着了。王经在桌上小篆内炷了香,悄悄出来了。良久,忽听有人掀的帘儿响,只见李瓶儿蓦地进来,身穿糁紫衫、白绢裙,乱挽乌云,黄恹恹面容,向床前叫道:“我的哥哥,你在这里睡哩,奴来见你一面。我被那厮告了一状,把我监在狱中,血水淋漓,与秽污在一处,整受了这些时苦。昨日蒙你堂上说了人情,减我三等之罪。那厮再三不肯,发恨还要告了来拿你。我待要不来对你说,诚恐你早晚暗遭毒手。我今寻安身之处去也,你须防范他。没事少要在外吃夜酒,往那去,早早来家。千万牢记奴言,休要忘了!”说毕,二人抱头而哭。西门庆便问:“姐姐,你往那去?对我说。”李瓶儿顿脱,撒手却是南柯一梦。西门庆从睡梦中直哭醒来,看见帘影射入,正当日午,由不的心中痛切。正是:

\[
花落土埋香不见,镜空鸾影梦初醒。
\]
有诗为证:

\[
残雪初晴照纸窗,地炉灰烬冷侵床。
个中邂逅相思梦,风扑梅花斗帐香。
\]

不想早晨送了乔亲家礼,乔大户娘子使了乔通来送请帖儿,请月娘众姊妹。小厮说:“爹在书房中睡哩。”都不敢来问。月娘在后边管待乔通,潘金莲说:“拿帖儿,等我问他去。”于是蓦地推开书房门,见西门庆\textShouWai 着,他一屁股就坐在旁边,说:“我的儿,独自个自言自语,在这里做甚么?嗔道不见你,原来在这里好睡也!”一面说话,一面看着西门庆,因问:“你的眼怎生揉的恁红红的?”西门庆道:“想是我控着头睡来。”金莲道:“到只象哭的一般。”西门庆道:“怪奴才,我平白怎的哭?”金莲道:“只怕你一时想起甚心上人儿来是的。”西门庆道:“没的胡说,有甚心上人、心下人?”金莲道:“李瓶儿是心上的,奶子是心下的,俺们是心外的人,入不上数。”西门庆道:“怪小淫妇儿,又六说白道起来。”因问:“我和你说正经话——前日李大姐装椁,你每替他穿了甚么衣服在身底下来?”金莲道:“你问怎的?”西门庆道:“不怎的,我问声儿。”金莲道:“你问必有缘故。上面穿两套遍地金缎子衣服,底下是白绫袄、黄绸裙,贴身是紫绫小袄、白绢裙、大红小衣。”西门庆点了点头儿。金莲道:“我做兽医二十年,猜不着驴肚里病?你不想他,问他怎的?”西门庆道:“我才方梦见他来。”金莲道:“梦是心头想,喷涕鼻子痒。饶他死了,你还这等念他。象俺每都是可不着你心的人,到明日死了,苦恼也没那人想念!”西门庆向前一手搂过他脖子来,就亲个嘴,说:“怪小油嘴,你有这些贼嘴贼舌的。”金莲道:“我的儿,老娘猜不着你那黄猫黑尾的心儿!”两个又咂了一回舌头,自觉甜唾溶心,脂满香唇,身边兰麝袭人。西门庆于是淫心辄起,搂他在怀里。他便仰靠梳背,露出那话来,叫妇人品箫。妇人真个低垂粉头,吞吐裹没,往来鸣咂有声。西门庆见他头上戴金赤虎分心,香云上围着翠梅花钿儿,后鬓上珠翘错落,兴不可遏。正做到美处,忽见来安儿隔帘说:“应二爹来了。”西门庆道:“请进来。”慌的妇人没口子叫:“来安儿贼囚,且不要叫他进来,等我出去着。”来安儿道:“进来了,在小院内。”妇人道:“还不去教他躲躲儿!”那来安儿走去,说:“二爹且闪闪儿,有人在屋里。”这伯爵便走到松墙旁边,看雪培竹子。王经掀着软帘,只听裙子响,金莲一溜烟后边走了。正是:

\[
雪隐鹭鸶飞始见,柳藏鹦鹉语方知。
\]

伯爵进来,见西门庆,唱喏坐下。西门庆道:“你连日怎的不来?”伯爵道:“哥,恼的我要不的在这里。”西门庆问道:“又怎的恼?你告我说。”伯爵道:“紧自家中没钱,昨日俺房下那个,平白又桶出个孩儿来。白日里还好挝挠,半夜三更,房下又七痛八病。少不得扒起来收拾草纸被褥,叫老娘去。打紧应保又被俺家兄使了往庄子上驮草去了。百忙挝不着个人,我自家打灯笼叫了巷口邓老娘来。及至进门,养下来了。”西门庆问:“养个甚么?”伯爵道:“养了个小厮。”西门庆骂道:“傻狗才,生了儿子倒不好,如何反恼?是春花儿那奴才生的?”伯爵笑道:“是你春姨。”西门庆道:“那贼狗掇腿的奴才,谁教你要他来?叫叫老娘还抱怨!”伯爵道:“哥,你不知,冬寒时月,比不的你们有钱的人家,又有偌大前程,生个儿子锦上添花,便喜欢。俺们连自家还多着个影儿哩,要他做甚么!家中一窝子人口要吃穿,巴劫的魂也没了。应保逐日该操当他的差事去了,家兄那里是不管的。大小女便打发出去了,天理在头上,多亏了哥你。眼见的这第二个孩儿又大了,交年便是十三岁。昨日媒人来讨帖儿。我说:‘早哩,你且去着。’紧自焦的魂也没了,猛可半夜又钻出这个业障来。那黑天摸地,那里活变钱去?房下见我抱怨,没奈何,把他一根银挖儿与了老娘去了。明日洗三,嚷的人家知道了,到满月拿甚么使?到那日我也不在家,信信拖拖到那寺院里且住几日去罢。”西门庆笑道:“你去了,好了和尚来赶热被窝儿。你这狗才,到底占小便益儿。”又笑了一回,那应伯爵故意把嘴谷都着不做声。西门庆道:“我的儿,不要恼,你用多少银子,对我说,等我与你处。”伯爵道:“有甚多少?”西门庆道:“也够你搅缠是的。到其间不够了,又拿衣服当去。”伯爵道:“哥若肯下顾,二十两银子就够了,我写个符儿在此。费烦的哥多了,不好开口的,也不敢填数儿,随哥尊意便了。”西门庆也不接他文约,说:“没的扯淡,朋友家,什么符儿!”正说着,只见来安儿拿茶进来。西门庆叫小厮:“你放下盏儿,唤王经来。”不一时,王经来到。西门庆吩咐:“你往后边对你大娘说,我里间床背阁上,有前日巡按宋老爹摆酒两封银子,拿一封来。”王经应诺,不多时拿了银子来。西门庆就递与应伯爵,说:“这封五十两,你都拿了使去。原封未动,你打开看看。”伯爵道:“忒多了。”西门庆道:“多的你收着,眼下你二令爱不大了?你可也替他做些鞋脚衣裳,到满月也好看。”伯爵道:“哥说的是。”将银子拆开,都是两司各府倾就分资,三两一锭,松纹足色,满心欢喜,连忙打恭致谢,说道:“哥的盛情,谁肯!真个不收符儿?”西门庆道:“傻孩儿,谁和你一般计较?左右我是你老爷老娘家,不然你但有事就来缠我?这孩子也不是你的孩子,自是咱两个分养的。实和你说,过了满月,把春花儿那奴才叫了来,且答应我些时儿,只当利钱不算罢。”伯爵道:“你春姨这两日瘦的象你娘那样哩!”两个戏了一回,伯爵因问:“黄四丈人那事怎样了?”西门庆说:“钱龙野书到,雷兵备旋行牌提了犯人上去从新问理,把孙文相父子两个都开出来,只认了十两烧埋钱。”伯爵道:“造化他了。他就点着灯儿,那里寻这人情去!你不受他的,干不受他的。虽然你不稀罕,留送钱大人也好。别要饶了他,教他好歹摆一席大酒,里边请俺们坐一坐。你不说,等我和他说。饶了他小舅一个死罪,当别的小可事儿!”这里说话不题。

且说月娘在上房,只见孟玉楼走来,说他兄弟孟锐:“不久又起身往川广贩杂货去。今来辞辞他爹,在我屋里坐着哩。他在那里?姐姐使个小厮对他说声儿。”月娘道:“他在花园书房和应二坐着哩。又说请他爹哩,头里潘六姐到请的好!乔通送帖儿来,等着讨个话儿,到明日咱们好去不去。我便把乔通留下,打发吃茶,长等短等不见来,熬的乔通也去了。半日,只见他从前边走将来,教我问他:‘你对他说了不曾?’他没的话回,只哕了一声:‘我就忘了。’帖子还袖在袖子里。原来是恁个没尾巴行货子!不知前头干甚么营生,那半日才进来,恰好还不曾说。吃我讧了两句,往前去了。”少顷,来安进来,月娘使他请西门庆,说孟二舅来了。西门庆便起身,留伯爵:“你休去了,我就来。”走到后边,月娘先把乔家送帖来请说了。西门庆说:“那日只你一人去罢。热孝在身,莫不一家子都出来!”月娘说:“他孟二舅来辞辞你,一两日就起身往川广去。在三姐屋里坐着哩。”又问:“头里你要那封银子与谁?”西门庆道:“应二哥房里春花儿,昨晚生了个儿子,问我借几两银子使。告我说,他第二个女儿又大,愁的要不的。”月娘道:“好,好。他恁大年纪,也才见这个孩子,应二嫂不知怎的喜欢哩!到明日,咱也少不的送些粥米儿与他。”西门庆道:“这个不消说。到满月,不要饶花子,奈何他好歹发帖儿,请你们往他家走走去,就瞧瞧春花儿怎么模样。”月娘笑道:“左右和你家一般样儿,也有鼻儿也有眼儿,莫不差别些儿!”一面使来安请孟二舅来。

不一时,孟玉楼同他兄弟来拜见。叙礼已毕,西门庆陪他叙了回话,让至前边书房内与伯爵相见。吩咐小厮看菜儿,放桌儿筛酒上来,三人饮酒。西门庆教再取双钟箸:“对门请温师父陪你二舅坐。”来安不一时回说:“温师父不在,望倪师父去了。”西门庆说:“请你姐夫来坐坐。”良久,陈敬济来,与二舅见了礼,打横坐下。西门庆问:“二舅几时起身,去多少时?”孟锐道:“出月初二日准起身。定不的年岁,还到荆州买纸,川广贩香蜡,着紧一二年也不止。贩毕货就来家了。此去从河南、陕西、汉州去,回来打水路从峡江、荆州那条路来,往回七八千里地。”伯爵问:“二舅贵庚多少?”孟锐道:“在下虚度二十六岁。”伯爵道:“亏你年小小的,晓的这许多江湖道路,似俺们虚老了,只在家里坐着。”须臾添换上来,杯盘罗列,孟二舅吃至日西时分,告辞去了。

西门庆送了回来,还和伯爵吃了一回。只见买了两座库来,西门庆委付陈敬济装库。问月娘寻出李瓶儿两套锦衣,搅金银钱纸装在库内。因向伯爵说:“今日是他六七,不念经,烧座库儿。”伯爵道:“好快光阴,嫂子又早没了个半月了。”西门庆道:“这出月初五日是他断七,少不的替他念个经儿。”伯爵道:“这遭哥念佛经罢了。”西门庆道:“大房下说,他在时,因生小儿,许了些《血盆经忏》,许下家中走的两个女僧做首座,请几众尼僧,替他礼拜几卷忏儿罢了。”说毕,伯爵见天晚,说道:“我去罢。只怕你与嫂子烧纸。”又深深打恭说:“蒙哥厚情,死生难忘!”西门庆道:“难忘不难忘,我儿,你休推梦里睡哩!你众娘到满月那日,买礼都要去哩。”伯爵道:“又买礼做甚?我就头着地,好歹请众嫂子到寒家光降光降。”西门庆道:“到那日,好歹把春花儿那奴才收拾起来,牵了来我瞧瞧。”伯爵道:“你春姨他说来,有了儿子,不用着你了。”西门庆道:“不要慌,我见了那奴才和他答话。”伯爵笑的去了。

西门庆令小厮收了家伙,走到李瓶儿房里。陈敬济和玳安已把库装封停当。那日玉皇庙、永福寺、报恩寺都送疏来。西门庆看着迎春摆设羹饭完备,下出匾食来,点上香烛,使绣春请了吴月娘众人来。西门庆与李瓶儿烧了纸,抬出库去,教敬济看着,大门首焚化。正是:

\[
芳魂料不随灰死,再结来生未了缘。
\]

\newpage
%# -*- coding:utf-8 -*-
%%%%%%%%%%%%%%%%%%%%%%%%%%%%%%%%%%%%%%%%%%%%%%%%%%%%%%%%%%%%%%%%%%%%%%%%%%%%%%%%%%%%%


\chapter{应伯爵戏衔玉臂\KG 玳安儿密访蜂媒}


词曰:

\[
钟情太甚,到老也无休歇。月露烟云都是态,况与玉人明说。软语叮咛,柔情婉恋,熔尽肝肠铁。岐亭把盏,水流花谢时节。
\]

话说西门庆与李瓶儿烧纸毕,归潘金莲房中歇了一夜。到次日,先是应伯爵家送喜面来。落后黄四领他小舅子孙文相,宰了一口猪、一坛酒、两只烧鹅、四只烧鸡、两盒果子来与西门庆磕头。西门庆再三不受,黄四打旋磨儿跪着说:“蒙老爹活命之恩,举家感激不浅。无甚孝顺,些微薄礼,与老爹赏人,如何不受!”推阻了半日,西门庆止受猪酒:“留下送你钱老爹罢。”黄四道:“既是如此,难为小人一点穷心,无处所尽。”只得把羹果抬回去。又请问:“老爹几时闲暇?小人问了应二叔,里边请老爹坐坐。”西门庆道:“你休听他哄你哩!又费烦你,不如不央我了。”那黄四和他小舅子千恩万谢出门去了。

到十一月初一日,西门庆往衙门中回来,又往李知县衙内吃酒去,月娘独自一人,素妆打扮,坐轿子往乔大户家与长姐做生日,都不在家。到后晌,有庵里薛姑子,听见月娘许下他初五日念经拜《血盆忏》,于是悄悄瞒着王姑子,买了两盒礼物来见月娘。月娘不在家,李娇儿、孟玉楼留他吃茶,说:“大姐姐往乔亲家做生日去了。你须等他来,他还和你说话哩。”那薛姑子就坐住了。潘金莲思想着玉箫告他说,月娘吃了他的符水药才坐了胎气,又见西门庆把奶子要了,恐怕一时奶子养出孩子来,搀夺了他宠爱。于是把薛姑子让到前边他房里,悄悄央薛姑子,与他一两银子,替他配坐胎气符药,不在话下。

到晚夕,等的月娘回家,留他住了一夜。次日,问西门庆讨了五两银子经钱写法与他。这薛姑子就瞒着王姑子、大师父,到初五日早请了八众女僧,在花园卷棚内建立道场,讽诵《华严》、《金刚》经咒,礼拜《血盆》宝忏。晚夕设放焰口施食。那日请了吴大妗子、花大嫂并官客吴大舅、应伯爵、温秀才吃斋。尼僧也不动响器,只敲木鱼,击手馨,念经而已。

那日伯爵领了黄四家人,具帖初七日在院中郑爱月儿家置酒请西门庆。西门庆看了帖儿,笑道:“我初七日不得闲,张西村家吃生日酒。倒是明日空闲。”问还有谁,伯爵道:“再没人。只请了我与李三相陪哥,又叫了四个女儿唱《西厢记》。”西门庆吩咐与黄四家人斋吃了,打发回去,改了初六。伯爵便问:“黄四那日买了分甚么礼来谢你?”西门庆如此这般:“我不受他的,再三磕头礼拜,我只受了猪酒。添了两匹白鹇紵丝、两匹京缎、五十两银子,谢了龙野钱公了。”伯爵道:“哥,你不接钱尽够了,这个是他落得的。少说四匹尺头值三十两银子,那二十两,那里寻这分上去?便益了他,救了他父子二人性命!”当日坐至晚夕方散。西门庆向伯爵说:“你明日还到这边。”伯爵说:“我知道。”作别去了。八众尼僧直乱到一更多,方才道场圆满,焚烧箱库散了。

至次日,西门庆早往衙门中去了。且说王姑子打听得知,大清早晨走来,说薛姑子揽了经去,要经钱。月娘怪他道:“你怎的昨日不来?他说你往王皇亲家做生日去了。”王姑子道:“这个就是薛家老淫妇的鬼。他对着我说咱家挪了日子,到初六念经。难道经钱他都拿的去了,一些儿不留下?”月娘道:“还等到这咱哩?未曾念经,经钱写法就都找与他了。早是我还与你留下一匹衬钱布在此。”教小玉连忙摆了些昨日剩下的斋食与他吃了,把与他一匹蓝布。这王姑子口里喃喃呐呐骂道:“这老淫妇,他印造经,赚了六娘许多银子。原说这个经儿,咱两个使,你又独自掉揽的去了。”月娘道:“老薛说你接了六娘《血盆经》五两银子,你怎的不替他念?”王姑子道:“他老人家五七时,我在家请了四位师父,念了半个月哩。”月娘道:“你念了,怎的挂口儿不对我题?你就对我说,我还送些衬施儿与你。”那王姑子便一声儿不言语,讪讪的坐了一回,往薛姑子家嚷去了。正是:

\[
佛会僧尼是一家,法轮常转度龙华。
此物只好图生育,枉使金刀剪落花。
\]

却说西门庆从衙门中回来,吃了饭,应伯爵又早到了。盔的新缎帽,沉香色\textuni{2773D}褶,粉底皂靴,向西门庆声喏,说:“这天也有晌午,好去了。他那里使人邀了好几遍了。”西门庆道:“咱今邀葵轩同走走去。”使王经:“往对过请你温师父来。”王经去不多时,回说:“温师父不在家,望朋友去了。”伯爵便说:“咱等不的他。秀才家有要没紧望朋友,知多咱来?倒没的误了勾当。”西门庆吩咐琴童:“备黄马与应二爹骑。”伯爵道:“我不骑。你依我:省的摇铃打鼓,我先走一步儿,你坐轿子慢慢来就是了。”西门庆道:“你说的是,你先行罢。”那伯爵举手先走了。

西门庆吩咐玳安、琴童、四个排军,收拾下暖轿跟随。才待出门,忽平安儿慌慌张张从外拿着双帖儿来报,说:“工部安老爹来拜。先差了个吏送帖儿,后边轿子便来也。”慌的西门庆吩咐家中厨下备饭,使来兴儿买攒盘点心伺候。良久,安郎中来到,西门庆冠冕出迎。安郎中穿着妆花云鹭补子员领,起花萌金带,进门拜毕,分宾主坐定,左右拿茶上来。茶罢,叙其间阔之情。西门庆道:“老先生荣擢,失贺,心甚缺然。前日蒙赐华扎厚仪,生正值丧事,匆匆未及奉候起居为歉。”安郎中道:“学生有失吊问,罪罪!生到京也曾道达云峰,未知可有礼到否?”西门庆道:“正是,又承翟亲家远劳致赙。”安郎中道:“四泉一定今岁恭喜。”西门庆道,“在下才微任小,岂敢非望。”又说:“老先生荣擢美差,足展雄才。治河之功,天下所仰。”安郎中道:“蒙四泉过誉。一介寒儒,辱蔡老先生抬举,谬典水利,修理河道,当此民穷财尽之时。前者皇船载运花石,毁闸折坝,所过倒悬,公私困弊之极。又兼贼盗梗阻,虽有神输鬼役之才,亦无如之何矣。”西门庆道:“老先生大才展布,不日就绪,必大升擢矣。”因问:“老先生敕书上有期限否?”安郎中道:“三年钦限。河工完毕,圣上还要差官来祭谢河神。”说话中间,西门庆令放桌儿,安郎中道:“学生实说,还要往黄泰宇那里拜拜去。”西门庆道:“既如此,少坐片时,教从者吃些点心。”不一时,就是春盛案酒,一色十六碗下饭,金钟暖酒斟来,下人俱有攒盘点心酒肉。安郎中席间只吃了三钟,就告辞起身,说:“学生容日再来请教。”西门庆款留不住,送至大门首,上轿而去。回到厅上,解去冠带,换了巾帻,止穿紫绒狮补直身。使人问:“温师父来了不曾?”玳安回说:“温师父尚未回哩。有郑春和黄四叔家来定儿来邀,在这里半日了。”

西门庆即出门上轿,左右跟随,迳往郑爱月儿家来。比及进院门,架儿们都躲过一边,只该日俳长两边站立,不敢跪接。郑春与来定儿先通报去了。应伯爵正和李三打双陆,听见西门庆来,连忙收拾不及。郑爱月儿、爱香儿戴着海獭卧兔儿,一窝丝杭州攒,打扮的花仙也似,都出来门首迎接。西门庆下了轿,进入客位内。西门庆吩咐不消吹打,止住鼓乐。先是李三、黄四见毕礼数,然后郑家鸨子出来拜见了。才是爱月儿姊妹两个磕头。正面安放两张交椅,西门庆与应伯爵坐下,李智、黄四与郑家姊妹打横。玳安在旁禀问:“轿子在这里,回了家去?”西门庆令排军和轿子都回去,又吩咐琴童:“到家看你温师父来了,拿黄马接了来。”琴童应喏去了。伯爵因问:“哥怎的这半日才来?”西门庆悉把安郎中来拜留饭之事说了一遍。

须臾,郑春拿上茶来,爱香儿拿了一盏递与伯爵。爱月儿便递西门庆,那伯爵连忙用手去接,说:“我错接,只说你递与我来。”爱月儿道:“我递与你?——没修这样福来!”伯爵道:“你看这小淫妇儿,原来只认的他家汉子,倒把客人不着在意里。”爱月儿笑道:“今日轮不着你做客人哩!”吃毕茶,须臾四个唱《西厢》妓女都出来与西门庆磕头,一一问了姓名。西门庆对黄四说:“等住回上来唱,只打鼓儿,不吹打罢。”黄四道:“小人知道。”鸨子怕西门庆冷,又教郑春放下暖帘来,火盆内添上许多兽炭。只见几个青衣圆社听见西门庆在郑家吃酒,走来门首伺候,探头舒脑,不敢进去。有认得玳安的,向玳安打恭,央及作成作成。玳安悄俏进来替他禀问,被西门庆喝了一声,唬的众人一溜烟走了。不一时,收拾果品案酒上来,正面放两张桌席:西门庆独自一席,伯爵与温秀才一席——留下温秀才座位在左首。旁边一席李三和黄四,右边是他姊妹二人。端的肴堆异品,花插金瓶。郑奉、郑春在旁弹唱。

才递酒安席坐下,只见温秀才到了。头戴过桥巾,身穿绿云袄,进门作揖。伯爵道:“老先生何来迟也?留席久矣。”温秀才道:“学生有罪,不知老先生呼唤,适往敝同窗处会书,来迟了一步。”慌的黄四一面安放钟箸,与伯爵一处坐下。不一时,汤饭上来,两个小优儿弹唱一回下去。四个妓女才上来唱了一折“游艺中原”,只见玳安来说:“后边银姨那里使了吴惠和蜡梅送茶来了。”原来吴银儿就在郑家后边住,止隔一条巷。听见西门庆在这里吃酒,故使送茶。西门庆唤入里面,吴惠、蜡梅磕了头,说:“银姐使我送茶来爹吃。”揭开盒儿,斟茶上去,每人一盏瓜仁香茶。西门庆道:“银姐在家做甚么哩?”蜡梅道:“姐儿今日在家没出门。”西门庆吃了茶,赏了他两个三钱银子,即令玳安同吴惠:“你快请银姨去。”郑爱月儿急俐,便就教郑春:“你也跟了去,好歹缠了银姨来。他若不来,你就说我到明日就不和他做伙计了。”应伯爵道:“我倒好笑,你两个原来是贩\textuni{23B48}的伙计。”温秀才道:“南老好不近人情。自古同声相应,同气相求。本乎天者亲上,本乎地者亲下。同他做伙计亦是理之当然。”爱月儿道:“应花子,你与郑春他们都是伙计,当差供唱都在一处。”伯爵道:“傻孩子,我是老王八!那咱和你妈相交,你还在肚子里!”说笑中间,妓女又上来唱了一套“半万贼兵”。西门庆叫上唱莺莺的韩家女儿近前,问:“你是韩家谁的女儿?”爱香儿说:“爹,你不认的?他是韩金钏侄女儿,小名消愁儿,今年才十三岁。”西门庆道:“这孩子到明日成个好妇人儿。举止伶俐,又唱的好。”因令他上席递酒。黄四下汤下饭,极尽殷勤。

不一时,吴银儿来到。头上戴着白绉纱\textuni{4BFC}髻、珠子箍儿、翠云钿儿,周围撇一溜小簪儿。上穿白绫对衿袄儿,妆花眉子,下着纱绿潞绸裙,羊皮金滚边。脚上墨青素缎鞋儿。笑嘻嘻进门,向西门庆磕了头,后与温秀才等各位都道了万福。伯爵道:“我倒好笑,来到就教我惹气。俺每是后娘养的?只认的你爹,与他磕头,望着俺每只一拜。原来你这丽春院小娘儿这等欺客!我若有五棍儿衙门,定不饶你。”爱月儿叫:“应花子,好没羞的孩儿。你行头不怎么,光一味好撇。”一面安座儿,让银姐就在西门庆桌边坐下。西门庆见他戴着白\textuni{4BFC}髻,问:“你戴的谁人孝?”吴银儿道:“爹故意又问个儿,与娘戴孝一向了。”西门庆一闻与李瓶儿戴孝,不觉满心欢喜,与他侧席而坐,两个说话。

须臾汤饭上来,爱月儿下来与他递酒。吴银儿下席说:“我还没见郑妈哩。”一面走到鸨子房内见了礼,出来,鸨子叫:“月姐,让银姐坐。只怕冷,教丫头烧个火笼来,与银姐烤手儿。”随即添换热菜上来,吴银儿在旁只吃了半个点心,喝了两口汤。放下箸儿,和西门庆攀话道:“娘前日断七念经来?”西门庆道:“五七多谢你每茶。”吴银儿道:“那日俺每送了些粗茶,倒教爹把人情回了,又多谢重礼,教妈惶恐的要不的。昨日娘断七,我会下月姐和桂姐,也要送茶来,又不知宅内念经不念。”西门庆道:“断七那日,胡乱请了几位女僧,在家拜了拜忏。亲眷一个都没请,恐怕费烦。”饮酒说话之间,吴银儿又问:“家中大娘众娘每都好?”西门庆道:“都好。”吴银儿道:“爹乍没了娘,到房里孤孤儿的,心中也想么?”西门庆道:“想是不消说。前日在书房中,白日梦见他,哭的我要不的。”吴银儿道:“热突突没了,可知想哩!”伯爵道:“你每说的知情话,把俺每只顾旱着,不说来递钟酒,也唱个儿与俺听。俺每起身去罢!”慌的李三、黄四连忙撺掇他姐儿两个上来递酒。安下乐器,吴银儿也上来。三个粉头一般儿坐在席上,躧着火盆,合着声儿唱了套《中吕·粉蝶儿》“三弄梅花”,端的有裂石流云之响。

唱毕,西门庆向伯爵说:“你索落他姐儿三个唱,你也下来酬他一杯儿。”伯爵道:“不打紧,死不了人。等我打发他:仰靠着,直舒着,侧卧着,金鸡独立,随我受用;又一件,野马踩场,野狐抽丝,猿猴献果,黄狗溺尿,仙人指路,——哥,随他拣着要。”爱香道:“我不好骂出来的,汗邪了你这贼花子,胡说乱道的。”应伯爵用酒碟安三个钟儿,说:“我儿,你每在我手里吃两钟。不吃,望身上只一泼。”爱香道:“我今日忌酒。”爱月儿道:“你跪着月姨,教我打个嘴巴儿,我才吃。”伯爵道:“银姐,你怎的说?”吴银儿道:“二爹,我今日心里不自在,吃半盏儿罢。”爱月儿道:“花子,你不跪,我一百年也不吃。”黄四道:“二叔,你不跪,显的不是趣人。也罢,跪着不打罢。”爱月儿道:“跪了也不打多,只教我打两个嘴巴儿罢。”伯爵道:“温老先儿,你看着,怪小淫妇儿只顾赶尽杀绝。”于是奈何不过,真个直撅儿跪在地下。那爱月儿轻揎彩袖,款露春纤,骂道:“贼花子,再可敢无礼伤犯月姨了?——高声儿答应。你不答应,我也不吃。”伯爵无法可处,只得应声道:“再不敢伤犯月姨了。”这爱月儿方连打了两个嘴巴,方才吃那钟酒。伯爵起来道:“好个没仁义的小淫妇儿,你也剩一口儿我吃。把一钟酒都吃的净净儿的。”爱月儿道:“你跪下,等我赏你一钟吃。”于是满满斟上一杯,笑望伯爵口里只一灌。伯爵道,“怪小淫妇儿,使促狭灌撒了我一身。我老实说,只这件衣服,新穿了才头一日儿,就污浊了我的。我问你家汉子要。”笑了一回,各归席上坐定。

看看天晚,掌烛上来。西门庆吩咐取个骰盆来。先让温秀才,秀才道:“岂有此理!还从老先生来。”于是西门庆与银儿用十二个骰儿抢红,下边四个妓女拿着乐器弹唱。饮过一巡,吴银儿却转过来与温秀才、伯爵抢红,爱香儿却来西门庆席上递酒猜枚。须臾过去,爱月儿近前与西门庆抢红,吴银儿却往下席递李三、黄四酒。原来爱月几旋往房中新妆打扮出来,上着烟里火回纹锦对衿袄儿、鹅黄杭绢点翠缕金裙、妆花膝裤、大红凤嘴鞋儿,灯下海獭卧兔儿,越显的粉浓浓雪白的脸儿。真是:

\[
芳姿丽质更妖烧,秋水精神瑞雪标。
白玉生香花解语,千金良夜实难消。
\]
西门庆见了,如何不爱。吃了几钟酒,半酣上来,因想着李瓶儿梦中之言:少贪在外夜饮。一面起身后边净手。慌的鸨子连忙叫丫鬟点灯,引到后边。解手出来,爱月随即跟来伺候。盆中净手毕,拉着他手儿同到房中。

房中又早月窗半启,银烛高烧,气暖如春,兰麝馥郁,于是脱了上盖,止穿白绫道袍,两个在床上腿压腿儿做一处。先是爱月儿问:“爹今日不家去罢了。”西门庆道:“我还去。今日一者银儿在这里,不好意思;二者我居着官,今年考察在迩,恐惹是非,只是白日来和你坐坐罢了。”又说:“前日多谢你泡螺儿。你送了去,倒惹的我心酸了半日。当初止有过世六娘他会拣。他死了,家中再有谁会拣他!”爱月道:“拣他不难,只是要拿的着禁节儿便好。那瓜仁都是我口里一个个儿嗑的,说应花子倒挝了好些吃了。”西门庆道:“你问那讪脸花子,两把挝去喃了好些。只剩下没多,我吃了。”爱月儿道:“倒便益了贼花子,恰好只孝顺了他。”又说:“多谢爹的衣梅。妈看见吃了一个儿,欢喜的要不的。他要便痰火发了,晚夕咳嗽半夜,把人聒死了。常时口干,得恁一个在口里噙着他,倒生好些津液。我和俺姐姐吃了没多几个儿,连罐儿他老人家都收在房内早晚吃,谁敢动他!”西门庆道:“不打紧,我明日使小厮再送一罐来你吃。”爱月又问:“爹连日会桂姐没有?”西门庆道:“自从孝堂内到如今,谁见他来?”爱月儿道:“六娘五七,他也送茶去来?”西门庆道:“他家使李铭送去来。”爱月道:“我有句话儿,只放在爹心里。”西门庆问:“甚么话?”那爱月又想了想说:“我不说罢。若说了,显的姐妹每恰似我背地说他一般,不好意思的。”西门庆一面搂着他脖子说道:“怪小油嘴儿,甚么话?说与我,不显出你来就是了。”

两个正说得入港,猛然应伯爵入来大叫一声:“你两个好人儿,撇了俺每走在这里说梯己话儿!”爱月儿道:“哕,好个不得人意怪讪脸花子!猛可走来,唬了人恁一跳!”西门庆骂:“怪狗才,前边去罢。丢的葵轩和银姐在那里,都往后头来了。”这伯爵一屁股坐在床上,说:“你拿胳膊来,我且咬口儿,我才去。你两个在这里尽着\textuni{34B2}捣!”于是不由分说,向爱月儿袖口边勒出那赛鹅脂雪白的手腕儿来,夸道:“我儿,你这两只手儿,天生下就是发\textuni{23B20}\textuni{23B36}的行货子。”爱月儿道:“怪攮刀子的,我不好骂出来!”被伯爵拉过来,咬了一口走了。咬得老婆怪叫,骂:“怪花子,平白进来鬼混人死了!”便叫桃花儿:“你看他出去了,把弄道子门关上。”爱月便把李桂姐如今又和王三官儿好一节说与西门庆:“怎的有孙寡嘴、祝麻子、小张闲,架儿于宽、聂钺儿,踢行头白回子、向三,日逐标着在他家行走。如今丢开齐香儿,又和秦家玉芝儿打热,两下里使钱。使没了,将皮袄当了三十两银子,拿着他娘子儿一副金镯子放在李桂姐家,算了一个月歇钱。”西门庆听了,口中骂道:“这小淫妇儿,我恁吩咐休和这小厮缠,他不听,还对着我赌身发咒,恰好只哄着我。”爱月儿道:“爹也没要恼。我说与爹个门路儿,管情教王三官打了嘴,替爹出气。”西门庆把他搂在怀里说道:“我的儿,有甚门路儿,说与我知道。”爱月儿道:“我说与爹,休教一人知道。就是应花子也休对他题,只怕走了风。”西门庆道:“你告我说,我傻了,肯教人知道!”郑爱月道:“王三官娘林太太,今年不上四十岁,生的好不乔样!描眉画眼,打扮的狐狸也似。他儿子镇日在院里,他专在家,只寻外遇。假托在姑姑庵里打斋,但去,就在说媒的文嫂儿家落脚。文嫂儿单管与他做牵头,只说好风月。我说与爹,到明日遇他遇儿也不难。又一个巧宗儿:王三官娘子儿今才十九岁,是东京六黄太尉侄女儿,上画般标致,双陆、棋子都会。三官常不在家,他如同守寡一般,好不气生气死。为他也上了两三遭吊,救下来了。爹难得先刮剌上了他娘,不愁媳妇儿不是你的。”当下,被他一席话儿说的西门庆心邪意乱,搂着粉头说:“我的亲亲,你怎的晓的就里?”爱月儿就不说常在他家唱,只说:“我一个熟人儿,如此这般和他娘在某处会过一面,也是文嫂儿说合。”西门庆问:“那人是谁?莫不是大街坊张大户侄儿张二官儿?”爱月儿道:“那张懋德儿,好\textuni{34B2}的货,麻着个脸蛋子,密缝两个眼,可不砢硶杀我罢了!只好蒋家百家奴儿接他。”西门庆道:“我猜不着,端的是谁?”爱月儿道:“教爹得知了罢:原是梳笼我的一个南人。他一年来此做买卖两遭,正经他在里边歇不的一两夜,倒只在外边常和人家偷猫递狗,干此勾当。”西门庆听了,见粉头所事,合着他的板眼,亦发欢喜,说:“我儿,你既贴恋我心,我每月送三十两银子与你妈盘缠,也不消接人了。我遇闲就来。”爱月儿道:“爹,你若有我心时,甚么三十两二十两,随着掠几两银子与妈,我自恁懒待留人,只是伺候爹罢了。”西门庆道:“甚么话!我决然送三十两银子来。”说毕,两个上床交欢。床上铺的被褥约一尺高,爱月道:“爹脱衣裳不脱?”西门庆道:“咱连衣耍耍罢,只怕他们前边等咱。“一面扯过枕头来,粉头解去下衣,仰卧枕畔,西门庆把他两只小小金莲扛在肩上,解开蓝绫裤子,那话使上托子。但见花心轻折,柳腰款摆。正是:

\[
花嫩不禁柔,春风卒未休。
花心犹未足,脉脉情无极。
低低唤粉郎,春宵乐未央。
\]

两个交欢良久,至精欲泄之际,西门庆干的气喘吁吁,粉头娇声不绝,鬓云拖枕,满口只教:“亲达达,慢着些儿!”少顷,乐极情浓,一泄如注。云收雨散,各整衣理容,净了手,同携手来到席上。

吴银儿和爱香儿正与葵轩、伯爵掷色猜枚,觥筹交错,耍在热闹处。众人见西门庆进入,俱立起身来让坐。伯爵道:“你也下般的,把俺每丢在这里,你才出来,拿酒儿且扶扶头着。”西门庆道:“俺每说句话儿,有甚闲勾当!”伯爵道:“好话,你两个原来说梯己话儿。”当下伯爵拿大钟斟上暖酒,众人陪西门庆吃。四个妓女拿乐器弹唱。玳安在旁说道:“轿子来了。”西门庆呶了个嘴儿与他,那玳安连忙吩咐排军打起灯笼,外边伺候。西门庆也不坐,陪众人执杯立饮。吩咐四个妓女:“你再唱个‘一见娇羞’我听。”那韩消愁儿拿起琵琶来,款放娇声,拿腔唱道:

\[
一见娇羞,雨意云情两意投。我见他千娇百媚,万种妖娆,一捻温柔。通书先把话儿勾,传情暗里秋波溜。记在心头。心头,未审何时成就。
\]
唱了一个,吴银儿递西门庆酒,郑香儿便递伯爵,爱月儿奉温秀才,李智、黄四都斟上。四妓女又唱了一个。吃毕,众人又彼此交换递了两转,妓女又唱了两个。

唱毕,都饮过,西门庆就起身。一面令玳安向书袋内取出大小十一包赏赐来:四个妓女每人三钱,厨役赏了五钱,吴惠、郑春、郑奉每人三钱,撺掇打茶的每人二钱,丫头桃花儿也与了他三钱。俱磕头谢了。黄四再三不肯放,道:“应二叔,你老人家说声,天还早哩。老爹大坐坐,也尽小人之情,如何就要起身?我的月姨,你也留留儿。”爱月儿道:“我留他,他白不肯坐。”西门庆道:“你每不知,我明日还有事。”一面向黄四作揖道:“生受打搅!”黄四道:“惶恐!没的请老爹来受饿,又不肯久坐,还是小人没敬心。”说着,三个唱的都磕头说道:“爹到家多顶上大娘和众娘们,俺每闲了,会了银姐往宅内看看大娘去。”西门庆道:“你每闲了去坐上一日来。”一面掌起灯笼,西门庆下台矶,郑家鸨子迎着道万福,说道:“老爹大坐回儿,慌的就起身,嫌俺家东西不美口?还有一道米饭儿未曾上哩!”西门庆道:“够了。我明日还要起早,衙门中有勾当。应二哥他没事,教他大坐回儿罢。”那伯爵就要跟着起来,被黄四使力拦住,说道:“我的二爷,你若去了,就没趣死了。”伯爵道:“不是,你休拦我。你把温老先生有本事留下,我就算你好汉。”那温秀才夺门就走,被黄家小厮来定儿拦腰抱住。西门庆到了大门首,因问琴童儿:“温师父有头口在这里没有?”琴童道:“备了驴子在此,画童儿看着哩。”西门庆向温秀才道:“既有头口,也罢,老先儿你再陪应二哥坐坐,我先去罢。”于是,都送出门来。那郑月儿拉着西门庆手儿悄悄捏了一把,说道:“我说的话,爹你在心些,法不传六耳。”西门庆道:“知道了。”爱月又叫郑春:“你送老爹到家。”西门庆才上轿去了。吴银儿就在门首作辞了众人并郑家姐儿两个,吴惠打着灯回家去了。郑月儿便叫:“银姐,见了那个流人儿,好歹休要说。”吴银儿道:“我知道。”众人回至席上,重添兽炭,再泛流霞,歌舞吹弹,欢娱乐饮,直耍了三更方散。黄四摆了这席酒,也与了他十两银子,不在话下。当日西门庆坐轿子,两个排军打着灯,迳出院门,打发郑春回家。

一宿晚景题过。到次日,夏提刑差答应的来请西门庆早往衙门中审问贼情等事,直问到晌午来家。吃了饭,早是沈姨夫差大官沈定,拿帖儿送了个后生来,在缎子铺煮饭做火头,名唤刘包。西门庆留下了,正在书房中,拿帖儿与沈定回家去了。只见玳安在旁边站立,西门庆便问道:“温师父昨日多咱来的?”玳安道:“小的铺子里睡了好一回,只听见画童儿打对过门,那咱有三更时分才来了。今早问,温师父倒没酒;应二爹醉了,唾了一地,月姨恐怕夜深了,使郑春送了他家去了。”西门庆听了,哈哈笑了,因叫过玳安近前,说道:“旧时与你姐夫说媒的文嫂儿在那里住?你寻了他来,对门房子里见我。我和他说话。”玳安道:“小的不认的文嫂儿家,等我问了姐夫去。”西门庆道:“你问了他快去。”

玳安走到铺子里问陈敬济,敬济道:“问他做甚么?”玳安道:“谁知他做甚么,猛可教我抓寻他去。”敬济道:“出了东大街一直往南去,过了同仁桥牌坊转过往东,打王家巷进去,半中腰里有个发放巡捕的厅儿,对门有个石桥儿,转过石桥儿,紧靠着个姑姑庵儿,旁边有个小胡同儿,进小胡同往西走,第三家豆腐铺隔壁上坡儿,有双扇红对门儿的就是他家。你只叫文妈,他就出来答应你。”玳安听了说道:“再没有?小炉匠跟着行香的走——琐碎一浪荡。你再说一遍我听,只怕我忘了。”那陈敬济又说了一遍,玳安道:“好近路儿!等我骑了马去。”一面牵出大白马来骑上,打了一鞭,那马跑踍跳跃,一直去了。出了东大街迳往南,过同仁桥牌坊,由王家巷进去,果然中间有个巡捕厅儿,对门亦是座破石桥儿,里首半截红墙是大悲庵儿,往西小胡同上坡,挑着个豆腐牌儿,门首只见一个妈妈晒马粪。玳安在马上就问:“老妈妈,这里有个说媒的文嫂儿?”那妈妈道:“这隔壁对门儿就是。”

玳安到他门首,果然是两扇红对门儿,连忙跳下马来,拿鞭儿敲着门叫道:“文嫂在家不在?”只见他儿子文\textSiTang 开了门,问道:“是那里来的?”玳安道:“我是县门前提刑西门老爹家,来请,教文妈快去哩。”文\textSiTang 听见是提刑西门大官府里来的,便让家里坐。那玳安把马拴住,进入里面。见上面供养着利市纸,有几个人在那里算进香帐哩。半日拿了钟茶出来,说道:“俺妈不在了。来家说了,明日早去罢。”玳安道:“驴子见在家里,如何推不在?”侧身迳往后走。不料文嫂和他媳妇儿,陪着几个道妈妈子正吃茶,躲不及,被他看见了,说道:“这个不是文妈?就回我不在家!”文嫂笑哈哈与玳安道了个万福,说道:“累哥哥到家回声,我今日家里会茶。不知老爹呼唤我做甚么,我明日早去罢。”玳安道:“只分忖我来寻你,谁知他做甚么。原来你在这咭溜搭剌儿里住,教我抓寻了个小发昏。”文嫂儿道:“他老人家这几年买使女,说媒,用花儿,自有老冯和薛嫂儿、王妈妈子走跳,稀罕俺每!今日忽剌八又冷锅中豆儿爆,我猜着你六娘没了,一定教我去替他打听亲事,要补你六娘的窝儿。”玳安道:“我不知道。你到那里,俺爹自有话和你说。”文嫂儿道:“既如此,哥哥你略坐坐儿,等我打发会茶人去了,同你去罢。”玳安道:“俺爹在家紧等的火里火发,吩咐了又吩咐,教你快去哩。和你说了话,还要往府里罗同知老爹家吃酒去哩。”文嫂道:“也罢,等我拿点心你吃了,同你去。”玳安道:“不吃罢。”文嫂因问:“你大娘生了孩儿没有?”玳安道:“还不曾见哩。”文嫂一面打发玳安吃了点心,穿上衣裳,说道:“你骑马先行一步儿,我慢慢走。”玳安道:“你老人家放着驴子,怎不备上骑?”文嫂儿道:“我那讨个驴子来?那驴子是隔壁豆腐铺里的,借俺院儿里喂喂儿,你就当我的。”玳安道:“记的你老人家骑着匹驴儿来,往那去了?”文嫂儿道:“这咱哩!那一年吊死人家丫头,打官司把旧房儿也卖了,且说驴子哩!”玳安道:“房子到不打紧,且留着那驴子和你早晚做伴儿也罢了。别的罢了,我见他常时落下来好个大鞭子。”文嫂哈哈笑道:“怪猴子,短寿命,老娘还只当好话儿,侧着耳朵听。几年不见,你也学的恁油嘴滑舌的。到明日,还教我寻亲事哩!”玳安道:“我的马走的快,你步行,赤道挨磨到多咱晚,不惹的爹说?你也上马,咱两个叠骑着罢。”文嫂儿道:“怪小短命儿,我又不是你影射的!街上人看着,怪剌剌的。”玳安道:“再不,你备豆腐铺里驴子骑了去,到那里等我打发他钱就是了。”文嫂儿道:“这还是话。”一面教文\textSiTang 将驴子备了,带上眼纱,骑上,玳安与他同行,迳往西门庆宅中来。正是:

\[
欲向深闺求艳质,全凭红叶是良媒。
\]


\newpage
%# -*- coding:utf-8 -*-
%%%%%%%%%%%%%%%%%%%%%%%%%%%%%%%%%%%%%%%%%%%%%%%%%%%%%%%%%%%%%%%%%%%%%%%%%%%%%%%%%%%%%


\chapter{招宣府初调林太太\KG 丽春院惊走王三官}


词曰:

\[
香烟袅,罗帏锦帐风光好。风光好,金钗斜軃,凤颠鸾倒。
恍疑身在蓬莱岛,邂逅相逢缘不小。缘不小,最开怀处,蛾眉淡扫。
\]

话说玳安同文嫂儿到家,平安说:“爹在对门房子里。”进去禀报。西门庆正在书房中和温秀才坐的,见玳安,随即出来,小客位内坐下。玳安道:“文嫂儿叫了来,在外边伺候。”西门庆即令:“叫他进来。”那文嫂悄悄掀开暖帘,进入里面,向西门庆磕头。西门庆道:“文嫂,许久不见你。”文嫂道:“小媳妇有。”西门庆道:“你如今搬在那里住了?”文嫂道:“小媳妇因不幸为了场官司,把旧时那房儿弃了,如今搬在大南首王家巷住哩。”西门庆吩咐道:“起来说话。”那文嫂一面站立在旁边。西门庆令左右都出去,那平安和画童都躲在角门外伺候,只玳安儿影在帘儿外边听。西门庆因问:“你常在那几家大人家走跳?”文嫂道:“就是大街皇亲家,守备府周爷家,乔皇亲、张二老爹、夏老爹家,都相熟。”西门庆道:“你认的王招宣府里不认的?”文嫂道:“是小媳妇定门主顾,太太和三娘常照顾我的花翠。”西门庆道:“你既相熟,我有桩事儿央及你,休要阻了我。”向袖中取出五两一锭银子与他,悄悄和他说:“如此这般,你怎的寻个路儿把他太太吊在你那里,我会他会儿,我还谢你。”那文嫂听了,哈哈笑道:“是谁对爹说来?你老人家怎的晓得来?”西门庆道:“常言:人的名儿,树的影儿。我怎得不知道!”文嫂道:“若说起我这太太来,今年属猪,三十五岁,端的上等妇人,百伶百俐,只好象三十岁的。他虽是干这营生,好不干的细密!就是往那里去,许多伴当跟随,径路儿来,迳路儿去。三老爹在外为人做人,他怎在人家落脚?——这个人传的讹了。倒是他家里深宅大院,一时三老爹不在,藏掖个儿去,人不知鬼不觉,倒还许。若是小媳妇那里,窄门窄户,敢招惹这个事?就是爹赏的这银子,小媳妇也不敢领去。宁可领了爹言语,对太太说就是了。”西门庆道:“你不收,便是推托,我就恼了。事成,我还另外赏几个绸缎你穿。”文嫂道:“愁你老人家没有也怎的?上人着眼觑,就是福星临。”磕了个头,把银子接了,说道:“待小媳妇悄悄对太太说,来回你老人家。”西门庆道:“你当件事干,我这里等着。你来时,只在这里来就是了,我不使小厮去了。”文嫂道:“我知道。不在明日,只在后日,随早随晚,讨了示下就来了。”一面走出来。玳安道:“文嫂,随你罢了,我只要你一两银子,也是我叫你一场。你休要独吃。”文嫂道:“猢狲儿隔墙掠筛箕,还不知仰着合着哩。”于是出门骑上驴子,他儿子笼着,一直去了。西门庆和温秀才坐了一回,良久,夏提刑来,就冠冕着同往府里罗同知——名唤罗万象那里吃酒去了。直到掌灯以后才来家。

且说文嫂儿拿着西门庆五两银子,到家欢喜无尽,打发会茶人散了。至后晌时分,走到王招宣府宅里,见了林太太,道了万福。林氏便道:“你怎的这两日不来看看我?”文嫂便把家中会茶,赶腊月要往顶上进香一节告诉林氏。林氏道:“你儿子去,你不去罢了。”文嫂儿道:“我如何得去?只教文\textSiTang 代进香去罢了。”林氏道:“等临期,我送些盘缠与你。”文嫂便道:“多谢太太布施。”说毕,林氏叫他近前烤火,丫鬟拿茶来吃了。这文嫂一面吃了茶,问道:“三爹不在家了?”林氏道:“他又有两夜没回家,只在里边歇哩。逐日搭着这伙乔人,只眠花卧柳,把花枝般媳妇儿丢在房里,通不顾,如何是好?”文嫂又问:“三娘怎的不见?”林氏道:“他还在房里未出来哩。”这文嫂见无人,便说道:“不打紧,太太宽心。小媳妇有个门路儿,管就打散了这伙人,三爹收心,也再不进院去了。太太容小媳妇,便敢说;不容便不敢说。”林氏道:“你说的话儿,那遭儿我不依你来?你有话只顾说不妨。”这文嫂方说道:“县门前西门大老爹,如今见在提刑院做掌刑千户,家中放官吏债,开四五处铺面:缎子铺、生药铺、绸绢铺、绒线铺,外边江湖又走标船,扬州兴贩盐引,东平府上纳香蜡,伙计主管约有数十。东京蔡太师是他干爷,朱太尉是他卫主,翟管家是他亲家,巡抚巡按都与他相交,知府知县是不消说。家中田连阡陌,米烂成仓,身边除了大娘子——乃是清河左卫吴千户之女,填房与他为继室——只成房头、穿袍儿的,也有五六个。以下歌儿舞女,得宠侍妾,不下数十。端的朝朝寒食,夜夜元宵。今老爹不上三十一二年纪,正是当年汉子,大身材,一表人物。也曾吃药养龟,惯调风情;双陆象棋,无所不通;蹴踘打毬,无所不晓;诸子百家,拆白道字,眼见就会。端的击玉敲金,百怜百俐。闻知咱家乃世代簪缨人家,根基非浅,又见三爹在武学肄业,也要来相交,只是不曾会过,不好来的。昨日闻知太太贵诞在迩,又四海纳贤,也一心要来与太太拜寿。小媳妇便道:‘初会,怎好骤然请见的。待小的达知老太太,讨个示下,来请老爹相见。’今老太太不但结识他来往相交,只央浼他把这干人断开了,须玷辱不了咱家门户。”林氏被文嫂这篇话说的心中迷留摸乱,情窦已开,便向文嫂儿较计道:“人生面不熟,怎好遽然相见?”文嫂道:“不打紧,等我对老爹说。只说太太先央浼他要到提刑院递状,告引诱三爹这起人,预先请老爹来私下先会一会,此计有何不可?”说得林氏心中大喜,约定后日晚夕等候。

这文嫂讨了妇人示下归家,到次日饭时,走来西门庆宅内。西门庆正在对门书院内坐的,忽玳安报:“文嫂来了。”西门庆听了,即出小客位,令左右放下帘儿。良久,文嫂进入里面,磕了头,玳安知局,就走出来了。文嫂便把怎的说念林氏:“夸奖老爹人品家道,怎样结识官府,又怎的仗义疏财,风流博浪,说得他千肯万肯,约定明日晚间,三爹不在家,家中设席等候。假以说人情为由,暗中相会。”西门庆听了,满心欢喜。又令玳安拿了两匹绸缎赏他。文嫂道,“爹明日要去,休要早了。直到掌灯,街上人静时,打他后门首扁食巷中——他后门旁有个住房的段妈妈,我在他家等着。爹只使大官儿弹门,我就出来引爹入港,休令左近人知道。”西门庆道:“我知道。你明日先去,不可离寸地,我也依期而至。”说毕,文嫂拜辞出门,又回林氏话去了。

西门庆那日,归李娇儿房中宿歇,一宿无话。巴不到次日,培养着精神。午间,戴着白忠靖巾,便同应伯爵骑马往谢希大家吃生日酒。席上两个唱的。西门庆吃了几杯酒,约掌灯上来,就逃席走出来了。骑上马,玳安、琴童两个小厮跟随。那时约十九日,月色朦胧,带着眼纱由大街抹过,迳穿到扁食巷王招宣府后门来。那时才上灯一回,街上人初静之后。西门庆离他后门半舍,把马勒住,令玳安先弹段妈妈家门。原来这妈妈就住着王招宣家后房,也是文嫂举荐,早晚看守后门,开门闭户。但有入港,在他家落脚做窝。文嫂在他屋里听见弹门,连忙开门。见西门庆来了,一面在后门里等的西门庆下了马,除去眼纱儿,引进来,吩咐琴童牵了马,往对门人家西首房檐下那里等候,玳安便在段妈妈屋里存身。这文嫂一面请西门庆入来,便把后门关了,上了栓,由夹道进内。转过一层群房,就是太太住的五间正房,旁边一座便门闭着。这文嫂轻敲敲门环儿,原来有个听头。少顷,见一丫鬟出来,开了双扉。文嫂导引西门庆到后堂,掀开帘拢,只见里面灯烛荧煌,正面供养着他祖爷太原节度颁阳郡王王景崇的影身图:穿着大红团袖,蟒衣玉带,虎皮交椅坐着观看兵书。有若关王之像,只是髯须短些。迎门朱红匾上写着“节义堂”三字,两壁隶书一联:“传家节操同松竹,报国勋功并斗山。”西门庆正观看之间,只听得门帘上铃儿响,文嫂从里拿出一盏茶来与西门庆吃。西门庆便道:“请老太太出来拜见。”文嫂道:“请老爹且吃过茶着,刚才禀过太太知道了。”不想林氏悄悄从房门帘里望外边观看,见西门庆身材凛凛,一表人物,头戴白缎忠靖冠,貂鼠暖耳,身穿紫羊绒鹤氅,脚下粉底皂靴,就是个——

\[
富而多诈奸邪辈,压善欺良酒色徒。
\]

林氏一见满心欢喜,因悄悄叫过文嫂来,问他戴的孝是谁的。文嫂道:“是他第六个娘子的孝,新近九月间没了不多些时。饶少杀,家中如今还有一巴掌人儿。他老人家,你看不出来?出笼儿的鹌鹑——也是个快斗的。”这婆娘听了,越发欢喜无尽。文嫂催逼他出去,妇人道:“我羞答答怎好出去?请他进来见罢。”文嫂一面走出来,向西门庆说:“太太请老爹房内拜见哩。”于是忙掀门帘,西门庆进入房中,但见帘幙垂红,毡毺铺地,麝兰香霭,气暖如春。绣榻则斗帐云横,锦屏则轩辕月映。妇人头上戴着金丝翠叶冠儿,身穿白绫宽绸袄儿,沉香色遍地金妆花缎子鹤氅,大红宫锦宽襕裙子,老鹳白绫高底鞋儿。就是个绮阁中好色的娇娘,深闺内施\textuni{23B48}的菩萨。有诗为证:

\[
云浓脂腻黛痕长,莲步轻移兰麝香。
醉后情深归绣帐,始知太太不寻常。
\]

西门庆一见便躬身施礼,说道:“请太太转上,学生拜见。”林氏道:“大人免礼罢。”西门庆不肯,就侧身磕下头去拜两拜。妇人亦叙礼相还。拜毕,西门庆正面椅子上坐了,林氏就在下边梳背炕沿斜佥相陪。文嫂又早把前边仪门闭上了,再无一个仆人在后边。三公子那边角门也关了。一个小丫鬟名唤芙蓉,拿茶上来,林氏陪西门庆吃了茶,文嫂就在旁说道:“太太久闻老爹执掌刑名,敢使小媳妇请老爹来央烦桩事儿,未知老爹可依允不依?”西门庆道:“不知老太太有甚事吩咐?”林氏道:“不瞒大人说,寒家虽世代做了这招宣,不幸夫主去世年久,家中无甚积蓄。小儿年幼优养,未曾考袭,如今虽入武学肄业,年幼失学。外边有几个奸诈不良的人,日逐引诱他在外飘酒,把家事都失了。几次欲待要往公门诉状,诚恐抛头露面,有失先夫名节。今日敢请大人至寒家诉其衷曲,就如同递状一般。望乞大人千万留情把这干人怎生处断开了,使小儿改过自新,专习功名,以承先业,实出大人再造之恩,妾身感激不浅,自当重谢。”西门庆道:“老太太怎生这般说。尊家乃世代簪缨,先朝将相。令郎既入武学,正当努力功名,承其祖武,不意听信游食所哄,留连花酒,实出少年所为。太太既吩咐,学生到衙门里,即时把这干人处分惩治,庶可杜绝将来。”这妇人听了,连忙起身,向西门庆道了万福,说道:“容日妾身致谢大人。”西门庆道:“你我一家,何出此言。”

说话之间,彼此眉目顾盼留情。不一时,文嫂放桌儿摆上酒来,西门庆故意辞道:“学生初来进谒,倒不曾送礼来,如何反承老太太盛情留坐!”林氏道:“不知大人下降,没作整备。寒天聊具一杯水酒,表意面已。”丫鬟筛上酒来,端的金壶斟美酿,玉盏贮佳肴。林氏起身捧酒,西门庆亦下席道:“我当先奉老太太一杯。”文嫂儿在旁插口说道:“老爹且不消递太太酒。这十一月十五日是太太生日,那日送礼来与太太祝寿就是了。”西门庆道:“阿呀!早时你说。今日是初九,差六日。我在下一定来与太太登堂拜寿。”林氏笑道:“岂敢动劳大人!”须臾,大盘大碗,就是十六碗美味佳肴,旁边绛烛高烧,下边金炉添火,交杯一盏,行令猜枚,笑雨嘲云。

酒为色胆。看看饮至莲漏已沉、窗月倒影之际,一双竹叶穿心,两个芳情已动。文嫂已过一边,连次呼酒不至。西门庆见左右无人,渐渐促席而坐,言颇涉邪,把手捏腕之际,挨肩擦膀之间。初时戏搂粉项,妇人则笑而不言;次后款启朱唇,西门庆则舌吐其口,鸣咂有声,笑语密切。妇人于是自掩房门,解衣松佩,微开锦帐,轻展绣衾,鸳枕横床,凤香薰被,相挨玉体,抱搂酥胸。原来西门庆知妇人好风月,家中带了淫器包在身边,又服了胡僧药。妇人摸见他阳物甚大,西门庆亦摸其牝户,彼此欢欣,情兴如火。展猿臂,不觉蝶浪蜂狂;跷玉腿,那个羞云怯雨!正是:

\[
纵横惯使风流阵,那管床头堕玉钗。
\]

西门庆当下竭平生本事,将妇人尽力盘桓了一场。缠至更深天气,方才精泄。妇人则发乱钗横,花憔柳困。两个并头交股,搂抱片时,起来穿衣。妇人款剔银灯,开了房门,照镜整容,呼丫鬟捧水净手。复饮香醪,再劝美酌。三杯之后,西门庆告辞起身,妇人挽留不已,叮咛频嘱。西门庆躬身领诺,谢扰不尽,相别出门。妇人送到角门首回去了。文嫂先开后门,呼唤玳安、琴童牵马过来,骑上回家。街上已喝号提铃,更深夜静,但见一天霜气,万籁无声。西门庆回家,一宿无话。

到次日,西门庆到衙门中发放已毕,在后厅叫过该地方节级缉捕,吩咐如此这般:“王招宣府里三公子,看有甚么人勾引他,院中在何人家行走,即查访出名字来,报我知道。”因向夏提刑说:“王三公子甚不学好,昨日他母亲再三央人来对我说,倒不关他儿子事,只被这干光棍勾引他。今若不痛加惩治,将来引诱坏了人家子弟。”夏提刑道:“长官所见不错,必该治他。”节级缉捕领了西门庆钧语,当日即查访出各人名姓来,打了事件,到后晌时分来西门庆宅内呈递揭帖。西门庆见上面有孙寡嘴、祝实念、小张闲、聂钺儿、向三、于宽、白回子,乐妇是李桂姐、秦玉芝儿。西门庆取过笔来,把李桂姐、秦玉芝儿并老孙、祝实念名字都抹了,吩咐:“这小张闲等五个光棍,即与我拿了,明日早带到衙门里来。”众公人应诺下去。至晚,打听王三官众人都在李桂姐家吃酒踢行头,都埋伏在房门首。深更时分,刚散出来,众公人把小张闲、聂钺、于宽、白回子、向三五人都拿了。孙寡嘴与祝实念扒李桂姐后房去了,王三官藏在李桂姐床底下,不敢出来。桂姐一家唬的捏两把汗,更不知是那里的人,乱央人打听实信。王三官躲了一夜不敢出来。李家鸨子又恐怕东京下来拿人,到五更时分,撺掇李铭换了衣服,送王三官来家。

节级缉捕把小张闲等拿在听事房吊了一夜。到次日早晨,西门庆进衙门与夏提刑升厅,两边刑杖罗列,带人上去。每人一夹二十大棍,打得皮开肉绽,鲜血迸流,响声震天,哀号恸地。西门庆嘱咐道:“我把你这起光棍,专一引诱人家子弟在院飘风,不守本分,本当重处,今姑从轻责你这几下儿。再若犯在我手里,定然枷号,在院门首示众!”喝令左右:“叉下去!”众人望外,金命水命,走投无命。

两位官府发放事毕,退厅吃茶。夏提刑因说起:“昨日京中舍亲崔中书那里书来,说衙门中考察本上去了,还未下来哩。今日会了长官,咱倒好差人往怀庆府同僚林苍峰那里,打听打听消息去。他那里临京近。”西门庆道:“长官所见甚明。”即唤走差的上来吩咐:“与你五钱银子盘缠,即拿俺两个拜帖,到怀庆府提刑林千户老爹那里,打听京中考察本示下,看经历司行下照会来不曾。务要打听的实,来回报。”那人领了银子、拜帖,又到司房结束行装,讨了匹马,长行去了。两位官府才起身回家。

却说小张闲等从提刑院打出来,走在路上各人思想,更不料今日受这场亏是那里药线,互相埋怨。小张闲道:“莫不还是东京那里的消息?”白回子道:“不是。若是那里消息,怎肯轻饶素放?”常言说得好:乖不过唱的,贼不过银匠,能不过架儿。聂钺儿一口就说道:“你每都不知道,只我猜得着。此一定是西门官府和三官儿上气,嗔请他表子,故拿俺每煞气。正是:龙斗虎伤,苦了小獐。”小张闲道:“列位倒罢了,只是苦了我在下了。孙寡嘴、祝麻子都跟着,只把俺每顶缸。”于宽道:“你怎的说浑话?他两个是他的朋友,若拿来跪在地下,他在上面坐着,怎生相处?”小张闲道:“怎的不拿老婆?”聂钺道:“两个老婆,都是他心上人。李家桂姐是他的表子,他肯拿来!也休怪人,是俺每的晦气,偏撞在这网里。才夏老爹怎生不言语,只是他说话?这个就见出情弊来了。如今往李桂姐家寻王三官去!白为他打了这一屁股疮来不成?便罢了,就问他要几两银子盘缠,也不吃家中老婆笑话。”于是迳入勾栏,见李桂姐家门关的铁桶相似。叫了半日,丫头隔门问是谁,小张闲道:“是俺每,寻三官儿说话。”丫头回说:“他从那日半夜就回家去了,不在这里。无人在家中,不敢开门。”这众人只得回来,到王招宣府内,迳入他客位里坐下。王三官听见众人来寻他,唬得躲在房里不敢出来。半日,使出小厮永定儿来说:“俺爹不在家了。”众人道:“好自在性儿!不在家了,往那里去了?叫不将来!”于宽道:“实和你说了罢,休推睡里梦里。刚才提刑院打了俺每,押将出来。如今还要他正身见官去哩!”搂起腿来与永定瞧,教他进里面去说:“为你打俺每,有甚要紧!”一个个都躺在凳上声疼叫喊。

那王三官儿越发不敢出来,只叫:“娘,怎么样儿?如何救我则可。”林氏道:“我女妇人家,如何寻人情去救得?”求了半日,见外边众人等得急了,要请老太太说话。那林氏又不出去,只隔着屏风说道:“你每略等他等,委的在庄上,不在家了。我这里使小厮叫他去。”小张闲道:“老太太,快使人情他来!这个疖子终要出脓,只顾脓着不是事。俺每为他连累打了这一顿。刚才老爹吩咐押出俺每来要他。他若不出来,大家都不得清净,就弄的不好了。”

林氏听言,连忙使小厮拿出茶来与众人吃。王三官唬的鬼也似,逼他娘寻人情。直到至急之处,林氏方才说道:“文嫂他只认的提刑西门官府家,昔年曾与他女儿说媒来,在他宅中走的熟。”王三官道:“就认的西门提刑也罢。快使小厮请他来。”林氏道:“他自从你前番说了他,使性儿一向不来走动,怎好又请他?他也不肯来。”王三官道:“好娘,如今事在至急,请他来,等我与他陪个礼儿便了。”林氏便使永定儿悄悄打后门出去,请了文嫂来。王三官再三央及他,一口一声只叫:“文妈,你认的提刑西门大官府,好歹说个人情救我。”这文嫂故意做出许多乔张致来,说道:“旧时虽故与他宅内大姑娘说媒,这几年谁往他门上走!大人家深宅大院,不去缠他。”王三官连忙跪下说道:“文妈,你救我,恩有重报,不敢有忘。那几个人在前边只要出官,我怎去得?”文嫂只把眼看他娘,他娘道:“也罢,你便替他说说罢了。”文嫂道:“我独自个去不得。三叔,你衣巾着,等我领你亲自到西门老爹宅上,你自拜见央浼他,等我在旁再说,管情一天事就了了。”王三官道:“见今他众人在前边催逼甚急,只怕一时被他看见怎了?”文嫂道:“有甚难处勾当?等我出去安抚他,再安排些酒肉点心茶水哄他吃着,我悄悄领你从后门出去,干事回来,他就便也不知道。”

这文嫂一面走出前厅,向众人拜了两拜,说道:“太太教我出来,多上覆列位哥每:本等三叔往庄上去了,不在家,使人请去了,便来也。你每略坐坐儿。吃打受骂,连累了列位。谁人不吃盐米,等三叔来,教他知遇你们。你们千差万差来人不差,恒属大家只要图了事。上司差派,不由自己。有了三叔出来,一天大事都了了。”众人听了,一齐道:“还是文妈见的多,你老人家早出来说恁句有南北的话儿,俺每也不急的要不的。执杀法儿只回不在家,莫不俺每自做出来的事?你恁带累俺每吃官棒,上司要你,假推不在家。吃酒吃肉,教人替你不成?文妈,你是晓道理的,你出来,俺每还透个路儿与你——破些东西儿,寻个分上儿说说,大家了事。你不出来见俺每,这事情也要消缴,一个缉捕问刑衙门,平不答的就罢了?”文嫂儿道:“哥每说的是。你每略坐坐儿,我对太太说,安排些酒饭儿管待你每。你每来了这半日也饿了。”众人都道:“还是我的文妈知人苦辣。不瞒文妈说,俺每从衙门里打出来,黄汤儿也没曾尝着哩!”这文嫂走到后边,一力窜掇,打了二钱银子酒,买了一钱银子点心,猪羊牛肉各切几大盘,拿将出去,一壁哄他众人在前边大酒大肉吃着。

这王三官儒巾青衣,写了揭帖,文嫂领着,带上眼纱,悄悄从后门出来,步行径往西门庆家来。到了大门首,平安儿认的文嫂,说道:“爹才在厅上,进去了。文妈有甚话说?”文嫂递与他拜帖,说道:“哥哥,累你替他禀禀去。”连忙问王三官要了二钱银子递与他,那平安儿方进去替他禀知西门庆。西门庆见了手本拜帖,上写着:“眷晚生王采顿首百拜。”一面先叫进文嫂,问了回话,然后才开大厅槅子门,使小厮请王三官进去。西门庆头戴忠靖巾,便衣出来迎接,见王三衣巾进来,故意说道:“文嫂怎不早说?我亵衣在此。”便令左右:“取我衣服来。”慌的王三官向前拦住道:“尊伯尊便,小侄敢来拜渎,岂敢动劳!”至厅内,王三官务请西门庆转上行礼。西门庆笑道:“此是舍下。”再三不肯。西门庆居先拜下去,王三官说道:“小侄有罪在身,久仰,欠拜。”西门庆道:“彼此少礼。”王三官因请西门庆受礼,说道:“小侄人家,老伯当得受礼,以恕拜迟之罪。”务让起来,受了两礼。西门庆让坐,王三官又让了一回,然后挪座儿斜佥坐的。

少顷,吃了茶,王三官向西门庆说道:“小侄有事,不敢奉渎尊严。”因向袖中取出揭帖递上,随即离座跪下。被西门庆一手拉住,说道:“贤契有甚话,但说何害!”王三官就说:“小侄不才,诚为得罪,望乞老伯念先父武弁一殿之臣,宽恕小侄无知之罪,完其廉耻,免令出官,则小侄垂死之日,实再生之幸也。衔结图报,惶恐,惶恐!”西门庆展开揭帖,上面有小张闲等五人名字,说道:“这起光棍,我今日衙门里,已各重责发落,饶恕了他,怎的又央你去?”王三官道:“他说老伯衙门中责罚了他,押出他来,还要小侄见官。在家百般辱骂喧嚷,索诈银两,不得安生,无处控诉,特来老伯这里请罪。”又把礼帖递上。西门庆一见,便道:“岂有此理!这起光棍可恶。我倒饶了他,如何倒往那里去搅扰!”把礼帖还与王三官收了,道:“贤契请回,我且不留你坐。如今就差人拿这起光棍去。容日奉招。”王三官道:“岂敢!蒙老伯不弃,小侄容当叩谢。”千恩万谢出门。西门庆送至二门首,说:“我亵服不好送的。”那王三官自出门来,还带上眼纱,小厮跟随去了。文嫂还讨了西门庆话。西门庆吩咐:“休要惊动他,我这里差人拿去。”

这文嫂同王三官暗暗到家。不想西门庆随即差了一名节级、四个排军,走到王招宣宅内。那起人正在那里饮酒喧闹,被公人进去不由分说都拿了,带上镯子。唬得众人面如土色,说道:“王三官干的好事,把俺每稳住在家,倒把锄头反弄俺每来了。”那个节级排军骂道:“你这厮还胡说,当的甚么?各人到老爹跟前哀告,讨你那命是正经。”小张闲道:“大爷教导的是。”

不一时,都拿到西门庆宅门首,门上排军并平安儿都张着手儿要钱,才替他禀。众人不免脱下褶儿,并拿头上簪圈下来,打发停当,方才说进去。半日,西门庆出来坐厅,节级带进去跪在厅下。西门庆骂道:“我把你这起光棍,我倒将就了你,你如何指称我衙门往他家讹诈去?实说诈了多少钱?若不说,令左右拿拶子与我着实拶起来!”当下只说了声,那左右排军登时拿了五六把新拶子来伺候。小张闲等只顾叩头哀告道:“小的每并没讹诈分文财物,只说衙门中打出来,对他说声。他家拿出些酒食来管待小的们,小的每并没需索他的。”西门庆道:“你也不该往他家去。你这些光棍,设骗良家子弟,白手要钱,深为可恨!既不肯实供,都与我带了衙门里收监,明日严审取供,枷号示众!”众人一齐哀告,哭道:“天官爷,超生小的每罢,小的再不敢上他门缠扰了。休说枷号,这一送到监里去,冬寒时月,小的每都是死数。”西门庆道:“我把你这起光棍,饶出你去,都要洗心改过,务要生理。不许你挨坊靠院,引诱人家子弟,诈骗财物。再拿到我衙门里来,都活打死了。”喝令:“叉出去!”众人得了个性命,往外飞跑。正是:

\[
敲碎玉笼飞彩凤,顿开金锁走蛟龙。
\]

西门庆发了众人去,回至后房,月娘问道:“这是那个王三官儿?”西门庆道:“此是王招宣府中三公子,前日李桂儿为那场事就是他。今日贼小淫妇儿不改,又和他缠,每月三十两银子教他包着。嗔道一向只哄着我!不想有个底脚里人儿又告我说,教我差干事的拿了这干人,到衙门里都夹打了。不想这干人又到他家里嚷赖,指望要诈他几两银子,只说衙门中要他。他从没见官,慌了,央文嫂儿拿了五十两礼帖来求我说人情。我刚才把那起人又拿了来,扎发了一顿,替他杜绝了。人家倒运,偏生这样不肖子弟出来。——你家祖父何等根基,又做招宣,你又见入武学,放着那名儿不干,家中丢着花枝般媳妇儿不去理论,白日黑夜只跟着这伙光棍在院里嫖弄。今年不上二十岁,年小小儿的,通不成器!”月娘道:“你乳老鸦笑话猪儿足,原来灯台不照自。你自道成器的?你也吃这井里水,无所不为,清洁了些甚么儿?还要禁人!”几句说的西门庆不言语了。

正摆上饭来吃,来安来报:“应二爹来了。”西门庆吩咐:“请书房里坐,我就来。”王经连忙开了厅上书房门,伯爵进里面坐了。良久,西门庆出来。声喏毕,就坐在炕上,两个说话。伯爵道:“哥,你前日在谢二哥家,怎老早就起身?”西门庆道:“我连日有勾当,又考察在迩,差人东京打听消息。我比你每闲人儿?”伯爵又问:“哥,连日衙门中有事没有?”西门庆道:“事,那日没有!”伯爵又道:“王三官儿说,哥衙门中把小张闲他每五个,初八日晚夕,在李桂姐屋里都拿的去了,只走了老孙、祝麻子两个。今早解到衙门里,都打出来了,众人都往招宣府缠王三官去了。怎的还瞒着我不说?”西门庆道:“傻狗才,谁对你说来?你敢错听了。敢不是我衙门里,敢是周守备府里?”伯爵道:“守备府中那里管这闲事!”西门庆道:“只怕是京中提人?”伯爵道:“也不是。今早李铭对我说,那日把他一家子唬的魂也没了,李桂儿至今唬的睡倒了,还没曾起炕儿。怕又是东京下来拿人,今早打听,方知是提刑院拿人。”西门庆道:“我连日不进衙门,并没知道。李桂儿既赌过誓不接他,随他拿乱去,又害怕睡倒怎的?”伯爵见西门庆迸着脸儿待笑,说道:“哥,你是个人,连我也瞒着起来。今日他告我说,我就知道哥的情。怎的祝麻子、老孙走了?一个缉捕衙门,有个走脱了人的?此是哥打着绵羊驹\textuni{29A07}战,使李桂儿家中害怕,知道哥的手段。若都拿到衙门去,彼此绝了情意,都没趣了。事情许一不许二。如今就是老孙、祝麻子见哥也有几分惭愧。此是哥明修栈道,暗度陈仓的计策。休怪我说,哥这一着做的绝了。这一个叫做真人不露相,露相不真人。若明逞了脸,就不是乖人儿了。还是哥智谋大,见的多。”几句说的西门庆扑吃的笑了,说道:“我有甚么大智谋?”伯爵道:“我猜一定还有底脚里人儿对哥说,怎得知道这等切?端的有鬼神不测之机!”西门庆道:“傻狗才,若要人不知,除非己莫为。”伯爵道:“哥衙门中如今不要王三官儿罢了。”西门庆道:“谁要他做甚么?当初干事的打上事件,我就把王三官、祝麻子、老孙并李桂儿、秦玉芝名字都抹了,只拿几个光棍来打了。”伯爵道:“他如今怎的还缠他?”西门庆道:“我实和你说罢,他指望讹诈他几两银子。不想刚才王三官亲上门来拜见,与我磕了头,陪了不是。我又差人把那几个光棍拿了,要枷号,他众人再三哀告说,再不敢上门缠他了。王三官一口一声称我是老伯,拿了五十两礼帖儿,我不受他的。他到明日还要请我家中知谢我去。”伯爵失惊道:“真个他来和哥陪不是来了?”西门庆道:“我莫不哄你?”因唤王经:“拿王三官拜帖儿与应二爹瞧。”那王经向房子里取出拜帖,上面写着:“眷晚生王采顿首百拜。”伯爵见了,极口称赞道:“哥的所算,神妙不测。”西门庆吩咐伯爵:“你若看见他每,只说我不知道。”伯爵道:“我晓得。机不可泄,我怎肯和他说!”坐了一回,吃了茶,伯爵道:“哥,我去罢,只怕一时老孙和祝麻子摸将来。只说我没到这里。”西门庆道。“他就来,我也不见他。”一面叫将门上人来,都吩咐了:“但是他二人,只答应不在家。”西门庆从此不与李桂姐上门走动,家中摆酒也不叫李铭唱曲,就疏淡了。正是:

\[
昨夜浣花溪上雨,绿杨芳草为何人?
\]

\newpage
%# -*- coding:utf-8 -*-
%%%%%%%%%%%%%%%%%%%%%%%%%%%%%%%%%%%%%%%%%%%%%%%%%%%%%%%%%%%%%%%%%%%%%%%%%%%%%%%%%%%%%


\chapter{老太监引酌朝房\KG 二提刑庭参太尉}


诗曰:

\[
帝曰简才能,旌贤在股肱。文章体一变,礼乐道逾弘。
芸阁英华人,宾门鹓鹭登。恩筵过所望,圣泽实超恒。
\]

话说西门庆自此与李桂姐断绝不题。却说走差人到怀庆府林千户处打听消息,林千户将升官邸报封付与来人,又赏了五钱银子,连夜来递与提刑两位官府。当厅夏提刑拆开,同西门庆先观本卫行来考察官员照会,其略曰:

\[
兵部一本,尊明旨,严考核,以昭劝惩,以光圣治事:先该金吾卫提督官校太尉太保兼太子太保朱题前事,考察禁卫官员,除堂上官自陈外,其余两厢诏狱缉捕、内外提刑所指挥千百户、镇抚等官,各挨次格,从公举劾,甄别贤否,具题上请,当下该部详议,黜陟升调降革等因。
奉圣旨:兵部知道,钦此钦遵。抄出到部。看得太尉朱题前事,遵奉旧例,委的本官殚力致忠,公于考核,皆出闻见之实,而无偏执之私。足以励人心而孚公议,无容臣等再喙。但恩威赏罚,出自朝廷,合候命下之日,一体照例施行等因。续奉钦依拟行。
内开山东提刑所正千户夏延龄,资望既久,才练老成,昔视典牧而坊隅安静,今理齐刑而绰有政声,宜加奖励,以冀甄升,可备卤簿之选者也。贴刑副千户西门庆,才干有为,精察素著。家称殷实而在任不贪,国事克勤而台工有绩。翌神运而分毫不索,司法令而齐民果仰。宜加转正,以掌刑名者也。怀庆提刑千户所正千户林承勋,年清优学,占籍武科,继祖职抱负不凡,提刑狱详明有法,可加奖励简任者也。副千户谢恩,年齿既残,昔在行犹有可观,今任理刑罹软尤甚,宜罢黜革任者也。
\]

西门庆看了他转正千户掌刑,心中大悦。夏提刑见他升指挥,管卤簿,大半日无言,面容失色。于是又展开工部工完的本观看,上面写道:

\[
工部一本,神运届京,天人胥庆,恳乞天恩,俯加渥典,以苏民困,以广圣泽事。
奉圣旨:这神运奉迎大内,奠安艮岳,以承天眷,朕心嘉悦。你每既效有勤劳,副朕事玄至意。所经过地方,委的小民困苦,着行抚按衙门,查勘明白,着行蠲免今岁田租之半。所毁坝闸,着部里差官会同巡按御史,即行修理。完日还差内侍孟昌龄前去致祭。蔡京、李邦彦、王炜、郑居中、高俅,辅弼朕躬,直赞内廷,勋劳茂著,京加太师,邦彦加柱国太子太师,王炜太傅,郑居中、高俅太保,各赏银五十两、四表礼。蔡京还荫一子为殿中监。国师林灵素,佐国宣化,远致神运,北伐虏谋,实与天通,加封忠孝伯,食禄一千石,赐坐龙衣一袭,肩舆人内,赐号玉真教主,加渊澄玄妙广德真人、金门羽客、达灵玄妙先生。朱勔、黄经臣,督理神运,忠勤可嘉。勔加太傅兼太子太傅,经臣加殿前都太尉,提督御前人船。各荫一子为金吾卫正千户。内侍李彦、孟昌龄、贾祥、何沂、蓝从颐着直延福五位宫近侍,各赐蟒衣玉带,仍荫弟侄一人为副千户,俱见任管事。礼部尚书张邦昌、左侍郎兼学士蔡攸、右侍郎白时中、兵部尚书余深、工部尚书林摅,俱加太子太保,各赏银四十两,彩缎二表礼。巡抚两浙佥都御史张阁,升工部右侍郎。巡抚山东都御史侯濛,升太常正卿。巡抚两浙、山东监察御史尹大谅、宋乔年,都水司郎中安忱、伍训,各升俸一级,赏银二十两。祗迎神运千户魏承勋、徐相、杨廷佩、司凤仪、赵友兰、扶天泽、西门庆、田九皋等,各升一级。内侍宋推等,营将王佑等,俱各赏银十两。所官薛显忠等,各赏银五两。校尉昌玉等,绢二匹。该衙门知道。
\]

夏提刑与西门庆看毕,各散回家。后晌时分,有王三官差永定同文嫂拿请书,十一日请西门庆往他府中赴席,少罄谢私之意。西门庆收下,不胜欢喜,以为其妻指日在于掌握。不期到初十日晚夕,东京本卫经历司差人行照会:“晓谕各省提刑官员知悉:火速赴京,赶冬节见朝谢恩,毋得违误取罪。”西门庆看了,到次日衙门中会了夏提刑,各人到家,即收拾行装,备办贽见礼物,约早晚起程。西门庆使玳安叫了文嫂儿,教他回王三官:“我今日不得来赴席,要上京见朝谢恩去。”文嫂连忙去回,王三官道:“既是老伯有事,容回来洁诚具请。”西门庆一面叫将贲四来,吩咐教他跟了去,与他五两银子,家中盘缠。留下春鸿看家,带了玳安、王经跟随答应。又问周守备讨了四名巡捕军人,四匹小马,打点驮装轿马,排军抬扛。夏提刑便是夏寿跟随。两家共有二十余人跟从。十二日起身离了清河县,冬天易晚,昼夜趱行。到了怀西怀庆府会林千户,千户已上东京去了。一路天寒坐轿,天暖乘马,朝登紫陌,暮践红尘。正是:

\[
意急款摇青帐幕,心忙敲碎紫丝鞭。
\]

话说一日到了东京,进得万寿门。西门庆主意要往相国寺下。夏提刑不肯,坚执要往他亲眷崔中书家投下。西门庆不免先具拜帖拜见。正值崔中书在家,即出迎接,至厅叙礼相见,与夏提刑道及寒温契阔之情。坐下茶毕,拱手问西门庆尊号。西门庆道:“贱号四泉。”因问:“老先生尊号?”崔中书道:“学生性最愚朴,名闲林下,贱名守愚,拙号逊斋。”因说道:“舍亲龙溪久称盛德,全仗扶持,同心协恭,莫此为厚。”西门庆道:“不敢。在下常领教诲,今又为堂尊,受益恒多,不胜感激。”夏提刑道:“长官如何这等称呼!便不见相知了。”崔中书道:“四泉说的也是,名分使然。”言毕,彼此笑了。不一时,收拾行李。天晚了,崔中书吩咐童仆放桌摆饭,无非是果酌肴馔之类,不必细说。当日,二人在崔中书家宿歇不题。

到次日,各备礼物拜帖,家人跟随,早往蔡太师府中叩见。那日太师在内阁还未出来,府前官吏人等如蜂屯蚁聚,挤匝不开。西门庆与夏提刑与了门上官吏两包银子,拿揭帖禀进去。翟管家见了,即出来相见,让他到外边私宅。先是夏提刑先见毕,然后西门庆叙礼,彼此道及往还酬答之意,各分宾位坐下。夏提刑先递上礼帖:两匹云鹤金缎、两匹色缎。翟管家是十两银子。西门庆礼帖上是一匹大红绒彩蟒、一匹玄色妆花斗牛补子员领、两匹京缎,另外梯己送翟管家一匹黑绿云绒、三十两银子。翟谦吩咐左右:“把老爷礼都收进府中去,上簿籍。”他只受了西门庆那匹云绒,将三十两银子连夏提刑的十两银子都不受,说道:“岂有此理。若如此,不见至交亲情。”一面令左右放桌儿摆饭,说道:“今日圣上奉艮岳,新盖上清宝箓宫,奉安牌匾,该老爷主祭,直到午后才散。到家同李爷又往郑皇亲家吃酒。只怕亲家和龙溪等不的,误了你每勾当。遇老爷闲,等我替二位禀就是一般。”西门庆道:“蒙亲家费心。”翟谦因问:“亲家那里住?”西门庆就把夏龙溪令亲家下歇说了。不一时,安放桌席端正,就是大盘大碗,汤饭点心一齐拿上来,都是光禄烹炮,美味极品无加。每人金爵饮酒三杯,就要告辞起身。翟谦款留,令左右又筛上一杯。西门庆因问:“亲家,俺每几时见朝?”翟谦道:“亲家,你同不得夏大人。夏大人如今是京堂官,不在此例。你与本卫新升的副千户何大监侄儿何永寿,他便贴刑,你便掌刑,与他作同僚了。他先谢了恩,只等着你见朝引奏毕,一同好领札付。你凡事只会他去。”夏提刑听了,一声儿不言语。西门庆道:“请问亲家,只怕我还要等冬至郊天回来见朝。”翟谦道:“亲家,你等不的冬至圣上郊天回来。那日天下官员上表朝贺,还要排庆成宴,你每怎等的?不如你今日先往鸿胪寺报了名,明日早朝谢了恩,直到那日堂上官引奏毕,领札付起身就是了。”西门庆谢道:“蒙亲家指教,何以为报!”临起身,翟谦又拉西门庆到侧净处说话,甚是埋怨西门庆说:“亲家,前日我的书上那等写了,大凡事要谨密,不可使同僚每知道。亲家如何对夏大人说了?教他央了林真人帖子来,立逼着朱太尉来对老爷说,要将他情愿不管卤簿,仍以指挥职衔在任所掌刑三年;何大监又在内廷,转央朝廷所宠安妃刘娘娘的分上,便也传旨出来,亲对老爷和朱太尉说了,要安他侄儿何永寿在山东理刑。两下人情阻住了,教老爷好不作难!不是我再三在老爷跟前维持,回倒了林真人,把亲家不撑下去了?”慌的西门庆连忙打躬,说道:“多承亲家盛情!我并不曾对一人说,此公何以知之?”翟谦道:“自古机事不密则害成,今后亲家凡事谨慎些便了。”

西门庆千恩万谢,与夏提刑作辞出门。来到崔中书家,一面差贲四鸿胪寺报了名。次日同夏提刑见朝,青衣冠带,正在午门前谢恩出来,刚转过西阙门来,只见一个青衣人走向前问道:“那位是山东提刑西门老爹?”贲四问道:“你是那里的?”那人道:“我是内府匠作监何公公来请老爹说话。”言未毕,只见一个太监,身穿大红蟒衣,头戴三山帽,脚下粉底皂靴,从御街定声叫道:“西门大人请了!”西门庆遂与夏提刑分别,被这太监用手一把拉在旁边一所值房内,相见作揖,慌的西门庆倒身还礼不迭。这太监说道:“大人,你不认的我,在下是匠作监太监何沂,见在延宁第四宫端妃马娘娘位下近侍。昨日内工完了,蒙万岁爷爷恩典,将侄儿何永寿升受金吾卫副千户,见在贵处提刑所理刑管事,与老大人作同僚。”西门庆道:“原来是何老太监,学生不知,恕罪,恕罪!”一面又作揖说道:“此禁地,不敢行礼,容日到老太监外宅进拜。”于是叙礼毕,让坐,家人捧茶来吃了。茶毕,就揭桌盒盖儿,桌上许多汤饭肴品,拿盏箸儿来安下。何太监道:“不消小杯了,我晓的大人朝下来,天气寒冷,拿个小盏来,没甚肴馔,亵渎大人,且吃个头脑儿罢。”西门庆道:“不当厚扰。”何太监于是满斟上一大杯,递与西门庆,西门庆道:“承老太监所赐,学生领下。只是出去还要见官拜部,若吃得面红,不成道理。”何太监道:“吃两盏儿烫寒何害!”因说道:“舍侄儿年幼,不知刑名,望乞大人看我面上,同僚之间,凡事教导他教导。”西门庆道:“岂敢。老太监勿得太谦,令侄长官虽是年幼,居气养体,自然福至心灵。”何太监道:“大人好说。常言:学到老不会到老。天下事如牛毛,孔夫子也只识的一腿。恐有不到处,大人好歹说与他。”西门庆道:“学生谨领。”因问:“老大监外宅在何处?学生好来奉拜长官。”何大监道:“舍下在天汉桥东,文华坊双狮马台就是。”亦问:“大人下处在那里?我教做官的先去叩拜。”西门庆道:“学生暂借崔中书家下。”

彼此问了住处,西门庆吃了一大杯就起身。何太监送出门,拱着手说道:“适间所言,大人凡事看顾看顾。他还等着你一答儿引奏,好领札付。”西门庆道:“老太监不消吩咐,学生知道。”于是出朝门,又到兵部,又遇见了夏提刑,同拜了部官来。比及到本卫参见朱太尉,递履历手本,缴札付,又拜经历司并本所官员,已是申刻时分。夏提刑改换指挥服色,另具手本参见了朱太尉,免行跪礼,择日南衙到任。刚出衙门,西门庆还等着,遂不敢与他同行,让他先上马。夏延龄那里肯?定要同行。西门庆赶着他呼“堂尊”,夏指挥道:“四泉,你我同僚在先,为何如此称呼?”西门庆道:“名分已定,自然之理,何故大谦。”因问:“堂尊高升美任,不还山东去了,宝眷几时搬取?”夏延龄道:“欲待搬来,那边房舍无人看守。如今且在舍亲这边权住,直待过年,差人取家小罢了。还望长官早晚看顾一二。房子若有人要,就央长官替我打发,自当报谢。”西门庆道:“学生谨领。请问府上那房价值若干?”夏延龄道:“舍下此房原是一千三百两买的,后边又盖了一层,使了二百两,如今卖原价也罢了。”

二人归到崔宅,王经向前禀说:“新升何老爹来拜,下马到厅。小的回部中还未来家。何老爹说多拜上夏老爹、崔老爹,都投下帖。午间又差人送了两匹金缎来。”宛红帖儿拿与西门庆看,上写着:“谨具缎帕二端,奉引贽敬。寅侍教生何永寿顿首拜。”西门庆看了,连忙差王经封了两匹南京五彩狮补员领,写了礼帖。吃了饭,连忙往何家回拜去。到于厅上,何千户忙出来迎接,乌纱皂履,年纪不上二十岁,生的面如傅粉,唇若涂朱,趋下阶来揖让,退逊谦恭特甚。二人到厅上叙礼,西门庆令玳安捧上贽见之礼,拜下去,说道:“适承光顾,兼领厚仪,又失迎迓。今早又蒙老公公值房赐馔,感德不尽。”何千户忙还礼说:“学生叨受微职,忝与长官同例,早晚得领教益,实为三生有幸。适间进拜不遇,又承垂顾,蓬筚光生。”令左右收下去,一面扯椅儿分宾主坐下,左右捧茶上来。吃茶之间,彼此问号,西门庆道:“学生贱号四泉。”何千户道:“学生贱号天泉。”又问:“长官今日拜毕部堂了?”西门庆道:“从内里蒙公公赐酒出来,拜毕部,又到本衙门见堂,缴了札付,拜了所司。出来就要奉谒长官,不知反先辱长官下顾。”何千户因问:“长官今日与夏公都见朝来?”西门庆道:“夏龙溪已升了指挥直驾,今日都见朝谢恩在一处,只到衙门见堂之时,他另具手本参见。”说毕,何千户道:“咱每还是先与本主老爹进礼,还是先领札付?”西门庆道:“依着舍亲说,咱每先在卫主宅中进了礼,然后大朝引奏,还在本衙门到堂同众领札付。”何千户道:“既是如此,咱每明早备礼进了罢。”于是都会下各人礼数,何千户是两匹蟒衣、一束玉带,西门庆是一匹大红麒麟金缎、一匹青绒蟒衣、一柄金镶玉绦环,各金华酒四坛。明早在朱太尉宅前取齐。约会已定,茶汤两换,西门庆告辞而回,并不与夏延龄题此事。一宿晚景题过。

到次日,早到何千户家。何千户又预备头脑小席,大盘大碗,齐齐整整,连手下人饱餐一顿,然后同往大尉宅门前来。贲四同何家人押着礼物。那时正值朱太尉新加太保,微宗天子又差使往南坛视牲未回,各家馈送贺礼并参见官吏人等,黑压压在门首等候。何千户同西门庆下了马,在左近一相识人家坐的,差人打听老爷道子响就来通报。直等到午后,忽见一人飞马而来,传报道:“老爷视牲回来,进南薰门了。”吩咐闲杂人打开。不一时,又骑报回来,传:“老爷过天汉桥了。”少顷,只见官吏军士各打执事旗牌,一对一对传呼,走了半日,才远远望见朱太尉八抬八簇肩舆明轿,头戴乌纱,身穿猩红斗牛绒袍,腰横荆山白玉,悬挂太保牙牌、黄金鱼钥,好不显赫威严!执事到了宅门首,都一字儿摆开,喝的肃静回避,无一人声嗽。那来见的官吏人等,黑压压一群跪在街前。良久,太尉轿到跟前,左右喝声:“起来伺候!”那众人一齐应诺,诚然声震云霄。只听东边咚咚鼓乐响动,原来本衙门六员太尉堂官,见朱太尉新加光禄大夫、太保,又荫一子为千户,都各备大礼,治酒庆贺,故有许多教坊伶官在此动乐。太尉才下轿,乐就止了。各项官吏人等,预备进见。忽然一声道子响,一青衣承差手拿两个红拜帖,飞走而来,递与门上人说:“礼部张爷与学士蔡爷来拜。”连忙禀报进去。须臾轿在门首,尚书张邦昌与侍郎蔡攸,都是红吉服孔雀补子,一个犀带,一个金带,进去拜毕,待茶毕,送出来。又是吏部尚书王祖道与左侍郎韩侣、右侍郎尹京也来拜,朱太尉都待茶送了。又是皇亲喜国公、枢密使郑居中、驸马掌宗人府王晋卿,都是紫花玉带来拜。唯郑居中坐轿,这两个都骑马。送出去,方是本衙堂上六员太尉到了:头一位是提督管两厢捉察使孙荣,第二位管机察梁应龙,第三管内外观察典牧皇畿童大尉侄儿童天胤,第四提督京城十三门巡察使黄经臣,第五管京营卫缉察皇城使窦监,第六督管京城内外巡捕使陈宗善。都穿大红,头戴貂蝉,惟孙荣是太子太保玉带,余者都是金带。下马进去。各家都有金币礼物。少顷,里面乐声响动,众太尉插金花,与朱太尉把盏递酒,阶下一派箫韶盈耳,两行丝竹和鸣。端的食前方丈,花簇锦筵。怎见得太尉的富贵?但见:

\[
官居一品,位列三台。赫赫公堂,潭潭相府。虎符玉节,门庭甲仗生寒;象板银筝,磈礧排场热闹。终朝谒见,无非公子王孙;逐岁追游,尽是侯门戚里。那里解调和燮理,一味能趋谄逢迎。端的谈笑起干戈,真个吹嘘惊海岳。假旨令八位大臣拱手,巧辞使九重天子点头。督择花石,江南淮北尽灾殃;进献黄杨,国库民财皆匮竭。
\]
正是:

\[
辇下权豪第一,人间富贵无双。
\]

须臾递毕,安席坐下。一班儿五个俳优,朝上筝\textuni{25C67}琵琶,方响箜篌,红牙象板,唱了一套“享富贵,受皇恩”。

当时酒进三巡,歌吟一套,六员太尉起身,朱太尉亲送出来,回到厅,乐声暂止,管家禀事,各处官员进见。朱太尉令左右抬公案,当厅坐下,吩咐出来,先令各勋戚中贵仕宦家人送礼的进去。须臾打发出来,才是本卫纪事、南北卫两厢、五所、七司捉察、讥察、观察、巡察、典牧、直驾、提牢、指挥、千百户等官,各具手本呈递。然后才传出来,叫两淮、两浙、山东、山西、关东、关西、河东、河北、福建、广南、四川十三省提刑官挨次进见。西门庆与何千户在第五起上,抬进礼物去,管家接了礼帖,铺在书案上,二人立在阶下,等上边叫名字。西门庆抬头见正面五间厂厅,上面朱红牌匾,悬着徽宗皇帝御笔钦赐“执金吾堂”斗大四个金字,甚是显赫。须臾叫名,二人应诺升阶,到滴水檐前躬身参谒,四拜一跪,听发放。朱太尉道:“那两员千户,怎的又叫你家太监送礼来?”令左右收了,吩咐:“在地方谨慎做官,我这里自有公道。伺候大朝引奏毕,来衙门中领札赴任。”二人齐声应诺。左右喝:“起去!”由左角门出来。刚出大门来,寻见贲四等抬担出来,正要走,忽见一人拿宛红帖飞马来报,说道:“王爷、高爷来了。”西门庆与何千户闪在人家门里观看。须臾,军牢喝道,只见总督京营八十万禁军陇西公王烨,同提督神策御林军总兵官太尉高俅,俱大红玉带,坐轿而至。那各省参见官员一涌出来,又不得见了。西门庆与何千户走到僻处,呼跟随人扯过马来,二人方骑上马回寓。正是:

\[
权奸误国祸机深,开国承家戒小人。
逆贼深诛何足道,奈何二圣远蒙尘。
\]

\newpage
%# -*- coding:utf-8 -*-
%%%%%%%%%%%%%%%%%%%%%%%%%%%%%%%%%%%%%%%%%%%%%%%%%%%%%%%%%%%%%%%%%%%%%%%%%%%%%%%%%%%%%


\chapter{李瓶儿何家托梦\KG 提刑官引奏朝仪}


词曰:

\[
花事阑珊芳草歇,客里风光,又过些时节。小院黄昏人忆别,泪痕点点成红血。咫尺江山分楚越,目断神惊,只道芳魂绝。梦破五更心欲折,角声吹落梅花月。
\]

话说西门庆同何千户回来,走到大街,何千户就邀请西门庆到家一饭。西门庆再三固辞。何千户令手下把马环拉住,说道:“学生还有一事与长官商议。”于是并辔同到宅前下马。贲四同抬盒迳往崔中书家去了。原来何千户盛陈酒筵在家等候。进入厅上,但见兽炭焚烧,金炉香霭。正中独设一席,下边一席相陪,旁边东首又设一席。皆盘堆异果,花插金瓶。西门庆问道:“长官今日筵何客?”何千户道:“家公公今日下班,敢屈长官一饭。”西门庆道:“长官这等费心,就不是同僚之情。”何千户道:“家公公粗酌屈尊,长官休怪。”一面看茶吃了。西门庆请老公公拜见,何千户道:“家公公便出来。”

不一时,何太监从后边出来,穿着绿绒蟒衣,冠帽皂鞋,宝石绦环。西门庆展拜四拜:“请公公受礼。”何大监不肯,说道:“使不的。”西门庆道:“学生与天泉同寅晚辈,老公公齿德俱尊,又系中贵,自然该受礼。”讲了半日,何大监受了半礼,让西门庆上坐,他主席相陪,何千户旁坐。西门庆道:“老公公,这个断然使不得。同僚之间,岂可旁坐!老公公叔侄便罢了,学生使不的。”何太监大喜道:“大人甚是知礼,罢罢,我阁老位儿旁坐罢,教做官的陪大人就是了。”西门庆道:“这等,学生坐的也安。”于是各照位坐下。何太监道:“小的儿们,再烧了炭来。今日天气甚是寒冷。”须臾,左右火池火叉,拿上一包水磨细炭,向火盆内只一倒。厅前放下油纸暖帘来,日光掩映,十分明亮。何太监道:“大人请宽了盛服罢。”西门庆道:“学生里边没穿甚么衣服,使小价下处取来。”何太监道:“不消取去。”令左右接了衣服,“拿我穿的飞鱼绿绒氅衣来,与大人披上。”西门庆笑道:“老先生职事之服,学生何以穿得?”何太监道:“大人只顾穿,怕怎的!昨日万岁赐了我蟒衣,我也不穿他了,就送了大人遮衣服儿罢。”不一时,左右取上来,西门庆令玳安接去员领,披上氅衣,作揖谢了。又请何千户也宽去上盖陪坐。

又拿上一道茶来吃了,何太监道:“叫小厮们来。”原来家中教了十二名吹打的小厮,两个师范领着上来磕头。何太监就吩咐动起乐来,然后递酒上坐。何太监亲自把盏,西门庆慌道:“老公公请尊便。有长官代劳,只安放钟箸儿就是一般。”何太监道:“我与大人递一钟儿。我家做官的初入芦苇,不知深浅,望乞大人凡事扶持一二,就是情了。”西门庆道:“老公公说那里话!常言:同僚三世亲。学生亦托赖老公公余光,岂不同力相助!”何太监道:“好说,好说。共同王事,彼此扶持。”西门庆也没等他递酒,只接了杯儿,领到席上,随即回奉一杯,安在何千户并何太监席上,彼此告揖过,坐下。吹打毕,三个小厮连师范,在筵前银筝象板,三弦琵琶,唱了一套《正宫·端正好》“雪夜访赵普”、“水晶宫鲛绡帐”。唱毕下去。

酒过数巡,食割两道,看看天晚,秉上灯来。西门庆唤玳安拿赏赐与厨役并吹打各色人役,就起身,说道:“学生厚扰一日了,就此告回。”那公公那里肯放,说道:“我今日正下班,要与大人请教。有甚大酒席,只是清坐而已,教大人受饥。”西门庆道:“承老公公赐这等美馔,如何反言受饥!学生回去歇息歇息,明早还要与天泉参谒参谒兵科,好领札付挂号。”何太监道:“既是大人要与我家做官的同干事,何不令人把行李搬过来我家住两日?我这后园儿里有几间小房儿,甚是僻静,就早晚和做官的理会些公事儿也方便些,强如在别人家。”西门庆道:“在这里最好,只是使夏公见怪,相学生疏他一般。”何太监道:“没的说。如今时年,早晨不做官,晚夕不唱喏,衙门是恁偶戏衙门。虽故当初与他同僚,今日前官已去,后官接管承行,与他就无干。他若这等说,他就是个不知道理的人了。今日我定要和大人坐一夜,不放大人去。”唤左右:“下边房里快放桌儿,管待你西门老爹大官儿饭酒。我家差几个人,跟他即时把行李都搬了来。”又吩咐:“打扫后花园西院干净,预备铺陈,炕中笼下炭火。”堂上一呼,阶下百诺,答应下去了。西门庆道:“老公公盛情,只是学生得罪夏公了。”何太监道:“他既出了衙门,不在其位,不谋其政。他管他那銮驾库的事,管不的咱提刑所的事了。难怪于你。”不由分说,就打发玳安并马上人吃了酒饭,差了几名军牢,各拿绳扛,迳往崔中书家搬取行李去了。

何太监道:“又一件相烦大人:我家做官的到任所,还望大人替他看所宅舍儿,好搬取家小。今先教他同大人去,待寻下宅子,然后打发家小起身。也不多,连几房家人也只有二三十口。”西门庆道:“老公公吩咐,要看多少银子宅舍?”何太监道:“也得千金外房儿才够住。”西门庆道:“夏龙溪他京任不去了,他一所房子倒要打发,老公公何不要了与天泉住,一举两得其便。此宅门面七间,到底五层,仪门进去大厅,两边厢房,鹿角顶,后边住房、花亭,周围群房也有许多,街道又宽阔,正好天泉住。”何太监道:“他要许多价值儿?”西门庆道:“他对我说原是一千三百两,又后边添盖了一层平房,收拾了一处花亭。老公公若要,随公公与他多少罢了。”何太监道:“我托大人,随大人主张就是了。趁今日我在家,差个人和他说去,讨他那原文书我瞧瞧。难得寻下这房舍儿,我家做官的去到那里,就有个归着了。”

不一时,只见玳安同众人搬了行李来回话。西门庆问:“贲四、王经来了不曾?”玳安道:“王经同押了衣箱行李先来了。还有轿子,叫贲四在那里看守着哩。”西门庆因附耳低言:“如此这般上覆夏老爹,借过那里房子的原契来,何公公要瞧瞧。就同贲四一答儿来。”这玳安应的去了。不一时,贲四青衣小帽,同玳安拿文书回西门庆说:“夏老爹多多上覆:既是何公公要,怎好说价钱!原文书都拿的来了。又收拾添盖,使费了许多,随爹主张了罢。”西门庆把原契递与何太监亲看了一遍,见上面写着一千二百两,说道:“这房儿想必也住了几年,未免有些糟烂,也别要说收拾,大人面上还与他原价。”那贲四连忙跪下说:“何爷说的是。自古道:使的憨钱,治的庄田。千年房舍换百主,一番拆洗一番新。”何太监听了喜欢道:“你是那里人?倒会说话儿。常言成大事者不惜小费,其实说的是。他教甚么名字?”西门庆道:“他名唤贲四。”何太监道:“也罢,没个中人儿,你就做个中人儿,替我讨了文书来。今日是个好日期,就把银子兑与他罢。”西门庆道:“如今晚了,待的明日也罢了。”何太监道:“到五更我早进去,明日大朝。今日不如先交与他银子,就了事。”西门庆问道:“明日甚时驾出?”何太监道:“子时驾出到坛,三更鼓祭了,寅正一刻就回宫。摆了膳,就出来设朝,升大殿,朝贺天下,诸司都上表拜冬。次日,文武百官吃庆成宴。你每是外任官,大朝引奏过就没事了。”说毕,何太监吩咐何千户进后边,打点出二十四锭大元宝来,用食盒抬着,差了两个家人,同贲四、玳安押送到崔中书家交割。夏公见抬了银子来,满心欢喜,随即亲手写了文契,付与贲四等,拿来递上。何太监不胜欢喜,赏了贲四十两银子,玳安、王经每人三两。西门庆道:“小孩子家,不当赏他。”何太监道:“胡乱与他买嘴儿吃。”三人磕头谢了。何太监吩咐管待酒饭,又向西门庆唱了两个喏:“全仗大人余光。”西门庆道:“还是看老公公金面。”何太监道:“还望大人对他说说,早把房儿腾出来,就好打发家小起身。”西门庆道:“学生一定与他说,教他早腾。长官这一去,且在衙门公廨中权住几日。待他家小搬到京,收拾了,长官宝眷起身不迟。”何太监道:“收拾直待过年罢了,先打发家小去才好。十分在衙门中也不方便。”

说话之间,已有一更天气,西门庆说道:“老公公请安置罢!学生亦不胜酒力了。”何大监方作辞归后边歇息去了。何千户教家乐弹唱,还与西门庆吃了一回,方才起身,送至后园。三间书院,台榭湖山,盆景花木,房内绛烛高烧,篆内香焚麝饼,十分幽雅。何千户陪西门庆叙话,又看茶吃了,方道安置,归后边去了。

西门庆摘去冠带,解衣就寝。王经、玳安打发了,就往下边暖炕上歇去了。西门庆有酒的人,睡在枕畔,见满窗月色,翻来复去。良久只闻夜漏沉沉,花阴寂寂,寒风吹得那窗纸有声,况离家已久。正要呼王经进来陪他睡,忽听得窗外有妇人语声甚低,即披衣下床,靸着鞋袜,悄悄启户视之。只见李瓶儿雾鬓云鬟,淡妆丽雅,素白旧衫笼雪体,淡黄软袜衬弓鞋,轻移莲步,立于月下。西门庆一见,挽之入室,相抱而哭,说道:“冤家,你如何在这里?”李瓶儿道:“奴寻访至此。对你说,我已寻了房儿了,今特来见你一面,早晚便搬去了。”西门庆忙问道:“你房儿在于何处?”李瓶儿道:“咫尺不远。出此大街迤东,造釜巷中间便是。”言讫,西门庆共他相偎相抱,上床云雨,不胜美快之极。已而整衣扶髻,徘徊不舍。李瓶儿叮咛嘱咐西门庆道:“我的哥哥,切记休贪夜饮,早早回家。那厮不时伺害于你,千万勿忘!”言讫,挽西门庆相送。走出大街上,见月色如昼,果然往东转过牌坊,到一小巷,见一座双扇白板门,指道:“此奴之家也。”言毕,顿袖而入。西门庆急向前拉之,恍然惊觉,乃是南柯一梦。但见月影横窗,花枝倒影矣。西门庆向褥底摸了摸,见精流满席,余香在被,残唾犹甜。追悼莫及,悲不自胜。正是:

\[
玉宇微茫霜满襟,疏窗淡月梦魂惊。
凄凉睡到无聊处,恨杀寒鸡不肯鸣。
\]

西门庆梦醒睡不着,巴不得天亮。比及天亮,又睡着了。次日早,何千户家童仆起来伺候,打发西门庆梳洗毕,何千户又早出来陪侍,吃了姜茶,放桌儿请吃粥。西门庆问:“老公公怎的不见?”何千户道:“家公公从五更就进内去了。”须臾拿上粥来。吃了粥,又拿上一盏肉圆子馄饨鸡蛋头脑汤。一面吃着,就吩咐备马。何千户与西门庆冠冕,仆从跟随,早进内参见兵科。出来,何千户便分路来家,西门庆又到相国寺拜智云长老。长老又留摆斋。西门庆只吃了一个点心,余者收与手下人吃了,就起身从东街穿过来,要往崔中书家拜夏龙溪去。因从造釜巷所过,中间果见有双扇白板门,与梦中所见一般。悄悄使玳安问隔壁卖豆腐老姬:“此家姓甚名谁?”老姬答道:“此袁指挥家也。”西门庆于是不胜叹异。到了崔中书家,夏公才待出门拜人,见西门庆到,忙令左右把马牵过,迎至厅上,拜揖叙礼。西门庆令玳安拿上贺礼:青织金绫紵一端、色缎一端。夏公道:“学生还不曾拜贺长官,到承长官先施。昨日小房又烦费心,感谢不尽。”西门庆道:“昨日何太监说起看房,我因堂尊分上,就说此房来。何公讨了房契去看了,一口就还原价。果是内臣性儿,立马盖桥就成了。还是堂尊大福!”说毕,二人笑了。夏公道:“何天泉,我也还未回拜他。”因问:“他此去与长官同行罢了。”西门庆道:“他已会定同学生一路去,家小且待后。昨日他老公公多致意,烦堂尊早些把房儿腾出来,搬取家眷。他如今权在衙门里住几日罢了。”夏公道:“学生也不肯久稽,待这里寻了房儿,就使人搬取家小。也只待出月罢了。”说毕,西门庆起身,又留了个拜帖与崔中书,夏公送出上马,归至何千户家。何千户又早有午饭等候。西门庆悉把拜夏公之事说了一遍:“腾房已在出月。”何千户大喜,谢道:“足见长官盛情。”

吃毕饭,二人正在厅上着棋,忽左右来报:“府里翟爹差人送下程来了。抓寻到崔老爹那里,崔老爹使他这里来了。”于是拿帖看,上写着:“谨具金缎一端、云紵一端、鲜猪一口、北羊一腔、内酒一坛、点心二盒。眷生翟谦顿首拜。”西门庆见来人,说道:“又蒙你翟爹费心。”一面收了礼物,写回帖,赏来人二两银子,抬盒人五钱,说道:“客中不便,有亵管家。”那人磕头收了。王经在旁悄悄说:“小的姐姐说,教我府里去看看爱姐,有物事捎与他。”西门庆问:“甚物事?”王经道:“是家中做的两双鞋脚手。”西门庆道:“单单儿怎好拿去?”吩咐玳安:“我皮箱内有带的玫瑰花饼,取两罐儿。”就把口帖付与王经,穿上青衣,跟了来人往府里看爱姐不题。这西门庆写了帖儿,送了一腔羊、一坛酒谢了崔中书,把一口猪、一坛酒、两盒点心抬到后边孝顺老公公。何千户拜谢道:“长官,你我一家,如何这等计较!”

且说王经到府内,请出韩爱姐,外厅拜见了。打扮的如琼林玉树一般,比在家出落自是不同,长大了好些。问了回家中事务,管待了酒饭,见王经身上单薄,与了一件天青紵丝貂鼠氅衣儿,又与了五两银子,拿来回覆西门庆话。西门庆大喜。正与何千户下棋,忽闻绰道之声,门上人来报:“夏老爹来拜,拿进两个拜帖儿。”两个忙迎接到厅叙礼,何千户又谢昨日房子之事。夏公具了两分缎帕酒礼,奉贺二公。西门庆与何千户再三致谢,令左右收了。夏公又赏了贲四、玳安、王经十两银子,一面分宾主坐下。茶罢,共叙寒温。夏公道:“请老公公拜见。”何千户道:“家公公进内去了。”夏公又留下了一个双红拜帖儿,说道:“多顶上老公公,拜迟,恕罪!”言毕,起身去了。何千户随即也具一分贺礼,一匹金缎,差人送去,不在言表。

到晚夕,何千户又在花园暖阁中摆酒与西门庆共酌,家乐歌唱,到二更方寝。西门庆因昨日梦遗之事,晚夕令王经拿铺盖来书房地平上睡。半夜叫上床,搂在被窝内。两个口吐丁香,舌融甜唾。正是:

\[
不能得与莺莺会,且把红娘去解馋。
\]

一晚题过。到次日,起五更与何千户一行人跟随进朝。先到待漏院伺候,等的开了东华门进入。但见:

\[
星斗依稀禁漏残,禁中环佩响珊珊。
欲知今日天颜喜,遥睹蓬莱紫气皤。
\]

少顷,只听九重门启,鸣哕哕之鸾声;阊阖天开,睹巍巍之衮冕。当时天子祀毕南郊回来,文武百官聚集,等候设朝。须臾钟响,天子驾出大殿,受百官朝贺。须臾,香球拨转,帘卷扇开。正是:

\[
晴日明开青锁闼,天风吹下御炉香。
千条瑞霭浮金阙,一朵红云捧玉皇。
\]

这皇帝生得尧眉舜目,禹背汤肩,才俊过人,口工诗韵,善写墨君竹,能挥薛稷书,通三教之书,晓九流之典。朝欢暮乐,依稀似剑阁孟商王;爱色贪花,仿佛如金陵陈后主。当下驾坐宝位,静鞭响罢,文武百官秉简当胸,向丹墀五拜三叩头,进上表章。已而有殿头官口传圣旨道:“朕今即位二十祀矣。艮岳于兹告成,上天降瑞,今值覆端之庆,与卿共之。”言未毕,班首中闪过一员大臣来,朝靴踏地响,袍袖列风生。视之,乃左丞相崇政殿大学士兼吏部尚书太师鲁国公蔡京也。幞头象简,俯伏金阶,口称:“万岁,万岁,万万岁!臣等诚惶诚恐,稽首顿首,恭惟皇上御极二十祀以来,海宇清宁,天下丰稔,上天降鉴,祯祥叠见。三边永息兵戈,万国来朝天阙。银岳排空,玉京挺秀。宝箓膺颁于昊阙,绛宵深耸于乾宫。臣等何幸,欣逢盛世,交际明良,永效华封之祝,常沾日月之光。不胜瞻天仰圣,激切屏营之至!谨献颂以闻。”良久,圣旨下来:“贤卿献颂,益见忠诚,朕心嘉悦。诏改明年为重和元年,正月元旦受定命宝,肄赦覃赏有差。”蔡大师承旨下来。殿头官口传圣旨:“有事出班早奏,无事卷帘退朝。”言未毕,见一人出离班部,倒笏躬身,绯袍象简,玉带金鱼,跪在金阶,口称:“光禄大夫掌金吾卫事太尉太保兼太子太保臣朱勔,引天下提刑官员章隆等二十六员,例该考察,已更改补、缴换札付,合当引奏。未敢擅便,请旨定夺。”于是二十六员提刑官都跪在后面。不一时,圣旨传下来:“照例给领。”朱太尉承旨下来。天子袍袖一展,群臣皆散,驾即回宫。百官皆从端礼门两分而出。那十二象不待牵而先走,镇将长随纷纷而散。朝门外车马纵横,侍仗罗列。人喧呼,海沸波翻;马嘶喊,山崩地裂。众提刑官皆出朝上马,都来本衙门伺候。良久,只见知印拿了印牌来,传道:“老爷不进衙门了,已往蔡爷、李爷宅内拜冬去了。”以此众官都散了。

西门庆与何千户回到家中。又过了一夕,到次日,衙门中领了札付,又挂了号,又拜辞了翟管家,打点残装,收拾行李,与何千户一同起身。何太监晚夕置酒饯行,嘱咐何千户:“凡事请教西门大人,休要自专,差了礼数。”从十一月二十日东京起身,两家也有二十人跟随,竟往山东大道而来。已是数九严寒之际,点水滴冻之时,一路上见了些荒郊野路,枯木寒鸦。疏林淡日影斜晖,暮雪冻云迷晚渡。一山未尽一山来,后村已过前村望。比及刚过黄河,到水关八角镇,骤然撞遇天起一阵大风。但见:

\[
非干虎啸,岂是龙吟?卒律律寒飙扑面,急飕飕冷气侵人。初时节无踪无影,次后来卷雾收云。吹花摆柳白茫茫,走石扬砂昏惨惨。刮得那大树连声吼,惊得那孤雁落深濠。须臾,砂石打地,尘土遮天。砂石打地,犹如满天骤雨即时来;尘土遮天,好似百万貔貅卷土至。这风大不大?真个是吹折地狱门前树,乱起酆都顶上尘;常娥急把蟾官闭,列子空中叫救人。险些儿玉皇住不得昆仑顶,只刮得大地乾坤上下摇。
\]
西门庆与何千户坐着两顶毡帏暖轿,被风刮得寸步难行。又见天色渐晚,恐深林中撞出小人来,西门庆吩咐手下:“快寻那里安歇一夜,明日风住再行罢。”抓寻了半日,远远望见路旁一座古刹,数株疏柳,半堵横墙。但见:

\[
石砌碑横梦草遮,回廊古殿半欹斜。
夜深宿客无灯火,月落安禅更可嗟。
\]
西门庆与何千户忙入寺中投宿,上题着“黄龙寺”。见方丈内几个僧人在那里坐禅,又无灯火,房舍都毁坏,半用篱遮。长老出来问讯,旋吹火煮茶,伐草根喂马。煮出茶来,西门庆行囊中带得干鸡腊肉果饼之类,晚夕与何千户胡乱食得一顿。长老爨一锅豆粥吃了,过得一宿。次日风止天晴,与了和尚一两银子相谢,作辞起身往山东来。正是:

\[
王事驱驰岂惮劳,关山迢递赴京朝。
夜投古寺无烟火,解使行人心内焦。
\]

\newpage
%# -*- coding:utf-8 -*-
%%%%%%%%%%%%%%%%%%%%%%%%%%%%%%%%%%%%%%%%%%%%%%%%%%%%%%%%%%%%%%%%%%%%%%%%%%%%%%%%%%%%%


\chapter{潘金莲抠打如意儿\KG 王三官义拜西门庆}


词曰:

\[
掉臂叠肩情态,炎凉冷暖纷纭。兴来阉竖长儿孙,石女须教有孕。莫使一朝势谢,亲生不若他生。爹爹妈妈向何亲?掇转窟臀不认。
\]

话说西门庆与何千户在路不题。单表吴月娘在家,因西门庆上东京,见家中妇女多,恐惹是非,吩咐平安无事关好大门,后边仪门夜夜上锁。姊妹每都不出来,各自在房做针指。若敬济要往后楼上寻衣裳,月娘必使春鸿或来安儿跟出跟入。常时查门户,凡事都严紧了。这潘金莲因此不得和敬济勾搭。只赖奶子如意备了舌,逐日只和如意儿合气。

一日,月娘打点出西门庆许多衣服、汗衫、小衣,教如意儿同韩嫂儿浆洗。不想这边春梅也洗衣裳,使秋菊问他借棒槌。这如意儿正与迎春捶衣,不与他,说道:“前日你拿了个棒槌,使着罢了,又来要!趁韩嫂在这里,要替爹捶裤子和汗衫儿哩。”那秋菊使性子走来对春梅说:“平白教我借,他又不与。迎春倒说拿去,如意儿拦住了不肯。”春梅道:“耶嚛,耶嚛!怎的这等生分?大白日里借不出个干灯盏来。借个棒槌使使儿,就不肯与将来,替娘洗了这裹脚,教拿甚么捶?秋菊,你往后边问他们借来使使罢。”这潘金莲正在房中炕上裹脚,忽然听得,又因怀着仇恨,寻不着头由儿,便骂道:“贼淫妇怎的不与?你自家问他要去,不与,骂那淫妇不妨事。”这春梅一冲性子,就一阵风走来李瓶儿那边,说道:“那个是外人也怎的?棒槌借使使就不与。如今这屋里又钻出个当家的来了!”如意儿道:“耶嚛,耶嚛!放着棒槌拿去使不是,谁在这里把住?就怒说起来。大娘吩咐,趁韩妈在这里,替爹浆出这汗衫子和绵绸裤子来。秋菊来要,我说待我把你爹这衣服捶两下儿着,就架上许多诳,说不与来?早是迎春姐听着。”不想潘金莲随即跟了来,便骂道:“你这个老婆不要说嘴!死了你家主子,如今这屋里就是你?你爹身上衣服不着你恁个人儿拴束,谁应的上他那心!俺这些老婆死绝了,教你替他浆洗衣服?你拿这个法儿降伏俺每,我好耐惊耐怕儿!”如意儿道:“五娘怎的说这话?大娘不吩咐,俺们好掉揽替爹整理的?”金莲道:“贼\textuni{22C49}剌骨,雌汉的淫妇,还强说甚么嘴!半夜替爹递茶儿扶被儿是谁来?讨披袄儿穿是谁来?你背地干的那茧儿,你说我不知道?就偷出肚子来,我也不怕!”如意道:“正经有孩子还死了哩,俺每到的那些儿!”这金莲不听便罢,听了心头火起,粉面通红,走向前一把手把老婆头发扯住,只用手抠他腹。亏得韩嫂儿向前劝开了。金莲骂道:“没廉耻的淫妇,嘲汉的淫妇!俺每这里还闲的声唤,你来雌汉子,你在这屋里是甚么人?你就是来旺儿媳妇子从新又出世来了,我也不怕你!”那如意儿一壁哭着,一壁挽头发,说道:“俺每后来,也不知甚么来旺儿媳妇子,只知在爹家做奶子。”金莲道:“你做奶子,行你那奶子的事,怎的在屋里狐假虎威,成起精儿来?老娘成年拿雁,教你弄鬼儿去了!”

正骂着,只见孟玉楼后边慢慢的走将来,说道:“六姐,我请你后边下棋,你怎的不去,却在这里乱些甚么?”一把手拉到他房里坐下,说道:“你告我说,因为什么起来?”这金莲消了回气,春梅递上茶来,喝了些茶,便道:“你看教这贼淫妇气的我手也冷了,茶也拿不起来。我在屋里正描鞋,你使小鸾来请我,我说且躺躺儿去。\textuni{22C49}在床上也未睡着,只见这小肉儿百忙且捶裙子。我说你就带着把我的裹脚捶捶出来。半日只听的乱起来,却是秋菊问他要棒槌使,他不与,把棒槌匹手夺下了,说道:‘前日拿个去不见了,又来要!如今紧等着与爹捶衣服哩!’教我心里就恼起来,使了春梅去骂那贼淫妇:‘从几时就这等大胆降服人,俺每手里教你降伏!你是这屋里什么儿?压折轿竿儿娶你来?你比来旺儿媳妇子差些儿!’我就随跟了去,他还嘴里\textuni{25456}里剥剌的,教我一顿卷骂。不是韩嫂儿死气力赖在中间拉着我,我把贼没廉耻雌汉的淫妇口里肉也掏出他的来!大姐姐也有些不是,想着他把死的来旺儿贼奴才淫妇惯的有些折儿?教我和他为冤结仇,落后一染脓带还垛在我身上,说是我弄出那奴才去了。如今这个老婆,又是这般惯他,惯的恁没张倒置的。你做奶子行奶子的事,许你在跟前花黎胡哨?俺每眼里是放不下沙子的人。有那没廉耻的货,人也不知死的那里去了,还在那屋里缠。但往那里回来,就望着他那影作个揖,口里一似嚼蛆的,不知说些甚么。到晚夕要茶吃,淫妇就连忙起来替他送茶,又替他盖被儿,两个就弄将起来。就是个久惯的淫妇!只该丫头递茶,许你去撑头获脑雌汉子?为什么问他要披袄儿,没廉耻的便连忙铺里拿了绸段来,替他裁披袄儿?你还没见哩:断七那日,他爹进屋里烧纸去,见丫头、老婆在炕上挝子儿,就不说一声儿,反说道:‘这供养的匾食和酒,也不要收到后边去,你每吃了罢。’这等纵容着他。这淫妇还说:‘爹来不来?俺每好等的。’不想我两三步叉进去,唬得他眼张失道,就不言语了。什么好老婆?一个贼活人妻淫妇,就这等饿眼见瓜皮,不管好歹的都收揽下。原来是一个眼里火烂桃行货子。那淫妇的汉子说死了。前日汉子抱着孩子,没在门首打探儿?还瞒着人捣鬼,张眼溜睛的。你看他如今别模改样的,又是个李瓶儿出世了!那大姐姐成日在后边只推聋装哑的,人但开口,就说不是了。”那玉楼听了,只是笑。因说:“你怎知道的这等详细?”金莲道:“南京沈万三,北京枯柳树。人的名儿,树的影儿,怎么不晓得?雪里埋死尸——自然消将出来。”玉楼道:“原说这老婆没汉子,如何又钻出汉子来了?”金莲道:“天下着风儿晴不的,人不着谎儿成不的!他不撺瞒着,你家肯要他!想着一来时,饿答的个脸,黄皮寡瘦的,乞乞缩缩那个腔儿!吃了这二年饱饭,就生事儿,雌起汉子来了。你如今不禁下他来,到明日又教他上头上脸的。一时捅出个孩子,当谁的?”玉楼笑道:“你这六丫头,到且是有权属。”说毕,坐了一回,两个往后边下棋去了。正是:

\[
三光有影遗谁系?万事无根只自生。
\]

话休饶舌,有日后晌时分,西门庆来到清河县。吩咐贲四、王经跟行李先往家去,他便送何千户到衙门中,看着收拾打扫公廨干净住下,方才骑马来家。进入后厅,吴月娘接着,舀水净面毕,就令丫鬟院子内放桌儿,满炉焚香,对天地位下告许愿心。月娘便问:“你为什么许愿心?”西门庆道:“休说起,我拾得性命来家。昨日十一月二十三日,刚过黄河,行到沂水县八角镇上,遭遇大风,沙石迷目,通行不得。天色又晚,百里不见人,众人都慌了。况驮垛又多,诚恐钻出个贼来怎了?比及投到个古寺中,和尚又穷,夜晚连灯火也没个儿,只吃些豆粥儿就过了一夜。次日风住,方才起身,这场苦比前日更苦十分。前日虽热,天还好些。这遭又是寒冷天气,又耽许多惊怕。幸得平地还罢了,若在黄河遭此风浪怎了?我在路上就许了愿心,到腊月初一日,宰猪羊祭赛天地。”月娘又问:“你头里怎不来家,却往衙门里做甚么?”西门庆道:“夏龙溪已升做指挥直驾,不得来了。新升是匠作监何太监侄儿何千户——名永寿,贴刑,不上二十岁,捏出水儿来的一个小后生,任事儿不知道。他太监再三央及我,凡事看顾教导他。我不送到衙门里安顿他个住处,他知道甚么?他如今一千二百两银子——也是我作成他——要了夏龙溪那房子,直待夏家搬取了家小去,他的家眷才搬来。前日夏大人不知什么人走了风与他,他又使了银子,央当朝林真人分上,对堂上朱太尉说,情愿以指挥职衔再要提刑三年。朱太尉来对老爷说,把老爷难的要不得。若不是翟亲家在中间竭力维持,把我撑在空地里去了。去时亲家好不怪我,说我干事不谨密。不知是什么人对他说来。”月娘道:“不是我说,你做事有些三慌子火燎腿样,有不的些事儿,告这个说一场,告那个说一场,恰似逞强卖富的。正是有心算无心,不备怎提备?人家悄悄干的事儿停停妥妥,你还不知道哩!”西门庆又说:“夏大人临来,再三央我早晚看顾看顾他家里,容日你买分礼儿走走去。”月娘道:“他娘子出月初二日生日,就一事儿去罢。你今后把这狂样来改了。常言道:‘逢人且说三分清,未可全抛一片心。’老婆还有个里外心儿,休说世人。”

正说着,只见玳安来说:“贲四问爹,要往夏大人家说去不去?”西门庆道:“你教他吃了饭去。”玳安应诺去了。李娇儿、孟玉楼、孙雪娥、潘金莲、大姐都来参见道万福,问话儿,陪坐的。西门庆又想起前番往东京回来,还有李瓶儿在,一面走到他房内,与他灵床作揖,因落了几点眼泪。如意儿、迎春、绣春都向前磕头。月娘随即使小玉请在后边,摆饭吃了,一面吩咐拿出四两银子,赏跟随小马儿上的人,拿帖儿回谢周守备去了。又叫来兴儿宰了半口猪、半腔羊、四十斤白面、一包白米、一坛酒、两腿火熏、两只鹅、十只鸡,又并许多油盐酱醋之类,与何千户送下程。又叫了一名厨役在那里答应。

正在厅上打点,忽琴童儿进来说道:“温师父和应二爹来望。”西门庆连忙请进温秀才、伯爵来。二人连连作揖,道其风霜辛苦。西门庆亦道:“蒙二公早晚看家。”伯爵道:“我早起来时,忽听房上喜鹊喳喳的叫。俺房下就先说:‘只怕大官人来家了,你还不快走了瞧瞧去?’我便说:‘哥从十二日起身,到今还未上半个月,怎能来得快?’房下说:‘来不来,你看看去!’教我穿衣裳到宅里,不想哥真个来家了。恭喜恭喜!”因见许多下饭酒米装在厅台上,便问道:“送谁家的?”西门庆道:“新同僚何大人,一路同来,家小还未到。今在衙门中权住,送份下程与他。又发柬明日请他吃接风酒,再没人,请二位与吴大舅奉陪。”伯爵道:“又一件:吴大舅与哥是官,温老先生戴着方巾,我一个小帽儿怎陪得他坐!不知把我当甚么人儿看,我惹他不笑话?”西门庆笑道:“这等把我买的缎子忠靖巾借与你戴着,等他问你,只说是我的大儿子,好不好?”说毕,众人笑了。伯爵道:“说正经话,我头八寸三,又戴不得你的。”温秀才道:“学生也是八寸三分,倒将学生方巾与老翁戴戴何如?”西门庆道:“老先生不要借与他,他到明日借惯了,往礼部当官身去,又来缠你。”温秀才笑道:“老先生好说,连我也扯下水去了。”少顷,拿上茶来吃了。温秀才问:“夏公已是京任,不来了?”西门庆道:“他已做堂尊了,直掌卤簿,穿麟服,使藤棍,如此华任,又来做甚么!”须臾,看写了帖子,抬下程出门,教玳安送去了。西门庆就拉温秀才、伯爵到厢房内暖炕上坐去了。又使琴童往院里叫吴惠、郑春、邵奉、左顺四名小优儿明日早来伺候。

不一时,放桌儿陪二人吃酒。西门庆吩咐:“再取双钟箸儿,请你姐夫来坐坐。”良久,陈敬济走来,作揖,打横坐下。四人围炉把酒来斟,因说起一路上受惊的话。伯爵道:“哥,你的心好,一福能压百祸,就有小人,一时自然都消散了。”温秀才道:“善人为邦百年,亦可以胜残去杀。休道老先生为王事驱驰,上天也不肯有伤善类。”西门庆因问:“家中没甚事?”敬济道:“家中无事。只是工部安老爹那里差人来问了两遭,昨日还来问,我回说还没来家哩。”

正说着,忽有平安来报:“衙门令史和众节级来禀事。”西门庆即到厅上站立,令他进见。二人跪下:“请问老爹几时上任?官司公用银两动支多少?”西门庆道:“你们只照旧时整理就是了。”令史道:“去年只老爹一位到任,如今老爹转正,何老爹新到任,两事并举,比旧不同。”西门庆道:“既是如此,添十两银子与他就是了。”二人应喏下去。西门庆又叫回来吩咐:“上任日期,你还问何老爹择几时。”二人道:“何老爹择定二十六日。”西门庆道:“既如此,你每伺候就是了。”二人去了。就是乔大人来拜望道喜。西门庆留坐不肯,吃茶起身去了。西门庆进来,陪二人饮至掌灯方散。西门庆往月娘房里歇了一宿。

到次日,家中置酒,与何千户接风。文嫂又早打听得西门庆来家,对王三官说了,具个柬帖儿来请。西门庆这里买了一副豕蹄、两尾鲜鱼、两只烧鸭、一坛南酒,差玳安送去,与太太补生日之礼。他那里赏了玳安三钱银子,不在话下。正厅上设下酒,锦屏耀目,桌椅鲜明。吴大舅、应伯爵、温秀才都来的早,西门庆陪坐吃茶,使人邀请何千户。不一时,小优儿上来磕头。伯爵便问:“哥,今日怎的不叫李铭?”西门庆道:“他不来我家来,我没的请他去!”

正说话,只见平安忙拿帖儿禀说:“帅府周爷来拜,下马了。”吴大舅、温秀才、应伯爵都躲在西厢房内。西门庆冠带出来,迎至厅上,叙礼毕,道及转升恭喜之事。西门庆又谢他人马。于是分宾主而坐。周守备问京中见朝之事,西门庆一一说了。周守备道:“龙溪不来,一定差人来取家小上京去。”西门庆道:“就取也待出月。如今何长官且在衙门权住着哩。夏公的房子与了他住,也是我替他主张的。”守备道:“这等更妙。”因见堂中摆设桌席,问道:“今日所延甚客?”西门庆道:“聊具一酌,与何大人接风。同僚之间,不好意思。”二人吃了茶,周守备起身,说道:“容日合卫列位,与二公奉贺。”西门庆道:“岂敢动劳,多承先施。”作揖出门,上马而去。西门庆回来,脱了衣服,又陪三人在书房中摆饭。何千户到午后方来,吴大舅等各相见叙礼毕,各叙寒温。茶汤换罢,各宽衣服。何千户见西门庆家道相称,酒筵齐整。四个小优银筝象板,玉阮琵琶,递酒上坐。直饮至起更时分,何千户方起身往衙门中去了。吴大舅、应伯爵、温秀才也辞回去了。

西门庆打发小优儿出门,吩咐收了家伙,就往前边金莲房中来。妇人在房内浓施朱粉,复整新妆,薰香澡牝,正盼西门庆进他房来,满面笑容,向前替他脱衣解带,连忙叫春梅点茶与他吃了,打发上床歇宿。端的被窝中相挨素体,枕席上紧贴酥胸,妇人云雨之际,百媚俱生。西门庆抽拽之后,灵犀已透,睡不着,枕上把离言深讲。交接后,淫情未足,又从下替他品箫。这妇人只要拴西门庆之心,又况抛离了半月在家,久旷幽怀,淫情似火,得到身,恨不得钻入他腹中。将那话品弄了一夜,再不离口。西门庆要下床溺尿,妇人还不放,说道:“我的亲亲,你有多少尿,溺在奴口里,替你咽了罢,省的冷呵呵的,热身子下去冻着,倒值了多的。”西门庆听了,越发欢喜无已,叫道:“乖乖儿,谁似你这般疼我!”于是真个溺在妇人口内。妇人用口接着,慢慢一口一口都咽了。西门庆问道:“好吃不好吃?”金莲道:“略有些咸味儿。你有香茶与我些压压。”西门庆道:“香茶在我白绫袄内,你自家拿。”这妇人向床头拉过他袖子来,掏摸了几个放在口内,才罢。正是:

\[
侍臣不及相如渴,特赐金茎露一杯。
\]

看官听说:大抵妾妇之道,鼓惑其夫,无所不至,虽屈身忍辱,殆不为耻。若夫正室之妻,光明正大,岂肯为也!是夜,西门庆与妇人盘桓无度。

次早往衙门中与何千户上任,吃公宴酒,两院乐工动乐承应。午后才回家,排军随即抬了桌席来。王三官那里又差人早来邀请。西门庆才收拾出来,左右来报:“工部安老爹来拜。”慌的西门庆整衣出来迎接。安郎中食寺丞的俸,系金镶带,穿白鹇补子,跟着许多官吏,满面笑容,相携到厅叙礼,彼此道及恭贺,分宾主坐下。安郎中道:“学生差人来问几次,说四泉还未回。”西门庆道:“正是。京中要等见朝引奏,才起身回来。”须臾,茶汤吃罢,安郎中方说:“学生敬来有一事不当奉渎:今有九江太府蔡少塘,乃是蔡老先生第九公子,来上京朝觐,前日有书来,早晚便到。学生与宋松泉、钱云野、黄泰宇四人作东,欲借府上设席请他,未知允否?”西门庆道:“老先生尊命,岂敢有违。约定几时?”安郎中道:“在二十七日。明日学生送分子过来,烦盛使一办,足见厚爱矣。”说毕,又上了一道茶,作辞,起身上马,喝道而去。

西门庆即出门,往王招宣府中来赴席。到门首,先投了拜帖。王三官连忙出来迎接,至厅上叙礼。大厅正面钦赐牌额,金字题曰“世忠堂”,两边门对写着“乔木风霜古,山河\textuni{25567}砺新”。王三官与西门庆行毕礼,尊西门庆上坐,他便傍设一椅相陪。须臾拿上茶来,交手递了茶,左右收了去。彼此扳了些说话,然后安排酒筵递酒。原来王三官叫了两名小优儿弹唱。西门庆道:“请出老太太拜见拜见。”慌的王三官令左右后边说。少顷,出来说道:“请老爹后边见罢。”王三官让西门庆进内。西门庆道:“贤契,你先导引。”于是迳入中堂。林氏又早戴着满头珠翠,身穿大红通袖袍儿,腰系金镶碧玉带,下着玄锦百花裙,搽抹的如银人也一般。西门庆一面施礼:“请太太转上。”林氏道:“大人是客,请转上。”让了半日,两个人平磕头,林氏道:“小儿不识好歹,前日冲渎大人。蒙大人又处断了那些人,知感不尽。今日备了一杯水酒,请大人过来,老身磕个头儿谢谢。如何又蒙大人赐将礼来?使我老身却之不恭,受之有愧。”西门庆道:“岂敢。学生因为公事往东京去了,误了与老太太拜寿。些须薄礼,胡乱送与老太太赏人。”因见文嫂儿在旁,便道:“老文,你取副盏儿来,等我与太太递一杯寿酒。”一面呼玳安上来。原来西门庆毡包内,预备着一套遍地金时样衣服,放在盘内献上。林氏一见,金彩夺目,满心欢喜。文嫂随即捧上金盏银台。王三官便要叫小优拿乐器进来弹唱。林氏道:“你叫他进来做甚么?在外答应罢了。”当下,西门庆把盏毕,林氏也回奉了一盏与西门庆谢了。然后王三官与西门庆递酒,西门庆才待还下礼去,林氏便道:“大人请起,受他一礼儿。”西门庆道:“不敢,岂有此礼?”林氏道:“好大人,怎这般说!你恁大职级,做不起他个父亲!小儿自幼失学,不曾跟着好人。若是大人肯垂爱,凡事指教他为个好人,今日我跟前,就教他拜大人做了义父。但有不是处,一任大人教诲,老身并不护短。”西门庆道:“老太太虽故说得是,但令郎贤契,赋性也聪明,如今年少,为小试行道之端,往后自然心地开阔,改过迁善。老太太倒不必介意。”当下教西门庆转上,王三官把盏,递了三钟酒,受其四拜之礼。递毕,西门庆亦转下与林氏作揖谢礼,林氏笑吟吟还了万福。自此以后,王三官见着西门庆以父称之。正是:常将压善欺良意,权作尤云殢雨心。复有诗以叹之:

\[
从来男女不通酬,卖俏营奸真可羞。
三官不解其中意,饶贴亲娘还磕头。
\]

递毕酒,林氏吩咐王三官:“请大人前边坐,宽衣服。”玳安拿忠靖巾来换了。不一时,安席坐下。小优弹唱起来,厨役上来割道,玳安拿赏赐伺候。当下食割五道,歌吟二套,秉烛上来,西门庆起身告辞。王三官再三款留,又邀到他书院中。独独的三间小轩里面,花竹掩映,文物潇洒。正面悬着一个金粉笺扁,曰“三泉诗舫”,四壁挂四轴古画。西门庆便问:“三泉是何人?”王三官只顾隐避,不敢回答。半日才说:“是儿子的贱号。”西门庆便一声儿没言语。抬过高壶来,又投壶饮酒。四个小优儿在旁弹唱。林氏后边只顾打发添换菜蔬果碟儿上来。

吃到二更时分,西门庆已带半酣,方才起身,赏了小优儿并厨役,作辞回家。到家迳往金莲房中。原来妇人还没睡,才摘去冠儿,挽着云髻,淡妆浓抹,正在房内茶烹玉蕊,香袅金猊等待。见西门庆进来,欢喜无限。忙向前接了衣裳,叫春梅点了一盏雀舌芽茶与西门庆吃。西门庆吃了,然后春梅脱靴解带,打发上床。妇人在灯下摘去首饰,换了睡鞋,上床并头交股而寝。西门庆将一只胳膊与妇人枕着,搂在怀中,犹如软玉温香一般,两个酥胸相贴,脸儿厮揾,鸣咂其舌。不一时,甜唾融心,灵犀春透。妇人不住手下边捏弄他那话。西门庆因问道:“我的儿,我不在家,你想我不想?”妇人道:“你去了这半个来月,奴那刻儿放下心来!晚间夜又长,独自一个偏睡不着。随问怎的暖床暖铺,只是害冷。腿儿触冷伸不开,只得忍酸儿缩着,白盼不到,枕边眼泪不知流了多少。落后春梅小肉儿见我短叹长吁,晚间逗着我下棋,坐到起更时分,俺娘儿两个一炕儿通厮脚儿睡。我的哥哥,奴心便是如此,不知你的心儿如何?”西门庆道:怪油嘴,这一家虽是有他们,谁不知我在你身上偏多。”妇人道:“罢么,你还哄我哩!你那吃着碗里看着锅里的心儿,你说我不知道?想着你和来旺儿媳妇子蜜调油也似的,把我来就不理了。落后李瓶儿生了孩子,见我如同乌眼鸡一般。今日都往那里去了?止是奴老实的还在。你就是那风里杨花,滚上滚下,如今又兴起如意儿贼\textuni{22C49}剌骨来了。他随问怎的,只是奶子,见放着他汉子,是个活人妻。不争你要了他,到明日又教汉子好在门首放羊儿剌剌。你为官为宦,传出去好听?你看这贼淫妇,前日你去了,同春梅两个为一个棒槌,和我大嚷大闹,通不让我一句儿。”西门庆道:“罢么,我的儿,他随问怎的,只是个手下人。他那里有七个头八个胆敢顶撞你?你高高手儿他过去了,低低手儿他敢过不去。”妇人道:“\textuni{35BF}嚛,说的倒好听!没了李瓶儿,他就顶了窝儿。学你对他说:‘你若伏侍的好,我把娘这分家当就与你罢。’你真个有这个话来?”西门庆道:“你休胡猜疑,我那里有此话!你宽恕他,我教他明日与你磕头陪不是罢。”妇人道:“我也不要他陪不是,我也不许你到那屋里睡。”西门庆道:“我在那边睡,非为别的,因越不过李大姐情,在那边守守灵儿,谁和他有私盐私醋!”妇人道:“我不信你这摭溜子。人也死了一百日来,还守什么灵?在那屋里也不是守灵,属米仓的,上半夜摇铃,下半夜丫头听的好梆声。”几句说的西门庆急了,搂过脖子来亲了个嘴,说道:“怪小淫妇儿,有这些张致的!”于是令他吊过身子去,隔山讨火,那话自后插入牝中,接抱其股,竭力扇磞的连声响亮。一面令妇人呼叫大东大西,问道:“你怕我不怕?再敢管着!”妇人道:“怪奴才,不管着你好上天也!我晓的你也丢不开这淫妇,到明日,问了我方许你那边去。他若问你要东西,须对我说,只不许你悄悄偷与他。若不依,我打听出来,看我嚷不嚷!我就摈兑了这淫妇,也不差甚么儿。又相李瓶儿来头,教你哄了,险些不把我打到赘字号去。你这烂桃行货子,豆芽莱——有甚正条捆儿也怎的?老娘如今也贼了些儿了。”说的西门庆笑了。当下两个殢雨尤云,缠到三更方歇。正是:

\[
带雨笼烟世所稀,妖娆身势似难支。
终宵故把芳心诉,留得东风不放归。
\]

两个并头交股睡到天明,妇人淫情未足,便不住手捏弄那话,登时把麈柄捏弄起来,叫道:“亲达达,我一心要你身上睡睡。”一面爬伏在西门庆身上倒浇烛,接着他脖子只顾揉搓,教西门庆两手扳住他腰,扳的紧紧的,他便在上极力抽提,一面爬伏在他身上揉一回,那话渐没至根,余者被托子所阻,不能入。妇人便道:“我的达达,等我白日里替你作一条白绫带子,你把和尚与你的那末子药装些在里面,我再坠上两根长带儿。等睡时,你扎他在根子上,却拿这两根带扎拴后边腰里,拴的紧紧的,又柔软,又得全放进,却不强如这托子硬硬的,格的人疼?”西门庆道:“我的儿,你做下,药在磁盒儿内,你自家装上就是了。”妇人道:“你黑夜好歹来,咱两个试试看好不好?”于是,两个玩耍一番。

只见玳安拿帖儿进来,问春梅:“爹起身不曾?安老爹差人送分资来了。又抬了两坛酒、四盆花树进来。”春梅道:“爹还没起身,教他等等儿。”玳安道:“他好少近路儿,还要赶新河口闸上回话哩。”不想西门庆在房中听见,隔窗叫玳安问了话,拿帖儿进去,拆开看,上写道:

\[
奉去分资四封,共八两。惟少塘桌席,余者散酌而已。仰冀从者留神,足见厚爱之至。外具时花四盆,以供清玩;浙酒二樽,少助待客之需。希莞纳,幸甚。
\]
西门庆看了,一面起身,且不梳头,戴着毡巾,穿着绒氅衣走出厅上,令安老爹人进见。递上分资。西门庆见四盆花草:一盆红梅、一盆白梅、一盆茉莉、一盆辛夷,两坛南酒,满心欢喜。连忙收了。发了回帖,赏了来人五钱银子,因问:“老爹们明日多咱时分来?用戏子不用?”来人道:“都早来。戏子用海盐的。”说毕,打发去了。西门庆叫左右把花草抬放藏春坞书房中摆放,一面使玳安叫戏子去,一面兑银子与来安儿买办。那日又是孟玉楼上寿,院中叫小优儿晚夕弹唱。

按下一头。却说应伯爵在家,拿了五个笺帖,教应保捧着盒儿,往西门庆对过房子内央温秀才写请书。要请西门庆五位夫人,二十八日家中做满月。刚出门转过街口,只见后边一人高叫道:“二爹请回来!”伯爵扭头回看是李铭,立住了脚。李铭走到跟前,问道:“二爹往那里去?”伯爵道:“我到温师父那里有些事儿去。”李铭道:“到家中还有句话儿说。”只见后边一个闲汉,掇着盒儿,伯爵不免又到家堂屋内。李铭连忙磕了个头,把盒儿掇进来放下,揭开却是烧鸭二只、老酒二瓶,说道:“小人没甚,这些微物儿孝顺二爹赏人。小的有句话迳来央及二爹。”一面跪在地下不起来。伯爵一把手拉起来,说道:“傻孩儿,你有话只管说,怎的买礼来?”李铭道:“小的从小儿在爹宅内,答应这几年,如今爹到看顾别人,不用小的了。就是桂姐那边的事,各门各户,小的实不知道。如今爹因怪那边,连小的也怪了。这负屈衔冤,没处伸诉,迳来告二爹。二爹到宅内见爹,千万替小的加句美言儿说说。就是桂姐有些一差半错,不干小的事。爹动意恼小的不打紧,同行中人越发欺负小的了。”伯爵道:“你原来这些时没往宅内答应去。”李铭道:“小的没曾去。”伯爵道:“嗔道昨日摆酒与何老爹接风,叫了吴惠、郑春、邵奉、左顺在那里答应,我说怎的不见你。我问你爹,你爹说:‘他没来,我没的请他去!’傻孩儿,你还不走跳些儿还好?你与谁赌气?”李铭道:“爹宅内不呼唤,小的怎的好去?前日他每四个在那里答应,今日三娘上寿,安官儿早晨又叫了两名去了;明日老爹摆酒,又是他们四个。倒没小的,小的心里怎么有个不急的!只望二爹替小的说个明白,小的还来与二爹磕头。”伯爵道:“我没有个不替你说的。我从前已往不知替人完美了多少勾当,你央及我这些事儿,我不替你说?你依着我,把这礼儿你还拿回去。你是那里钱儿,我受你的!你如今就跟了我去,等我慢慢和你爹说。”李铭道:“二爹不收此礼,小的也不敢去了。虽然二爹不希罕,也尽小的一点穷心。”再三央告,伯爵把礼收了。讨出三十文钱,打发拿盒人回去。于是同出门,来到西门庆对门房子里。进到书院门首,摇的门环儿响,说道:“葵轩老先生在家么?”温秀才正在书窗下写帖儿,忙应道:“请里面坐。”画童开门,伯爵在明间内坐的。温秀才即出来相见,叙礼让坐,说道:“老翁起来的早,往那里去来?”伯爵道:“敢来烦渎大笔写几个请书儿。如此这般,二十八日小儿满月,请宅内他娘们坐坐。”温秀才道:“帖在那里?将来学生写。”伯爵即令应保取出五个帖儿,递过去。温秀才拿到房内,才写得两个,只见棋童慌走来说道:“温师父,再写两个帖儿——大娘的名字,要请乔亲家娘和大妗子去。头里琴童来取门外韩大姨和孟二妗子那两个帖儿,打发去了不曾?”温秀才道:“你姐夫看着,打发去这半日了。”棋童道:“温师父写了这两个,还再写上四个,请黄四婶、傅大娘、韩大婶和甘伙计娘子的,我使来安儿来取。”不一时打发去了。只见来安来取这四个帖儿,伯爵问:“你爹在家里,是衙门中去了?”来安道:“爹今日没往衙门里去,在厅上看收礼哩。”温秀才道:“老先生昨日王宅赴席来晚了。”伯爵问起那王宅,温秀才道:“是招宣府中。”伯爵就知其故。良久,来安等了帖儿去,方才与伯爵写完。伯爵即带了李铭过这边来。

西门庆蓬着头,只在厅上收礼,打发回帖,旁边排摆桌面。见伯爵来,唱喏让坐。伯爵谢前日厚情,因问:“哥定这桌席做什么?”西门庆把安郎中来央浼作东,请蔡知府之事,告他说了一遍。伯爵道:“明日是戏子是小优?”西门庆道:“叫了一起海盐子弟,我这里又预备四名小优儿答应。”伯爵道:“哥,那四个?”西门庆道:“吴惠、邵奉、郑春、左顺。”伯爵道:“哥怎的不用李铭?”西门庆道:“他已有了高枝儿,又稀罕我这里做什么?”伯爵道:“哥怎的说这个话?你唤他,他才敢来。我也不知道你一向恼他。但是各人勾当,不干他事。三婶那边干事,他怎的晓得?你到休要屈了他。他今早到我那里,哭哭啼啼告诉我:‘休说小的姐姐在爹宅内,只小的答应该几年,今日有了别人,到没小的。’他再三赌身罚咒,并不知他三婶那边一字儿。你若恼他,却不难为他了。他小人有什么大汤水儿?你若动动意儿,他怎的禁得起!”便教李铭:“你过来,亲自告诉你爹。你只顾躲着怎的?自古丑媳妇免不得见公婆。”

那李铭站在槅子边,低头敛足,就似僻厅鬼儿一般看着二人说话。听得伯爵叫他,连忙走进去,跪着地下,只顾磕头,说道:“爹再访,那边事小的但有一字知道,小的车碾马踏,遭官刑揲死。爹从前已往,天高地厚之恩,小的一家粉身碎骨也报不过来。不争今日恼小的,惹的同行人耻笑,他也欺负小的,小的再向那里寻个主儿?”说毕,号淘痛哭,跪在地下只顾不起来。伯爵在旁道:“罢么,哥也是看他一场。大人不见小人之过,休说没他不是,就是他有不是处,他既如此,你也将就可恕他罢。”又叫李铭:“你过来,自古穿青衣抱黑柱,你爹既说开,就不恼你了,你往后也要谨慎些。”李铭道:“二爹说的是,知过必改,往后知道了。”西门庆沉吟半晌,便道:“既你二爹再三说,我不恼你了,起来答应罢。”伯爵道:”你还不快磕头哩!”那李铭连忙磕个头,立在旁边。伯爵方才令应保取出五个请帖儿来,递与西门庆道:“二十八日小儿弥月,请列位嫂子过舍光降光降。”西门庆看毕,教来安儿:“连盒儿送与大娘瞧去。——管情后日去不成。实和你说,明日是你三娘生日,家中又是安郎中摆酒,二十八日他又要看夏大人娘子去,如何去的成?”伯爵道:“哥杀人哩!嫂子不去,满园中果子儿,再靠着谁哩!我就亲自进屋里请去。”少顷,只见来安拿出空盒子来了:“大娘说,多上覆,知道了。”伯爵把盒儿递与应保接去,笑了道:“哥,你就哄我起来。若是嫂子不去,我就把头磕烂了,也好歹请嫂子走走去。”西门庆教伯爵:“你且休去,等我梳起头来,咱每吃饭。”说毕,入后边去了。

这伯爵便向李铭道:“如何?刚才不是我这般说着,他甚是恼你。他有钱的性儿,随他说几句罢了。常言:嗔拳不打笑面。如今时年,尚个奉承的。拿着大本钱做买卖,还带三分和气。你若撑硬船儿,谁理你!全要随机应变,似水儿活,才得转出钱来。你若撞东墙,别人吃饭饱了,你还忍饿。你答应他几年,还不知他性儿?明日交你桂姐赶热脚儿来,两当一:就与三娘做生日,就与他陪了礼儿来,一天事都了了。”李铭道:“二爹说的是。小的到家,过去就对三妈说。”说着,只见来安儿放桌儿,说道:“应二爹请坐,爹就出来。”

不一时,西门庆梳洗出来,陪伯爵坐的,问他:“你连日不见老孙、祝麻子?”伯爵道:“我令他来,他知道哥恼他。我便说:‘还是哥十分情分,看上顾下,那日蜢虫蚂炸一例扑了去,你敢怎样的!’他每发下誓,再不和王家小厮走。说哥昨日在他家吃酒来?他每也不知道。”西门庆道:“昨日他如此这般,置了一席大酒请我,拜认我做干老子,吃到二更来了。他每怎的再不和他来往?只不干碍着我的事,随他去,我管他怎的?我不真是他老子,管他不成!”伯爵道:“哥这话说绝了。他两个,一二日也要来与你服个礼儿,解释解释。”西门庆道:“你教他只顾来,平白服甚礼?”一面来安儿拿上饭来,无非是炮烹美口肴馔。西门庆吃粥,伯爵用饭。吃毕,西门庆问:“那两个小优儿来了不曾?”来安道:“来了这一日了。”西门庆叫他和李铭一答儿吃饭。一个韩佐,一个邵谦,向前来磕了头,下边吃饭去了。

良久,伯爵起身,说道:“我去罢,家里不知怎样等着我哩。小人家儿干事最苦,从炉台底下直买到堂屋门首,那些儿不要买?”西门庆道:“你去干了事,晚间来坐坐,与你三娘上寿,磕个头儿,也是你的孝顺。”伯爵道:“这个一定来,还教房下送人情来。”说毕,一直去了。正是:

\[
酒深情不厌,知己话偏长。
莫负相钦重,明朝到草堂。
\]

\newpage
%# -*- coding:utf-8 -*-
%%%%%%%%%%%%%%%%%%%%%%%%%%%%%%%%%%%%%%%%%%%%%%%%%%%%%%%%%%%%%%%%%%%%%%%%%%%%%%%%%%%%%


\chapter{潘金莲不愤忆吹箫\KG 西门庆新试白绫带}


词曰:

\[
唤多情,忆多情,谁把多情唤我名?唤名人可憎。为多情,转多情,死向多情心不平。休教情重轻。
\]

话说应伯爵回家去了。西门庆就在藏春坞坐着,看泥水匠打地炕。墙外烧火,安放花草,庶不至煤烟熏触。忽见平安拿进帖儿,禀说:“帅府周爷差人送分资来了。”盒内封着五封分资:周守备、荆都监、张团练、刘薛二内相,每人五星,粗帕二方,奉引贺敬。西门庆令左右收入后边,拿回帖打发去了。

且说那日,杨姑娘与吴大妗子、潘姥姥坐轿子先来了,然后薛姑子、大师父、王姑子,并两个小姑子妙趣、妙凤,并郁大姐,都买了盒儿来,与玉楼做生日。月娘在上房摆茶,众姊妹都在一处陪侍。须臾吃了茶,各人取便坐了。

潘金莲想着要与西门庆做白绫带儿,即便走到房里,拿过针线匣,拣一条白绫儿,将磁盒内颤声娇药末儿装在里面,周围用倒口针儿撩缝的甚是细法,预备晚夕要与西门庆云雨之欢。不想薛姑子蓦地进房来,送那安胎气的衣胞符药与他。这妇人连忙收过,一面陪他坐的。薛姑子见左右无人,便悄悄递与他,说道:“你拣个壬子日空心服,到晚夕与官人在一处,管情一度就成胎气。你看后边大菩萨,也是贫僧替他安的胎,今已有了半肚子了。我还说个法儿与你:缝个锦香囊,我书道朱砂符儿安在里面,带在身边,管情就是男胎,好不准验。”这妇人听了,满心欢喜,一面接了符药,藏放在箱内。拿过历日来看,二十九日是壬子日。于是就称了三钱银子送与他,说:“这个不当什么,拿到家买菜吃。等坐胎之时,我寻匹绢与你做衣穿。”薛姑子道:“菩萨快休计较,我不象王和尚那样利心重。前者因过世那位菩萨念经,他说我搀了他的主顾,好不和我嚷闹,到处拿言语丧我。我的爷,随他堕业,我不与他争执。我只替人家行好事,救人苦难。”妇人道:“薛爷,你只行你的事,各人心地不同。我这勾当,你也休和他说。”薛姑子道:“法不传六耳,我肯和他说!去年为后边大菩萨喜事,他还说我背地得多少钱,擗了一半与他才罢了。一个僧家,戒行也不知,利心又重,得了十方施主钱粮,不修功果,到明日死后,披毛戴角还不起。”说了回话,妇人教春梅:“看茶与薛爷吃。”那姑子吃了茶,又同他到李瓶儿那边参了灵,方归后边来。

约后晌时分,月娘放桌儿炕屋里,请众堂客并三个姑子坐的。又在明间内放八仙桌儿,铺着火盆摆下案酒,与孟玉楼上寿。不一时,琼浆满泛,玉\textuni{659D}高擎,孟玉楼打扮的粉妆玉琢,先与西门庆递了酒,然后与众姊妹叙礼,安席而坐。陈敬济和大姐又与玉楼上寿,行毕礼,就在旁边坐下。厨下寿面点心添换,一齐拿上来。众人才吃酒,只见来安拿进盒儿来说:“应保送人情来了。”西门庆叫月娘收了,就教来安:“送应二娘帖儿去,就请你应二爹和大舅来坐坐。我晓的他娘子儿,明日也是不来,请你二爹来坐坐罢,改日回人情与他就是了。”来安拿帖儿同应保去了。西门庆坐在上面,不觉想起去年玉楼上寿还有李大姐,今日妻妾五个,只少了他,由不得心中痛酸,眼中落泪。

不一时,李铭和两个小优儿进来了。月娘吩咐:“你会唱‘比翼成连理’不会?”韩佐道:“小的记得。”才待拿起乐器来弹唱,被西门庆叫近前,吩咐:“你唱一套‘忆吹箫’我听罢。”两个小优连忙改调唱《集贤宾》“忆吹箫,玉人何处也。”唱了一回,唱到“他为我褪湘裙杜鹃花上血”,潘金莲见唱此词,就知西门庆念思李瓶儿之意。及唱到此句,在席上故意把手放在脸儿上,这点儿那点儿羞他,说道:“孩儿,那里猪八戒走在冷铺中坐着——你怎的丑的没对儿!一个后婚老婆,又不是女儿,那里讨‘杜鹃花上血’来?好个没羞的行货子!”西门庆道:“怪奴才,听唱罢么,我那里晓得什么。单管胡枝扯叶的。”只见两个小优又唱到:“一个相府内怀春女,忽剌八抛去也。我怎肯恁随邪,又去把墙花乱折!”那西门庆只顾低着头留心细听。须臾唱毕,这潘金莲就不愤他,两个在席上只顾拌嘴起来。月娘有些看不上,便道:“六姐,你也耐烦,两个只顾强什么?杨姑奶奶和他大妗子丢在屋里,冷清清的,没个人儿陪他,你每着两个进去陪他坐坐儿,我就来。”当下金莲和李娇儿就往房里去了。

不一时,只见来安来说:“应二娘帖儿送到了。二爹来了,大舅便来。”西门庆道:“你对过请温师父来坐坐。”因对月娘说:“你吩咐厨下拿菜出来,我前边陪他坐去。”又叫李铭:“你往前边唱罢。”李铭即跟着西门庆出来,到西厢房内陪伯爵坐的。又谢他人情:“明日请令正好歹来走走。”伯爵道:“他怕不得来,家下没人。”良久,温秀才到,作揖坐下。伯爵举手道:“早晨多有累老先生。”温秀才道:“岂敢。”吴大舅也到了,相见让位毕,一面琴童儿秉烛来,四人围暖炉坐定。来安拿春盛案酒摆在桌上。伯爵灯下看见西门庆白绫袄子上,罩着青缎五彩飞鱼蟒衣,张牙舞爪,头角峥嵘,扬须鼓鬣,金碧掩映,蟠在身上,唬了一跳,问:“哥,这衣服是那里的?”西门庆便立起身来,笑道:“你每瞧瞧,猜是那里的?”伯爵道:“俺每如何猜得着。”西门庆道:“此是东京何太监送我的。我在他家吃酒,因害冷,他拿出这件衣服与我披。这是飞鱼,因朝廷另赐了他蟒龙玉带,他不穿这件,就送我了。此是一个大分上。”伯爵极口夸道:“这花衣服,少说也值几个钱儿。此是哥的先兆,到明日高转做到都督上,愁没玉带蟒衣?何况飞鱼!只怕穿过界儿去哩!”说着,琴童安放钟箸,拿酒上来。李铭在面前弹唱。伯爵道:“也该进去与三嫂递杯酒儿才好,如何就吃酒?”西门庆道:“我儿,你既有孝顺之心,往后边与三嫂磕个头儿就是了,说他怎的?”伯爵道:“磕头到不打紧,只怕惹人议论我做大不尊,到不如你替我磕个儿罢。”被西门庆向他头上打了一下,骂道:“你这狗才,单管恁没大小!”伯爵道:“有大小到不教孩儿们打了。”两个戏说了一回,琴童拿将寿面来,西门庆让他三人吃。自己因在后边吃了,就递与李铭吃。那李铭吃了,又上来弹唱。伯爵叫吴大舅:“吩咐曲儿叫他唱。”大舅道:“不要索落他,随他拣熟的唱罢。”西门庆道:“大舅好听《瓦盆儿》这一套。”一面令琴童斟上酒,李铭于是筝排雁柱,款定冰弦,唱了一套“叫人对景无言,终日减芳容”,下边去了。只见来安上来禀说:“厨子家去,请问爹,明日叫几名答应?”西门庆吩咐:“六名厨役、二名茶酒,酒筵共五桌,俱要齐备。”来安应诺去了。吴大舅便问:“姐夫明日请甚么人?”西门庆悉把安郎中作东请蔡九知府说了。吴大舅道:“既明日大巡在姐夫这里吃酒,又好了。”西门庆道:“怎的说?”吴大舅道:“还是我修仓的事,要在大巡手里题本,望姐夫明日说说,教他青目青目,到年终考满之时保举一二,就是姐夫情分。”西门庆道:“这不打紧。大舅明日写个履历揭帖来,等我取便和他说。”大舅连忙下来打恭。伯爵道:“老舅,你老人家放心,你是个都根主子,不替你老人家说,再替谁说?管情消不得吹嘘之力,一箭就上垛。”前边吃酒到二更时分散了,西门庆打发李铭等出门,就吩咐:“明日俱早来伺候。”李铭等应诺去了。小厮收进家伙,上房内挤着一屋里人,听见前边散了,都往那房里去了。

却说金莲,只说往他屋里去,慌的往外走不迭。不想西门庆进仪门来了,他便藏在影壁边黑影儿里,看着西门庆进入上房,悄悄走来窗下听觑。只见玉箫站在堂屋门首,说道:“五娘怎的不进去?”又问:“姥姥怎的不见?”金莲道:“老行货子,他害身上疼,往房里睡去了。”良久,只听月娘问道:“你今日怎的叫恁两个新小王八子?唱又不会唱,只一味‘三弄梅花’。”玉楼道:“只你临了教他唱‘鸳鸯浦莲开’,他才依了你唱。好两个猾小王八子,不知叫什么名字,一日在这里只是顽。”西门庆道:“一个叫韩佐,一个叫邵谦。”月娘道:“谁晓的他叫什么谦儿李儿!”不防金莲蹑足潜踪进去,立在暖炕儿背后,忽说道:“你问他?正经姐姐吩咐的曲儿不叫他唱,平白胡枝扯叶的教他唱什么‘忆吹箫’,支使的小王八子乱腾腾的,不知依那个的是。”玉楼“哕”了一声,扭回头看见是金莲,便道:“这个六丫头,你在那里来?猛可说出话来,倒唬我一跳。单爱行鬼路儿。你从多咱走在我背后?”小玉道:“五娘在三娘背后,好少一回儿。”金莲点着头儿向西门庆道:“哥儿,你脓着些儿罢了。你那小见识儿,只说人不知道。他是甚‘相府中怀春女’?他和我都是一般的后婚老婆。什么他为你‘褪湘裙杜鹃花上血’,三个官唱两个喏,谁见来?孙小官儿问朱吉,别的都罢了,这个我不敢许。可是你对人说的,自从他死了,好应心的菜儿也没一碟子儿。没了王屠,连毛吃猪!你日逐只噇屎哩?俺们便不是上数的,可不着你那心罢了。一个大姐姐这般当家立纪,也扶持不过你来,可可儿只是他好。他死,你怎的不拉住他?当初没他来时,你怎的过来?如今就是诸般儿称不上你的心了。题起他来,就疼的你这心里格地地的!拿别人当他,借汁儿下面,也喜欢的你要不的。只他那屋里水好吃么?”月娘道:“好六姐,常言道:好人不长寿,祸害一千年。自古镟的不圆砍的圆。你我本等是迟货,应不上他的心,随他说去罢了。”金莲道:“不是咱不说他,他说出来的话灰人的心。只说人愤不过他。”那西门庆只是笑,骂道:“怪小淫妇儿,胡说了你,我在那里说这个话来?”金莲道:“还是请黄内官那日,你没对着应二和温蛮子说?怪不的你老婆都死绝了,就是当初有他在,也不怎么的。到明日再扶一个起来,和他做对儿就是了。贼没廉耻撒根基的货!”说的西门庆急了,跳起来,赶着拿靴脚踢他,那妇人夺门一溜烟跑了。

这西门庆赶出去不见他,只见春梅站在上房门首,就一手搭伏春梅肩背往前边来。月娘见他醉了,巴不的打发他前边去睡,要听三个姑子宣卷。于是教小玉打个灯笼,送他前边去。金莲和玉箫站在穿廊下黑影中,西门庆没看见,迳走过去。玉箫向金莲道:“我猜爹管情向娘屋里去了。”金莲道:“他醉了,快发讪,由他先睡,等我慢慢进去。”这玉箫便道:“娘,你等等,我取些果子儿捎与姥姥吃去。”于是走到床房内,拿些果子递与妇人,妇人接的袖了,一直走到他前边。只见小玉送了回来,说道:“五娘在那边来?爹好不寻五娘。”

金莲到房门首,不进去,悄悄向窗眼望里张觑,看见西门庆坐在床上,正搂着春梅做一处顽耍。恐怕搅扰他,连忙走到那边屋里,将果子交付秋菊。因问:“姥姥睡没有?”秋菊道:“睡了一大回了。”金莲嘱咐他:“果子好生收在拣妆内。”又复往后边来。只见月娘、李娇儿、孟玉楼、西门大姐、大妗子、杨姑娘,并三个姑子带两个小姑子,坐了一屋里人。薛姑子便盘膝坐在月娘炕上,当中放着一张炕桌儿,炷了香,众人都围着他,听他说佛法。只见金莲笑掀帘子进来,月娘道:“你惹下祸来,他往屋里寻你去了。你不打发他睡,如何又来了?我还愁他到屋里要打你。”金莲笑道:“你问他敢打我不敢?”月娘道:“你头里话出来的忒紧了,他有酒的人,一时激得恼了,不打你打狗不成?俺每倒替你捏两把汗,原来你到这等泼皮。”金莲道:“他就恼,我也不怕他,看不上那三等儿九做的。正经姐姐吩咐的曲儿不教唱,且东沟犁西沟耙,唱他的心事。就是今日孟三姐的好日子,也不该唱这离别之词。人也不知死到那里去了,偏有那些佯慈悲假孝顺,我是看不上。”大妗子道:“你姐妹每乱了这一回,我还不知因为什么来。姑夫好好的进来坐着,怎的又出去了?”月娘道:“大妗子,你还不知道,那一个因想起李大姐来,说年时孟三姐生日还有他,今年就没他,落了几点眼泪,教小优儿唱了一套‘忆吹箫,玉人儿何处也’。这一个就不愤他唱这词,刚才抢白了他爹几句。抢白的那个急了,赶着踢打,这贼就走了。”杨姑娘道:“我的姐姐,你随官人教他唱罢了,又抢白他怎的?想必每常见姐姐每都全全儿的,今日只不见了李家姐姐,汉子的心怎么不惨切个儿。”孟玉楼道:“好奶奶,若是我每,谁嗔他唱!俺这六姐姐平昔晓的曲子里滋味,见那个夸死了的李大姐,比古人那个不如他,又怎的两个相交情厚,又怎么山盟海誓,你为我,我为你。这个牢成的又不服气,只顾拿言语抢白他,整厮乱了这半日。”杨姑娘道:“我的姐姐,原来这等聪明!”月娘道:“他什么曲儿不知道!但题起头儿,就知尾儿。象我每叫唱老婆和小优儿来,只晓的唱出来就罢了。偏他又说那一段儿唱的不是了,那一句儿唱的差了,又那一节儿稍了。但是他爹说出个曲儿来,就和他白搽白乱,必须搽恼了才罢。”孟玉楼在旁边戏道:“姑奶奶你不知,我三四胎儿只存了这个丫头子,这般精灵古怪的。”金莲笑向他打了一下,说道:“我到替你争气,你到没规矩起来了。”杨姑娘道:“姐姐,你今后让官人一句儿罢。常言:一夜夫妻百夜恩,相随百步也有个徘徊之意。一个热突突人儿,指头儿似的少了一个,有个不想不疼不题念的?”金莲道:“想怎不想,也有个常时儿。一般都是你的老婆,做什么抬一个灭一个?只嗔俺们不替他戴孝,他又不是婆婆,胡乱戴过断七罢了,只顾戴几时?”杨姑娘道:“姐姐每见一半不见一半儿罢。”大妗子道:“好快!断七过了,这一向又早百日来了。”杨姑娘问:“几时是百日?”月娘道:“早哩,腊月二十六日。”王姑子道:“少不的念个经儿。”月娘道:“挨年近节,念什么经!他爹只好过年念罢了。”说着,只见小玉拿上一道茶来,每人一盏。

须臾吃毕。月娘洗手,向炉中炷了香,听薛姑子讲说佛法。薛姑子就先宣念偈言,讲了一段五戒禅师破戒戏红莲女子,转世为东坡佛印的佛法。讲说了良久方罢。只见玉楼房中兰香,拿了两方盒细巧素菜果碟、茶食点心来,收了香炉,摆在桌上。又是一壶茶,与众人陪三个师父吃了。然后又拿荤下饭来,打开一坛麻姑酒,众人围炉吃酒。月娘便与大妗子掷色抢红。金莲便与李娇儿猜枚,玉箫在旁边斟酒,便替金莲打桌底下转子儿。须臾把李娇儿赢了数杯。玉楼道:“等我和你猜,你只顾赢他罢。”却要金莲拿出手来,不许褪在袖子里,又不许玉箫近前。一连反赢了金莲几大钟。

金莲坐不住,去了。到前边叫了半日,角门才开,只见秋菊揉眼。妇人骂道:“贼奴才,你睡来?”秋菊道:“我没睡。”妇人道:“见睡起来,你哄我。你到自在,就不说往后来接我接儿去。”因问:“你爹睡了?”秋菊道:“爹睡了这一日了。”妇人走到炕房里,搂起裙子来就在炕上烤火。妇人要茶吃,秋菊连忙倾了一盏茶来。妇人道:“贼奴才,好干净手儿,我不吃这陈茶,熬的怪泛汤气。你叫春梅来,叫他另拿小铫儿顿些好甜水茶儿,多着些茶叶,顿的苦艳艳我吃。”秋菊道:“他在那边床房里睡哩,等我叫他来。”妇人道:“你休叫他,且教他睡罢。”这秋菊不依,走在那边屋里,见春梅\textuni{22C49}在西门庆脚头睡得正好。被他摇推醒了,道:“娘来了,要吃茶,你还不起来哩。”这春梅哕他一口,骂道:“见鬼的奴才,娘来了罢了,平白唬人剌剌的!”一面起来,慢条厮礼、撒腰拉裤走来见妇人,只顾倚着炕儿揉眼。妇人反骂秋菊:“恁奴才,你睡的甜甜儿的,把你叫醒了。”因叫他:“你头上汗巾子跳上去了,还不往下扯扯哩。”又问:“你耳朵上坠子怎的只戴着一只?”这春梅摸了摸,果然只有一只。便点灯往那边床上寻去,寻不见。良久,不想落在那脚踏板上,拾起来。妇人问:“在那里来?”春梅道:“都是他失惊打怪叫我起来,吃帐钩子抓下来了,才在踏板上拾起来。”妇人道:“我那等说着,他还只当叫起你来。”春梅道:“他说娘要茶吃来。”妇人道:“我要吃口茶儿,嫌他那手不干净。”这春梅连忙舀了一小铫子水,坐在火上,使他挝了些炭在火内,须臾就是茶汤。涤盏干净,浓浓的点上去,递与妇人。妇人问春梅:“你爹睡下多大回了?”春梅道:“我打发睡了这一日了。问娘来,我说娘在后边还未来哩。”

这妇人吃了茶,因问春梅:“我头里袖了几个果子和蜜饯,是玉箫与你姥姥吃的,交付这奴才接进来,你收了?”春梅道:“我没见,他知道放在那里?”妇人叫秋菊,问他果子在那里,秋菊道:“我放在拣妆内哩。”走去取来,妇人数了数儿,少了一个柑子,问他那里去了。秋菊道:“我拿进来就放在拣妆内,那个害馋痨、烂了口吃他不成!”妇人道:“贼奴才,还涨漒嘴!你不偷,那去了?我亲手数了交与你的,怎就少了一个?原来只孝顺了你!”教春梅:“你与我把那奴才一边脸上打与他十个嘴巴子。”春梅道:“那臜脸蛋子,倒没的龌龊了我的手。”妇人道:“你与我拉过他来。”春梅用双手推颡到妇人跟前。妇人用手拧着他腮颊,骂道:“贼奴才,这个柑子是你偷吃了不是?你实实说了,我就不打你。不然,取马鞭子来,我这一旋剥就打个不数。我难道醉了?你偷吃了,一径里鬼混我。”因问春梅:“我醉不醉?”那春梅道:“娘清省白醒,那讨酒来?娘不信只掏他袖子,怕不的还有柑子皮儿在袖子里哩。”妇人于是扯过他袖子来,用手去掏,秋菊慌用手撇着不教掏。春梅一面拉起手来,果然掏出些柑子皮儿来。被妇人尽力脸上拧了两把,打了两下嘴巴,骂道:“贼奴才,你诸般儿不会,象这说舌偷嘴吃偏会。真赃实犯拿住,你还赖那个?我如今茶前酒后且不打你,到明日清省白醒,和你算帐。”春梅道:“娘到明日,休要与他行行忽忽的,好生旋剥了,叫个人把他实辣辣打与他几十板子,叫他忍疼也惧怕些。甚么逗猴儿似汤那几棍儿,他才不放在心上!”那秋菊被妇人拧得脸胀肿的,谷都着嘴往厨下去了。妇人把那一个柑子平分两半,又拿了个苹婆石榴,递与春梅,说道:“这个与你吃,把那个留与姥姥吃。”这春梅也不瞧,接过来似有如无,掠在抽屉内。妇人把蜜饯也要分开,春梅道:“娘不要分,我懒得吃这甜行货子,留与姥姥吃罢。”以此妇人不分,都留下了。

妇人走到桶子上小解了,叫春梅掇进坐桶来,澡了牝,又问春梅:“这咱天有多时分了?”春梅道:“睡了这半日,也有三更了。”妇人摘了头面,走来那边床房里,见桌上银灯已残,从新剔了剔,向床上看西门庆正打鼾睡。于是解松罗带,卸褪湘裙,上床钻入被窝里,与西门庆并枕而卧。

睡下不多时,向他腰间摸他那话。弄了一回,白不起。原来西门庆与春梅才行房不久,那话绵软,急切捏弄不起来。这妇人酒在腹中,欲情如火,蹲身在被底,把那话用口吮咂。挑弄蛙口,吞裹龟头,只顾往来不绝。西门庆猛然醒了,便道:“怪小淫妇儿,如何这咱才来?”妇人道:“俺每在后边吃酒,孟三儿又安排了两大方盒酒菜,郁大姐唱着,俺每猜枚掷骰儿,又顽了这一日,被我把李娇儿赢醉了。落后孟三儿和我五子三猜,俺到输了好几钟酒。你到是便宜,睡这一觉儿来好熬我,你看我依你不依?”西门庆道:“你整治那带子有了?”妇人道:“在褥子底下不是?”一面探手取出来,与西门庆看了,替他扎在麈柄根下,系在腰间,拴的紧紧的。又问:“你吃了不曾?”西门庆道:“我吃了。”须臾,那话吃妇人一壁厢弄起来,只见奢棱跳脑,挺身直舒,比寻常更舒半寸有余。妇人爬在身上,龟头昂大,两手扇着牝户往里放。须臾突入牝中,妇人两手搂定西门庆脖项,令西门庆亦扳抱其腰,在上只顾揉搓,那话渐没至根。妇人叫西门庆:“达达,你取我的柱腰子垫在你腰底下。”这西门庆便向床头取过他大红绫抹胸儿,四折叠起垫着腰,妇人在他身上马伏着,那消几揉,那话尽入。妇人道:“达达,你把手摸摸,都全放进去了,撑的里头满满儿的。你自在不自在?”西门庆用手摸摸,见尽没至根,间不容发,止剩二卵在外,心中觉翕翕然畅美不可言。妇人道:“好急的慌,只是寒冷,咱不得拿灯儿照着干,赶不上夏天好。”因问西门庆,说道:“这带子比那银托子好不好?又不格的阴门生痛的,又长出许多来。你不信,摸摸我小肚子,七八顶到奴心。”又道:“你搂着我,等我一发在你身上睡一觉。”西门庆道:“我的儿,你睡,达达搂着。”那妇人把舌头放在他口里含着,一面朦胧星眼,款抱香肩。睡不多时,怎禁那欲火烧身,芳心撩乱,于是两手按着他肩膊,一举一坐,抽彻至首,复送至根,叫:“亲心肝,罢了,六儿的心了。”往来抽卷,又三百回。比及精泄,妇人口中只叫:“我的亲达达,把腰扱紧了。”一面把奶头教西门庆咂,不觉一阵昏迷,淫水溢下,妇人心头小鹿突突的跳。登时四肢困软,香云撩乱。那话拽出来犹刚劲如故,妇人用帕搽之,说道:“我的达达,你不过却怎么的?”西门庆道:“等睡起一觉来再耍罢。”妇人道:“我的身子已软瘫热化的。”当下云收雨散,两个并肩交股,相与枕籍于床上,不知东方之既白。正是:

\[
等闲试把银缸照,一对天生连理人。
\]

\newpage
%# -*- coding:utf-8 -*-
%%%%%%%%%%%%%%%%%%%%%%%%%%%%%%%%%%%%%%%%%%%%%%%%%%%%%%%%%%%%%%%%%%%%%%%%%%%%%%%%%%%%%


\chapter{潘金莲香腮偎玉\KG 薛姑子佛口谈经}


诗曰:

\[
富贵如朝露,交游似聚沙。不如竹窗里,对卷自趺跏。
静虑同聆偈,清神旋煮茶。惟忧晓鸡唱,尘里事如麻。
\]

话说西门庆搂抱潘金莲,一觉睡到天明。妇人见他那话还直竖一条棍相似,便道:“达达,你饶了我罢,我来不得了。待我替你咂咂罢。”西门庆道:“怪小淫妇儿,你若咂的过了,是你造化。”这妇人真个蹲向他腰间,按着他一只腿,用口替他吮弄那话。吮够一个时分,精还不过,这西门庆用手按着粉项,往来只顾没棱露脑摇撼,那话在口里吞吐不绝。抽拽的妇人口边白沫横流,残脂在茎。妇人一面问西门庆:“二十八日应二家请俺每,去不去?”西门庆道:“怎的不去!”妇人道:“我有桩事儿央你,依不依?”西门庆道:“怪小淫妇儿,你有甚事,说不是。”妇人道:“你把李大姐那皮袄拿出来与我穿了罢。明日吃了酒回来,他们都穿着皮袄,只奴没件儿穿。”西门庆道:“有王招宣府当的皮袄,你穿就是了。”妇人道:“当的我不穿他,你与了李娇儿去。把李娇儿那皮袄却与雪娥穿。你把李大姐那皮袄与了我,等我\textuni{3A5F}上两个大红遍地金鹤袖,衬着白绫袄儿穿,也是与你做老婆一场,没曾与了别人。”西门庆道:“贼小淫妇儿,单管爱小便宜儿。他那件皮袄值六十两银子哩,你穿在身上是会摇摆!”妇人道:“怪奴才,你与了张三、李四的老婆穿了?左右是你的老婆,替你装门面,没的有这些声儿气儿的。好不好我就不依了。”西门庆道:“你又求人又做硬儿。”妇人道:“怪硶货,我是你房里丫头,在你跟前服软?”一面说着,把那话放在粉脸上只顾偎晃,良久,又吞在口里挑弄蛙口,一回又用舌尖抵其琴弦,搅其龟棱,然后将朱唇裹着,只顾动动的。西门庆灵犀灌顶,满腔春意透脑,良久精来,呼:“小淫妇儿,好生裹紧着,我待过也!”言未绝,其精邈了妇人一口。妇人口口接着,都咽了。正是:

\[
自有内事迎郎意,殷勤爱把紫箫吹。
\]

当日是安郎中摆酒,西门庆起来梳头净面出门。妇人还睡在被里,便说道:“你趁闲寻寻儿出来罢。等住回,你又不得闲了。”这西门庆于是走到李瓶儿房中,奶子、丫头又早起来顿下茶水供养。西门庆见如意儿薄施脂粉,长画蛾眉,笑嘻嘻递了茶,在旁边说话儿。西门庆一面使迎春往后边讨床房里钥匙去,如意儿便问:“爹讨来做甚么?”西门庆道:“我要寻皮袄与你五娘穿。”如意道:“是娘的那貂鼠皮袄?”西门庆道:“就是。他要穿穿,拿与他罢。”迎春去了,就把老婆搂在怀里,摸他奶头,说道:“我儿,你虽然生了孩子,奶头儿到还恁紧。”就两个脸对脸儿亲嘴咂舌头做一处。如意儿道:“我见爹常在五娘身边,没见爹往别的房里去。他老人家别的罢了,只是心多容不的人。前日爹不在,为个棒槌,好不和我大嚷了一场。多亏韩嫂儿和三娘来劝开了。落后爹来家,也没敢和爹说。不知甚么多嘴的人对他说,说爹要了我。他也告爹来不曾?”西门庆道:“他也告我来,你到明日替他陪个礼儿便了。他是恁行货子,受不的人个甜枣儿就喜欢的。嘴头子虽利害,到也没什么心。”如意儿道:“前日我和他嚷了,第二日爹到家,就和我说好活。说爹在他身边偏多,‘就是别的娘都让我几分,你凡事只有个不瞒我,我放着河水不洗船?’”西门庆道:“既是如此,大家取和些。”又许下老婆:“你每晚夕等我来这房里睡。”如意道:“爹真个来?休哄俺每!”西门庆道:“谁哄你来!”正说着,只见迎春取钥匙来。西门庆教开了床房门,又开橱柜,拿出那皮祆来抖了抖,还用包袱包了,教迎春拿到那边房里去。如意儿就悄悄向西门庆说:“我没件好裙袄儿,爹趁着手儿再寻件儿与了我罢。有娘小衣裳儿,再与我一件儿。”西门庆连忙又寻出一套翠盖缎子袄儿、黄绵绸裙子,又是一件蓝潞绸绵裤儿,又是一双妆花膝裤腿儿,与了他。老婆磕头谢了。西门庆锁上门,就使他送皮袄与金莲房里来。

金莲才起来,在床上裹脚,只见春梅说:“如意儿送皮袄来了。”妇人便知其意,说道:“你教他进来。”问道:“爹使你来?”如意道:“是爹教我送来与娘穿。”金莲道:“也与了你些什么儿没有?”如意道:“爹赏了我两件绸绢衣裳年下穿。叫我来与娘磕头。”于是向前磕了四个头。妇人道:“姐姐每这般却不好?你主子既爱你,常言:船多不碍港,车多不碍路,那好做恶人?你只不犯着我,我管你怎的?我这里还多着个影儿哩!”如意儿道:“俺娘已是没了,虽是后边大娘承揽,娘在前边还是主儿,早晚望娘抬举。小媳妇敢欺心!那里是叶落归根之处?”妇人道:“你这衣服少不得还对你大娘说声。”如意道:“小的前者也问大娘讨来,大娘说:‘等爹开时,拿两件与你。’”妇人道:“既说知罢了。”这如意就出来,还到那边房里,西门庆已往前厅去了。如意便问迎春:“你头里取钥匙去,大娘怎的说?”迎春说:“大娘问:‘你爹要钥匙做什么?’我也没说拿皮袄与五娘,只说我不知道。大娘没言语。”

却说西门庆走到厅上看设席,海盐子弟张美、徐顺、苟子孝都挑戏箱到了,李铭等四名小优儿又早来伺候,都磕头见了。西门庆吩咐打发饭与众人吃,吩咐李铭三个在前边唱,左顺后边答应堂客。那日韩道国娘子王六儿没来,打发申二姐买了两盒礼物,坐轿子,他家进财儿跟着,也来与玉楼做生日。王经送到后边,打发轿子出去了。不一时,门外韩大姨、孟大妗子都到了,又是傅伙计、甘伙计娘子、崔本媳妇儿段大姐并贲四娘子。西门庆正在厅上,看见夹道内玳安领着一个五短身子,穿绿缎袄儿、红裙子,不搽胭粉,两个密缝眼儿,一似郑爱香模样,便问是谁。玳安道:“是贲四嫂。”西门庆就没言语。往后见了月娘。月娘摆茶,西门庆进来吃粥,递与月娘钥匙。月娘道:“你开门做什么?”西门庆道:“潘六儿他说,明日往应二哥家吃酒没皮袄,要李大姐那皮袄穿。”被月娘瞅了一眼,说道:“你自家把不住自家嘴头了。他死了,嗔人分散他房里丫头,象你这等,就没的话儿说了。他见放皮袄不穿,巴巴儿只要这皮袄穿。——早时他死了,他不死,你只好看一眼儿罢了。”几句说的西门庆闭口无言。忽报刘学官来还银子,西门庆出去陪坐,在厅上说话。只见玳安拿进帖儿说:“王招宣府送礼来了。”西门庆问:“是什么礼?”玳安道:“是贺礼:一匹尺头、一坛南酒、四样下饭。”西门庆即叫王经拿眷生回帖儿谢了,赏了来人五钱银子,打发去了。

只见李桂姐门首下轿,保儿挑四盒礼物。慌的玳安替他抱毡包,说道:“桂姨,打夹道内进去罢,厅上有刘学官坐着哩。”那桂姐即向夹道内进去,来安儿把盒子挑进月娘房里。月娘道:“爹看见不曾?”玳安道:“爹陪着客,还不见哩。”月娘便说道:“且连盒放在明间内着。”一回客去了,西门庆进来吃饭,月娘道:“李桂姐送礼在这里。”西门庆道:“我不知道。”月娘令小玉揭开盒儿,见一盒果馅寿糕、一盒玫瑰糖糕、两只烧鸭、一副豕蹄。只见桂姐从房内出来,满头珠翠,穿着大红对衿袄儿,蓝缎裙子,望着西门庆磕了四个头。西门庆道:“罢了,又买这礼来做什么?”月娘道:“刚才桂姐对我说,怕你恼他。不干他事,说起来都是他妈的不是:那日桂姐害头疼来,只见这王三官领着一行人,往秦玉芝儿家去,打门首过,进来吃茶,就被人惊散了。桂姐也没出来见他。”西门庆道:“那一遭儿没出来见他,这一遭儿又没出来见他,自家也说不过。论起来,我也难管你。这丽春院拿烧饼砌着门不成?到处银钱儿都是一样,我也不恼。”那桂姐跪在地下只顾不起来,说道:“爹恼的是。我若和他沾沾身子,就烂化了,一个毛孔儿里生一个天疱疮。都是俺妈,空老了一片皮,干的营生没个主意。好的也招惹,歹的也招惹,平白叫爹惹恼。”月娘道:“你既来说开就是了,又恼怎的?”西门庆道:“你起来,我不恼你便了。”那桂姐故作娇态,说道:“爹笑一笑儿我才起来。你不笑,我就跪一年也不起来。”潘金莲在旁插口道:“桂姐你起来,只顾跪着他,求告他黄米头儿,叫他张致!如今在这里你便跪着他,明日到你家他却跪着你,——你那时却别要理他。”把西门庆、月娘都笑了,桂姐才起来了。只见玳安慌慌张张来报:“宋老爹、安老爹来了。”西门庆便拿衣服穿了,出去迎接。桂姐向月娘说道:“耶嚛嚛,从今后我也不要爹了,只与娘做女儿罢。”月娘道:“你的虚头愿心,说过道过罢了。前日两遭往里头去,没在那里?”桂姐道:“天么,天么,可是杀人!爹何曾往我家里?若是到我家里,见爹一面,沾沾身子儿,就促死了!娘你错打听了,敢不是我那里,是往郑月儿家走了两遭,请了他家小粉头子了。我这篇是非,就是他气不愤架的。不然,爹如何恼我?”金莲道:“各人衣饭,他平白怎么架你是非?”桂姐道:“五娘,你不知,俺们里边人,一个气不愤一个,好不生分!”月娘接过来道:“你每里边与外边差甚么?也是一般,一个不愤一个。那一个有些时道儿,就要躧下去。”月娘摆茶与他吃,不在话下。

却说西门庆迎接宋御史、安郎中,到厅上叙礼。每人一匹缎子、一部书,奉贺西门庆。见了桌席齐整,甚是称谢不尽。一面分宾主坐下,吃了茶,宋御史道:“学生有一事奉渎四泉:今有巡抚侯石泉老先生,新升太常卿,学生同两司作东,三十日敢借尊府置杯酒奉饯,初二日就起行上京去了。未审四泉允否?”西门庆道:“老先生吩咐,敢不从命!但未知多少桌席?”宋御史道:“学生有分资在此。”即唤书吏取出布、按两司连他共十二两分资来,要一张大插桌、六张散桌,叫一起戏子。西门庆答应收了,就请去卷棚坐的。不一时,钱主事也到了。三员官会在一处下棋。宋御史见西门庆堂庑宽广,院字幽深,书画文物极一时之盛。又见屏风前安着一座八仙捧寿的流金鼎,约数尺高,甚是做得奇巧。炉内焚着沉檀香,烟从龟鹤鹿口中吐出。只顾近前观看,夸奖不已。问西门庆:“这副炉鼎造得好!”因向二官说:“我学生写书与淮安刘年兄那里,央他替我捎带一副来,送蔡老先,还不见到。四泉不知是那里得来的?”西门庆道:“也是淮上一个人送学生的。”说毕下棋。西门庆吩咐下边,看了两个桌盒细巧菜蔬果馅点心上来,一面叫生旦在上唱南曲。宋御史道:“客尚未到,主人先吃得面红,说不通。”安郎中道:“天寒,饮一杯无碍。”宋御史又差人去邀,差人禀道:“邀了,在砖厂黄老爹那里下棋,便来也。”一面下棋饮酒,安郎中唤戏子:“你们唱个《宜春令》奉酒。”于是生旦合声唱一套“第一来为压惊”。

唱未毕,忽吏进报:“蔡老爹和黄老爹来了。”宋御史忙令收了桌席,各整衣冠出来迎接。蔡九知府穿素服金带,先令人投一“侍生蔡修”拜帖与西门庆。进厅上,安郎中道:“此是主人西门大人,见在本处作千兵,也是京中老先生门下。”那蔡知府又是作揖称道:“久仰,久仰。”西门庆道:“容当奉拜。”叙礼毕,各宽衣服坐下。左右上了茶,各人扳话。良久,就上坐。蔡九知府居上,主位四坐。厨役割道汤饭,戏子呈递手本,蔡九知府拣了《双忠记》,演了两折。酒过数巡,小优儿席前唱一套《新水令》“玉鞭骄马出皇都”。蔡知府笑道:“松原直得多少,可谓‘御史青骢马’,三公乃‘刘郎旧萦髯’。”安郎中道:“今日更不道‘江州司马青衫湿’。”言罢,众人都笑了。西门庆又令春鸿唱了一套“金门献罢平胡表”,把宋御史喜欢的要不的,因向西门庆道:“此子可爱。”西门庆道:“此是小价,原是扬州人。”宋御史携着他手儿,教他递酒,赏了他三钱银子,磕头谢了。正是:

\[
窗外日光弹指过,席前花影坐间移。
一杯未尽笙歌送,阶下申牌又报时。
\]

不觉日色沉西,蔡九知府见天色晚了,即令左右穿衣告辞。众位款留不住,俱送出大门而去。随即差了两名吏典,把桌席羊酒尺头抬送到新河口去讫。宋御史亦作辞西门庆,因说道:“今日且不谢,后日还要取扰。”各上轿而去。

西门庆送了回来,打发戏子,吩咐:“后日还是你们来,再唱一日。叫几个会唱的来,宋老爹请巡抚侯爷哩。”戏子道:“小的知道了。”西门庆令攒上酒桌,使玳安:“去请温师父来坐坐。”再叫来安儿:“去请应二爹去。”不一时,次第而至,各行礼坐下。三个小优儿在旁弹唱,把酒来斟。西门庆问伯爵:“你娘们明日都去,你叫唱的是杂耍的?”伯爵道:“哥到说得好,小人家那里抬放?将就叫两个唱女儿唱罢了。明日早些请众位嫂子下降。”这里前厅吃酒不题。

后边,孟大姨与盂三妗子先起身去了。落后杨姑娘也要去,月娘道:“姑奶奶你再住一日儿不是,薛师父使他徒弟取了卷来,咱晚夕叫他宣卷咱们听。”杨姑娘道:“老身实和姐姐说,要不是我也住,明日俺第二个侄儿定亲事,使孩子来请我,我要瞧瞧去。”于是作辞而去。众人吃到掌灯以后,三位伙计娘子也都作辞去了,止留下段大姐没去,潘姥姥也往金莲房内去了。只有大吟子、李桂姐、申二姐和三个姑子,郁大姐和李娇儿、孟玉楼、潘金莲,在月娘房内坐的。忽听前边散了,小厮收下家伙来。这金莲忙抽身就往前走,到前边悄悄立在角门首。只见西门庆扶着来安儿,打着灯,趔趄着脚儿就要往李瓶儿那边走,看见金莲在门首立着,拉了手进入房来。那来安儿便往上房交钟箸。

月娘只说西门庆进来,把申二姐、李桂姐、郁大姐都打发往李娇儿房内去了。问来安道:“你爹来没有?”来安道:“爹在五娘房里,不耐烦了。”月娘听了,心内就有些恼,因向玉楼道:“你看恁没来头的行货子,我说他今日进来往你房里去,如何三不知又摸到他屋里去了?这两日又浪风发起来,只在他前边缠。”玉楼道:“姐姐,随他缠去!这等说,恰似咱每争他的一般。可是大师父说的笑话儿,左右这六房里,由他串到。他爹心中所欲,你我管的他!”月娘道:“干净他有了话!刚才听见前头散了,就慌的奔命往前走了。”因问小玉:“灶上没人,与我把仪门拴上。后边请三位师父来,咱每且听他宣一回卷着。”又把李桂姐、申二姐、段大姐、郁大姐都请了来。月娘向大妗子道:“我头里旋叫他使小沙弥请了《黄氏女卷》来宣,今日可可儿杨姑娘又去了。”吩咐玉箫顿下好茶。玉楼对李娇儿说:“咱两家轮替管茶,休要只顾累大姐姐。”于是各房里吩咐预备茶去。

不一时,放下炕桌儿,三个姑子来到,盘膝坐在炕上。众人俱各坐了,听他宣卷。月娘洗手炷了香,这薛姑子展开《黄氏女卷》,高声演说道:

\[
盖闻法初不灭,故归空。道本无生,每因生而不用。由法身以垂八相,由八相以显法身。朗朗惠灯,通开世户;明明佛镜,照破昏衢。百年景赖刹那间,四大幻身如泡影。每日尘劳碌碌,终朝业试忙忙。岂知一性圆明,徒逞六根贪欲。功名盖世,无非大梦一场;富贵惊人,难免无常二字。风火散时无老少,溪山磨尽几英雄!
\]

演说了一回,又宣念偈子,又唱几个劝善的佛曲儿,方才宣黄氏女怎的出身,怎的看经好善,又怎的死去转世为男子,又怎的男女五人一时升天。

慢慢宣完,已有二更天气。先是李娇儿房内元宵儿拿了一道茶来,众人吃了。落后孟玉楼房中兰香,又拿了几样精制果菜、一大壶酒来,又是一大壶茶来,与大妗子、段大姐、桂姐众人吃。月娘又教玉箫拿出四盒儿茶食饼糖之类,与三位师父点茶。李桂姐道:“三个师父宣了这一回卷,也该我唱个曲儿孝顺。”月娘道:“桂姐,又起动你唱?”郁大姐道:“等我先唱。”月娘道:“也罢,郁大姐先唱。”申二姐道:“等姐姐唱了,我也唱个儿与娘们听。”桂姐不肯,道:“还是我先唱。”因问月娘要听什么,月娘道:“你唱个‘更深静悄’罢。”当下桂姐送众人酒,取过琵琶来,轻舒玉笋,款跨鲛绡,唱了一套。桂姐唱毕,郁大姐才要接琵琶,早被申二姐要过去了,挂在胳膊上,先说道:“我唱个《十二月儿挂真儿》与大妗子和娘每听罢。”于是唱道:“正月十五闹元宵,满把焚香天地烧……”那时大妗子害夜深困的慌,也没等的申二姐唱完,吃了茶就先往月娘房内睡去了。须臾唱完,桂姐便归李娇儿房内,段大姐便往孟玉楼房内,三位师父便往孙雪娥房里,郁大姐、申二姐就与玉箫、小玉在那边炕屋里睡。月娘同大妗子在上房内睡,俱不在话下。

看官听说:古妇人怀孕,不侧坐,不偃卧,不听淫声,不视邪色,常玩诗书金玉,故生子女端正聪慧,此胎教之法也。今月娘怀孕,不宜令僧尼宣卷,听其死生轮回之说。后来感得一尊古佛出世,投胎夺舍,幻化而去,不得承受家缘。盖可惜哉!正是:

\[
前程黑暗路途险,十二时中自着迷。
\]

\newpage
%# -*- coding:utf-8 -*-
%%%%%%%%%%%%%%%%%%%%%%%%%%%%%%%%%%%%%%%%%%%%%%%%%%%%%%%%%%%%%%%%%%%%%%%%%%%%%%%%%%%%%


\chapter{因抱恙玉姐含酸\KG 为护短金莲泼醋}


诗曰:

\[
双双蛱蝶绕花溪,半是山南半水西。
故园有情风月乱,美人多怨雨云迷。
频开檀口言如织,温托香腮醉如泥。
莫道佳人太命薄,一莺啼罢一莺啼。
\]

话说月娘听宣毕《黄氏宝卷》,各房宿歇不题。单表潘金莲在角门边,撞见西门庆,相携到房中。见西门庆只顾坐在床上,因问:“你怎的不脱衣裳?”那西门庆搂定妇人,笑嘻嘻说道:“我特来对你说声,我要过那边歇一夜儿去。你拿那淫器包儿来与我。”妇人骂道:“贼牢,你在老娘手里使巧儿,拿这面子话儿来哄我!我刚才不在角门首站着,你过去的不耐烦了,又肯来问我?这是你早辰和那歪剌骨商定了腔儿,嗔道头里使他来送皮袄儿,又与我磕了头。小贼歪剌骨,把我当甚么人儿?在我手内弄剌子。我还是李瓶儿时,教你活埋我!雀儿不在那窝儿里,我不醋了!”西门庆笑道:“那里有此勾当,他不来与你磕个头儿,你又说他的不是。”妇人沉吟良久,说道:“我放你去便去,不许你拿了这包子去,与那歪剌骨弄答的龌龌龊龊的,到明日还要来和我睡,好干净儿。”西门庆道:“我使惯了,你不与我却怎样的!”缠了半日,妇人把银托子掠与他,说道:“你要,拿了这个行货子去。”西门庆道:“与我这个也罢。”一面接的袖了,趔趄着脚儿就往外走。妇人道:“你过来,我问你,莫非你与他一铺儿长远睡?惹得那两个丫头也羞耻。无故只是睡那一回儿,还放他另睡去。”西门庆道:“谁和他长远睡?”说毕就走。妇人又叫回来,说道:“你过来,我分付你,慌怎的?”西门庆道:“又说甚么?”妇人道:“我许你和他睡便睡,不许你和他说甚闲话,教他在俺们跟前欺心大胆的。我到明日打听出来,你就休要进我这屋里来,我就把你下截咬下来。”西门庆道:“怪小淫妇儿,琐碎死了。”一直走过那边去了。春梅便向妇人道:“由他去,你管他怎的?婆婆口絮,媳妇耳顽,倒没的教人与你为冤结仇,误了咱娘儿两个下棋。”一面叫秋菊关上角门,放卓儿摆下棋子。两个下棋不题。

且说西门庆走过李瓶儿房内,掀开帘子。如意儿正与迎春、绣春炕上吃饭,见了西门庆,慌的跳起身来。西门庆道:“你们吃饭。”于是走出明间李瓶儿影跟前一张交椅上坐下。不一时,如意儿笑嘻嘻走出来,说道:“爹,这里冷,你往屋里坐去罢。”这西门庆就一把手搂过来,就亲了个嘴。一面走到房中床正面坐了。火炉上顿着茶,迎春连忙点茶来吃了。如意儿在炕边烤着火儿站立,问道:“爹,你今日没酒,还有头里与娘供养的一桌菜儿,一素儿金华酒,留下预备筛来与爹吃。”西门庆道:“下饭你们吃了罢,只拿几个果碟儿来,我不吃金华酒。”一面教绣春:“你打个灯笼,往藏春坞书房内,还有一坛葡萄酒,你问王经要了来,筛与我吃。”绣春应诺,打着灯笼去了。迎春连忙放桌儿,拿菜儿。如意儿道:“姐,你揭开盒子,等我拣两样儿与爹下酒。”于是灯下拣了几碟精味果菜,摆在桌上。良久,绣春取了酒来,打开筛热了。如意儿斟在钟内,递上。西门庆尝了尝,十分精美。如意儿就挨近桌边站立,侍奉斟酒,又亲剥炒栗子儿与他下酒。迎春知局,就往后边厨房内与绣春坐去了。

西门庆见无人在跟前,就叫老婆坐在他膝盖儿上,搂着与他一递一口儿饮酒。一面解开他对襟袄儿,露出他白馥馥酥胸,用手揣摸他奶头,夸道:“我的儿,你达达不爱你别的,只爱你到好白净皮肉儿,与你娘一般样儿,我搂你就如同搂着他一般。”如意儿笑道:“爹,没的说,还是娘的身上白。我见五娘虽好模样儿,皮肤也中中儿的,红白肉色儿,不如后边大娘、三娘到白净。三娘只是多几个麻儿。倒是他雪姑娘生得清秀,又白净。”又道:“我有句话对爹说,迎春姐有件正面戴仙子儿要与我,他要问爹讨娘家常戴的金赤虎,正月里戴,爹与了他罢。”西门庆道:“你没正面戴的,等我叫银匠拿金子另打一件与你,你娘的头面箱儿,你大娘都拿的后边去了,怎好问他要的。”老婆道:“也罢,你还另打一件赤虎与我罢。”一面走下来就磕头谢了。两个吃了半日酒。如意儿道:“爹,你叫姐来也与他一杯酒吃,惹他不恼么?”西门庆便叫迎春,不应。老婆亲到走到厨房内,说道:“姐,爹叫你哩。”迎春一面到跟前。西门庆令如意儿斟了一瓯酒与他,又拣了两箸菜儿放在酒托儿上。那迎春站在旁边,一面吃了。如意道:“你叫绣春姐来也吃些儿。”迎春去了,回来说道:“他不吃了。”就向炕上抱他铺盖,和绣春厨房炕上睡去了。

这老婆陪西门庆吃了一回酒,收拾家火,又点茶与西门庆吃了。原来另预备着一床儿铺盖与西门庆睡,都是绫绢被褥,扣花枕头,在薰笼内薰的暖烘烘的。老婆便问:“爹,你在炕上睡,床上睡?”西门庆道:“我在床上睡罢。”如意儿便将铺盖抱在床上铺下,打发西门庆解衣上床。他又在明间内打水洗了牝,掩上房门,将灯移近床边,方才脱衣裤上床,与西门庆相搂相抱,并枕而卧。妇人用手捏弄他那话儿,上边束着银托子,狰狞跳脑,又喜又怕。两个口吐丁香,交搂在一处。西门庆见他仰卧在被窝内,脱的精赤条条,恐怕冻着他,又取过他的抹胸儿替他盖着胸膛上。两手执其两足,极力抽提。老婆气喘吁吁,被他\textuni{34B2}得面如火热。又道:“这衽腰子还是娘在时与我的。”西门庆道:“我的心肝,不打紧处,到明日铺子里,拿半个红段子,做小衣儿穿在身上伏侍我。”老婆道:“可知好哩。”西门庆道:“我只要忘了,你今年多少年纪?你姓甚么?排行几姐?我只记你男子汉姓熊。”老婆道:“他便姓熊,叫熊旺儿。我娘家姓章,排行第四,今三十二岁。”西门庆道:“我原来还大你一岁。”一壁干首,一面口中呼叫他:“章四儿,你用心伏侍我,等明日后边大娘生了孩子,你好生看奶着。你若有造化,也生长一男半女,我就扶你起来,与我做一房小,就顶你娘的窝儿,你心下何如?”老婆道:“奴男子汉已是没了,娘家又没人,奴情愿一心伏侍爹,就死也不出爹这门。若爹可怜见,可知好哩。”西门庆见他言语儿投着机会,心中越发喜欢,攥着他雪白两只腿儿,只顾没棱探脑,两个扇干,抽提的老婆在下,无不叫出来。娇声怯怯,星眼朦朦。良久,却令他马伏在下,自舒双足,西门庆披着红绫被,骑在他身上,那话插入牝中。灯光下,两手按着他雪白的屁股,只顾扇打,口中叫:“章四儿,你好生叫着亲达达,休要住了,我丢与你罢。”那妇人在下举股相就,真个口中颤声柔语,呼叫不绝,足顽了一个时辰,西门庆方才精泄。良久,拽出麈柄来,老婆取帕儿替他搽拭。搂着睡到五更鸡叫时方醒,老婆又替他吮咂。西门庆告他说:“你五娘怎的替我咂半夜,怕我害冷,连尿也不教我下来溺,都替我咽了。”这西门太真个把胞尿都溺在老婆口内。当下两个旖旎温存,万千罗唣,\textuni{34B2}捣了一夜。

次日,老婆先起来,开了门,预备火盆,打发西门庆穿衣梳洗出门。到前边分付玳安:“教两名排军把卷棚放的流金八仙鼎,写帖儿抬送到宋御史老爹察院内,交付明白,讨回贴来。”又叫陈敬济,封了一匹金段,一匹色段,教琴童用毡包拿着,预备下马,要早往清河口,拜蔡知府去。正在月娘房内吃粥,月娘问他:“应二那里,俺们莫不都去,也留一个儿看家?留下他姐在家,陪大妗子做伴儿罢。”西门庆道:“我已预备下五分人情,都去走走罢。左右有大姐在家陪大妗子,就是一般。我已许下应二了。”月娘听了,一声儿没言语。李桂姐便拜辞说道:“娘,我今日家去罢。”月娘道:“慌去怎的,再住一日儿不是?”桂姐道:“不瞒娘说,俺妈心里不自在,家中没人,改日正月间来住两回儿罢。”拜辞了西门庆。月娘装了两盘茶食,又与桂姐一两银子,吃了茶,打发出门。

西门庆才穿上衣服,往前边去,忽有平安儿来报:“荆都监老爹来拜。”西门庆即出迎接,至厅上叙礼。荆都监叩拜堂上道:“久违,欠礼,高转失贺。”西门庆道:“多承厚贶,尚未奉贺。”叙毕契阔之情,分宾主坐下,左右献上茶汤。荆都监便道:“良骑俟候何往?”西门庆道:“京中太师老爷第九公子九江蔡知府,昨日巡按宋公祖与工部安凤山、钱云野、黄泰宇,都借学生这里作东,请他一饭。蒙他具拜贴与我,我岂可不回拜他拜去?诚恐他一时起身去了。”荆都监道:“正是。小弟有一事特来奉渎。巡按宋公正月间差满,只怕年终举劾地方官员,望乞四泉借重与他一说。闻知昨日在宅上吃酒,故此斗胆恃爱。倘得寸进,不敢有忘。”西门庆道:“此是好事,你我相厚,敢不领命?你写个说贴来,幸得他后日还有一席酒在我这里,等我抵面和他说又好说些。”荆都监连忙下位来,又与西门庆打一躬道:“多承盛情,衔结难忘。”便道:“小弟已具了履历手本在此。”一面叫写字的取出,荆都监亲手递上,与西门庆观看。上面写着:“山东等处兵马都监清河左卫指挥佥事荆忠,年三十二岁。系山后檀州人。由祖后军功累升本卫正千户。从某年由武举中式,历升今职,管理济州兵马。”一一开载明白。西门庆看毕,荆都监又向袖中取出礼贴来,递上说道:“薄仪望乞笑留。”西门庆见上面写着“白米二千石”,说道:“岂有此理,这个学生断不敢领,以此视人,相交何在?”荆都监道:“不然。总然四泉不受,转送宋公也是一般,何见拒之深耶?倘不纳,小弟亦不敢奉渎。”推让再三,西门庆只得收了,说道:“学生暂且收下。”一面接了,说道:“学生明日与他说了,就差人回报。”茶汤两换,荆都监拜谢起身去了。西门庆上马,琴童跟随,拜蔡知府去了。

却说玉箫打发西门庆出门,就走到金莲房中,说:“五娘,昨日怎的不往后边去坐?俺娘好不说五娘哩。说五娘听见爹前边散了,往屋里走不迭。昨日三娘生日,就不放往他屋里去,把拦的爹恁紧。三娘道:‘没的羞人子剌剌的,谁耐烦争他。左右是这几房里,随他串去。’”金莲道:“我待说,就没好口,\textuni{34B2}瞎了他的眼来!昨日你道他在我屋里睡来么?”玉箫道:“前边老到只娘屋里。六娘又死了,爹却往谁屋里去?”金莲道:“鸡儿不撒尿——各自有去处。死了一个,还有一个顶窝儿的。”玉箫又说:“俺娘又恼五娘问爹讨皮袄不对他说。落后爹送钥匙到房里,娘说了爹几句好的,说:‘早是李大姐死了,便指望他的,他不死只好看一眼儿罢了。’”金莲道:“没的扯那\textuni{23B48}淡!有一个汉子做主儿罢了,你是我婆婆?你管着我。我把拦他,我拿绳子拴着他腿儿不成?偏有那些\textuni{23B48}声浪气的!”玉箫道:“我来对娘说,娘只放在心里,休要说出我来。今日桂姐也家去了,俺娘收拾戴头面哩,五娘也快些收拾了罢。”说毕,玉箫后边去了。这金莲向镜台前搽胭抹粉,插茶戴翠,又使春梅后边问玉楼,今日穿甚颜色衣裳。玉楼道:“你爹嗔换孝,都教穿浅色衣服。”五个妇人会定了,都是白\textuni{4BFC}髻,珠子箍儿,浅色衣服。惟吴月娘戴着白绉纱金梁冠儿,上穿着沉香遍地金妆花补子袄儿,纱绿遍地金裙。一顶大轿,四顶小轿,排军喝路,棋童、来安三个跟随,拜辞了吴大妗子、三位师父、潘姥姥,径往应伯爵家吃满月酒去了。不题。

却说如意儿和迎春,有西门庆晚夕来吃的一桌菜,安排停当,还有一壶金华酒,向坛内又打出一壶葡萄酒来,午间请了潘姥姥、春梅,郁大姐弹唱着,在房内做一处吃。吃到中间,也是合当有事,春梅道:“只说申二姐会唱的好《挂真儿》,没个人往后边去叫他来,好歹教他唱个咱们听。”迎春才待使绣春叫去,只见春鸿走来烘火。春梅道:“贼小蛮囚儿,你不是冻的那腔儿,还不寻到这屋里来烘火。”因叫迎春:“你酾半瓯子酒与他吃。”分付:“你吃了,替我后边叫将申二姐来。就说我要他唱曲儿与姥姥听。”春鸿把酒勾了,一直走到后边,不想申二姐伴着大妗子、大姐、三个姑子、玉箫都在上房里坐的,正吃茶哩。忽见春鸿掀帘子进来,叫道:“申二姐,你来,俺大姑娘前边叫你唱个曲儿与他听去哩。”这申二姐道:“你大姑娘在这里,又有个大姑娘出来了?”春鸿道:“是俺前边春梅姑娘叫你。”申二姐道:“你春梅姑娘他稀罕怎的,也来叫我?有郁大姐在那里,也是一般。我这里唱与大妗奶奶听哩。”大妗子道:“也罢,申二姐,你去走走再来。”那申二姐坐住了,不动身。

春鸿一直走到前边,对春梅说:“我叫他,他不来哩。”春梅道:“你说我叫他,他就来了。”春鸿道:“我说前边大姑娘叫你,他意思不动,说这是大姑娘,那里又钻出个大姑娘来了?我说是春梅姑娘,他说你春梅姑娘便怎的,有郁大姐罢了,他从几时来也来叫我,我不得闲,在这里唱与大妗奶奶听哩。大妗奶奶到说你去走走再来,他不肯来哩。”这春梅不听便罢,听了三尸神暴跳,五脏气冲天,一点红从耳畔起,须臾紫遍了双腮。众人拦阻不住,一阵风走到上房里,指着申二姐一顿大骂道:“你怎么对着小厮说我‘那里又钻出个大姑娘来了’,‘稀罕他也来叫我’?你是甚么总兵官娘子,不敢叫你!俺们在那毛里夹着,是你抬举起来,如今从新钻出来了?你无非是个走千家门、万家户,贼狗攮的瞎淫妇!你来俺家才走了多少时儿,就敢恁量视人家?你会晓的甚么好成样的套数儿,左右是那几句东沟篱,西沟坝,油嘴狗舌,不上纸笔的那胡歌野词,就拿班做势起来!俺家本司三院唱的老婆,不知见过多少,稀罕你。韩道国那淫妇家兴你,俺这里不兴你。你就学与那淫妇,我也不怕。你好不好趁早儿去,贾妈妈与我离门离户。”那大妗子拦阻说道:“快休要破口。”把申二姐骂的睁睁的,敢怒而不敢言,说道:“耶嚛嚛,这位大姐,怎的恁般粗鲁性儿,就是刚才对着大官儿,我也没曾说甚歹话,怎就这般言语,泼口骂出来!此处不留人,更有留人处。”春梅越发恼了,骂道:“贼食,唱与人家听。趁早儿与我走,再也不要来了。”申二娘道:“我没的赖在你家!”春梅道:“赖在我家,叫小厮把鬓毛都撏光了你的。”大妗子道:“你这孩儿,今日怎的恁样儿的,还不往前边去罢。”那春梅只顾不动身。这申二姐一面哭哭啼啼下炕来,拜辞了大妗子,收拾衣裳包子,也等不的轿子来,央及大妗子使平安对过叫将画童儿来,领他往韩道国家去了。春梅骂了一顿,往前边去了。大妗子看着大姐和玉箫说道:“他敢前边吃了酒进来,不然如何恁冲言冲语的!骂的我也不好看的了。你叫他慢慢收拾了去就是了,立逼着撵他去了,又不叫小厮领他,十分水深人不过。”玉箫道:“他们敢在前头吃酒来?”

却说春梅走到前边,还气狠狠的向众人说道:“方才把贼瞎淫妇两个耳刮子才好。他还不知道我是谁哩!叫着他张儿致儿,拿班做势儿的。”迎春道:“你砍一枝损百枝,忌口些,郁大姐在这里。”春梅道:“不是这等说。像郁大姐在俺家这几年,大大小小,他恶讪了那个来?教他唱个儿,他就唱。那里像这贼瞎淫妇大胆。他记得甚么成样的套数,左来右去,只是那几句《山坡羊》、《琐南枝》,油里滑言语,上个甚么抬盘儿也怎的?我才乍听这个曲儿也怎的?我见他心里就要把郁大姐挣下来一般。”郁大姐道:“可不怎的。昨日晚夕,大娘教我唱小曲儿,他就连忙把琵琶夺过去,他要唱。大姑娘你也休怪,他怎知道咱家里深浅?他还不知把你当谁人看成。”春梅道:“我刚才不骂的:你上覆韩道国老婆那贼淫妇,你就学与他,我也不怕他。”潘姥姥道:“我的姐姐,你没要紧气的恁样儿的。”如意儿道:“我倾杯儿酒,与大姐姐消消儿恼。”迎春道:“我这女儿着恼就是气。”便道:“郁大姐,你拣套好曲儿唱个伏侍他。”这郁大姐拿过琵琶来,说道:“等我唱个“莺莺闹卧房”《山坡羊》儿。与姥姥和大姑娘听罢。”如意儿道:“你用心唱,等我斟上酒。”那迎春拿起杯儿酒来,望着春梅道:“罢罢,我的姐姐,你也不要恼了,胡乱且吃你妈妈这钟酒儿罢。”那春梅忍不住笑骂道:“怪小淫妇儿,你又做起我妈妈来了!”又说道:“郁大姐,休唱《山坡羊》,你唱个《江儿水》俺们听罢。”这郁大姐在旁弹着琵琶,慢慢唱“花娇月艳”,与众人吃酒不题。

且说西门庆从新河口拜了蔡九知府,回来下马,平安就禀:“今日有衙门里何老爹差答应的来,请爹明日早进衙门中,拿了一起贼情审问。又本府胡老爹送了一百本新历日。荆都监老爹差人送了一口鲜猪,一坛豆酒,又是四封银子。姐夫收下,交到后边去了,没敢与他回贴儿。晚上,他家人还来见爹说话哩。只胡老爹家与了回贴,赏了来人一钱银子。又是乔亲家爹送贴儿,明日请爹吃酒。”玳安儿又拿宋御史回贴儿来回话:“小的送到察院内,宋老爹说,明日还奉价过来。赏了小的并抬盒人五钱银子,一百本历日。”西门庆走到厅上,春鸿连忙报与春梅众人,说道:“爹来家了,还吃酒哩。”春梅道:“怪小蛮囚儿,爹来家随他来去,管俺们腿事!没娘在家,他也不往俺这边来。”众人打伙儿吃酒顽笑,只顾不动身。西门庆到上房,大妗子和三个姑子,都往那边屋里去了。玉箫向前与他接了衣裳,坐下,放桌儿打发他吃饭。教来兴儿定桌席:三十日与宋巡按摆酒;初一日刘、薛二内相,帅府周爷众位,吃庆官酒。分付去了。玉箫在旁请问:“爹吃酒,筛甚么酒吃?”西门庆道:“有刚才荆都监送来的那豆酒取来,打开我尝尝,看好不好。”只见来安儿进来,禀问接月娘去。玉箫便使他提酒来,打破泥头,倾在钟内,递与西门庆呷了一呷,碧靛般清,其味深长。西门庆令:“斟来我吃。”须臾,摆上菜来,西门庆在房中吃酒。

却说来安同排军拿灯笼,晚夕接了月娘众人来家。都穿着皮袄,都到上房来拜西门庆。惟雪娥与西门庆磕头,起来又与月娘磕头。拜完了,又都过那边屋里,去拜大妗子与三个姑子。月娘便坐着与西门庆说话:“应二嫂见俺们都去,好不喜欢!酒席上有隔壁马家娘子和应大嫂、杜二娘,也有十来位娘子。叫了两个女儿弹唱。养了好个平头大脸的小厮儿。原来他房里春花儿,比旧时黑瘦了好些,只剩下个大驴脸一般的,也不自在哩。今日乱的他家里大小不安,本等没人手。临来时,应二歌与俺们磕头,谢了又谢,多多上覆你,多谢重礼。”西门庆道:“春花儿那成精奴才,也打扮出来见人?”月娘道:“他比那个没鼻子?没眼儿?是鬼儿?出来见不的?”西门庆道:“那奴才,撒把黑豆只好教猪拱罢。”月娘道:“我就听不上你恁说嘴。只你家的好,拿掇的,出来见的人!”那王经在旁立着,说道:“应二爹见娘们去,先头不敢出来见,躲在下边房里,打窗户眼儿望前瞧。被小的看见了,说道:‘你老人家没廉耻,平日瞧甚么!”他赶着小的打。”西门庆笑的没眼缝儿,说道:“你看这贼花子,等明日他来,着老实抹他一脸粉。”王经笑道:“小的知道了。”月娘喝道:“这小厮别要胡说。他几时瞧来?平白枉口拔舌的。一日谁见他个影儿?只临来时,才与俺们磕头。”王经站了一回出来了。

月娘也起身过这边屋里,拜大妗子并三个师父。大姐与玉箫众丫头媳妇都来磕头。月娘便问:“怎的不见申二姐?”众人都不作声。玉箫说:“申二姐家去了。”月娘道:“他怎的不等我来就去?”大妗子隐瞒不住,把春梅骂他之事,说了一遍。月娘就有几分恼,说道:“他不唱便罢了,这丫头恁惯的没张倒置的,平白骂他怎么的?怪不的俺家主子也没那正主了,奴才也没个规矩,成甚么道理!”望着金莲道:“你也管他管儿,惯的他通没些摺儿。”金莲在旁笑着说道:“也没见这个瞎曳么的,风不摇,树不动。你走千家门,万家户,在人家无非只是唱。人叫你唱个儿,也不失了和气,谁教他拿班儿做势的,他不骂他嫌腥。”月娘道:“你到且是会说话儿的。都像这等,好人歹人都吃他骂了去?也休要管他一管儿了!”金莲道:“莫不为瞎淫妇打他几棍儿?”月娘听了他这句话,气的他脸通红了,说道:“惯着他,明日把六邻亲戚都教他骂遍了罢!”于是起身,走过西门庆这边来。西门庆便问:“怎么的?”月娘道:“情知是谁,你家使的有好规矩的大姐,如此这般,把申二姐骂的去了。”西门庆笑道:“谁教他不唱与他听来。也不打紧处,到明日使小厮送他一两银子,补伏他,也是一般。”玉箫道:“申二姐盒子还在这里,没拿去哩。”月娘见西门庆笑,便说道:“不说教将来嗔喝他两句,亏你还雌着嘴儿,不知笑的是甚么?”玉楼、李娇儿见月娘恼起来,就都先归房去了。西门庆只顾吃酒,良久,月娘进里间内,脱衣裳摘头,便问玉箫:“这箱上四包银子是那里的?”西门庆说:“是荆都监的二百两银子,要央宋巡按,图干升转。”玉箫道:“头里姐夫送进来,我就忘了对娘说。”月娘道:“人家的,还不收进柜里去哩。”玉箫一面安放在厨柜中。

金莲在那边屋里只顾坐的,要等西门庆一答儿往前边去,今日晚夕要吃薛姑子符药,与他交媾,图壬子日好生子。见西门庆不动身,走来掀帘子儿叫他说:“你不往前边去,我等不得你,我先去也。”西门庆道:“我儿,你先走一步儿,我吃了这些酒来。”那金莲一直往前去了。月娘道:“我偏不要你去,我还和你说话哩。你两个合穿着一条裤子也怎的?强汗世界,巴巴走来我屋里,硬来叫你。没廉耻的货,只你是他的老婆,别人不是他的老婆?你这贼皮搭行货子,怪不的人说你。一视同仁,都是你的老婆,休要显出来便好。就吃他在前边把拦住了,从东京来,通影边儿不进后边歇一夜儿,教人怎么不恼?你冷灶着一把儿,热灶着一把儿才好,通教他把拦住了,我便罢了,不和你一般见识,别人他肯让的过?口儿内虽故不言语,好杀他心儿里也有几分恼。今日孟三姐在应二嫂那里,通一日没吃甚么儿,不知掉了口冷气,只害心凄恶心。来家,应二嫂递了两钟酒,都吐了。你还不往屋里瞧他瞧去?”

西门庆听了,说道:“真个?分付收了家火罢,我不吃酒了。”于是走到玉楼房中。只见妇人已脱了衣裳,摘去首饰,浑衣儿歪在炕上,正倒着身子呕吐。西门庆见他呻吟不止,慌问道:“我的儿,你心里怎么的来?对我说,明日请人来看你。”妇人一声不言语,只顾呕吐。被西门庆一面抱起他来,与他坐的,见他两只手只揉胸前,便问:“我的心肝,心里怎么?告诉我。”妇人道:“我害心凄的慌,你问他怎的?你干你那营生去。”西门庆道:“我不知道,刚才上房对我说,我才晓的。”妇人道:“可知你不晓的。俺每不是你老婆,你疼你那心爱的去罢。”西门庆于是搂过粉项来亲个嘴,说道:“怪油嘴,就奚落我起来。”便叫兰香:“快顿好苦艳茶儿来,与你娘吃。”兰香道:“有茶伺候着哩。”一面捧茶上来。西门庆亲手拿在他口儿边吃。妇人道:“拿来,等我自吃。会那等乔劬劳,旋蒸热卖儿的,谁这里争你哩!今日日头打西出来,稀罕往俺这屋里来走一走儿。也有这大娘,平白说怎的,争出来\textuni{24238}包气。”西门庆道:“你不知,我这两日七事八事,心不得个闲。”妇人道:“可知你心不得闲,自有那心爱的扯落着你哩。把俺们这僻时的货儿,都打到赘字号听题去了,后十年挂在你那心里。”见西门庆嘴揾着他那香腮,便道:“吃的那酒气,还不与我过一边去。人一日黄汤辣水儿谁尝着来,那里有甚么神思和你两个缠!”西门庆道:“你没吃甚么儿?叫丫头拿饭来咱们吃,我也还没吃饭哩。”妇人道:“你没的说,人这里凄疼的了不得,且吃饭!你要吃,你自家吃去!”西门庆道:“我不吃,我敢也不吃了,咱两个收拾睡了罢。明日早,使小厮请任医官来看你。”妇人道:“由他去,请甚么任医官、李医官,教刘婆子来,吃他服药也好了。”西门庆道:“你睡下,等我替你心口内扑撒扑撒,管情就好了。你不知道,我专一会揣骨捏病。”西门庆忽然想起道:“昨日刘学官送了十圆广东牛黄蜡丸,那药,酒儿吃下极好。”即使兰香:“问你大娘要去,在上房磁罐儿内盛着哩。就拿素儿带些酒来。吃了管情手到病除。”妇人道:“我不好骂出来,你会揣甚么病?要酒,俺这屋里有酒。”

不一时,兰香到上房要了两丸来。西门庆看筛热了酒,剥去腊,里面露出金丸来,拿与玉楼吃下去。西门庆因令兰香:“趁着酒,你筛一钟儿来,我也吃了药罢。”被玉楼瞅了一眼,说道:“就休要汗邪,你要吃药,往别人房里去吃。你这里且做甚么哩,却这等胡作做。你见我不死,来撺掇上路儿来了。紧要教人疼的魂也没了,还要那等掇弄人,亏你也下般的,谁耐烦和你两个只顾涎缠。”西门庆笑道:“罢罢,我的儿,我不吃药了,咱两个睡罢。”那妇人一面吃毕药,与西门庆两个解衣上床同寝。西门庆在被窝内,替他手撒扑着酥胸,揣摸香乳,一手搂其粉项,问道:“我的亲亲,你心口这回吃下药觉好些?”妇人道:“疼便止了,还有些嘈杂。”西门庆道:“不打紧,消一回也好了。”因说道:“你不在家,我今日兑了五十两银子与来兴儿,后日宋御史摆酒,初一日烧纸还愿心,到初三日,再破两日工夫,把人都请了罢。受了人家许多人情礼物,只顾挨着,也不是事。”妇人道:“你请也不在我,不请也不在我。明日三十日,我教小厮来攒帐,交与你,随你交付与六姐,教他管去。也该教他管管儿,却是他昨日说的:‘甚么打紧处,雕佛眼儿便难,等我管。’”西门庆道:“你听那小淫妇儿,他勉强,着紧处他就慌了。亦发摆过这几席酒儿,你交与他就是了。”玉楼道:“我的哥哥,谁养的你恁乖!还说你不护他,这些事儿就见出你那心儿来了。摆过酒儿交与他,俺们是合死的?像这清早辰,得梳个头儿?小厮你来我去,称银换钱,气也掏干了。饶费了心,那个道个是也怎的!”西门庆道:“我的儿,常言道:‘当家三年狗也嫌。’”说着,一面慢慢搊起一只腿儿,跨在胳膊上,搂抱在怀里,揝着他白生生的小腿儿,穿着大红绫子的绣鞋儿,说道:“我的儿,你达不爱你别,只爱你这两只白腿儿,就是普天下妇人选遍了,也没你这等柔嫩可爱。”妇人道:“好个说嘴的货,谁信那棉花嘴儿,可可儿的就是普天下妇人选遍了没有来!不说俺们皮肉儿粗糙,你拿左话儿右说着哩。”西门庆道:“我的心肝,我有句谎就死了我。”妇人道:“行货子,没要紧赌什么誓。”这西门庆说着就把那话带上了银托子,插放入他牝中。妇人道:“我说你行行就下道儿来了。”因摸见银托子,说道:“从多咱三不知就带上这行货子了,还不趁早除下来哩。”那西门庆那里肯依,抱定他一只腿在怀里,只顾没棱露脑,浅抽深送。须臾淫水浸出,往来有声,如狗茶镪子一般,妇人一面用绢抹尽了去,口里内不住作柔颤声,叫他:“达达,你省可往里边去,奴这两日好不腰酸,下边流白浆子出来。”西门庆道:“我到明日问任医官讨服暖药来,你吃就好了。”

不说两个在床上欢娱顽耍,单表吴月娘在上房陪着大妗子、三位师父,晚夕坐的说话。因说起春梅怎的骂申二姐,骂的哭涕,又不容他坐轿子去,旋央及大妗子,对过叫画童儿送他往韩道国家去。大妗子道:“本等春梅出来的言语粗鲁,饶我那等说着,还刀截的言语骂出来,他怎的不急了!他平昔不晓的恁口泼骂人,我只说他吃了酒。”小玉道:“他们五个在前头吃酒来。”月娘道:“恁不合理的行货子,生生把丫头惯的恁没大没小的,还嗔人说哩。到明日不管好歹,人都吃他骂了去罢,要俺们在屋里做甚么?一个女儿,他走千家门,万家户,教他传出去好听?敢说西门庆家那大老婆,也不知怎么出来的。乱世不知那个是主子,那个是奴才。不说你们这等惯的没些规矩,恰似俺们不长俊一般,成个甚么道理!”大妗子道:“随他去罢,他姑夫不言语,怎好惹气?”当夜无辞,同归到房中歇了。

次日,西门庆早起往衙门中去了。潘金莲见月娘拦了西门庆不放来,又误了壬子日期,心中甚是不悦。次日,老早就使来安叫了一顶轿子,把潘姥姥打发往家去了。吴月娘早辰起来,三个姑子要告辞家去,月娘每个一盒茶食,五钱银子,又许下薛姑子正月里庵里打斋,先与他一两银子,请香烛纸马,到腊月还送香油、白面、细米素食与他斋僧供佛。因摆下茶,在上房内管待,同大妗子一处吃。先请了李娇儿、孟玉楼、大姐,都坐下。问玉楼:“你吃了那蜡丸,心口内不疼了?”玉楼道:“今早吐了两口酸水,才好了。”叫小玉往前边:“请潘姥姥和五娘来吃点心。”玉箫道:“小玉在后边蒸点心哩。我去请罢。”于是一直走了前边金莲房中,便问他:“姥姥怎的不见?后边请姥姥和五娘吃茶哩。”金莲道:“他今日早辰,我打发他家去了。”玉箫说:“怎的不说声,三不知就去了?”金莲道:“住的人心淡,只顾住着怎的!”玉箫道:“我拿了块腊肉儿,四个甜酱瓜茄子,与他老人家,谁知他就去了。五娘你替老人家收着罢。”于是递与秋菊,放在抽替内。这玉箫便向金莲说道:“昨日晚夕五娘来了,俺娘如此这般对着爹好不说五娘强汗世界,与爹两个合穿着一条裤子,没廉耻,怎的把拦老爹在前边,不往后边来。落后把爹打发三娘房里歇了一夜,又对着大妗子、三位师父,怎的说五娘惯的春梅没规矩,毁骂申二姐。爹到明日还要送一两银子与申二姐遮羞。”一五一十说了一时。这金莲听记在心。玉箫先来回月娘说:“姥姥起早往家去了,五娘便来也。”月娘便望着大妗子道:“你看,昨日说了他两句儿,今日就使性子,也不进来说声儿,老早打发他娘去了。我猜姐姐又不知心里安排着要起甚么水头儿哩。”

当下月娘自知屋里说话,不防金莲暗走到明间帘下,听觑多时了,猛可开言说道:“可是大娘说的,我打发了他家去,我好把拦汉子?”月娘道:“是我说来,你如今怎么我?本等一个汉子,从东京来了,成日只把拦在你那前头,通不来后边傍个影儿。原来只你是他的老婆,别人不是他的老婆?行动题起来,别人不知道,我知道。就是昨日李桂姐家去了,大妗子问了声:‘李桂姐住了一日儿,如何就家去了?他姑夫因为甚么恼他?’我还说:‘谁知为甚么恼他?’你便就撑着头儿说:‘别人不知道,只我晓的。’你成日守着他,怎么不晓的!”金莲道:“他不往我那屋里去,我莫不拿猪毛绳子套了他去不成!那个浪的慌了也怎的?”月娘道:“你不浪的慌,他昨日在我屋里好好儿坐的,你怎的掀着帘子硬入来叫他前边去,是怎么说?汉子顶天立地,吃辛受苦,犯了甚么罪来,你拿猪毛绳子套他?贱不识高低的货,俺每倒不言语了,你倒只顾赶人。一个皮袄儿,你悄悄就问汉子讨了,穿在身上,挂口儿也不来后边题一声儿。都是这等起来,俺每在这屋里放小鸭儿?就是孤老院里也有个甲头。一个使的丫头,和他猫鼠同眠,惯的有些摺儿!不管好歹就骂人。说着你,嘴头子不伏个烧埋。”金莲道:“是我的丫头也怎的?你每打不是!我也在这里,还多着个影儿哩。皮袄是我问他要来。莫不只为我要皮袄,开门来也拿了几件衣裳与人,那个你怎的就不说了?丫头便是我惯了他,是我浪了图汉子喜欢。像这等的却是谁浪?”吴月娘吃他这两句,触在心上,便紫漒了双腮,说道:“这个是我浪了,随你怎的说。我当初是女儿填房嫁他,不是趁来的老婆。那没廉耻趁汉精便浪,俺每真材实料,不浪。”吴大妗子便在跟前拦说:“三姑娘,你怎的,快休舒口。”孟玉楼道:“耶嚛,耶嚛,大娘,你今日怎的这等恼的大发了,连累俺每,一俸打着好几个。也没见这六姐,你让大娘一句儿也罢了,只顾拌起嘴来了。”大妗子道:“常言道,要打没好手,厮骂没好口。不争你姊妹每嚷斗,俺每亲戚在这里住着也羞。姑娘,你不依我,想是嗔我在这里,叫轿子来我家去罢!”被李娇儿一面拉住大妗子,那潘金莲见月娘骂他这等言语,坐在地下就打滚撒泼。自家打几个嘴巴,头上\textuni{4BFC}髻都撞落一边,放声大哭,叫起来说道:“我死了罢,要这命做什么,你家汉子说条念款说将来,我趁将你家来了!这也不难的勾当,等他来家,与了我休书,我去就是了。你赶人不得赶上。”月娘道:“你看就是了,泼脚子货。别人一句儿还没说出来,你看他嘴头子,就相淮洪一般。他还打滚儿赖人,莫不等的汉子来家,把我别变了!你放恁个刁儿,那个怕你么?”金莲道:“你是真材实料的,谁敢辩别你?”月娘越发大怒,说道:“我不真材实料,我敢在这家里养下汉来?”金莲道:“你不养下汉,谁养下汉来?你就拿主儿来与我!”玉楼见两个拌的越发不好起来,一面拉金莲往前边去,说道:“你恁怪剌剌的,大家都省口些罢了。只顾乱起来,左右是两句话,教三位师父笑话。你起来,我送你前边去罢。”那金莲只顾不肯起来,被玉楼和玉箫一齐扯起来,送他前边去了。

大妗子便劝住月娘,说道:“姑娘,你身上又不方便,好惹气,分明没要紧。你姐妹们欢欢喜喜,俺每在这里住着有光。似这等合气起来,又不依个劝,却怎样儿的?”那三个姑子见嚷闹起来,打发小姑儿吃了点心,包了盒子,告辞月娘众人,月娘道:“三位师父,休要笑话。”薛姑子道:“我的佛菩萨,没的说,谁家灶内无烟?心头一点无明火,些儿触着便生烟。大家尽让些就罢了。佛法上不说的好:‘冷心不动一孤舟,净扫灵台正好修。’若还绳头松松,就是万个金刚也降不住。为人只把这心猿意马牢拴住了,成佛作祖都打这上头起。贫僧去也,多有打搅菩萨。好好儿的。”一面打了两个问讯。月娘连忙还万福,说道:“空过师父,多多有慢。另日着人送斋衬去。”即叫大姐:“你和二娘送送三位师父出去,看狗。”于是打发三个姑子出门去了。

月娘陪大妗子坐着,说道:“你看这回气的我,两只胳膊都软了,手冰冷的。从早辰吃了口清茶,还汪在心里。”大妗子道:“姑娘,我这等劝你少揽气,你不依我。你又是临月的身子,有甚要紧。”月娘道:“早是你在这里住看着,又是我和他合气?如今犯夜的倒拿住巡更的。我倒容了人,人倒不肯容我。一个汉子,你就通身把拦住了,和那丫头通同作弊,在前头干的那无所不为的事,人干不出来的,你干出来。女妇人家,通把个廉耻也不顾。他灯台不照自己,还张着嘴儿说人浪。想着有那一个在,成日和那一个合气,对着俺每,千也说那一个的不是,他就是清净姑姑儿了。单管两头和番,曲心矫肚,人面兽心。行说的话儿,就不承认了。赌的那誓唬人子。我洗着眼儿看着他,到明日还不知怎么样儿死哩。刚才摆着茶儿,我还好意等他娘来吃,谁知他三不知的就打发去了。就安排要嚷的心儿,悄悄儿走来这里听。听怎的?那个怕你不成!待等汉子来,轻学重告,把我休了就是了。”小玉道:“俺每都在屋里守着炉台站着,不知五娘几时走来,也不听见他脚步儿响。”孙雪娥道:“他单会行鬼路儿,脚上只穿毡底鞋,你可知听不见。想着起头儿一来时,该和我合了多少气!背地打伙儿嚼说我,教爹打我那两顿,娘还说我和他偏生好斗的。”月娘道:“他活埋惯了人,今日还要活埋我哩。你刚才不见他那等撞头打滚儿,一径使你爹来家知道,管就把我翻倒底下。”李娇儿笑道:“大娘没的说,反了世界!”月娘道:“你不知道,他是那九条尾的狐狸精,把好的吃他弄死了,且稀罕我能多少骨头肉儿!你在俺家这几年,虽是个院中人,不像他久惯牢头。你看他昨日那等气势,硬来我屋里叫汉子:‘你不往前边去,我等不的你,先去。’恰似只他一个人的汉子一般,就占住了。不是我心中不恼,他从东京来家,就不放一夜儿进后边来。一个人的生日,也不往他屋里走走儿去。十个指头,都放在你口内才罢了。”大妗子道:“姑娘,你耐烦,你又常病儿痛儿的,不贪此事,随他去罢。不争你为众好,与人为怨结仇。”劝了一回,玉箫安排上饭来,也不吃,说道:“我这回好头疼,心口内有些恶没没的上来。”教玉箫:“那边炕上,放下枕头,我且躺躺去。”分付李娇儿:“你们陪大妗子吃饭。”那日,郁大姐也要家去,月娘分付:“装一盒子点心,与他五钱银子。”打发去了。

却说西门庆衙门中审问贼情,到午牌时分才来家。正值荆都监家人讨回帖,西门庆道:“多谢你老爹重礼。如何这等计较?你还把那礼扛将回去,等我明日说成了取家来。”家人道:“家老爹没分付,小的怎敢将回去,放在老爹这里也是一般。”西门庆道:“既恁说,你多上覆,我知道了。”拿回贴,又赏家人一两银子。因进上房,见月娘睡在炕上,叫了半日,白不答应。问丫鬟,都不敢说。走到前边金莲房里,见妇人蓬头撒脑,拿着个枕头睡,问着又不言语,更不知怎的。一面封银子,打发荆都监家人去了,走到孟玉楼房中问。玉楼隐瞒不住,只得把月娘和金莲早辰嚷闹合气之事,备说一遍。

这西门庆慌了,走到上房,一把手把月娘拉起来,说道:“你甚要紧,自身上不方便,理那小淫妇儿做甚么?平白和他合甚么气?”月娘道:“我和他合气,是我偏生好斗寻趁他来?他来寻趁将我来!你问众人不是?早辰好意摆下茶儿,请他娘来吃。他使性子把他娘打发去了,便走来后边撑着头儿和我嚷,自家打滚撞头,鬟髻都踩扁了,皇帝上位的叫,只是没打在我脸上罢了。若不是众人拉劝着,是也打成一块。他平白欺负惯了人,他心里也要把我降伏下来。行动就说:‘你家汉子说条念款将我来了,打发了我罢,我不在你家了。’一句话儿出来,他就是十句说不下来,嘴一似淮洪一般,我拿甚么骨秃肉儿拌的他过?专会那泼皮赖肉的,气的我身子软瘫儿热化,甚么孩子李子,就是太子也成不的。如今倒弄的不死不活,心口内只是发胀,肚子往下鳖坠着疼,头又疼,两只胳膊都麻了。刚才桶子上坐了这一回,又不下来。若下来也干净了,省的死了做带累肚子鬼。到半夜寻一条绳子,等我吊死了,随你和他过去。往后没的又像李瓶儿,吃他害死了。我晓的你三年不死老婆,也是大悔气。”西门庆不听便罢,听的说,越发慌了,一面把月娘搂抱在怀里,说道:“我的好姐姐,你别和那小淫妇儿一般见识,他识什么高低香臭?没的气了你,倒值了多的。我往前边骂这贼小淫妇儿去。”月娘道:“你还敢骂他,他还要拿猪毛绳子套你哩。”西门庆道:“你教他说,恼了我,吃我一顿好脚。”因问月娘:“你如今心内怎么的?吃了些甚么儿没有?”月娘道:“谁尝着些甚么儿?大清早辰才拿起茶,等着他娘来吃,他就走来和我嚷起来。如今心内只发胀,肚子往下鳖坠着疼,脑袋又疼,两只胳膊都麻了。你不信,摸我这手,恁半日还同握过来。”西门庆听了,只顾跌脚,说道:“可怎样儿的,快着小厮去请任医官来看看。”月娘道:“请什么任医官?随他去,有命活,没命教他死,才趁了人的心。什么好的老婆?是墙上土坯,去了一层又一层。我就死了,把他扶了正就是了。恁个聪明的人儿,当不的家?”西门庆道:“你也耐烦,把那小淫妇儿只当臭屎一般丢着他去便罢了。你如今不请任后溪来看你看,一时气裹住了这胎气,弄的上不上,下不下,怎么了?”月娘道:“这等,叫刘婆子来瞧瞧,吃他服药,再不,头上剁两针,由他自好了。”西门庆道:“你没的说,那刘婆子老淫妇,他会看甚胎产?叫小厮骑马快请任医官来看。”月娘道:“你敢去请!你就请了来,我也不出去。”西门庆不依他,走到前边,即叫琴童:“快骑马往门外请任老爹,紧等着,一答儿就来。”琴童应诺,骑上马云飞一般去了。西门庆只在屋里厮守着月娘,分付丫头,连忙熬粥儿拿上来,劝他吃,月娘又不吃。等到后晌时分,琴童空回来说:“任老爹在府里上班,未回来。他家知道咱这里请,说明日任老爹绝早就来了。”

月娘见乔大户一替两替来请,便道:“太医已是明日来了,你往乔亲家那里去罢。天晚了,你不去,惹的乔亲家怪。”西门庆道:“我去了,谁看你?”月娘笑道:“傻行货子,谁要你做恁个腔儿。你去,我不妨事。等我消一回儿,慢慢挣痤着起来,与大妗子坐的吃饭。你慌的是些甚么?”西门庆令玉箫:“快请你大妗子来,和你娘坐的。”又问:“郁大姐在那里?叫他唱与娘听。”玉箫道:“郁大姐往家去,不耐烦了。”西门庆道:“谁教他去来?留他两住两日儿也罢了。”赶着玉箫踢了两脚。月娘道:“他见你家反宅乱,要去,管他腿事?”玉箫道:“正经骂申二姐的倒不踢。”那西门庆只做不听见,一面穿了衣裳,往乔大户家吃酒去了。未到起更时分,就来家,到了上房。月娘正和大妗子、玉楼、李娇儿四个坐的。大妗子见西门庆进来,忙往后边去了。西门庆便问月娘道:“你这咱好些了么?”月娘道:“大妗子陪我吃了两口粥儿,心口内不大十分胀了,还只有些头疼腰酸。”西门庆道:“不打紧,明日任后溪来看,吃他两服药,解散散气,安安胎就好了。”月娘道:“我那等样教你休请他,你又请他。白眉赤眼,教人家汉子来做甚么?你明日看我出去不出去!”因问:“乔亲家请你做甚么?”西门庆道:“他说我从东京来了,与我坐坐。今日他也费心,整治许多菜蔬,叫两个唱的,落后又邀过来台官来陪我。我热着你,心里不自在,吃了几钟酒,老早就来了。”月娘道:“好个说嘴的货!我听不上你这巧言花语,可可儿就是热着我来?我是那活佛出现,也不放在你那惦。就死了也不值个破沙锅片子。”又问:“乔亲家再没和你说什么话?”西门庆方告说:“乔亲家如今要趁着新例,上三十两银子纳个义官。银子也封下了,教我对胡府尹说。我说不打紧,胡府尹昨日送了我一百本历日,我还没曾回他礼。等我送礼时,稍了贴子与他,问他讨一张义官札付来与你就是了。他不肯,他说纳些银子是正理。如今央这里分上讨讨儿,免上下使用,也省十来两银子。”月娘道:“既是他央及你,替他讨讨儿罢。你没拿他银子来?”西门庆道:“他银子明日送过来。还要买分礼来,我止住他了。到明日,咱佥一口猪,一坛酒,送胡府尹就是了。”说毕,西门庆晚夕就在上房睡了一夜。

到次日,宋巡按摆酒,后厅筵席治酒,装定果品。大清早辰,本府出票拨了两院三十名官身乐人,两名伶官、四名排长领着,来西门庆宅中答应。只见任医官从早辰就骑马来了,西门庆忙迎到厅上陪坐,道连日阔怀之事。任医官道:“昨日盛使到,学生该班,至晚才来家,见尊剌,今日不俟驾而来。敢问何人欠安?”西门庆道:“大贱内偶然有些失调,请后溪一诊。”须臾茶至。吃了茶,任医官道:“昨日闻得明川说,老先生恭喜,容当奉贺。”西门庆道:“菲才备员而已,何贺之有。”一面西门庆分付:“后边对你大娘说,任老爹来了,明间内收拾。”琴童应诺,到后边。大妗子、李娇儿、孟玉楼都在房内,只见琴童来说:“任医官来了,爹分付教收拾明间里坐的。”月娘只不动身,说道:“我说不要请他,平白教人家汉子,睁着活眼,把手捏腕的,不知做甚么!叫刘妈妈子来,吃两服药,由他好了。好这等摇铃打鼓的,好与人家汉子喂眼。”玉楼道:“大娘,已是请人来了,你不出去却怎样的,莫不回了人去不成?”大妗子又在旁边劝着说:“姑娘,他是个太医,你教他看看你这脉息,还知道你这病源,不知你为甚起气恼,伤犯了那一经。吃了他药,替你分理理气血,安安胎气也好。刘婆子他晓得甚么病源脉理?一时耽误怎了。”月娘方动身梳头,戴上冠儿,玉箫拿镜子,孟玉楼跳上炕去,替他拿抿子掠后鬓。李娇儿替他勒钿儿。孙雪娥预备拿衣裳。不一时,打扮的粉妆玉琢,正是:

\[
罗浮仙子临凡世,月殿婵娟出画堂。
\]


\newpage
%# -*- coding:utf-8 -*-
%%%%%%%%%%%%%%%%%%%%%%%%%%%%%%%%%%%%%%%%%%%%%%%%%%%%%%%%%%%%%%%%%%%%%%%%%%%%%%%%%%%%%


\chapter{春梅娇撒西门庆\KG 画童哭躲温葵轩}


诗曰:

\[
相劝频携金粟杯,莫将闲事系柔怀。
年年只是人依旧,处处何曾花不开?
歌咏且添诗酒兴,醉酣还命管弦来。
尊前百事皆如昨,简点惟无温秀才。
\]

话说西门庆见月娘半日不出去,又亲自进来催促,见月娘穿衣裳,方才请任医官进明间内坐下。少顷,月娘从房内出来,望上道了万福,慌的任医官躲在旁边,屈身还礼。月娘就在对面椅上坐下。琴童安放桌儿锦茵,月娘向袖口边伸玉腕,露青葱,教任医官诊脉。良久诊完,月娘又道了个万福。抽身回房去了。房中小厮拿出茶来。吃毕茶,任医官说道:“老夫人原来禀的气血弱,尺脉来的浮涩。虽是胎气,有些荣卫失调,易生嗔怒,又动了肝火。如今头目不清,中膈有些阻滞烦闷,四肢之内,血少而气多。”月娘使出琴童来说:“娘如今只是有些头疼心胀,胳膊发麻,肚腹往下坠着疼,腰酸,吃饮食无味。”任医官道:“我已知道,说得明白了。”西门庆道:“不瞒后溪说,房下如今见怀临月身孕,因着气恼,不能运转,滞在胸膈间。望乞老先生留神加减一二,足见厚情。”任医官道:“岂劳分付,学生无不用心。此去就奉过安胎理气和中养荣蠲痛之剂来。老夫人服过,要戒气恼,就厚味也少吃。”西门庆道:“望乞老先生把他这胎气好生安一安。”任医官道:“已定安胎理气,养其荣卫,不劳分付,学生自有斟酌。”西门庆复说:“学生第三房下有些肚疼,望乞有暖宫丸药,并见赐些。”任医官道:“学生谨领,就封过来。”说毕起身,走到前厅院内,见许多教坊乐工伺候,因问:“老翁,今日府上有甚事?”西门庆道:“巡按宋公连两司官,请巡抚侯石泉老先生,在舍摆酒。”这任医官听了,越发骇然尊敬,在前门揖让上马,打了恭又打恭,比寻常不同,倍加敬重。西门庆送他回来,随即封了一两银子,两方手帕,使琴童骑马讨药去。

李娇儿、孟玉楼众人,都在月娘房里装定果盒,搽抹银器。因说:“大娘,你头里还要不出去,怎么他看了就知道你心中的病?”月娘道:“甚么好成样的老婆,由他死便死了罢,可是他说的:‘你是我婆婆?无故只是大小之分罢了。我还大他八个月哩,汉子疼我,你只好看我一眼儿罢了。’他不讨了他口里话,他怎么和我大嚷大闹?若不是你们撺掇我出去,我后十年也不出去。随他死,教他死去!常言道:‘一鸡死,一鸡鸣,新来鸡儿打鸣忒好听。’我死了,把他立起来,也不乱,也不嚷,才‘拔了萝卜地皮宽”。”玉楼道:“大娘,耶嚛,耶嚛!那里有此话,俺每就替他赌个大誓。这六姐,不是我说他,有些不知好歹,行事要便勉强,恰似咬群出尖儿的一般,一个大有口没心的行货子。大娘你恼他,可知错恼了哩。”月娘道:“他是比你没心?他一团儿心机。他怎的会悄悄听人,行动拿话儿讥讽人。”玉楼道:“娘,你是个当家人,恶水缸儿,不恁大量些,却怎样儿的!常言一个君子待了十个小人。你手放高些,他敢过去了;你若与他一般见识起来,他敢过不去。”月娘道:“只有了汉子与他做主儿着,那大老婆且打靠后。”玉楼道:“哄那个哩?如今像大娘心里恁不好,他爹敢往那屋里去么!”月娘道:“他怎的不去?可是他说的,他屋里拿猪心绳子套,他不去?一个汉子的心,如同没笼头的马一般,他要喜欢那一个,只喜欢那个。谁敢拦他拦,他又说是浪了。”玉楼道:“罢么,大娘,你已是说过,通把气儿纳纳儿。等我教他来与娘磕头,赔个不是。趁着他大妗子在这里,你们两个笑开了罢。你不然,教他爹两个里不作难?就行走也不方便。但要往他屋里去,又怕你恼;若不去,他又不敢出来。今日前边恁摆酒,俺们都在这里定果盒,忙的了不得,他到落得在屋里躲猾儿。俺每也饶不过他。大妗子,我说的是不是?”大妗子道:“姑娘,也罢,他三娘也说的是。不争你两个话差,只顾不见面,教他姑夫也难,两下里都不好行走的。”月娘通一声也不言语。

孟玉楼抽身往前走。月娘道:“孟三姐,不要叫他去,随他来不来罢。”玉楼道:“他不敢不来,若不来,我可拿猪毛绳子套了他来。”一直走到金莲房中,见他头也不梳,把脸黄着,坐在炕上。玉楼道:“五姐,你怎的装憨儿?把头梳起来,今日前边摆酒,后边恁忙乱,你也进去走走儿,怎的只顾使性儿起来?刚才如此这般,俺每劝了他这一回。你去到后边,把恶气儿揣在怀里,将出好气儿来,看怎的与他下个礼,赔个不是儿罢。你我既在矮檐下,怎敢不低头。常言:‘甜言美语三冬暖,恶语伤人六月寒’。你两个已是见过话,只顾使性儿到几时?人受一口气,佛受一炉香,你去与他赔个不是儿,天大事都了了。不然,你不教爹两下里也难。待要往你这边来,他又恼。”金莲道:“耶嚛,耶嚛!我拿甚么比他?可是他说的,他是真材实料,正经夫妻,你我都是趁来的露水,能有多大汤水儿?比他的脚指头儿也比不的儿。”玉楼道:“你又说,我昨日不说的,一棒打三四个人。就是后婚老婆,也不是趁将来的,当初也有个三媒六证,难道只恁就跟了往你家来!砍一枝,损百株,就是六姐恼了你,还有没恼你的。有势休要使尽,有话休要说尽。凡事看上顾下,留些儿防后才好。不管蜢虫、蚂蚱,一例都说着。对着他三位师父、郁大姐。人人有面,树树有皮,俺每脸上就没些血儿?他今日也觉不好意思的。只是你不去,却怎样儿的?少不的逐日唇不离腮,还有一处儿。你快些把头梳了,咱两个一答儿到后边去。”那潘金莲见他恁般说,寻思了半日,忍气吞声,镜台前拿过抿镜,只抿了头,戴上\textuni{4BFC}髻,穿上衣裳,同玉楼径到后边上房来。

玉楼掀开帘儿先进去,说道:“我怎的走了去就牵了他来!他不敢不来!”便道:“我儿,还不过来与你娘磕头!”在旁边便道:“亲家,孩儿年幼,不识好歹,冲撞亲家。高抬贵手,将就他罢,饶过这一遭儿。到明日再无礼,犯到亲家手里,随亲家打,我老身也不敢说了。”那潘金莲与月娘磕了四个头,跳起来,赶着玉楼打道:“汗邪了你这麻淫妇,你又做我娘来了。”连众人都笑了,那月娘忍不住也笑了。玉楼道:“贼奴才,你见你主子与了你好脸儿,就抖毛儿打起老娘来了。”大妗子道:“你姐妹们笑开,恁欢喜欢喜却不好?就是俺这姑娘一时间一言半语咭咶你们,大家厮抬厮敬,尽让一句儿就罢了。常言:‘牡丹花儿虽好,还要绿叶扶持。’”月娘道:“他不言语,那个好说他?”金莲道:“娘是个天,俺每是个地。娘容了俺每,俺每骨秃叉着心里。”玉楼打了他肩背一下,说道:“我的儿,你这回才像老娘养的。且休要说嘴,俺每做了这一日话,也该你来助助忙儿。”这金莲便向炕上与玉楼装定果盒,不在话下。

琴童讨将药来,西门庆看了药贴,就叫送进来与月娘、玉楼。月娘便问玉楼:“你也讨药来?”玉楼道:“还是前日看根儿,下首里只是有些怪疼,我教他爹对任医官说,稍带两服丸子药来我吃。”月娘道:“你还是前日空心掉了冷气了,那里管下寒的是!”

按下后边。却说前厅宋御史先到了,西门庆陪他在卷棚内坐。宋御史深谢其炉鼎之事:“学生还当奉价。”西门庆道:“奉送公祖,犹恐见却,岂敢云价。”宋御史道:“这等,何以克当?”一面又作揖致谢。茶罢,因说起地方民情风俗一节,西门庆大略可否而答之。次问及有司官员,西门庆道:“卑职只知本府胡正堂民望素著,李知县吏事克勤。其余不知其详,不敢妄说。”宋御史问道:“守备周秀曾与执事相交,为人却也好不好?”西门庆道:“周总兵虽历练老成,还不如济州荆都监,青年武举出身,才勇兼备,公祖倒看他看。”宋御史道:“莫不是都监荆忠?执事何以相熟?”西门庆道:“他与我有一面之交,昨日递了个手本与我,望乞公祖青盼一二。”宋御史道:“我也久闻他是个好将官。”又问其次者,西门庆道:“卑职还有妻兄吴铠,见任本衙右所正千户之职。昨日委管修义仓,例该升指挥,亦望公祖提拔,实卑职之沾恩惠也。”宋御史道:“既是令亲,到明日类本之时,不但加升本等职级,我还保举他见任管事。”西门庆连忙作揖谢了,因把荆都监并吴大舅履历手本递上。宋御史看了,即令书吏收执,分付:“到明日类本之时,呈行我看。”那吏典收下去了。西门庆又令左右悄悄递了三两银子与他,不在话下。

正说话间,前厅鼓乐响,左右来报:“两司老爷都到了。”慌的西门庆即出迎接,到厅上叙礼。这宋御史慢慢才走出花园角门。众官见礼毕数,观看正中摆设大插卓一张,五老定胜方糖,高顶簇盘,甚是齐正,周围卓席俱丰胜,心中大悦。都望西门庆谢道:“生受,容当奉补。”宋御史道:“分资诚为不足,四泉看我分上罢了,诸公不消奉补。”西门庆道:“岂有此理。”一面各分次序坐下,左右拿上茶来。众官又一面差官邀去。

看看等到午后,只见一匹报马来到说:“侯爷来了。”这里两边鼓乐一齐响起,众官都出大门迎接。宋御史只在二门里相候。不一时,蓝旗马道过尽,侯巡抚穿大红孔雀,戴貂鼠暖耳,浑金带,坐四人大轿,直至门首下轿。众官迎接进来。宋御史亦换了大红金云白豸暖耳,犀角带,相让而入。到于大厅上,叙毕礼数,各官廷参毕,然后是西门庆拜见。侯巡抚因前次摆酒请六黄太尉,认得西门庆。即令官吏拿双红友生侯濛单拜贴,递与西门庆。西门庆双手接了,分付家人捧上去。一面参拜毕,宽衣上坐。众官两旁佥坐,宋御史居主位。奉毕茶,阶下动起乐来。宋御史递酒簪花,捧上尺头,随即抬下卓席来,装在盒内,差官吏送到公厅去了。然后上坐,献汤饭,割献花猪,俱不必细说。先是教坊吊队舞,撮弄百戏,十分齐整。然后才是海盐子弟上来磕头,呈上关目揭贴。侯公分付搬演《裴晋公还带记》。唱了一折下来,又割锦缠羊。端的花簇锦攒,吹弹歌舞,箫韶盈耳,金貂满座。有诗为证:

\[
华堂非雾亦非渐,歌遏行云酒满筵。
不但红娥垂玉佩,果然绿鬓插金蝉。
\]

侯巡抚只坐到日西时分,酒过数巡,歌唱两折下来,令左右拿五两银子,分赏厨役、茶酒、乐工、脚下人等,就穿衣起身。众官俱送出大门,看着上轿而去。回来,宋御史与众官谢了西门庆,亦告辞而归。

西门庆送了回来,打发乐工散了。因见天色尚早,分付把卓席休动。一面使小厮请吴大舅并温秀才、应伯爵、傅伙计、甘伙计、贲第传、陈敬济来坐,听唱。又拿下两卓酒肴,打发子弟吃了。等的人来,教他唱《四节记(冬景)韩熙载夜宴陶学士》抬出梅花来,放在两边卓上,赏梅饮酒。先是三伙计来旁坐下。不一时,温秀才也过来了,吴大舅、吴二舅、应伯爵都来了。应伯爵与西门庆唱喏:“前日空过众位嫂子,又多谢重礼。”西门庆笑骂道:“贼天杀的狗材,你打窗户眼儿内偷瞧的你娘们好!”伯爵道:“你休听人胡说,岂有此理。我想来也没人。”指王经道:“就是你这贼狗骨秃儿,干净来家就学舌。我到明日把你这小狗骨秃儿肉也咬了。”说毕,吃了茶。

吴大舅要到后边,西门庆陪下来,向吴大舅如此这般说:“对宋大巡已替大舅说,他看了揭贴,交付书办收了。我又与了书办三两银子,连荆大人的都放在一处。他亲口许下,到明日类本之时,自有意思。”吴大舅听了,满心欢喜,连忙与西门庆唱喏:“多累姐夫费心。”西门庆道:“我就说是我妻兄,他说既是令亲,我已定见过分上。”于是同到房中,见了月娘。月娘与他哥道万福。大舅向大妗子说道:“你往家去罢了,家里没人,如何只顾不去了?”大妗子道:“三姑娘留下,教我过了初三日去哩。”吴大舅道:“既是姑娘留你,到初四日去便了。”说毕,来到前边,同众坐下饮酒。不一时,下边戏子锣鼓响动,搬演《韩熙载夜宴(邮亭佳遇)》。正在热闹处,忽见玳安来说:“乔亲家爹那里,使了乔通在下边请爹说话。”西门庆随即下席见乔通。乔通道:“爹说昨日空过亲家。爹使我送那援纳例银子来,一封三十两,另外又拿着五两与吏房使用。”西门庆道:“我明日早封过与胡大尹,他就与了札付来。又与吏房银子做甚么?你还带回去。”一面分付玳安拿酒饭点心,管待乔通,打发去了。

话休饶舌。当日唱了《邮亭》两折,有一更时分,西门庆前边人散了,看收了家火,就进入月娘房来。大妗子正坐的,见西门庆进来,连忙往那边屋里去了。西门庆因向月娘说:“我今日替你哥如此这般对宋巡按说,他许下除加升一级,还教他见任管事,就是指挥佥事。我刚才已对你哥说了,他好不喜欢,只在年终就题本。”月娘便道:“没的说,他一个穷卫家官儿,那里有二三百银子使?”西门庆道:“谁问他要一百文钱儿。我就对宋御史说是我妻兄,他亲口既许下,无有个不做分上的。”月娘道:“随你与他干,我不管你。”西门庆便问玉箫:“替你娘煎了药,拿来我瞧着,打发你娘吃了罢。”月娘道:“你去,休管他,等我临睡自家吃。”那西门庆才待往外走,被月娘又叫回来,问道:“你往那里去?若是往前头去,趁早儿不要去。他头里与我陪过不是了,只少你与他陪不是去哩。”西门庆道:“我不往他屋里去。”月娘道:“你不往他屋里去,往谁屋里去?那前头媳妇子跟前也省可去。惹的他昨日对着大妗子,好不拿话儿咂我,说我纵容着你要他,图你喜欢哩。你又恁没廉耻的。”西门庆道:“你理那小淫妇儿怎的!”月娘道:“你只依我说,今日偏不要你往前边去,也不要你在我这屋里,你往下边李娇姐房里睡去。随你明日去不去,我就不管了。”西门庆见恁说,无法可处,只得往李娇儿房里歇了一夜。

到次日,腊月初一日,早往衙门中同何千户发牌升厅画卯,发放公文。一早辰才来家,又打点礼物猪酒,并三十两银子,差玳安往东平府送胡府尹去。胡府尹收下礼物,即时封过札付来。西门庆在家,请了阴阳徐先生,厅上摆设猪羊酒果,烧纸还愿心毕,打发徐先生去了。因见玳安到了,看了回贴,札付上面用着许多印信,填写乔洪本府义官名目。一面使玳安送两盒胙肉与乔大户家,就请乔大户来吃酒,与他札付瞧。又分送与吴大舅、温秀才、应伯爵、谢希大并众伙计,每人都是一盒,不在话下。一面又发贴儿,初三日请周守备、荆都监、张团练、刘、薛二内相、何千户、范千户、吴大舅、乔大户、王三官儿,共十位客,叫一起杂耍乐工,四个唱的。

那日孟玉楼攒了帐,递与西门庆,就交代与金莲管理,他不管了。因来问月娘道:“大娘,你昨日吃了药儿,可好些?”月娘道:“怪的不人说怪浪肉,平白教人家汉子捏了捏手,今日好了。头也不疼,心口也不发胀了。”玉楼笑道:“大娘,你原来只少他一捏儿。”连大妗子也笑了。西门庆拿了攒的帐来,又问月娘。月娘道:“该那个管,你交与那个就是了。来问我怎的,谁肯让的谁?”这西门庆方打帐兑三十两银子,三十吊钱,交与金莲管理,不在话下。

良久,乔大户到了。西门庆陪他厅上坐的,如此这般拿胡府尹札付与他看。看见上写义官乔洪名字:“援例上纳白米三千石,以济边饷”,满心欢喜,连忙向西门庆失恭致谢:“多累亲家费心,容当叩谢。”因叫乔通:“好生送到家去。”又说:“明日若亲家见招,在下有此冠带,就敢来陪。”西门庆道:“初三日亲家好歹早些下降。”一面吃茶毕,分付琴童,西厢书房里放卓儿。“亲家请那里坐,还暖些。”同到书房,才坐下,只见应伯爵到了。敛了几分人情,交与西门庆,说:“此是列位奉贺哥的分资。”西门庆接了,看头一位就是吴道官,其次应伯爵、谢希大、祝实念、孙寡嘴、常峙节、白赉光、李智、黄四、杜三哥,共十分人情。西门庆道:“我这边还有吴二舅、沈姨夫,门外任医官、花大哥并三个伙计、温蔡轩,也有二十多人,就在初四日请罢。”一面令左右收进人情去,使琴童儿:“拿马请你吴大舅来,陪你乔家亲爹坐。”因问:“温师父在家不在?”来安儿道:“温师父不在家,望朋友去了。”不一时,吴大舅来到,连陈敬济五人共坐,把酒来斟。卓上摆列许多下饭。饮酒中间,西门庆因向吴大舅说:“乔亲家恭喜的事,今日已领下札付来了。容日我这里备礼写文轴,咱每从府中迎贺迎贺。”乔大户道:“惶恐,甚大职役,敢起动列位亲家费心。”忽有本县衙差人送历日来了,共二百五十本。西门庆拿回贴赏赐,打发来人去了。应伯爵道:“新历日俺每不曾见哩。”西门庆把五十本拆开,与乔大户、吴大舅、伯爵三人分开。伯爵看了看,开年改了重和元年,该闰正月。

不说当日席间猜枚行令。饮酒至晚,乔大户先告家去。西门庆陪吴大舅、伯爵坐到起更时分方散。分付伴当:“早伺候备马,邀你何老爹到我这里起身,同往郊外送侯爷,留下四名排军,与来安、春鸿两个,跟大娘轿往夏家去。”说毕,就归金莲房中来。那妇人未等他进房,就先摘了冠儿,乱挽乌云,花容不整,朱粉懒施,浑衣儿歪在床小,叫着只不做声。西门庆便坐在床上问道:“怪小油嘴,你怎的恁个腔儿?”也不答应。被西门庆用手拉起他来,说道:“你如何悻悻的?”那妇人便做出许多乔张致来,把脸扭着,止不住纷纷香腮上滚下泪来。那西门庆就是铁石人,也把心肠软了。连忙一只手搂着他脖子说:“怪油嘴,好好儿的,平白你两个合甚么气?”那妇人半日方回说道:“谁和他合气来?他平白寻起个不是,对着人骂我是拦汉精,趁汉精,趁了你来了。他是真材实料,正经夫妻。谁教你又到我这屋里做甚么!你守着他去就是了,省的我把拦着你。说你来家,只在我这房里缠,早是肉身听着,你这几夜只在我这屋里睡来?白眉赤眼儿的嚼舌根。一件皮袄,也说我不问他,擅自就问汉子讨了。我是使的奴才丫头,莫不往你屋里与你磕头去?为这小肉儿骂了那贼瞎淫妇,也说不管,偏有那些声气的。你是个男子汉,若是有主张,一拳柱定,那里有这些闲言帐语。怪不的俺每自轻自贱,常言道:‘贱里买来贱里卖,容易得来容易舍。’趁将你家来,与你家做小老婆,不气长。你看昨日,生怕气了他,在屋里守着的是谁?请太医的是谁?在跟前撺拨侍奉的是谁?苦恼俺每这阴山背后,就死在这屋里,也没个人儿来揪问。这个就是出那人的心来了!还教我含着眼泪儿,走到后边与他赔不是。”说着,那桃花脸上止不住又滚下珍珠儿,倒在西门庆怀里,呜呜咽咽,哭的捽鼻涕弹眼泪。西门庆一面搂抱着劝道:“罢么,我的儿,我连日心中有事,你两家各省一句儿就罢了。你教我说谁的是?昨日要来看你,他说我来与你赔不是,不放我来。我往李娇儿房里睡了一夜。虽然我和人睡,一片心只想着你。”妇人道:“罢么,我也见出你那心来了。一味在我面上虚情假意,倒老还疼你那正经夫妻。他如今替你怀着孩子,俺每一根草儿,拿甚么比他!”被西门庆搂过脖子来亲了个嘴,道:“小油嘴,休要胡说。”只见秋菊拿进茶来。西门庆便道:“贼奴才,好干净儿,如何教他拿茶?”因问:“春梅怎的不见?”妇人道:“你还问春梅哩,他饿的还有一口游气儿,那屋里躺着不是。带今日三四日没吃点汤水儿了,一心只要寻死在那里。说他大娘,对着人骂了他奴才,气生气死,整哭了三四日了。”这西门庆听了,说道:“真个?”妇人道:“莫不我哄你不成,你瞧去不是!”

这西门庆慌过这边屋里,只见春梅容妆不整,云髻歪斜,睡在炕上。西门庆叫道:“怪小油嘴,你怎的不起来?”叫着他,只不做声,推睡。被西门庆双关抱将起来。那春梅从酩子里伸腰,一个鲤鱼打挺,险些儿没把西门庆扫了一交,早是抱的牢,有护炕倚住不倒。春梅道:“达达,放开了手。你又来理论俺每这奴才做甚么?也玷辱了你这两只手。”西门庆道:“小油嘴儿,你大娘说了你两句儿罢了,只顾使起性儿来了。说你这两日没吃饭?”春梅道:“吃饭不吃饭,你管他怎的!左右是奴才货儿,死便随他死了罢。我做奴才,也没干坏了甚么事,并没教主子骂我一句儿,打我一下儿,做甚么为这\textuni{34B2}遍街捣遍巷的贼瞎妇,教大娘这等骂我,嗔俺娘不管我,莫不为瞎淫妇打我五板儿?等到明日,韩道国老婆不来便罢,若来,你看我指着他一顿好骂。原来送了这瞎淫妇来,就是个祸根。”西门庆道:“就是送了他来,也是好意,谁晓的为他合起气来。”春梅道:“他若肯放和气些,我好骂他?他小量人家!”西门庆道:“我来这里,你还不倒钟茶儿我吃?那奴才手不干净,我不吃他倒的茶。”春梅道:“死了王屠,连毛吃猪。我如今走也走不动在这里,还教我倒甚么茶?”西门庆道:“怪小油嘴儿,谁教你不吃些甚么儿?”因说道:“咱每往那边屋里去。我也还没吃饭哩,教秋菊后边取菜儿,筛酒,烤果馅饼儿,炊鲜汤咱每吃。”于是不由分诉,拉着春梅手到妇人房内。分付秋菊:“拿盒子后边取吃饭的菜儿去。”不一时,拿了一方盒菜蔬来。西门庆分付春梅:“把肉鲊拆上几丝鸡肉,加上酸笋韭菜,和成一大碗香喷喷馄饨汤来。”放下卓儿摆上,一面盛饭来。又烤了一盒果馅饼儿。西门庆和金莲并肩而坐,春梅也在旁陪着同吃。三个你一杯,我一杯,吃到一更方睡。

到次日,西门庆起早,约会何千户来到,吃了头脑酒,起身同往郊外送侯巡抚去了。吴月娘先送礼往夏指挥家去,然后打扮,坐大轿,排军喝道,来安、春鸿跟随来吃酒,看他娘子儿,不在话下。

且说玳安、王经看家,将到晌午时分,只见县前卖茶的王妈妈领着何九,来大门首寻问玳安:“老爹在家不在家?”玳安道:“何老人家、王奶奶稀罕,今日那阵风儿吹你老人家来这里走走?”王婆子道:“没勾当怎好来踅门踅户?今日不因老九,为他兄弟的事,要央烦你老爹,老身还不敢来。”玳安道:“老爷今日与侯爷送行去了,俺大娘也不在家。你老人家站站,等我进去对五娘说声。”进入不多时出来,说道:“俺五娘请你老人家进去哩。”王婆道:“我敢进去?你引我引儿,只怕有狗。”那玳安引他进入花园金莲房门首,掀开帘子,王婆进去。见妇人家常戴着卧免儿,穿着一身锦段衣裳,搽抹的粉妆玉琢,正在炕上脚登着炉台儿坐的。进去不免下礼,慌的妇人答礼,说道:“老王免了罢。”那婆子见毕礼,坐在炕边头。妇人便问:“怎的一向不见你?”王婆子道:“老身心中常想着娘子,只是不敢来亲近。”问:“添了哥哥不曾?”妇人道:“有倒好了。小产过两遍,白不存。”问:“你儿子有了亲事来?”王婆道:“还不曾与他寻。他跟客人淮上来家这一年多,家中积攒了些,买个驴儿,胡乱磨些面儿卖来度日。”因问:“老爹不在家了?”妇人道:“他今日往门外与抚按官送行去了,他大娘也不在家,有甚话说?”王婆道:“何老九有桩事,央及老身来对老爹说:他兄弟何十吃贼攀了,见拿在提刑院老爹手里问。攀他是窝主。本等与他无干,望乞老爹案下与他分豁分豁。贼若指攀,只不准他就是了。何十出来,到明日买礼来重谢老爹,有个说贴儿在此。”一面递与妇人。妇人看了,说道:“你留下,等你老爹来家,我与他瞧。”婆子道:“老九在前边伺候着哩,明日教他来讨话罢。”

妇人一面叫秋菊看茶来,须臾,秋菊拿了一盏茶来,与王婆吃了。那婆子坐着,说道:“娘子,你这般受福勾了。”妇人道:“甚么勾了,不惹气便好,成日欧气不了在这里。”婆子道:“我的奶奶,你饭来张口,水来湿手,这等插金戴银,呼奴使婢,又惹甚么气?”妇人道:“常言说得好,三窝两块,大妇小妻,一个碗内两张匙,不是汤着就抹着。如何没些气儿?”婆子道:“好奶奶,你比那个不聪明!趁着老爹这等好时月,你受用到那里是那里。”说道:“我明日使他来讨话罢。”于是拜辞起身。妇人道:“老王,你多坐回去不是?”那婆子道:“难为老九,只顾等我,不坐罢。改日再来看你。”妇人也不留他留儿,就放出他来了。到了门首,又叮咛玳安。玳安道:“你老人家去,我知道,等俺爹来家我就禀。”何九道:“安哥,我明日早来讨话罢。”于是和王婆一路去了。

至晚,西门庆来家。玳安便把此事禀知。西门庆到金莲房看了贴子,交付与答应的收着:“明日到衙门中禀我。”一面又令陈敬济发初四日请人贴子。瞒着春梅,又使琴童儿送了一两银子并一盒点心到韩道国家,对着他说:“是与申二姐的,教他休恼。”那王六儿笑嘻嘻接了,说:“他不敢恼。多上覆爹娘,冲撞他春梅姑娘。”俱不在言表。

至晚,月娘来家,先拜见大妗子众人,然后见西门庆,道了万福,就告诉:“夏大人娘子见了我去,好不喜欢。今日也有许多亲邻堂客。原来夏大人有书来了,也有与你的书,明日送来与你。也只在这初六、七起身,搬取家小上京。说了又说,好歹央贲四送他到京就回来。贲四的那孩子长儿,今日与我磕头,好不出跳的好个身段儿。嗔道他旁边捧着茶把眼只顾偷瞧我。我也忘了他,倒是夏大人娘子叫他改换的名字,叫做瑞云,‘过来与你西门奶奶磕头’,他才放下茶托儿,与我磕了四个头。我与了他两枝金花儿。夏大人娘子好不喜欢,抬举他,也不把他当房里人,只做亲儿女一般看他。”西门庆道:“还是这孩子有福,若是别人家手里,怎么容得,不骂奴才少椒末儿,又肯抬举他!”被月娘瞅了一眼,说道:“碜说嘴的货,是我骂了你心爱的小姐儿了!”西门庆笑了,说道:“他借了贲四押家小去,我线铺子教谁看?”月娘道:“关两日也罢了。”西门庆道:“关两日,阻了买卖,近年近节,绸绢绒线正快,如何关闭了铺子?到明日再处。”说毕,月娘进里间脱衣裳摘头,走到那边房内,和大妗子坐的。家中大小都来参见磕头。

是日,西门庆在后边雪娥房中歇了一夜,早往衙门中去了。只见何九走来问玳安讨信,与了玳安一两银子。玳安道:“昨日爹来家,就替你说了。今日到衙门中,敢就开出你兄弟来了。你往衙门首伺候。”何九听言,满心欢喜,一直走到衙门前去了。西门庆到衙门中坐厅,提出强盗来,每人又是一夹,二十大板,把何十开出来,放了。另拿了弘化寺一名和尚顶缺,说强盗曾在他寺内宿了一夜。正是:张公吃酒李公醉,桑树上脱枝柳树上报。有诗为证:

\[
宋朝气运已将终,执掌提刑甚不公。
毕竟难逃天下眼,那堪激浊与扬清。
\]

那日西门庆家中叫了四个唱的:吴银儿、郑爱月儿、洪四儿、齐香儿,日头晌午就来了,都到月娘房内,与月娘、大妗子众人磕头。月娘摆茶与他们吃了。正弹着乐器,唱曲儿与众人听,忽见西门庆从衙门中来家,进房来。四个唱的都放了乐器,笑嘻嘻向前,与西门庆磕头。坐下,月娘便问:“你怎的衙门中这咱才来?”西门庆告诉:“今日向理好几桩事情。”因望着金莲说:“昨日王妈妈来说何九那兄弟,今日我已开除来放了。那两名强盗还攀扯他,教我每人打了二十,夹了一夹,拿了门外寺里一个和尚顶缺,明日做文书送过东平府去。又是一起奸情事,是丈母养女婿的。那女婿不上二十多岁,名唤宋得,原与这家是养老不归宗女婿。落后亲丈母死了,娶了个后丈母周氏,不上一年,把丈人死了。这周氏年小,守不得,就与这女婿暗暗通奸,后因为责使女,被使女传于两邻,才首告官。今日取了供招,都一日送过去了。这一到东平府,奸妻之母,系缌麻之亲,两个都是绞罪。”潘金莲道:“要着我,把学舌的奴才打的烂糟糟的,问他个死罪也不多。你穿青衣抱黑柱,一句话就把主子弄了。”西门庆道:“也吃我把那奴才拶了几拶子好的。为你这奴才,一时小节不完,丧了两个人性命。”月娘道:“大不正则小不敬。母狗不掉尾,公独不上身。大凡还是女人心邪,若是那正气的,谁敢犯他!”四个唱的都笑道:“娘说的是。就是俺里边唱的,接了孤老的朋友还使不的,休说外头人家。”说毕,摆饭与西门庆吃了。

忽听前厅鼓乐响,荆都监来了。西门庆连忙冠带出迎,接至厅上叙礼,分宾主坐下。茶罢,如此这般告说:“宋巡按收了说贴,已慨然许下,执事恭喜,必然在迩。”荆都监听了,又下坐作揖致谢:“老翁费心,提携之力,铭刻难忘。”西门庆又说起:“周老总兵,生也荐言一二,宋公必有主意。”谈话间,忽然刘薛二公公到。鼓乐迎接进来,西门太相让入厅,叙礼。二内相皆穿青缧绒蟒衣,宝石绦环,正中间坐下。次后周守备到了,一处叙话。荆都监又向周守备说:“四泉厚情,昨日宋公在尊府摆酒,曾称颂公之才猷。宋公已留神于中,高转在即。”周守备亦欠身致谢不尽。落后张团练、何千户、王三官、范千户、吴大舅、乔大户陆续都到了。乔大户冠带青衣,四个伴当跟随,进门见毕诸公,与西门庆拜了四拜。众人问其恭喜之事,西门庆道:“舍亲家在本府援例新受恩荣义官之职。”周守备道:“四泉令亲,吾辈亦当奉贺。”乔大户道:“蒙列位老爹盛情,岂敢动劳。”说毕,各分次序坐下。遍递了一道茶,然后递酒上坐。锦屏前玳筵罗列,画堂内宝玩争辉,阶前动一派笙歌,席上堆满盘异果。良久,递酒安席毕,各归席坐下。王三官再三不肯上来坐,西门庆道:“寻常罢了,今日在舍,权借一日陪诸公上坐。”王三官必不得已,左边垂首坐了。须臾,上罢汤饭,下边教坊撮弄杂耍百戏上来。良久,才是四个唱的,拿着银筝玉板,放娇声当筵弹唱。正是:

\[
舞裙歌板逐时新,散尽黄金只此身。
寄与富儿休暴殄,俭如良药可医贫。
\]

当日刘内相坐首席,也赏了许多银子。饮酒为欢,至一更时分方散。西门庆打发乐工赏钱出门。四个唱的都在月娘房内弹唱,月娘留下吴银儿过夜,打发三个唱的去。临去,见西门庆在厅上,拜见拜见。西门庆分付郑爱月儿:“你明日就拉了李桂姐,两个还来唱一日。”郑爱月儿就知今日有王三官儿,不叫李桂姐来唱,笑道:“爹,你兵马司倒了墙——贼走了?”又问:“明日请谁吃酒?”西门庆道:“都是亲朋。”郑爱月儿道:“有应二那花子,我不来,我不要见那丑冤家怪物。”西门庆道:“明日没有他。”爱月儿道:“没有他才好。若有那怪攮刀子的,俺们不来。”说毕,磕了头去了。西门庆看着收了家伙,回到李瓶儿那边,和如意儿睡了。一宿晚景题过。

次日,早往衙门送问那两起人犯过东平府去。回来家中摆酒,请吴道官、吴二舅、花大舅、沈姨父、韩姨夫、任医官、温秀才、应伯爵,并会众人李智、黄四、杜三哥并家中三个伙计,十二张桌儿。席中止是李桂姐、吴银儿、郑爱月儿三个粉头递酒,李铭、吴惠、郑奉三个小优儿弹唱。正递酒中间,忽平安儿来报:“云二叔新袭了职,来拜爹,送礼来。”西门庆听言,忙道:“有请。”只见云理守穿着青纻丝补服员领,冠冕着,腰系金带,后面伴当抬着礼物,先递上揭贴,与西门庆观看。上写:“新袭职山东清河右卫指挥同知门下生云理守顿首百拜。谨具土仪:貂鼠十个,海鱼一尾,虾米一包,腊鹅四只,腊鸭十只,油低帘二架,少申芹敬。”西门庆即令左右收了,连忙致谢。云理守道:“在下昨日才来家,今日特来拜老爹。”于是四双八拜,说道:“蒙老爹莫大之恩,些少土仪,表意而已。”然后又与众人叙礼拜见。西门庆见他居官,就待他不同,安他与吴二舅一桌坐了,连忙安钟箸,下汤饭。脚下人俱打发攒盘酒肉。因问起发丧替职之事,这云理守一一数言:“蒙兵部余爷怜先兄在镇病亡,祖职不动,还与了个本卫见任佥书。”西门庆欢喜道:“恭喜恭喜,容日已定来贺。”当日众人席上每位奉陪一杯,又令三个唱的奉酒,须臾把云理守灌的醉了。那应伯爵在席上,如线儿提的一般,起来坐下,又与李桂姐、郑月儿彼此互相戏骂不绝。当日酒筵笑声,花攒锦簇,觥筹交错,耍顽至二更时分方才席散。打发三个唱的去了,西门庆归上房宿歇。

到次日起来迟,正在上房摆粥吃了,穿衣要拜云理守。只见玳安来说:“贲四在前边请爹说话。”西门庆就知为夏龙溪送家小之事,一面出来厅上。只见贲四向袖中取出夏指挥书来呈上,说道:“夏老爹要教小人送送家小往京里去,小人禀问老爹去不去?”西门庆看了书中言语,无非是叙其阔别,谢其早晚看顾家小,又借贲四携送家小之事,因说道:“他既央你,你怎的不去!”因问:“几时起身?”贲四道:“今早他大官儿叫了小人去,分付初六日家小准起身。小人也得半月才回来。”说毕,把狮子街铺内钥匙交递与西门庆。西门庆道:“你去,我教你吴二舅来,替你开两日罢。”那贲四方才拜辞出门,往家中收拾行装去了。西门庆就冠冕着出门,拜云指挥去了。

那日大妗子家去,叫下轿子门首伺候。也是合当有事,月娘装了两盒子茶食点心下饭,送出门首上轿。只见画童儿小厮躲在门房,大哭不止。那平安儿只顾扯他,那小厮越扯越哭起来。被月娘等听见,送出大妗子去了,便问平安儿:“贼囚,你平白扯他怎的?惹的他恁怪哭。”平安道:“温师父那边叫扯,他白不去,只是骂小的。”月娘道:“你教他好好去罢。”因问道:“小厮,你师父那边叫,去就是了,怎的哭起来?”那画童嚷平安道:“又不关你事,我不去罢了,你扯我怎的?”月娘道:“你因何不去?”那小厮又不言语。金莲道:“这贼小囚儿,就是个肉佞贼。你大娘问你,怎的不言语?被平安向前打了一个嘴巴,那小厮越发大哭了。月娘道:“怪囚根子,你平白打他怎的?你好好教他说,怎的不去?”正问着,只见玳安骑了马进来。月娘问道:“你爹来了?”玳安道:“被云二叔留住吃酒哩。使我送衣裳来了,要还毡巾去。”看见画童儿哭,便问:“小大官儿,怎的号啕痛也是的?”平安道:“对过温师父叫他不去,反哭骂起我来了。玳安道:“我的哥哥,温师父叫,你仔细,有名的温屁股,他一日没屁股也成不的。你每常怎么挨他的,今日又躲起来了?”月娘骂道:“怪囚根子,怎么温屁股?”玳安道:“娘只问他就是。”潘金莲得不的风儿就是雨儿,一面叫过画童儿来,只顾问他:“小奴才,你实说他叫你做甚么?你不说,看我教你大娘打你。”逼问那小厮急了,说道:“他只要哄着小的,把他那行货子放在小的屁股里,弄和胀胀的疼起来。我说你还不快拔出来,他又不肯拔,只顾来回动。且教小的拿出,跑过来,他又来叫小的。”月娘听了便喝道:“怪贼小奴才儿,还不与我过一边去!也有这六姐,只管审问他,说的碜死了。我不知道,还当是好话儿,侧着耳朵儿听他。这蛮子也是个不上芦帚的行货子,人家小厮与你使,却背地干这个营生。”金莲道:“大娘,那个上芦帚的肯干这营生,冷铺睡的花子才这般所为。”孟玉楼道:“这蛮子,他有老婆,怎生这等没廉耻?”金莲道:“他来了这一向,俺们就没见他老婆怎生样儿。”平安道:“娘每会胜也不看见他。他但往那边去就锁了门。住了这半年,我只见他会轿子往娘家去了一遭,没到晚就来家了。往常几时出个门儿来,只好晚夕门首倒杩子走走儿罢了。”金莲道:“他那老婆也是个不长俊的行货子,嫁了他,怕不的也没见个天日儿,敢每日只在屋里坐天牢哩。”说了回,月娘同众人回后边去了。

西门庆约莫日落时分来家,到上房坐下。月娘问道:“云伙计留你坐来?”西门庆道:“他在家,见我去,旋放桌儿留我坐,打开一坛酒和我吃。如今卫中荆南岗升了,他就挨着掌印。明日连他和乔亲家,就是两分贺礼,众同僚都说了,要与他挂轴子,少不得教温葵轩做两篇文章,买轴子写。”月娘道:“还缠甚么温葵轩、鸟葵轩哩!平白安扎恁样行货子,没廉耻,传出去教人家知道,把丑来出尽了。”西门庆听言,唬了一跳,便问:“怎么的?”月娘道:“你别要来问我,你问你家小厮去。”西门庆道:“是那个小厮?”金莲道:“情知是谁?画童贼小奴才,俺去送大妗子去,他正在门首哭,如此这般,温蛮子弄他来。”西门庆听了,还有些不信,便道:“你叫那小奴才来,等我问他。”一面使玳安儿前边把画童儿叫到上房,跪下,西门庆要拿拶子拶他,便道:“贼奴才,你实说,他叫你做甚么?”画童儿道:“他叫小的,要灌醉了小的,干那小营生儿。今日小的害疼,躲出来了,不敢去。他只顾使平安叫,又打小的,教娘出来看见了。他常时问爹家中各娘房里的事,小的不敢说。昨日爹家中摆酒,他又教唆小的偷银器家火与他。又某日他望倪师父去,拿爹的书稿儿与倪师父瞧,倪师父又与夏老爷瞧。”这西门庆不听便罢,听了便道:“画虎画皮难画骨,知人知面不知心。我把他当个人看,谁知他人皮包狗骨东西,要他何用?”一面喝令画童起去,分付:“再不消过那边去了。”那画童磕了头,起来往前边去了。西门庆向月娘道:“怪道前日翟亲家说我机事不密则害成,我想来没人,原来是他把我的事透泄与人,我怎的晓得?这样的狗骨秃东西,平白养在家做甚么?”月娘道:“你和谁说?你家又没孩子上学,平白招揽个人在家养活,只为写礼贴儿,饶养活着他,还教他弄乾坤儿。”西门庆道:“不消说了,明日教他走道儿就是了。”一面叫将平安来,分付:“对过对他说,家老爹要房子堆货,教温师父转寻房儿便了。等他来见我,你在门首,只回我不在家。”那平安儿应诺去了。

西门庆告月娘说:“今日贲四来辞我,初六日起身,与夏龙溪送家小往东京去。我想来,线铺子没人,倒好教二舅来替他开两日儿。好不好?”月娘道:“好不好,随你叫他去。我不管你,省的人又说照顾了我的兄弟。”西门庆不听,于是使棋童儿:“请你二舅来。”不一时,请吴二舅到,在前厅陪他吃酒坐的,把钥匙交付与他:“明日同来昭早往狮子街开铺子去。”不在话下。

却说温秀才见画童儿一夜不过来睡,心中省恐。到次日,平安走来说:“家老爹多上覆温师父,早晚要这房子堆货,教师父别寻房儿罢。”这温秀才听了,大惊失色,就知画童儿有甚话说,穿了衣巾,要见西门庆说话。平安道:“俺爹往衙门中去了,还未来哩。”比及来,这温秀才又衣巾过来伺候,具了一篇长柬,递与琴童儿。琴童又不敢接,说道:“俺爹才从衙门中回家,辛苦,后边歇去了,俺每不敢禀。”这温秀才就知疏远他,一面走到倪秀才家商议,还搬移家小往旧处住去了。正是:谁人汲得西江水,难洗今朝一面羞。

\[
靡不有初鲜克终,交情似水淡长浓。
自古人无千日好,果然花无摘下红。
\]

\newpage
%# -*- coding:utf-8 -*-
%%%%%%%%%%%%%%%%%%%%%%%%%%%%%%%%%%%%%%%%%%%%%%%%%%%%%%%%%%%%%%%%%%%%%%%%%%%%%%%%%%%%%


\chapter{西门庆踏雪访爱月\KG 贲四嫂带水战情郎}


词曰:

\[
梅其雪,岁暮斗新妆。月底素华同弄色,风前轻片半含香,不比柳花狂。双雀影,堪比雪衣娘。六出光中曾结伴,百花头上解寻芳,争似两鸳鸯。
\]

话说温秀才求见西门庆不得,自知惭愧,随移家小,搬过旧家去了。西门庆收拾书院,做了客坐,不在话下。

一日,尚举人来拜辞,上京会试,问西门庆借皮箱毡衫。西门庆陪坐待茶,因说起乔大户、云理守:“两位舍亲,一受义官,一受祖职,见任管事,欲求两篇轴文奉贺。不知老翁可有相知否?借重一言,学生具币礼相求。”尚举人笑道:“老翁何用礼,学生敝同窗聂两湖,见在武库肄业,与小儿为师,本领杂作极富。学生就与他说,老翁差盛使持轴来就是了。”西门庆连忙致谢。茶毕起身。西门庆随即封了两方手帕、五钱白金,差琴童送轴子并毡衫、皮箱,到尚举人处放下。那消两日,写成轴文差人送来。西门庆挂在壁上,但见金字辉粕,文不加点,心中大喜。只见应伯爵来问:“乔大户与云二哥的事,几时举行?轴文做了不曾?温老先儿怎的连日不见?”西门庆道:“又题什么温老先儿,通是个狗类之人!”如此这般,告诉一遍。伯爵道:“哥,我说此人言过其实,虚浮之甚,早时你有后眼,不然,教他调坏了咱家小儿每了。”又问他:“二公贺轴,何人写了?”西门庆道:“昨日尚小塘来拜我,说他朋友聂两湖善于词藻,央求聂两湖作了。文章已写了来,你瞧!”于是引伯爵到厅上观看,喝采不已,又说道:“人情都全了,哥,你早送与人家,好预备。”西门庆道:“明日好日期,早差人送去。”

正说着,忽报:“夏老爹儿来拜辞,说初六日起身去。小的回爹不在家。他说教对何老爹那里说声,差人那边看守去。”西门太看见贴儿上写着“寅家晚生夏承恩顿首拜,谢辞”。西门庆道:“连尚举人搭他家,就是两分程仪香绢。”分付琴童:“连忙买了,教你姐夫封了,写贴子送去。”正在书房中留伯爵吃饭,忽见平安儿慌慌张张拿进三个贴儿来报:“参议汪老爹、兵备雷老爹、郎中安老爹来拜。”西门庆看贴儿:“汪伯彦、雷启元、安忱拜。”连忙穿衣系带。伯爵道:“哥,你有事,我去罢。”西门庆道:“我明日会你哩。”一面整衣出迎。三官员皆相让而入。进入大厅,叙礼,道及向日叨扰之事。少顷茶罢,坐话间,安郎中便道:“雷东谷、汪少华并学生,又来干渎:有浙江本府赵大尹,新升大理寺正,学生三人借尊府奉请,已发柬,定初九日。主家共五席。戏子学生那里叫来。未知肯允诺否?”西门庆道:“老先生分付,学生扫门拱候。”安郎中令吏取分资三两递上,西门庆令左右收了,相送出门。雷东谷向西门庆道:“前日钱云野书到,说那孙文相乃是贵伙计,学生已并他除开了,曾来相告不曾?”西门庆道:“正是,多承老先生费心,容当叩拜。”雷兵备道:“你我相爱间,何为多数。”言毕,相揖上轿而去。

原来潘金莲自从当家管理银钱,另定了一把新等子。每日小厮买进菜蔬来,拿到跟前与他瞧过,方数钱与他。他又不数,只教春梅数钱,提等子。小厮被春鸿骂的狗血淋头,行动就说落,教西门庆打。以此众小厮互相抱怨,都说在三娘手儿里使钱好。

却说次日,西门庆衙门中散了,对何千户说:“夏龙溪家小已是起身去了,长官可曾委人那里看守门户去?”何千户道:“正是,昨日那边着人来说,学生已令小价去了。”西门庆道:“今日同长官那边看看去。”于是出衙门,并马到了夏家宅内。家小已是去尽了,伴当在门首伺候。两位官府下马,进到厅上。西门庆引着何千户前后观看了,又到前边花亭上,见一片空地,无甚花草。西门庆道:“长官到明日还收拾个耍子所在,栽些花柳,把这座亭子修理修理。”何千户道:“这个已定。学生开春从新修整修整,盖三间卷棚,早晚请长官来消闲散闷。”看了一回,分付家人收拾打扫,关闭门户。不日写书往东京回老公公话,赶年里搬取家眷。西门庆作别回家。何千户还归衙门去了。到次日才搬行李来住,不在言表。

西门庆刚到家下马,见何九买了一匹尺头、四样下饭、一坛酒来谢。又是刘内相差人送了一食盒蜡烛,二十张桌围,八十股官香,一盒沉速料香,一坛自造内酒,一口鲜猪。西门庆进门,刘公公家人就磕头,说道:“家公多多上履,这些微礼,与老爹赏人。”西门庆道:“前日空过老公公,怎又送这厚礼来?”便令左右:“快收了,请管家等等儿。”少顷,画童儿拿出一钟茶来,打发吃了。西门庆封了五钱银子赏钱,拿回贴,打发去了。一面请何九进去。西门庆见何九,一把手扯在厅上来。何九连忙倒身磕下头去,道:“多蒙老爹天心,超生小人兄弟,感恩不浅。”请西门庆受礼,西门庆不肯受磕头,拉起来,说道:“老九,你我旧人,快休如此。”就让他坐。何九说道:“小人微末之人,岂敢僭坐。”只说立在旁边。西门庆也站着,陪吃了一盏茶,说道:“老九,你如何又费心送礼来?我断然不受,若有甚么人欺负你,只顾来说,我替你出气。倘县中派你甚差事,我拿贴儿与你李老爹说。”何九道:“蒙老爹恩典,小人知道。小人如今也老了,差事已告与小人何钦顶替了。”西门庆道:“也罢,也罢,你清闲些好。”又说道:“既你不肯,我把这酒礼收了,那尺头你还拿去,我也不留你坐了。”那何九千恩万谢,拜辞去了。

西门庆就坐在厅上,看看打点礼物果盒、花红羊酒、轴文并各人分资。先差玳安送往乔大户家去,后叫王经送往云理守家去。玳安回来,乔家与了五钱银子。王经到云理守家,管待了茶食,与了一匹真青大布、一双琴鞋,回“门下辱爱生”双贴儿:“多上覆老爹,改日奉请。”西门庆满心欢喜,到后边月娘房中摆饭吃,因向月娘说:“贲四去了,吴二舅在狮子街卖货,我今日倒闲,往那里看看去。”月娘道:“你去不是,若是要酒菜儿,蚤使小厮来家说。”西门庆道:“我知道。”一面分付备马,就戴着毡忠靖巾,貂鼠暖耳,绿绒补子氅褶,粉底皂靴,琴童、玳安跟随,径往狮子街来。到房子内,吴二舅与来昭正挂着花拷拷儿,发买绸绢、绒线、丝绵,挤一铺子人做买卖,打发不开。西门庆下马,看了看,走到后边暖房内坐下。吴二舅走来作揖,因说:“一日也攒银二三十两。”西门庆又分付来昭妻一丈青:“二舅每日茶饭休要误了。”来昭妻道:“逐日伺候酒饭,不敢有误。”

西门庆见天色阴晦,彤云密布,冷气侵人,将有作雪的模样。忽然想起要往郑月儿家去,即令琴童:“骑马家中取我的皮袄来,问你大娘,有酒菜儿稍一盒与你二舅吃。”琴童应诺。到家,不一时,取了貂鼠皮袄,并一盒酒菜来。西门庆陪二舅在房中吃了三杯,分付:“二舅,你晚夕在此上宿,慢慢再用。我家去罢。”于是带上眼纱,骑马,玳安、琴童跟随,径进构栏,往郑爱月儿家来。转过东街口,只见天上纷纷扬扬,飘起一天瑞雪来。但见:

\[
漠漠严寒匝地,这雪儿下得正好。扯絮撏绵,裁成片片,大如拷拷。见林间竹笋茆茨,争些被他压倒。富豪侠却言:消灾障犹嫌少。围向那红炉兽炭,穿的是貂裘绣袄。手拈梅花,唱道是国家祥瑞,不念贫民些小。高卧有幽人,吟咏多诗草。
\]

西门庆踏着那乱琼碎玉,进入构栏,到于郑爱月儿家门首下马。只见丫鬟飞报进来,说:“老爹来了。”郑妈妈看见,出来,至于中堂见礼,说道:“前日多谢老爹重礼,姐儿又在宅内打搅,又教他大娘、三娘赏他花翠汗巾。”西门庆道:“那日空了他来。”一面坐下。西门庆令玳安:“把马牵进来,后边院落安放。”老妈道:“请爹后边明间坐罢。月姐才起来梳头,只说老爹昨日来,到伺候了一日,今日他心中有些不快,起来的迟些。”这西门庆一面进入他后边明间内,但见绿穿半启、毡幕低张,地平上黄铜大盆生着炭火。西门庆坐在正面椅上。先是郑爱香儿出来相见了,递了茶。然后爱月儿才出来,头挽一窝丝杭州缵,翠梅花钮儿,金趿钗梳,海獭卧兔儿。打扮的雾霭云鬟,粉妆玉琢。笑嘻嘻向西门庆道了万福,说道:“爹,我那一日来晚了。紧自前边散的迟,到后边,大娘又只顾不放俺每,留着吃饭,来家有三更天了。”西门庆笑道:“小油嘴儿,你倒和李桂姐两个把应花子打的好响瓜儿。”郑爱月儿道:“谁教他怪叨唠,在酒席上屎口儿伤俺每来!那一日祝麻子也醉了,哄我,要送俺每来。我便说:‘没爹这里灯笼送俺每,蒋胖子吊在阴沟里——缺臭了你了。’”西门庆道:“我昨日听见洪四儿说,祝麻子又会着王三官儿,大街上请了荣娇儿。”郑月儿道:“只在荣娇儿家歇了一夜,烧了一炷香,不去了。如今还在秦玉芝儿走着哩。”说了一回话,道:“爹,只怕你冷,往房里坐。”

这西门庆到于房中,脱去貂裘,和粉头围炉共坐,房中香气袭人。须臾,丫头拿了三瓯儿黄芽韭菜肉包、一寸大的水角儿来。姊妹二人陪西门庆,每人吃了一瓯儿。爱月儿又拨上半瓯儿,添与西门庆。西门庆道:“我勾了,才吃了两个点心来了。心里要来你这里走走,不想恰好天气又落下雪来了。”爱月儿道:“爹前日不会下我?我昨日等了一日不见爹,不想爹今日才来。”西门庆道:“昨日家中有两位士夫来望,乱着就不曾来得。”爱月儿道:“我要问爹,有貂鼠买个儿与我,我要做了围脖儿戴。”西门庆道:“不打紧,昨日韩伙计打辽东来,送了我几个好貂鼠。你娘们都没围脖儿,到明日一总做了,送两个一家一个。”于是爱香、爱月儿连忙起身道了万福。西门庆分付:“休见了桂姐、银姐说。”郑月儿道:“我知道。”因说:“前日李桂姐见吴银儿在那里过夜,问我他几时来的,我没瞒他,教我说:‘昨日请周爷,俺每四个都在这里唱了一日。爹说有王三官儿在这里,不好请你的。今日是亲朋会中人吃酒,才请你来唱。’他一声儿也没言语。”西门庆道:“你这个回的他好。前日李铭,我也不要他唱来,再三央及你应二爹来说。落后你三娘生日,桂姐买了一分礼来,再一与我陪不是。你娘们说着,我不理他。昨日我竟留下银姐,使他知道。”爱月儿道:“不知三娘生日,我失误了人情。”西门庆道:“明日你云老爹摆酒,你再和银姐来唱一日。”爱月儿道:“爹分付,我去。”说了回话,粉头取出三十二扇象牙牌来,和西门庆在炕毡条上抹牌顽耍。爱香儿也坐在旁边同抹。三人抹了回牌,须臾,摆上酒来,爱香与爱月儿一边一个捧酒,不免筝排雁柱,款跨鲛绡,姊妹两个弹唱。唱了一套,姐妹两个又拿上骰盆儿来,和西门庆抢红顽笑。杯来盏去,各添春色。西门庆忽看见郑爱月儿房中,床旁侧锦屏风上,挂着一轴《爱月美人图》,题诗一首:

\[
有美人兮迥出群,轻风斜拂石榴裙。
花开金谷春三月,月转花阴夜十分。
玉雪精神联仲琰,琼林才貌过文君。
少年情思应须慕,莫使无心托白云。
\]

西门庆看了,便问:“三泉主人是王三官儿的号?”慌的郑爱月儿连忙摭说道:“这还是他旧时写下的。他如今不号三泉了,号小轩了。他告人说,学爹说:‘我号四泉,他怎的号三泉?’他恐怕爹恼,因此改了号小轩。”一面走向前,取笔过来,把那“三”字就涂抹了。西门庆满心欢喜,说道:“我并不知他改号一节。”粉头道:“我听见他对一个人说来,我才晓的。说他去世的父亲号逸轩,他故此改号小轩。”说毕,郑爱香儿往下边去了,独有爱月儿陪西门庆在房内。两个并肩叠股,抢红饮酒,因说起林太太来,怎的大量,好风月:“我在他家吃酒,那日王三官请我到后边拜见。还是他主意,教三官拜认我做义父,教我受他礼,委托我指教他成人。”粉头拍手大笑道:“还亏我指与爹这条路儿,到明日,连三官儿娘子不怕不属了爹。”西门庆道:“我到明日,我先烧与他一炷香。到正月里,请他和三官娘子往我家看灯吃酒,看他去不去。”粉头道:“爹,你还不知三官娘子生的怎样标致,就是个灯人儿也没他那一段风流妖艳。今年十九岁儿,只在家中守寡,王三官儿通不着家。爹,你肯用些工夫儿,不愁不是你的人。”两个说话之间,相挨相凑。只见丫鬟又拿上许多细果碟儿来,粉头亲手奉与西门庆下酒。又用舌头噙凤香蜜饼送入他口中,又用纤手解开西门庆裤带,露出那话来,教他弄。那话狰狞跳脑,紫强光鲜,西门庆令他品之。这粉头真个低垂粉项,轻启朱唇,半吞半吐,或进或出,呜咂有声,品弄了一回。灵犀已透,淫心似火,便欲交欢。粉头便往后边去了。西门庆出房更衣,见雪越下得甚紧。回到房中,丫鬟向前打发脱靴解带,先上牙床。粉头澡牝回来,掩上双扉,共入鸳帐。正是:得多少动人春色娇还媚,惹蝶芳心软欲浓。有诗为证:

\[
聚散无凭在梦中,起来残烛映纱红。
钟情自古多神合,谁道阳台路不通。
\]

两个云雨欢娱,到一更时分起来。整衣理鬓,丫鬟复酾美酒,重整佳肴,又饮勾几杯。问玳安:“有灯笼、伞没有?”玳安道:“琴童家去取灯笼、伞来了。”这西门庆方才作别,鸨子、粉头相送出门,看着上马。郑月儿扬声叫道:“爹若叫我,蚤些来说。”西门庆道:“我知道。”一面上马,打着伞出院门,一路踏雪到家中。对着吴月娘,只说在狮子街和吴二舅饮酒,不在话下。一宿晚景题过。

到次日,却是初八日,打听何千户行李,都搬过夏家房子内去了,西门庆送了四盒细茶食、五钱折帕贺仪过去。只见应伯爵蓦地走来。西门庆见雪晴,风色甚冷,留他前边书房中向火,叫小厮拿菜儿,留他吃粥,因说道:“昨日乔亲家、云二哥礼并折帕,都送去了。你的人情,我也替你封了二钱出上了。你不消与他罢,只等发柬请吃酒。”应伯爵举手谢了,因问:“昨日安大人三位来做甚么?那两位是何人?”西门庆道:“那两个,一个是雷兵备,一个是汪参议,都是浙江人,要在我这里摆酒。明日请杭州赵霆知府,新升京堂大理寺丞,是他每本府父母官,相处分上,又不可回他的。通身只三两分资。”伯爵道:“大凡文职好细,三两银子勾做甚么!哥少不得赔些儿。”西门庆道:“这雷兵备,就是问黄四小舅子孙文相的,昨日还对我题起开除他罪名哩。”伯爵道:“你说他不仔细,如今还记着,折准摆这席酒才罢了。”

说话之间,伯爵叫:“应宝,你叫那个人来见你大爹。”西门庆便问:“是何人?”伯爵道:“一个小后生,倒也是旧人家出身。父母都没了,自幼在王皇亲宅内答应。已有了媳妇儿,因在庄子上和一般家人不和,出来了。如今闲着,做不的甚么。他与应宝是朋友,央及应宝要投个人家。今早应宝对我说:‘爹倒好举荐与大爹宅内答应。’我便说:‘不知你大爹用不用?’”因问应宝:“他叫甚么名字?你叫他进来。”应宝道:“他姓来,叫来友儿。”只见那来友儿,扒在地上磕了个头起来,帘外站立。伯爵道:“若论他这身材膂力尽有,掇轻负重却去的。”因问:“你多少年纪了?”来友儿道:“小的二十岁了。”又问:“你媳妇没子女?”那人道:“只光两口儿。”应宝道:“不瞒爹说,他媳妇才十九岁儿,厨灶针线,大小衣裳都会做。”西门庆见那人低头并足,为人朴实,便道:“既是你应二爹来说,用心在我这里答应。”分付:“拣个好日期,写纸文书,两口儿搬进来罢。”那来友儿磕了个头。西门庆就叫琴童儿领到后边,见月娘众人磕头去。月娘就把来旺儿原住的那一间房与他居住。伯爵坐了回,家去了。应宝同他写了一纸投身文书,交与西门庆收了,改名来爵,不在话下。

却说贲四娘子,自从他家长儿与了夏家,每日买东买西,只央及平安儿和来安、画童儿。西门庆家中这些大官儿,常在他屋里打平和儿吃酒。贲四娘子和气,就定出菜儿来,或要茶水,应手而至。就是贲四一时铺中归来撞见,亦不见怪。以此今日他不在家,使着那个不替他动?玳安儿与平安儿,在他屋里坐的更多。

初九日,西门庆与安郎中、汪参议、雷兵备摆酒,请赵知府,俱不必细说。那日蚤辰,来爵两口儿就搬进来。他媳妇儿后边见月娘众人磕头。月娘见他穿着紫绸袄,青布披袄,绿布裙子,生的五短身材,瓜子面皮儿,搽脂抹粉,缠的两只脚翘翘的,问起来,诸般针指都会做。取了他个名字,叫做惠元,与惠秀、惠祥一递三日上灶,不题。

一日,门外杨姑娘没了。安童儿来报丧。西门庆整治了一张插桌,三牲汤饭,又封了五两香仪。吴月娘、李娇儿、孟玉楼、潘金莲四顶轿子,都往北边与他烧纸吊孝,琴童儿、棋童儿、来爵儿、来安儿四个,都跟轿子,不在家。西门庆在对过段铺子书房内,看着毛袄匠与月娘做貂鼠围脖,先攒出一个围脖儿,使玳安送与院中郑月儿去,封了十两银子与他过节。郑家管待酒馔,与了他三钱银子。玳安走来,回西门庆话,说:“月姨多上覆,多谢了,前日空过了爹来。与了小的三钱银子。”西门庆道:“你收了罢。”因问他:“贲四不在家,你头里从他屋里出来做甚么?”玳安道:“贲四娘子从他女孩儿嫁了,没人使,常央及小的每替他买买甚么儿。”西门庆道:“他既没人使,你每替他勤勤儿也罢。”又悄悄向玳安道:“你慢慢和他说,如此这般,爹要来看你看儿,你心下如何?看他怎的说。他若肯了,你问他讨个汗巾儿来与我。”玳安道:“小的知道了。”领了西门庆言语,应诺下去。西门庆就走到家中来。只见王经向顾银铺内取了金赤虎,并四对金头银簪儿,交与西门庆。西门庆留下两对在书房内,余者袖进李瓶儿房内,与了如意儿那赤虎,又是一对簪儿。把那一对簪儿就与了迎春。二人接了,连忙磕头。西门庆就令迎春取饭去。须臾,拿饭来吃了,出来又到书房内坐下。只见玳安慢慢走到跟前,见王经在旁,不言语。西门庆使王经后边取茶去。那玳安方说:“小的将爹言语对他说了,他笑了。约会晚上些伺候,等爹进去。叫小的拿了这汗巾儿来。”西门庆见红绵纸儿,包着一方红绫织锦回纹汗巾儿,闻了闻喷鼻香,满心欢喜,连忙袖了。只见王经拿茶来,吃了,又走过对门,看匠人做生活去。

忽报:“花大舅来了。”西门庆道:“请过来这边坐。”花子繇走到书房暖阁儿里,作揖坐下。致谢外日相扰。叙话间,画童儿拿过茶来吃了。花子繇道:“门外一个客人,有五百包无锡米,冻了河,紧等要卖了回家去。我想着姐夫,倒好买下等价钱。”西门庆道:“我平白要他做甚么?冻河还没人要,到开河船来了,越发价钱跌了。如今家中也没银子。”即分付玳安:“收拾放桌儿,家中说,看菜儿来。”一面使画童儿:“请你应二爹来,陪你花爹坐。”不一时,伯爵来到。三人共在一处,围炉饮酒。又叫烙了两炷饼吃,良久,只见吴道官徒弟应春,送节礼疏诰来。西门庆请来同坐吃酒。就揽李瓶儿百日经,与他银子去。吃至日落时分,花子繇和应春二人先起身去了。次后甘伙计收了铺子,又请来坐,与伯爵掷骰猜枚谈话,不觉到掌灯已后。吴月娘众人轿子到了,来安走来回话。伯爵道:“嫂子们今日都往那里去来?”西门庆道:“杨姑娘没了,今日三日念经,我这里备了张祭卓,又封了香仪儿,都去吊问。”伯爵道:“他老人家也高寿了。”西门庆道:“敢也有七十五六。男花女花都没有,只靠侄儿那里养活,材儿也是我替他备下这几年了。”伯爵道:“好好,老人家有了黄金入柜,就是一场事了,哥的大阴骘。”说毕,酒过数巡,伯爵与甘伙计作辞去了。西门庆就起身走过来,分付后生王显:“仔细火烛。”王显道:“小的知道。”看着把门关上了。

这西门庆见没人,两天步就走入贲四家来。只见卉四娘子儿在门首独自站立已久,见对门关的门响,西门庆从黑影中走至跟前。这妇人连忙把封门一开,西门庆钻入里面。妇人还扯上封门,说道:“爹请里边纸门内坐罢。”原来里间槅扇厢着后半间,纸门内又有个小炕儿,笼着旺旺的火。桌上点着灯,两边护炕糊的雪白。妇人勒着翠蓝销金箍儿,上穿紫绸袄,青绡丝披袄,玉色绡裙子,向前与西门庆道了万福,连忙递了一盏茶与西门庆吃,因悄悄说:“只怕隔壁韩嫂儿知道。”西门庆道:“不妨事。黑影子里他那里晓的。”于是不由分说,把妇人搂到怀中就亲嘴。拉过枕头来,解衣按在炕沿子上,扛起腿来就耸。那话上已束着托子,刚插入牝中,就拽了几拽,妇人下边淫水直流,把一条蓝布裤子都湿了。西门庆拽出那话来,向顺袋内取出包儿颤声娇来,蘸了些在龟头上,攮进去,方才涩住淫津,肆行抽拽。妇人双手扳着西门庆肩膊,两厢迎凑,在下扬声颤语,呻吟不绝。这西门庆乘着酒兴,架起两腿在胳膊上,只顾没棱露脑,锐进长驱,肆行扇蹦,何止二三百度。须臾,弄的妇人云髻蓬松,舌尖冰冷,口不能言。西门庆则气喘吁吁,灵龟畅美,一泄如注。良久,拽出那话来,淫水随出,用帕搽之。两个整衣系带,复理残妆。西门庆向袖中掏出五六两一包碎银子,又是两对金头簪儿,递与妇人节间买花翠带。妇人拜谢了,悄悄打发出来。那边玳安在铺子里,专心只听这边门环儿响,便开大门,放西门庆进来。自知更无一人晓的。后次朝来暮往,也入港一二次。正是:若要人不知,除非己莫为。不想被韩嫂儿冷眼睃见,传的后边金莲知道了。这金莲亦不说破他。

一日,腊月十五日,乔大户家请吃酒。西门庆会同应伯爵、吴大舅一齐起身。那日有许多亲朋看戏饮酒,至二更方散。第二日,每家一张卓面,俱不必细说。

单表崔本治了二千两湖州绸绢货物,腊月初旬起身,雇船装载,赶至临清马头。教后生荣海看守货物,便雇头口来家,取车锐银两,到门首下头口。琴童道:“崔大哥来了,请厅上坐。爹在对门房子里,等我请去。”一面走到对门,不见西门庆,因问平安儿,平安儿道:“爹敢进后边去了。”这琴童走到上房问月娘,月娘道:“见鬼的,你爹从蚤辰出去,再几时进来?”又到各房里,并花园、书房都瞧遍了,没有。琴童在大门首扬声道:“省恐杀人,不知爹往那里去了,白寻不着!大白日里把爹来不见了。崔大哥来了这一日,只顾教他坐着。”那玳安分明知道,只不做声。不想西门庆忽从前边进来,把众人唬了一惊。原来西门庆在贲四屋里入港,才出来。那平安打发西门庆进去了,望着琴童儿吐舌头,都替他捏两把汗道:“管情崔大哥去了,有几下子打。”不想西门庆走到厅上,崔本见了,磕头毕,交了书帐,说:“船到马头,少车税银两。我从腊月初一日起身,在扬州与他两个分路。他每往杭州去了,俺每都到苗青家住了两日。”因说:“苗青替老爹使了十两银子,抬了扬州卫一个千户家女子,十六岁了,名唤楚云。说不尽生的花如脸,玉如肌,星如眼,月如眉,腰如柳,袜如钩,两只脚儿,恰刚三寸。端的有沉鱼落雁之容,闭月羞花之豹。腹中有三千小曲,八百大曲。苗青如此还养在家,替他打妆奁,治衣服。待开春,韩伙计、保官儿船上带来,伏侍老爹,消愁解闷。”西门庆听了,满心欢喜,说道:“你船上稍了来也罢。又费烦他治甚衣服,打甚妆砹,愁我家没有?”于是恨不的腾云展翅,飞上扬州,搬取娇姿,赏心乐事。正是:鹿分郑相应难辨,蝶化庄周未可。有诗为证:

\[
闻道扬州一楚云,偶凭青鸟语来真。
不知好物都离隔,试把梅花问主人。
\]

西门庆陪崔本吃了饭,兑了五十两银子做车税钱,又写书与钱主事,烦他青目。崔本言讫,作辞,往乔大户家回话去了。平安见西门庆不寻琴童儿,都说:“我儿,你不知有多少造化。爹今日不知有甚事喜欢,若不是,绑着鬼有几下打。”琴童笑道:“只你知爹性儿。”

比及起了货,来到狮子街卸下,就是下旬时分。西门庆正在家打发送节礼,忽见荆都监差人拿贴儿来,问:“宋大巡题本已上京数日,未知旨意下来不曾?伏惟老翁差人察院衙门一打听为妙。”西门庆即差答应节级,拿了五钱银子,往巡按公衙打听。果然昨日东京邸报下来,写抄得一纸,全报来与西门庆观看。上面写着:

\[
山东巡按监察御史宋乔年一本:循例举劾地方文武官员,以励人心,以隆圣治事。窃惟吏以抚民,武以御乱,所以保障地方,以司民命者也。苟非其人,则处置乖方,民受其害,国何赖焉!臣奉命按临山东等处,吏政民瘼,监司守御,无不留心咨访。复命按抚大臣,详加鉴别,各官贤否,颇得其实。兹当差满之期,敢不一一陈之。访得山东左布政陈四箴操履忠贞,抚民有方;廉使赵讷,纲纪肃清,士民服习;兵备副使雷启元,军民咸服其恩威,僚幕悉推其练达;济南府知府张叔夜,经济可观,才堪司牧;东平府知府胡师父,居任清慎,视民如伤。此数臣者,皆当荐奖而优擢者也。又访得左参议冯廷鹄,伛偻之形,桑榆之景,形若木偶,尚肆贪婪;东昌府知府徐松,纵父妾而通贿,毁谤腾于公堂,慕羡余而诛求,詈言遍于间里。此二臣者,所当亟赐置斥者也。再访得左军院佥书守备周秀,器宇恢弘,操持老练,军心允服,贼盗潜消;济州兵马都监荆忠,年力精强,才犹练达,冠武科而称为儒将,胜算可以临戎,号令而极其严明,长策卒能御侮。此二臣者,所当亟赐迁擢者也。清河县千户吴铠,以练达之才,得卫守之法,驱兵以\textShouShou 中坚,靡攻不克;储食以资粮饷,无人不饱。推心置腹,人思效命。实一方之保障,为国家之屏藩。宜特加超擢,鼓舞臣寮。陛下如以臣言可采,举而行之,庶几官爵不滥而人思奋,守牧得人而圣治有赖矣。等因。
奉饮依:该部知道。续该吏、兵二部题前事:看得御史宋乔年所奏内,劾举地方文武官员,无非体国之忠,出于公论,询访事实,以裨圣治之事。优乞圣明俯赐施行,天下幸甚,生民幸甚。奉钦依:拟行。
\]

西门庆一见,满心欢喜。拿着邸报,走到后边,对月娘说:“宋道长本下来了。已是保举你哥升指挥佥事,见任管屯。周守备与荆大人都有奖励,转副参、统制之任。如今快使小厮请他来,对他说声。”月娘道:“你使人请去,我交丫鬟看下酒菜儿。我愁他这一上任,也要银子使。”西门庆道:“不打紧,我借与他几两银子也罢了。”不一时,请得吴大舅到了。西门庆送那题奏旨意与他瞧。吴大舅连忙拜谢西门庆与月娘,说道:“多累姐夫、姐姐扶持,恩当重报,不敢有忘。”西门庆道:“大舅,你若上任摆酒没银子,我这里兑些去使。”那大舅又作揖谢了。于是就在月娘房中,安排上酒来吃酒。月娘也在旁边陪坐。西门庆即令陈敬济把全抄写了一本,与大舅拿着。即差玳安拿贴送邸报往荆都监、周守御两家报喜去。正是:

\[
劝君不费镌研石,路上行人口似碑。
\]

\newpage
%# -*- coding:utf-8 -*-
%%%%%%%%%%%%%%%%%%%%%%%%%%%%%%%%%%%%%%%%%%%%%%%%%%%%%%%%%%%%%%%%%%%%%%%%%%%%%%%%%%%%%


\chapter{林太太鸳帏再战\KG 如意儿茎露独尝}


词曰:

\[
凤髻金泥带,龙纹玉掌梳。去来窗下笑来扶,爱道画眉深浅入时无?弄笔偎人久,描花试手初。等闲含笑问狂夫,笑问欢情不减旧时么?
\]

话说西门庆陪大舅饮酒,至晚回家。到次日,荆都监早辰骑马来拜谢,说道:“昨日见旨意下来,下官不胜欢喜,足见老翁爱厚,费心之至,实为衔结难忘。”说毕,茶汤两换,荆都监起身,因问:“云大人到几时请俺们吃酒?”西门庆道:“近节这两日也是请不成,直到正月间罢了。”送至大门,上马而去。西门庆宰了一口鲜猪,两坛浙江酒,一匹大红绒金豸员领,一匹黑青妆花纻丝员领,一百果馅金饼,谢宋御史。就差春鸿拿贴儿,送到察院去。门吏人报进去,宋御史唤至后厅火房内,赏茶吃。等写了回帖,又赏了春鸿三钱银子。来见西门庆,拆开观看,上写着:

\[
两次造扰华府,悚愧殊甚。今又辱承厚贶,何以克当?外令亲荆子事,已具本矣,相已知悉。连日渴仰丰标,容当面悉。使旋谨谢。\named{侍生宋乔年拜大锦衣西门先生大人门下}
\]
宋御史随即差人,送了一百本历日,四万纸,一口猪来回礼。

一日,上司行下文书来,令吴大舅本卫到任管事。西门庆拜去,就与吴大舅三十两银子,四匹京段,交他上下使用。到二十四日,封了印来家,又备羊酒花红轴文,邀请亲朋,等吴大舅从卫中上任回来,迎接到家,摆大酒席与他作贺。又是何千户东京家眷到了,西门庆写月娘名字,送茶过去。到二十六日,玉皇庙吴道官十二个道众,在家与李瓶儿念百日经,整做法事,大吹大打,各亲朋都来送茶,请吃斋供,至晚方散,俱不在言表。

至廿七日,西门庆打发各家送礼,应伯爵、谢希大、常峙节、傅伙计、甘伙计、韩道国、贲第传、崔本,每家半口猪,半腔羊,一坛酒,二包米,一两银子,院中李桂姐、吴银儿、郑爱月儿,每人一套衣服,三两银子。吴月娘又与庵里薛姑子打斋,令来安儿送香油、米面、银钱去,不在言表。看看到年除之日,穿梅表月,檐雪滚风,竹爆千门万户,家家贴春胜,处处挑桃符。西门庆烧了纸,又到于李瓶儿房,灵前祭奠。祭毕,置酒于后堂,合家大小欢乐。手下家人小厮并丫头媳妇,都来磕头。西门庆与吴月娘,俱有手帕、汗巾、银钱赏赐。

到次日,重和元年新正月元旦,西门庆早起冠冕,穿大红,天地上烧了纸,吃了点心,备马就拜巡按贺节去了。月娘与众妇人早起来,施朱傅粉,插花插翠,锦裙绣袄,罗袜弓鞋,妆点妖娆,打扮可喜,都来月娘房里行礼。那平安儿与该日节级在门首接拜贴,上门簿,答应往来官长士夫。玳安与王经穿着新衣裳,新靴新帽,在门首踢毽子,放炮仗,磕瓜子儿。众伙计主管,伺候见节者,不计其数,都是陈敬济一人管待。约晌午,西门庆往府县拜了人回来,刚下马,招宣府王三官儿衣巾着来拜。到厅上拜了西门庆四双八拜,然后请吴月娘见。西门庆请到后边,与月娘见了,出来前厅留坐。才拿起酒来吃了一盏,只见何千户来拜。西门庆就叫陈敬济管待陪王三官儿,他便往卷棚内陪何千户坐去了。王三官吃了一回,告辞起身。陈敬济送出大门,上马而去。落后又是荆都监、云指挥、乔大户,皆络绎而至。西门庆待了一日人,已酒带半酣,至晚打发人去了,回到上房歇了一夜。到次早,又出去贺节,至晚归来,家中已有韩姨夫、应伯爵、谢希大、常峙节、花子繇来拜。陈敬济陪在厅上坐的。西门庆到了,见毕礼,重新摆上酒来饮酒。韩姨夫与花子繇隔门,先去了。剩下伯爵、希大、常峙节,坐个定光油儿不去。又撞见吴二舅来了,见了礼,又往后边拜见月娘,出来一处坐的。直吃到掌灯已后方散。

西门庆已吃的酩酊大醉,送出伯爵,等到门首众人去了。西门庆见玳安在旁站立,捏了一把手。玳安就知其意,说道:“他屋里没人。”这西门庆就撞入他房内。老婆早已在门里迎接进去。两个也无闲话,走到里间,脱衣解带就干起来。原来老婆好并着腿干,两只手扇着,只教西门庆攮他心子。那浪水热热一阵流出来,把床褥皆湿。西门庆龟头蘸了药,攮进去,两手扳着腰,只顾揉搓,麈柄尽入至根,不容毫发,妇人瞪目,口中只叫“亲爷。”那西门庆问他:“你小名叫甚么?说与我。”老婆道:“奴娘家姓叶,排行五姐。”西门庆口中喃喃呐呐,就叫叶“五儿”不绝。那老婆原是奶子出身,与贲四私通,被拐出来,占为妻子。今年三十二岁,甚么事儿不知道!口里如流水连叫“亲爷”不绝,情浓一泄如注。西门庆扯出麈柄要抹,妇人拦住:“休抹,等淫妇下去,替你吮净了罢。”西门庆满心欢喜,妇人真个蹲下身子,双手捧定那话,吮咂得干干净净,才系上裤子。因问西门庆:“他怎的去恁些时不来?”西门庆道:“我这里也盼他哩。只怕京中你夏老爹留住他使。”又与了老婆二、三两银子盘缠,因说:“我待与你一套衣服,恐贲四知道不好意思。不如与你些银子儿,你自家治买罢。”开门送出来。玳安又早在铺子里掩门等候。西门庆便往后边去了。

看官听说,自古上梁不正则下梁歪,原来贲四老婆先与玳安有奸,这玳安刚打发西门庆进去了,因傅伙计又没在铺子里上宿,他与平安儿打了两大壶酒,就在老婆屋里吃到有二更时分,平安在铺子里歇了,他就和老婆在屋里睡了一宿。有这等的事!正是:

\[
满眼风流满眼迷,残花何事滥如泥?
拾琴暂息商陵操,惹得山禽绕树啼。
\]

却说贲四老婆晚夕同玳安睡了,因对他说:“我一时依了爹,只怕隔壁韩嫂儿传嚷的后边知道,也似韩伙计娘子,一时被你娘们说上几句,羞人答答的,怎好相见?”玳安道:“如今家中,除了俺大娘和五娘不言语,别的不打紧。俺大娘倒也罢了,只是五娘快出尖儿。你依我,节间买些甚么儿,进去孝顺俺大娘。别的不稀罕,他平昔好吃蒸酥,你买一钱银子果馅蒸酥、一盒好大壮瓜子送进去达初九日是俺五娘生日,你再送些礼去,梯己再送一盒瓜子与俺五娘。管情就掩住许多口嘴。”这贲四老婆真个依着玳安之言,第二日赶西门庆不在家,玳安就替他买了盒子,掇进月娘房中。月娘便道:“是那里的?”玳安道:“是贲四嫂子送与娘吃的。”月娘道:“他男子汉又不在家,那讨个钱来,又交他费心。”连忙收了,又回出一盒馒头,一盒果子,说:“上覆他,多谢了。”

那日西门庆拜人回家,早又玉皇庙吴道官来拜,在厅上留坐吃酒。刚打发吴道官去了,西门庆脱了衣服,使玳安:“你骑了马,问声文嫂儿去:‘俺爹今日要来拜拜太太。’看他怎的说?”玳安道:“爹,不消去,头里文嫂儿骑着驴子打门首过去了。他说明日初四,王三官儿起身往东京,与六黄公公磕头去了。太太说,交爷初六日过去见节,他那里伺候。”西门庆便道:“他真个这等说来?”玳安道:“莫不小的敢说谎!”这西门庆就入后边去了。

刚到上房坐下,忽来安儿来报:“大舅来了。”只见吴大舅冠冕着,束着金带,进入后堂,先拜西门庆,说道:“我吴铠多蒙姐夫抬举看顾,又破费姐夫,多谢厚礼。昨日姐夫下降,我又不在家,失迎。今日敬来与姐夫磕个头儿,恕我迟慢之罪。”说着,磕下头去。西门庆慌忙顶头相还,说道:“大舅恭喜,至亲何必计较。”拜毕,月娘出来与他哥磕头。慌的大舅忙还半礼,说道:“姐姐,两礼儿罢,哥哥嫂嫂不识好歹,常来扰害你两口儿。你哥老了,看顾看顾罢。”月娘道:“一时有不到处,望哥耽带便了。”吴大舅道:“姐姐没的说,累你两口儿还少哩?”拜毕,西门庆留吴大舅坐,说道:“这咱晚了,料大舅也不拜人了,宽了衣裳,咱房里坐罢。”不想孟玉楼与潘金莲两个都在屋里,听见嚷吴大舅进来,连忙走出来,与大舅磕头。磕了头,径往各人房里去了。西门庆让大舅房内坐的,骑火盆安放桌儿,摆上菜儿来。小玉、玉箫都来与大舅磕头。月娘用小金镶钟儿,斟酒递与大舅,西门庆主位相陪。吴大舅让道:“姐姐你也来坐的。”月娘道:“我就来。”又往里间房内,拿出数样配酒的果菜来。饮酒之间,西门庆便问:“大舅的公事都停当了?”吴大舅道:“蒙姐夫抬举,卫中任便到了,上下人事,倒也都周给的七八。只有屯所里未曾去到到任。胆日是个好日期,卫中开了印,来家整理些盒子,须得抬到屯所里到任,行牌拘将那屯头来参见,分付分付。前官丁大人坏了事情,已被巡扶侯爷参劾去了。如今我接管承行,须要振刷在册花户,警励屯头,务要把这旧管新增开报明白,到明日秋粮夏税,才好下屯征收。”西门庆道:“通共约有多少屯田?”吴大舅道:“太祖旧例,为养兵省转输之劳,才立下这屯田。那时只是上纳秋粮,后吃宰相王安石立青苗法,增上这夏税。而今济州管内,除了抛荒、苇场、港隘,通共二万七千顷屯地。每顷秋税夏税只征收一两八钱,不上五百两银子。到年终总倾销了,往东平府交纳,转行招商,以备军粮马草作用。”西门庆又问:“还有羡余之利?”吴大舅道:“虽故还有些抛零人户不在册者,乡民顽滑,若十分征紧了,等秤斛斗量,恐声口致起公论。”西门庆道:“若是多寡有些儿也罢,难道说全征?”吴大舅道:“不瞒姐夫说,若会管此屯,见一年也有百十两银子。到年终,人户们还有些鸡鹅豕米相送,那个是各人取觅,不在数内的。只是多赖姐夫力量扶持。”西门庆道:“得勾你老人家搅给,也尽我一点之心。”说了回,月娘也走来旁边陪坐,三人饮酒。到掌灯已后,吴大舅才起身去了。西门庆就在金莲房中歇了一夜。到次日早往衙门中开印,升厅画卯,发放公事。先是云理守家发贴儿,初五日请西门庆并合卫官员吃庆官酒。次日,何千户娘子蓝氏下贴儿,初六日请月娘姊妹相会。

且说那日西门庆同应伯爵、吴大舅三人起身到云理守家。原来旁边又典了人家一所房子,三间客位内摆酒,叫了一起吹打鼓乐迎接,都有桌面,吃至晚夕来家。巴不到次日,月娘往何千户家吃酒去了。西门庆打选衣帽齐整,骑马带眼纱,玳安、琴童跟随,午后时分,径来王招宣府中拜节。王三官儿不在,送进贴儿去。文嫂儿又早在那里,接了贴儿,连忙报与林太太说,出来,请老爷后边坐。转过大厅,到于后边,掀起明帘,只见里边氍毹匝地,帘幕垂红。少顷,林氏穿着大红通袖袍儿,珠翠盈头,与西门庆见毕礼数,留坐待茶,分付:“大官,把马牵于后槽喂养。”茶罢,让西门庆宽衣房内坐,说道:“小儿从初四日往东京与他叔岳父六黄太尉磕头去了,只过了元宵才来。”西门庆一面唤玳安,脱去上盖,里边穿着白绫袄子,天青飞鱼氅衣,十分绰耀。妇人房里安放桌席。须臾,丫鬟拿酒菜上来,杯盘罗列,肴馔堆盈,酒泛金波,茶烹玉蕊。妇人玉手传杯,秋波送意,猜枚掷骰,笑语烘春。话良久,意洽情浓;饮多时,目邪心荡。看看日落黄昏,又早高烧银烛。玳安、琴童自有文嫂儿管待,等闲不过这边来。妇人又倒扣角门,僮仆谁敢擅入。酒酣之际,两人共入里间房内,掀开绣帐,关上窗户,轻剔银缸,忙掩朱户。男子则解衣就寝,妇人即洗牝上床,枕设宝花,被翻红浪。原来西门庆带了淫器包儿来,安心要鏖战这婆娘,早把胡僧药用酒吃在腹中,那话上使着双托子,在被窝中,架起妇人两股,纵麈柄入牝中,举腰展力,一阵掀腾鼓捣,连声响亮。妇人在下,没口叫亲达达如流水。正是:

\[
招海旌幢秋色里,击天鼙鼓月明中。
\]
但见:

\[
迷魂阵罢,摄魄旗开。迷魂阵上,闪出一员酒金刚,色魔王能争惯战;摄魂旗下,拥一个粉骷髅,花狐狸百媚千娇。这阵上,扑冬冬,鼓震春雷;那阵上,闹挨挨,麝兰叆叇。这阵上,复溶溶,被翻红浪精神健;那阵上,刷剌剌,帐控银钩情意乖。这一个急展展,二十四解任徘徊;那一个忽剌剌,一十八滚难挣扎。斗良久,汗浸浸,钗横鬓乱;战多时,喘吁吁,枕侧衾歪。顷刻间,肿眉\textuni{268D8}眼;霎时下,肉绽皮开。正是:几番鏖战贪淫妇,不是今番这一遭。
\]

当下西门庆就在这婆娘心口与阴户烧了两炷香,许下胆日家中摆酒,使人请他同三官儿娘子去看灯耍子。这妇人一段身心已被他拴缚定了,于是满口应承都去。西门庆满心欢喜,起来与他留连痛饮,至二更时分,把马从后门牵出,作别回家。正是:

\[
尽日思君倚画楼,相逢不舍又频留。
刘郎莫谓桃花老,浪把轻红逐水流。
\]

西门庆到家,有平安拦门禀说:“今日有薛公公家差人送请贴儿,请爹早往门外皇庄看春。又是云二叔家送了五个贴儿,请五位娘吃节酒。”西门庆听了,进入月娘房来。只见孟玉楼、潘金莲都在房内坐的。月娘从何千户家赴了席来家,正坐着说话。见西门庆进来,连忙道了万福。因问:“你今日往那里,这咱才来?”西门庆没得说,只说:“我在应二哥家留坐。”月娘便说起今日何千户家酒席上事:“原来何千户娘子年还小哩,今年才十八岁,生的灯上人儿也似,一表人物,好标致,知今博古,见我去,恰似会了几遍,好不喜洽。嫁了何大人二年光景,房里到使着四个丫头,两个养娘,两房家人媳妇。”西门庆道:“他是内府生活所蓝太监侄女儿,嫁与他陪了好少钱儿!”月娘道:“明日云伙计家,又请俺每吃节酒,送了五个贴儿业,端的去不去?”西门庆说:“他既请你每,都去走走罢。”月娘道:“留雪姐在家罢,只怕大节下,一时有个人客闯将来,他每没处挝挠。”西门庆道:“也罢,留雪姐在家里,你每四个去罢。明日薛太监请我看春,我也懒待去。这两日春气发也怎的,只害这腰腿疼。”月娘道:“你腰腿疼只怕是痰火,问任医官讨两服药吃不是,只顾挨着怎的?”西门庆道:“不妨事,由他。一发过了这两日吃,心净些。”因和月娘计较:“到明日灯节,咱少不的置席酒儿,请请何大人娘子。连周守备娘子,荆南岗娘子,张亲家母,云二哥娘子,连王三官儿母亲,和大妗子、崔亲家母,这几位都会会。也只在十二三,挂起灯来。还叫王皇亲家那起小厮扮戏耍一日。去年还有贲四在家,扎几架烟火放,今年他东京去了,只顾不见来,却教谁人看着扎?”那金莲在旁插口道:“贲四去了,他娘子儿扎也是一般。”这西门庆就瞅了金莲道:“这个小淫妇儿,三句话就说下道儿去了。”那月娘、玉楼也不采顾,就罢了。因说道:“那王官儿娘,咱每与他没会过,人生面不熟,怎么好请他?只怕他也不肯来。”西门庆道:“他既认我做亲,咱送个贴儿与他,来不来,随他就是了。”月娘又道:“我明日不往云家去罢,怀着个临月身子,只管往人家撞来撞去的,交人家唇齿。”玉楼道:“怕怎的,你身子怀的又不显,怕还不是这个月的孩子,不妨事。大节下自恁散心,去走走儿才好。”说毕,西门庆吃了茶,就往后边孙雪娥房里去了。那潘金莲见他往雪娥房中去,叫了大姐,也就往前边去了。西门庆到于雪娥房中,交他打腿捏身上,捏了半夜。一宿晚景题过。

到次日早辰,只见应伯爵走来,对西门庆说:“昨日云二嫂送了个贴儿,今日请房下陪众嫂子坐。家中旧时有几件衣服儿,都倒塌了。大正月不穿件好衣服,惹的人家笑话。敢来上覆嫂子,有上盖衣服,借约两套儿,头面簪环,借约几件儿,交他穿戴了去。”西门庆令王经:“你里边对你大娘说去。”伯爵道:“应宝在外边拿着毡包并盒儿哩。哥哥,累你拿进去,就包出来罢。”那王经接毡包进去,良久抱出来,交与应宝,说道:“里面两套上色段子织金衣服,大小五件头面,一双环儿。”应宝接的去了。西门庆陪伯爵吃茶,说道:“今日薛内相又请我门外看春,怎么得工夫去?吴亲家庙里又送贴儿,初九日年例打醮,也是去不成,教小婿去罢了。这两日不知酒多了也怎的,只害腰疼,懒待动旦。”伯爵道:“哥,你还是酒之过,湿痰流注在这下部,也还该忌忌。”西门庆道:“这节间到人家,谁肯轻放了你,怎么忌的住?”

正说着,只见玳安拿进盒儿来,说道:“何老爹家差人送请贴儿来,初九日请吃节酒。”西门庆道:“早是你看着,人家来请,你怎不去?”于是看盒儿内,放着三个请贴儿,一个双红佥儿,写着“大寅丈四泉翁老先生大人”,一个写“大都阃吴老先生大人”,一个写着“大乡望应老先生大人”,俱是“侍教生何永寿顿首拜”。玳安说:“他说不认的,教咱这里转送送儿去。”伯爵一见便说:“这个却怎样儿的?我还没送礼儿去与他,怎好去?”西门庆道:“我这里替你封上分帕礼儿,你差应宝早送去就是了。”一面令王经:“你封二钱银子,一方手帕,写你应二爹名字,与你应二爹。”因说:“你把这请贴儿袖了去,省的我又教人送。”只把吴大舅的差来安儿送去了。须臾,王经封了帕礼递与伯爵。伯爵打恭说道:“又多谢哥,我后日早来会你,咱一同起身。”说毕,作辞去了。午间,吴月娘等打扮停当,一顶大轿,三顶小轿,后面又带着来爵媳妇儿惠元,收叠衣服,一顶小轿儿,四名排军喝道,琴童、春鸿、棋童、来安四个跟随,往云指挥家来吃酒。正是:

\[
翠眉云鬓画中人,袅娜宫腰迥出尘。
天上嫦娥元有种,娇羞酿出十分春。
\]

不说月娘众人吃酒去了。且说西门庆分付大门上平安儿:“随问甚么人,只说我不在。有贴儿接了就是了。”那平安经过一遭,那里再敢离了左右,只在门首坐的。但有人客来望,只回不在家。西门庆因害腿疼,猛然想起任医官与他延寿丹,用人乳吃。于是来到李瓶儿房中,叫迎春拿菜儿,筛酒来吃。迎春打发了,就走过隔壁,和春梅下棋去了。要茶要水,自有如意儿打发。西门庆见丫鬟不在屋里,就在炕上斜靠着。露出那话,带着银托子,教他用口吮咂。一面斟酒自饮,因呼道:“章四儿,我的儿,你用心替达达咂,我到明日,寻出件好妆花段子比甲儿来,你正月十二日穿。”老婆道:“看他可怜见。”咂弄勾一顿饭时,西门庆道:“我儿,我心里要在你身上烧炷香儿。”老婆道:“随爹拣着烧。”西门庆令他关上房门,把裙子脱了,仰卧在炕上。西门庆袖内还有烧林氏剩下的三个烧酒浸的香马儿,撇去他抹胸儿,一个坐在他心口内,一个坐在他小肚儿底下,一个安在他盖子上,用安息香一齐点着,那话下边便插进牝中,低着头看着拽,只顾没棱露脑,往来迭进不已。又取过镜台来旁边照看,须臾,那香烧到肉根前,妇人蹙眉啮齿,忍其疼痛,口里颤声柔语,哼成一块,没口子叫:“达达,爹爹,罢了我了,好难忍他。”西门庆便叫道:“章四淫妇儿,你是谁的老婆?”妇人道:“我是爹的老婆。”西门庆教与他:“你说是熊旺的老婆,今日属了我的亲达达了。”那妇人回应道:“淫妇原是熊旺的老婆,今日属了我的亲达达了。”西门庆又问道:“我会\textuni{34B2}不会?”妇人道:“达达会\textuni{34B2}。”两个淫声艳语,无般言语不说出来。西门庆那话粗大,撑得妇人牝中满满,往来出入,带的花心红如鹦鹉舌,黑似蝙蝠翅,翻复可爱。西门庆于是把他两股扳拘在怀内,四体交匝,两厢迎凑,那话尽没至根,不容毫发,妇人瞪目失声,淫水流下,西门庆情浓乐极,精邈如泉涌。正是:

\[
不知已透春消息,但觉形骸骨节熔。
\]

西门庆烧了老婆身上三处春,开门寻了一件玄色段子妆花比甲儿与他。至晚,月娘众人来家,对西门庆说:“原来云二嫂也怀着个大身子,俺两今日酒席上都递了酒,说过,到明日两家若分娩了,若是一男一女,两家结亲做亲家;若都是男子,同堂攻书;若是女儿,拜做姐妹,一处做针指,来往亲戚耍子。应二嫂做保证。”西门庆听的笑了。

话休饶舌。到第二日,却是潘金莲上寿。西门庆早起往衙门中去了,分付小厮每抬出灯来,收拾揩抹干净,各处张挂。叫来兴买鲜果,叫小优晚夕上寿。潘金莲早辰打扮出来,花妆粉抹,翠袖朱唇,走来大厅上。看见玳安与琴童站在高凳上挂灯,因笑嘻嘻说道:“我道是谁在这里,原来是你每挂灯哩。”琴童道:“今日是五娘上寿,爹分付叫俺每挂了灯,明日娘生日好摆酒。晚夕小的每与娘磕头,娘已定赏俺每哩。”妇人道:“要打便有,要赏可没有。”琴童道:“耶嚛,娘怎的没打不说话,行动只把打放在头里,小的每是娘的儿女,娘看顾看顾儿便好,如何只说打起来。”妇人道:“贼囚,别要说嘴,你好生仔细挂那灯,没的例儿扯儿的,拿不牢吊将下来。前日年里,为崔本来,说你爹大白里不见了,险了险赦了一顿打,没曾打,这遭儿可打的成了。”琴童道:“娘只说破话,小的命儿薄薄的,又唬小的。”玳安道:“娘也会打听,这个话儿娘怎得知?”妇人道:“宫外有株松,宫内有口钟。钟的声儿,树的影儿,我怎么有个不知道的?昨日可是你爹对你大娘说,去年有贲四在家,还扎了几架烟火放,今年他不在家,就没人会扎。吃我说了两句:‘他不在家,左右有他老婆会扎,教他扎不是!’”玳安道:“娘说的甚么话,一个伙计家,那里有此事!”妇人道:“甚么话?檀木靶,有此事,真个的。画一道儿,只怕\textuni{34B2}过界儿去了。”琴童道:“娘也休听人说,只怕贲四来家知道。”妇人道:“可不瞒那王八哩。我只说那王八也是明王八,怪不的他往东京去的放心,丢下老婆在家,料莫他也不肯把\textuni{23B48}闲着。贼囚根子们,别要说嘴,打伙儿替你爹做牵头,引上了道儿,你每好图躧狗尾儿。说的是也不是?敢说我知道?嗔道贼淫妇买礼来,与我也罢了,又送蒸酥与他大娘,另外又送一大盒瓜子儿与我,要买住我的嘴头子,他是会养汉儿。我就猜没别人,就知道是玳安这贼囚根子,替他铺谋定计。”玳安道:“娘屈杀小的。小的平白管他这勾当怎的?小的等闲也不往他屋里去。娘也少听韩回子老婆说话,他两个为孩子好不嚷乱。常言‘要好不能勾,要歹登时就’,‘房倒压不杀人,舌头倒压人’,‘听者有,不听者无’。论起来,贲四娘子为人和气,在咱门首住着,家中大小没曾恶识了一个人。谁不在他屋里讨茶吃,莫不都养着?倒没处放。”金莲道:“我见那水眼淫妇,矮着个靶子,像个半头砖儿也是的,把那水济济眼挤着,七八拿杓儿舀。好个怪淫妇!他和那韩道国老婆,那长大摔瓜的淫妇,我不知怎的,掐了眼儿不待见他。”正说着,只见小玉走来说:“俺娘请五娘,潘姥姥来了,要轿子钱哩。”金莲道:“我在这里站着,他从多咱进去了?”琴童道:“姥姥打夹道里进去的。一来的轿子,该他六分银子。”金莲道:“我那得银子?来人家来,怎不带轿子钱儿走!”一面走到后边,见了他娘,只顾不与他轿子钱,只说没有。月娘道:“你与姥姥一钱银子,写帐就是了。”金莲道:“我是不惹他,他的银子都有数儿,只教我买东西,没教我打发轿子钱。”坐了一回,大眼看小眼,外边挨轿的催着要去。玉楼见不是事,向袖中拿出一钱银子来,打发抬轿的去了。不一时,大妗子、二妗子、大师父来了,月娘摆茶吃了。潘姥姥归到前边他女儿房内来,被金莲尽力数落了一顿,说道:“你没轿子钱,谁教你来?恁出丑划划的,教人家小看!”潘姥姥道:“姐姐,你没与我个钱儿,老身那讨个钱儿来?好容易筹办了这分礼儿来。”妇人道:“指望问我要钱,我那里讨个钱儿与你?你看七个窟窿到有八个眼儿等着在这里。今后你看有轿子钱便来他家来,没轿子钱别要来。料他家也没少你这个究亲戚!休要做打踊的献世包!‘关王卖豆腐——人硬货不硬’。我又听不上人家那等\textuni{23B48}声颡气。前日为你去了,和人家大嚷大闹的,你知道也怎的?驴粪球儿面前光,却不知里面受凄惶。”几句说的潘姥姥呜呜咽咽哭起来了。春梅道:“娘今日怎的,只顾说起姥姥来了。”一面安抚老人家,在里边炕上坐的,连忙点了盏茶与他吃。潘姥姥气的在炕上睡了一觉,只见后边请吃饭,才起来往后边去了。

西门庆从衙门中来家,正在上房摆饭,忽有玳安拿进贴儿来说:“荆老爹升了东南统制,来拜爹。”西门庆见贴儿上写:“新东南统制兼督漕运总兵官荆忠顿首拜。”慌的西门庆连忙穿衣,冠带迎接出来。只见都总制穿着大红麒麟补服、浑金带进来,后面跟着许多僚掾军牢。一面让至大厅上叙礼毕,分宾主而坐,茶汤上来。荆统制说道:“前日升官敕书才到,还未上任,径来拜谢老翁。”西门庆道:“老总兵荣擢恭喜,大才必有大用,自然之道。吾辈亦有光矣,容当拜贺。”一面请宽尊服,少坐一饭。即令左右放卓儿,荆统制再三致谢道:“学生奉告老翁,一家尚未拜,还有许多薄冗,容日再来请教罢。”便要起身,西门庆那里肯放,随令左右上来,宽去衣服,登时打抹春台,收拾酒果上来。兽炭顿烧,暖帘低放。金壶斟下液,翠盏贮羊羔,才斟上酒来,只见郑春、王相两个小优儿来到,扒在面前磕头。西门庆道:“你两个如何这咱才来?”问郑春:“那一个叫甚名字?”郑春道:“他唤王相,是王桂的兄弟。”西门庆即令拿乐器上来弹唱。须臾,两个小优哥唱了一套“霁景融和”。左右拿上两盘攒盒点心嗄饭,两瓶酒,打发马上人等。荆统制道:“这等就不是了。学生叨扰,下人又蒙赐馔,何以克当?”即令上来磕头。西门庆道:“一二日房下还要洁诚请尊正老夫人赏灯一叙,望乞下降。在座者惟老夫人、张亲家夫人、同僚何天泉夫人,还有两位舍亲,再无他人。”荆统制道:“若老夫人尊票制,贱荆已定趋赴。”又问起:“周老总兵怎的不见升转?”荆统制道:“我闻得周菊轩也只在三月间有京荣之转。”西门庆道:“这也罢了。”坐不多时,荆统制告辞起身,西门庆送出大门,看着上马喝道而去。

晚夕,潘金莲上寿,后厅小优弹唱,递了酒,西门庆便起身往金莲房中去了。月娘陪着大妗子、潘姥姥、女儿郁大姐、两个姑子在上房会的饮酒。潘金莲便陪西门庆在他房内,从新又安排上酒来,与西门庆梯己递酒磕头。落后潘姥姥来了,金莲打发他李瓶儿这边歇卧。他陪着西门庆自在饮酒,顽耍做一处。

却说潘姥姥到那边屋里,如意、迎春让他热炕上坐着。先是姥姥看明间内灵前,供摆着许多狮仙五老定胜桌,旁边挂着他影,因向前道了个问讯,说道:“姐姐好处生天去了。”进来坐在炕上,向如意儿、迎春道:“你娘勾了。官人这等费心追荐,受这般大供养,勾了。他是有福的。”如意儿道:“前日娘的生日,请姥姥,怎的不来?门外花大妗子和大妗子都在这里来,十二个道士念经,好不大吹大打,扬幡道场,水火炼度,晚上才去了。”潘姥姥道:“帮年逼节,丢着个孩子在家,我来家中没人,所以就不曾来。今日你杨姑娘怎的不见?”如意儿道:“姥姥还不知道,杨姑娘老病死了,从年里俺娘念经就没来,俺娘们都往北边与他上祭去来。”潘姥姥道:“可伤,他大如我,我还不晓的他老人家没了。嗔道今日怎的不见他。”说了一回,如意儿道:“姥姥,有钟甜酒儿,你老人家用些儿。”一面叫:“迎春姐,你放小卓儿在炕上,筛甜酒与姥姥吃杯。”不一时取到。饮酒之间,婆子又题起李瓶儿来:“你娘好人,有仁义的姐姐,热心肠儿。我但来这里,没曾把我老娘当外人看承,一到就是热茶热水与我吃,还只恨我不吃。晚间和我坐着说话儿,我临家去,好歹包些甚么儿与我拿了去,再不曾空了我。不瞒你姐姐每说,我身上穿的这披袄儿,还是你娘与我的。正经我那冤家,半分折针儿也迸不出来与我。我老身不打诳语,阿弥陀佛,水米不打牙。他若肯与我一个钱儿,我滴了眼睛在地。你娘与了我些甚么儿,他还说我小眼薄皮,爱人家的东西。想今日为轿子钱,你大包家拿着银子,就替老身出几分便怎的?咬定牙儿只说没有,到教后边西房里姐姐,拿出一钱银子来,打发抬轿的去了。归到屋里,还数落了我一顿,到明日有轿子钱,便教我来,没轿子钱,休叫我上门走。我这去了不来了。来到这里没的受他的气。随他去,有天下人心狠,不似俺这短寿命。姐姐你每听着我说,老身若死了,他到明日不听人说,还不知怎么收成结果哩!想着你从七岁没了老子,我怎的守你到如今,从小儿交你做针指,往余秀才家上女学去,替你怎么缠手缠脚儿的,你天生就是这等聪明伶俐,到得这步田地?他把娘喝过来断过去,不看一眼儿。”如意儿道:“原来五娘从小儿上学来,嗔道恁题起来就会识字深。”潘姥姥道:“他七岁儿上女学,上了三年,字仿也曾写过,甚么诗词歌赋唱本上字不认的!”

正说着,只见打的角门子响,如意儿道:“是谁叫门?”使绣春:“你瞧瞧去。”那绣春走来说:“是春梅姐姐来了。”如意儿连忙捏了潘姥姥一把手,就说道:“姥姥悄悄的,春梅来了。”潘姥姥道:“老身知道他与我那冤家一条腿儿。”只见春梅进来,见众人陪着潘姥姥吃酒,说道:“我来瞧瞧姥姥来了。”如意儿让他坐,这春梅把裙子搂起,一屁股坐在炕上。迎春便挨着他坐,如意坐在右边炕头上,潘姥姥坐在当中。因问:“你爹和你娘睡了不曾?”春梅道:“刚才打发他两个睡下了。我来这边瞧瞧姥姥,有几样菜儿,一壶儿酒,取过来和姥姥坐的。”因央及绣春:“你那边教秋菊掇了来,我已是攒下了。”绣春去了,不一时,秋菊用盒儿掇着菜儿,绣春提了一锡壶金华酒来。春梅分付秋菊:“你往房里看去,若叫我,来这里对我说。”秋菊去了。一面摆酒在炕卓上,都是烧鸭、火腿、海味之类,堆满春台。绣春关上角门,走进在旁边陪坐,于是筛上酒来。春梅先递了一钟与潘姥姥,然后递如意儿与迎春、绣春。又将护衣碟儿内,每样拣出,递与姥姥众人吃,说道:“姥姥,这个都是整菜,你用些儿。”那婆子道:“我的姐姐,我老身吃。”因说道:“就是你娘,从来也没费恁个心儿,管待我管待儿。姐姐,你倒有惜孤爱老的心,你到明日管情一步好一步。敢是俺那冤家,没人心没人义,几遍为他心龌龊,我也劝他,就扛的我失了色。今日早是姐姐你看着,我来你家讨冷饭来了,你下老实那等扛我!”春梅道:“姥姥,罢,你老人家只知其一,不知其二。俺娘是争强不伏弱的性儿。比不的六娘,银钱自有,他本等手里没钱,你只说他不与你。别人不知道,我知道。想俺爹虽是有的银子放在屋里,俺娘正眼儿也不看他的。若遇着买花儿东西,明公正义问他要。不恁瞒瞒藏藏的,教人看小了他,怎么张着嘴儿说人!他本没钱,姥姥怪他,就亏了他了。莫不我护他?也要个公道。”如意儿道:“错怪了五娘。自古亲儿骨肉,五娘有钱,不孝顺姥姥,再与谁?常言道,要打看娘面,千朵桃花一树儿生,到明日你老人家黄金入柜,五娘他也没个贴皮贴肉的亲戚,就如死了俺娘样儿。”婆子道:“我有今年没明年,知道今日死明日死?我也不怪他。”春梅见婆子吃了两钟酒,韶刀上来,便叫迎春:“二姐,你拿骰盆儿来,咱每掷个骰儿,抢红耍子儿罢。”不一时,取了四十个骰儿的骰盆来。春梅先与如意儿掷,掷了一回,又与迎春掷,都是赌大钟子。你一盏,我一钟。须臾,竹叶穿心,桃花上脸,把一锡瓶酒吃的罄净。迎春又拿上半坛麻姑酒来,也都吃了。约莫到二更时分,那潘姥姥老人家熬不的,又早前靠后仰,打起盹来,方才散了。

春梅便归这边来,推了推角门,开着,进入院内。只见秋菊正在明间板壁缝儿内,倚着春凳儿,听他两个在屋里行房,怎的作声唤,口中呼叫甚么。正听在热闹,不防春梅走到根前,向他腮颊上尽力打了个耳刮子,骂道:“贼少死的囚奴,你平白在这里听甚么?”打的秋菊睁睁的,说道:“我这里打盹,谁听甚么来,你就打我?”不想房里妇人听见,便问春梅,他和谁说话。春梅道:“没有人,我使他关门,他不动。”于是替他摭过了。秋菊揉着眼,关上房门。春梅走到炕上,摘头睡了。正是:

\[
鸧鹒有意留残景,杜宇无情恋晚晖。
\]

一宿晚景题过。次日,潘金莲生日,有傅伙计、甘伙计、贲四娘子、崔本媳妇、段大姐、吴舜臣媳妇、郑三姐、吴二妗子,都在这里。西门庆约会吴大舅、应伯爵,整衣冠,尊瞻视,骑马喝道,往何千户家赴席。那日也有许多官客,四个唱的,一起杂耍,周守备同席饮酒。至晚回家,就在前边和如意儿歇了。

到初十日,发贴儿请众官娘子吃酒,月娘便问西门庆说:“趁着十二日看灯酒,把门外的孟大姨和俺大姐,也带着请来坐坐,省的教他知道恼,请人不请他。”西门庆道:“早是你说。”分付陈敬济:“再写两个贴,差琴童儿请去。”这潘金莲在旁,听着多心,走到屋里,一面撺掇潘姥姥就要起身。月娘道:“姥姥你慌去怎的?再消住一日儿是的。”金莲道:“姐姐,大正月里,他家里丢着孩子,没人看,教他去罢。”慌的月娘装了两个盒子点心茶食,又与了他一钱轿子钱,管待打发去了。金莲因对着李娇儿说:“他明日请他有钱的大姨儿来看灯吃酒,一个老行货子,观眉观眼的,不打发去了,平白教他在屋里做甚么?待要说是客人,没好衣服穿。待要说是烧火的妈妈子,又不像。倒没的教我惹气。”因西门庆使玳安儿送了两个请书儿,往招宣府,一个请林太太,一个请王三官儿娘子黄氏。又使他院中早叫李桂儿、吴银儿、郑爱月儿、洪四儿四个唱的,李铭、吴惠、郑奉三个小优儿。不想那日贲四从东京来家,梳洗头脸,打选衣帽齐整,来见西门庆磕头。递上夏指挥回书。西门庆问道:“你如何这些时不来?”贲四具言在京感冒打寒一节,“直到正月初二日,才收拾起身回来,夏老爹多上覆老爹,多承看顾。”西门庆照旧还把钥匙教与他管绒线铺。另打开一间,教吴二舅开铺子卖绸绢,到明日松江货舡到,都卸在狮子街房内,同来保发卖。且叫贲四叫花儿匠在家攒造两架烟火,十二日要放与堂客看。

只见应伯爵领了李三见西门庆,先道外面承携之事。坐下吃毕茶,方才说起:“李三哥今有一宗买卖与你说,你做不做?”西门庆道:“甚么买卖?”李三道:“你东京行下文书,天下十三省,每省要几万两银子的古器。咱这东平府,坐派着二万两,批文在巡按处,还未下来。如今大街上张二官府,破二百两银子干这宗批要做,都看有一万两银子寻。小人会了二叔,敬来对老爹说。老爹若做,张二官府拿出五千两来,老爹拿出五千两来,两家合着做这宗买卖。左右没人,这边是二叔和小人与黄四哥,他那边还有两个伙计,二分八利钱。未知老爹意下何如?”西门庆问道:“是甚么古器?”李三道:“老爹还不知,如今朝廷皇城内新盖的艮岳,改为寿岳,上面起盖许多亭台殿阁,又建上清宝箓宫、会真堂、璇神殿,又是安妃娘娘梳妆阁,都用着这珍禽奇兽,周彝商鼎,汉篆秦炉,宣王石鼓,历代铜鞮,仙人掌承露盘,并希世古董玩器摆设,好不大兴工程,好少钱粮!”西门庆听了,说道:“比是我与人家打伙而做,不如我自家做了罢,敢量我拿不出这一二万银子来?”李三道:“得老爹全做又好了,俺每就瞒着他那边了。左右这边二叔和俺每两个,再没人。”伯爵道:“哥,家里还添个人儿不添?”西门庆道:“到根前再添上贲四,替你们走跳就是了。”西门庆又问道:“批文在那里?”李三道:“还在巡按上边,没发下来哩。”西门庆道:“不打紧,我差人写封书,封些礼,问宋松原讨将来就是了。”李三道:“老爹若讨去,不可迟滞,自古兵贵神速,先下米的先吃饭,诚恐迟了,行到府里。吃别人家干的去了。”西门庆笑道:“不怕他,就行到府里,我也还教宋松原拿回去。就是胡府尹,我也认的。”于是留李三、伯爵同吃了饭,约会:“我如今就写书,明日差小价去。”李三道:“又一件,宋老爹如今按院不在这里了,从前日起身往兖州府盘查去了。”西门庆道:“你明日就同小价往兖州府走遭。”李三道:“不打紧,等我去,来回破五六日罢了。老爹差那位管家,等我会下,有了书,教他往我那里歇,明日我同他好早起身。”西门庆道:“别人你宋老爹不信的,他常喜的是春鸿,叫春鸿、来爵两个去罢。”于是叫他二人到面前,会了李三,晚夕往他家宿歇。伯爵道:“这等才好,事要早干,高材疾足者先得之。”于是与李三吃毕饭,告辞而去。西门庆随即教陈敬济写了书,又封了十两叶子黄金在书帕内,与春鸿、来爵二人。分付:“路上仔细,若讨了批文,即便早来。若是行到府里,问你宋老爹讨张票,问府里要。”来爵道:“爹不消分付,小的曾在充州答应过徐参议,小的知道。”于是领了书礼,打在身边,径往李三家去了。

不说十一日来爵、春鸿同李三早雇了长行头口,往兖州府去了。却说十二日,西门庆家中请各堂客饮酒。那日在家不出门,约下吴大舅、谢希大、常峙节四位,晚夕来在卷棚内赏灯饮酒。王皇亲家小厮,从早辰就挑了箱子来了,等堂客到,打铜锣鼓迎接。周守备娘子有眼疾不得来,差人来回。止是荆统制娘子、张团练娘子、云指挥娘子,并乔亲家母、崔亲家母、吴大姨、孟大姨,都先到了。只有何千户娘子、王三官母亲林太太并王三官娘子不见到。西门庆使排军、玳安、琴童儿来回催邀了两三遍,又使文嫂儿催邀。午间,只见林氏一顶大轿,一顶小轿跟了来。见了礼,请西门庆拜见,问:“怎的三官娘子不来?”林氏道:“小儿不在,家中没人。”拜毕下来。止有何千户娘子,直到晌午半日才来,坐着四人大轿,一个家人媳妇坐小轿跟随,排军抬着衣箱,又是两个青衣人紧扶着轿扛,到二门里才下轿。前边鼓乐吹打迎接,吴月娘众姊妹迎至仪门首。西门庆悄悄在西厢房,放下帘来偷瞧,见这蓝氏年约不上二十岁,生的长挑身材,打扮的如粉妆玉琢,头上珠翠堆满,凤翘双插,身穿大红通袖五彩妆花四兽麒麟袍儿,系着金镶碧玉带,下衬着花锦蓝裙,两边禁步叮咚,麝兰扑鼻。但见:

\[
仪容娇媚,体态轻盈。姿性儿百伶百俐,身段儿不短不长。细弯弯两道蛾眉,直侵入鬓;滴流流一双凤眼,来往踅人。娇声儿似啭日流莺,嫩腰儿似弄风杨柳。端的是绮罗队里生来,却厌豪华气象,珠翠丛中长大,那堪雅淡梳汝。开遍海棠花,也不问夜来多少;标残杨柳絮,竟不知春意如何。轻移莲步,有蕊珠仙子之风流;款蹙湘裙,似水月观音之态度。正是:比花花解语,比玉玉生香。
\]
这西门庆不见则已,一则魂飞天外,魄丧九霄,未曾体交,精魄先失。少顷,月娘等迎接进入后堂,相见叙礼已毕,请西门太拜见。西门庆得了这一声,连忙整衣冠行礼,恍若琼林玉树临凡,神女巫山降下,躬身施礼,心摇目荡,不能禁止。拜见毕下来,月娘先请在卷棚内摆过茶,然后大厅吹打,安席上坐,各依次序,当下林太太上席。戏文扮的是《小天香半夜朝元记》。唱的两折下来,李桂姐、吴银儿、郑月儿、洪四儿四个唱的上去,弹唱灯词。

西门庆在卷棚内,自有吴大舅、应伯爵、谢希大、常峙节、李铭、吴惠、郑奉三个小优儿弹唱、饮酒,不住下来大厅格子外往里观觑。看官听说,明月不常圆,彩云容易散,乐极悲生,否极泰来,自然之理。西门庆但知争名夺利,纵意奢淫,殊不知天道恶盈,鬼录来追,死限临头。到晚夕堂中点起灯来,小优儿弹唱。还未到起更时分,西门庆陪人坐的,就在席上齁齁的打起睡来。伯爵便行令猜枚鬼混他,说道:“哥,你今日没高兴,怎的只打睡?”西门庆道:“我昨日没曾睡,不知怎的,今日只是没精神,要打睡。”只见四个唱的下来,伯爵教洪四儿与郑月儿两个弹唱,吴银儿与李桂姐递酒。

正耍在热闹处,忽玳安来报:“王太太与何老爹娘子起身了。”西门庆就下席来,黑影里走到二门里首,偷看他上轿。月娘众人送出来,前边天井内看放烟火。蓝氏已换了大红遍地金貂鼠皮袄,林太太是白绫袄儿,貂鼠披风,带着金钏玉珮。家人打灯笼,簇拥上轿而去。这西门庆正是饿眼将穿,馋涎空咽,恨不能就要成双。见蓝氏去了,悄悄从夹道进来。当时没巧不成语,姻缘会凑,可霎作怪,来爵儿媳妇见堂客散了,正从后边归来,开房门,不想顶头撞见西门庆,没处藏躲。原来西门庆见媳妇子生的乔样,安心已久,虽然不及来旺妻宋氏风流,也颇充得过第二。于是乘着酒兴儿,双关抱进他房中亲嘴。这老婆当初在王皇亲家,因是养主子,被家人不忿攘闹,打发出来,今日又撞着这个道路,如何不从了?一面就递舌头在西门庆口中。两个解衣褪裤,就按在炕沿子上,掇起腿来,被西门庆就耸了个不亦乐乎。正是:未曾得遇莺娘面,且把红娘去解馋。有诗为证:

\[
灯月交光浸玉壶,分得清光照绿珠。
莫道使君终有妇,教人桑下觅罗敷。
\]

\newpage
%# -*- coding:utf-8 -*-
%%%%%%%%%%%%%%%%%%%%%%%%%%%%%%%%%%%%%%%%%%%%%%%%%%%%%%%%%%%%%%%%%%%%%%%%%%%%%%%%%%%%%


\chapter{西门庆贪欲丧命\KG 吴月娘失偶生儿}


词曰:

\[
人生南北如岐路,世事悠悠等风絮,造化弄人无定据。翻来覆去,倒横直竖,眼见都如许。到如今空嗟前事,功名富贵何须慕,坎止流行随所寓。玉堂金马,竹篱茅舍,总是伤心处。
\]

话说西门庆,奸耍了来爵老婆,复走到卷棚内,陪吴大舅、应伯爵、谢希大、常峙节饮酒。荆统制娘子、张团练娘子、乔亲家母、崔亲家母、吴大妗子、段大姐,坐了好一会,上罢元宵圆子,方才起身去了。大妗子那日同吴舜臣媳妇都家去了。陈敬济打发王皇亲戏子二两银子唱钱,酒食管待出门。只四个唱的并小优儿,还在卷棚内弹唱递酒。伯爵向西门庆说道:“明日花大哥生日,哥,你送了礼去不曾?”西门庆说道:“我早辰送过去了。”玳安道:“花大舅头里使来定儿送请贴儿来了。”伯爵道:“哥,你明日去不去?我好来会你。”西门庆道:“到明日看。再不,你先去罢。”少顷,四个唱的后边去了,李铭等上来弹唱,那西门庆不住只在椅子上打睡。吴大舅道:“姐夫连日辛苦了,罢罢,咱每告辞罢。”于是起身。那西门庆又不肯,只顾拦着,留坐到二更时分才散。西门庆先打发四个唱的轿子去了,拿大钟赏李铭等三人每人两钟酒,与了六钱唱钱,临出门,叫回李铭分付:“我十五日要请你周爷和你荆爷、何老爹众位,你早替我叫下四个唱的,休要误了。”李铭跪下禀问:“爹叫那四个?”西门庆道:“樊百家奴儿,秦玉芝儿,前日何老爹那里唱的一个冯金宝儿,并吕赛儿,好歹叫了来。”李铭应诺:“小的知道了。”磕了头去了。

西门庆归后边月娘房里来。月娘告诉:“今日林太太与荆大人娘子好不喜欢,坐到那咱晚才去了。酒席上再三谢我说:蒙老爹扶持,但得好处,不敢有忘。在出月往淮上催攒粮运去也。”又说:“何大娘子今日也吃了好些酒,喜欢六姐,又引到那边花园山子上瞧了瞧。今日各项也赏了许多东西。”说毕,西门庆就在上房歇了。到半夜,月娘做了一梦,天明告诉西门庆说道:“敢是我日里看着他王太太穿着大红绒袍儿,我黑夜就梦见你李大姐箱子内寻出一件大红绒袍儿,与我穿在身上,被潘六姐匹手夺了去,披在他身上,教我就恼了,说道:‘他的皮袄,你要的去穿了罢了,这件袍儿你又来夺。’他使性儿把袍儿上身扯了一道大口子,吃我大吆喝,和他骂嚷,嚷着就醒了。不想是南柯一梦。”西门庆道:“不打紧,我到明日替你寻一件穿就是了。自古梦是心头想。”

到次日起来,头沉,懒待往衙门中去,梳头净面,穿上衣裳,走来前边书房中坐的。只见玉箫问如意儿挤了半瓯子奶,径到书房与西门庆吃药。西门庆正倚靠床上,叫王经替他打腿。王经见玉箫来,就出去了。玉箫打发他吃了药,西门庆就使他拿了一对金镶头簪儿,四个乌银戒指儿,送到来爵媳妇子屋里去。那玉箫明见主子使他干此营生,又似来旺媳妇子那一本帐,连忙钻头觅缝,袖的去了。送到了物事,还走来回西门庆话,说道:“收了,改日与爹磕头。”就拿回空瓯子儿到上房去了。月娘叫小玉熬下粥,约莫等到饭时前后,还不见进来。

原来王经稍带了他姐姐王六儿一包儿物事,递与西门庆瞧,就请西门庆往他家去。西门庆打开纸包儿,却是老婆剪下的一柳黑臻臻、光油油的青丝,用五色绒缠就了一个同心结托儿,用两根锦带儿拴着,做的十分细巧。又一件是两个口的鸳鸯紫遍地金顺袋儿,里边盛着瓜穰儿。西门庆观玩良久,满心欢喜,遂把顺袋放在书厨内,锦托儿褪于袖中。正在凝思之际,忽见吴月娘蓦地走来,掀开帘子,见他躺在床上,王经扒着替他打腿,便说道:“你怎的只顾在前头,就不进去了,屋里摆下粥了。你告我说,你心里怎的,只是恁没精神?”西门庆道:“不知怎的,心中只是不耐烦,害腿疼。”月娘道:“想必是春气起了。你吃了药,也等慢慢来。”一面请到房中,打发他吃粥。因说道:“大节下,你也打起精神儿来,今日门外花大舅生日,请你往那里走走去。再不,叫将应二哥来,同你坐坐。”西门庆道:“他也不在,与花大舅做生日去了。你整治下酒菜儿,等我往灯市铺子内和他二舅坐坐罢。”月娘道:“你骑马去,我教丫鬟整理。”这西门庆一面分付玳安备马,王经跟随,穿上衣穿,径到狮子街灯市里来。但见灯市中车马轰雷,灯球灿彩,游人如蚁,十分热闹。

\[
太平时序好风催,罗绮争驰斗锦回。
鳌山高耸青云上,何处游人不看来。
\]

西门庆看了回灯,到房子门首下马,进入里面坐下。慌的吴二舅、贲四都来声喏。门首买卖,甚是兴盛。来昭妻一丈青又早书房内笼下火,拿茶吃了。不一时,吴月娘使琴童儿、来安儿拿了两方盒点心嗄饭菜蔬,铺内有南边带来豆酒,打开一坛,摆在楼上,请吴二舅与贲四轮番吃酒。楼窗外就看见灯市,来往人烟不断。

吃至饭后时分,西门庆使王经对王六儿说去。王六儿听见西门庆来,连忙整治下春台,果盒酒肴等候。西门庆分付来昭:“将这一桌酒菜,晚夕留着吴二舅、贲四在此上宿吃,不消拿回家去了。”又教琴童提送一坛酒,过王六儿这边来。西门庆于是骑马径到他家。妇人打扮迎接到明间内,插烛也似磕了四个头。西门庆道:“迭承你厚礼,怎的两次请你不去?”王六儿说道:“爹倒说的好,我家中再有谁来?不知怎的,这两日只是心里不好,茶饭儿也懒待吃,做事没入脚处。”西门庆道:“敢是想你家老公?”妇人道:“我那里想他!倒是见爹这一向不来,不知怎的怠慢着爹了,爹把我网巾圈儿打靠后了,只怕另有个心上人儿了。”西门庆笑道:“那里有这个理!倒因家中节间摆酒,忙了两日。”妇人道:“说昨日爹家中请堂客来。”西门庆道:“便是你大娘吃过人家两席节酒,须得请人回席。”妇人道:“请了那几位堂客?”西门庆便说某人某人,从头诉说一遍。妇人道:“看灯酒儿,只请要紧的,就不请俺每请儿。”西门庆道:“不打紧,到明日十六,还有一席酒,请你每众伙计娘子走走去。是必到跟前又推故不去了。”妇人道:“娘若赏个贴儿来,怎敢不去?”因前日他小大姐骂了申二姐,教他好不抱怨,说俺每。他那日原要不去来,倒是俺每撺掇了他去,落后骂了来,好不在这里哭。俺每倒没意思剌涑的。落后又教爹娘费心,送了盒子并一两银子来,安抚了他,才罢了。原来小大姐这等躁暴性子,就是打狗也看主人面。”西门庆道:“你不知这小油嘴,他好不兜达的性儿,着紧把我也擦刮的眼直直的。也没见,他叫你唱,你就唱个儿与他听罢了,谁教你不唱,又说他来?”妇人道:“耶嚛,耶嚛!他对我说,他几时说他来,说小大姐走来指着脸子就骂起来,在我这里好不三行鼻涕两行眼泪的哭。我留他住了一夜,才打发他去了。”说了一回,丫头拿茶吃了。老冯婆子又走来与西门庆磕头。西门庆与了他约三四钱一块银子,说道:“从你娘没了,就不往我那里走走去。”妇人道:“没他的主儿,那里着落?倒常时来我这里,和我做伴儿。”

不一时,请西门庆房中坐的,问:“爹和了午饭不曾?”西门庆道:“我早辰家中吃了些粥,刚才陪你二舅又吃了两个点心,且不吃甚么哩。”一面放桌儿,安排上酒来。妇人令王经打开豆酒,筛将上来,陪西门庆做一处饮酒。妇人问道:“我稍来的那物件儿,爹看见来?都是奴旋剪下顶中一溜头发,亲手做的。管情爹见了爱。”西门庆道:“多谢你厚情。”饮至半酣,见房内无人,西门庆袖中取出来,套在龟身下,两根锦带儿扎在腰间,用酒服下胡僧药去,那妇人用手搏弄,弄得那话登时奢棱跳脑,横筋皆现,色若紫肝,比银托子和白绫带子又不同。西门庆搂妇人坐在怀内,那话插进牝中,在上面两个一递一口饮酒,咂舌头顽笑。吃至掌灯,冯妈妈又做了些韭菜猪肉饼儿拿上来。妇人陪西门庆每人吃了两个,丫鬟收下去。两个就在里间暖炕上,撩开锦幔,解衣就寝。妇人知道西门庆好点着灯行房,把灯台移在里间炕边桌上,一面将纸门关上,澡牝干净,脱了裤儿,钻在被窝里,与西门庆做一处相搂相抱,睡了一回。原来西门庆心中只想着何千户娘子蓝氏,欲情如火,那话十分坚硬。先令妇人马伏在下,那话放入庭花内,极力扇蹦了约二三百度,扇蹦的屁股连声响亮,妇人用手在下揉着心子,口中叫达达如流水。西门庆还不美意,又起来披上白绫小袄,坐在一只枕头上,令妇人仰卧,寻出两条脚带,把妇人两只脚拴在两边护炕柱儿上,卖了个金龙探爪,将那话放入牝中,少时,没棱露脑,浅抽深送。恐妇人害冷,亦取红绫短襦,盖在他身上。这西门庆乘其酒兴,把灯光挪近跟前,垂首玩其出入之势。抽撤至首,复送至根,又数百回。妇人口中百般柔声颤语,都叫将出来。西门庆又取粉红膏子药,涂在龟头上攮进去,妇人阴中麻痒不能当,急令深入,两厢迎就。这西门庆故作逗留,戏将龟头濡晃其牝口,又操弄其花心,不肯深入,急的妇人淫津流出,如蜗之吐涎。灯光里,见他两只腿儿着红鞋,跷在两边,吊的高高的,一往一来,一冲一撞,其兴不可遏。因口呼道:“淫妇,你想我不想?”妇人道:“我怎么不想达达,只要你松柏儿冬夏长青便好。休要日远日疏,顽耍厌了,把奴来不理。奴就想死罢了,敢和谁说?有谁知道?就是俺那王八来家,我也不和他说。想他恁在外做买卖,有钱,他不会养老婆的?他肯挂念我?”西门庆道:“我的儿,你若一心在我身上,等他来家,我爽利替他另娶一个,你只长远等着我便了。”妇人道:“好达达,等他来家,好歹替他娶了一个罢,或把我放在外头,或是招我到家去,随你心里。淫妇爽利把不直钱的身子,拼与达达罢,无有个不依你的。”西门庆道:“我知道。”两个说话之间,又干勾两顿饭时,方才精泄。解御下妇人脚带来,搂在被窝内,并头交股,醉眼朦胧,一觉直睡到三更时分方起。西门庆起来,穿衣净手。妇人开了房门,叫丫鬟进来,再添美馔,复饮香醪,满斟暖酒,又陪西门庆吃了十数杯。不觉醉上来,才点茶漱口,向袖中掏出一纸贴儿递与妇人:“问甘伙计铺子里取一套衣服你穿,随你要甚花样。”那妇人万福谢了,方送出门。

王经打着灯笼,玳安、琴童笼着马,那时也有三更天气,阴云密布,月色朦胧,街市上人烟寂寞,闾巷内犬吠盈盈。打马刚走到西首那石桥儿跟前,忽然一阵旋风,只见个黑影子,从桥底下钻出来,向西门庆一扑。那马见了只一惊跳,西门庆在马上打了个冷战,醉中把马加了一鞭,那马摇了摇鬃,玳安、琴童两个用力拉着嚼环,收煞不住,云飞般望家奔将来,直跑到家门首方止。王经打着灯笼,后边跟不上。西门庆下马腿软了,被左右扶进,径往前边潘金莲房中来。此这一来,正是:

\[
失脱人家逢五道,滨冷饿鬼撞钟馗。
\]

原来金莲从后边来,还没睡,浑衣倒在炕上,等待西门庆。听见来了,连忙一骨碌扒起来,向前替他接衣服。见他吃的酩酊大醉,也不敢问他。西门太一只手搭伏着他肩膀上,搂在怀里,口中喃喃呐呐说道:“小淫妇儿,你达达今日醉了,收拾铺,我睡也。”那妇人持他上炕,打发他歇下。那西门庆丢倒头在枕上鼾睡如雷,再摇也摇他不醒。然后妇人脱了衣裳,钻在被窝内,慢慢用手腰里摸他那话,犹如绵软,再没硬朗气儿,更不知在谁家来。翻来覆去,怎禁那欲火烧身,淫心荡漾,不住用手只顾捏弄,蹲下身子,被窝内替他百计品咂,只是不起,急的妇人要不的。因问西门庆:“和尚药在那里放着哩?”推了半日推醒了。西门庆酩子里骂道:“怪小淫妇,只顾问怎的?你又教达达摆布你,你达今日懒待动弹。药在我袖中穿心盒儿内。你拿来吃了,有本事品弄的他起来,是你造化。”那妇人便去袖内摸出穿心盒来打开,里面只剩下三四丸药儿。这妇人取过烧酒壶来,斟了一钟酒,自己吃了一丸,还剩下三丸。恐怕力不效,千不合,万不合,拿烧酒都送到西门庆口内。醉了的人,晓的甚么?合着眼只顾吃下去。那消一盏热茶时,药力发作起来,妇人将白绫带子拴在根上,那话跃然而起,妇人见他只顾去睡,于是骑在他身上,又取膏子药安放在马眼内,顶入牝中,只顾揉搓,那话直抵苞花窝里,觉翕翕然,浑身酥麻,畅美不可言。又两手据按,举股一起一坐,那话坐棱露脑,一二百回。初时涩滞,次后淫水浸出,稍沾滑落,西门庆由着他掇弄,只是不理。妇人情不能当,以舌亲于西门庆口中,两手搂着他脖项,极力揉搓,左右偎擦,麈柄尽没至根,止剩二卵在外,用手摸之,美不可言,淫水随拭随出。比三鼓天,五换巾帕。妇人一连丢了两次,西门庆只是不泄。龟头越发胀的犹如炭火一般,害箍胀的慌,令妇人把根下带子去了,还发胀不已,令妇人用口吮之。这妇人扒伏在他身上,用朱唇吞裹龟头,只顾往来不已,又勒勾约一顿饭时,那管中之精猛然一股冒将出来,犹水银之淀筒中相似,忙用口接咽不及,只顾流将出来。初时还是精液,往后尽是血水出来,再无个收救。西门庆已昏迷去,四肢不收。妇人也慌了,急取红枣与他吃下去。精尽继之以血,血尽出其冷气而已。良久方止。妇人慌做一团,便搂着西门庆问道:“我的哥哥,你心里觉怎么的!”西门庆亦苏醒了一回,方言:“我头目森森然,莫知所以。”金莲问:“你今日怎的流出恁许多来?”更不说他用的药多了。看官听说,一己精神有限,天下色欲无穷。又曰“嗜欲深者生机浅”,西门庆只知贪淫乐色,更不知油枯灯灭,髓竭人亡。正是起头所说:

\[
二八佳人体似酥,腰间仗剑斩愚夫。
虽然不见人头落,暗里教君骨髓枯。
\]

一宿晚景题过。到次日清早辰,西门庆起来梳头,忽然一阵昏晕,望前一头抢将去。早被春梅双手扶住,不曾跌着磕伤了头脸。在椅上坐了半日,方才回过来。慌的金莲连忙问道:“只怕你空心虚弱,且坐着,吃些甚么儿着,出去也不迟。”一面使秋菊:“后边取粥来与你爹吃。”那秋菊走到后边厨下,问雪娥:“熬的粥怎么了?爹如此这般,今早起来害了头晕,跌了一交,如今要吃粥哩。”不想被月娘听见,叫了秋菊,问其端的。秋菊悉把西门庆梳头,头晕跌倒之事,告诉一遍。月娘不听便了,听了魂飞天外,魄散九霄,一面分付雪娥快熬粥,一面走来金莲房中看视。见西门庆坐在椅子上,问道:“你今日怎的头晕?”西门庆道:“我不知怎的,刚才就头晕起来。”金莲道:“早时我和春梅要跟前扶住了,不然好轻身子儿,这一交和你善哩!”月娘道:“敢是你昨日来家晚了,酒多了头沉。”金莲道:“昨日往谁家吃酒?那咱晚才来。”月娘道:“他昨日和他二舅在铺子里吃酒来。”不一时,雪娥熬了粥,教春梅拿着,打发西门庆吃。那西门庆拿起粥来,只吃了半瓯儿,懒待吃,就放下了。月娘道:“你心里觉怎的?”西门庆道:“我不怎么,只是身子虚飘飘的,懒待动旦。”月娘道:“你今日不往衙门中去罢。”西门庆道:“我不去了。消一回,我往前边看着姐夫写贴儿,十五日请周菊轩、荆南岗、何大人众官客吃酒。”月娘道:“你今日还没吃药,取奶来把那药再吃上一服。是你连日着辛苦忙碌了。”一面教春梅问如意儿挤了奶来,用盏儿盛着,教西门庆吃了药,起身往前边去。春梅扶着,刚走到花园角门首,觉眼便黑了,身子晃晃荡荡,做不的主儿,只要倒。春梅又扶回来了。月娘道:“依我且歇两日儿,请人也罢了,那里在乎这一时。且在屋里将息两日儿,不出去罢。”因说:“你心里要吃甚么,我往后边做来与你吃。”西门庆道:“我心里不想吃。”

月娘到后边,从新又审问金莲:“他昨日来家醉不醉?再没曾吃酒?与你行甚么事?”金莲听了,恨不的生出几个口来,说一千个没有:“姐姐,你没的说,他那咱晚来了,醉的行礼儿也没顾的,还问我要烧酒吃,教我拿茶当酒与他吃,只说没了酒,好好打发他睡了。自从姐姐那等说了,谁和他有甚事来,倒没的羞人子剌剌的。倒只怕别处外边有了事来,俺每不知道。若说家里,可是没丝毫事儿。”月娘和玉楼都坐在一处,一面叫了玳安、琴童两个到跟前审问他:“你爹昨日在那里吃酒来?你实说便罢,不然有一差二错,就在你这两个囚根子身上。”那玳安咬定牙,只说狮子街和二舅、贲四吃酒,再没往那里去。落后叫将吴二舅来,问他,二舅道:“姐夫只陪俺每吃了没多大回酒,就起身往别处去了。”这吴月娘听了,心中大怒,待二舅去了,把玳安、琴童尽力数骂了一遍,要打他二人。二人慌了,方才说出:“昨日在韩道国老婆家吃酒来。”那潘金莲得不的一声就来了,说道:“姐姐刚才就埋怨起俺每来,正是冤杀旁人笑杀贼。俺每人人有面,树树有皮,姐姐那等说来,莫不俺每成日把这件事放在头里?”又道:“姐姐,你再问这两个囚根子,前日你往何千户家吃酒,他爹也是那咱时分才来,不知在谁家来。谁家一个拜年,拜到那咱晚!”玳安又恐怕琴童说出来,隐瞒不住,遂把私通林太太之事,备说一遍。月娘方才信了,说道:“嗔道教我拿贴儿请他,我还说人生面不熟,他不肯来,怎知和他有连手。我说恁大年纪,描眉画鬓,搽的那脸倒像腻抹儿抹的一般,干净是个老浪货!”玉楼道:“姐姐,没见一个儿子也长恁大人儿,娘母还干这个营生。忍不住,嫁了个汉子,也休要出这个丑。”金莲道:“那老淫妇有甚么廉耻!”月娘道:“我只说他决不来,谁想他浪\textShan 着来了。”金莲道:“这个,姐姐才显出个皂白来了!像韩道国家这个淫妇,姐姐还嗔我骂他!干净一家子都养汉,是个明王八,把个王八花子也裁派将来,早晚好做勾使鬼。”月娘道:“王三官儿娘,你还骂他老淫妇,他说你从小儿在他家使唤来。”那金莲不听便罢,听了把脸掣耳朵带脖子都红了,便骂道:“汗邪了那贼老淫妇!我平日在他家做甚么?还是我姨娘在他家紧隔壁住,他家有个花园,俺每小时在俺姨娘家住,常过去和他家伴姑儿耍子,就说我在他家来,我认的他是谁?也是个张眼露睛的老淫妇!”月娘道:“你看那嘴头子!人和你说话,你骂他。”那金莲一声儿就不言语了。

月娘主张叫雪娥做了些水角儿,拿了前边与西门庆吃。正走到仪门首,只见平安儿径直往花园中走。被月娘叫住问道:“你做甚么?”平安儿道:“李铭叫了四个唱的,十五日摆酒,因来回话。问摆的成摆不成。我说未发贴儿哩。他不信,教我进来禀爹。”月娘骂道:“怪贼奴才,还摆甚么酒,问甚么,还不回那王八去哩,还来禀爹娘哩。”把平安儿骂的往外金命水命去了。月娘走到金莲房中,看着西门庆只吃了三四个水角儿,就不吃了。因说道:“李铭来回唱的,教我回倒他,改日子了,他去了。”西门庆点头儿。

西门庆只望一两日好些出来,谁知过了一夜,到次日,内边虚阳肿胀,不便处发出红瘰来,连肾囊都肿得明滴溜如茄子大。但溺尿,尿管中犹如刀子犁的一般。溺一遭,疼一遭。外边排军、伴当备下马伺候,还等西门庆往衙门里大发放,不想又添出这样症候来。月娘道:“你依我拿贴儿回了何大人,在家调理两日儿,不去罢。你身子恁虚弱,趁早使小厮请了任医官,教瞧瞧。你吃他两贴药过来。休要只顾耽着,不是事。你偌大的身量,两日通没大好吃甚么儿,如何禁的?”那西门庆只是不肯吐口儿请太医,只说:“我不妨事,过两日好了,我还出去。”虽故差人拿贴儿送假牌往衙门里去,在床上睡着,只是急躁,没好气。西门庆只望一两日好些出来,谁知过了一夜,到次日,内边虚阳肿胀,不便处发出红瘰来,连肾囊都肿得明滴溜如茄子大。但溺尿,尿管中犹如刀子犁的一般。溺一遭,疼一遭。外边排军、伴当备下马伺候,还等西门庆往衙门里大发放,不想又添出这样症候来。月娘道:“你依我拿贴儿回了何大人,在家调理两日儿,不去罢。你身子恁虚弱,趁早使小厮请了任医官,教瞧瞧。你吃他两贴药过来。休要只顾耽着,不是事。你偌大的身量,两日通没大好吃甚么儿,如何禁的?”那西门庆只是不肯吐口儿请太医,只说:“我不妨事,过两日好了,我还出去。”虽故差人拿贴儿送假牌往衙门里去,在床上睡着,只是急躁,没好气。

应伯爵打听得知,走来看他。西门庆请至金莲房中坐的。伯爵声喏道:“前日打搅哥,不知哥心中不好,嗔道花大舅那里不去。”西门庆道:“我心中若好时,也去了。不知怎的懒待动旦。”伯爵道:“哥,你如今心内怎样的?”西门庆道:“不怎的,只是有些头晕,起来身子软,走不的。”伯爵道:“我见你面容发红色,只怕是火。教人看来不曾?”西门庆道:“房下说请任后溪来看我,我说又没甚大病,怎好请他的。”伯爵道:“哥,你这个就差了,还请他来看看,怎的说。吃两贴药,散开这火就好了。春气起,人都是这等痰火举发举发。昨日李铭撞见我,说你使他叫唱的,今日请人摆酒,说你心中不好,改了日子。把我唬了一跳,我今日才来看哥。”西门庆道:“我今日连衙门中拜牌也没去,送假牌去了。”伯爵道:“可知去不的,大调理两日儿出门。”吃毕茶道:“我去罢,再来看哥。李桂姐会了吴银儿,也要来看你哩。”西门庆道:“你吃了饭去。”伯爵道:“我一些不吃。”扬长出去了。

西门庆于是使琴童往门外请了任医官来,进房中诊了脉,说道:“老先生此贵恙,乃虚火上炎,肾水下竭,不能既济,此乃是脱阳之症。须是补其阴虚,方才好得。”说毕,作辞起身去了。一面封了五钱银子,讨将药来,吃了。止住了头晕,身子依旧还软,起不来。下边肾囊越发肿痛,溺尿甚难。西门庆于是使琴童往门外请了任医官来,进房中诊了脉,说道:“老先生此贵恙,乃虚火上炎,肾水下竭,不能既济,此乃是脱阳之症。须是补其阴虚,方才好得。”说毕,作辞起身去了。一面封了五钱银子,讨将药来,吃了。止住了头晕,身子依旧还软,起不来。下边肾囊越发肿痛,溺尿甚难。

到后晌时分,李桂姐、吴银儿坐轿子来看。每人两个盒子,进房与西门庆磕头,说道:“爹怎的心里不自在?”西门庆道:“你姐儿两个自恁来看看便了,如何又费心买礼儿。”因说道:“我今年不知怎的,痰火发的重些。”桂姐道:“还是爹这节间酒吃的多了,清洁他两日儿,就好了。”坐了一回,走到李瓶儿那边屋里,与月娘众人见节。请到后边,摆茶毕,又走来到前边,陪西门庆坐的说话儿。只见伯爵又陪了谢希大、常峙节来望。西门庆教玉箫搊扶他起来坐的,留他三人在房内,放桌儿吃酒。谢希大道:“哥,用了些粥不曾?”玉箫把头扭着不答应。西门庆道:“我还没吃粥,咽不下去。”希大道:“拿粥,等俺每陪哥吃些粥儿还好。”不一时,拿将粥来。西门庆拿起粥来,只扒了半盏儿,就吃不下了。月娘和李桂姐、吴银儿都在李瓶儿那边坐的。伯爵问道:“李桂姐与银姐来了,怎的不见?”西门庆道:“在那边坐的。”伯爵因令来安儿:“你请过来,唱一套儿与你爹听。”吴月娘恐西门庆不耐烦,拦着,只说吃酒哩,不教过来。众人吃了一回酒,说道:“哥,你陪着俺每坐,只怕劳碌着你。俺每去了,你自在侧侧儿罢。”西门庆道:“起动列位挂心。”三人于是作辞去了。

应伯爵走出小院门,叫玳安过来分付:“你对你大娘说,应二爹说来,你爹面上变色,有些滞气,不好,早寻人看他。大街上胡太医最治的好痰火,何不使人请他看看,休要耽迟了。”玳安不敢怠慢,走来告诉月娘。月娘慌进房来,对西门庆说:“方才应二哥对小厮说,大街上胡太医看的痰火好,你何不请他来看看你?”西门庆道:“胡太医前番看李大姐不济,又请他?”月娘道:“药医不死病,佛度有缘人。看他不济,只怕你有缘,吃了他的药儿好了是的。”西门庆道:“也罢,你请他去。”不一时,使棋童儿请了胡太医来。适有吴大舅来看,陪他到房中看了脉。对吴大舅、陈敬济说:“老爹是个下部蕴毒,若久而不治,卒成溺血之疾。乃是忍便行房。”又卦了五星药金,讨将药来吃下去,如石沉大海一般,反溺不出来。月娘慌了,打发桂姐、吴银儿去了,又请何老人儿子何春泉来看。又说:“是癃闭便毒,一团膀胱邪火,赶到这下边来。四肢经络中,又有湿痰流聚,以致心肾不交。”封了五钱药金,讨将药来,越发弄的虚阳举发,麈柄如铁,昼夜不倒。潘金莲晚夕不管好歹,还骑在他身上,倒浇蜡烛掇弄,死而复苏者数次。

到次日,何千户要来望,先使人来说。月娘便对西门庆道:“何大人要来看你,我扶你往后边去罢,这边隔二骗三,不是个待人的。”那西门庆点头儿。于是月娘替他穿上暖衣,和金莲肩搭搊扶着,方离了金莲房,往后边上房,铺下被褥高枕,安顿他在明间炕上坐的。房中收拾干净,焚下香。不一时,何千户来到,陈敬济请他到于后边卧房,看见西门庆坐在病榻上,说道:“长官,我不敢作揖。”因问:“贵恙觉好些?”西门庆告诉:“上边火倒退下了,只是下边肿毒,当不的。”何千户道:“此系便毒。我学生有一相识,在东昌府探亲,昨日新到舍下,乃是山西汾州人氏,姓刘号桔斋,年半百,极看的好疮毒。我就使人请他来看看长官贵恙。”西门庆道:“多承长官费心,我这里就差人请去。”何千户吃毕茶,说道:“长官,你耐烦保重。衙门中事,我每日委答应的递事件与你,不消挂意。”西门庆举手道:“只是有劳长官了。”作辞出门。西门庆这里随即差玳安拿贴儿,同何家人请了这刘桔斋来。看了脉,并不便处,连忙上了药,又封一贴煎药来。西门庆答贺了一匹杭州绢,一两银子。吃了他头一盏药,还不见动静。

那日不想郑月儿送了一盒鸽子雏儿,一盒果饼顶皮酥,坐轿子来看。进门与西门庆磕头,说道:“不知道爹不好,桂姐和银姐好人儿,不对我说声儿,两个就先来了。看的爹迟了,休怪。”西门庆道:“不迟,又起动你费心,又买礼来。”爱月儿笑道:“甚么大礼,惶恐。”因说:“爹清减的恁样的,每日饮馔也用些儿?”月娘道:“用的倒好了,吃不多儿。今日早辰,只吃了些粥汤儿,刚才太医看了去了。”爱月儿道:“娘,你分付姐把鸽子雏儿顿烂一个儿来,等我劝爹进些粥儿。你老人家不吃,恁偌大身量,一家子金山也似靠着你,却怎么样儿的。”月娘道:“他只害心口内拦着,吃不下去。”爱月儿道:“爹,你依我说,把这饮撰儿就懒待吃,须也强吃些儿,怕怎的?人无根本,水食为命。终须用的有柱\textShouQiang 些儿。不然,越发淘渌的身子空虚了。”不一时,顿烂了鸽子雏儿,小玉拿粥上来,十香甜酱瓜茄,粳粟米粥儿。这郑月儿跳上炕去,用盏儿托着,跪在西门庆身边,一口口喂他。强打着精神,只吃了上半盏儿。拣两箸儿鸽子雏儿在口内,就摇头儿不吃了。爱月儿道:“一来也是药,二来还亏我劝爹,却怎的也进了些饮馔儿!”玉箫道:“爹每常也吃,不似今日月姐来,劝着吃的多些。”月娘一面摆茶与爱月儿吃,临晚管待酒馔,与了他五钱银子,打发他家去。爱月儿临出门,又与西门庆磕头,说道:“爹,你耐烦将息两日儿,我再来看你。”

比及到晚夕,西门庆又吃了刘桔斋第二贴药,遍身疼痛,叫了一夜。到五更时分,那不便处肾囊胀破了,流了一滩鲜血,龟头上又生出疳疮来,流黄水不止。西门庆不觉昏迷过去。月娘众人慌了,都守着看视,见吃药不效,一面请了刘婆子,在前边卷棚内与西门庆点人灯挑神,一面又使小厮往周守备家内访问吴神仙在那里,请他来看,因他原相西门庆今年有呕血流脓之灾,骨瘦形衰之病。贲四说:“也不消问周老爹宅内去,如今吴神仙见在门外土地庙前,出着个卦肆儿,又行医,又卖卦。人请他,不争利物,就去看治。”月娘连忙就使琴童把这吴神仙请将来。进房看了西门庆不似往时,形容消减,病体恹恹,勒着手帕,在于卧榻。先诊了脉息,说道:“官人乃是酒色过度,肾水竭虚,太极邪火聚于欲海,病在膏肓,难以治疗。吾有诗八句,说与你听。只因他:

\[
醉饱行房恋女娥,精神血脉暗消磨。
遗精溺血与白浊,灯尽油干肾水枯。
当时只恨欢娱少,今日翻为疾病多。
玉山自倒非人力,总是卢医怎奈何!”
\]

月娘见他说治不的了,道:“既下药不好,先生看他命运如何?”吴神仙掐指寻纹,打算西门庆八字,说道:“属虎的,丙寅年,戊申月,壬午日,丙辰时。今年戊戌,流年三十三年,算命,见行癸亥运。虽然是火土伤官,今年戊土来克壬水。正月又是戊寅月,三戊冲辰,怎么当的?虽发财发福,难保寿源。有四句断语不好。说道:

\[
命犯灾星必主低,身轻煞重有灾危。
时日若逢真太岁,就是神仙也皱眉。
\]
月娘道:“命不好,请问先生还有解么?”神仙道:“白虎当头,丧门坐命,神仙也无解,太岁也难推。造物已定,神鬼莫移。”月娘只得拿了一匹布,谢了神仙,打发出门。月娘见求神问卜皆有凶无吉,心中慌了。到晚夕,天井内焚香,对天发愿,许下“儿夫好了,要往泰安州顶上与娘娘进香挂袍三年”。孟玉楼又许下逢七拜斗,独金莲与李娇儿不许愿心。

西门庆自觉身体沉重,要便发昏过去,眼前看见花子虚、武大在他跟前站立,问他讨债,又不肯告人说,只教人厮守着他。见月娘不在跟前,一手拉着潘金莲,心中舍他不的,满眼落泪,说道:“我的冤家,我死后,你姐妹们好好守着我的灵,休要失散了。”那金莲亦悲不自胜,说道:“我的哥哥,只怕人不肯容我。”西门庆道:“等他来,等我和他说。”不一时,吴月娘进来,见他二人哭的眼红红的,便道:“我的哥哥,你有甚话,对奴说几句儿,也是我和你做夫妻一场。”西门庆听了,不觉哽咽哭不出声来,说道:“我觉自家好生不济,有两句遗言和你说:我死后,你若生下一男半女,你姊妹好好待着,一处居住,休要失散了,惹人家笑话。”指着金莲说:“六儿从前的事,你耽待他罢。”说毕,那月娘不觉桃花脸上滚下珍珠来,放声大哭,悲恸不止。西门庆嘱付了吴月娘,又把陈敬济叫到跟前,说道:“姐夫,我养儿靠儿,无儿靠婿。姐夫就是我的亲儿一般。我若有些山高水低,你发送了我入土。好歹一家一计,帮扶着你娘儿每过日子,休要教人笑话。”又分付:“我死后,段子铺里五万银子本钱,有你乔亲家爹那边,多少本利都找与他。教傅伙计把贷卖一宗交一宗,休要开了。贲四绒线铺,本银六千五百两,吴二舅绸绒铺是五千两,都卖尽了货物,收了来家。又李三讨了批来,也不消做了,教你应二叔拿了别人家做去罢。李三、黄四身上还欠五百两本钱,一百五十两利钱未算,讨来发送我。你只和傅伙计守着家门这两个铺子罢。印子铺占用银二万两,生药铺五千两,韩伙计、来保松江船上四千两。开了河,你早起身,往下边接船去。接了来家,卖了银子并进来,你娘儿每盘缠。前边刘学官还少我二百两,华主簿少我五十两,门外徐四铺内,还欠我本利三百四十两,都有合同见在,上紧使人摧去。到日后,对门并狮子街两处房子都卖了罢,只怕你娘儿们顾揽不过来。”说毕,哽哽咽咽的哭了。陈敬济道:“爹嘱咐,儿子都知道了。”不一时,傅伙计、甘伙计、吴二舅、贲四、崔本都进来看视问安。西门庆一一都分付了一遍。众人都道:“你老人家宽心,不妨事。”一日来问安看者,也有许多。见西门庆不好的沉重,皆嗟叹而去。

过了两日,月娘痴心,只指望西门庆还好,谁知天数造定,三十三岁而去。到于正月二十一日,五更时分,相火烧身,变出风来,声若牛吼一般,喘息了半夜。挨到巳牌时分,呜呼哀哉,断气身亡。正是:三寸气在千般用,一旦无常万事休。古人有几句格言,说得好:

\[
为人多积善,不可多积财。积善成好人,积财惹祸胎。
石崇当日富,难免杀身灾。邓通饥饿死,钱山何用哉!
今人非古比,心地不明白。只说积财好,反笑积善呆。
多少有钱者,临了没棺材。
\]

原来西门庆一倒头,棺材尚未曾预备。慌的吴月娘叫了吴二舅与贲四到跟前,开了箱子拿四四锭元宝,教他两个看材板去。刚才打发去了,不防忽一阵就害肚里疼,急扑进去床上倒下,就昏晕不省人事。孟玉楼与潘金莲、孙雪娥都在那边屋里,七手八脚,替西门庆戴唐巾,装柳穿衣服。忽听见小玉来说:“俺娘跌倒在床上。”慌的玉楼、李娇儿就来问视,月娘手按着害肚内疼,就知道决撒了。玉楼教李娇儿守着月娘,他就来使小厮快请蔡老娘去。李娇儿又使玉箫前边教如意儿来。比及玉楼回到上房里面,不见了李娇儿。原来李娇儿赶月娘昏沉,房内无人,箱子开着,暗暗拿了五锭元宝,往他屋里去了。手中拿将一搭纸,见了玉楼,只说:“寻不见草纸,我往房里寻草纸去来。”那玉楼也不留心,且守着月娘,拿杩子伺候,见月娘看看疼的紧了。

不一时,蔡老娘到了,登时生下一个孩儿来。这屋里装柳西门庆停当,口内才没气儿,合家大小放声号哭起来。蔡老娘收裹孩儿,剪去脐带,煎定心汤与月娘吃了。扶月娘暖炕上坐的。月娘与了蔡老娘三两银子,蔡老娘嫌少,说道:“养那位哥儿赏了我多少,还与我多少便了。休说这位哥儿是大娘生养的。”月娘道:“比不得当时,有当家的老爹在此,如今没了老爹,将就收了罢。待洗三来,再与你一两就是了。”那蔡老娘道:“还赏我一套衣服儿罢。”拜谢去了。

月娘苏醒过来,看见箱子大开着,便骂玉箫:“贼臭肉,我便昏了,你也昏了?箱子大开着,恁乱烘烘人走,就不说锁锁儿。”玉箫道:“我只说娘锁了箱子,就不曾看见。”于是取锁来锁。玉楼见月娘多心,就不肯在他屋里,走出对着金莲说:“原来大姐姐恁样的,死了汉子,头一日就防范起人来了。”殊不知李娇儿已偷了五锭元宝在屋里去了。

当下吴二舅、贲四往尚推官家买了一付棺材板来,教匠人解锯成椁。众小厮把西门庆抬出,停当在大厅上,请了阴阳徐先生来批书。不一时,吴大舅也来了。吴二舅、众伙计都在前厅热乱,收灯卷画,盖上纸被,设放香灯几席。来安儿专一打磨。徐先生看了手,说道:“正辰时断气,合家都不犯凶煞。”请问月娘:“三日大殓,择二月十六破土,三十出殡,有四七多日子。”一面管待徐先生去了,差人各处报丧,交牌印往何千户家去,家中披孝搭棚,俱不必细说。

到三日,请僧人念倒头经,挑出纸钱去。合家大小都披麻带孝。女婿陈敬济斩衰泣杖,灵前还礼。月娘在暗房中出不来。李娇儿与玉楼陪待堂客;潘金莲管理库房,收祭桌;孙雪娥率领家人媳妇,在厨下打发各项人茶饭。傅伙计、吴二舅管帐、贲四管孝帐;来兴管厨;吴大舅与甘伙计陪待人客。蔡老娘来洗了三,月娘与了一套绸绢衣裳打发去了。就把孩儿起名叫孝哥儿,未免送些喜面。亲邻与众街坊邻舍都说:“西门庆大官人正头娘子生了一个墓生儿子,就与老子同日同时,一头断气,一头生儿,世间有这等蹊跷古怪事。”

不说众人理乱这桩事。且说应伯爵闻知西门庆没了,走来吊孝哭泣,哭了一回。吴大舅、二舅正在卷棚内看着与西门庆传影,伯爵走来,与众人见礼,说道:“可伤,做梦不知哥没了。”要请月娘拜见,吴大舅便道:“舍妹暗房出不来,如此这般,就是同日添了个娃儿。”伯爵愕然道:“有这等事!也罢也罢,哥有了个后代,这家当有了主儿了。”落后陈敬济穿着一身重孝,走来与伯爵磕头。伯爵道:“姐夫姐夫,烦恼。你爹没了,你娘儿每是死水儿了,家中凡事要你仔细。有事不可自家专,请问你二位老舅主张。不该我说,你年幼,事体还不大十分历练。”吴大舅道:“二哥,你没的说。我自也有公事,不得闲,见有他娘在。”伯爵道:“好大舅,虽故有嫂子,外边事怎么理的?还是老舅主张。自古没舅不生,没舅不长。一个亲娘舅,比不的别人。你老人家就是个都根主儿,再有谁大?”因问道:“有了发引日期没有?”吴大舅道:“择二月十六日破土,三十日出殡,也在四七之外。”不一时,徐先生来到,祭告入殓,将西门庆装入棺材内,用长命丁钉了,安放停当,题了名旌:“诰封武略将军西门公之柩”。

那日何千户来吊孝。灵前拜毕,吴大舅与伯爵陪侍吃茶,问了发引的日期。何千户分付手下该班排军,原答应的,一个也不许动,都在这里伺候。直过发引之后,方许回衙门当差。又委两名节级管领,如有违误,呈来重治。又对吴大舅说:“如有外边人拖欠银两不还者,老舅只顾说来,学生即行追治。”吊老毕,到衙门里一面行文开缺,申报东京本卫去了。

话分两头。却说来爵、春鸿同李三,一日到兖州察院,投下了书礼,宋御史见西门庆书上要讨古器批文一节,说道:“你早来一步便好。昨日已都派下各府买办去了。”寻思间,又见西门庆书中封着金叶十两,又不好违阻了的。便留下春鸿、来爵、李三在公廨驻札。随即差快手拿牌,赶回东平府批文来,封回与春鸿书中,又与了一两路费,方取路回清河县。往返十日光景。走进城,就闻得路上人说:“西门大官人死了,今日三日,家中念经做斋哩。”这李三就心生奸计,路上说念来爵、春鸿:“将此批文按下,只说宋老爷没与来。咱每都投到大街张二老爹那里去罢。你二人不去,我每人与你十两银子,到家隐住,不拿出来就是了。”那来爵见财物倒也肯了,只春鸿不肯,口里含糊应诺。

到家,见门首挑着纸钱,僧人做道场,亲朋吊丧者不计其数,这李三就分路回家去了。来爵、春鸿见吴大舅、陈敬济磕了头,问:“讨批文如何?怎的李三不来?”那来爵欲说不肯,这春鸿把宋御史书连批都拿出来,递与大舅,悉把李三路上与的十两银子,说的言语,如此这般教他隐下,休拿出来,同他投往张二官家去:“小的怎敢忘恩负义?径奔家来。”吴大舅一面走到后边,告诉月娘:“这个小的儿,就是个知恩的。叵耐李三这厮短命,见姐夫没了几日,就这等坏心。”因把这件事就对应伯爵说:“李智、黄四借契上本利还欠六百五十两银子,趁着刚才何大人分付,把这件事写纸状子,呈到衙门里,教他替俺追追这银子来,发送姐夫。他同寮间自恁要做分上,这些事儿莫道不依。”伯爵慌了,说道:“李三却不该行此事。老舅快休动意,等我和他说罢。”于是走到李三家,请了黄四来,一处计较。说道:“你不该先把银子递与小厮,倒做了管手。狐狸打不成,倒惹了一屁股臊。如今恁般,要拿文书提刑所告你每哩。常言道官官相护,何况又同寮之间,你等怎抵斗的他过!依我,不如悄悄遂二十两银子与吴大舅,只当兖州府干了事来了。我听得说,这宗钱粮他家已是不做了,把这批文难得掣出来,咱投张二官那里去罢。你每二人再凑得二百两,少不也拿不出来,再备办一张祭桌,一者祭奠大官人,二者交这银子与他。另立一纸欠结,你往后有了买卖,慢慢还他就是了。这个一举两得,又不失了人情,有个始终。”黄四道:“你说的是。李三哥,你干事忒慌速了些。”真个到晚夕,黄四同伯爵送了二十两银子到吴大舅家,如此这般,“讨批文一节,累老舅张主张主。”这吴大舅已听见他妹子说不做钱粮,何况又黑眼见了白晃晃银子,如何不应承,于是收了银子。

到次日,李智、黄四备了一张插桌,猪首三牲,二百两银子,来与西门庆祭奠。吴大舅对月娘说了,拿出旧文书,从新另立了四百两一纸欠帖,饶了他五十两,余者教他做上买卖,陆续交还。把批文交付与伯爵手内,同往张二官处合伙,上纳钱粮去了,不在话下。正是:

\[
金逢火炼方知色,人与财交便见心。
\]
有诗为证:

\[
造物于人莫强求,劝君凡事把心收。
你今贪得收人业,还有收人在后头。
\]

\newpage
%# -*- coding:utf-8 -*-
%%%%%%%%%%%%%%%%%%%%%%%%%%%%%%%%%%%%%%%%%%%%%%%%%%%%%%%%%%%%%%%%%%%%%%%%%%%%%%%%%%%%%


\chapter{潘金莲售色赴东床\KG 李娇儿盗财归丽院}


诗曰:

\[
倚醉无端寻旧约,却因惆怅转难胜。
静中楼阁深春雨,远处帘栊半夜灯。
抱柱立时风细细,绕廊行处思腾腾。
分明窗下闻裁剪,敲遍栏杆唤不应。
\]

话说西门庆死了,首七那日,却是报国寺十六众僧人做水陆。这应伯爵约会了谢希大、花子繇、祝实念、孙天化、常峙节、白赉光七人,坐在一处,伯爵先开口说:“大官人没了,今一七光景。你我相交一场,当时也曾吃过他的,也曾用过他的,也曾使过他的,也曾借过他的。今日他死了,莫非推不知道?洒土也眯眯后人眼睛儿,他就到五阎王跟前,也不饶你我。如今这等计较,你我各出一钱银子,七人共凑上七钱,办一桌祭礼,买一幅轴子,再求水先生作一篇祭文,抬了去,大官人灵前祭奠祭奠,少不的还讨了他七分银子一条孝绢来,这个好不好?”众人都道:“哥说的是。”当下每人凑出银子来,交与伯爵,整备祭物停当,买了轴子,央水秀才做了祭文。这水秀才平昔知道应伯爵这起人,与西门庆乃小人之朋,于是暗含讥刺,作就一篇祭文。伯爵众人把祭祀抬到灵前摆下,陈敬济穿孝在旁还礼。伯爵为首,各人上了香,人人都粗俗,那里晓得其中滋味。浇了奠酒,只顾把祝文宣念。其文略曰:

\[
维重和元年,岁戊戌,二月戊子期,越初三日庚寅,侍教生应伯爵、谢希大、花子繇、祝实念、孙天化、常峙节、白赉光,谨以清酌庶馐之仪,致祭于故锦衣西门大官人之灵曰:维灵生前梗直,秉性坚刚;软的不怕,硬的不降。常济人以点水,恒助人以精光。囊箧颇厚,气概轩昂。逢乐而举,遇阴伏降。锦裆队中居住,齐腰库里收藏。有八角而不用挠掴,逢虱虮而骚痒难当。受恩小子,常在胯下随帮。也曾在章台而宿柳,也曾在谢馆而猖狂。正宜撑头活脑,久战熬场,胡为罹一疾不起之殃?见今你便长伸着脚子去了,丢下小子辈,如班鸠跌脚,倚靠何方?难上他烟花之寨,难靠他八字红墙。再不得同席而儇软玉,再不得并马而傍温香。撇的人垂头落脚,闪的人牢温郎当。今特奠兹白浊,次献寸觞。灵其不昧,来格来歆。尚飨。
\]
众人祭毕,陈敬济下来还礼,请去卷棚内三汤五割,管待出门不题。

且说那日院中李家虔婆,听见西门庆死了,铺谋定计,备了一张祭桌,使了李桂卿、李桂姐坐轿子来上纸吊问。月娘不出来,都是李娇儿、孟玉楼在上房管待。李家桂卿、桂姐悄悄对李娇儿说:“俺妈说,人已是死了,你我院中人,守不的这样贞节!自古千里长棚,没个不散的筵席。教你手里有东西,悄悄教李铭稍了家去防后。你还恁傻!常言道:‘扬州虽好,不是久恋之家。’不拘多少时,也少不的离他家门。”那李娇儿听记在心。

不想那日韩道国妻王六儿,亦备了张祭桌,乔素打扮,坐轿子来与西门庆烧纸。在灵前摆下祭祀,只顾站着。站了半日,白没个人儿出来陪待。原来西门庆死了,首七时分,就把王经打发家去不用了。小厮每见王六儿来,都不敢进去说。那来安儿不知就里,到月娘房里,向月娘说:“韩大婶来与爹上纸,在前边站了一日了,大舅使我来对娘说。”这吴月娘心中还气忿不过,便喝骂道:“怪贼奴才,不与我走,还来甚么韩大婶、\textuni{23B48}大婶,贼狗攮的养汉淫妇,把人家弄的家败人亡,父南子北,夫逃妻散的,还来上甚么\textuni{23B48}纸!”一顿骂的来安儿摸门不着,来到灵前。吴大舅问道:“对后边说了不曾?”来安儿把嘴谷都着不言语。问了半日,才说:“娘稍出四马儿来了。”这吴大舅连忙进去,对月娘说:“姐姐,你怎么这等的?快休要舒口!自古人恶礼不恶。他男子汉领着咱偌多的本钱,你如何这等待人?好名儿难得,快休如此。你就不出去,教二姐姐、三姐姐好好待他出去,也是一般。做甚么恁样的,教人说你不是。”那月娘见他哥这样说,才不言语了。良久,孟玉楼出来,还了礼,陪他在灵前坐的。只吃一钟茶,妇人也有些省口,就坐不住,随即告辞起身去了。正是:

\[
谁人汲得西江水,难免今朝一面羞。
\]

那李桂卿、桂姐、吴银儿都在上房坐着,见月娘骂韩道国老婆淫妇长、淫妇短,砍一株损百枝,两个就有些坐不住,未到日落,就要家去。月娘再三留他姐儿两个:“晚夕伙计每伴宿,你每看了提偶,明日去罢。”留了半日,桂姐、银姐不去了,只打发他姐姐桂卿家去了。到了晚夕,僧人散了,果然有许多街坊、伙计、主管,乔大户、吴大舅、吴二舅、沈姨父、花子繇、应伯爵、谢希大、常峙节,也有二十余人,叫了一起偶戏,在大卷棚内,摆设酒席伴宿。提演的是“孙荣、孙华杀狗劝夫”戏文。堂客都在灵旁厅内,围着帏屏,放下帘来,摆放桌席,朝外观看。李铭、吴惠在这里答应,晚夕也不家去了。不一时,众人都到齐了。祭祀已毕,卷棚内点起烛来,安席坐下,打动鼓乐,戏文上来。直搬演到三更天气,戏文方了。

原来陈敬济自从西门庆死后,无一日不和潘金莲两个嘲戏,或在灵前溜眼,帐子后调笑。于是赶人散一乱,众堂客都往后边去了,小厮每都收家活,这金莲赶眼错,捏了敬济一把,说道:“我儿,你娘今日成就了你罢。趁大姐在后边,咱就往你屋里去罢。”敬济听了,得不的一声,先往屋里开门去了。妇人黑影里,抽身钻入他房内,更不答话,解开裤子,仰卧在炕上,双凫飞首,教陈敬济好耍。正是:色胆如天怕甚事,鸳帏云雨百年情。真个是:

\[
二载相逢,一朝配偶;数年姻眷,一旦和谐。一个柳腰款摆,一个玉茎忙舒。耳边诉雨意云情,枕上说山盟海誓。莺恣蝶采,旖妮搏弄百千般;狂雨羞云,娇媚施逞千万态。一个不住叫亲亲,一个搂抱呼达达。得多少柳色乍翻新样绿,花容不减旧时红。
\]

霎时云雨了毕,妇人恐怕人来,连忙出房,往后边去了。到次日,这小伙儿尝着这个甜头儿,早辰走到金莲房来,金莲还在被窝里未起来。从窗眼里张看,见妇人被拥红云,粉腮印玉,说道:“好管库房的,这咱还不起来!今日乔亲家爹来上祭,大娘分付把昨日摆的李三、黄四家那祭桌收进来罢。你快些起来,且拿钥匙出来与我。”妇人连忙教春梅拿钥匙与敬济,敬济先教春梅楼上开门去了。妇人便从窗眼里递出舌头,两个咂了一回。正是得多少脂香满口涎空咽,甜唾颙心溢肺奸。有词为证:

\[
恨杜鹃声透珠帘。心似针签,情似胶粘。我则见笑脸腮窝愁粉黛,瘦损春纤宝髻乱,云松翠钿。睡颜酡,玉减红添。檀口曾沾。到如今唇上犹香,想起来口内犹甜。
\]

良久,春梅楼上开了门,敬济往前边看搬祭祀去了。不一时,乔大户家祭来摆下。乔大户娘子并乔大户许多亲眷,灵前祭毕。吴大舅、吴二舅、甘伙计陪侍,请至卷棚内管待。李铭、吴惠弹唱。那日郑爱月儿家也来上纸吊孝。月娘俱令玉楼打发了孝裙束腰,后边与堂客一同坐的。郑爱月儿看见李桂姐、吴银姐都在这里,便嗔他两个不对他说:“我若知道爹没了,有个不来的!你每好人儿,就不会我会儿去。”又见月娘生了孩儿,说道:“娘一喜一忧。惜乎爹只是去世太早了些儿,你老人家有了主儿,也不愁。”月娘俱打发了孝,留坐至晚方散。

到二月初三日,西门庆二七,玉皇庙吴道官十六众道士,在家念经做法事。那日衙门中何千户作创,约会了刘、薛二内相,周守备、荆都统、张团练、云指挥等数员武官,合着上了坛祭。月娘这里请了乔大户、吴大舅、应伯爵来陪待,李铭、吴惠两个小优儿弹唱,卷棚管待去了。俱不必细说。到晚夕念经送亡。月娘分付把李瓶儿灵床连影抬出去,一把火烧了。将箱笼都搬到上房内堆放。奶子如意儿并迎春收在后边答应,把绣春与了李娇儿房内使唤。将李瓶儿那边房门,一把锁锁了。可怜正是:画栋雕梁犹未干,堂前不见痴心客。有诗为证:

\[
襄王台下水悠悠,一种相思两样愁。
月色不如人事改,夜深还到粉墙头。
\]

那时李铭日日假以孝堂助忙,暗暗教李娇儿偷转东西与他掖送到家,又来答应,常两三夜不往家去,只瞒过月娘一人眼目。吴二舅又和李娇儿旧有首尾,谁敢道个不字。初九日念了三七经,月娘出了暗房,四七就没曾念经。十二日,陈敬济破了土回来。二十日早发引,也有许多冥器纸札,送殡之人终不似李瓶儿那时稠密。临棺材出门,也请了报恩寺朗僧官起棺,坐在轿上,捧的高高的,念了几句偈文。念毕,陈敬济摔破纸盆,棺材起身,合家大小孝眷放声号哭。吴月娘坐魂轿,后面坐堂客上轿,都围随材走,径出南门外五里原祖茔安厝。陈敬济备了一匹尺头,请云指挥点了神主,阴阳徐先生下了葬。众孝眷掩土毕。山头祭桌,可怜通不上几家,只是吴大舅、乔大户、何千户、沈姨夫、韩姨夫与众伙计五六处而已。吴道官还留下十二众道童回灵,安于上房明间正寝。阴阳洒扫已毕,打发众亲戚出门。吴月娘等不免伴夫灵守孝。一日暖了墓回来,答应班上排军节级,各都告辞回衙门去了。西门庆五七,月娘请了薛姑子、王姑子、大师父、十二众尼僧,在家诵经礼忏,超度夫主生天。吴大妗子并吴舜臣媳妇,都在家中相伴。

原来出殡之时,李桂卿同桂姐在山头,悄悄对李娇儿如此这般:“妈说,你摸量你手中没甚细软东西,不消只顾在他家了。你又没儿女,守甚么?教你一场嚷乱,登开了罢。昨日应二哥来说,如今大街坊张二官府,要破五百两金银,娶你做二房娘子,当家理纪。你那里便图出身,你在这里守到老死,也不怎么。你我院中人家,弃旧迎新为本,趋火附势为强,不可错过了时光。”这李娇儿听记在心,过了西门庆五七之后,因风吹火,用力不多。不想潘金莲对孙雪娥说,出殡那日,在坟上看见李娇儿与吴二舅在花园小房内,两个说话来。春梅孝堂中又亲眼看见李娇儿帐子后递了一包东西与李铭,塞在腰里,转了家去。嚷的月娘知道,把吴二舅骂了一顿,赶去铺子里做买卖,再不许进后边来。分付门上平安,不许李铭来往。这花娘恼羞变成怒,正寻不着这个由头儿哩。一日因月娘在上房和大妗子吃茶,请孟玉楼,不请他,就恼了,与月娘两个大闹大嚷,拍着西门庆灵床子,啼啼哭哭,叫叫嚎嚎,到半夜三更,在房中要行上吊。丫头来报与月娘。月娘慌了,与大妗子计议,请将李家虔婆来,要打发他归院。虔婆生怕留下他衣服头面,说了几句言语:“我家人在你这里做小伏低,顶缸受气,好容易就开交了罢!须得几十两遮羞钱。”吴大舅居着官,又不敢张主,相讲了半日,教月娘把他房中衣服、首饰、箱笼、床帐、家活尽与他,打发出门。只不与他元宵、绣春两个丫头去。李娇儿生死要这两个丫头。月娘生死不与他,说道:“你倒好,买良为娼。”一句慌了鸨子,就不敢开言,变做笑吟吟脸儿,拜辞了月娘,李娇儿坐轿子,抬的往家去了。

看官听说,院中唱的,以卖俏为活计,将脂粉作生涯;早辰张风流,晚夕李浪子;前门进老子,后门接儿子;弃旧怜新,见钱眼开,自然之理。饶君千般贴恋,万种牢笼,还锁不住他心猿意马。不是活时偷食抹嘴,就是死后嚷闹离门。不拘几时,还吃旧锅粥去了。正是:蛇入筒中曲性在,鸟出笼轻便飞腾。有诗为证:

\[
堪笑烟花不久长,洞房夜夜换新郎。
两只玉腕千人枕,一点朱唇万客尝。
造就百般娇艳态,生成一片假心肠。
饶君总有牢笼计,难保临时思故乡。
\]

月娘打发李娇儿出门,大哭了一场。众人都在旁解劝,潘金莲道:“姐姐,罢,休烦恼了。常言道,娶淫妇,养海青,食水不到想海东。这个都是他当初干的营生,今日教大姐姐这等惹气。”

家中正乱着,忽有平安来报:“巡盐蔡老爹来了,在厅上坐着哩,我说家老爹没了。他问没了几时了,我回正月二十一日病故,到今过了五七。他问有灵没灵,我回有灵,在后边供养着哩。他要来灵前拜拜,我来对娘说。”月娘分付:“教你姐夫出去见他。”不一时,陈敬济穿上孝衣出去,拜见了蔡御史。良久,后边收拾停当,请蔡御史进来西门庆灵前参拜了。月娘穿着一身重孝,出来回礼,再不交一言,就让月娘说:“夫人请回房。”又向敬济说道:“我昔时曾在府相扰,今差满回京去,敬来拜谢拜谢,不期作了故人。”便问:“甚么病症?”陈敬济道:“是痰火之疾。”蔡御史道:“可伤,可伤。”即唤家人上来,取出两匹杭州绢,一双绒袜,四尾白鲞,四罐蜜饯,说道:“这些微礼,权作奠仪罢。”又拿出五十两一封银子来,“这个是我向日曾贷过老先生些厚惠,今积了些俸资奉偿,以全终始之交。”分付平安道:“大官,交进房去。”敬济道:“老爹忒多计较了。”月娘说:“请老爹前厅坐。”蔡御史道:“也不消坐了。拿茶来,吃了一钟就是了。”左右须臾拿茶上来。蔡御史吃了,扬长起身上轿去了。月娘得了这五十两银子,心中又是那欢喜,又是那惨戚。想有他在时,似这样官员来到,肯空放去了?又不知吃酒到多咱晚。今日他伸着脚子,空有家私,眼看着就无人陪待。正是:

\[
人得交游是风月,天开图画即江山。
\]

话说李娇儿到家,应伯爵打听得知,报与张二官知,就拿着五两银子来,请他歇了一夜。原来张二官小西门庆一岁,属兔的,三十二岁了。李娇儿三十四岁,虔婆瞒了六岁,只说二十八岁,教伯爵瞒着。使了三百两银子,娶到家中,做了二房娘子。祝实念、孙寡嘴依旧领着王三官儿,还来李家行走,与桂姐打热,不在话下。

伯爵、李三、黄四借了徐内相五千两银子,张二官出了五千两,做了东平府古器这批钱粮,逐日宝鞍大马,在院内摇摆。张二官见西门庆死了,又打点了上千两金银,往东京寻了枢密院郑皇亲人情,对堂上朱太尉说,要讨提刑所西门庆这个缺。家中收拾买花园,盖房子。应伯爵无日不在他那边趋奉,把西门庆家中大小之事,尽告诉与他,说:“他家中还有第五个娘子潘金莲,排行六姐,生的上画儿般标致,诗词歌赋,诸子百家,拆牌道字,双陆象棋,无不通晓。又写的一笔好字,弹的一手好琵琶。今年不上三十岁,比唱的还乔。”说的那张二官心中火动,巴不的就要了他,便问道:“莫非是当初卖炊饼的武大郎那老婆么?”伯爵道:“就是他。占来家中,今也有五六年光景,不知他嫁人不嫁。”张二官道:“累你打听着,待有嫁人的声口,你来对我说,等我娶了罢。”伯爵道:“我身子里有个人,在他家做家人,名来爵儿。等我对他说,若有出嫁声口,就来报你知道。难得你娶过他这个人来家,也强似娶个唱的。当时西门庆大官人在时,为娶他,不知费了许多心。大抵物各有主,也说不的,只好有福的匹配,你如有了这般势耀,不得此女貌,同享荣华,枉自有许多富贵。我只叫来爵儿密密打听,但有嫁人的风缝儿,凭我甜言美语,打动春心,你却用几百两银子,娶到家中,尽你受用便了。”看官听说,但凡世上帮闲子弟,极是势利小人。当初西门庆待应伯爵如胶似漆,赛过同胞弟兄,那一日不吃他的,穿他的,受用他的。身死未几,骨肉尚热,便做出许多不义之事。正是画虎画皮难画骨,知人知面不知心。有诗为证:

\[
昔年音气似金兰,百计趋奉不等闲。
自从西门身死后,纷纷谋妾伴人眠。
\]

\newpage
%# -*- coding:utf-8 -*-
%%%%%%%%%%%%%%%%%%%%%%%%%%%%%%%%%%%%%%%%%%%%%%%%%%%%%%%%%%%%%%%%%%%%%%%%%%%%%%%%%%%%%


\chapter{韩道国拐财远遁\KG 汤来保欺主背恩}


诗曰:

\[
燕入非傍舍,鸥归只故池。断桥无复板,卧柳自生枝。
遂有山阳作,多惭鲍叔知。素交零落尽,白首泪双垂。
\]

话说韩道国与来保,自从拿着西门庆四千两银子,江南买货物,到于扬州,抓寻苗青家内宿歇。苗青见了西门庆手札,想他活命之恩,尽力趋奉。又讨了一个女子,名唤楚云,养在家里,要送与西门庆,以报其恩。韩道国与来保两个且不置货,成日寻花问柳,饮酒宿妇。只到初冬天气,景物萧瑟,不胜旅思。方才将银往各处买布匹,装在扬州苗青家安下,待货物买完起身。先是韩道国请个表子,是扬州旧院王玉枝儿,来保便请了林彩虹妹子小红。一日,请扬州盐客王海峰和苗青游宝应湖,游了一日,归到院中。又值玉枝儿鸨子生日,这韩道国又邀请众人,摆酒与鸨子王一妈做生日。使后生胡秀,请客商汪东桥与钱晴川两个,白不见到。不一时,汪东桥与钱晴川就同王海峰来了。至日落时分,胡秀才来,被韩道国带酒骂了两句,说:“这厮不知在那里噇酒,噇到这咱才来,口里喷出来的酒气。客人到先来了这半日,你不知那里来,我到明日定和你算帐。”那胡秀把眼斜瞅着他,走到下边,口里喃喃呐呐,说:“你骂我,你家老婆在家里仰扇着挣,你在这里合蓬着丢!宅里老爹包着你家老婆,\textuni{34B2}的不值了,才交你领本钱出来做买卖。你在这里快活,你老婆不知怎么受苦哩!得人不化白出你来,你落得为人就勾了。”对玉枝儿鸨子只顾说。鸨子便拉出他院子里,说:“胡官人,你醉了,你往房里睡去罢。”那胡秀大吆大喝,白不肯进房。不料韩道国正陪众客商在席上吃酒,听见胡秀口内放屁辣臊,心中大怒,走出来踢了他两脚,骂道:“贼野囚奴,我有了五分银子,雇你一日,怕寻不出人来!”即时赶他去。那胡秀那里肯出门,在院子内声叫起来,说道:“你如何赶我?我没坏了管帐事!你倒养老婆,倒赶我,看我到家说不说!”被来保劝住韩道国,一手扯他过一边,说道:“你这狗骨头,原来这等酒硬!”那胡秀道:“叔叔,你老人家休管他。我吃甚么酒来,我和他做一做。”被来保推他往屋里挺觉去了。正是:

\[
酒不醉人人自醉,色不迷人人自迷。
\]

来保打发胡秀房里睡去不题。韩道国恐怕众客商耻笑,和来保席上觥筹交错,递酒哄笑。林彩虹、小红姊妹二人并王玉枝儿三个唱的,弹唱歌舞,花攒锦簇,行令猜枚,吃至三更方散。次日,韩道国要打胡秀,胡秀说:“小的通不晓一字。”道国被苗青做好做歹劝住了。

话休饶舌。有日货物置完,打包装载上船。不想苗青讨了送西门庆的那女子楚云,忽生起病来,动身不得。苗青说:“等他病好了,我再差人送了来罢。”只打点了些人事礼物,抄写书帐,打发二人并胡秀起身。王玉枝并林彩虹姊妹,少不的置酒马头,作别饯行。从正月初十日起身,一路无词。一日到临清闸上,这韩道国正在船头站立,忽见街坊严四郎,从上流坐船而来,往临清接官去。看见韩道国,举手说:“韩西桥,你家老爹从正月间没了。”说毕,船行得快,就过去了。这韩道国听了此言,遂安心在怀,瞒着来保不说。不想那时河南、山东大旱,赤地千里,田蚕荒芜不收,棉花布价一时踊贵,每匹布帛加三利息,各处乡贩都打着银两远接,在临清一带马头迎着客货而买。韩道国便与来保商议:“船上布货约四千余两,见今加三利息,不如且卖一半,又便宜钞关纳税,就到家发卖也不过如此。遇行市不卖,诚为可惜。”来保道:“伙计所言虽是,诚恐卖了,一时到家,惹当家的见怪,如之奈何?”韩道国便说:“老爹见怪,都在我身上。”来保强不过他,就在马头上,发卖了一千两布货。韩道国说:“双桥,你和胡秀在船上等着纳税,我打旱路同小郎王汉,打着这一千两银子,先去报老爹知道。”来保道:“你到家,好歹讨老爹一封书来,下与钞关钱老爹,少纳税钱,先放船行。”韩道国应诺。同小郎王汉装成驮垛,往清河县家中来。

有日进城,在瓮城南门里,日色渐落,忽撞遇着坟的张安,推着车辆酒米食盐,正出南门。看见韩道国,便叫:“韩大叔,你来家了。”韩道国看见他带着孝,问其故,张安说:“老爹死了,明日三月初九日断七。大娘交我拿此酒米食盒往坟上去,明日与老爹烧纸。”这韩道国听了,说:“可伤,可伤!果然路上行人口似碑,话不虚传。”打头口径进城中。到了十字街上,心中算计:“且住。有心要往西门庆家去,况今他已死了,天色又晚,不如且归家停宿一宵,和浑家商议了,明日再去不迟。”于是和王汉打着头口,径到狮子街家中。二人下了头口,打发赶脚人回去,叫开门,王汉搬行李驮垛进入堂中,径到狮子街家中。二人下了头口,打发赶脚人回去,叫开门,王汉搬行李驮垛进入堂中。老婆一面迎接入门,拜了佛祖。王六儿替他脱衣坐下,丫头点茶吃。韩道国先告诉往回一路之事,道:“我在路上撞遇严四哥与张安,才知老爹死了。好好的,怎的就死了?”王六儿道:“天有不测风云,人有暂时祸福。谁人保得无常!”韩道国一面把驮垛打开,取出他江南置的许多衣裳细软等物,并那一千两银子,一封一封都放在炕上。老婆打开看,都是白光光雪花银两,便问:“这是那里的?”韩道国说:“我在路上闻了信,就先卖了这一千两银子来了。”又取出两包梯己银子一百两,因问老婆:“我去后,家中他也看顾你不曾?”王六儿道:“他在时倒也罢了,如今你这银子还送与他家去?”韩道国道:“正是要和你商议,咱留下些,把一半与他如何?”老婆道:“呸,你这傻奴才料,这遭再休要傻了。如今他已是死了,这里无人,咱和他有甚瓜葛?不急你送与他一半,交他招暗道儿,问你下落。到不如一狠二狠,把他这一千两,咱雇了头口,拐了上东京,投奔咱孩儿那里。愁咱亲家太师爷府中,安放不下你我!”韩道国道:“丢下这房子,急切打发不出去,怎了?”老婆道:“你看没才料!何不叫将第二个来,留几两银子与他,就叫他看守便了。等西门庆家人来寻你,保说东京咱孩儿叫了两口去了。莫不他七个头八个胆,敢往太师府中寻咱们去?就寻去,你我也不怕他。”韩道国道:“争奈我受大官人好处,怎好变心的?没天理了!”老婆道:“自古有天理到没饭吃哩。他占用着老娘,使他这几两银子,不差甚么。想着他孝堂里,我到好意备了一张插桌三牲,往他家烧纸。他家大老婆那不贤良的淫妇,半日不出来,在屋里骂的我好讪的。我出又出不来,坐又坐不住,落后他第三个老婆出来陪我坐,我不去坐,就坐轿子来家了,想着他这个情儿,我也该使他这几两银子。”一席话,说得韩道国不言语了。夫妻二人,晚夕计议已定。到次日五更,叫将他兄弟韩二来,如此这般,叫他看守房子,又把与他一二十两银子盘缠。那二捣鬼千肯万肯,说:“哥嫂只顾去,等我打发他。”这韩道国就把王汉小郎并两个丫头,也跟他带上东京去。雇了二十辆车,把箱笼细软之物都装在车上。投天明出西门,径上东京去了。正是:

\[
撞碎玉笼飞彩凤,顿开金锁走蛟龙。
\]

这里韩道国夫妇东京去了不题。单表吴月娘次日带孝哥儿,同孟玉楼、潘金莲、西门大姐、奶子如意儿、女婿陈敬济,往坟上与西门庆烧纸。张安就告诉月娘,昨日撞见韩大叔来家一节,月娘道:“他来了,怎的不到我家来?只怕他今日来。”在坟上刚烧了纸,坐了没多回,老早就起身来家。使陈敬济往他家,“叫韩伙计去,问他船到那里了?”初时叫着不闻人言,次则韩二出来,说:“俺侄女儿东京叫了哥嫂去了,船不知在那里。”让陈敬济回月娘。月娘不放心,使敬济骑头口往河下寻船。去了一日,到临清马头船上,寻着来保船只。来保问:“韩伙计先打了一千两银子家去了。”敬济道:“谁见他来?张安看见他进城,次日坟上来家,大娘使我问他去,他两口子夺家连银子都拐的上东京去了。如今爹死了,断七过了,大娘不放心,使我来找寻船只。”这来保口中不言,心内暗道:“这天杀,原来连我也瞒了,嗔道路上定要卖这一千两银子,干净要起毛心。正是人面咫尺,心隔千里。”这来保见西门庆已死,也安心要和他一路。把敬济小伙儿引诱在马头上各唱店中、歌楼上饮酒,请表子顽耍。暗暗船上搬了八百两货物,卸在店家房内,封记了。一日钞关上纳了税,放船过来,在新河口起脚装车,往清河县城里来,家中东厢房卸下。

自从西门庆死了,狮子街丝绵铺已关了。对门段铺,甘伙计、崔本卖了银两都交付明白,各辞归房去了。房子也卖了,止有门首解当、生药铺,敬济与傅伙坟开着。原来这来保妻惠祥,有个五岁儿子,名僧宝儿。韩道国老婆王六儿有个侄女儿四岁,二人割衿做了亲家。家中月娘通不知道。这来保交卸了货物,就一口把事情都推在韩道国身上,说他先卖了二千两银子来家。那月娘再三使他上东京,问韩道国银子下落。被他一顿话说:“咱早休去!一个太师老爷府中,谁人敢到?没的招事惹非。得他不来寻你,咱家念佛。到没的招惹虱子头上挠!”月娘道:“翟亲家也亏咱家替他保亲,莫不看些分上儿。”来保道:“他家女儿见在他家得时,他敢只护他娘老子,莫不护咱不成?此话只好在家对我说罢了,外人知道,传出去到不好了。只当丢这几两银子罢,更休题了。”月娘听了无法,也只得罢了。又交他会买头,发卖布货。他会了主儿来,月娘交陈敬济兑银讲价钱,主儿都不服,拿银出去了。来保硬说:“姐夫,你不知买卖甘苦。俺在江湖上走的多,晓得行情,宁可卖了悔,休要悔了卖。这货来家得此价钱就勾了。你十分把弓儿拽满,迸了主儿,显的不会做生意。我不是托大说话,你年少不知事体。我莫不胳膊儿往外撇?不如卖吊了,是一场事。”那敬济听了,使性儿不管了。他也不等月娘来分付,匹手夺过算盘,邀回主儿来。把银子兑了二千余两,一件件交付与敬济经手,交进月娘收了,推货出门。月娘与了他二三十两银子房中盘缠,他便故意儿昂昂大意不收,说道:“你老人家还收了。死了爹,你老人家死水儿,自家盘缠,又与俺们做甚?你收了去,我决不要。”一日晚夕,外边吃的醉醉儿,走进月娘房中,搭伏着护炕,说念月娘:“你老人家青春少小,没了爹,你自家守着这点孩子儿,不害孤另么?”月娘一声儿没言语。

一日,东京翟管家寄书来,知道西门庆死了,听见韩道国说,他家中有四个弹唱出色女子,该多少价钱,说了去,兑银子来,要载到京中答应老太太。月娘见书,慌了手脚,叫将来保来计议,与他去好,不与他去好。来保进入房中,也不叫娘,只说:“你娘子人家不知事,不与他去,就惹下祸了。这个都是过世老头儿惹的,恰似卖富一般,但摆酒请人,就叫家乐出去,有个不传出去的?何况韩伙计女儿又在府中答应老太太,有个不说的?我前日怎么说来,今果然有此勾当钻出来。你不与他,他裁派府县,差人坐名儿来要,不怕你不双手儿奉与他,还是迟了。难说四个都与他,不如今日胡乱打发两个与他,还做面皮。”这月娘沉吟半晌。孟玉楼房中兰香,与金莲房中春梅,都不好打发。绣春又要看哥儿,不出门。因问他房中玉箫与迎春,情愿要去。以此就差来保,雇车辆装载两个女子,往东京太师府中来。不料来保这厮,在路上把这两个女子都奸了。有日到东京,会见韩道国夫妇,把前后事都说了。韩道国谢来保道:“若不是亲戚看顾我,在家阻住,我虽然不怕他,也未免多一番唇舌。”翟谦看见迎春、玉箫两个都生的好模样儿,一个会筝,一个会弦子,都不上十七八岁,进入府中伏侍老太太,赏出两锭元宝来。这来保还克了一锭,到家只拿出一锭元宝来与月娘,还将言语恐吓月娘说:“若不是我去,还不得他这锭元宝拿家来。你还不知,韩伙计两口儿在那府中好不受用富贵,独自住着一所宅子,呼奴使婢,坐五行三。翟管家以老爹呼之,他家女儿韩爱姐,日逐上去答应老太太,寸步不离,要一奉十,拣口儿吃用,换套穿衣。如今又会写,又会算,福至心灵,出落得好长大身材,姿容美貌。前日出来见我,打扮得如琼林玉树一般,百伶百俐,一口一声叫我保叔。如今咱家这两个家乐到那里,还在他手里坟针线哩。”说毕,月娘还甚是知感他不尽。打发他酒馔吃了,与他银子又不受,拿了一匹段子与他妻惠祥做衣服穿,不在话下。

这来保一日同他妻弟刘仓,往临清马头上,将封寄店内布货,尽行卖了八百两银子,暗卖下一所房子,就在刘仓右边门首,就开杂货铺儿。他便日逐随倚祀会茶。他老婆惠祥,要便对月娘说,假推往娘家去。到房子里,从新换了头面衣服,珠子箍儿,插金戴银,往王六儿娘家王母猪家扳亲家,行人情,坐轿看他家女儿去来。到房子里,依旧换了惨淡衣裳,才往西门庆家中来,只瞒过月娘一人不知。来保这厮,常时吃醉了,来月娘房中,嘲话调戏,两番三次。不是月娘为人正大,也被他说念的心邪,上了道儿。又有一般小厮媳妇,在月娘根前,说他媳妇子在外与王母猪作亲家,插金戴银,行三坐五。潘金莲也对月娘说了几次,月娘不信。

惠祥听了此言,在厨房中骂大骂小。来保便装胖字蠢,自己夸奖,说众人:“你每只好在家里说炕头子上嘴罢了!相我水皮子上,顾瞻将家中这许多银子货物来家。若不是我,都吃韩伙计老年箝嘴,拐了往东京去。只呀的一声,干丢在水里也不响。如今还不道俺每一个‘是’,说俺转了主子的钱了,架俺一篇是非。正是割股的也不知,烯香的也不知。自古信人调,丢了瓢。”媳妇子惠祥便骂:“贼嚼舌根的淫妇!说俺两口子转的钱大了,在外行三坐五扳亲。老道出门,问我姊那里借的几件子首饰衣裳,就说是俺落的主子银子治的!要挤撮俺两口子出门,也不打紧。等俺每出去,料莫天也不着饿水鸦儿吃草。我洗净着眼儿,看你这些淫妇奴才,在西门庆家里住牢着!”月娘见他骂大骂小,寻由头儿和人嚷,闹上吊;汉子又两番三次,无人处在根前无礼,心里也气得没入脚处,只得交他两口子搬离了家门。这来保就大剌剌和他舅子开起个布铺来,发卖各色细布,日逐会亲友,行人情,不在话下。正是:

\[
势败奴欺主,时衰鬼弄人。
\]

\newpage
%# -*- coding:utf-8 -*-
%%%%%%%%%%%%%%%%%%%%%%%%%%%%%%%%%%%%%%%%%%%%%%%%%%%%%%%%%%%%%%%%%%%%%%%%%%%%%%%%%%%%%


\chapter{陈敬济弄一得双\KG 潘金莲热心冷面}


词曰:

\[
闻道双衔凤带,不妨单着鲛绡。夜香知为阿谁烧?怅望水沉烟枭。云鬓风前绿卷,玉颜想处红潮,莫交空负可怜宵,月下双湾步俏。\named{右调《西江月》}
\]

话说潘金莲与陈敬济,自从在厢房里得手之后,两个人尝着甜头儿,日逐白日偷寒,黄昏送暖。或倚肩嘲笑,或并坐调情,掐打揪撏,通无忌惮。或有人跟前不得说话,将心事写了,搓成纸条儿,丢在地下,你有话传与我,我有话传与你。一日,四月天气,潘金莲将自己袖的一方银丝汗贴儿,裹着一个纱香袋儿,里面装一缕头发并些松柏儿,封的停当,要与敬济。不想敬济不在厢房内,遂打窗眼内投进去。后敬济进房,看见弥封甚厚,打开却是汗巾香袋儿,纸上写一词,名《寄生草》:

\[
将奴这银丝帕,并香囊寄与他。当初结下青丝发。松柏儿要你常牵挂,泪珠儿滴写相思话。夜深灯照的奴影儿孤,休负了夜深潜等荼縻架。
\]
敬济见词上约他在荼縻架下等候,私会佳期。随即封了一柄湘妃笔金扇儿,亦写了一词在上回答他,袖入花园内。不想月娘正在金莲房中坐着,这敬济三不知,走进角门就叫:“可意人在家不在?”这金莲听见是他语音,恐怕月娘听见决撒了,连忙掀帘子走出来。看着他摆手儿,佯说:“我道是谁,原来是陈姐夫来寻大姐。大姐刚才在这里,和他每往花园亭子上摘花儿去了。”这敬济见有月娘在房里,就把物事暗暗递与妇人袖了,他就出去了。月娘便问:“陈姐夫来做甚么?”金莲道:“他来寻大姐,我回他往花园中去了。”以此瞒过月娘。少顷,月娘起身回后边去了。金莲向袖中取出拆开,却是湘妃竹金扇儿一柄,上面一种青蒲,半溪流水,有《水仙子》一首词儿:

\[
紫竹白纱甚逍遥,绿囗青蒲巧制成,金铰银钱十分妙。美人儿堪用着,遮炎天少把风招。有人处常常袖着,无人处慢慢轻摇,休教那俗人见偷了。
\]

妇人看见其词,到于晚夕月上时,早把春梅、秋菊两个丫头打发些酒与他吃,关在那边炕屋睡。然后自在房中,绿半启,绛烛高烧,收拾床铺衾枕,薰香澡牝,独立木香棚下,专等敬济来赴佳期。西门大姐那夜恰好被月娘请去后边,听王姑子宣卷去了,只有元宵儿在屋里。敬济梯己与了他一方手帕,分付他:“看守房中,我往你五娘那边下棋去。等大姑娘进来,你快来。”元宵儿应诺了。敬济得手,走来花园中,只见花筛月影,参差提成映。走到荼縻架下,远望见妇人摘去冠儿,乱挽乌云,悄悄在木香棚下独立。这敬济猛然从荼縻架下突出,双手把妇人抱住。把妇人唬了一跳,说:“呸,小短命!猛然外事出来,唬了我一跳。早是我,你搂便将就罢了,若是别人,你也恁胆大搂起来?”敬济吃得半酣儿,笑道:“早是搂了你,就错搂了红娘,也是没奈何。”两个于是相搂相抱,携手进入房中。房中荧煌煌掌着灯烛,桌上设着酒肴,一面顶了角门,并肩而坐饮酒。妇人便问:“你来,大姐在那里?”敬济道:“大姐后边听宣卷去了,我分付下元宵儿,有事来这里叫,我只说在这里下棋。”说毕,上欢笑做一处。饮酒多时,常言“风流茶说合,酒是色媒人”,不觉竹叶穿心,桃花上脸,一个嘴儿相亲,一个腮儿厮揾,罩了灯,上床交接。有《六娘子》小词为证:

\[
入门来,奴搂抱在怀。奴把锦被儿伸开,俏冤家顽的十分怪。嗏,将奴脚儿抬。脚儿抬,揉乱了乌云,\textuni{4BFC}髻儿歪。
\]

两人云雨才毕,只听得元宵叫门说:“大姑娘进房中来了。”这敬济慌的穿衣去了。正是:

\[
狂蜂浪蝶有时见,飞入梨花无处寻。
\]

原来潘金莲那边三间楼上,中间供养佛像,两边稍间堆放生药香料。两个自此以后,情沾肺腑,意密如漆,无日不相会做一处。一日也是合当有事,潘金莲早辰梳妆打扮,走来楼上观音菩萨前烧香。不想陈敬济正拿钥匙上楼,开库房门拿药材香料,撞遇在一处。这妇人且不烧香,见楼上无人,两个搂抱着亲嘴咂舌,一个叫“亲亲五娘”,一个呼“心肝短命”,因说:“趁无人,咱在这里干了罢。”一面解褪衣裤,就在一张春凳上双凫飞肩,灵根半入,不胜绸缪。当初没巧不成话,两个正干得好,不防春梅正上楼来,拿盒子取茶叶看见。两个凑手脚不迭,都吃了一惊。春梅恐怕羞了他,连忙倒退回身子,走下胡梯。慌的敬济兜小衣不迭,妇人穿上裙子,忙叫春梅:“我的好姐姐,你上来,我和你说话。”那春梅于是走上楼来。金莲道:“我的好姐姐,你姐夫不是别人,我今叫你知道了罢。俺两个情孚意合,拆散不开。你千万休对人说,只放在你心里。”春梅便说:“好娘,说那里话。奴伏侍娘这几年,岂不知娘心腹,肯对人说!”妇人道:“你若肯遮盖俺们,趁你姐夫在这里,你也过来和你姐夫睡一睡,我方信你。你若不肯,只是不可怜见俺每了。”那春梅把脸羞的一红一白,只得依他。卸下湘裙,解开裤带,仰在凳上,尽着这小伙儿受用。有这等事!正是:明珠两颗皆无价,可奈檀郎尽得钻。有《红绣鞋》为证:

\[
假认做女婿亲厚,往来和丈母歪偷。人情里包藏鬼胡油。明讲做儿女礼,暗结下燕莺俦,他两个见今有。
\]

当下尽着敬济与春梅耍完,大家方才走散。自此以后,潘金莲便与春梅打成一家,与这小伙儿暗约偷期,非只一日,只背着秋菊。

六月初一日,潘姥姥老病没了,有人来说。吴月娘买一张插桌,三牲冥纸,教金莲坐轿子往门外探丧祭祀,去了一遭回来。到次日,六月初三日,金莲起来得早,在月娘房里坐着,说了半日话出来,走在大厅院子里墙根下,急了溺尿。正撩起裙子,蹲踞溺尿。原来西门庆死了,没人客来往,等闲大厅仪门只是关闭不开。敬济在东厢房住,才起来,忽听见有人在墙根溺的尿刷刷的响,悄悄向窗眼里张看,却不想是他,便道:“是那个撒野,在这里溺尿?撩起衣服,看溅湿了裙子?”这妇人连忙系上裙子,走到窗下问道:“原来你在屋里,这咱才起来,好自在。大姐没在房里么?”敬济道:“在后边,几时出来!昨夜三更才睡,大娘后边拉着我听宣《红罗宝卷》,坐到那咱晚,险些儿没把腰累■■了,今日白扒不起来。”金莲道:“贼牢成的,就休捣谎哄我!昨日我不在家,你几时在上房内听宣卷来?丫鬟说你昨日在孟三儿房里吃饭来。”敬济道:“早是大姐看着,俺每都在上房内,几时在他屋里去来!”说着,这小伙儿站在炕上,把那话弄得硬硬的,直竖的一条棍,隔窗眼里舒过来。妇人一见,笑的要不得,骂道:“怪贼牢拉的短命,猛可舒出你老子头来,唬了我一跳。你趁早好好抽进去,我好不好拿针刺与你一下子,教你忍痛哩!”敬济笑道:“你老人家这回儿又不待见他起来,你好歹打发他个好处去,也是你一点阴骘。”妇人骂道:“好个怪牢成久惯的囚根子!”一面向腰里摸出面青铜小镜来,放在窗棂上,假做匀脸照镜,一面用朱唇吞裹吮咂他那话,吮咂的这小郎君一点灵犀灌顶,满腔春意融心。正咂在热闹处,忽听得有人走的脚步儿响,这妇人连忙摘下镜子,走过一边。敬济便把那话抽回去。却不想是来安儿小厮走来,说:“傅大郎前边请姐夫吃饭哩。”敬济道:“教你傅大郎且吃着,我梳头哩,就来。”来安儿回去了。妇人便悄悄向敬济说:“晚夕你休往那里去了,在屋里,我使春梅叫你。好歹等我,有话和你说。”敬济道:“谨依来命。”妇人说毕,回房去了。敬济梳洗毕,往铺中自做买卖。不题。

不一时,天色晚来。那日,月黑星密,天气十分炎热。妇人令春梅烧汤热水,要在房中洗澡,修剪足甲。床上收拾衾枕,赶了蚊子,放下纱帐子,小篆内炷了香。春梅便叫:“娘不,今日是头伏,你不要些凤仙花染指甲?我替你寻些来。”妇人道:“你那里寻去?”春梅道:“我直往那边大院子里才有,我去拔几根来。娘教秋菊寻下杵臼,捣下蒜。”妇人附耳低言,悄悄分付春梅:“你就厢房中请你姐夫晚夕来,我和他说话。”春梅去了,这妇人在房中,比及洗了香肌,修了足甲,也有好一回。只见春梅拔了几颗凤仙花来,整叫秋菊捣了半日。妇人又与他他几钟酒吃,打发他厨下先睡了。妇人灯光下染了十指春葱,令春梅拿凳子放在天井内,铺着凉簟衾枕纳凉。约有更阑时分,但见朱户无声,玉绳低转,牵牛、织女二星隔在天河两岸。又忽闻一阵花香,几点萤火。妇人手拈纨扇,伏枕而待。春梅把角门虚掩。正是:

\[
待月西厢下,迎风户半开。
隔墙花影动,疑是玉人来。
\]

原来敬济约定摇木瑾花树为号,就知他来了。妇人见花枝摇影,知是他来,便在院内咳嗽接应。他推开门进来,两个并肩而坐。妇人便问:“你来,房中有谁?”敬济道:“大姐今日没出来,我已分付元宵儿在房里,有事先来叫我。”因问:“秋菊睡了?”妇人道:“已睡熟了。”说毕,相搂相抱,二人就在院内凳上,赤身露体,席上交欢。不胜缱绻。但见:

\[
情兴两和谐,搂定香肩脸揾腮。手捻香乳绵似软,实奇哉!掀起脚儿脱绣鞋,玉体着郎怀。舌送丁香口便开,倒凤填鸾云雨罢,嘱多才:明朝千万早些来。
\]

两个云雨毕,妇人拿出五两碎银子来,递与敬济说:“门外你潘姥姥死了,棺材已是你爹在日与了他。三日入殓时,你大娘教我去探丧烧纸来了。明日出殡,你大娘不放我去,说你爹热孝在身,只见出门。这五两银子交与你,明早央你蚤去门外发送发送你潘姥姥,打发抬钱,看着下入土内,你来家。就同我去一般。”这敬济一手接了银子,说:“这个不打紧。我明日绝早就出门,干毕事,来回你老人家。”说毕,恐大姐进房,老早归厢房中去了。

一宿晚景休题。到次日,到饭时就来家。金莲才起来,在房中梳头。敬济走来回话,就门外昭化寺里,拿了两枝茉莉花儿来妇人戴。妇人问:“棺材下了葬了?”敬济道:“我管何事,不打发他老人家黄金入了柜,我敢来回话!还剩了二两六七钱银子,交付与你妹子收了,盘缠度日。千恩万谢,多多上覆你。”妇人听见他娘入土,落下泪来。便叫春梅:“把花儿浸在盏内,看茶来与你姐夫吃。”不一时,两盒儿蒸酥,四碟小菜,打发敬济吃了茶,往前边去了。由是越发与这小伙儿日亲日近。

一日,七月天气,妇人早辰约下他:“你今日休往那里去,在房中等着,我往你房里,和你顽耍。”这敬济答应了,不料那日被崔本邀了他,和几个朋友往门外耍子。去了一日,吃的大醉来家,倒在床上就睡着了,不知天高地下。黄昏时分,金莲蓦地到他房中,见他挺在床上,推他推不醒,就知他在那里吃了酒来。可霎作怪,不想妇人摸到他袖子里,吊下一根金头莲瓣簪儿来,上面趿着两溜字儿:“金勒马嘶芳草地,玉楼人醉杏花天。”迎亮一看,认的是孟玉楼簪子:“怎生落在他袖中?想必他也和玉楼有些首尾。不然,他的簪子如何他袖着?怪道这短命,几次在我面上无情无绪。我若不留几个字儿与他,只说我没来。等我写四句诗在壁上,使他知道。待我见了,慢慢追问他下落。”于是取笔在壁上写了四句。诗曰:

\[
独步书斋睡未醒,空劳神女下巫云。
襄王自是无情绪,辜负朝朝暮暮情。
\]

写毕,妇人回房去了。却说敬济一觉酒醒起来,房中掌上灯,因想起今日妇人来相会,我却醉了。回头见壁上写了四句诗在壁上,墨迹犹新,念了一遍,就知他来到,空回去了。心中懊悔不已。“这咱已是起更时分,大姐、元宵儿都在后边未出来,我若往他那边去,角门又关了。”走来木槿花下,摇花枝为号,不听见里面动静,不免踩着太湖石扒过粉墙去。那妇人见他有酒,醉了挺觉,大恨归房,闷闷在心,就浑衣上床歪睡。不料半夜他扒过墙来,见院内无人,想丫鬟都睡了,悄悄蹑足潜踪走到房门首,见门虚掩,就挨身进来。窗间月色照见床上妇人独自朝里歪着,低声叫“可意人”,数声不应,说道:“你休怪我,今日崔大哥众朋友,邀了我往门外五里原庄上射箭耍子了一日,来家就醉了。不知你到,有负你之约,恕罪恕罪。”那妇人也不理他。敬济见他不理,慌了,一面跪在地下,说了一遍又重复一遍。被妇人反手望脸上挝了一下,骂道:“贼牢拉负心短命,还不悄悄的,丫头听见!我知道你有了人,把我不放到心上。你今日端的那去来?”敬济道:“我本被崔大哥拉了门外射箭去,灌醉了来,就睡着了,失误你约,你休恼。我看见你留诗在壁上,就知恼了你。”妇人道:“怪捣鬼牢拉的,别要说嘴,与我禁声!你捣的鬼如泥弹儿圆,我手内放不过。你今日便是崔本叫了你吃酒,醉了来家,你袖子里这根簪子,却是那里的?”敬济道:“是那日花园中拾的,今两三日了。”妇人道:“你还\textuni{34B2}神捣鬼,是那花园里拾的?你再拾一根来,我才信你。这簪子是孟碱儿那麻淫妇的头上簪子,我认的千真万真,上面还趿着他名字,你还哄我。嗔道前日我不在,他叫你房里吃饭,原来你和他七个八个。我问你,还不肯认。你不和他两个有首尾,他的簪子缘何到你手里?原来把我的事都透露与他,怪道他前日见了我笑,原来有你的话在里头。自今以后,你是你,我是我,绿豆皮儿——请退了。”敬济听了,急的赌神发咒,继之以哭,道:“我敬济若与他有一字丝麻皂线,灵的是东岳城隍,活不到三十岁,生来碗大疔疮,害三五年黄病,要汤不汤,要水不水。”那妇人终是不信,说道:“你这贼才料,说来的牙疼誓,亏你口内不害碜!”两个絮聒了一回,见夜深了,不免解卸衣衫,挨身上床躺下。那妇人把身子扭过,倒背着他,使个性儿不理他,由着他姐姐长、姐姐短,只是反手望脸上挝过去。唬的敬济气也不敢出一口儿来,干霍乱了一夜。将天明,敬济恐怕丫头起身,依旧越墙而过,往前边厢房中去了。正是:

\[
三光有影遣谁系?万事无根只自生。
\]

\newpage
%# -*- coding:utf-8 -*-
%%%%%%%%%%%%%%%%%%%%%%%%%%%%%%%%%%%%%%%%%%%%%%%%%%%%%%%%%%%%%%%%%%%%%%%%%%%%%%%%%%%%%


\chapter{秋菊含恨泄幽情\KG 春梅寄柬谐佳会}


诗曰:

\[
如此钟情古所稀,吁嗟好事到头非。
汪汪两眼西风泪,犹向阳台作雨飞。
月有阴晴与圆缺,人有悲欢与会别。
拥炉细语鬼神知,空把佳期为君说。
\]

话说潘金莲见陈敬济天明越墙过去了,心中又后悔。次日却是七月十五日,吴月娘坐轿子往地藏庵薛姑子那里,替西门庆烧盂兰会箱库去。金莲众人都送月娘到大门首。回来,孟玉楼、孙雪娥、大姐,都往后边去了。独金莲落后,走到前厅仪门首,撞遇敬济正在李瓶儿那边楼上,寻了解当库衣物抱出来。金莲叫住,便向他说:“昨日我说了你几句,你如何使性儿今早就跳出来了,莫不真个和我罢了?”敬济道:“你老人家还说哩,一夜谁睡着来!险些儿一夜不曾把我麻烦死了,你看把我脸上肉也挝的去了!”妇人骂道:“贼短命,既不与他有首尾,贼人胆儿虚,你平白走怎的?”敬济道:“天将明了,不走来,不教人看见了?谁与他有甚事来?”金莲道:“既无此事,你今晚再来,我慢慢问你。”敬济道:“吃你麻犯了人,一夜谁合眼儿来?等我白日里睡一觉儿去。”妇人道:“你不去,和你算帐。”说毕,妇人回房去了。

敬济拿衣物往铺子里来,做了一回买卖,归到厢房,歪在床上睡了一觉。盼望天色晚了,要往金莲那边去。不想到黄昏时分,天色一阵黑阴来,窗外簌簌下起雨来。正是:

\[
萧萧庭院黄昏雨,点点芭蕉不住声。
\]
这敬济见那雨下得紧,说道:“好个不做美的天!他甫能教我对证话去,今日不想又下起雨来,好闷倦人也。”于是长等短等,那雨不住,簌簌直下到初更时分,下的房檐上流水。这小郎君等不的雨住,披着一条茜红毯子卧单在身上。那时吴月娘来家,大姐与元宵儿都在后边没出来。于是锁了房门,从西角门大雨里走入花园,推了推角门。妇人知他今晚必来,早已分付春梅灌了秋菊几钟酒,同他在炕房里先睡了,以此把角门虚掩。这敬济推开角门,便挨身而入。进到妇人卧房,见纱房半启,银烛高烧,桌上酒果已陈,金尊满泛。两个并肩叠股而坐。妇人便问:“你既不曾与孟三儿勾搭,这簪子怎得到你手里?”敬济道:“本是我昨日在花园荼縻架下拾的,若哄你,便促死促灰。”妇人道:“既无此事,还把这簪子与你关头,我不要你的。只要把我与你的簪子、香囊、帕儿物事收好着,少了我一件儿,钱与你答话。”两个吃酒下棋,到一更方上床安寝。颠鸾倒凤,整狂了半夜。妇人把昔日西门庆枕边风月,一旦尽付与情郎身上。

却说秋菊在那边屋里,忽听见这边屋里恰似有男子声音说话,更不知是那个。到天明鸡叫时分,秋菊起来溺尿,忽听那边房内开的门响,朦胧月色,雨尚未止,打窗眼看见一人,披着红卧单,从房中出去了。“恰似陈姐夫一般。原来夜夜和我娘睡。我娘自来会撇净,干净暗里养着女婿!”次日,径走到后边厨房里,就如此这般对小玉说。不想小玉和春梅好,又告诉春梅说:“秋菊说你娘养着陈姐夫,昨日在房里睡了一夜,今早出去了。大姑娘和元宵又没在前边睡。”这春梅归房一五一十对妇人说:“娘不打与这奴才几下,教他骗口张舌,葬送主子。”金莲听了大怒,就叫秋菊到面前跪着,骂道:“教你煎熬粥儿,就把锅来打破了。你敢屁股大,吊了心也怎的?我这几日没曾打你这奴才,骨朵痒了!”于是拿棍子向他脊背上尽力狠抽了三十下,打得秋菊杀猪也似叫,身上都破了。春梅走将来说:“娘没的打他这几下儿,只好与他挝痒儿罢了。旋剥了,叫将小厮来,拿大板子尽力砍与他二三十板,看他怕不怕?汤他这几下儿,打水不深的,只像斗猴儿一般。他好小胆儿,你想他怕也怎的?做奴才,里言不出,外言不入,都似你这般,好养出家生哨儿来了。”秋菊道:“谁说甚么来?”妇人道:“还说嘴哩!贼破家害主的奴才,还说甚么!”几声喝的秋菊往厨下去了。正是:

\[
蚊虫遭扇打,只为嘴伤人。
\]

一日,八月中秋时分,金莲夜间暗约敬济赏月饮酒,和春梅同下鳌棋儿。晚夕贪睡失晓,至茶时前后还未起来,颇露圭角。不想被秋菊睃到眼里,连忙走到后边上房,对月娘说。不想月娘才梳头,小玉正在上房门首站立。秋菊拉过他一边,告他说:“俺姐夫如此这般,昨日又在我娘房里歇了一夜,如今还未起来哩。前日为我告你说,打了我一顿。今日真实看见,我原不赖他,请奶奶快去瞧去。”小玉骂道:“张眼露睛奴才,又来葬送主子,俺奶奶梳头哩,还不快走哩。”月娘便问:“他说甚么?”小玉不能隐讳,只说:“五娘使秋菊来请奶奶说话。”更不说出别的事。

这月娘梳了头,轻移莲步,蓦然来到前边金莲房门首。早被春梅看见,慌的先进来,报与金莲。金莲与敬济两个还在被窝内未起,听见月娘到,两个都吃了一惊,慌做手脚不迭,连忙藏敬济在床身子里,用一床锦被遮盖的沿沿的。教春梅放小桌儿在床上,拿过珠花来,且穿珠花。不一时,月娘到房中坐下,说:“六姐,你这咱还不见出门,只道你做甚,原来在屋里穿珠花哩。”一面拿在手中观看,夸道:“且是穿的好,正面芝麻花,两边槅子眼方胜儿,辕围蜂赶菊,刚凑着同心结,且是好看。到明日,你也替我穿恁条箍儿戴。”妇人见月娘说好话儿,那心头小鹿儿才不跳了,一面令春梅:、倒茶来与大娘吃。”少顷,月娘吃了茶,坐了回去了,说:“六姐快梳了头,后边坐。”金莲道:“晓得。”打发月娘出来,连忙撺掇敬济出港,往前边去了。春梅与妇人整捏两把汗,妇人说:“你大娘等闲无事再不来,今日大清早辰来做甚么?”春梅道:“左右是咱家这奴才嚼舌来。”不一时,只见小玉走来,如此这般:“秋菊后边说去,说姐夫在这屋里明睡到夜,夜睡到明,被我骂喝了他两声,他还不动。俺奶奶问我,没的说,只说五娘请奶奶说话,方才来了。你老人家只放在心里,大人不见小人之过,只堤防着这奴才就是了。”

看官听说,虽是月娘不信秋菊说话,只恐金莲少女嫩妇没了汉子,日久一时心邪,着了道儿。恐传出去,被外人唇舌。又以爱女之故,不教大姐远出门,把李娇儿厢房挪与大姐住,教他两口儿搬进后边仪门里来。遇着傅伙计家去,方教敬济轮番在铺子里上宿。取衣物药材,俱同玳安儿出入。各处门户都上了锁钥,丫鬟妇女无事不许往外边去。凡事都严紧,这潘金莲与敬济两个热突突恩情都间阻了。正是:世间好事多间阻,就里风光不久长。有诗为证:

\[
几向天台访玉真,三山不见海沉沉。
侯门一日深如海,从此萧郎是路人。
\]

潘金莲自被秋菊泄露之后,与敬济约一个多月不曾相会。金莲每日难挨,怎禁绣帏孤冷,画阁凄凉,未免害些木边之目,田下之心。脂粉懒匀,茶饭顿减,带围宽褪,恹恹瘦损,每日只是思睡,扶头不起。春梅道:“娘,你这等虚想也无用,昨日大娘留下两个姑子,我听见说今晚要宣卷,后边关的仪门早。晚夕,我推往前边马房内取草装枕头,等我到铺子里叫他去。我好歹叫了姐夫和娘会一面,娘心下如何?”妇人道:“我的好姐姐,你若肯可怜见,叫得他来,我恩有重报,决不有忘。”春梅道:“娘说的是那里话!你和我是一个人,爹又没了,你明日往前后进,我情愿跟娘去。咱两个还在一处。”妇人道:“你有此心,可知好哩。”

到于晚夕,妇人先在后边月娘前,假托心中不自在,用了个金蝉脱壳,归到前边。月娘后边仪门老早开了,丫鬟妇人都放出来,要听尼僧宣卷。金莲央及春梅,说道:“好姐姐,你快些请他去罢。”春梅道:“等我先把秋菊那奴才,与他几钟酒,灌醉了,倒扣他在厨房内。我方好去。”于是筛了两大碗酒,打发秋菊吃了,扣他在厨房内,拿了个筐儿,走到前边,先撮了一筐草,就悄悄到印子铺门首,低声叫门。正值傅伙计不在铺中,往家去了。独有敬济在炕上才歪下,忽见有人叫门,声音像是春梅,连忙开门,见是他,满面笑道:“果然是小大姐,没人,请里面坐。”春梅走入房内,便问:“小厮们在那里?”敬济道:“玳安和平安,都在那边生药铺中睡哩,独我一个在此受孤凄,挨冷淡。”春梅道:“俺娘多上覆你,说你好人儿,这几日就门边儿也不往俺那屋里走走去。说你另有了对门主顾儿了,不稀罕俺娘儿每了。”敬济道:“说那里话,自从那日着了唬,惊散了,又见大娘紧门紧户,所以不敢走动。”春梅道:“俺娘为你这几日心中好生不快,逐日无心无绪,茶饭懒吃,做事没入脚处。今日大娘留他后边听宣卷,也没去,就来了。一心只是牵挂想你,巴巴使我来,好歹教你快去哩。”敬济道:“多感你娘称们厚情,何以报答?你略先走一步儿,我收拾了,随后就去。”一面开橱门,取出一方白绫汗巾,一副银三事挑牙儿与他。就和春梅两个搂抱,按在炕上,且亲嘴咂舌,不胜欢谑。正是:

\[
无缘得会莺莺面,且把红娘去解谗。
\]

两个戏了一回,春梅先拿着草归到房来,一五一十对妇人说:“姐夫我叫了,他便来也。见我去,好不喜欢,又与了我一方汗巾,一付银挑牙儿。”妇人便叫春梅:“你在外边看着,只怕他来。”

原来那日正值九月十二三,月色正明。陈敬济旋到生药铺,叫过来安儿来这边来。他只推月娘叫他听宣卷,径往后边去了。因前边花园门关了,打后边角门走入金莲那边,摇木瑾花为号。春梅连忙接应,引入房中。妇人迎门接着,笑骂道:“贼短命,好人儿,就不进来走走儿。”敬济道:“我巴不得要来哩,只怕弄出是非来,带累你老人家,不好意思。”说着,二人携手进房坐下。春梅关上角门,房中放桌儿,摆上酒肴。妇人和敬济并肩叠股而坐,春梅打横,把酒来斟,穿杯换盏,倚翠偎红,吃了一回。吃的酒浓上来,妇人娇眼乜斜,乌云半軃,取出西门庆淫器包儿,里面包着相思套、颤声娇、银托子、勉铃一弄儿淫器。教敬济便在灯光影下,妇人便赤身露体,仰卧在一张醉翁椅儿上。敬济亦脱的上下没条丝,又拿出春意二十四解本儿,放在灯下,照着样儿行事。妇人便叫春梅:“你在后边推着你姐夫,只怕他身子乏了。”那春梅真个在后边推送,敬济那话插入妇人牝中,往来抽送,十分畅美,不可尽言。不想秋菊在后边厨下,睡到半夜里起来净手,见房门倒扣着,推不开。于是伸手出来,拨开鸟吊儿,大月亮地里,蹑足潜踪,走到前房窗下。打窗眼里望里张看,见房中掌着明晃晃灯烛,三个人吃得大醉,都光赤着身子,正做得好。两个对面坐着,春梅便在身后推车,三人串作一处。但见:

一个不顾夫主名分,一个那管上下尊卑。一个椅上逞雨意云情,一个耳畔说山盟海誓。一个寡妇房内翻为快活道场,一个丈母根前变作污淫世界。一个把西门庆枕边风月尽付与娇婿,一个将韩寿偷香手段悉送与情娘。正是:写成今世不休书,结下来生欢喜带。

秋菊看到眼里,口中不说,心内暗道:“他们还在人前撇清要打我,今日却真实被我看见了。到明日对大娘说,莫非又说骗嘴张舌赖我不成!”于是瞧了个不亦乐乎,依旧还往厨房中睡去了。

三个整狂到三更时分才睡。春梅未曾天明先起来,走到厨房,见厨房门开了,便问秋菊。秋菊道:“你还说哩。我尿急了,往那里溺?我拔开鸟吊,出来院子里溺尿来。”春梅道:“成精奴才,屋里放着杩子,溺不是!”秋菊道:“我不知杩子在屋里。”两个后边聒噪,敬济天明起来,早往前边去了。正是:

\[
两手劈开生死路,翻身跳出是非门。
\]

那妇人便问春梅:“后边乱甚么?”这春梅如此这般,告说秋菊夜里开门一节。妇人发恨要打秋菊。这秋菊早辰又走来后边,报与月娘知道,被月娘喝了一声,骂道:“贼葬弄主子的奴才!前日平空走来,轻事重报,说他主子窝藏陈姐夫在房里,明睡到夜,夜睡到明,叫了我去。他主子正在床上放炕桌儿穿珠花儿,那得陈姐夫来?落后陈姐夫打前边来,恁一个弄主子的奴才!一个大人放在屋里,端的是糖人儿,不拘那里安放了?一个砂子那里发落?莫不放在眼里不成?传出去,知道的是你这奴才葬送主子。不知道的,只说西门庆平日要的人强多了,人死了多少时儿,老婆们一个个都弄的七颠八倒。恰似我的这孩子,也有些甚根儿不正一般。”于是要打秋菊。唬得秋菊往前边疾走如飞,再不敢来后边说了。

妇人听见月娘喝出秋菊,不信其事,心中越发放大胆了。西门大姐听见此言,背地里审问敬济。敬济道:“你信那汗邪了的奴才!我昨日见在铺里上宿,几时往花园那边去来?花园门成日关着。”大姐骂道:“贼囚根子,你别要说嘴,你若有风吹草动,到我耳朵内,惹娘说我,你就信信脱脱去了,再也休想在这屋里了。”敬济道:“是非终日有,不听自然无。大娘眼见不信他。”大姐道:“得你这般说就好了。”正是:

\[
谁料郎心轻似絮,那知妾意乱如丝。
\]

\newpage
%# -*- coding:utf-8 -*-
%%%%%%%%%%%%%%%%%%%%%%%%%%%%%%%%%%%%%%%%%%%%%%%%%%%%%%%%%%%%%%%%%%%%%%%%%%%%%%%%%%%%%


\chapter{吴月娘大闹碧霞宫\KG 曾静师化缘雪涧洞}


诗曰:

\[
一自当年折凤凰,至今情绪几惶惶。
盖棺不作横金妇,入地还从折桂郎。
彭泽晓烟归宿梦,潇湘夜雨断愁肠。
新诗写向空山寺,高挂云帆过豫章。
\]

说话一日,吴月娘请将吴大舅来商议,要往泰安州顶上与娘娘进香,因西门庆病重之时许的愿心。吴大舅道:“既要去,须是我同了你去。”一面备办香烛纸马祭品之物,玳安、来安儿跟随,雇了三个头口,月娘便坐一乘暖轿,分付孟玉楼、潘金莲、孙雪娥、西门大姐:“好生看家,同奶子如意儿、众丫头好生看孝哥儿。后边仪门无事早早关了,休要出外边去。”又分付陈敬济:“休要那去,同傅伙计大门首看顾。我约莫到月尽就来家了。”十五日早辰烧纸通信,晚夕辞了西门庆灵,与众姊妹置酒作别,把房门、各库门房钥匙交付与小玉拿着。次日早五更起身,离了家门,一行人奔大路而去。那秋深时分,天寒日短,一日行程六七十里之地。未到黄昏,投客店村房安歇,次日再行。一路上,秋云淡淡,寒雁凄凄,树木凋落,景物荒凉,不胜悲怆。

话休饶舌。一路无词,行了数日,到了泰安州,望见泰山,端的是天下第一名山,根盘地脚,顶接天心,居齐鲁之邦,有岩岩之气象。吴大舅见天晚,投在客店歇宿一宵。次日早起上山,望岱岳庙来。那岱岳库就在山前,乃累朝祀典,历代封禅,为第一庙貌也。但见:

\[
庙居岱岳,山镇乾坤,为山岳之尊,乃万福之领袖。山头倚槛,直望弱水蓬莱;绝顶攀松,都是浓云薄雾。楼台森耸,金乌展翅飞来;殿宇棱层,玉兔腾身走到。雕梁画栋,碧瓦朱檐,凤扉亮槅映黄纱,龟背绣帘垂锦带。遥观圣像,九猎舞舜目尧眉;近观神颜,衮龙袍汤肩禹背。御香不断,天神飞马报丹书;祭祀依时,老幼望风祈护福。嘉宁殿祥云香霭,正阳门瑞气盘旋。
\]
正是:

\[
万民朝拜碧霞宫,四海皈依神圣帝。
\]

吴大舅领月娘到了岱岳庙,正殿上进了香,瞻拜了圣像,庙祝道士在旁宣念了文书。然后两廊都烧化了纸钱,吃了些斋食。然后领月娘上顶,登四十九盘,攀藤揽葛上去。娘娘金殿在半空中云烟深处,约四五十里,风云雷雨都望下观看。月娘众人从辰牌时分岱岳庙起身,登盘上顶,至申时已后方到。娘娘金殿上朱红牌扁,金书“碧霞宫”三字。进入宫内,瞻礼娘娘金身。怎生模样?但见:

\[
头绾九龙飞凤髻,身穿金缕绛绡衣。蓝田玉带曳长裾,白玉圭璋檠彩袖。脸如莲萼,天然眉目映云鬟;唇似金朱,自在规模端雪体。犹如王母宴瑶池,却似嫦娥离月殿。正大仙云描不就,威严形象画难成。
\]
月娘瞻拜了娘娘仙容,香案边立着一个庙祝道士,约四十年纪,生的五短身材,三溜髭须,明眸牿齿,头戴簪冠,身披绛服,足登云履,向前替月娘宣读了还愿文疏,金炉内炷了香,焚化了纸马金银,令小童收了祭供。

原来这庙祝道士,也不是个守本分的,乃是前边岱岳庙里金住持的大徒弟,姓石,双名伯才,极是个贪财好色之辈,趋时揽事之徒。这本地有个殷太岁,姓殷,双名天锡,乃是本州知州高廉的妻弟。常领许多不务本的人,或张弓挟弹,牵架鹰犬,在这上下二宫,专一睃看四方烧香妇女,人不敢惹他。这道士石伯才,专一藏奸蓄诈,替他赚诱妇女到方丈,任意奸淫,取他喜欢。因见月娘生的姿容非俗,戴着孝冠儿,若非官户娘子,定是豪家闺眷;又是一位苍白髭髯老子跟随,两个家童,不免向前稽首,收谢神福:“请二位施主方丈一茶。”吴大舅便道:“不劳生受,还要赶下山去。”伯才道:“就是下山也还早哩。”

不一时,请至方丈,里面糊的雪白,正面放一张芝麻花坐床,柳黄锦帐,香几上供养一幅洞宾戏白牡丹图画,左右一对联,大书着:“两袖清风舞鹤,一轩明月谈经。”伯才问吴大舅上姓,大舅道:“在下姓吴,这个就是舍妹吴氏,因为夫主来还香愿,不当取扰上宫。”伯才道:“既是令亲,俱延上坐。”他便主位坐了,便叫徒弟看茶。原来他手下有两个徒弟,一个叫郭守清,一个名郭守礼,皆十六岁,生得标致,头上戴青段道髻,身穿青绢道服,脚上凉鞋净袜,浑身香气袭人。客至则递茶递水,斟酒下菜。到晚来,背地便拿他解馋填馅。不一时,守清、守礼安放桌儿,就摆斋上来,都是美口甜食,蒸堞饼馓,各样菜蔬,摆满春台。每人送上甜水好茶,吃了茶,收下家火去。就摆上案酒。大盘大碗肴馔,都是鸡鹅鱼鸭上来。用琥珀镶盏,满泛金波。吴月娘见酒来,就要起身,叫玳安近前,用红漆盘托出一匹大布、二两白金,与石道士作致谢之礼。吴大舅便说:“不当打搅上宫,这些微礼致谢仙长。不劳见赐酒食,天色晚来,如今还要赶下山去。”慌的石伯才致谢不已,说:“小道不才,娘娘福荫,在本山碧霞宫做个住持,仗赖四方钱粮,不管待四方财主,作何项下使用?今聊备粗斋薄馔,倒反劳见赐厚礼,使小道却之不恭,受之有愧。”辞谢再三,方令徒弟收下去。一面留月娘、吴大舅坐:“好歹坐片时,略饮三杯,尽小道一点薄情而已。”吴大舅见款留恳切,不得已和月娘坐下。不一时,热下饭上来。石道士分付徒弟:“这个酒不中吃,另打开昨日徐知府老爷送的那一坛透瓶香荷花酒来,与你吴老爹用。”不一时,徒弟另用热壶筛热酒上来。先满斟一杯,双手递与月娘,月娘不肯接。吴大舅道:“舍妹他天性不用酒。”伯才道:“老夫人一路风霜,用些何害?好歹浅用些。”一面倒去半钟,递上去与月娘接了。又斟一杯递与吴大舅,说:“吴老爹,你老人家试用此酒,其味如何?”吴大舅饮了一口,觉香甜绝美,其味深长,说道:“此酒甚好。”伯才道:“不瞒你老人家说,此是青州徐知府老爹送与小道的酒。他老夫人、小姐、公子,年年来岱岳庙烧香建醮,与小道相交极厚。他小姐;衙内又寄名在娘娘位下。见小道立心平淡,殷勤香火,一味至诚,甚是敬爱小道。常年,这岱岳庙上下二宫钱粮,有一半征收入库。近年多亏了我这恩主徐知府老爹题奏过,也不征收,都全放常住用度,侍奉娘娘香火,余者接待四方香客。”这里说话,下边玳安、来安、跟从轿夫,下边自有坐处,汤饭点心,大盘大碗酒肉,都吃饱了。

吴大舅饮了几杯,见天晚要起身。伯才道:“日色将落,晚了赶不下山去。倘不弃,在小道方丈权宿一宵,明早下山从容些。”吴大舅道:“争奈有些小行李在店内,诚恐一时小人罗唣。”伯才笑道:“这个何须挂意!决无丝毫差池。听得是我这里进香的,不拘村坊店面,闻风害怕,好不好把店家拿来本州来打,就教他寻贼人下落。”吴大舅听了,就坐住了。伯才拿大钟斟上酒来。吴大舅见酒利害,便推醉更衣,遂往后边阁上观看随喜去了。这月娘觉身子乏困,便在床上侧侧儿。这石伯才一面把房门拽上,外边去了。

月娘方才床上歪着,忽听里面响亮了一声,床背后纸门内跳出一个人来,淡红面貌,三柳髭须,约三十年纪,头戴渗青巾,身穿紫锦袴衫,双手抱住月娘,说道:“小生殷天锡,乃高太守妻弟。久闻娘子乃官豪宅眷,天然国色,思慕如渴。今既接英标,乃三生有幸,倘蒙见怜,死生难忘也。”一面按着月娘在床上求欢。月娘唬的慌做一团,高声大叫:“清平世界,朗朗乾坤,没事把良人妻室,强霸拦在此做甚!”就要夺门而走。被天锡抵死拦挡不放,便跪下说:“娘子禁声,下顾小生,恳求怜允。”那月娘越高声叫的紧了,口口大叫:“救人!”平安、玳安听见是月娘声音,慌慌张张走去后边阁上,叫大舅说:“大舅快去,我娘在方丈和人合口哩。”这吴大舅慌的两步做一步奔到方丈推门,那里推得开。只见月娘高声:“清平世界,拦烧香妇女在此做甚么?”这吴大舅便叫:“姐姐休慌,我来了!”一面拿石头把门砸开。那殷天锡见有人来,撇开手,打床背后一溜烟走了。原来这石道士床背后都有出路。

吴大舅砸开方丈门。问月娘道:“姐姐,那厮玷污不曾?”月娘道:“不曾玷污。那厮打床背后走了。”吴大舅寻道士,那石道士躲去一边,只教徒弟来支调。大舅大怒,喝令手下跟随玳安、来安儿把道士门窗户壁都打碎了。一面保月娘出离碧霞宫,上了轿子,便赶下山来。

约黄昏时分起身,走了半夜,方到山下客店内。如此这般,告店小二说。小二叫苦连声,说:“不合惹了殷太岁,他是本州知州相公妻弟,有名殷太岁。你便去了,俺开店之家,定遭他凌辱,怎肯干休!”吴大舅便多与他一两店钱,取了行李,保定月娘轿子,急急奔走。后面殷天锡气不舍,率领二三十闲汉,各执腰刀短棍,赶下山来。

吴大舅一行人,两程做一程,约四更时分,赶到一山凹里。远远树木丛中有灯光,走到跟前,却是一座石洞,里面有一老僧秉烛念经。吴大舅问:“老师,我等顶上烧香,被强人所赶,奔下山来,天色昏黑,迷踪失路至此。敢问老师,此处是何地名?从那条路回得清河县去?”老僧说:“此是岱岳东峰,这洞名唤雪涧洞。贫僧就叫雪洞禅师,法名普静,在此修行二三十年。你今遇我,实乃有缘。休往前去,山下狼虽虎豹极多。明日早行,一直大道就是你清河县了。”吴大舅道:“只怕有人追赶。”老师把眼一观说:“无妨,那强人赶至半山,已回去了。”因问月娘姓氏。吴大舅道:“此乃吾妹,西门庆之妻。因为夫主,来此进香。得遇老师搭救,恩有重报,不敢有忘。”于是在洞内歇了一夜。

次日天不亮,月娘拿出一匹大布谢老师。老师不受,说:“贫曾只化你亲生一子作个徒弟,你意下何如?”吴大舅道:“吾妹止生一子,指望承继家业。若有多余,就与老师作徒弟。”月娘道:“小儿还小,今才不到一周岁儿,如何来得?”老师道:“你只许下,我如今不问你要,过十五年才问你要哩。”月娘口中不言,过十五年再作理会,遂含糊许下老师。一面作辞老师,竟奔清河县大道而来。正是:

\[
世上只有人心歹,万物还教天养人。
但交方寸无诸恶,狼虎丛中也立身。
\]

\newpage
%# -*- coding:utf-8 -*-
%%%%%%%%%%%%%%%%%%%%%%%%%%%%%%%%%%%%%%%%%%%%%%%%%%%%%%%%%%%%%%%%%%%%%%%%%%%%%%%%%%%%%


\chapter{吴月娘识破奸情\KG 春梅姐不垂别泪}


词曰:

\[
情若连环总不解,无端招引旁人怪。好事多磨成又败,应难捱,相冷眼谁揪采?镇日愁眉和敛黛,阑干倚遍无聊赖。但愿五湖明月在,权宁耐,终须还了鸳鸯债。
\]

话说月娘取路来家,不题。单表金莲在家,和陈敬济两个就如鸡儿赶蛋相似,缠做一处。一日,金莲眉黛低垂,腰肢宽大,终日恹恹思睡,茶饭懒咽,教敬济到房中说:“奴有件事告你说,这两日眼皮儿懒待开,腰肢儿渐渐大,肚腹中扑扑跳,茶饭儿怕待吃,身子好生沉困。有你爹在时,我求薛姑子符药衣胞那等安胎,白没见个踪影。今日他没了,和你相交多少时儿,便有了孩子。我从三月内洗身上,今方六个月,已有半肚身孕。往常时我排磕人,今日却轮到我头上。你休推睡里梦里,趁你大娘未来家,那里讨贴坠胎的药,趁早打落了这胎气。不然,弄出个怪物来,我就寻了无常罢了,再休想抬头见人。”敬济听了,便道:“咱家铺中诸样药都有,倒不知那几样儿坠胎,又没方修治。你放心,不打紧处,大街坊胡太医,他大小方脉,妇人科,都善治,常在咱家看病。等我问他那里赎取两贴,与你下胎便了。”妇人道:“好哥哥,你上紧快去,救奴之命。”

这陈敬济包了三钱银子,径到胡太医家来。胡太医正在家,出来相见声喏,认的敬济是西门大官人女婿,让坐说:“一向稀面,动问到舍有何见教?”敬济道:“别无干渎。”向袖中取出白金三星:“充药资之礼,敢求下胎良剂一二贴,足见盛情。”胡太医道:“天地之间,以好生为德。人家十个九个只要安胎的药,你如何倒要打胎?没有,没有。”敬济见他掣肘,又添了二钱药资,说:“你休管他,各人家自有用处。此妇女子生落不顺,情愿下胎。”这胡太医接了银子,说道:“不打紧,我与你一服红花一扫光。吃下去,如人行五里,其胎自落矣。”于是取了两贴,付与敬济。敬济得了药,作辞胡太医,到家递与妇人。妇人到晚夕,煎汤吃下去,登时满肚里生疼,睡在炕上,教春梅按在肚上只情揉揣。可霎作怪,须臾坐净桶,把孩子打下来了。只说身上来,令秋菊搅草纸倒在毛司里。次日,掏坑的汉子挑出去,一个白胖的孩子儿。常言好事不出门,恶事传千里,不消几日,家中大小都知金莲养女婿,偷出私孩子来了。

且说吴月娘有日来家。往回去了半个月光景,来时正值十月天气。家中大小接着,知前拜罢,就对玉楼众姐妹,把岱岳庙中的事,从头告诉一遍,因大哭一场。合家大小都来参见了。月娘见奶子抱孝哥儿到跟前,子母相会在一处。烧纸,置酒管待吴大舅回家。晚夕,众姊妹与月娘接风,俱不在话下。

到第二日,月娘因路上风霜跋涉,着了辛苦,又吃了惊怕,身上疼痛沉困,整不好了两三日。那秋菊在家,把金莲、敬济两人干的勾当,听的满耳满心,要告月娘说。走到上房门首,又被小玉哕骂在脸上,大耳刮子打在他脸上,骂道:“贼说舌的奴才,趁早与我走!俺奶奶远路来家,身子不快活,还未起来。气了他,倒值了多的。”骂的秋菊忍气吞声,喏喏而退。

一日,也是合当有事,敬济进来寻衣服,妇人和他又在玩花楼上两个做得好。被秋菊走到后边,叫了月娘来看,说道;“奴婢两番三次告大娘说不信。娘不在,两个在家明睡到夜,夜睡到明,偷出私孩子来。与春梅两个都打成一家。今日两人又在楼上干歹事,不是奴婢说谎,娘快些瞧去。”月娘急忙走到前边,两个正干的好,还未下楼。春梅在房中,忽然看见,连忙上楼去说:“不好了,大娘来了。”两人忙了手脚,没处躲避。敬济只得拿衣服下楼往外走,被月娘撞见喝骂了几句,说:“小孩儿家没记性,有要没紧进来撞甚么?”敬济道:“铺子内人等着,没人寻衣服。”月娘道:“我那等分付你,教小厮进来取,如何又进来寡妇房里做甚么?没廉耻!”几句骂得敬济往外金命水命,走投无命。妇人羞的半日不敢下来。然后下来,被月娘尽力数说了一顿,说道:“六姐,今后再休这般没廉耻!你我如今是寡妇,比不得有汉子,香喷喷在家里。瓶儿罐儿有耳朵,有要没紧和这小厮缠甚么!教奴才们背地排说的碜死了!常言道,男儿没性,寸铁无钢;女人无性,烂如麻糖。其身正,不令而行;其身不正,虽令不行。你若长俊正条,肯教奴才排说?他在我跟前说了几遍,我不信;今日亲眼看见,说不的了。我今日说过,你要自家立志,替汉子争气。像我进香去,被强人逼勒,若是不正气的,也来不到家了。”金莲吃月娘数说,羞的脸上红一块白一块,口里说一千个没有,只说:“我在楼上烧香,陈姐夫自去那边寻衣裳,谁和他说甚话来!”当日月娘乱了一回,归后边去了。

晚夕,西门大姐在房内又骂敬济:“贼囚根子,敢说又没真赃实犯拿住你?你还那等嘴巴巴的!今日两个又在楼上做甚么?说不的了!两个弄的好碜儿,只把我合在缸底下一般。那淫妇要了我汉子,还在我面前拿话儿拴缚人,毛司里砖儿——又臭又硬,恰似降伏着那个一般。他便羊角葱靠南墙——老辣已定。你还要在这里雌饭吃!”敬济骂道:“淫妇,你家收着我银子,我雌你家饭吃?”使性子往前边来了。

自此已后,敬济只在前边,无事不敢进入后边来。取东取西,只是玳安、平安两个往楼上取去。每日饭食,晌午还不拿出来,把傅伙计饿的只拿钱街上烫面吃。正是龙斗虎伤,苦了小獐。各处门户,日头半天就关了。由是与金莲两个恩情又间阻了。敬济那边陈宅的房子,一向教他母舅张团练看守居住。张团练革任在家闲住,敬济早晚往那里吃饭去,月娘也不追问。

两个隔别,约一月不得会面。妇人独在那边,挨一日似三秋,过一宵如半夏,怎禁这空房寂静,欲火如蒸,要见他一面,难上之难。两下音信不通,这敬济无门可入。忽一日见薛嫂儿打门首过,有心要托他寄一纸柬儿与金莲,诉其间阻之事,表此肺腑之情。一日,推门外讨帐,骑头口径到薛嫂家,拴了驴儿,掀帘便问:“薛妈在家?”有他儿子薛纪媳妇儿金大姐抱孩子在炕上,伴着人家卖的两个使女,听见有人叫薛妈,出来问:“是谁?”敬济道:“是我。”问:“薛妈在家不在?”金大姐道:“姑夫请家来坐,俺妈往人家兑了头面,讨银子去了。有甚话说,使人叫去。”连忙点茶与敬济吃。坐不多时,只见薛嫂儿来了,与敬济道了万福,说:“姑夫那阵风儿吹来我家!”叫金大姐:“倒茶与姑夫吃。”金大姐道:“刚才吃了茶了。”敬济道:“无事不来。如此这般,与我五娘勾搭日久,今被秋菊丫头戳舌,把俺两个姻缘拆散。大娘与大姐是疏淡我。我与六姐拆散不开,二人离别日久,音信不通,欲稍寄数字进去与他。无人得到内里,须央及你,如此这般通个消息。”向袖中取出一两银子来:“这些微礼,权与薛妈买茶吃。”那薛嫂一闻其言,拍手打掌笑起来,说道:“谁家女婿戏丈母?世间那里有此事!姑夫,你实对我说,端的你怎么得手来?”敬济道:“薛嫂禁声,且休取笑。我有这柬贴封好在此,好歹明日替我送与他去。”薛嫂一手接了说:“你大娘从进香回来,我还没看他去,两当一节,我去走走。”敬济道:“我在那里讨你信?”薛嫂道:“往铺子里寻你回话。”说毕,敬济骑头口来家。

次日,薛嫂提着花箱儿,先进西门庆家上房看月娘。坐了一回,又到孟玉楼房中,然后才到金莲这边。金莲正放桌儿吃粥。春梅见妇人闷闷不乐,说道:“娘,你老人家也少要忧心。是非有无,随人说去。如今爹也没了,大娘他养不出个墓生儿来,莫不是也来路不明?他也难管你我暗地的事。你把心放开,料天塌了还有撑天大汉哩。人生在世,且风流了一日是一日。”于是筛上酒来,递一钟与妇人说:“娘且吃一杯儿暖酒,解解愁闷。”因见阶下两只犬儿交恋在一处,说道:“畜生尚有如此之乐,何况人而反不如此乎?”正饮酒,只见薛嫂儿来到,向金莲道个万福,又与春梅拜了拜,笑道:“你娘儿们好受用。”因观二犬恋在一处,又笑道:“你家好祥瑞,你娘儿每看着怎不解闷!”妇人道:“那阵风儿今日刮你来,怎的一向不来走走?”一面让薛嫂坐。薛嫂儿道:“我整日干的不知甚么,只是不得闲。大娘顶上进了香来,也不曾看的他,刚才好不怪我。西房三娘也在跟前,留了我两对翠花,一对大翠围发,好快性,就称了八钱银子与我。只是后边雪姑娘,从八月里要了我两对线花儿,该二钱银子,白不与我。好悭吝的人!我对你说,怎的不见你老人家?”妇人道:“我这两日身中有些不自在,不曾出去走动。”春梅一面筛了一钟酒,递与薛嫂儿。薛嫂忙又道万福,说:“我进门就吃酒。”妇人道:“你到明日养个好娃娃。”薛嫂儿道:“我养不的,俺家儿子媳妇儿金大姐,倒新添了个娃儿,才两个月来。”又道:“你老人家没了爹,终日这般冷清清了。”妇人道:“说不得,有他在好了,如今弄的俺娘儿们一折一磨的。不瞒老薛说,如今俺家中人多舌头多,他大娘自从有了这孩儿,把心肠儿也改变了,姊妹不似那咱亲热了。这两日一来我心里不自在,二来因些闲话,没曾往那边去。”春梅道:“都是俺房里秋菊这奴才,大娘不在,霹空架了俺娘一篇是非,把我也扯在里面,好不乱哩。”薛嫂道:“就是房里使的那大姐?他怎的倒弄主子?自古穿青衣,抱黑柱。这个使不的。”妇人使春梅:“你瞧瞧那奴才,只怕他又来听。”春梅道:“他在厨下拣米哩!这破包篓奴才,在这屋就是走水的槽,单管屋里事儿往外学舌。”薛嫂道:“这里没人,咱娘儿每说话。昨日陈姐夫到我那里,如此这般告诉我,干净是他戳犯你每的事儿了。陈姐夫说,他大娘数说了他,各处门户都紧了,不许他进来取衣裳拿药材了。把大姐搬进东厢房里住。每日晌午还不拿饭出去与他吃,饿的他只往他母舅张老爹那里吃去。一个亲女婿不托他,倒托小厮,有这个道理?他有好一向没得见你老人家,巴巴央及我,稍了个柬儿,多多拜上你老人家,少要心焦,左右爹也是没了,爽利放倒身,大做一做,怕怎的?点根香怕出烟儿;放把火,倒也罢了。”于是取出敬济封的柬贴儿递与妇人。拆开观看,别无甚话,上写《红绣鞋》一词:

\[
袄庙火烧皮肉,蓝桥水淹过咽喉,紧按纳风声满南州。洗净了终是染污,成就了倒是风流,不怎么也是有。\named{六姐妆次敬济百拜上}
\]
妇人看毕,收入袖中。薛嫂道:“他教你回个记色与他,或写几个字儿稍了去,方信我送的有个下落。”妇人教春梅陪着薛嫂吃酒,他进入里间,半晌拿了一方白绫帕,一个金戒指儿。帕儿上又写了一首词儿,叙其相思契阔之怀。写完,封得停当,走出来交与薛嫂,便说:“你上覆他,教他休要使性儿,往他母舅张家那里吃饭,惹他张舅蜃齿,说你在丈人家做买卖,却来我家吃饭。显得俺们都是没生活的一般,教他张舅怪。或是未有饭吃,教他铺子里拿钱买些点心和伙计吃便了。你使性儿不进来,和谁鳖气哩!却相是贼人胆儿虚一般。”薛嫂道:“等我对他说。”妇人又与了薛嫂五钱银子。

作别出门,来到前边铺子里,寻见敬济。两个走到僻静处说话,把封的物事递与他:“五娘说,教你休使性儿赌鳖气,教你常进来走走,休往你张舅家吃饭去,惹人家怪。”因拿出五钱银子与他瞧:“此是里面与我的,漏眼不藏丝,久后你两个愁不会在一答里?对出来,我脸放在那里?”敬济道:“老薛多有累你。”深深与他唱喏。那薛嫂走了两步,又回来说:“我险些儿忘了一件事,刚才我出来,大娘又使丫头绣春叫我进去,叫我晚上来领春梅,要打发卖他。说他与你们做牵头,和他娘通同养汉。”敬济道:“薛妈,你且领在家。我改日到你家见他一面,有话问他。”那薛嫂说毕,回家去了。

果然到晚夕月上的时分,走来领春梅。到月娘房中,月娘开口说:“那咱原是你手里十六两银子买的,你如今拿十六两银子来就是了。”分付小玉:“你看着,到前边收拾了,教他罄身儿出去,休要带出衣裳去了。”那薛嫂儿到前边,向妇人如此这般:“他大娘教我领春梅姐来了。对我说,他与你老人家通同作弊,偷养汉子,不管长短,只问我要原价。”妇人听见说领卖春梅,就睁了眼,半日说不出话来,不觉满眼落泪,叫道:“薛嫂儿,你看我娘儿两个没汉子的,好苦也!今日他死了多少时儿,就打发我身边人。他大娘这般没人心仁义,自恃他身边养了个尿胞种,就把人躧到泥里。李瓶儿孩子周半还死了哩,花麻痘疹未出,知道天怎么算计,就心高遮了太阳!”薛嫂道:“春梅姐说,爹在日曾收用过他。”妇人道:“收用过二字儿!死鬼把他当心肝肺肠儿一般看待!说一句,听十句,要一奉十,正经成房立纪老婆且打靠后。他要打那个小厮十棍儿,他爹不敢打五棍儿。”薛嫂道:“可又来,大娘差了!爹收用的恁个出色姐儿,打发他,箱笼儿也不与,又不许带一件衣服儿,只教他罄身儿出去,邻舍也不好看的。”妇人道:“他对你说,休教带出衣裳去?”薛嫂道:“大娘分付,小玉姐便来。教他看着,休教带衣裳出去。”那春梅在旁,听见打发他,一点眼泪也没有。见妇人哭,说道:“娘你哭怎的?奴去了,你耐心儿过,休要思虑坏了你。你思虑出病来,没人知你疼热。等奴出去,不与衣裳也罢,自古好男不吃分时饭,好女不穿嫁时衣。”正说着,只见小玉进来,说道:“五娘,你信我奶奶,倒三颠四的。小大姐扶持你老人家一场,瞒上不瞒下,你老人拿出他箱子来,拣上色的包与他两套,教薛嫂儿替他拿了去,做个一念儿,也是他番身一场。”妇人道:“好姐姐,你到有点仁义。”小玉道:“你看,谁人保得常无事!虾蟆、促织儿,都是一锹土上人。兔死狐悲,物伤其类。”一面拿出春梅箱子来,是戴的汗巾儿、翠簪儿,都教他拿去。妇人拣了两套上色罗段衣服鞋脚,包了一大包,妇人梯己与了他几件钗梳簪坠戒指,小玉也头上拔下两根簪子来递与春梅。余者珠子缨络、银丝云髻、遍地金妆花裙袄,一件儿没动,都抬到后边去了。春梅当下拜辞妇人、小玉,洒泪而别。临出门,妇人还要他拜辞拜辞月娘众人,只见小玉摇手儿。这春梅跟定薛嫂,头也不回,扬长决裂,出大门去了。

小玉和妇人送出大门回来。小玉到上房回大娘,只说:“罄身子去了,衣服都留下,没与他。”这金莲归到房中,往常有春梅,娘儿两个相亲相热,说知心话儿,今日他去了,丢得屋里冷冷落落,甚是孤凄,不觉放声大哭。有诗为证:

\[
耳畔言犹在,于今恩爱分。
房中人不见,无语自消魂。
\]

\newpage
%# -*- coding:utf-8 -*-
%%%%%%%%%%%%%%%%%%%%%%%%%%%%%%%%%%%%%%%%%%%%%%%%%%%%%%%%%%%%%%%%%%%%%%%%%%%%%%%%%%%%%


\chapter{雪娥唆打陈敬济\KG 金莲解渴王潮儿}


诗曰:

\[
雨打梨花倍寂寥,几回肠断泪珠抛。
睽违一载犹三载,情绪千丝与万条。
好句每从秋里得,离魂多自梦中消。
香罗重解知何日,辜负巫山几暮朝。
\]

话说潘金莲自从春梅去后,房中纳闷,不题。单表陈敬济,次日上饭时出去,假作讨帐,骑头口到于薛嫂儿家。薛嫂儿正在屋里,一面让进来坐。敬济拴了头口,进房坐下,点茶吃了。薛嫂故意问:“姐夫来有何话说?”敬济道:“我往前街讨帐,竟到这里。昨晚大小姐出来了,和他说句话儿。”薛嫂故作乔张致,说:“好姐夫,昨日你家丈母好不分付我,因为你每通同作弊,弄出丑事来,才把他打发出门,教我防范你们,休要与他会面说话。你还不趁早去哩,只怕他一时使将小厮来看见,到家学了,又是一场儿。倒没的弄的我也上不的门。”那敬济便笑嘻嘻袖中拿出一两银子来:“权作一茶,你且收了,改日还谢你。”那薛嫂见钱眼开,便道:“好姐夫,自恁没钱使,将来谢我!只是我去年腊月,你铺子当了人家两付扣花枕顶,将有一年来,本利该八钱银子,你寻与我罢。”敬济道:“这个不打紧,明日就寻与你。”

这薛嫂儿一面请敬济里间房里去,与春梅厮见,一面叫他媳妇金大姐定菜儿,“我去买茶食点心。”又打了一壶酒,并肉鲊之类,教他二人吃。这春梅看见敬济,说道:“姐夫,你好人儿,就是个弄人的刽子手!把俺娘儿两个弄的上不上下不下,出丑惹人嫌,到这步田地。”敬济道:“我的姐姐,你既出了他家门,我在他家也不久了。‘妻儿赵迎春,各自寻投奔’。你教薛妈妈替你寻个好人家去罢,我‘腌韭菜——已是入不的畦”了。我往东京俺父亲那里去计较了回来,把他家女儿休了,只要我家寄放的箱子。”说毕,不一时,薛嫂买将茶食酒菜来,放炕桌儿摆了,两个做一处饮酒叙话。薛嫂也陪他吃了两盏,一递一句,说了回月娘心狠:“宅里恁个出色姐儿出来,通不与一件儿衣服簪环。就是往人家上主儿去,装门面也不好看。还要旧时原价。就是清水,这碗里倾倒那碗内,也抛撒些儿。原来这等夹脑风。临时出门,倒亏了小玉丫头做了个分上,教他娘拿了两件衣服与他。不是,往人家相去,拿甚么做上盖?”比及吃得酒浓时,薛嫂教他媳妇金大姐抱孩子,躲去人家坐的,教他两个在里间自在坐个房儿。正是:

\[
云淡淡天边鸾凤,水沉沉波底鸳鸯。
写成今世不休书,结下来生欢喜带。
\]
两个干讫,一度作别,比时难割难舍。薛嫂恐怕月娘使人来瞧,连忙撺掇敬济出港,骑上头口来家。

迟不上两日,敬济又稍了两方销金汗巾,两双膝裤与春梅,又寻枕头出来与薛嫂儿。又拿银子打酒,在薛嫂儿房内正和春梅吃酒,不想月娘使了来安小厮来催薛嫂儿:“怎的还不上主儿?”看见头口拴在门首,来安儿到家学了舌,说:“姐夫也在那里来。”月娘听了,心中大怒,使人一替两替叫了薛嫂儿去,尽力数说了一遍,道:“你领了奴才去,今日推明日,明日推后日,只顾不上紧替我打发,好窝藏着养汉挣钱儿与你家使。若是你不打发,把丫头还与我领了来,我另教冯妈妈子卖,你再休上我门来。”这薛嫂儿听了,到底还是媒人的嘴,说道:“天么天么!你老人家怪我差了。我赶着增福神着棍打?你老人家照顾我,怎不打发?昨日也领着走了两三个主儿,都出不上,你老人家要十六两原价,俺媒人家那里有这些银子陪上。”月娘又道:“小厮说陈家种子今日在你家和丫头吃酒来。”薛嫂慌道:“耶嚛!耶嚛!又是一场儿。还是去年腊月,当了人家两付枕顶,在咱狮子街铺内,银子收了,今日姐夫送枕顶与我。我让他吃茶,他不吃,忙忙就上头口来了。几时进屋里吃酒来!原来咱家这大官儿,恁快捣谎驾舌!”月娘吃他一篇,说的不言语了,说道:“我只怕一时被那种子设念随邪,差了念头。”薛嫂道:“我是三岁小孩儿?岂可恁些事儿不知道。你那等分付了我,我长吃好,短吃好?他在那里也没的久停久坐,与了我枕头,茶也没吃就来了。几曾见咱家小大姐面儿来!万物也要个真实,你老人家就上落我起来。既是如此,如今守备周老爷府中,要他图生长,只出十二两银子。看他若添到十三两上,我兑了银子来罢。说起来,守备老爷前者在咱家酒席上,也曾见过小大姐来。因他会这几套唱,好模样儿,才出这几两银子。又不是女儿,其余别人出不上。”薛嫂当下和月娘砸死了价钱。

次日,早把春梅收拾打扮,妆点起来,戴着围发云髻儿,满头珠翠,穿上红段袄儿,蓝段裙子,脚上双鸾尖翘翘,一顶轿子送到守备府中。周守备见了春梅生的模样儿,比旧时越又红又白,身段儿不短不长,一双小脚儿,满心欢喜,就兑出五十两一锭元宝来,这薛嫂儿拿出家,凿下十三两银子,往西门庆家交与月娘,另外又拿出一两来,说:“是周爷赏我的喜钱,你老人家这边不与我些儿?”那吴月娘免不过,只得又秤出五钱银子与他,恰好他还禁了三十七两五钱银子。十个九个媒人,都是如此赚钱养家。

却表陈敬济见卖了春梅,又不得往金莲那边去,见月娘凡事不理他,门户都严禁,到晚夕亲自出来,打灯笼前后照看,上了锁,方才睡去,因此弄不得手脚。敬济十分急了,先和西门大姐嚷了两场,淫妇前淫妇后骂大姐:“我在你家做女婿,不道的雌饭吃,吃伤了!你家收了我许多金银箱笼,你是我老婆,不顾赡我,反说我雌你家饭吃!我白吃你家饭来?”骂的大姐只是哭涕。

十一月念七日,孟玉楼生日。玉楼安排了几碗酒菜点心,好意教春鸿拿出前边铺子,教敬济陪傅伙计吃。月娘便拦说:“他不是才料。休要理他。要与傅伙计,自与傅伙计自家吃就是了,不消叫他。”玉楼不肯。春鸿拿出来,摆在水柜上。一大壶酒都吃了,不勾,又使来巡儿后边要去。傅伙计便说:“姐夫不消要酒去了,这酒勾了,我也不吃了。”敬济不肯,定要来安要去。等了半晌,来安儿出来,回说没了酒了。这陈敬济也有半酣酒儿在肚内,又使他要去,那来安不动。又另拿钱,打了酒来吃着。骂来安儿:“贼小奴才儿,你别要慌!你主子不待见我,连你这奴才每也欺负我起来了,使你使儿不动。我与你家做女婿,不道的酒肉吃伤了,有爹在怎么行来?今日爹没了,就改变了心肠,把我来不理,都乱来挤撮我。我大丈母听信奴才言语,凡事托奴才,不托我。由他,我好耐凉耐怕儿!”傅伙计劝道:“好姐夫,快休舒言。不敬奉姐夫,再敬奉谁?想必后边忙。怎不与姐夫吃?你骂他不打紧,墙有缝,壁有耳,恰似你醉了一般。”敬济道:“老伙计,你不知道,我酒在肚里,事在心头。俺丈母听信小人言语,骂我一篇是非。就算我\textuni{34B2}了人,人没\textuni{34B2}了我?好不好我把这一屋子里老婆都刮剌了,到官也只是后丈母通奸,论个不应罪名。如今我先把你家女儿休了,然后一纸状子告到官。再不,东京万寿门进一本,你家见收着我家许多金银箱笼,都是杨戬应没官赃物。好不好把你这几间业房子都抄没了,老婆便当官办卖。我不图打鱼,只图混水耍子。会事的把俺女婿收笼着,照旧看待,还是大家便益。”傅伙计见他话头儿来的不好,说道:“姐夫,你原来醉了。王十九,只吃酒,且把散话革起。”这敬济眼瞅着傅伙计,骂道:“老贼狗,怎的说我散话!揭跳我醉了,吃了你家酒来?我不才是他家女婿娇客,你无故只是他家行财,你也挤撮我起来!我教你这老狗别要慌,你这几年赚的俺丈人钱勾了,饭也吃饱了,心里要打伙儿把我疾发了去,要夺权儿做买卖,好禁钱养家。我明日本状也带你一笔。教他打官司!”那傅伙计最是个小胆儿的人,见头势不好,穿上衣裳,悄悄往家一溜烟走了。小厮收了家活,后边去了,敬济倒在炕上睡下,一宿晚景题过。

次日,傅伙计早辰进后边,见月娘把前事具诉一遍,哭哭啼啼,要告辞家去,交割帐目,不做买卖了。月娘便劝道:“伙计,你只安心做买卖,休要理那泼才料,如臭屎一般丢着他。当初你家为官事投到俺家来权住着,有甚金银财宝?也只是大姐几件妆奁,随身箱笼。你家老子便躲上东京去了,那时恐怕小人不足,教俺家昼夜耽心。你来时才十六七岁,黄毛团儿也一般。也亏在丈人家养活了这几年,调理的诸般买卖儿都会。今日翅膀毛儿干了,反恩将仇报,一扫帚扫的光光的。小孩儿家说话欺心,恁没天理,到明日只天照看他!伙计,你自安心做你买卖,休理他便了。他自然也羞。”一面把傅伙计安抚住了不题。

一日,也是合当有事,印了铺挤着一屋里人赎讨东西。只见奶子如意儿,抱着孝哥儿送了一壶茶来与傅伙计吃,放在桌上。孝哥儿在奶子怀里,哇哇的只管哭。这陈敬济对着那些人,作耍当真说道:“我的哥哥,乖乖儿,你休哭了。”向众人说:“这孩子倒相我养的,依我说话,教他休哭,他就不哭了。”那些人就呆了。如意儿说:“姐夫,你说的好妙话儿,越发叫起儿来了,看我进房里说不说。”这陈敬济赶上踢了奶子两脚,戏骂道:“怪贼邋遢,你说不是!我且踢个响屁股儿着。”那奶子抱孩子走到后边,如此这般向月娘哭说:“姐夫对众人将哥儿这般言语发出来。”这月娘不听便罢,听了此言,正在镜台边梳着头,半日说不出话来,往前一撞,就昏倒在地,不省人事。但见:

\[
荆山玉损,可惜西门庆正室夫妻;宝鉴花残,枉费九十日东君匹配。花容掩淡,犹如西园芍药倚朱栏;檀口无言,一似南海观音来入定。小园昨日春风急,吹折江梅就地花。
\]
慌了小玉,叫将家中大小,扶起月娘来炕上坐的。孙雪娥跳上炕,撅救了半日,舀姜汤灌下去,半日苏醒过来。月娘气堵心胸,只是哽咽,哭不出声来。奶子如意儿对孟玉楼、孙雪娥,将敬济对众人将哥儿戏言之事,说了一遍:“我好意说他,又赶着我踢了两脚,把我也气的发昏在这里。”雪娥扶着月娘,待的众人散去,悄悄在房中对月娘说:“娘也不消生气,气的你有些好歹,越发不好了。这小厮因卖了春梅,不得与潘家那淫妇弄手脚,才发出话来。如今一不做,二不休,大姐已是嫁出女,如同卖出田一般,咱顾不得他这许多。常言养虾蟆得水蛊儿病,只顾教那小厮在家里做甚么!明日哄赚进后边,下老实打与他一顿,即时赶离门,教他家去。然后叫将王妈妈子来,把那淫妇教他领了去,变卖嫁人,如同狗臭尿,掠将出去,一天事都没了。平空留着他在家里做甚么!到明日,没的把咱们也扯下水去了。”月娘道:“你说的也是。”当下计议已定了。

到次日,饭时已后,月娘埋伏了丫鬟媳妇七八个人,各拿短棍棒槌。使小厮来安儿请进陈敬济来后边,只推说话。把仪门关了,教他当面跪下,问他:“你知罪么?”那陈敬济也不跪,转把脸儿高扬,佯佯不采。月娘大怒,于是率领雪娥并来兴儿媳妇、来昭妻一丈青、中秋儿、小玉、绣春众妇人,七手八脚,按在地下,拿棒槌短棍打了一顿。西门大姐走过一边,也不来救。打的这小伙儿急了,把裤子脱了,露出那直竖一条棍来。唬的众妇人看见,却丢下棍棒乱跑了。月娘又是那恼,又是那笑,口里骂道:“好个没根基的王八羔子!”敬济口中不言,心中暗道:“若不是我这个法儿,怎得脱身。”于是扒起来,一手兜着裤子,往前走了。月娘随令小厮跟随,教他算帐,交与傅伙计。敬济自知也立脚不定,一面收拾衣服铺盖,也不作辞,使性儿一直出离西门庆家,径往他母舅张团练家,他旧房子自住去了。正是:

\[
唯有感恩并积恨,万年千载不生尘。
\]

潘金莲在房中,听见打了陈敬济,赶离出门去了,越发忧上加忧,闷上添闷。一日,月娘听信雪娥之言,使玳安儿去叫了王婆来。那王婆自从他儿子王潮跟淮上客人,拐了起车的一百两银子来家,得其发迹,也不卖茶了,买了两个驴儿,安了盘磨,一张罗柜,开起磨房来。听见西门庆宅里叫他,连忙穿衣就走,到路上问玳安说:“我的哥哥,几时没见你,又早笼起头去了,有了媳妇儿不曾?”玳安道:“还不曾有哩。”王婆子道:“你爹没了,你家谁人请我做甚么?莫不是你五娘养了儿子了,请我去抱腰?”玳安道:“俺五娘倒没养儿子,倒养了女婿。俺大娘请你老人家,领他出来嫁人。”王婆子道:“天么,天么,你看么!我说这淫妇,死了你爹,怎守的住。只当狗改不了吃屎,就弄碜儿来了。就是你家大姐那女婿子?他姓甚么?”玳安道:“他姓陈,名唤陈敬济。”王婆子道:“想着去年,我为何老九的事,去央烦你爹。到宅内,你爹不在,贼淫妇他就没留我房里坐坐儿,折针也迸不出个来,只叫丫头倒一钟清茶我吃了,出来了。我只道千年万岁在他家,如何今日也还出来!好个浪蹄子淫妇,休说我是你个媒王,替你作成了恁好人家,就是闲人进去,也不该那等大意。”玳安道:“为他和俺姐夫在家里炒嚷作乱,昨日差些儿没把俺大娘气杀了哩。俺姐夫已是打发出去了,只有他老人家,如今教你领他去哩。”王婆子道:“他原是轿儿来,少不得还叫顶轿子。他也有个箱笼来,这里少不的也与他个箱子儿。”玳安道:“这个少不的,俺大娘自有个处。”

两个说话间,到了门首。进入月娘房里,道了万福坐下,丫鬟拿茶吃了。月娘便道:“老王,无事不请你来。”悉把潘金莲如此这般,上项说了一遍:“今来是是非人,去是非者。一客不烦二王,还起动你领他出去,或聘嫁,或打发,叫他吃自在饭去罢。我男子汉已是没了,招揽不过这些人来。说不的当初死鬼为他丢了许多钱底那话了,就打他恁个人儿也有。如今随你聘嫁,多少儿交得来,我替他爹念个经儿,也是一场勾当。”王婆道:“你老人家,是稀罕这钱的?只要把祸害离了门就是了。我知道,我也不肯差了。”又道:“今日好日,就出去罢。又一件,他当初有个箱笼儿,有顶轿儿来,也少不的与他顶轿儿坐了去。”月娘道:“箱子与他一个,轿子不容他坐。”小玉道:“俺奶奶气头上便是这等说,到临岐,少不的雇顶轿儿。不然街坊人家看着,抛头露面的,不吃人笑话?”月娘不言语了,一面使丫鬟绣春,前边叫金莲来。

这金莲一见王婆子在房里,就睁了,向前道了万福,坐下。王婆子开言便道:“你快收拾了。刚才大娘说,教我今日领你出去哩。”金莲道:“我汉子死了多少时儿,我为下甚么非,作下甚么歹来?如何平空打发我出去?”王婆道:“你休稀里打哄,做哑装聋!自古蛇钻窟窿蛇知道,各人干的事儿,各人心里明。金莲你休呆里撒奸,说长道短,我手里使不的巧语花言,帮闲钻懒。自古没个不散的筵席,出头椽儿先朽烂,人的名儿,树的影儿。苍蝇不钻没缝儿蛋,你休把养汉当饭,我如今要打发你上阳关。”金莲见势头不好,料难久住,便也发话道:“你打人休打脸,骂人休揭短!有势休要使尽了,赶人不可赶上。我在你家做老婆,也不是一日儿,怎听奴才淫妇戳舌,便这样绝情绝义的打发我出去!我去不打紧,只要大家硬气,守到老没个破字儿才好。”当下金莲与月娘乱了一回。月娘到他房中,打点与了他两个箱子,一张抽替桌儿,四套衣服,几件钗梳簪环,一床被褥。其余他穿的鞋脚,都填在箱内。把秋菊叫到后边来,一把锁就把房门锁了。金莲穿上衣服,拜辞月娘,在西门庆灵前大哭了一回。又走到孟玉楼房中,也是姊妹相处一场,一旦分离,两个落了一回眼泪。玉楼瞒着月娘,悄悄与了他一对金碗簪子,一套翠蓝段袄、红裙子,说道:“六姐,奴与你离多会少了,你看个好人家,往前进了罢。自古道,千里长篷,也没个不散的筵席。你若有了人家,使个人来对我说声,奴往那里去,顺便到你那里看你去,也是姐妹情肠。”于是洒泪而别。临出门,小玉送金莲,悄悄与了金莲两根金头簪儿。金莲道:“我的姐姐,你倒有一点人心儿在我。”王婆又早雇人把箱笼桌子抬的先去了。独有玉楼、小玉送金莲到门首,坐了轿子才回。正是:

\[
世上万般哀苦事,无非死别共生离。
\]

却说金莲到王婆家,王婆安插他在里间,晚夕同他一处睡。他儿子王潮儿,也长成一条大汉,笼起头去了,还未有妻室,外间支着床睡。这潘金莲次日依旧打扮,乔眉乔眼在帘下看人。无事坐在炕上,不是描眉画眼,就是弹弄琵琶。王婆不在,就和王潮儿斗叶儿、下棋。那王婆自去扫面,喂养驴子,不去管他。朝来暮去,又把王潮儿刮剌上了。晚间等的王婆子睡着了,妇人推下炕溺尿,走出外间床上,和王潮儿两个干,摇的床子一片响声。被王婆子醒来听见,问那里响。王潮儿道:“是柜底下猫捕老鼠响。”王婆子睡梦中,喃喃呐呐,口里说道:“只因有这些麸面在屋里,引的这扎心的半夜三更耗爆人,不得睡。”良久,又听见动旦,摇的床子格支支响,王婆又问那里响。王潮道:“是猫咬老鼠,钻在炕洞下嚼的响。”婆子侧耳,果然听见猫在炕洞里咬的响,方才不言语了。妇人和小厮干完事,依旧悄悄上炕睡去了。有几句双关,说得这老鼠好:

\[
你身躯儿小,胆儿大,嘴儿尖,忒泼皮。见了人藏藏躲躲,耳边厢叫叫唧唧,搅混人半夜三更不睡。不行正人伦,偏好钻穴隙。更有一桩儿不老实,到底改不的偷馋抹嘴。
\]

有日,陈敬济打听得潘金莲出来,还在王婆家聘嫁,因提着两吊铜钱,走到王婆家来。婆子正在门前扫驴子撒的粪。这敬济向前深深地唱个喏。婆子问道:“哥哥,你做甚么?”敬济道:“请借里边说话。”王婆便让进里面。敬济便道:“动问西门大官人宅内,有一位娘子潘六姐,在此出嫁?”王婆便道:“你是他甚么人?”那敬济嘻嘻笑道:“不瞒你老人家说,我是他兄弟,他是我姐姐。”那王婆子眼上眼下,打量他一回,说:“他有甚兄弟,我不知道,你休哄我。你莫不是他家女婿姓陈的,在此处撞蠓子,我老娘手里放不过。”敬济笑向腰里解下两吊铜钱来,放在面前,说:“这两吊钱权作王奶奶一茶之费,教我且见一面,改日还重谢你老人家。”婆子见钱,越发乔张致起来,便道:“休说谢的话。他家大娘子分付将来,不许教闲杂人来看他。咱放倒身说话,你既要见这雌儿一面,与我五两银子,见两面与我十两。你若娶他,便与我一百两银子,我的十两媒人钱在外。我不管闲帐。你如今两串钱儿,打水不浑的,做甚么?”敬济见这虔婆口硬,不收钱,又向头上拔下一对金头银脚簪子,重五钱,杀鸡扯腿跪在地下,说道:“王奶奶,你且收了,容日再补一两银子来与你,不敢差了。且容我见他一面,说些话儿则个。”那婆子于是收了簪子和钱,分付:“你进去见他,说了话就与我出来。不许你涎眉睁目,只顾坐着。所许那一两头银子,明日就送来与我。”于是掀帘,放敬济进里间。妇人正坐在炕上,看见敬济,便埋怨他道:“你好人儿!弄的我前不着村,后不着店,有上稍,没下稍,出丑惹人嫌。你就影儿也不来看我看儿了。我娘儿们好好的,拆散的你东我西,皆是为谁来?”说着,扯住敬济,只顾哭泣。王婆又嗔哭,恐怕有人听见。敬济道:“我的姐姐,我为你剐皮剐肉,你为我受气耽羞,怎不来看你?昨日到薛嫂儿家,已知春梅卖在守备府里去了,才打听知你出离了他家门,在王奶奶这边聘嫁。今日特来见你一面,和你计议。咱两个恩情难舍,拆散不开,如之奈何?我如今要把他家女儿休了,问他要我家先前寄放金银箱笼。他若不与我,我东京万寿门一本一状进下来,那里他双手奉与我还是迟了。我暗地里假名托姓,一顶轿子娶到你家去,咱两个永远团圆,做上个夫妻,有何不可?”妇人道:“现今王干娘要一百两银子,你有这些银子与他?”敬济道:“如何人这许多?”婆子说道:“你家大丈母说,当初你家爹,为他打个银人儿也还多,定要一百两银子,少一丝毫也成不的。”敬济道:“实不瞒你老人家说,我与六姐打得热了,拆散不开,看你老人家下顾,退下一半儿来,五六十两银子也罢,我往母舅那里典上两三间房子,娶了六姐家去,也是春风一度。你老人家少转些儿罢。”婆子道:“休说五六十两银子,八十两也轮不到你手里了。昨日湖州贩绸绢何官人,出到七十两;大街坊张二官府,如今见在提刑院掌刑,使了两个节级来,出到八十两上,拿着两卦银子来兑,还成不的,都回去了。你这小孩儿家,空口来说空话,倒还敢奚落老娘,老娘不道的吃伤了哩!”当下一直走出街上,大吆喝说:“谁家女婿要娶丈母,还来老娘屋里放屁!”敬济慌了,一手扯进婆子来,双膝跪下央及:“王奶奶噤声,我依王奶奶价值一百两银子罢。争奈我父亲在东京,我明日起身往东京取银子去。”妇人道:“你既为我一场,休与干娘争执,上紧取去,只恐来迟了,别人娶了奴去,就不是你的人了。”敬济道:“我雇头口连夜兼程,多则半月,少则十日就来了。”婆子道:“常言先下米先食饭,我的十两银子在外,休要少了,我先与你说明白着。”敬济道:“这个不必说,恩有重报,不敢有忘。”说毕,敬济作辞出门,到家收拾行李,次日早雇头口,上东京取银子去。此这去,正是:

\[
青龙与白虎同行,吉凶事全然未保。
\]

\newpage
%# -*- coding:utf-8 -*-
%%%%%%%%%%%%%%%%%%%%%%%%%%%%%%%%%%%%%%%%%%%%%%%%%%%%%%%%%%%%%%%%%%%%%%%%%%%%%%%%%%%%%


\chapter{王婆子贪财忘祸\KG 武都头杀嫂祭兄}


诗曰:

\[
悠悠嗟我里,世乱各东西。存者问消息,死者为尘泥。
贱子家既败,壮士归来时。行久见空巷,日暮气惨凄。
但逢狐与狸,竖毛怒裂眦。我有镯镂剑,对此吐长霓。
\]

话说陈敬济雇头口起身,叫了张团练一个伴当跟随,早上东京去不题。却表吴月娘打发潘金莲出门,次日使春鸿叫薛嫂儿来,要卖秋菊。这春鸿正走到大街,撞见应伯爵,叫住问:“春鸿,你往那里去?”春鸿道:“大娘使小的叫媒人薛嫂儿去。”伯爵问:“叫媒人做甚么?”春鸿道:“卖五娘房里秋菊丫头。”伯爵又问:“你五娘为甚么打发出来嫁人?”这春鸿便如此这般,“因和俺姐夫有些说话,大娘知道了,先打发了春梅小大姐,然后打了俺姐夫一顿,赶出往家去了。昨日才打发出俺五娘来。”伯爵听了,点了点头儿,说道:“原来你五娘和你姐夫有楂儿,看不出人来。”又向春鸿说:“孩儿,你爹已是死了,你只顾还在他家做甚么?终是没出产。你心里还要归你南边去?还是这里寻个人家跟罢。”春鸿道:“便是这般说。老爹已是没了,家中大娘好不严禁,各处买卖都收了,房子也卖了,琴童儿、画童儿都走了,也揽不过这许多人口来。小的待回南边去,又没顺便人带去。这城内寻个人家跟,又没个门路。”伯爵道:“傻孩儿,人无远见,安身不牢。千山万水,又往南边去做甚?你肚里会几句唱,愁这城内寻不出主儿来答应。我如今举保个门路与你。如今大街坊张二老爹家,有万万贯家财,见顶补了你爹在提刑院做掌刑千户。如今你二娘又在他家做了二房,我把你送到他宅中答应,他见你会唱南曲,管情一箭就上垛,留下你做个亲随大官儿,又不比在你家里。他性儿又好,年纪小小,又倜傥,又爱好,你就是个有造化的。”这春鸿扒倒地下就磕了个头:“有累二爹。小的若见了张老爹,得一步之地,买礼与二爹磕头。”伯爵一把手拉着春鸿说:“傻孩儿,你起来,我无有个不作成人的,肯要你谢?你那得钱儿来!”春鸿道:“小的去了,只怕家中大娘抓寻小的怎了?”伯爵道:“这个不打紧。我问你张二老爹讨个贴儿,封一两银子与他家。他家银子不敢受,不怕不把你不双手儿送了去。”说毕,春鸿往薛嫂儿家,叫了薛嫂儿。见月娘,领秋菊出来,只卖了五两银子,交与月娘,不在话下。

却说应伯爵领春鸿到张二官宅里见了。张二官见他生的清秀,又会唱南曲,就留下他答应。便拿拜贴儿,封了一两银子,送往西门庆家,讨他箱子。那日吴月娘家中正陪云离守娘子范氏吃酒。先是云离守补在清河左卫做同知,见西门庆死了,吴月娘守寡,手里有东西,就安心有垂涎图谋之意。此日正买了八盘羹果礼物,来看月娘。见月娘生了孝哥,范氏房内亦有一女,方两月儿,要与月娘结亲。那日吃酒,遂两家割衫襟,做了儿女亲家,留下一双金环为定礼。听见玳安儿拿进张二官府贴儿,并一两银子,说春鸿投在他家答应去了,使人来讨他箱子衣服。月娘见他见做提刑官,不好不与他,银子也不曾收,只得把箱子与将出来。

初时,应伯爵对张二官说:“西门庆第五娘子潘金莲生得标致,会一手琵琶。百家词曲,双陆象棋,无不通晓,又会写字。因为年小守不的,又和他大娘合气,今打发出来,在王婆家嫁人。”这张二官一替两替使家人拿银子往王婆家相看,王婆只推他大娘子分付,不倒口要一百两银子。那人来回讲了几遍,还到八十两上,王婆还不吐口儿。落后春鸿到他宅内,张二官听见春鸿说,妇人在家养育女婿方打发出来。这张二官就不要了,对着伯爵说:“我家现放着十五岁未出幼儿子上学攻书,要这样妇人来家做甚?”又听见李娇儿说,金莲当初用毒药摆布死了汉子,被西门庆占将来家,又偷小厮,把第六个娘子娘儿两个,生生吃他害杀了。以此张二官就不要了。

话分两头。却说春梅卖到守备府中,守备见他生的标致伶俐,举止动人,心中大喜。与了他三间房住,手下使一个小丫鬟,就一连在他房中歇了三夜。三日,替他裁了两套衣服。薛嫂儿去,赏了薛嫂五钱银子。又买了个使女扶持他,立他做第二房。大娘子一目失明,吃长斋念佛,不管闲事。还有生姐儿孙二娘,在东厢居住。春梅在西厢房,各处钥匙都教他掌管,甚是宠爱他。一日,听薛嫂儿说,金莲出来在王婆家聘嫁,这春梅晚夕啼啼哭哭对守备说:“俺娘儿两个,在一处厮守这几年,他大气儿不着呵着我,把我当亲女儿一般看承。只知拆散开了,不想今日他也出来了,你若肯娶将他来,俺娘儿每还在一处,过好日子。”又说他怎的好模样儿,诸般词曲都会,又会弹琵琶。聪明俊俏,百伶百俐。属龙的,今才三十二岁儿。“他若来,奴情愿做第三也罢。”于是把守备念转了,使手下亲随张胜、李安封了二方手帕,二钱银子,往王婆家相看,果然生的好个出色的妇人。王婆开口指称他家大娘子要一百两银子。张胜、李安讲了半日,还了八十两,那王婆不肯,不转口儿,要一百两:“媒人钱不要便罢了,天也不使空人。”这张胜、李安只得又拿回银子来禀守备。丢了两日,怎禁这春梅晚夕啼啼哭哭:“好歹再添几两银子,娶了来和奴做伴儿,死也甘心。”守备见春梅只是哭泣,只得又差了大管家周忠,同张胜《李安,毡包内拿着银子,打开与婆子看,又添到九十两上。婆子越发张致起来,说:“若九十两,到不的如今,提刑张二老爹家抬的去了。”这周忠就恼了,分付李安把银子包了,说道:“三只脚蟾便没处寻,两脚老婆愁寻不出来!这老淫妇连人也不识。你说那张二官府怎的,俺府里老爹管不着你?不是新娶的小夫人再三在老爷跟前说念,要娶这妇人,平白出这些银子,要他何用!”李安道:“勒掯俺两番三次来回,贼老淫妇,越发鹦哥儿风了!”拉着周忠说:“管家,咱去来,到家回了老爷,好不好教牢子拿去,拶与他一顿好拶子。”这婆子终是贪着陈敬济那口食,由他骂,只是不言语。二人到府中,回禀守备说:“已添到九十两,还不肯。”守备说:“明日兑与他一百两,拿轿子抬了来罢。”周忠说:“爷就与了一百两,王婆还要五两媒人钱。且丢他两日,他若张致,拿到府中拶与他一顿拶子,他才怕。”看官听说,大段金莲生有地而死有处,不争被周忠说这两句话。有分交:这妇人从前作过事,今朝没兴一齐来。有诗为证:

\[
人生虽未有前知,祸福因由更问谁。
善恶到头终有报,只争来早与来迟。
\]

按下一头。单表武松自从垫发孟州牢城充军之后,多亏小管营施恩看顾。次后,施恩与蒋门神争夺快活林酒店,被蒋门神打伤,央武松出力,反打了蒋门神一顿。不想蒋门神妹子玉兰,嫁与张都监为妾,赚武松去,假捏贼情,将武松拷打,转又发安平寨充军。这武松走到飞云浦,又杀了两个公人,复回身杀了张都监、蒋门神全家老小,逃躲在施恩家。施恩写了一封书,皮箱内封了一百两银子,教武松到安平寨与知寨刘高,教看顾他。不想路上听见太子立东宫,放郊天大赦,武松就遇赦回家,到清河县下了文书,依旧在县当差,还做都头。来到家中,寻见上邻姚一郎,交付迎儿。那时迎儿已长大十九岁了,收揽来家,一处居住。就有人告他说:“西门庆已死,你嫂子又出来了,如今还在王婆家,早晚嫁人。”这汉子扣了,旧仇在心。正是:

\[
踏破铁鞋无觅处,得来全不费工夫。
\]

次日,理帻穿衣,径走过间壁王婆门首。金莲正在帘下站着,见武松来,连忙闪入里间去。武松掀开帘子便问:“王妈妈在家?”那婆子正在磨上扫面,连忙出来应道:“是谁叫老身?”见是武松,道了万福。武松深深唱喏。婆子道:“武二哥,且喜,几时回家来了?”武松道:“遇赦回家,昨日才到。一向多累妈妈看家,改日相谢。”婆子笑嘻嘻道:“武二哥比旧时保养,胡子楂儿也有了,且是好身量,在外边又学得这般知礼。”一面请他上坐,点茶吃了。武松道:“我有一桩事和妈妈说。”婆子道:“有甚事?武二哥只顾说。”武松道:“我闻的人说,西门庆已是死了,我嫂子出来,在你老人家这里居住。敢烦妈妈对嫂子说,他若不嫁人便罢,若是嫁人,如是迎儿大了,娶得嫂子家去,看管迎儿,早晚招个女婿,一家一计过日子,庶不教人笑话。”婆子初时还不吐口儿,便道:“他在便在我这里,倒不知嫁人不嫁人。”次后听见说谢他,便道:“等我慢慢和他说。”

那妇人在帘内听见武松言语,要娶他看管迎儿,又见武松在外出落得长大身材,胖了,比昔时又会说话儿,旧心不改,心下暗道:“我这段姻缘还落在他手里。”就等不得王婆叫他,自己出来,向武松道了万福,说道:“既是叔叔还要奴家去看管迎儿,招女婿成家,可知好哩。”王婆道:“我一件,只如今他家大娘子,要一百两银子才嫁人。”武松道:“如何要这许多?”王婆道:“西门大官人,当初为他使了许多,就打恁个银人儿也勾了。”武松道:“不打紧,我既要请嫂嫂家去,就使一百两也罢。另外破五两银子,与你老人家。”这婆子听见,喜欢的屁滚尿流,没口说道:“还是武二哥知礼,这几年江湖上见的事多,真是好汉。”妇人听了此言,走到屋里,又浓浓点了一钟瓜仁泡茶,双手递与武松吃了。婆子问道:“如今他家要发脱的紧,又有三四个官户人家争着娶,都回阻了,价钱不兑。你这银子,作速些便好。常言先下米先吃饭,千里姻缘着线牵,休要落在别人手内。”妇人道:“既要娶奴家,叔叔上紧些。”武松便道:“明日就来兑银子,晚夕请嫂嫂过去。”那王婆还不信武松有这些银子,胡乱答应去了。

到次日,武松打开皮箱,拿出施恩与知寨刘高那一百两银子来,又另外包了五两碎银子,走到王婆家,拿天平兑起来。那婆子看见白晃晃摆了一桌银子,口中不言,心内暗道:“虽是陈敬济许下一百两,上东京去取,不知几时到来。仰着合着,我见钟不打,去打铸钟?”又见五两谢他,连忙收了。拜了又拜,说道:“还是武二哥知人甘苦。”武松道:“妈妈收了银子,今日就请嫂嫂过门。”婆子道:“武二哥,且是好急性。门背后放花儿——你等不到晚了?也待我往他大娘那里交了银子,才打发他过去。”又道:“你今日帽儿光光,晚夕做个新郎。”那武松紧着心中不自在,那婆子不知好歹,又奚落他。打发武松出门,自己寻思:“他家大娘只叫我发脱,又没和我断定价钱,我今胡乱与他一二十两银子就是了,绑着鬼也落他一半多养家。”就把银凿下二十两银子,往月娘家里交割明白。月娘问:“甚么人家娶去了?”王婆道:“兔儿沿山跑,还来归旧窝。嫁了他家小叔,还吃旧锅里粥去了。”月娘听了,暗中跌脚,常言“仇人见仇人,分外眼睛明”,与孟玉楼说:“往后死在他小叔子手里罢了。那汉子杀人不斩眼,岂肯干休!”

不说月娘家中叹息,却表王婆交了银子到家,下午时,教王潮先把妇人箱笼桌儿送过去。这武松在家中又早收拾停当,打下酒肉,安排下菜蔬。晚上婆子领妇人过门,换了孝,带着新\textuni{4BFC}髻,身穿红衣服,搭着盖头。进门来,见明间内明亮亮点着灯烛,重立武大灵牌供养在上面,先有些疑忌,由不的发似人揪,肉如钩搭。进入门来,到房中,武松分付迎儿把前门上了拴,后门也顶了。王婆见了,说道:“武二哥,我去罢,家里没人。”武松道:“妈妈请进房里吃盏酒。”武松教迎儿拿菜蔬摆在桌上,须臾烫上酒来,请妇人和王婆吃酒。那武松也不让,把酒斟上,一连吃了四五碗酒。婆子见他吃得恶,便道:“武二哥,老身酒勾了,放我去,你两口儿自在吃罢。”武松道:“妈妈,且休得胡说!我武二有句话问你!”只闻飕的一声响,向衣底掣出一把二尺长刃薄背厚的朴刀来,一只手笼着刀靶,一只手按住掩心,便睁圆怪眼,倒竖刚须,说道:“婆子休得吃惊!自古冤有头,债有主,休推睡里梦里。我哥哥性命都在你身上!”婆子道:“武二哥,夜晚了,酒醉拿刀弄杖,不是耍处。”武松道:“婆子休胡说,我武二就死也不怕!等我问了这淫妇,慢慢来问你这老猪狗!若动一动步儿,先吃我五七刀子。”一面回过脸来,看着妇人骂道:“你这淫妇听着!我的哥哥怎生谋害了?从实说来,我便饶你。”那妇人道:“叔叔如何冷锅中豆儿炮?好没道理!你哥哥自害心疼病死了,干我甚事?”说由未了,武松把刀子忔楂的插在桌子上,用左手揪住妇人云髻,右手匹胸提住,把桌子一脚踢番,碟儿盏儿都打得粉碎。那妇人能有多大气脉,被这汉子隔桌子轻轻提将起来,拖出外间灵桌子前。那婆子见势头不好,便去奔前门走,前门又上了栓。被武松大叉步赶上,揪番在地,用腰间缠带解下来,四手四脚捆住,如猿猴献果一般,便脱身不得,口中只叫:“都头不消动意,大娘子自做出来,不干我事。”武松道:“老猪狗,我都知道了,你赖那个?你教西门庆那厮垫发我充军去,今日我怎生又回家了!西门庆那厮却在那里?你不说时,先剐了这个淫妇,后杀你这老猪狗!”提起刀来,便望那妇人脸上撇了两撇。

妇人慌忙叫道:“叔叔且饶,放我起来,等我说便了。”武松一提,提起那婆娘,旋剥净了,跪在灵桌子前。武松喝道:“淫妇快说!”那妇人唬得魂不附体,只得从实招说,将那时收帘子打了西门庆起,并做衣裳入马通奸,后怎的踢伤武大心窝,王婆怎地教唆下毒,拨置烧化,又怎的娶到家去,一五一十,从头至尾,说了一遍。王婆听见,只是暗中叫苦,说:“傻才料,你实说了,却教老身怎的支吾。”这武松一面就灵前一手揪着妇人,一手浇奠了酒,把纸钱点着,说道:“哥哥,你阴魂不远,今日武松与你报仇雪恨。”那妇人见势头不好,才待大叫。被武松向炉内挝了一把香灰,塞在他口,就叫不出来了。然后劈脑揪番在地。那妇人挣扎,把\textuni{4BFC}髻簪环都滚落了。武松恐怕他挣扎,先用油靴只顾踢他肋肢,后用两只手去摊开他胸脯,说时迟,那时快,把刀子去妇人白馥馥心窝内只一剜,剜了个血窟窿,那鲜血就冒出来。那妇人就星眸半闪,两只脚只顾登踏。武松口噙着刀子,双手去斡开他胸脯,扎乞的一声,把心肝五脏生扯下来,血沥沥供养在灵前。后方一刀割下头来,血流满地。迎儿小女在旁看见,唬的只掩了脸。武松这汉子端的好狠也。可怜这妇人,正是三寸气在千般用,一日无常万事休。亡年三十二岁。但见:

\[
手到处青春丧命,刀落时红粉亡身。七魄悠悠,已赴森罗殿上;三魂渺渺,应归枉成城中。好似初春大雪压折金钱柳,腊月狂风吹折玉梅花。这妇人娇媚不知归何处,芳魂今夜落谁家?
\]
古人有诗一首,单悼金莲死的好苦也:

\[
堪悼金莲诚可怜,衣裳脱去跪灵前。
谁知武二持刀杀,只道西门绑腿顽。
往事看嗟一场梦,今身不值半文钱。
世间一命还一命,报应分明在眼前。
\]

武松杀了妇人,那婆子便叫:“杀人了!”武松听见他叫,向前一刀,也割下头来。拖过尸首。一边将妇人心肝五脏,用刀插在后楼房檐下。

那时有初更时分,倒扣迎儿在屋里。迎儿道:“叔叔,我害怕!”武松道:“孩儿,我顾不得你了。”武松跳过王婆家来,还要杀他儿子王潮。不想王潮合当不该死,听见他娘这边叫,就知武松行凶,推前门不开,叫后门也不应,慌的走去街上叫保甲。那两邻明知武松凶恶,谁敢向前。武松跳过墙来,到王婆房内,只见点着灯,房内一人也没有。一面打开王婆箱笼,就把他衣服撇了一地。那一百两银子止交与吴月娘二十两,还剩了八十五两,并些钗环首饰,武松都包裹了。提了朴刀,越后墙,赶五更挨出城门,投十字坡张青夫妇那里躲住,做了头佗,上梁山为盗去了。正是:

\[
平生不作绉眉事,世上应无切齿人。
\]


\newpage
%# -*- coding:utf-8 -*-
%%%%%%%%%%%%%%%%%%%%%%%%%%%%%%%%%%%%%%%%%%%%%%%%%%%%%%%%%%%%%%%%%%%%%%%%%%%%%%%%%%%%%


\chapter{陈敬济感旧祭金莲\KG 庞大姐埋尸托张胜}


诗曰:

\[
梦中虽暂见,及觉始知非。展转不成寐,徒倚独披衣。
凄凄晓风急,腌腌月光微。空床常达旦,所思终不归。
\]

话说武松杀了妇人、王婆,劫去财物,逃上梁山去了,不题。且说王潮儿街上叫了保甲来,见武松家前后门都不开,又王婆家被劫去财物,房中衣服丢的横三竖四,就知是武松杀人劫财而去。未免打开前后门,见血沥沥两个死尸倒在地下,妇人心肝五脏用刀插在后楼房檐下。迎儿倒扣在房中。问其故,只是哭泣。次日早衙,呈报到本县,杀人凶刃都拿放在面前。本县新任知县也姓李,双名昌期,乃河北真定府枣强县人氏。听见杀人公事,即委差当该吏典,拘集两邻保甲,并两家苦主王潮、迎儿。眼同当街,如法检验。生前委被武松因忿带酒,杀潘氏、王婆二命,叠成文案,就委地方保甲瘗埋看守。挂出榜文,四厢差人跟寻,访拿正犯武松,有人首告者,官给赏银五十两。

守备府中张胜、李安打着一百两银子到王婆家,看见王婆、妇人俱已被武松杀死,县中差人检尸,捉拿凶犯。二人回报到府中。春梅听见妇人死了,整哭了两三日,茶饭都不吃。慌了守备,使人门前叫调百戏的货郎儿进去,耍与他观看,只是不喜欢。日逐使张胜、李安打听,拿住武松正犯,告报府中知道,不在话下。

按下一头。且表陈敬济前往东京取银子,一心要赎金莲,成其夫妇。不想走到半路,撞见家人陈定从东京来,告说家爷病重之事:“奶奶使我来请大叔往家去,嘱托后事。”这敬济一闻其言,两程做一程,路上趱行。有日到东京他姑夫张世廉家。张世廉已死,止有姑娘见在。他父亲陈洪已是没了三日,满家带孝。敬济参见他父亲灵座。与他母亲张氏并姑娘磕头。张氏见他成人,母子哭做一处,通同商议:“如今一则以喜,一则以忧。”敬济便道:“如何是喜,如何是忧?”张氏道:“喜者,如今朝廷册立东宫,郊天大赦;忧则不想你爹爹病死在这里,你姑夫又没了,姑娘守寡,这里住着不是常法,如今只得和你打发你爹爹灵柩回去,葬埋乡井,也是好处。”敬济听了,心内暗道:“这一回发送,装载灵柩家小粗重上车,少说也得许多日期耽阁,却不误了六姐?不如先诓了两车细软箱笼家去,待娶了六姐,再来搬取灵柩不迟。”一面对张氏说道:“如今随路盗贼,十分难走。假如灵柩家小箱笼一同起身,未免起眼,倘遇小人怎了?宁可耽迟不耽错。我先押两车细软箱笼家去,收拾房屋。母亲随后和陈定、家眷并父亲灵柩,过年正月同起身回家,寄在城外寺院,然后做斋念经、筑坟安葬,也是不迟。”张氏终是妇人家,不合一时听信敬济巧言,就先打点细软箱笼,装载两大车,上插旗号,扮做香车。从腊月初一日东京起身,不上数日,到了山东清河县家门首,对他母舅张团练说:“父亲已死,母亲押灵车,不久就到。我押了两车行李,先来收拾打扫房屋。”他母舅听说:“既然如此,我仍搬回家去便了。”一面就令家人搬家活,腾出房子来。敬济见母舅搬去,满心欢喜,说:“且得冤家离眼前,落得我娶六姐来家,自在受用。我父亲已死,我娘又疼我。先休了那个淫妇,然后一纸状子,把俺丈母告到官,追要我寄放东西,谁敢道个不字?又挟制俺家充军人数不成!”正是:

\[
人便如此如此,天理不然不然。
\]

这敬济就打了一百两银子在腰里,另外又袖着十两谢王婆,来到紫石街王婆门首。可霎作怪,只见门前街旁埋着两个尸首,上面两杆枪交叉挑着个灯笼,门前挂着一张手榜,上书:“本县为人命事:凶犯武松,杀死潘氏、王婆二命,有人捕获首告官司者,官给赏银五十两。”这敬济仰头看见,便立睁了。只见窝铺中站出两个人来,喝声道:“甚么人?看此榜文做甚?见今正身凶犯捉拿不着,你是何人?”大叉步便来捉获。敬济慌的奔走不迭,恰走到石桥下酒楼边,只见一个人,头戴万字巾,身穿青衲袄,随后赶到桥下,说道:“哥哥,你好大胆,平白在此看他怎的?”这敬济扭回头看时,却是一个识熟朋友——铁指甲杨二郎。二人声喏。杨二道:“哥哥一向不见,那里去来?”敬济便把东京父死往回之事,告说一遍:“恰才这杀死妇人,是我丈人的小,潘氏。不知他被人杀了。适才见了榜文,方知其故。”杨二郎告道:“他是小叔武松,充配在外,遇赦回还,不知因甚杀了妇人,连王婆子也不饶。他家还有个女孩儿,在我姑夫姚二郎家养活了三四年。昨日他叔叔杀了人,走的不知下落。我姑夫将此女县中领出,嫁与人为妻小去了。见今这两个尸首,日久只顾埋着,只是苦了地方保甲看守,更不知何年月日才拿住凶犯武松。”说毕,杨二郎招了敬济,上酒楼饮酒:“与哥拂尘。”敬济见妇人已死,心中痛苦不了,那里吃得下酒。约莫饮勾三杯,就起身下楼,作别来家。

到晚夕,买了一陌钱纸,在紫石街离王婆门首远远的石桥边,叫着妇人:“潘六姐,我小兄弟陈敬济,今日替你烧陌钱纸。皆因我来迟了一步,误了你性命。你活时为人,死后为神,早佑佑捉获住仇人武松,替你报仇雪恨。我在法场上看着剐他,方趁我平生之志。”说毕哭泣,烧化了钱纸。敬济回家,闭了门户。走归房中,恰才睡着,似睡不睡,梦见金莲身穿素服,一身带血,向敬济哭道:“我的哥哥,我死的好苦也!实指望与你相处在一处,不期等你不来,被武松那厮害了性命。如今阴司不收,我白日游游荡荡,夜归各处寻讨浆水,适间蒙你送了一陌钱纸与我。但只是仇人未获,我的尸首埋在当街,你可念旧日之情,买具棺材盛了葬埋,免得日久暴露。”敬济哭道:“我的姐姐,我可知要葬埋你。但恐我丈母那无仁义的淫妇知道。他只恁赖我,倒趁了他机会。姐姐,你须往守备府中,对春梅说知,教他葬埋你身尸便了。”妇人道:“刚才奴到守备府中,又被那门神户尉拦挡不放,奴须慢慢再哀告他则个。”敬济哭着,还要拉着他说话,被他身上一阵血腥气,撇气挣脱,却是南柯一梦。枕上听那更鼓时,正打三更三点,说道:“怪哉!我刚才分明梦见六姐向我诉告衷肠,教我葬埋之意,又不知甚年何日拿着武松,是好伤感人也!”正是:

\[
梦中无限伤心事,独坐空房哭到明。
\]

按下一头。却表县中访拿武松,约两个月有余,捕获不着,已知逃遁梁山为盗。地方保甲邻佑呈报到官,所有两个尸首,相应责令家属领埋。王婆尸首,便有他儿子王潮领的埋葬。止有妇人身尸,无人来领。却说府中春梅,两三日一遍,使张胜、李安来县中打听。回去只说凶犯还未拿住,尸首照旧埋瘗,地方看守,无人敢动。直挨过年,正月初旬时节,忽一日晚间,春梅作一梦。恍恍惚惚,梦见金莲云髻蓬松,浑身是血,叫道:“庞大姐,我的好姐姐,奴死的好苦也!所有奴的尸首,在街暴露日久,风吹雨洒,鸡犬作践,无人领埋。奴举眼无亲,你若念旧日母子之情,买具棺木,把奴埋在一个去处,奴在阴司口眼皆闭。”说毕大哭不止。春梅扯住他,还要再问他别的话,被他挣开,撇手惊觉,却是南柯一梦。从睡梦中直哭醒来,心内犹疑不定。

次日叫进张胜、李安分付:“你二人去县中打听,那埋的妇人、婆子尸首还有也没有。”张胜、李安应诺去了。不多时,来回报:“正犯凶身已自逃走脱了。所有杀死身尸,地方看守,日久不便,相应责令各人家属领埋。那婆子尸首,他儿子招领的去了。那妇人无人来领,还埋在街心。”春梅道:“既然如此,我这桩事儿,累你二人替我干得来,我还重赏你。”二人跪下道:“小夫人说那里话,若肯在老爷前抬举小人一二,便消受不了。虽赴汤跳水,敢说不去?”春梅走到房中,拿出十两银子,两匹大布,委付二人道:“这死的妇人,是我一个嫡亲姐姐,嫁在西门庆家,今日出来,被人杀死。你二人休教你老爷知道,拿这银子替我买一具棺材,把他装殓了,抬出城外,择方便地方埋葬停当,我还重赏你。”二人道“这个不打紧,小人就去。”李安说:“只怕县中不教你我领尸怎了?须拿老爷个贴儿,下与县官才好。”张胜道:“只说小夫人是他妹子,嫁在府中,那县官不敢不依,何消贴子。”于是领了银子,来到班房内。张胜便向李安说:“想必这死的妇人,与小夫人曾在西门庆家做一处,相结的好,今日方这等为他费心。想着死了时,整哭了三四日,不吃饭,直教老爷门前叫了调百戏货郎儿,调与他观看,还不喜欢。今日他无亲人领去,小夫人岂肯不葬埋他?咱每若替他干得此事停当,早晚他在老爷跟前,只方便你我,就是一点福星。见今老爷百依百随,听他说话,正经大奶奶、二奶奶且打靠后。”说毕,二人拿银子到县前递了领状,就说他妹子在老爷府中,来领尸首。使了六两银子,合了一具棺材,把妇人尸首掘出,把心肝填在肚内,用线缝上,用布装殓停当,装入材内。张胜说:“就埋在老爷香火院永福寺里罢,那里有空闲地。”就叫了两名伴当,抬到永福寺,对长老说:“这是宅内小夫人的姐姐,要一块地儿葬埋。”长老不敢怠慢,就在寺后拣一块空心白杨树下那里葬埋。已毕,走来宅内回春梅话,说:“除买棺材装殓,还剩四两银子。”交割明白。春梅分付:“多有起动,你二人将这四两银子,拿二两与长老道坚,教他早晚替他念些经忏,超度他升天。”又拿出一大坛酒,一腿猪肉,一腿羊肉:“这二两银子,你每人将一两家中盘缠。”二人跪下,那里敢接?只说:“小夫人若肯在老爷面前抬举小人,消受不了。这些小劳,岂敢接受银两。”春梅道:“我赏你,不收,我就恼了。”二人只得磕头领了出来。两个班房吃酒,甚是称念小夫人好处。次日,张胜送银子与长老念经,春梅又与五钱银子买纸,与金莲烧,俱不在话下。

却说陈定从东京载灵柩家眷到清河县城外,把灵柩寄在永福寺,等念经发送,归葬坟内。敬济在家听见母亲张氏家小车辆到了,父亲灵柩寄停在城外永福寺,收卸行李已毕,与张氏磕了头。张氏怪他:“就不去接我一接。”敬济只说:“心中不好,家里无人看守。”张氏便问:“你舅舅怎的不见?”敬济道:“他见母亲到,连忙搬回家去了。”张氏道:“且教你舅舅住着,慌搬去怎的?”一面他母舅张团练来看姐姐。姊妹抱头而哭,置酒叙说,不必细说。

次日,张氏早使敬济拿五两银子、几陌金银钱纸,往门外与长老,替他父亲念经。正骑头口街上走,忽撞遇他两个朋友陆大郎、杨大郎,下头口声喏。二人问道:“哥哥那里去?”敬济悉言:“先父灵柩寄在门外寺里,明日二十日是终七,家母使我送银子与长老,做斋念经。”二人道:“兄弟不知老伯灵柩到了,有失吊问。”因问:“几时发引安葬?”敬济道:“也只在一二日之间,念经毕,入坟安葬。”说罢,二人举手作别。这敬济又叫住,因问杨大郎:“县前我丈人的小,那潘氏尸首怎不见?被甚人领的去了?”杨大郎便道:“半月前,地方因捉不着武松,禀了本县相公,令各家领去葬埋。王婆是他儿子领去。这妇人尸首,丢了三四日,被守备府中买了一口棺材,差人抬出城外永福寺去葬了。”敬济听了,就知是春梅在府中收葬了他尸首。因问二郎:“城外有几个永福寺?”二郎道:“南门外只有一个永福寺,是周秀老爷香火院,那里有几个永福寺来?”敬济听了,暗喜:“就是这个永福寺,也是缘法凑巧,喜得六姐亦葬在此处。”一面作别二人,打头口出城,径到永福寺中。见了长老,且不说念经之事,就先问长老道坚:“此处有守备府中新近葬的一个妇人,在那里?”长老道:“就在寺后白杨树下。说是宅内小夫人的姐姐。”这陈敬济且不参见他父亲灵柩,先拿钱祭物,至于金莲坟上,与他祭了,烧化钱纸,哭道:“我的六姐,你兄弟陈敬济来与你烧一陌纸钱,你好处安身,苦处用钱。”祭毕,然后才到方丈内他父亲灵柩跟前烧纸祭祀。递与长老经钱,教他二十日请八众禅僧,念断七经。长老接了经衬,备办斋供。敬济到家,回了张氏话。二十日都去寺中拈香,择吉发引,把父亲灵柩归到祖茔。安葬已毕,来家母子过日不题。

却表吴月娘,一日二月初旬,天气融和,孟玉楼、孙雪娥、西门大姐、小玉,出来大门首站立,观看来往车马,人烟热闹。忽见一簇男女,跟着个和尚,生的十分胖大,头顶三尊铜佛,身上构着数枝灯树,杏黄袈裟风兜袖,赤脚行来泥没踝。当时古人有几句,赞的这行脚僧好处:

\[
打坐参禅,讲经说法。铺眉苦眼,习成佛祖家风;赖教求食,立起法门规矩。白日里卖杖摇铃,黑夜间舞枪弄棒。有时门首磕光头,饿了街前打响嘴。空色色空,谁见众生离下土?去来来去,何曾接引到西方。
\]
那和尚见月娘众妇人在门首,便向前道了个问讯,说道:“在家老菩萨施主,既生在深宅大院,都是龙华一会上人。贫僧是五台山下来的,结化善缘,盖造十王功德,三宝佛殿。仰赖十方施主菩萨,广种福田,舍资才共成胜事,种来生功果。贫僧只是挑脚汉。”月娘听了他这般言语,便唤小玉往房中以一顶僧帽,一双僧鞋,一吊铜钱,一斗白米。原来月娘平昔好斋僧布施,常时发心做下僧帽、僧鞋,预备来施。这小玉取出来,月娘分付:“你叫那师父近前来,布施与他。”这小玉故做娇态,高声叫道:“那变驴的和尚,过不过来!俺奶奶布施与你这许多东西,还不磕头哩。”月娘便骂道:“怪堕业的小臭肉儿,一个僧家,是佛家弟子,你有要没紧,恁谤他怎的?不当家化化的,你这小淫妇儿,到明日不知堕多少罪业!”小玉笑道:“奶奶,这贼和尚,我叫他,他怎的把一双贼眼,眼上眼下打量我?”那和尚双手接了鞋帽钱来,打问讯说道:“多谢施主老菩萨布施。”小玉道:“这秃厮好无礼。这些人站着,只打两个问讯儿,就不与我打一个儿?”月娘道:“小肉儿,还恁说白道黑道。他一个佛家之子,你也消受不的他这个问讯。”小玉道:“奶奶,他是佛爷儿子,谁是佛爷女儿?”月娘道:“相这比丘尼姑僧,是佛的女儿。”小玉道:“譬若说,相薛姑子、王姑子、大师父,都是佛爷女儿,谁是佛爷女婿?”月娘忍不住笑,骂道:“这贼小淫妇儿,也学的油嘴滑舌,见见就说下道儿去了。”小玉道:“奶奶只骂我,本等这秃和尚贼眉竖眼的只看我。”孟玉楼道:“他看你,想必认得你,要度脱你去。”小玉道:“他若度我,我就去。”说着,众妇女笑了一回。月娘喝道:“你这小淫妇儿,专一毁僧谤佛。”那和尚得了布施,顶着三尊佛扬长而去了。小玉道:“奶奶还嗔我骂他,你看这贼秃,临去还看了我一眼才去了。”有诗单道月娘修善施僧好处:

\[
守寡看经岁月深,私邪空色久违心。
奴身好似天边月,不许浮云半点侵。
\]

月娘众人正在门首说话,忽见薛嫂儿提着花箱儿,从街上过来。见月娘众人道了万福。月娘问:“你往那里去来?怎的影迹儿也不来我这里走走?”薛嫂儿道:“不知我终日穷忙的是些甚么。这两日,大街上掌刑张二老爹家,与他儿子和北边徐公公家做亲,娶了他侄女儿,也是我和文嫂儿说的亲事。昨日三朝,摆大酒席,忙的连守备府里咱家小大姐那里叫我,也没去,不知怎么恼我哩。”月娘问道:“你如今往那里去?”薛嫂道:“我有桩事,敬来和你老人家说来。”月娘道:“你有话进来说。”一面让薛嫂儿到后边上房里坐下,吃了茶。薛嫂道:“你老人家还不知道,你陈亲家从去年在东京得病没了,亲家母叫了姐夫去,搬取老小灵柩。从正月来家,已是念经发送,坟上安葬毕。我听说你老人家这边知道,怎不去烧张纸儿,探望探望。”月娘道:“你不来说,俺怎得晓的,又无人打听。倒只知道潘家的吃他小叔儿杀了,和王婆子都埋在一处,却不知如今怎样了。”薛嫂儿道:“自古生有地儿死有处。五娘他老人家,不因那些事出去了,却不好来。平日不守本分,干出丑事来,出去了,若在咱家里,他小叔儿怎得杀了他?还是冤有头,债有主。倒还亏了咱家小大姐春梅,越不过娘儿们情场,差人买了口棺材,领了他尸首,葬埋了。不然只顾暴露着,又拿不着小叔子,谁去管他?”孙雪娥在旁说:“春梅在守备府中多少时儿,就这等大了?手里拿出银子,替他买棺材埋葬,那守备也不嗔,当他甚么人?”薛嫂道:“耶嚛,你还不知,守备好不喜他,每日只在他房里歇卧,说一句依十句,一娶了他,见他生的好模样儿,乖觉伶俐,就与他西厢房三间房住,拨了个使女伏侍他。老爷一连在他房里歇了三夜,替他裁四季衣服,上头。三日吃酒,赏了我一两银子,一匹段子。他大奶奶五十岁,双目不明,吃长斋,不管事。东厢孙二娘生了小姐,虽故当家,挝着个孩子。如今大小库房钥匙,倒都是他拿着,守备好不听他说话哩。且说银子,手里拿不出来?”几句说的月娘、雪娥都不言语。坐了一回,薛嫂起身。月娘分付:“你明日来,我这里备一张祭桌,一匹尺头,一分冥纸,你来送大姐与他公公烧纸去。”薛嫂儿道:“你老人家不去?”月娘道:“你只说我心中不好,改日望亲家去罢。”那薛嫂约定:“你教大姐收拾下等着我。饭罢时候我来。”月娘道:“你如今到那里去?守备府中不去也罢。”薛嫂道:“不去,就惹他怪死了。他使小伴当叫了我好几遍了。”月娘道:“他叫你做甚么?”薛嫂道:“奶奶,你不知。他如今有了四五个月身孕了,老爷好不喜欢,叫了我去,已定赏我。”提着花箱,作辞去了。雪娥便说:“老淫妇说的没个行款也!他卖与守备多少时,就有了半肚孩子,那守备身边少说也有几房头,莫就兴起他来,这等大道?”月娘道:“他还有正景大奶奶,房里还有一个生小姐的娘子儿哩。”雪娥道:“可又来!到底还是媒人嘴,一尺水十丈波的。”不因今日雪娥说话,正是:从天降下钩和线,就地引来是非来。有诗为证:

\[
曾记当年侍主旁,谁知今日变风光。
世间万事皆前定,莫笑浮生空自忙。
\]

\newpage
%# -*- coding:utf-8 -*-
%%%%%%%%%%%%%%%%%%%%%%%%%%%%%%%%%%%%%%%%%%%%%%%%%%%%%%%%%%%%%%%%%%%%%%%%%%%%%%%%%%%%%


\chapter{清明节寡妇上新坟\KG 永福寺夫人逢故主}


词曰:

\[
佳人命薄,叹艳代红粉,几多黄土。岂是老天浑不管,好恶随人自取?既赋娇容,又全慧性,却遣轻归去。不平如此,问天天更不语。
可惜国色天香,随时飞谢,埋没今如许。借问繁华何处在?多少楼台歌舞,紫陌春游,绿窗晚秀,姊妹娇眉妩。人生失意,从来无问今古。\named{右调《翠楼吟》}
\]

话说月娘次日备了一张桌,并冥纸尺头之类,大姐身穿孝服,坐轿子,先叫薛嫂押祭礼,到陈宅来。只见陈敬济正在门首站立,便问:“是那里的?”薛嫂道了万福,说:“姐夫,你休推不知。你丈母家来与你爹烧纸,送大姐来了。”敬济便道:“我鸡巴\textuni{34B2}的才是丈母!正月十六贴门神——来迟了半个月。人也入了土,才来上祭。”薛嫂道:“好姐夫,你丈母说,寡妇家没脚蟹,不知亲家灵柩来家,迟了一步,休怪。”正说着,只见大姐轿子落在门首。敬济问:“是谁?”薛嫂道:“再有谁?你丈母心内不好,一者送大姐来家,二者敬与你爹烧纸。”敬济骂道:“趁早把淫妇抬回去!好的死了万万千千,我要他做甚么?”薛嫂道:“常言道:嫁夫着主。怎的说这个话?”敬济道:“我不要这淫妇了,还不与我走?”那抬轿的只顾站立不动,被敬济向前踢了两脚,骂道:“还不与我抬了去,我把你花子脚砸折了,把淫妇鬓毛都蒿净了!”那抬轿子的见他踢起来,只得抬轿子往家中走不迭。比及薛嫂叫出他娘张氏来,轿子已抬去了。

薛嫂儿没奈何,教张氏收下祭礼,走来回覆吴月娘。把吴月娘气的一个发昏,说道:“恁个没天理的短命囚根子!当初你家为了官事,搬来丈人家居住,养活了这几年,今日反恩将仇报起来了。只恨死鬼当初揽的好货在家里,弄出事来,到今日教我做臭老鼠,教他这等放屁辣臊。”对着大姐说:“孩儿,你是眼见的,丈人、丈母那些儿亏了他来?你活是他家人,死是他家鬼,我家里也留以留你。你明日还去,休要怕他,料他挟你不到井里。他好胆子,恒是杀不了人,难道世间没王法管他也怎的!”当晚不题。

到次日,一顶轿子,教玳安儿跟随着,把大姐又送到陈敬济家来。不想陈敬济不在家,往坟上替他父亲添土叠山子去了。张氏知礼,把大姐留下,对着玳安说:“大官到家多多上覆亲家,多谢祭礼,休要和他一般见识。他昨日已有酒了,故此这般。等我慢慢说他。”一面管待玳安儿,安抚来家。

至晚,陈敬济坟上回来,看见了大姐,就行踢打,骂道:“淫妇,你又来做甚么?还说我在你家雌饭吃,你家收着俺许多箱笼,因起这大产业,不道的白养活了女婿!好的死了万千,我要你这淫妇做甚?”大姐亦骂:“没廉耻的囚根子!没天理的囚根子!淫妇出去吃人杀了,没的禁拿我煞气。”被敬济扯过头发,尽力打了几拳头。他娘走来解劝,把他娘推了一交。他娘叫骂哭喊,说:“好囚根子,红了眼,把我也不认的了!”到晚上,一顶轿子,把大姐又送将来,分付道:“不讨将寄放妆奁箱笼来家,我把你这淫妇活杀了。”这大姐害怕,躲在家中居住,再不敢去了。这正是:谁知好事多更变,一念翻成怨恨媒。这里不去。不题。

且说一日,三月清明佳节。吴月娘备办香烛、金钱冥纸、三牲祭物,抬了两大食盒,要往城外坟上与西门庆上新坟祭扫。留下孙雪娥和大姐、众丫头看家。带了孟玉楼和小玉,并奶子如意儿抱着孝哥儿,都坐轿子往坟上去。又请了吴大舅和大妗子二人同去。出了城门,只见那郊原野旷,景物芳菲,花红柳绿,仕女游人不断。一年四季,无过春天,最好景致。日谓之丽日,风谓之和风,吹柳眼,绽花心,拂香尘。天色暖,谓之暄。天色寒,谓之料峭。骑的马,谓之宝马。坐的轿,谓之香车。行的路,谓之芳径。地下飞的尘,谓之香尘。千花发蕊,万草生芽,谓之春信。韶光淡荡,淑景融和。小桃深妆脸妖娆,嫩柳袅宫腰细腻。百转黄鹂惊回午梦,数声紫燕说破春愁。日舒长暖澡鹅黄,水渺茫浮香鸭绿。隔水不知谁院落,秋千高挂绿杨烟。端的春景果然是好。有诗为证:

\[
清明何处不生烟,郊外微风挂纸钱。
人笑人歌芳草地,乍晴乍雨杏花天。
海棠枝上绵莺语,杨柳堤边醉客眠。
红粉佳人争画板,彩绳摇拽学飞仙。
\]

吴月娘等轿子到五里原坟上,玳安押着食盒,先到厨下生起火来,厨役落作整理不题。月娘与玉楼、小玉、奶子如意儿抱着孝哥儿,到于庄院客坐内坐下吃茶,等着吴大妗子,不见到。玳安向西门庆坟上祭台儿,摆设桌面三牲,羹饭祭物,列下纸钱,只等吴大妗子。原来大妗子雇不出轿子来,约已牌时分,才同吴大舅雇了两个驴儿骑将来。月娘便说:“大妗子雇不出轿子来,这驴儿怎的骑?”一面吃了茶,换了衣服,同来西门庆坟上祭扫。那月娘手拈着五根香,自拿一根,递一根与玉楼,又递一根与奶子如意儿替孝哥上,那两根递与吴大舅、大妗子。月娘插在香炉内,深深拜下去,说道:“我的哥哥,你活时为人,死后为神。今日三月清明佳节,你的孝妻吴氏三姐、孟三姐和你周岁孩童孝哥儿,敬来与你坟前烧一陌钱纸。你保佑他长命百岁,替你做坟前拜扫之人。我的哥哥,我和你做夫妻一场,想起你那模样儿并说的话来,是好伤感人也。”拜毕,掩面痛哭。玉楼向前插上香,也深深拜下,同月娘大哭了一场。玉楼上了香,奶子如意儿抱着哥儿也跪下上香,磕了头。吴大舅、大妗子都炷了香。行毕礼数,玳安把钱纸烧了。让到庄上卷棚内,放桌席摆饭,收拾饮酒。月娘让吴大舅、大妗子上坐。月娘与玉楼下陪。小玉和奶子如意儿,同大妗子家使的老姐兰花,也在两边打横列坐,把酒来斟。按下这里吃酒不题。

却表那日周守备府里也上坟。先是春梅隔夜和守备睡,假推做梦,睡梦中哭醒了。守备慌的问:“你怎的哭?”春梅便说:“我梦见我娘向我哭泣,说养我一场,怎地不与他清明寒食烧纸,因此哭醒了。”守备道:“这个也是养女一场,你的一点孝心。不知你娘坟在何处?”春梅道:“在南门外永福寺后面便是。”守备说:“不打紧,永福寺是我家香火院,明日咱家上坟,你叫伴当抬些祭物,往那里与你娘烧分纸钱,也是好处。”至次日,守备令家人收拾食盒酒果祭品,径往城南祖坟上。那里有大庄院、厅堂、花园、享堂、祭台。大奶奶、孙二娘并春梅,都坐四人轿,排军喝路,上坟耍子去了。

却说吴月娘和大舅、大妗子吃了回酒,恐怕晚来,分付玳安、来安儿收拾了食盒酒果,先往杏花村酒楼下,拣高阜去处,人烟热闹,那里设放桌席等候。又见大妗子没轿子,都把轿子抬着,后面跟随不坐,领定一簇男女,吴大舅牵着驴儿,压后同行,踏青游玩。三月桃花店,五里杏花村,只见那随路上坟游玩的王孙士女,花红柳绿,闹闹喧喧,不知有多少。正走之间,也是合当有事,远远望见绿槐影里,一座庵院,盖造得十分齐整。但见:

\[
山门高耸,梵宇清幽。当头敕额字分明,两下金刚形势猛。五间大殿,龙鳞瓦砌碧成行;两下僧房,龟背磨砖花嵌缝。前殿塑风调雨顺,后殿供过去未来。钟鼓楼森立,藏经阁巍峨。旗竿高峻接青云,宝塔依稀侵碧汉。木鱼横挂,云板高悬。佛前灯烛莹煌,炉内香烟缭绕。幢旗不断,观音殿接祖师堂;宝盖相连,鬼母位通罗汉殿。时时护法诸天降,岁岁降魔尊者来。
\]
吴月娘便问:“这座寺叫做甚么寺?”吴大舅便说:“此是周秀老爷香火院,名唤永福禅林。前日姐夫在日,曾舍几拾两银子在这寺中,重修佛殿,方是这般新鲜。”月娘向大妗子说:“咱也到这寺里看一看。”于是领着一簇男女,进入寺中来。不一时,小沙弥看见,报与长老知道:“见有许多男女……”便出方丈来迎请,见了吴大舅、吴月娘,向前合掌道了问讯,连忙唤小和尚开了佛殿:“请施主菩萨随喜游玩,小僧看茶。”那小沙弥开了殿门,领月娘一簇男女,前后两廊参拜观看了一回,然后到长老方丈。长老连忙点上茶来,吴大舅请问长老道号,那和尚答说:“小僧法名道坚。这寺是恩主帅府周爷香火院,小僧忝在本寺长老,廊下管百十众僧行,后边禅堂中还有许多云游僧行,常时坐禅,与四方檀越答报功德。”一面方丈中摆斋,让月娘:“众菩萨请坐。”月娘道:“不当打搅长老宝刹。”一面拿出五钱银子,教大舅递与长老,佛前请香烧。那和尚打问讯谢了,说道:“小僧无甚管待,施主菩萨稍坐,略备一茶而已,何劳费心赐与布施。”不一时,小和尚放下桌儿,拿上素菜斋食饼馓上来。那和尚在旁陪坐,才举箸儿让众人吃时,忽见两个青衣汉子,走的气喘吁吁,暴雷也一般报与长老,说道:“长老还不快出来迎接,府中小奶奶来祭祀来了!”慌的长老披袈裟,戴僧帽不迭,分付小沙弥连忙收了家活,“请列位菩萨且在小房避避,打发小夫人烧了纸,祭毕去了,再款坐一会不迟。”吴大舅告辞,和尚死活留住,又不肯放。

那和尚慌的鸣起钟鼓来,出山门迎接,远远在马道口上等候。只见一族青衣人,围着一乘大轿,从东云飞般来,轿夫走的个个汗流满面,衣衫皆湿。那长老躬身合掌说道:“小僧不知小奶奶前来,理合远接,接待迟了,万勿见罪。”这春梅在轿内答道:“起动长老。”那手下伴当,又早向寺后金莲坟上,忙将祭桌纸钱来摆设下。春梅轿子来到,也不到寺,径入寺后白杨树下金莲坟前下轿。两边青衣人伺候。这春梅不慌不忙,来到坟前,摆了香,拜了四拜,说道:“我的娘,今日庞大姐特来与你烧陌纸钱,你好处升天,苦处用钱。早知你死在仇人之手,奴随问怎的也娶来府中,和奴做一处。还是奴耽误了你,悔已是迟了。”说毕,令左右把钱纸烧了。这春梅向前放声大哭不已。

吴月娘在僧房内,只知有宅内小夫人来到,长老出山门迎接,又不见进来。问小和尚,小和尚说:“这寺后有小奶奶的一个姐姐,新近葬下,今日清明节,特来祭扫烧纸。”孟玉楼便道:“怕不就是春梅来了?也不见的。”月娘道:“他那得个姐来死了葬在此处?”又问小和尚:“这府里小夫人姓甚么?”小和尚道:“姓庞,前日与了长老四五两经钱,教替他姐姐念经,荐拔生天。”玉楼道:“我听见他爹说春梅娘家姓庞,叫庞大姐,莫不是他?”正说话,只见长老先来,分付小沙弥:“好看好茶。”不一时,轿子抬进方丈二门里才下。月娘和玉楼众人打僧房帘内望外张看,怎样的小夫人。定睛仔细看时,却是春梅。但比昔时出落得长大身材,面如满月,打扮的粉妆玉琢,头上戴着冠儿,珠翠堆满,凤钗半卸,上穿大红妆花袄,下着翠兰缕金宽斓裙子,带着丁当禁步,比昔不同许多。但见:

\[
宝髻巍峨,凤钗半卸。胡珠环耳边低挂,金挑凤鬓后双拖。红绣袄偏衬玉香肌,翠纹裙下映金莲小。行动处,胸前摇响玉丁当;坐下时,一阵麝兰香喷鼻。腻粉妆成脖颈,花钿巧帖眉尖。举止惊人,貌比幽花殊丽;姿容闲雅,性如兰蕙温柔。若非绮阁生成,定是兰房长就。俨若紫府琼姬离碧汉,宛如蕊宫仙子下尘寰。
\]
那长老上面独独安放一张公座椅儿,让春梅坐下。长老参见已毕,小沙弥拿上茶来。长老递茶上去,说道:“今日小僧不知小奶奶来这里祭祀,有失迎接,万望恕罪。”春梅道:“外日多有起动长老诵经追荐。”那和尚说:“小僧岂敢。有甚殷勤补报恩主?多蒙小奶奶赐了许多钱衬施。小僧请了八众禅僧,整做道场,看经礼忏一日。晚夕,又与他老人家装些厢库焚化。道场圆满,才打发两位管家进城,宅里回小奶奶话。”春梅吃了茶,小和尚接下钟盏来。长老只顾在旁一递一句与春梅说话,把吴月娘众人拦阻在内,又不好出来的。

月娘恐怕天晚,使小和尚请下长老来,要起身。那长老又不肯放,走来方丈禀春梅说:“小僧有件事禀知小奶奶。”春梅道:“长老有话,但说无妨。”长老道:“适间有几位游玩娘子,在寺中随喜,不知小奶奶来。如今他要回去,未知小奶奶尊意如何。”春梅道:“长老何不请来相见。”那长老慌的来请。吴月娘又不肯出来,只说:“长老不见罢。天色晚了,俺们告辞去了。”长老见收了他布施,又没管待,又意不过,只顾再三催促。吴月娘与孟玉楼、吴大妗子推阻不过,只得出来,春梅一见便道:“原来是二位娘与大妗子。”于是先让大妗子转上,花枝招展磕下头去。慌的大妗子还礼不迭,说道:“姐姐,今非昔比,折杀老身。”春梅道:“好大妗子,如何说这话,奴不是那样人。尊卑上下,自然之礼。”拜了大妗子,然后向月娘、孟玉楼插烛也似磕头。月娘、玉楼亦欲还礼,春梅那里肯,扶起,磕下四个头,说:“不知是娘们在这里,早知也请出来相见。”月娘道:“姐姐,你自从出了家门在府中,一向奴多缺礼,没曾看你,你休怪。”春梅道:“好奶奶,奴那里出身,岂敢说怪。”因见奶子如意儿抱着孝哥儿,说道:“哥哥也长的恁大了。”月娘说:“你和小玉过来,与姐姐磕过头儿。”那如意儿和小玉二人笑嘻嘻过来,亦与春梅都平磕了头。月娘道:“姐姐,你受他两个一礼儿。”春梅向头上拔下一对金头银簪儿来,插在孝哥儿帽儿上。月娘说:“多谢姐姐簪儿,还不与姐姐唱个喏儿。”如意儿抱着哥儿,真个与春梅唱个喏,把月娘喜欢的要不得。玉楼道:“姐姐,你今日不到寺中,咱娘儿们怎得遇在一处相见。”春梅道:“便是因俺娘他老人家新埋葬在这寺后,奴在他手里一场,他又无亲无故,奴不记挂着替他烧张纸儿,怎生过得去。”月娘道:“我记的你娘没了好几年,不知葬在这里。”孟玉楼道:“大娘还不知庞大姐说话,说的是潘六姐死了。多亏姐姐,如今把他埋在这里。”月娘听了,就不言语了。吴大妗子道:“谁似姐姐这等有恩,不肯忘旧,还葬埋了。你逢节令题念他,来替他烧钱化纸。”春梅道:“好奶奶,想着他怎生抬举我来!今日他死的苦,这般抛露丢下,怎不埋葬他?”说毕,长老教小和尚放桌儿,摆斋上来。两张大八仙桌子,蒸酥点心,各样素馔菜蔬,堆满春台,绝细春芽雀舌甜水好茶。众人吃了,收下家活去。吴大舅自有僧房管待,不在话下。

孟玉楼起身,心里要往金莲坟上看看,替他烧张纸,也是姊妹一场。见月娘不动身,拿出五分银子,教小沙弥买纸去。长老道:“娘子不消买去,我这里有金银纸,拿几分烧去。”玉楼把银子递与长老,使小沙弥领到后边白杨树下金莲坟上,见三尺坟堆,一堆黄土,数柳青蒿。上了根香,把纸钱点着,拜了一拜,说道:“六姐,不知你埋在这里。今日孟三姐误到寺中,与你烧陌钱纸,你好处升天,苦处用钱。”一面放声大哭。那奶子如意儿见玉楼往后边,也抱了孝哥儿来看一看。月娘在方丈内和春梅说话,教奶子休抱了孩子去,只怕唬了他。如意儿道:“奶奶,不妨事,我知道。”径抱到坟上,看玉楼烧纸哭罢回来。

春梅和月娘匀了脸,换了衣裳,分付小伴当将食盒打开,将各样细果甜食,肴品点心攒盒,摆下两桌子,布甑内筛上酒来,银钟牙箸,请大妗子、月娘、玉楼上坐,他便主位相陪。奶子、小玉,都在两边打横。吴大舅另放一张桌子在僧房内。正饮酒中间,忽见两个青衣伴当走来,跪下禀道:“老爷在新庄,差小的来请小奶奶看杂耍调百戏的。大奶奶、二奶奶都去了,请奶奶快去哩。”这春梅不慌不忙,说:“你回去,知道了。”那二人应诺下来,又不敢去,在下边等候。大妗子、月娘便要起身,说:“姐姐,不可打搅。天色晚了,你也有事,俺们去罢。”那春梅那里肯放,只顾令左右将大钟来劝道:“咱娘儿们会少离多,彼此都见长着,休要断了这门亲路。奴也没亲没故,到明日娘的好日子,奴往家里走走去。”月娘道:“我的姐姐,说一声儿就勾了,怎敢起动你?容一日,奴去看姐姐去。”饮过一杯,月娘说:“我酒勾了,你大妗子没轿子,十分晚了,不好行的。”春梅道:“大妗子没轿子,我这里有跟随小马儿,拨一匹与妗子骑,关了家去。”大妗子再三不肯,辞了,方一面收拾起身。春梅叫过长老来,令小伴当拿出一匹大布、五钱银子与长老。长老拜谢了,送出山门。春梅与月娘拜别,看着月娘、玉楼众人上了轿子,他也坐轿子,两下分路,一簇人明随喝道,往新庄上去了。正是:

\[
树叶还有相逢时,岂可人无得运时。
\]

\newpage
%# -*- coding:utf-8 -*-
%%%%%%%%%%%%%%%%%%%%%%%%%%%%%%%%%%%%%%%%%%%%%%%%%%%%%%%%%%%%%%%%%%%%%%%%%%%%%%%%%%%%%


\chapter{来旺偷拐孙雪娥\KG 雪娥受辱守备府}


诗曰:

\[
菟丝附蓬麻,引蔓原不长。失身与狂夫,不如弃道旁。
暮夜为侬好,席不暖侬床。昏来晨一别,无乃太匆忙。
行将滨死地,老痛迫中肠。
\]

话说吴大舅领着月娘等一簇男女,离了永福寺,顺着大树长堤前来。玳安又早在杏花酒楼下边,人烟热闹,拣高阜去处,幕天席地设下酒肴,等候多时了。远远望月娘众人轿子驴子到了,问道:“如何这咱才来?”月娘又把永福寺中遇见春梅告诉一遍。不一时斟上酒来。众人坐下正饮酒,只见楼下香车绣毂往来,人烟喧杂。月娘众人骊着高阜,把眼观看,只见人山人海围着,都看教师走马耍解。

原来是本县知县相公儿子李衙内,名唤李拱璧,年约三十余岁,见为国子上舍,一生风流博浪,懒习诗书,专好鹰犬走马,打球蹴踘,常在三瓦两巷中走,人称他为“李棍子”。那日穿着一弄儿轻罗软滑衣裳,头戴金顶缠棕小帽,脚踏乾黄靴,同廊吏何不韦带领二三十好汉,拿弹弓、吹筒、球棒在于杏花村大酒楼下,看教师李贵走马卖解,竖肩桩、隔肚带,轮枪舞棒,做各样技艺顽耍,引了许多男女围着烘笑。那李贵诨名为山东夜叉,头带万字巾,身穿紫窄衫,销金裹肚,坐下银鬃马,手执朱红杆明枪,背插招风令字旗,在街心扳鞍上马,往来卖弄手段。这李衙内正看处,忽抬头看见一簇妇人在高阜处饮酒,内中一个长挑身材妇人,不觉心摇目荡,观之不足,看之有余,口中不言,心内暗道:“不知是谁家妇女,有男子汉没有?”一面叫过手下答应的小张闲架儿来,悄悄分付:“你去那高坡上,打听那三个穿白的妇人是谁家的。访得的实,告我知道。”那小张闲应诺,云飞跑去。不多时,走到跟前附耳低言回报说:“如此这般,是县门前西门庆家妻小。一个年老的姓吴,是他妗子;一个五短身材,是他大娘子吴月娘;那个长挑身材,有白麻子的,是第三个娘子,姓孟,名玉楼;如今都守寡在家。”这李衙内听了,独看上孟玉楼,重赏小张闲,不在话下。

吴月娘和大舅众人观看了半日,见日色衔山,令玳安收拾了食盒,上轿骑驴一径回家。有诗为证:

\[
柳底花阴压路尘,一回游赏一回新。
有缘千里来相会,无缘对面不相亲。
\]

这里月娘众人回家不题。却说那日,孙雪娥与西门大姐在家,午后时分无事,都出大门首站立。也是天假其便,不想一个摇惊闺的过来。那时卖脂粉、花翠生活,磨镜子,都摇惊闺。大姐说:“我镜子昏了。”使平安儿:“叫住那人,与我磨镜子。”那人放下担儿,说道:“我不会磨镜子,我只卖些金银生活,首饰花翠。”站立在门前,只顾眼上眼下看着雪娥。雪娥便道:“那汉子,你不会磨镜子,去罢,只顾看我怎的!”那人说:“雪姑娘,大姑娘,不认的我了?”大姐道:“眼熟,急忙想不起来。”那人道:“我是爹手里出去的来旺儿。”雪娥便道:“你这几年在那里来?出落得恁胖了。”来旺儿道:“我离了爹门,到原籍徐州,家里闲着没营生,投跟了老爹上京来做官。不想到半路里,他老爷儿死了,丁忧家去了。我便投在城内顾银铺,学会了此银行手艺,各样生活。这两日行市迟,顾银铺教我挑副担儿,出来街上发卖些零碎。看见娘每在门首,不敢来相认,恐怕踅门瞭户的。今日不是你老人家叫住,还不敢相认。”雪娥道:“原来是你。教我只顾认了半日,白想不起。既是旧儿女,怕怎的?”因问:“你担儿里卖的是甚么生活?挑进里面,等俺每看一看。”那来旺儿一面把担儿挑入里边院子里来。打开箱子,用箧儿托出几样首饰来:金银镶嵌不等,打造得十分奇巧。大姐与雪娥看了一回,问来旺儿:“你还有花翠,拿出来。”这孙雪娥便留了他一对翠凤,一对柳穿金鱼儿。大姐便称出银子来与他。雪娥两样生活,欠他一两二钱银子,约下他:“明日早来取罢。今日你大娘不在家,和你三娘和哥儿都往坟上与你爹烧纸去了。”来旺道:“我去年在家里,就听见人说爹死了。大娘生了哥儿,怕不的好大了。”雪娥道:“你大娘孩儿如今才周半儿。一家儿大大小小,如宝上珠一般,全看他过日子哩。”说话中间,来昭妻一丈青出来,倾了盏茶与他吃,那来旺儿接了茶,与他唱了个喏。来旺也在跟前,同叙了回话。分付:“你明日来见见大娘。”那来旺儿挑担出门。

到晚上,月娘众人轿子来家。雪娥、大姐、众人丫头接着,都磕了头。玳安跟盒担走不上,雇了匹驴儿骑来家,打发抬盒人去了。月娘告诉雪娥、大姐,说今日寺里遇见春梅一节:“原来他把潘家的就葬在寺后首,俺每也不知。他来替他娘烧纸,误打误撞遇见他。娘儿每又认了回亲。先是寺里长老摆斋吃了。落后他又教伴当摆上他家的四五十攒盒,各样菜蔬下饭,筛酒上来,通吃不了。他看见哥儿,又与了他一对簪儿,好不和气。起解行三坐五,坐着大轿子,许多跟随。又且是出落的比旧时长大了好些,越发白胖了。”吴大妗子道:“他倒也不改常忘旧。那时在咱家时,我见他比众丫鬟行事儿正大,说话儿沉稳,就是个才料儿。你看今日福至心灵,恁般造化。”孟玉楼道:“姐姐没问他,我问他来。果然半年没洗换,身上怀着喜事哩。也只是八九月里孩子,守备好不喜欢哩。薛嫂儿说的倒不差。”说了一回,雪娥题起:“今日娘不在,我和大姐在门首,看见来旺儿。原来他又在这里学会了银匠,挑着担儿卖金银生活花翠。俺每就不认得了,买了他几枝花翠,他问娘来,我说往坟上烧纸去了。”月娘道:“你怎的不教他等着我来家?”雪娥道:“俺每教他明日来。”

正坐着说话,只见奶子如意儿向前对月娘说:“哥儿来家这半日,只是昏睡不醒,口中出冷气,身上汤烧火热的。”这月娘听见慌了,向炕上抱起孩儿来,口揾着口儿,果然出冷汗,浑身发热,骂如意儿:“好淫妇,此是轿子冷了孩儿了。”如意儿道:“我拿小被儿裹的紧紧的,怎得冻着?”月娘道:“再不是抱了往那死鬼坟上,唬了他来了。那等分付教你休抱他去,你不依,浪着抱的去了。”如意儿道:“早小玉姐姐看着,只抱了他那里看看就来了,几时唬着他来!”月娘道:“别要说嘴,看那看儿便怎的?却把他唬了。”急忙叫来安儿:“快请刘婆子去。”不一时,刘婆来到。看了脉息,摸了身上,说:“着了些凉寒,撞见邪祟了。”留了两服朱砂丸,用姜汤灌下去。分付奶子抱着他,热炕上睡到半夜,出了些冷汗,身上才凉了。于是管待刘婆子吃了茶,与了他三钱银子,叫他明日还来看看。一家子慌的要不的,起起倒倒,整乱了半夜。

却说来旺,次日依旧挑将生活担儿,来到西门庆门首,与来昭唱喏,说:“昨日雪姑娘留下我些生活,许下今日教我来取银子,就见见大娘。”来昭道:“你且去着,改日来。昨日大娘来家,哥儿不好,叫医婆、太医看,下药,整乱了一夜,好不心,今日才好些,那得工夫称银子与你。”正说着,只见月娘、玉楼、雪娥送出刘婆子,来到大门首,看见来旺儿。那来旺儿扒在地下,与月娘、玉楼磕下两个头。月娘道:“几时不见你,就不来这里走走。”来旺儿悉将前事说了一遍,“要来不好来的。”月娘道:“旧儿女人家,怕怎的?你爹又没了。当初只因潘家那淫妇,一头放火,一头放水,架的舌,把个好媳妇儿生生逼勒的吊死了,将有作没,把你垫发了去。今日天也不容,他往那去了!”来旺儿道:“也说不的,只是娘心里明白就是了。”说了回话,月娘问他:“卖的是甚样生活?拿出来瞧。”拣了他几件首饰,该还他三两二钱银子,都用等子称了与他。叫他进入仪门里面,分付小玉取一壶酒来,又是一盘点心,教他吃。那雪娥在厨上一力撺掇,又热了一大碗肉出来与他。吃的酒饭饱了,磕头出门。月娘、玉楼众人归到后边去。雪娥独自悄悄和他说话:“你常常来走着,怕怎的!奴有话教来昭嫂子对你说。我明日晚夕,在此仪门里紫墙儿跟前耳房内等你。”两个递了眼色,这来旺儿就知其意,说:“这仪门晚夕关不关?”雪娥道:“如此这般,你来先到来昭屋里,等到晚夕,踩着梯凳,越过墙,顺着遮墙,我这边接你下来。咱二人会合一回,还有细话与你说。”这来旺得了此话,正是欢从额起,喜向腮生,作辞雪娥,挑担儿出门。正是:不着家神,弄不得家鬼。有诗为证:

\[
闲来无事倚门阑,偶遇多情旧日缘。
对人不敢高声语,故把秋波送几番。
\]

这来旺儿欢喜来家,一宿无话。到次日,也不挑担儿出来卖生活,慢慢踅来西门庆门首,等来昭出来与他唱喏。那来昭便说:“旺哥稀罕,好些时不见你了。”来旺儿笑道:“不是也不来,里边雪姑娘少我几钱生活银,讨讨。”来昭一面把来旺儿让到房里坐下。来旺儿道:“嫂子怎不见?”来昭道:“你嫂子今日后边上灶哩。”那来旺儿拿出一两银子,递与来昭,说:“这银子取壶酒来,和哥嫂吃。”来昭道:“何消这许多。”即叫他儿子铁棍儿过来。那铁棍吊起头去——十五岁了,拿壶出来,打了一大注酒,使他后边叫一丈青来。不一时,一丈青盖了一锡锅热饭,一大碗杂熬下饭,两碟菜蔬,说道:“好呀,旺官儿在这里。”来昭便拿出银子与一丈青瞧,说:“兄弟破费,要打壶酒咱两口儿吃。”一丈青笑道:“无功消受,怎生使得?”一面放了炕桌,让来旺炕上坐。摆下酒菜,把酒来斟。来旺儿先倾头一盏,递与来昭,次递一盏与一丈青,深深唱喏,说:“一向不见哥嫂,这盏水酒孝顺哥嫂。”一丈青便说:“哥嫂不道酒肉吃伤了!你对真人休说假话。里边雪姑娘昨日已央及达知我了,你两个旧情不断,托俺每两口儿如此这般周全你。你休推睡里梦里,要知山下路,须问过来人。你若入港相会,有东西出来,休要独吃,须把些汁水教我呷一呷,俺替你每须耽许多利害。”那来旺便跪下说:“只望哥嫂周全,并不敢有忘。”说毕,把酒吃了一回。一丈青往后边和雪娥答了话出来,对他说,约定晚上来,来昭屋里窝藏,待夜里关上仪门,后边人歇下,越墙而过,于中取事。有诗为证:

\[
报应本无私,影响皆相似。
要知祸福因,但看所为事。
\]

这来旺得了此言,回来家,巴不到晚,踅到来昭屋里,打酒和他两口儿吃。至更深时分,更无一人觉的,直待的大门关了,后边仪门上了拴,家中大小歇息定了,彼此都有个暗号儿,只听墙内雪娥咳嗽之声。这来旺儿踏着梯凳,黑暗中扒过粉墙,雪娥那边用凳子接着。两个就在西耳房堆马鞍子去处,两个相搂相抱,云雨做一处。彼此都是旷夫寡妇,欲心如火。那来旺儿缨枪强壮,尽力弄了一回,乐极精来,一泄如注。干毕,雪娥递与他一包金银首饰,几两碎银子,两件段子衣服,分付:“明日晚夕你再来,我还有些细软与你。你外边寻下安身去处。往后这家中过不出好来,不如和你悄悄出去,外边寻下房儿,成其夫妇。你又会银行手艺,愁过不得日子?”来旺儿便说:“如今东门外细米巷,有我个姨娘,有名收生的屈老娘。你那里曲弯小巷,倒避眼,咱两个投奔那里去。迟些时,看无动静,我带你往原籍家里,买几亩地种去也好。”两个商量已定。这来旺就作别雪娥,依旧扒过墙来,到来昭屋里。等至天明,开了大门,挨身出去。到黄昏时分,又来门首,踅入来昭屋里。晚夕依旧跳过墙去,两个干事。朝来暮往,非止一日,也抵盗了许多细软东西,金银器皿,衣服之类。来昭两口子也得抽分好些肥己,俱不必细说。

一日,后边月娘看孝哥儿出花儿,心中不快,睡得早。这雪娥房中使女中秋儿,原是大姐使的,因李娇儿房中元宵儿被敬济要了,月娘就把中秋儿与了雪娥,把元宵儿伏侍大姐。那一日,雪娥打发中秋儿睡下,房里打点一大包钗环头面,装在一个匣内,用手帕盖了头,随身衣服,约定来旺儿在来昭屋里等候,两个要走。来昭便说:“不争你走了,我看守大门,管放水鸭儿!若大娘知道,问我要人怎的?不如你每打房上去,就骊破些瓦,还有踪迹。”来旺儿道:“哥也说得是。”雪娥又留一个银折盂,一根金耳斡,一件青绫袄,一条黄绫裙,谢了他两口儿。直等五更鼓,月黑之时,隔房扒过去。来昭夫妇又筛上两大钟暖酒,与来旺、雪娥吃,说:“吃了好走,路上壮胆些。”吃到五更时分,每人拿着一根香,骊着梯子,打发两个扒上房去,一步一步把房上瓦也跳破许多。比及扒到房檐跟前,街上人还未行走,听巡捕的声音,这来旺儿先跳下去,后却教雪娥骊着他肩背,接搂下来。两个往前边走,到十字路口上,被巡捕的拦住,便问:“往那里去的男女?”雪娥便唬慌了手脚。这来旺儿不慌不忙,把手中官香弹了一弹,说道:“俺是夫妇二人,前往城外岳庙里烧香,起的早了些,长官勿怪。”那人问:“背的包袱内是甚么?”来旺儿道:“是香烛纸马。”那人道:“既是两口儿岳庙烧香,也是好事,你快去罢。”这来旺儿得不的一声,拉着雪娥,往前飞走。走到城下,城门才开。打人闹里挨出城去,转了几条街巷。

原来细米巷在个僻静去处,住着不多几家人家,都是矮房低厦。到于屈姥姥家,屈姥姥还未开门。叫了半日,屈姥姥才起来开了门,见来旺儿领了个妇人来。原来来旺儿本姓郑,名唤郑旺,说:“这妇人是我新寻的妻小。姨娘这里有房子,且借一间,寄住些时,再寻房子。”递与屈姥姥三两银子,教买柴米。那屈姥姥得了银子,只得留下。他儿子屈铛,因见郑旺夫妻二人,带着许多金银首饰东西,夜晚见财起意,就掘开房门偷盗出来去耍钱,致被捉获,具了事件,拿去本县见官。李知县见系贼赃之事,赃物见在,即差人押着屈铛到家,把郑旺、孙雪娥一条索子都拴了。那雪娥唬的脸蜡黄也似黄了,换了渗淡衣裳,带着眼纱,把手上戒指都勒下来打发了公人,押去见官。当下烘动了一街人观看,有认得的,说是西门庆家小老婆,今被这走出的小厮来旺儿——改名郑旺通奸,拐盗财物在外居住。又被这屈铛掏摸了,今事发见官。当下一个传十个,十个传百个,路上行人口似飞。

月娘家中自从雪娥走了,房中中秋儿见箱内细软首饰都没了,衣服丢的乱三搅四,报与月娘。月娘吃了一惊,便问中秋儿:“你跟着他睡,走了,你岂不知?”中秋儿便说:“他要便晚夕悄悄偷走出外边,半日方回,不知详细。”月娘又问来昭:“你看守大门,人出去你怎不晓的?”来昭便说:“大门每日上锁,莫不他飞出去!”落后看见房上瓦骊破许多,方知越房而去了。又不敢使人骊访,只得按纳含忍。不想本县知县当堂理问这件事,先把屈铛夹了一顿,追出金头面四件,银首饰三件,金环一双,银钟二个,碎银五两,衣服二件,手帕一个,匣一个。向郑旺名下追出银三十两,金碗簪一对,金仙子一件,戒指四个。向雪娥名下追出金挑心一件,银镯一付,金钮五付,银簪四对,碎银一包。屈姥姥名下追出银三两。就将来旺儿问拟奴婢因奸盗取财物,屈铛系窃盗,俱系杂犯死罪,准徒五年,赃物入官。雪娥孙氏系西门庆妾,与屈姥姥当下都当官拶了一拶。屈姥姥供明放了。雪娥责令本县差人到西门庆家,教人递领状领孙氏。那吴月娘叫吴大舅来商议:“已是出丑,平白又领了来家做甚么?没的玷污了家门,与死的装幌子。”打发了差人钱,回了知县话。知县拘将官媒人来,当官辩卖。

却说守备府中,春梅打听得知,说西门庆家中孙雪娥如此这般,被来旺儿拐出,盗了财物去在外居住,事发到官,如今当官辨卖。这春梅听见,要买他来家上灶,要打他嘴,以报平昔之仇。对守备说:“雪娥善能上灶,会做的好茶饭汤水,买来家中伏侍。”这守备即差张胜、李安。拿贴儿对知县说。知县自恁要做分上,只要八两银子官价。交完银子,领到府中,先见了大奶奶并二奶奶孙氏,次后到房中来见春梅。春梅正在房里缕金床上,锦帐之中,才起来。手下丫鬟领雪娥见面。那雪娥见是春梅,不免低头进见。望上倒身下拜,磕了四个头。这春梅把眼瞪一瞪,唤将当直的家人媳妇上来,“与我把这贱人撮去了\textuni{4BFC}髻,剥了上盖衣裳,打入厨下,与我烧火做饭。”这雪娥听了,暗暗叫苦。自古世间打墙板儿翻上下,扫米却做管仓人。既在他檐下,怎敢不低头?孙雪娥到此地步,只得摘了髻儿,换了艳服,满脸悲恸,往厨下去了。有诗为证:

\[
布袋和尚到明州,策杖芒鞋任处游。
饶你化身千百亿,一身还有一身愁。
\]

\newpage
%# -*- coding:utf-8 -*-
%%%%%%%%%%%%%%%%%%%%%%%%%%%%%%%%%%%%%%%%%%%%%%%%%%%%%%%%%%%%%%%%%%%%%%%%%%%%%%%%%%%%%


\chapter{孟玉楼爱嫁李衙内\KG 李衙内怒打玉簪儿}


诗曰:

\[
簟展湘纹浪欲生,幽怀自感梦难成。
倚床剩觉添风味,开户羞将待月明。
拟倩蜂媒传密意,难将萤火照离情。
遥怜织女佳期近,时看银河几曲横。
\]

话说一日,陈敬济听见薛嫂儿说知孙雪娥之事。这陈敬济乘着这个根由,就如此这般,使薛嫂儿往西门庆家对月娘说。薛嫂只得见月娘,说:“陈姑夫在外声言发话,说不要大姐,要写状子,巡抚、巡按处告示,说老爹在日,收着他父亲寄放的许多金银箱笼细软之物。”这月娘一来因孙雪娥被来旺儿盗财拐去,二者又是来安儿小厮走了,三者家人来兴媳妇惠秀又死了,刚打发出去,家中正七事八事,听见薛嫂儿来说此话,唬的慌了手脚,连忙雇轿子,打发大姐家去。但是大姐床奁箱厨陪嫁之物,交玳安雇人,都抬送到陈敬济家。敬济说:“这是他随身嫁我的床帐妆奁,还有我家寄放的细软金银箱笼,须索还我。”薛嫂道:“你大丈母说来,当初丈人在时,止收下这个床奁嫁妆,并没见你别的箱笼。”敬济又要使女元宵儿。薛嫂儿和玳安儿来对月娘说。月娘不肯把元宵与他,说:“这丫头是李娇儿房中使的,如今留着晚早看哥儿哩。”把中秋儿打发将来,说:“原是买了伏侍大姐的。”这敬济又不要中秋儿,两头来回只教薛嫂儿走。他娘张氏向玳安说:“哥哥,你到家拜上你大娘,你家姐儿们多,也不稀罕这个使女看守哥儿。既是与了大姐房里好一向,你姐夫已是收用过了他,你大娘只顾留怎的?”玳安一面到家,把此话对月娘说了。月娘无言可对,只得把元宵儿打发将来。敬济收下,满心欢喜,说道:“可怎的也打我这条道儿来?”正是:

\[
饶你奸似鬼,吃我洗脚水。
\]

按下一头。单说李知县儿子李衙内,自从清明郊外看见吴月娘、孟玉楼两人一般打扮,生的俱有姿色,知是西门庆妻小。衙内有心,爱孟玉楼生的长挑身材,瓜子面皮,模样儿风流俏丽。原来衙内丧偶,鳏居已久,一向着媒妇各处求亲,都不遂意。及见玉楼,便觉动心,但无门可入,未知嫁与不嫁,从违如何。不期雪娥缘事在官,已知是西门庆家出来的,周旋委曲,在伊父案前,将各犯用刑研审,追出赃物数目,望其来领。月娘害怕,又不使人见官。衙内失望,因此才将赃物入官,雪娥官卖。至是衙内谋之于廊吏何不韦,径使官媒婆陶妈妈来西门庆家访求亲事,许说成此门亲事,免县中打卯,还赏银五两。

这陶妈妈听了,喜欢的疾走如飞,一日到于西门庆门首。来昭正在门首立,只见陶妈妈向前道了万福,说道:“动问管家哥一声,此是西门老爹家?”来昭道:“你是那里来的?老爹已下世了,有甚话说?”陶妈妈道:“累及管家进去禀声,我是本县官媒人,名唤陶妈妈,奉衙内小老爹钧语,分付说咱宅内有位奶奶要嫁人,敬来说亲。”那来昭喝道:“你这婆子,好不近理!我家老爹没了一年有余,止有两位奶奶守寡,并不嫁人。常言疾风暴雨,不入寡妇之门。你这媒婆,有要没紧,走来胡撞甚亲事?还不走快着,惹的后边奶奶知道,一顿好打。”那陶妈妈笑道:“管家哥,常言官差吏差,来人不差。小老爹不使我,我敢来?嫁不嫁,起动进去禀声,我好回话去。”来昭道:“也罢,与人方便,自己方便,你少待片时,等我进去。两位奶奶,一位奶奶有哥儿,一位奶奶无哥儿,不知是那一位奶奶要嫁人?”陶妈妈道:“衙内小老爹说,清明那日郊外曾看见来,是面上有几点白麻子的那位奶奶。”

来昭听了,走到后边,如此这般告诉月娘说:“县中使了个官媒人在外面。”倒把月娘吃了一惊,说:“我家并没半个字儿迸出,外边人怎得晓的?”来昭道:“曾在郊外,清明那日见来,说脸上有几个白麻子儿的。”月娘便道:“莫不孟三姐也‘腊月里罗卜——动人心’?忽剌八要往前进嫁人?正是‘世间海水知深浅,惟有人心难忖量’”。一面走到玉楼房中坐下,便问:“孟三娘,奴有件事儿来问你,外面有个保山媒人,说是县中小衙内,清明那日曾见你一面,说你要往前进。端的有此话么?”看官听说,当时没巧不成话,自古姻缘着线牵。那日郊外,孟玉楼看见衙内生的一表人物,风流博浪,两家年甲多相仿佛,又会走马拈弓弄箭,彼此两情四目都有意,已在不言之表。但未知有妻子无妻子,口中不言,心内暗度:“男子汉已死,奴身边又无所出。虽故大娘有孩儿,到明日长大了,各肉儿各疼。闪的我树倒无阴,竹篮儿打水。”又见月娘自有了孝哥儿,心肠改变,不似往时,“我不如往前进一步,寻上个叶落归根之处,还只顾傻傻的守些甚么?到没的担阁了奴的青春年少。”正在思慕之间,不想月娘进来说此话,正是清明郊外看见的那个人,心中又是欢喜,又是羞愧,口里虽说:“大娘休听人胡说,奴并没此话。”不觉把脸来飞红了,正是:

\[
含羞对众休开口,理鬓无言只揾头。
\]
月娘说:“此是各人心里事,奴也管不的许多。”一面叫来昭:“你请那保山进来。”来昭门首唤陶妈妈,进到后边见月娘,行毕了礼数,坐下。小丫鬟倒茶吃了。月娘便问:“保山来,有甚事?”陶妈妈便道:“小媳妇无事不登三宝殿,奉本县正宅衙内分付,说贵宅上有一位奶奶要嫁人,讲说亲事。”月娘道:“俺家这位娘子嫁人,又没曾传出去,你家衙内怎得知道?”陶妈妈道:“俺家衙内说来,清明那日,在郊外亲见这位娘子,生的长挑身材,瓜子面皮,脸上有稀稀几个白麻子,便是这位奶奶。”月娘听了,不消说就是孟三姐了。于是领陶妈妈到玉楼房中明间内坐下。

等勾多时,玉楼梳洗打扮出来。陶妈妈道了万福,说道:“就是此位奶奶,果然话不虚传,人材出众,盖世无双,堪可与俺衙内老爹做个正头娘子。”玉楼笑道:“妈妈休得乱说。且说你衙内今年多大年纪?原娶过妻小没有?房中有人也无?姓甚名谁?有官身无官身?从实说来,休要捣谎。”陶妈妈道:“天么,天么!小媳妇是本县官媒,不比外边媒人快说谎。我有一句说一句,并无虚假。俺知县老爹年五十多岁,止生了衙内老爹一人,今年属马的,三十一岁,正月二十三日辰时建生。见做国子监上舍,不久就是举人、进士。有满腹文章,弓马熟闲,诸子百家,无不通晓。没有大娘子二年光景,房内止有一个从嫁使女答应,又不出众。要寻个娘子当家,敬来宅上说此亲事。若是咱府上做这门亲事,老爹说来,门面差摇,坟茔地土钱粮,一例尽行蠲免,有人欺负,指名说来,拿到县里,任意拶打。”玉楼道:“你衙内有儿女没有?原籍那里人氏?诚恐一时任满,千山万水带去,奴亲都在此处,莫不也要同他去?”陶妈妈道:“俺衙内身边,儿花女花没有,好不单径。原籍是咱北京真定府枣强县人氏,过了黄河不上六七百里。他家中田连阡陌,骡马成群,人丁无数,走马牌楼,都是抚按明文,圣旨在上,好不赫耀吓人。如今娶娘子到家,做了正房,过后他得了官,娘子便是五花官诰,坐七香车,为命妇夫人,有何不好?”这孟玉楼被陶妈妈一席话,说得千肯万肯,一面唤兰香放桌儿,看茶食点心与保山吃。因说:“保山,你休怪我叮咛盘问。你这媒人们说谎的极多,奴也吃人哄怕了。”陶妈妈道:“好奶奶,只要一个比一个。清自清,浑自浑,好的带累了歹的。小媳妇并不捣谎,只依本分做媒。奶奶若肯了,写个婚帖儿与我,好回小老爹话去。”玉楼取了一条大红段子,使玳安交铺子里傅伙计写了生时八字。吴月娘便说:“你当初原是薛嫂儿说的媒,如今还使小厮叫将薛嫂儿来,两个同拿了贴儿去,说此亲事,才是礼。”不多时,使玳安儿叫了薛嫂儿来,见陶妈妈道了万福。当行见当行,拿着贴儿出离西门庆家门,往县中回衙内话去。一个是这里冰人,一个是那头保山,两张口四十八个牙,这一去管取说得月里嫦娥寻配偶,巫山神女嫁襄王。

陶妈妈在路上问薛嫂儿:“你就是这位娘子的原媒?”薛嫂道:“便是。”陶妈妈问他:“原先嫁这里,根儿是何人家的女儿?嫁这里是女儿,是再婚?”这薛嫂儿便一五一十,把西门庆当初从杨家娶来的话告诉一遍。因见婚贴儿上写“女命三十七岁,十一月二十七日子时生”,说:“只怕衙内嫌年纪大些,怎了?他今才三十一岁,倒大六岁。”薛嫂道:“咱拿了这婚贴儿,交个过路的先生,算看年命妨碍不妨碍。若是不对,咱瞒他几岁儿,也不算说谎。”

二人走来,再不见路过响板的先生,只见路南远远的一个卦肆,青布帐幔,挂着两行大字:“子平推贵贱,铁笔判荣枯;有人来算命,直言不容情。”帐子底下安放一张桌子,里面坐着个能写快算灵先生。这两个媒人向前道了万福,先生便让坐下。薛嫂道:“有个女命累先生算一算。”向袖中拿出三分命金来,说:“不当轻视,先生权且收了,路过不曾多带钱来。”先生道:“请说八字。”陶妈妈递与他婚帖看,上面有八字生日年纪,先生道:“此是合婚。”一百捏指寻纹,把算子摇了一摇,开言说道:“这位女命今年三十七岁了,十一月廿七日子时生。甲子月,辛卯日,庚子时,理取印绶之格。女命逆行,见在丙申运中。丙合辛生,往后大有威权,执掌正堂夫人之命。四柱中虽夫星多,然是财命,益夫发福,受夫宠爱,这两年定见妨克,见过了不曾?”薛嫂道:“已克过两位夫主了。”先生道:“若见过,后来好了。”薛嫂儿道:“他往后有子没有?”先生道:“子早哩。直到四十一岁才有一子送老。一生好造化,富贵荣华无比。”取笔批下命词四句道:

\[
娇姿不失江梅态,三揭红罗两画眉。
会看马首升腾日,脱却寅皮任意移。
\]
薛嫂问道:“先生,如何是‘会看马首升腾日,脱却寅皮任意移’?这两句俺每不懂,起动先生讲说讲说。”先生道:“马首者,这位娘子如今嫁个属马的夫主,才是贵星,享受荣华。寅皮是克过的夫主,是属虎的,虽是宠爱,只是偏房。往后一路功名,直到六十八岁,有一子,寿终,夫妻偕老。”两个媒人说道:“如今嫁的倒果是个属马的,只怕大了好几岁,配不来。求先生改少两岁才好。”先生道:“既要改,就改做丁卯三十四岁罢。”薛嫂道:“三十四岁,与属马的也合的着么?”先生道:“丁火庚金,火逢金炼,定成大器,正合得着。”当下改做三十四岁。

两个拜辞了先生,出离卦肆,径到县中。门子报入,衙内便唤进陶、薛二媒人,旋磕了头。衙内便问:“那个妇人是那里的?”陶妈妈道:“是那边媒人。”因把亲事说成,告诉一遍,说:“娘子人才无比的好,只争年纪大些。小媳妇不敢擅便,随衙内老爹尊意,讨了个婚贴在此。”于是递上去。李衙内看了,上写着“三十四岁,十一月廿七日子时生”,说道:“就大三两岁,也罢。”薛嫂儿插口道:“老爹见的是,自古道,妻大两,黄金长;妻大三,黄金山。这位娘子人材出众,性格温柔,诸子百家,当家理纪,自不必说。”衙内道:“我已见过,不必再相。只择吉日良时,行茶礼过去就是了。”两个媒人禀说:“小媳妇几时来伺候?”衙内道:“事不迟稽迟,你两个明日来讨话,往他家说。”每个赏了一两银子,做脚步钱。两个媒人欢喜出门,不在话下。

这李衙内见亲事已成,喜不自胜,即唤廊吏何不韦来商议,对父亲李知县说了。令阴阳生择定四月初八日行礼,十五日准娶妇人过门。就兑出银子来,委托何不韦、小张闲买办茶红酒礼,不必细说。两个媒人次日讨了日期,往西门庆家回月娘、玉楼话。正是:

\[
姻缘本是前生定,曾向蓝田种玉来。
\]

四月初八日,县中备办十六盘羹果茶饼,一副金丝冠儿,一副金头面,一条玛瑙带,一副丁当七事,金镯银钏之类,两件大红宫锦袍儿,四套妆花衣服,三十两礼钱,其余布绢绵花,共约二十余抬。两个媒人跟随,廊吏何不韦押担,到西门庆家下了茶。

十五日,县中拨了许多快手闲汉来,搬抬孟玉楼床帐嫁妆箱笼。月娘看着,但是他房中之物,尽数都交他带去。原旧西门庆在日,把他一张八步彩漆床陪了大姐,月娘就把潘金莲房中那张螺钿床陪了他。玉楼交兰香跟他过去,留下小鸾与月娘看哥儿。月娘不肯,说:“你房中丫头,我怎好留下你的?左右哥儿有中秋儿、绣春和奶子,也勾了。”玉楼止留下一对银回回壶与哥儿耍子,做一念儿,其余都带过去了。到晚夕,一顶四人大轿,四对红纱灯笼,八个皂隶跟随来娶。玉楼戴着金梁冠儿,插着满头珠翠、胡珠子,身穿大红通袖袍儿,先辞拜西门庆灵位,然后拜月娘。月娘说道:“孟三姐,你好狠也!你去了,撇的奴孤另另独自一个,和谁做伴儿?”两个携手哭了一回。然后家中大小都送出大门。媒人替他带上红罗销金盖袱,抱着金宝瓶,月娘守寡出不的门,请大姨送亲,送到知县衙里来。满街上人看见说:“此是西门大官人第三娘子,嫁了知县相公儿子衙内,今日吉日良时娶过门。”也有说好的,也有说歹的。说好者,当初西门大官人怎的为人做人,今日死了,止是他大娘子守寡正大,有儿子,房中搅不过这许多人来,都交各人前进,甚有张主。有那说歹的,街谈巷议,指戳说道:“西门庆家小老婆,如今也嫁人了。当初这厮在日,专一违天害理,贪财好色,奸骗人家妻女。今日死了,老婆带的东西,嫁人的嫁人,拐带的拐带,养汉的养汉,做贼的做贼,都野鸡毛儿零撏了。常言三十年远报,而今眼下就报了。”旁人纷纷议论不题。

且说孟大姨送亲到县衙内,铺陈床帐停当,留坐酒席来家。李衙内赏薛嫂儿、陶妈妈每人五两银子,一段花红利市,打发出门。至晚,两个成亲,极尽鱼水之欢,于飞之乐。到次日,吴月娘送茶完饭。杨姑娘已死,孟大妗子、二妗子、孟大姨都送茶到县中。衙内这边下回书,请众亲戚女眷做三日,扎彩山,吃筵席。都是三院乐人妓女,动鼓乐扮演戏文。吴月娘那日亦满头珠翠,身穿大红通袖袍儿,百花裙,系蒙金带,坐大轿来衙中,进入后边院落,静俏俏无个人接应。想起当初,有西门庆在日,姊妹们那样闹热,往人家赴席来家,都来相见说话,一条板凳坐不了,如今并无一个儿了。一面扑着西门庆灵床儿,不觉一阵伤心,放声大哭。哭了一回,被丫鬟小玉劝止。正是:

\[
平生心事无人识,只有穿窗皓月知。
\]

这里月娘忧闷不题。却说李衙内和玉楼两个,女貌郎才,如鱼如水,正合着油瓶盖。每日燕尔新婚,在房中厮守,一步不离。端详玉楼容貌,越看越爱。又见带了两个从嫁丫鬟,一个兰香,年十八岁,会弹唱;一个小鸾,年十五岁,俱有颜色。心中欢喜没入脚处。有诗为证:

\[
堪夸女貌与郎才,天合姻缘礼所该。
十二巫山云雨会,两情愿保百年偕。
\]

原来衙内房中,先头娘子丢了一个大丫头,约三十年纪,名唤玉簪儿。专一搽胭抹粉,作怪成精。头上打着盘头揸髻,用手贴苫盖,周围勒销金箍儿,假充作\textuni{4BFC}髻,身上穿一套怪绿乔红的裙袄,脚上穿着双拨船样四个眼的剪绒鞋,约长尺二。在人根前,轻身浪颡,做势拿班。衙内未娶玉楼时,他便逐日顿羹顿饭,殷勤伏侍,不说强说,不笑强笑,何等精神。自从娶过玉楼来,见衙内和他如胶似漆,把他不去揪采,这丫头就使性儿起来。一日,衙内在书房中看书,这玉簪儿在厨下顿了一盏好果仁炮茶,双手用盘儿托来书房里,笑嘻嘻掀开帘儿,送与衙内。不想衙内看了一回书,搭伏定书桌就睡着了。这玉簪儿叫道:“爹,谁似奴疼你,顿了这盏好茶儿与你吃。你家那新娶的娘子,还在被窝里睡得好觉儿,怎不交他那小大姐送盏茶来与你吃?”因见衙内打盹,在眼前只顾叫不应,说道:“老花子,你黑夜做夜作使乏了也怎的?大白日里盹磕睡,起来吃茶!”叫衙内醒了,看见是他,喝道:“怪碜奴才!把茶放下,与我过一边去。”这玉簪儿满脸羞红,使性子把茶丢在桌上,出来说道:“好不识人敬重!奴好意用心,大清早辰送盏茶儿来你吃,倒吆喝我起来。常言:‘丑是家中宝,可喜惹烦恼’。我丑,你当初瞎了眼,谁交你要我来?”被衙内听见,赶上尺力踢了两靴脚。这玉簪儿登时把那付奴脸膀的有房梁高,也不搽脸了,也不顿茶了。赶着玉楼,也不叫娘,只你也我也,无人处,一屁股就在玉楼床上坐下。玉楼亦不去理他。他背地又压伏兰香、小鸾说:“你休赶着我叫姐,只叫姨娘。我与你娘系大小之分。”又说:“你只背地叫罢,休对着你爹叫。你每日跟随我行,用心做活,你若不听我说,老娘拿煤锹子请你。”后来几次见衙内不理他,他就撒懒起来,睡到日头半天还不起来,饭儿也不做,地儿也不扫。玉楼分付兰香、小鸾:“你休靠玉簪儿了,你二人自去厨下做饭,打发你爹吃罢。”这玉簪又气不愤,使性谤气,牵家打伙,在厨房内打小鸾,骂兰香:“贼小奴才,小淫妇儿!碓磨也有个先来后到,先有你娘来,先有我来?都是你娘儿们占了罢,不献这个勤儿也罢了!当原先俺死的那个娘也没曾失口叫我声玉簪儿,你进门几日,就题名道姓叫我。我是你手里使的人也怎的?你未来时,我和俺爹同床共枕,那一日不睡到斋时才起来。和我两个如糖拌蜜,如蜜搅酥油一般打热。房中事,那些儿不打我手里过。自从你来了,把我蜜罐儿也打碎了,把我姻缘也拆散开了,一撵撵到我明间,冷清清支板凳打官铺,再不得尝着俺爹那件东西儿如今甚么滋味了。我这气苦也没处声诉。你当初在西门庆家,也曾做第三个小老婆来,你小名儿叫玉楼,敢说老娘不知道?你来在俺家,你识我见,大家脓着些罢了。会那等乔张致,呼张唤李,谁是你买到的?属你管辖?”不知玉楼在房听见,气的发昏,又不好声言对衙内说。

一日热天,也是合当有事。晚夕衙内分付他厨下热水,拿浴盆来房中,要和玉楼洗澡。玉楼便说:“你交兰香热水罢,休要使他。”衙内不从,说道:“我偏使他,休要惯了这奴才。”玉簪儿见衙内要水,和妇人共浴兰汤,效鱼水之欢,心中正没好气,拿浴盆进房,往地下只一墩,用大锅浇上一锅滚水,只中喃喃呐呐说道:“也没见这娘淫妇,刁钻古怪,禁害老娘!无故也只是个浪精毴,没三日不拿水洗。像我与俺主子睡,成月也不见点水儿,也不见展污了甚么佛眼儿。偏这淫妇会,两番三次刁蹬老娘。”直骂出房门来。玉楼听见,也不言语。衙内听了此言,心中大怒,澡也洗不成,精脊梁趿着鞋,向床头取拐子,就要走出来。妇人拦阻住,说道:“随他骂罢,你好惹气。只怕热身子出去,风试着你,倒值了多的。”衙内那里按纳得住,说道:“你休管。这奴才无礼!”向前一把手采住他头发,拖踏在地下,轮起拐子,雨点打将下来。饶玉楼在旁劝着,也打了二三十下在身。打的这丫头急了,跪在地下告说:“爹,你休打我,我想爹也看不上我在家里了,情愿卖了我罢。”衙内听了,亦发恼怒起来,又狠了几下。玉楼劝道:“他既要出去,你不消打,倒没得气了你。”衙内随令伴当即时叫将陶妈妈来,把玉簪儿领出去,便卖银子来交,不在话下。正是:蚊虫遭扇打,只为嘴伤人。有诗为证:

\[
百禽啼后人皆喜,惟有鸦鸣事若何。
见者多言闻者唾,只为人前口嘴多。
\]

\newpage
%# -*- coding:utf-8 -*-
%%%%%%%%%%%%%%%%%%%%%%%%%%%%%%%%%%%%%%%%%%%%%%%%%%%%%%%%%%%%%%%%%%%%%%%%%%%%%%%%%%%%%


\chapter{陈敬济被陷严州府\KG 吴月娘大闹授官厅}


诗曰:

\[
猛虎冯其威,往往遭急缚。雷吼徒暴哮,枝撑已在脚。
忽看皮寝处,无复晴闪烁。人有甚于斯,尽以劝元恶。
\]

话说李衙内打了玉簪儿一顿,即时叫陶妈妈来领出,卖了八两银子,另买了个十八岁使女,名唤满堂儿上灶,不在话下。

却表陈敬济,自从西门大姐来家,交还了许多床帐妆奁,箱笼家伙,三日一场嚷,五日一场闹,问他娘张氏要本钱做买卖。他母舅张团练,来问他母亲借了五十两银子,复谋管事。被他吃醉了,往张舅门上骂嚷。他张舅受气不过,另问别处借了银子,干成管事,还把银子交还交来。他母亲张氏,着了一场重气,染病在身,日逐卧床不起,终日服药,请医调治。吃他逆殴不过,只得兑出三百两银子与他,叫陈定在家门首,打开两间房子开布铺,做买卖。敬济便逐日结交朋友陆三郎、杨大郎狐朋狗党,在铺中弹琵琶,抹骨牌,打双陆,吃半夜酒,看看把本钱弄下去了。陈定对张氏说他每日饮酒花费。张氏听信陈定言语,便不肯托他。敬济反说陈定染布去,克落了钱,把陈定两口儿撵出来外边居住,却搭了杨大郎做伙计。这杨大郎名唤杨光彦,绰号为铁指甲,专一粜风卖雨,架谎凿空。他许人话,如捉影捕风,骗人财,似探囊取物。这敬济问娘又要出二百两银子来添上,共凑了五百两银子,信着他往临清贩布去。

这杨大郎到家收拾行李,跟着敬济从家中起身,前往临清马头上寻缺货去。到了临清,这临清闸上是个热闹繁华大马头去处,商贾往来之所,车辆辐凑之地,有三十二条花柳巷,七十二座管弦楼。这敬济终是年小后生,被这杨大郎领着游娼楼,登酒店,货物到贩得不多。因走在一娼楼,见了一个粉头,名唤冯金宝,生的风流俏丽,色艺双全。问青春多少,鸨子说:“姐儿是老身亲生之女,止是他一人挣钱养活。今年青春才交二九一十八岁。”敬济一见,心目荡然,与了鸨子五两银子房金,一连和他歇了几夜。杨大郎见他爱这粉头,留连不舍,在旁花言说念,就要娶他家去。鸨子开口要银一百二十两,讲到一百两上,兑了银子,娶了来家。一路上用轿抬着,杨大郎和敬济都骑马,押着货物车走,一路扬鞭走马,那样欢喜。正是:

\[
多情燕子楼,马道空回首。
载得武陵春,陪作鸾凰友。
\]
张氏见敬济货到贩得不多,把本钱到娶了一个唱的来家,又着了口重气,呜呼哀哉,断气身亡。这敬济不免买棺装殓,念经做七,停放了一七光景,发送出门,祖茔合葬。他母舅张团练看他娘面上,亦不和他一般见识。这敬济坟上覆墓回来,把他娘正房三间,中间供养灵位,那两间收拾与冯金宝住,大姐到住着耳房。又替冯金宝买了丫头重喜儿伏侍。门前杨大郎开着铺子,家里大酒大肉买与唱的吃。每日只和唱的睡,把大姐丢着不去揪采。

一日,打听孟玉楼嫁了李知县儿子李衙内,带过许多东西去。三年任满,李知县升在浙江严州府做了通判,领凭起身,打水路赴任去了。这陈敬济因想起昔日在花园中拾了孟玉楼那根簪子,就要把这根簪子做个证儿,赶上严州去。只说玉楼先与他有了奸,与了他这根簪子,不合又带了许多东西,嫁了李衙内,都是昔日杨戬寄放金银箱笼,应没官之物。“那李通判一个文官,多大汤水!听见这个利害口声,不怕不叫他儿子双手把老婆奉与我。我那时娶将来家,与冯金宝做一对儿,落得好受用。”正是:计就月中擒月兔,谋成日里捉金乌。敬济不来到好,此一来,正是:失晓人家逢五道,溟泠饿鬼撞钟馗。有诗为证:

\[
赶到严州访玉人,人心难忖似石沉。
侯门一旦深似海,从此萧郎落陷坑。
\]

一日,陈敬济打点他娘箱中,寻出一千两金银,留下一百两与冯金宝家中盘缠,把陈定复叫进来看家,并门前铺子发卖零碎布匹。他与杨大郎又带了家人陈安,押着九百两银子,从八月中秋起身,前往湖州贩了半船丝绵绸绢,来到清江浦马头上,湾泊住了船只,投在个店主人陈二店内。交陈二杀鸡取酒,与杨大郎共饮。饮酒中间,和杨大郎说:“伙计,你暂且看守船上货物,在二郎店内略住数日。等我和陈安拿些人事礼物,往浙江严州府,看看家姐嫁在府中。多不上五日,少只三日就来。”杨大郎道:“哥去只顾去。兄弟情愿店中等候。哥到日,一同起身。”

这陈敬济千不合万不合和陈安身边带了些银两、人事礼物,有日取路径到严州府。进入城内,投在寺中安下。打听李通判到任一个月,家小船只才到三日。这陈敬济不敢怠慢,买了四盘礼物,四匹纻丝尺头,陈安押着。他便拣选衣帽齐整,眉目光鲜,径到府衙前,与门吏作揖道:“烦报一声,说我是通判老爹衙内新娶娘子的亲,孟二舅来探望。”这门吏听了,不敢怠慢,随即禀报进去。衙内正在书房中看书,听见是妇人兄弟,令左右先把礼物抬进来,一面忙整衣冠,道:“有请。”把陈敬济请入府衙厅上叙礼,分宾主坐下,说道:“前日做亲之时,怎的不会二舅?”敬济道:“在下因在川广贩货,一年方回。不知家姐嫁与府上,有失亲近。今日敬备薄礼,来看看家姐。”李衙内道:“一向不知,失礼,恕罪,恕罪。”须臾,茶汤已罢,衙内令左右:“把礼贴并礼物取进去,对你娘说,二舅来了。”孟玉楼正在房中坐的,只听小门子进来,报说:“孟二舅来了。”玉楼道:“再有那个舅舅,莫不是我二哥孟锐来家了,千山万水来看我?”只见伴当拿进礼物和贴儿来,上面写着:“眷生孟锐”,就知是他兄弟,一面道:“有请。”令兰香收拾后堂干净。

玉楼装点打扮,俟候出见。只见衙内让直来,玉楼在帘内观看,可霎作怪,不是他兄弟,却是陈姐夫。“他来做甚么?等我出去,见他怎的说话?常言,亲不亲,故乡人;美不美,乡中水。虽然不是我兄弟,也是我女婿人家。”一面整妆出来拜见。那敬济说道:“一向不知姐姐嫁在这里,没曾看得……”才说得这句,不想门子来请衙内,外边有客来了。这衙内分付玉楼款待二舅,就出去待客去了。玉楼见敬济磕下头去,连忙还礼,说道:“姐夫免礼,那阵风儿刮你到此?”叙毕礼数,上坐,叫兰香看茶出来。吃了茶,彼此叙了些家常话儿,玉楼因问:“大姐好么?”敬济就把从前西门庆家中出来,并讨箱笼的一节话告诉玉楼。玉楼又把清明节上坟,在永福寺遇见春梅,在金莲坟上烧纸的话告诉他。又说:“我那时在家中,也常劝你大娘,疼女儿就疼女婿,亲姐夫,不曾养活了外人。他听信小人言语,把姐夫打发出来。落后姐夫讨箱子,我就不知道。”敬济道:“不瞒你老人家说,我与六姐相交,谁人不知?生生吃他听奴才言语,把他打发出去,才吃武松杀了。他若在家,那武松有七个头八个胆,敢往你家来杀他?我这仇恨,结的有海来深。六姐死在阴司里,也不饶他。”玉楼道:“姐夫也罢,丢开手的事,自古冤仇只可解,不可结。”

说话中间,丫鬟放下桌儿,摆下酒来,杯盘肴品,堆满春台。玉楼斟上一杯酒,双手递与敬济说:“姐夫远路风尘,无可破费,且请一杯儿水酒。”这敬济用手接了,唱了喏,也斟一杯回奉妇人,叙礼坐下,因见妇人“姐夫长,姐夫短”叫他,口中不言,心内暗道:“这淫妇怎的不认犯,只叫我姐夫?等我慢慢的探他。”当下酒过三巡,肴添五道,无人在跟前,先丢几句邪言说入去,道:“我兄弟思想姐姐,如渴思浆,如热思凉,想当初在丈人家,怎的在一处下棋抹牌,同坐双双,似背盖一般。谁承望今日各自分散,你东我西。”玉楼笑道:“姐夫好说。自古清者清而浑者浑,久而自见。”这敬济笑嘻嘻向袖中取出一包双人儿的香茶,递与妇人,说:“姐姐,你若有情,可怜见兄弟,吃我这个香茶儿。”说着,就连忙跪下。那妇人登时一点红从耳畔起,把脸飞红了,一手把香茶包儿掠在地下,说道:“好不识人敬重!奴好意递酒与你吃,到戏弄我起来。”就撇了酒席往房里去了。敬济见他不理,一面拾起香茶来,就发话道:“我好意来看你,你到变了卦儿。你敢说你嫁了通判儿子好汉子,不采我了。你当初在西门庆家做第三个小老婆,没曾和我两个有首尾?”因向袖中取出旧时那根金头银簪子,拿在手内说:“这个是谁人的?你既不和我有奸,这根簪儿怎落在我手里?上面还刻着玉楼名字。你和大老婆串同了,把我家寄放的八箱子金银细软、玉带宝石东西,都是当朝杨戬寄放应没官之物,都带来嫁了汉子。我教你不要慌,到八字八\textJinXia 儿上和你答话!”

玉楼见他发话,拿的簪子委是他头上戴的金头莲瓣簪儿:“昔日在花园中不见,怎的落在这短命手里?”恐怕嚷的家下人知道,须臾变作笑吟吟脸儿,走将出来,一把手拉敬济,说道:“好阻夫,奴斗你耍子,如何就恼起来。”因观看左右无人,悄悄说:“你既有心,奴亦有意。”两个不由分说,搂着就亲嘴。这陈敬济把舌头似蛇吐信子一般,就舒到他口里交他咂,说道:“你叫我声亲亲的丈夫,才算你有我之心。”妇人道:“且禁声,只怕有人听见。”敬济悄悄向他说:“我如今治了半船货,在清江浦等候。你若肯下顾时,如此这般,到晚夕假扮门子,私走出来,跟我上船家去,成其夫妇,有何不可?他一个文职官,怕是非,莫不敢来抓寻你不成?”妇人道:“既然如此,也罢。”约会下:“你今晚在府墙后等着,奴有一包金银细软,打墙上系过去,与你接了,然后奴才扮做门子,打门里出来,跟你上船去罢。”看官听说,正是佳人有意,那怕粉墙高万丈;红粉无情,总然共坐隔千山。当时孟玉楼若嫁得个痴蠢之人,不如敬济,敬济便下得这个锹镢着;如今嫁这李衙内,有前程,又且人物风流,青春年少,恩情美满,他又勾你做甚?休说平日又无连手。这个郎君也是合当倒运,就吐实话,泄机与他,倒吃婆娘哄赚了。正是:

\[
花枝叶下犹藏刺,人心难保不怀毒。
\]

当下二人会下话,这敬济吃了几杯酒,告辞回去。李衙内连忙送出府门,陈安跟随而去。衙内便问妇人:“你兄弟住那里下处?我明日回拜他去,送些嗄程与他。”妇人便说:“那里是我兄弟,他是西门庆家女婿,如此这般,来勾搭要拐我出去。奴已约下他,今晚三更在后墙相等。咱不如将计就计,把他当贼拿下,除其后患如何?”衙内道:“叵耐这厮无端,自古无毒不丈夫,不是我去寻他,他自来送死。”一面走出外边,叫过左右伴当,心腹快手,如此这般预备去了。

这陈敬济不知机变,至半夜三更,果然带领家人陈安,来府衙后墙下,以咳嗽为号,只听墙内玉楼声音,打墙上掠过一条索子去,那边系过一大包银子。原来是库内拿的二百两赃罚银子。这敬济才待教陈安拿着走,忽听一阵梆子响,黑影里闪出四五条汉,叫声:“有贼了!”登时把敬济连陈安都绑了,禀知李通判,分付:“都且押送牢里去,明日问理。”

原来严州府正堂知府姓徐,名唤徐崶,系陕西临洮府人氏,庚戌进士,极是个清廉刚正之人。次早升堂,左右排两行官吏,这李通判上去,画了公座,库子呈禀贼情事,带陈敬济上去,说:“昨夜至一更时分,有先不知名今知名贼人二名:陈敬济、陈安,锹开库门锁钥,偷出赃银二百两,越墙而过,致被捉获,来见老爷。”徐知府喝令:“带上来!”把陈敬济并陈安揪采驱拥至当厅跪下。知府见敬济年少清俊,便问:“这厮是那里人氏?因何来我这府衙公廨,夜晚做贼,偷盗官库赃银,有何理说?”那陈敬济只顾磕头声冤。徐知府道:“你做贼如何声冤?”李通判在旁欠身便道:“老先生不必问他,眼见得赃证明白,何不回刑起来。”徐知府即令左右:“拿下去打二十板。”李通判道:“人是苦虫,不打不成。不然,这贼便要展转。”当下两边皂隶,把敬济、陈安拖番,大板打将下来。这陈敬济口内只骂:“谁知淫妇孟三儿陷我至此,冤哉!苦哉!”这徐知府终是黄堂出身官人,听见这一声,必有缘故,才打到十板上,喝令:“住了,且收下监去,明日再问。”李通判道:“老先生不该发落他,常言‘人心似铁,官法如炉’,从容他一夜不打紧,就翻异口词。”徐知府道:“无妨,吾自有主意。”当下狱卒把敬济、陈安押送监中去讫。

这徐知府心中有些疑忌,即唤左右心腹近前,如此这般,下监中探听敬济所犯来历,即便回报。这干事人假扮作犯人,和敬济晚间在一\textuni{3B71}上睡,问其所以:“我看哥哥青春年少,不是做贼的,今日落在此,打屈官司。”敬济便说:“一言难尽,小人本是清河县西门庆女婿,这李通判儿子新娶的妇人孟氏,是俺丈人的小,旧与我有奸的。今带过我家老爷杨戬寄放十箱金银宝玩之物来他家,我来此间问他索讨,反被他如此这般欺负,把我当贼拿了。苦打成招,不得见其天日,是好苦也!”这人听了,走来退厅告报徐知府。知府道:“如何?我说这人声冤叫孟氏,必有缘故。”

到次日升堂,官吏两旁侍立。这徐知府把陈敬济、陈安提上来,摘了口词,取了张无事的供状,喝令释放。李通判在旁不知,还再三说:“老先生,这厮贼情既的,不可放他。”反被徐知府对佐贰官尽力数说了李通判一顿,说:“我居本府正官,与朝廷干事,不该与你家官报私仇,诬陷平人作贼。你家儿子娶了他丈人西门庆妾孟氏,带了许多东西,应没官赃物,金银箱笼来。他是西门庆女婿,径来索讨前物,你如何假捏贼情,拿他入罪,教我替你家出力?做官养儿养女,也要长大,若是如此,公道何堪?”当厅把李通判数说的满面羞惭,垂首丧气而不敢言。陈敬济与陈安便释放出去了。良久,徐知府退堂。

这李通判回到本宅,心中十分焦燥。便对夫人大嚷大叫道:“养的好不肖子,今日吃徐知府当堂对众同僚官吏,尽力数落了我一顿,可不气杀我也!”夫人慌了,便道:“甚么事?”李通判即把儿子叫到跟前,喝令左右:“拿大板子来,气杀我也!”说道:“你拿得好贼,他是西门庆女婿。因这妇人带了许多妆奁、金银箱笼来,他口口声声称是当朝逆犯杨戬寄放应没官之物,来问你要。说你假盗出库中官银,当贼情拿他。我通一字不知,反被正堂徐知府对众数说了我这一顿。此是我头一日官未做,你照顾我的。我要你这不肖子何用?”即令左右雨点般大板子打将下来。可怜打得这李衙内皮开肉绽,鲜血迸流。夫人见打得不像模样,在旁哭泣劝解。孟玉楼立在后厅角门首,掩泪潜听。当下打了三十大板,李通判分付左右:“押着衙内,即时与我把妇人打发出门,令他任意改嫁,免惹是非,全我名节。”那李衙内心中怎生舍得离异,只顾在父母跟前啼哭哀告:“宁把儿子打死爹爹跟前,并舍不的妇人。”李通判把衙内用铁索墩锁在后堂,不放出去,只要囚禁死他。夫人哭道:“相公,你做官一场,年纪五十余岁,也只落得这点骨血。不争为这妇人,你囚死他,往后你年老休官,倚靠何人?”李通判道:“不然,他在这里,须带累我受人气。”夫人道:“你不容他在此,打发他两口儿回原籍真定府家去便了。”通判依听夫人之言,放了衙内,限三日就起身,打点车辆,同妇人归枣强县里攻书去了。

却表陈敬济与陈安出离严州府,到寺中取了行李,径往清江浦陈二店中来寻杨大郎。陈二说:“他三日前,说你有信来说不得来,他收拾了货船,起身往家中去了。”这敬济未信,还向河下去寻船只,扑了个空。说道:“这天杀的,如何不等我来就起身去了!”况新打监中出来,身边盘缠已无,和陈安不免搭在人船上,把衣衫解当,讨吃归家,忙忙似丧家之犬,急急如漏网之鱼,随行找寻杨大郎,并无踪迹。那时正值秋暮天气,树木凋零,金风摇落,甚是凄凉。有诗八句,单道这秋天行人最苦:

\[
栖栖芰荷枯,叶叶梧桐坠。蛩鸣腐草中,雁落平沙地。
细雨湿青林,霜重寒天气。不见路行人,怎晓秋滋味。
\]

有日敬济到家。陈定正在门首,看见敬济来家,衣衫褴褛,面貌黧黑,唬了一跳。接到家中,问货船到于何处。敬济气得半日不言,把严州府遭官司一节说了:“多亏正堂徐知府放了我,不然性命难保。今被杨大郎这天杀的,把我货物不知拐的往那里去了。”先使陈定往他家探听,他家说还不曾来家。敬济又亲去问了一遭,并没下落,心中着慌,走入房中。那冯金宝又和西门大姐首南面北,自从敬济出门,两个合气,直到如今。大姐便说:“冯金宝拿着银子钱,转与他鸨子去了。他家保儿成日来,瞒藏背掖,打酒买肉,在屋里吃。家中要的没有,睡到晌午,诸事儿不买,只熬俺们。”冯金宝又说:“大姐成日模草不拈,竖草不动,偷米换烧饼吃。又把煮的腌肉偷在房里,和丫头元宵儿同吃。”这陈敬济就信了,反骂大姐:“贼不是才料淫妇,你害馋痨谗痞了,偷米出去换烧饼吃,又和丫头打伙儿偷肉吃。”把元宵儿打了一顿,把大姐踢了几脚。这大姐急了,赶着冯金宝儿撞头,骂道:“好养汉的淫妇!你偷盗的东西与鸨子不值了,到学舌与汉子,说我偷米偷肉,犯夜的倒拿住巡更的了,教汉子踢我。我和你这淫妇兑换了罢,要这命做甚么!”这敬济道:“好淫妇,你换兑他,你还不值他几个脚指头儿哩。”也是合当有事,于是一把手采过大姐头发来,用拳撞脚踢、拐子打,打得大姐鼻口流血,半日苏醒过来。这敬济便归唱的房里睡去了。由着大姐在下边房里呜呜咽咽,只顾哭泣。元宵儿便在外间睡着了。可怜大姐到半夜,用一条索子悬梁自缢身死,亡年二十四岁。

到次日早辰,元宵起来,推里间不开。上房敬济和冯金宝还在被窝里,使他丫头重喜儿来叫大姐,要取木盆洗坐脚,只顾推不开。敬济还骂:“贼淫妇,如何还睡?这咱晚不起来!我这一跺开门进去,把淫妇鬓毛都拔净了。”重喜儿打窗眼内望里张看,说道:“他起来了,且在房里打秋千耍子儿哩。”又说:“他提偶戏耍子儿哩。”只见元宵瞧了半日,叫道:“爹,不好了,俺娘吊在床顶上吊死了。”这小郎才慌了,和唱的齐起来,跺开房门,向前解卸下来,灌救了半日,那得口气儿来。不知多咱时分,呜呼哀哉死了。正是:

\[
不知真性归何处,疑在行云秋水中。
\]

陈定听见大姐死了,恐怕连累,先走去报知月娘。月娘听见大姐吊死了,敬济娶唱的在家,正是冰厚三尺,不是一日之寒,率领家人小厮、丫鬟媳妇七八口,往他家来。见了大姐尸首吊的直挺挺的,哭喊起来,将敬济拿住,揪采乱打,浑身锥了眼儿也不计数。唱的冯金宝躲在床底下,采出来,也打了个臭死。把门窗户壁都打得七零八落,房中床帐妆奁都还搬的去了。归家请将吴大舅、二舅来商议。大舅说:“姐姐,你趁此时咱家人死了不到官,到明日他过不得日子,还来缠要箱笼。人无远虑,必有近忧。不如到官处断开了,庶杜绝后患。”月娘道:“哥见得是。”一面写了状子。

次日,月娘亲自出官,来到本县授官厅下,递上状去。原来新任知县姓霍,名大立,湖广黄冈县人氏,举人出身,为人鲠直。听见系人命重事,即升厅受状。见状上写着:

\[
告状人吴氏,年三十四岁,系已故千户西门庆妻。状告为恶婿欺凌孤孀,听信娼妇,熬打逼死女命,乞怜究治,以存残喘事。比有女婿陈敬济,遭官事投来氏家,潜住数年。平日吃酒行凶,不守本分,打出吊入。氏惧法逐离出门。岂期敬济怀恨,在家将氏女西门氏,时常熬打,一向含忍。不料伊又娶临清娼妇冯金宝来家,夺氏女正房居住,听信唆调,将女百般痛辱熬打,又采去头发,浑身踢伤,受忍不过,比及将死,于本年八月廿三日三更时分,方才将女上吊缢死。切思敬济,恃逞凶顽,欺氏孤寡,声言还要持刀杀害等语,情理难容。乞赐行拘到案,严究女死根由,尽法如律。庶凶顽知警,良善得以安生,而死者不为含冤矣。为此具状上告本县青天老爷施行。
\]
这霍知县在公座上看了状子,又见吴月娘身穿缟素,腰系孝裙,系五品职官之妻,生的容貌端庄,仪容闲雅。欠身起来,说道:“那吴氏起来,据我看,你也是个命官娘子,这状上情理,我都知了。你请回去,今后只令一家人在此伺候就是了。我就出牌去拿他。”那吴月娘连忙拜谢了知县,出来坐轿子回家,委付来昭厅下伺候。须臾批了呈状,委两个公人,一面白牌,行拘敬济、娼妇冯金宝,并两邻保甲,正身赴官听审。

这敬济正在家里乱丧事,听见月娘告下状来,县中差公人发牌来拿他,唬的魂飞天外,魄丧九霄。那冯金宝已被打得浑身疼痛,睡在床上。听见人拿他,唬的魂也不知有无。陈敬济没高低使钱,打发公人吃了酒饭,一条绳子连唱的都拴到县里。左邻范纲,右邻孙纪,保甲王宽。霍知县听见拿了人来,即时升厅。来昭跪在上首,陈敬济、冯金宝一行人跪在阶下。知县看了状子,便叫敬济上去说:“你这厮可恶!因何听信娼妇,打死西门氏,方令上吊,有何理说?”敬济磕头告道:“望乞青天老爷察情,小的怎敢打死他。因为搭伙计在外,被人坑陷了资本,着了气来家,问他要饭吃。他不曾做下饭,委被小的踢了两脚。他到半夜自缢身死了。”知县喝道:“你既娶下娼妇,如何又问他要饭吃?尤说不通。吴氏状上说你打死他女儿,方才上吊,你还不招认!”敬济说:“吴氏与小的有仇,故此诬陷小的,望老爷察情。”知县大怒,说:“他女儿见死了,还推赖那个?”喝令左右拿下去,打二十大板。提冯金宝上来,拶了一拶,敲一百敲。令公人带下收监。次日,委典史臧不息带领吏书、保甲、邻人等,前至敬济家,抬出尸首,当场检验。身上俱有青伤,脖项间亦有绳痕,生前委因敬济踢打伤重,受忍不过,自缢身死。取供具结,回报县中。知县大怒,又打了敬济十板。金宝褪衣,也是十板。问陈敬济夫殴妻至死者绞罪,冯金宝递决一百,发回本司院当差。

这陈敬济慌了,监中写出贴子,对陈定说,把布铺中本钱,连大姐头面,共凑了一百两银子,暗暗送与知县。知县一夜把招卷改了,止问了个逼令身死,系杂犯,准徒五年,运灰赎罪。吴月娘再三跪门哀告。知县把月娘叫上去,说道:“娘子,你女儿项上已有绳痕,如何问他殴杀条律?人情莫非忒偏向么?你怕他后边缠扰你,我这里替你取了他杜绝文书,令他再不许上你门就是了。”一面把陈敬济提到跟前,分付道:“我今日饶你一死,务要改过自新,不许再去吴氏家缠扰。再犯到我案下,决然不饶。即便把西门氏买棺装殓,发送葬埋来回话,我这里好申文书往上司去。”这敬济得了个饶,交纳了赎罪银子,归到家中,抬尸入棺,停放一七,念经送葬,埋城外。前后坐了半个月监,使了许多银两,唱的冯金宝也去了,家中所有都干净了,房儿也典了,刚刮剌出个命儿来,再也不敢声言丈母了。正是:祸福无门人自招,须知乐极有悲来。有诗为证:

\[
风波平地起萧墙,义重恩深不可忘。
水溢蓝桥应有会,三星权且作参商。
\]

\newpage
%# -*- coding:utf-8 -*-
%%%%%%%%%%%%%%%%%%%%%%%%%%%%%%%%%%%%%%%%%%%%%%%%%%%%%%%%%%%%%%%%%%%%%%%%%%%%%%%%%%%%%


\chapter{王杏庵义恤贫儿\KG 金道士娈淫少弟}


诗曰:

\[
阶前潜制泪,众里自嫌身。气味如中酒,情怀似别人。
暖风张乐席,晴日看花尘。尽是添愁处,深居乞过春。
\]

话说陈敬济,自从西门大姐死了,被吴月娘告了一状,打了一场官司出来,唱的冯金宝又归院中去了,刚刮剌出个命儿来。房儿也卖了,本钱儿也没了,头面也使了,家伙也没了。又说陈定在外边打发人,克落了钱,把陈定也撵去了。家中日逐盘费不周,坐吃山空,不时往杨大郎家中,问他这半船货的下落。一日,来到杨大郎门首,叫声:“杨大郎在家不在?”不想杨光彦拐了他半船货物,一向在外,卖了银两,四散躲闪。及打听得他家中吊死了老婆,他丈母县中告他,坐了半个月监,这杨大郎就蓦地来家住着。听见敬济上门叫他,问货船下落,一径使兄弟杨二风出来,反问敬济要人:“你把我哥哥叫的外面做买卖,这几个月通无音信,不知抛在江中,推在河内,害了性命,你倒还来我家寻货船下落?人命要紧,你那货物要紧?”这杨二风平昔是个刁徒泼皮,耍钱捣子,胳膊上紫肉横生,胸前上黄毛乱长,是一条直率光棍。走出来一把扯住敬济,就问他要人。那敬济慌忙挣开手跑出回家来。这杨二风故意拾了块三尖瓦楔,将头颅钻破,血流满面,赶将敬济来,骂道:“我\textuni{34B2}你娘娘!我见你家甚么银子来?你来我屋里放屁,吃我一顿好拳头。”那敬济金命水命,走投无命,奔到家,把大门关闭如铁桶相似,由着杨二风牵爹娘,骂父母,拿大砖砸门,只是鼻口内不敢出气儿。又况才打了官司出来,梦条绳蛇也害怕,只得含忍过了。正是:

\[
嫩草怕霜霜怕日,恶人自有恶人磨。
\]

不消几时,把大房卖了,找了七十两银子,典了一所小房,在僻巷内居住。落后两个丫头,卖了一个重喜儿,只留着元宵儿和他同铺歇。又过了不上半月,把小房倒腾了,却去赁房居住。陈安也走了,家中没营运,元宵儿也死了,止是单身独自,家伙桌椅都变卖了,只落得一贫如洗。未几,房钱不给,钻入冷铺内存身。花子见他是个富家勤儿,生得清俊,叫他在热炕上睡,与他烧饼儿吃。有当夜的过来教他顶火夫,打梆子摇铃。

那时正值腊月,残冬时分,天降大雪,吊起风来,十分严寒。这工敬济打了回梆子,打发当夜的兵牌过去,不免手提铃串了几条街巷。又是风雪,地下又踏着那寒冰,冻得耸肩缩背,战战兢兢。临五更鸡叫,只见个病花子躺在墙底下,恐怕死了,总甲分付他看守着,寻了把草叫他烤。这敬济支更一夜,没曾睡,就歪下睡着了。不想做了一梦,梦见那时在西门庆家,怎生受荣华富贵,和潘金莲勾搭,顽耍戏谑,从睡梦中就哭醒来。众花子说:“你哭怎的?”这敬济便道:“你众位哥哥,我的苦楚,你怎得知?

\[
频年困苦痛妻亡,身上无衣口绝粮。
马死奴逃房又卖,只身独自在他乡。
朝依肆店求遗馔,暮宿庄园倚败墙。
只有一条身后路,冷铺之中去打梆。”
\]

陈敬济晚夕在冷铺存身,白日间街头乞食。

清河县城内有一老者,姓王名宣,字廷用,年六十余岁,家道殷实,为人心慈,仗义疏财,专一济贫拔苦,好善敬神。所生二子,皆当家成立。长子王乾,袭祖职为牧马所掌印正千户;次子王震,充为府学庠生。老者门首搭了个主管,开着个解当铺儿。每日丰衣足食,闲散无拘,在梵宇听经,琳宫讲道。无事在家门首施药救人,拈素珠念佛。因后园中有两株杏树,道号为杏庵居士。

一日,杏庵头戴重檐幅巾,身穿水合道服,在门首站立。只见陈敬济打他门首过,向前扒在地下磕了个头。忙的杏庵还礼不迭,说道:“我的哥,你是谁?老拙眼昏,不认的你。”这敬济战战兢兢,站立在旁边说道:“不瞒你老人家,小人是卖松槁陈洪儿子。”老者想了半日,说:“你莫不是陈大宽的令郎么?”因见他衣服褴褛,形容憔悴,说道:“贤侄,你怎的弄得这般模样?”便问:“你父亲、母亲可安么?”敬济道:“我爹死在东京,我母亲也死了。”杏庵道:“我闻得你在丈人家住来?”敬济道:“家外父死了,外母把我撵出来。他女儿死了,告我到官,打了一场官司。把房儿也卖了,有些本钱儿,都吃人坑了,一向闲着没有营生。”杏庵道:“贤侄,你如今在那里居住?”敬济半日不言语,说:“不瞒你老人家说,如此如此。”杏庵道:“可怜,贤侄你原来讨吃哩。想着当初,你府上那样根基人家。我与你父亲相交,贤侄,你那咱还小哩,才扎着总角上学堂,怎就流落到此地位?可伤,可伤。你政治家甚亲家?也不看顾你看顾儿。”敬济道:“正是。俺张舅那里,一向也久不上门,不好去的。”

问了一回话,老者把他让到里面客位里,令小厮放桌儿,摆出点心嗄饭来,教他尽力吃了一顿。见他身上单寒,拿出一件青布绵道袍儿,一顶毡帽,又一双毡袜、绵鞋,又秤一两银子,五百铜钱,递与他,分付说:“贤侄,这衣服鞋袜与你身上,那铜钱与你盘缠,赁半间房儿住;这一两银子,你拿着做上些小买卖儿,也好糊口过日子,强如在冷铺中,学不出好人来。每月该多少房钱,来这里,老拙与你。”这陈敬济扒在地下磕头谢了,说道:“小侄知道。”拿着银钱,出离了杏庵门首。也不寻房子,也不做买卖,把那五百文钱,每日只在酒店面店以了其事。那一两银子,捣了些白铜顿罐,在街上行使。吃巡逻的当土贼拿到该坊节级处,一顿拶打,使的罄尽,还落了一屁股疮。不消两日,把身上绵衣也输了,袜儿也换嘴来吃了,依旧原在街上讨吃。

一日,又打王杏庵门首所过,杏庵正在门首,只见敬济走来磕头,身上衣袜都没了,止戴着那毡帽,精脚趿鞋,冻的乞乞缩缩。老者便问:“陈大官,做的买卖如何?房钱到了,来取房钱来了?”那陈敬济半日无言可对。问之再三,方说如此这般,都没了。老者便道:“阿呀,贤侄,你这等就不是过日子的道理。你又拈不的轻,负不的重,但做了些小活路儿,不强如乞食,免教人耻笑,有玷你父祖之名。你如何不依我说?”一面又让到里面,教安童拿饭来与他吃饱了。又与了他一条夹裤,一领白布衫,一双裹脚,一吊铜钱,一斗米:“你拿去务要做上了小买卖,卖些柴炭、豆儿、瓜子儿,也过了日子,强似这等讨吃。”这敬济口虽答应,拿钱米在手,出离了老者门,那消几日,熟食肉面,都在冷铺内和花子打伙儿都吃了。耍钱,又把白布衫、夹裤都输了。大正月里,又抱着肩儿在街上走,不好来见老者,走在他门首房山墙底下,向日阳站立。

老者冷眼看见他,不叫他。他挨挨抢抢,又到根前扒在地下磕头。老者见他还依旧如此,说道:“贤侄,这不是常策。咽喉深似海,日月快如梭,无底坑如何填得起?你进来,我与你说,有一个去处,又清闲,又安得你身,只怕你不去。”敬济跪下哭道:“若得老伯见怜,不拘那里,但安下身,小的情愿就去。”杏庵道:“此去离城不远,临清马头上,有座晏公庙。那里鱼米之乡,舟船辐辏之地,钱粮极广,清幽潇洒。庙主任道士,与老拙相交极厚,他手下也有两三个徒弟徒孙。我备分礼物,把你送与他做个徒弟出家,学些经典吹打,与人家应福,也是好处。”敬济道:“老伯看顾,可知好哩。”杏庵道:“既然如此,你去,明日是个好日子,你早来,我送你去。”敬济去了。这王老连忙叫了裁缝来,就替敬济做了两件道袍,一顶道髻,鞋袜俱全。

次日,敬济果然来到。王老教他空屋里洗了澡,梳了头,戴上道髻,里外换了新袄新裤,上盖表绢道衣,下穿云履毡袜,备了四盘羹果,一坛酒,一匹尺头,封了五两银子。他便乘马,雇了一匹驴儿与敬济骑着,安童、喜童跟随,两个人担了盒担,出城门,径往临清马头晏公庙来。止七十里,一日路程。比及到晏公庙,天色已晚,王老下马,进入庙来。只见青松郁郁,翠柏森森,两边八字红墙,正面三间朱户,端的好座庙宇。但见:

\[
山门高耸,殿阁棱层。高悬敕额金书,彩画出朝入相。五间大殿,塑龙王一十二尊;两下长廊,刻水族百千万众。旗竿凌汉,帅字招风。四通八达,春秋社礼享依时;雨顺风调,河道民间皆祭赛。万年香火威灵在,四境官民仰赖安。
\]

山门下早有小童看见,报入方丈,任道士忙整衣出迎。王杏庵令敬济和礼物且在外边伺候。不一时,任道士把杏庵让入方丈松鹤轩叙礼,说:“王老居上,怎生一向不到敝庙随喜?今日何幸,得蒙下顾。”杏庵道:“只因家中俗冗所羁,久失拜望。”叙礼毕,分宾主而坐,小童献茶。茶罢,任道士道:“老居士,今日天色已晚,你老人家不去罢了。”分付把马牵入后槽喂息。杏庵道:“没事不登三宝殿。老拙敬来有一事干渎,未知尊意肯容纳否?”任道士道:“老居士有何见教?只顾分付,小道无不领命。”杏庵道:“今有故人之子,姓陈,名敬济,年方二十四岁。生的资格清秀,倒也伶俐。只是父母去世太早,自幼失学。若说他父祖根基,也不是无名少姓人家,有一分家当,只因不幸遭官事没了,无处栖身。老拙念他乃尊旧日相交之情,欲送他来贵宫作一徒弟,未知尊意如何?”任道士便道:“老居士分付,小道怎敢违阻?奈因小道命蹇,手下虽有两三个徒弟,都不省事,没一个成立的,小道常时惹气,未知此人诚实不诚实?”杏庵道:“这个小的,不瞒尊师说,只顾放心,一味老实本分,胆儿又小,所事儿伶范,堪可作一徒弟。”任道士问:“几时送来?”杏庵道:“见在山门外伺候。还有些薄礼,伏乞笑纳。”慌的任道士道:“老居干何不早说?”一面道:“有请。”于是抬盒人抬进礼物。任道士见帖儿上写着:“谨具粗段一端,鲁酒一樽,豚蹄一副,烧鸭二只,树果二盒,白金五两。知生王宣顿首拜。”连忙稽首谢道:“老居士何以见赐许多重礼,使小道却之不恭,受之有愧。”

只见陈敬济头戴金梁道髻,身穿青绢道衣,脚下云履净袜,腰系丝绦,生的眉清目秀,齿白唇红,面如傅粉,走进来向任道士倒身下拜,拜了四双八拜。任道士因问他:“多少青春?”敬济道:“属马,交新春二十四岁了。”任道士见他果然伶俐,取了他个法名,叫做陈宗美。原来任道士手下有两个徒弟,大徒弟姓金,名宗明;二徒弟姓徐,名宗顺。他便叫陈宗美。王杏庵都请出来,见了礼数。一面收了礼物,小童掌上灯来,放卓儿,先摆饭,后吃酒。肴品杯盘,堆满桌上,无非是鸡蹄鹅鸭鱼肉之类。王老吃不多酒,徒弟轮番劝勾几巡,王老不胜酒力告辞。房中自有床铺,安歇一宿。

到次日清晨,小童舀水净面,梳洗盥漱毕,任道士又早来递茶。不一时,摆饭,又吃了两杯酒,喂饱头口,与了抬盒人力钱。王老临起身,叫过敬济来分付:“在此好生用心习学经典,听师父指教。我常来看你,按季送衣服鞋袜来与你。”又向任道士说:“他若不听教训,一任责治,老拙并不护短。”一面背地又嘱付敬济:“我去后,你要洗心改正,习本等事业。你若再不安分,我不管你了。”那敬济应诺道:“儿子理会了。”王老当下作辞任道士,出门上马,离晏公庙,回家去了。

敬济自此就在晏公庙做了道士。因见任道士年老赤鼻,身体魁伟,声音洪亮,一部髭髯,能谈善饮,只专迎宾送客。凡一应大小事,都在大徒弟金宗明手里。那时,朝廷运河初开,临清设二闸,以节水利。不拘官民,船到闸上,都来庙里,或求神福,或来祭愿,或设卦与笤,或做好事。也有布施钱米的,也有馈送香油纸烛的,也有留松蒿芦席的。这任道士将常署里多余钱粮,都令家下徒弟在马头上开设钱米铺,卖将银子来,积攒私囊。

他这大徒弟金宗明,也不是个守本分的。年约三十余岁,常在娼楼包占乐妇,是个酒色之徒。手下也有两个清洁年少徒弟,同铺歇卧,日久絮繁。因见敬济生的齿白唇红,面如傅粉,清俊乖觉,眼里说话,就缠他同房居住。晚夕和他吃半夜酒,把他灌醉了,在一铺歇卧。初时两头睡,便嫌敬济脚臭,叫过一个枕头上睡。睡不多回,又说他口气喷着,令他吊转身子,屁股贴着肚子。那敬济推睡着,不理他。他把那话弄得硬硬的,直竖一条棍,抹了些唾津在头上,往他粪门里只一顶。原来敬济在冷铺里,被花子飞天鬼侯林儿弄过的,眼子大了,那话不觉就进去了。这敬济口中不言,心内暗道:“这厮合败。他讨得十方便宜多了,把我不知当做甚么人儿。与他个甜头儿,且教他在我手内纳些钱钞。”一面故意声叫起来。这金宗明恐怕老道士听见,连忙掩住他口,说:“好兄弟,噤声!随你要的,我都依你。”敬济道:“你既要勾搭我,我不言语,须依我三件事。”宗明道:“好兄弟,休说三件,就是十件事,我也依你。”敬济道:“第一件,你既要我,不许你再和那两个徒弟睡;第二件,大小房门钥匙,我要执掌;第三件,随我往那里去,你休嗔我。你都依了我,我方依你此事。”金宗明道:“这个不打紧,我都依你。”当夜两个颠来倒去,整狂了半夜。这陈敬济自幼风月中撞,甚么事不知道。当下被底山盟,枕边海誓,淫声艳语,抠吮舔品,把这金宗明哄得欢喜无尽。到第二日,果然把各处钥匙都交与他手内,就不和那两个徒弟在一处,每日只同他一铺歇卧。

一日两,两日三,这金宗明便再三称赞他老实。任道士听信,又替他使钱讨了一张度牒。自此以后,凡事并不防范。这陈敬济因此常拿着银钱往马头上游玩,看见院中架儿陈三儿说:“冯金宝儿他鸨子死了,他又卖在郑家,叫郑金宝儿。如今又在大酒楼上赶趁哩,你不看他看去?”这小伙儿旧情不改,拿着银钱,跟定陈三儿,径往马头大酒楼上来。此不来倒好,若来,正是:五百载冤家来聚会,数年前姻眷又相逢。有诗为证:

\[
人生莫惜金缕衣,人生莫负少年时。
有花欲折须当折,莫待无花空折枝。
\]

原来这座酒楼乃是临清第一座酒楼,名唤谢家酒楼。里面有百十座阁儿,周围都是绿栏杆,就紧靠着山冈,前临官河,极是人烟闹热去处,舟船往来之所。怎见得这座酒楼齐整?但见:

\[
雕檐映日,面栋飞云。绿栏杆低接轩窗,翠帘栊高悬户牖。吹笙品笛,尽都是公子王孙;执盏擎杯,摆列着歌妪舞女。消磨醉眼,依青天万叠云山;勾惹吟魂,翻瑞雪一河烟水。楼畔绿杨啼野鸟,门前翠柳系花骢。
\]
这陈三儿引敬济上楼,到一个阁儿里坐下。便叫店小二打抹春台,安排一分上品酒果下饭来摆着,使他下边叫粉头去了。须臾,只见楼梯响,冯金宝上来,手中拿着个厮锣儿,见了敬济,深深道了万福。常言情人见情人,不觉簇地两行泪下。正是:

\[
数声娇语如莺啭,一串珍珠落线买。
\]
敬济一见,便拉他一处坐,问道:“姐姐,你一向在那里来?不见你。”这冯金宝收泪道:“自从县中打断出来,我妈着了惊谎,不久得病死了,把我卖在郑五妈家。这两日子弟稀少,不免又来在临清马头上赶趁酒客。昨日听见陈三儿说你在这里开钱铺,要见你一见。不期今日会见一面。可不想杀我也!”说毕,又哭了。敬济取出袖中帕儿,替他抹了眼泪,说道:“我的姐姐,你休烦恼。我如今又好了,自从打出官司来,家业都没了,投在这晏公庙,做了道士。师父甚是托我,往后我常来看你。”因问:“你如今在那里安下?”金宝便道:“奴就在这桥西洒家店刘二那里。有百十房子,四外行院窠子,妓女都在那里安下,白日里便是这各酒楼赶趁。”说着,两个挨身做一处饮酒。陈三儿烫酒上楼,拿过琵琶来。金宝弹唱了个曲儿与敬济下酒,名《普天乐》:

\[
泪双垂,垂双泪。三杯别酒,别酒三杯。鸾凤对拆开,折开鸾凤对。岭外斜晖看看坠,看看坠,岭外晖。天昏地暗,徘徊不舍,不舍徘徊。
\]

两人吃得酒浓时,朱免解衣云雨,下个房儿。这陈敬济一向不曾近妇女,久渴的人,今得遇金宝,尽力盘桓,尤云殢雨,未肯即休。须臾事毕,各整衣衫。敬济见天色晚了,与金宝作别,与了金宝一两银子,与了陈三儿百文铜钱,嘱付:“姐姐,我常来看你,咱在这搭儿里相会。你若想我,使陈三儿叫我去。”下楼来,又打发了店主人谢三郎三钱银子酒钱。敬济回庙中去了。冯金宝送至桥边方回。正是:

\[
盼穿秋水因钱钞,哭损花容为邓通。
\]

\newpage
%# -*- coding:utf-8 -*-
%%%%%%%%%%%%%%%%%%%%%%%%%%%%%%%%%%%%%%%%%%%%%%%%%%%%%%%%%%%%%%%%%%%%%%%%%%%%%%%%%%%%%


\chapter{大酒楼刘二撒泼\KG 洒家店雪娥为娼}


诗曰:

\[
骨肉伤残产业荒,一身何忍去归娼。
泪垂玉箸辞官舍,步蹴金莲入教坊。
览镜自怜倾国色,向人初学倚门妆。
春来雨露宽如海,嫁得刘郎胜阮郎。
\]

话说陈敬济自从谢家酒楼上见了冯金宝,两个又勾搭上前情。往后没三日不和他相会,或一日敬济有事不去,金宝就使陈三儿稍寄物事,或写情书来叫他去。一次或五钱,或一两。以后日间供其柴米,纳其房钱。归到庙中便脸红。任道士问他何处吃酒来,敬济只说:“在米铺和伙计畅饮三杯,解辛苦来。”他师兄金宗明一力替他遮掩,晚夕和他一处盘弄那勾当,是不必说。朝来暮往,把任道士囊箧中细软的本钱,也抵盗出大半花费了。

一日,也是合当有事。这洒家店的刘二,有名坐地虎,他是帅府周守备府中亲随张胜的小舅子,专一在马头上开娼店,倚强凌弱,举放私债,与巢窝中各娼使用,加三讨利。有一不给,捣换文书,将利作本,利上加利。嗜酒行凶,人不敢惹他。就是打粉头的班头,欺酒客的领袖。因见陈敬济是宴公庙任道士的徒弟,白脸小厮,谢三家大酒上把粉头郑金宝儿占住了,吃的楞楞睁睁,提着碗头大的拳头,走来谢家楼下,问:“金宝在那里?”慌的谢三郎连忙声喏,说道:“刘二叔叔,他在楼上第二间阁儿里便是。”这刘二大叉步上楼来。敬济正与金宝在阁儿里面饮酒,做一处快活,把房门关闭,外边帘子挂着。被刘二一把手扯下帘子,大叫:“金宝儿出来!”唬的陈敬济鼻口内气儿也不敢出。这刘二用脚把门跺开,金宝儿只得出来相见,说:“刘二叔叔,有何说话?”刘二骂道:“贼淫妇,你少我三个月房钱,却躲在这里,就不去了。”金宝笑嘻嘻说道:“二叔叔,你家去,我使妈妈就送房钱来。”这刘二只搂心一拳,打了老婆一交,把头颅抢在阶沿下磕破,血流满地,骂道:“贼淫妇,还等甚送来,我如今就要!”看见陈敬济在里面,走向前把桌子只一掀,碟儿打得粉碎。那敬济便道:“阿呀,你是甚么人?走来撒野。”刘二骂道:“我\textuni{34B2}你道士秫秫娘!”一手采过头发来,按在地下,拳捶脚踢无数。那楼上吃酒的人,看着都立睁了。店主人谢三初时见刘二醉了,不敢惹他,次后见打得人不像模样,上楼来解劝,说道:“刘二叔,你老人家息怒。他不晓得你老人家大名,误言冲撞,休要和他一般见识,看小人薄面,饶他去罢。”这刘二那里依从,尽力把敬济打了个发昏章第十一。叫将地方保甲,一条绳子,连粉头都拴在一处墩锁,分付:“天明早解到老爷府里去。”原来守备敕书上命他保障地方,巡捕盗贼,兼管河道。这里拿了敬济,任道士庙中尚还不知,只说晚夕米铺中上宿未回。

却说次日,地方保甲、巡河快手押解敬济、金宝,雇头口赶清晨早到府前伺候。先递手本与两个管事张胜、李安看,说是刘二叔地方喧闹一起,宴公庙道士一名陈宗美,娼妇郑金宝。众军牢都问他要钱,说道:“俺们是厅上动刑的,一班十二人,随你罢。正经两位管事的,你倒不可轻视了他。”敬济道:“身边银钱倒有,都被夜晚刘二打我时,被人掏摸的去了。身上衣服都扯碎了,那得钱来?止有头上关顶一根银簪儿,拔下来,与二位管事的罢。”众牢子拿着那根簪子,走来对张胜、李安如此这般说:“他一个钱儿不拿出来,止与了这根簪儿,还是闹银的。”张胜道:“你叫他近前,等我审问他。”众军牢不一时拥到跟前跪下,问:“你几时与任道士做徒弟?俗名叫甚么?我从未见你。”敬济道:“小的俗名叫陈敬济,原是好人家儿女,做道士不久。”张胜道:“你既做道士,便该习学经典,许你在外宿娼饮酒喧嚷?你把俺帅府衙门当甚么些小衙门,不拿了钱儿来,这根簪子打水不浑,要他做甚?”还掠与他去。分付牢子:“等住回老爷升厅,把他放在头一起。眼见这狗男女道士,就是个吝钱的,只许你白要四方施主钱粮!休说你为官事,你就来吃酒赴席,也带方汗巾儿揩嘴。等动刑时,着实加力拶打这厮。”又把郑金宝叫上去。郑家有忘八跟着,上下打发了三四两银子。张胜说:“你系娼门,不过趁熟赶些衣食为生,没甚大事。看老爷喜怒不同,看恼只是一两拶子;若喜欢,只恁放出来也不知。”不一时,只见里面云板响,守备升厅,两边僚掾军牢森列,甚是齐整。但见:

\[
绯罗缴壁,紫绶桌围。当厅额挂茜罗,四下帘垂翡翠。勘官守正,戒石上刻御制四行;人从谨廉,鹿角旁插令旗两面。军牢沉重,僚掾威仪。执大棍授事立阶前,挟文书厅旁听发放。虽然一路帅臣,果是满堂神道。
\]

当时,没巧不成话,也是五百劫冤家聚会,姻缘合当凑着。春梅在府中,从去岁八月间,已生了个哥儿小衙内。今方半岁光景,貌如冠玉,唇若涂朱。守备喜似席上之珍,爱如无价之宝。未几,大奶奶下世,守备就把春梅册正,做了夫人。就住着五间正房,买了两个养娘抱奶哥儿,一名玉堂,一名金匮;两个小丫鬟服侍,一名翠花,一名兰花;又有两个身边得宠弹唱的姐儿,都十六七岁,一名海棠,一名月桂,都在春梅房中侍奉。那孙二娘房中止使着一个丫鬟,名唤荷花儿,不在话下。每常这小衙内,只要张胜抱他外边顽耍,遇着守备升厅,便在旁边观看。

当日,守备升厅坐下,放了告牌出去,各地方解进人来。头一起就叫上陈敬济并娼妇郑金宝儿去。守备看了呈状,便说道:“你这厮是个道士,如何不守清规,宿娼饮酒,骚扰地方,行止有亏。左右拿下去,打二十棍,追了度牒还俗。那娼妇郑氏,拶一拶,敲五十敲,责令归院当差。”两边军牢向前,才待扯翻敬济,摊去衣服,用绳索绑起,转起棍来,两边招呼要打时,可霎作怪,张胜抱着小衙内,正在月台上站立观看,那小衙内看见打敬济,便在怀里拦不住,扑着要敬济抱。张胜恐怕守备看见,忙走过来。那小衙内亦发大哭起来,直哭到后边春梅跟前。春梅问:“他怎的哭?”张胜便说:“老爷厅上发放事,打那宴公庙陈道士,他就扑着要他抱,小的走下来,他就哭了。”

这春梅听见是姓陈的,不免轻移莲步,款蹙湘裙,走到软屏后面探头观觑:“打的那人,声音模样,倒好似陈姐夫一般,他因何出家做了道士?”又叫过张胜,问他:“此人姓甚名谁?”张胜道:“这道士我曾问他来,他说俗名叫陈敬济。”春梅暗道:“正是他了。”一面使张胜:“请下你老爷来。”这守备厅上打敬济才打到十棍,一边还拶着唱的,忽听后边夫人有请,分付牢子把棍且阁住休打,一面走下厅来。春梅说道:“你打的那道士,是我姑表兄弟,看奴面上,饶了他罢。”守备道:“夫人何不早说,我已打了他十棍,怎生奈何?”一面出来,分付牢子:“都与我放了。”唱的便归院去了。守备悄悄使张胜:“叫那道士回来,且休去。问了你奶奶,请他相见。”这春梅才待使张胜请他到后堂相见,忽然沉吟想了一想,便又分付张胜:“你且叫那人去着,待我慢慢再叫他。”度牒也不曾追。

这陈敬济打了十棍,出离了守备府,还奔来晏公庙。不想任道士听见人来说:“你那徒弟陈宗美,在大酒楼上包着唱的郑金宝儿,惹了洒家店坐地虎刘二,打得臭死,连老婆都拴了,解到守备府去了。行止有亏,便差军牢来拿你去审问,追度牒还官。”这任道士听了,一者老年的着了惊怕,二来身体胖大,因打开囊箧,内又没有许多细软东西,着了口重气,心中痰涌上来,昏倒在地。众徒弟慌忙向前扶救,请将医者来灌下药去,通不省人事。到半夜,呜呼断气身亡。亡年六十三岁。第二日,陈敬济来到,左右邻人说:“你还敢庙里去?你师父因为你,如此这般,得了口重气,昨夜三更鼓死了。”这敬济听了,唬的忙忙似丧家之犬,急急如漏网之鱼,复回清河县城中来。正是:

\[
鹿随郑相应难辩,蝶化庄周未可知。
\]

话分两头。却说春梅一面使张胜叫敬济且去着,一面走归房中,摘了冠儿,脱了绣服,倒在床上,便扪心挝被,声疼叫唤起来。唬的合宅大小都慌了。下房孙二娘来问道:“大奶奶才好好的,怎的就不好起来?”春梅说:“你每且去,休管我。”落后守备退厅进来,见他躺在床上叫唤,也慌了。扯着他手儿问道:“你心里怎的来?”也不言语,又问:“那个惹着你来?”也不做声。守备道:“不是我刚才打了你兄弟,你心内恼么?”亦不应答。这守备无计奈何,走出外边麻犯起张胜、李安来了:“你两个早知他是你奶奶兄弟,如何不早对我说?却教我打了他十下,惹的你奶奶心中不自在。我曾教你留下他,请你奶奶相见,你如何又放他去了?你这厮每却讨分晓!”张胜说:“小的曾禀过奶奶来,奶奶说且教他去着,小的才放他去了。”一面走入房中,哭哭啼啼,哀告春梅:“望乞奶奶在爷前方便一言。不然,爷要见责小的每哩。”这春梅睁圆星眼,剔起蛾眉,叫过守备近前说:“我自心中不好,干他们甚事?那厮他不守本分,在外边做道士,且奈他些时,等我慢慢招认他。”这守备才不麻犯张胜、李安了。

守备见他只管声唤,又使张胜请下医官来看脉,说:“老安人染了六欲七情之病,着了重气在心。”讨将药来又不吃,都放冷了。丫头每都不敢向前说话,请将守备来看着吃药,只呷了一口,就不吃了。守备出去了,大丫鬟月桂拿过药来,“请奶奶吃药。”被春梅拿过来,匹脸只一泼,骂道:“贼浪奴才,你只顾拿这苦水来灌我怎的?我肚子里有甚么?”教他跪在面前。孙二娘走来,问道:“月桂怎的?奶奶教他跪着。”海棠道:“奶奶因他拿药与奶奶吃来,奶奶说:‘我肚子里有甚么?拿这药来灌我。’教他跪着。”孙二娘道:“奶奶,你委的今一日没曾吃甚么。这月桂他不晓得,奶奶休打他,看我面上,饶他这遭罢。”分付海棠:“你往厨下熬些粥儿来,与你奶奶吃口儿。”春梅于是把月桂放起来。

那海棠走到厨下,用心用意熬了一小锅粳米浓浓的粥儿,定了四碟小菜儿,用瓯儿盛着,热烘烘拿到房中。春梅躺在床上面朝里睡,又不敢叫,直待他番身,方才请他:“有了粥儿在此,请奶奶吃粥。”春梅把眼合着,不言语。海棠又叫道:“粥晾冷了,请奶奶起来吃粥。”孙二娘在旁说道:“大奶奶,你这半日没吃甚么,这回你觉好些,且起来吃些个。”那春梅一骨碌子扒起来,教奶子拿过灯来,取粥在手,只呷了一口,往地下只一推。早是不曾把家伙打碎,被奶子接住了。就大吆喝起来,向孙二娘说:“你平白叫我起来吃粥,你看贼奴才熬的好粥!我又不坐月子,熬这照面汤来与我吃怎么?”分付奶子金匮:“你与我把这奴才脸上打与他四个嘴巴!”当下真个把海棠打了四个嘴巴。孙二娘便道:“奶奶,你不吃粥,却吃些甚么儿?却不饿着你。”春梅道:“你教我吃,我心内拦着,吃不下去。”良久,叫过小丫鬟兰花儿来,分付道:“我心内想些鸡尖汤儿吃。你去厨房内,对那淫妇奴才,教他洗手做碗好鸡尖汤儿与我吃。教他多放些酸笋,做的酸酸辣辣的我吃。”孙二娘便说:“奶奶分付他,教雪娥做去。你心下想吃的就是药。”

这兰花不敢怠慢,走到厨下对雪娥说:“奶奶教你做鸡尖汤,快些做,等着要吃哩。”原来这鸡尖汤,是雏鸡脯翅的尖儿碎切的做成汤。这雪娥一面洗手剔甲,旋宰了两只小鸡,退刷干净,剔选翅尖,用快刀碎切成丝,加上椒料、葱花、芫荽、酸笋、油酱之类,揭成清汤。盛了两瓯儿,用红漆盘儿,热腾腾,兰花拿到房中。春梅灯下看了,呷了一口,怪叫大骂起来:“你对那淫妇奴才说去,做的甚么汤!精水寡淡,有些甚味?你们只教我吃,平白叫我惹气!”慌的兰花生怕打,连忙走到厨下对雪娥说:“奶奶嫌汤淡,好不骂哩。”这雪娥一声儿不言语,忍气吞声,从新洗锅,又做了一碗。多加了些椒料,香喷喷,教兰花儿拿到房里来。春梅又嫌忒咸了,拿起来照地下只一泼,早是兰花躲得快,险些儿泼了一身。骂道:“你对那奴才说去,他不愤气做与我吃。这遭做的不好,教他讨分晓。”这雪娥听见,千不合,万不合,悄悄说了一句:“姐姐几时这般大了,就抖搂起人来!”不想兰花回到房里,告春梅说了。这春梅不听便罢,听了此言,登时柳眉剔竖,星眼圆睁,咬碎银牙,通红了粉面,大叫:“与我采将那淫妇奴才来!”

须臾,使了奶娘丫鬟三四个,登时把雪娥拉到房中。春梅气狠狠的一手扯住他头发,把头上冠子跺了,骂道:“淫妇奴才,你怎的说几时这般大?不是你西门庆家抬举的我这般大!我买将你来伏侍我,你不愤气,教你做口子汤,不是精淡,就是苦咸。你倒还对着丫头说我几时恁般大起来,搂搜索落我,要你何用?”一面请将守备来,采雪娥出去,当天井跪着。前边叫将张胜、李安,旋剥褪去衣裳,打三十大棍。两边家人点起明晃晃灯笼,张胜、李安各执大棍伺候。那雪娥只是不肯脱衣裳。守备恐怕气了他,在跟前不敢言语。孙二娘在旁边再三劝道:“随大奶奶分付打他多少,免褪他小衣罢。不争对着下人,脱去他衣服,他爷体面上不好看的。只望奶奶高抬贵手,委的他的不是了。”春梅不肯,定要去他衣服打,说道:“那个拦我,我把孩子先摔杀了,然后我也一条绳子吊死就是了。留着他便是了。”于是也不打了,一头撞倒在地,就直挺挺的昏迷,不省人事。守备唬的连忙扶起,说道:“随你打罢,没的气着你。”当下可怜把这孙雪娥拖番在地,褪去衣服,打了三十大棍,打的皮开肉绽。一面使小牢子半夜叫将薛嫂儿来,即时罄身领出去办卖。

春梅把薛嫂儿叫在背地,分付:“我只要八两银子,将这淫妇奴才好歹与我卖在娼门。随你转多少,我不管你。你若卖在别处,我打听出来,只休要见我。”那薛嫂儿道:“我靠那里过日子,却不依你说?”当夜领了雪娥来家。那雪娥悲悲切切,整哭到天明。薛嫂便劝道:“你休哭了,也是你的晦气,冤家撞在一处。老爷见你到罢了,只恨你与他有些旧仇旧恨,折挫你。连老爷也做不得主儿,见他有孩子,凡事依随他。正经下边孙二娘也让他几分。常言拐米倒做了仓官,说不的了,你休气哭。”雪娥收泪,谢薛嫂:“只望早晚寻个好头脑我去,只有饭吃罢。”薛嫂道:“他千万分付,只教我把你送在娼门。我养儿养女,也要天理。等我替你寻个单夫独妻,或嫁个小本经纪人家,养活得你来也罢。”那雪娥千恩万福谢了。

薛嫂过了两日,只见邻居一个开店张妈走来叫:“薛妈,你这壁厢有甚娘子?怎的哭的悲切?”薛嫂便道:“张妈,请进来坐。”说道:“便是这位娘子,他是大人家出来的,因和大娘子合不着,打发出来,在我这里嫁人。情愿个单夫独妻,免得惹气。”张妈妈道:“我那边下着一个山东卖绵花客人,姓潘,排行第五,年三十七岁,几车花果,常在老身家安下。前日说他家有个老母有病,七十多岁,死了浑家半年光景,没人伏侍。再三和我说,替他保头亲事,并无相巧的。我看来这位娘子年纪到相当,嫁与他做个娘子罢。”薛嫂道:“不瞒你老人家说,这位娘子大人家出身,不拘粗细都做的,针指女工,自不必说,又做的好汤水。今才三十五岁。本家只要三十两银子,倒好保与他罢。”张妈妈道:“有箱笼没有?”薛嫂道:“止是他随身衣服、簪环之类,并无箱笼。”张妈妈道:“既是如此,老身回去对那人说,教他自家来看一看。”说毕,吃茶,坐回去了。晚夕对那人说了,次日饭罢以后,果然领那人来相看。一见了雪娥好模样儿,年小,一口就还了二十五两,另外与薛嫂一两媒人钱。薛嫂也没争竞,就兑了银子,写了文书。晚夕过去,次日就上车起身。薛嫂教人改换了文书,只兑了八两银子交到府中,春梅收了,只说卖与娼门去了。

那人娶雪娥到张妈家,止过得一夜,到第二日,五更时分,谢了张妈妈,作别上了车,径到临清去了。此是六月天气,日子长,到马头上才日西时分。到于洒家店,那里有百十间房子,都下着各处远方来的窠子行院唱的。这雪娥一领入一个门户,半间房子,里面炕上坐着个五六十岁的婆子,还有个十七顶老丫头,打着盘头揸髻,抹着铅粉红唇,穿着一弄儿软绢衣服,在炕边上弹弄琵琶。这雪娥看见,只叫得苦,才知道那汉子潘五是个水客。买他来做粉头。起了他个名叫玉儿。这小妮子名唤金儿,每日拿厮锣儿出去,酒楼上接客供唱,做这道路营生。这潘五进门不问长短,把雪娥先打了一顿,睡了两日,只与他两碗饭吃,教他学乐器弹唱,学不会又打,打得身上青红遍了。引上道儿,方与他好衣穿,妆点打扮,门前站立,倚门献笑,眉目嘲人。正是:遗踪堪入府人眼,不买胭脂画牡丹。有诗为证:

\[
穷途无奔更无投,南去北来休更休。
一夜彩云何处散,梦随明月到青楼。
\]

这雪娥在洒家店,也是天假其便。一日,张胜被守备差遣往河下买几十石酒曲,宅中造酒。这洒家店坐地虎刘二,看见他姐夫来,连忙打扫酒楼干净,在上等阁儿里安排酒肴杯盘,请张胜坐在上面饮酒。酒博士保儿筛酒,禀问:“二叔,下边叫那几个唱的上来递酒?”刘二分付:“叫王家老姐儿,赵家娇儿,潘家金儿,玉儿四个上来,伏侍你张姑夫。”酒博士保儿应诺下楼。不多时,只听得胡梯畔笑声儿,一般儿四个唱的,打扮得如花似朵,都穿着轻纱软绢衣裳,上的楼来,望上拜了四拜,立在旁边。这张胜猛睁眼观看,内中一个粉头,可霎作怪,“到相老爷宅里打发出来的那雪娥娘子。他如何做这道路在这里?”那雪娥亦眉眼扫见是张胜,都不做声。这张胜便问刘二:“那个粉头是谁家的?”刘二道:“不瞒姐夫,他是潘五屋里玉儿、金儿,这个是王老姐,一个是赵娇儿。”张胜道:“这潘家玉儿,我有些眼熟。”因叫他近前,悄悄问他:“你莫不是雪姑娘么?怎生到于此处?”那雪娥听见他问,便簇地两行泪下,便道:“一言难尽。”如此这般,具说一遍。“被薛嫂撺瞒,把我卖了二十五两银子,卖在这里供筵席唱,接客迎人。”这张胜平昔见他生的好,常是怀心。这雪娥席前殷勤劝酒,两个说得入港。雪娥和金儿不免拿过琵琶来,唱个词儿,与张胜下酒。唱毕,彼此穿杯换盏,倚翠偎红,吃得酒浓时,常言:“世财红粉歌楼酒,谁为三般事不迷?”这张胜就把雪娥来爱了。两个晚夕留在阁儿里,就一处睡了。这雪娥枕边风月,耳畔山盟,和张胜尽力盘桓,如鱼似水,百般难述。次日起来,梳洗了头面,刘二又早安排酒肴上来,与他姐夫扶头。大盘大碗,饕食一顿,收起行装,喂饱头口,装载米曲,伴当跟随。临出门,与了雪娥三两银子,分付刘二:“好生看顾他,休教人欺负。”自此以后,张胜但来河下,就在洒家店与雪娥相会。往后走来走去,每月与潘五几两银子,就包住了他,不许接人。那刘二自恁要图他姐夫欢喜,连房钱也不问他要了。各窠窝刮刷将来,替张胜出包钱,包定雪娥柴米。有诗为证:

\[
岂料当年纵意为,贪淫倚势把心欺。
祸不寻人人自取,色不迷人人自迷。
\]

\newpage
%# -*- coding:utf-8 -*-
%%%%%%%%%%%%%%%%%%%%%%%%%%%%%%%%%%%%%%%%%%%%%%%%%%%%%%%%%%%%%%%%%%%%%%%%%%%%%%%%%%%%%


\chapter{玳安儿窃玉成婚\KG 吴典恩负心被辱}


诗曰:

\[
寺废僧居少,桥滩客过稀。家贫奴负主,官懦吏相欺。
水浅鱼难住,林稀鸟不栖。人情皆若此,徒堪悲复凄。
\]

话说孙雪娥在洒家店为娼,不题。却说吴月娘,自从大姐死了,告了陈敬济一状,大家人来昭也死了,他妻子一丈青带着小铁棍儿,也嫁人去了。来兴儿看守门户,房中绣春,与了王姑子做徒弟,出家去了。那来兴儿自从他媳妇惠秀死了,一向没有妻室。奶子如意儿,要便引着孝哥儿在他屋里顽耍,吃东西。来兴儿又打酒和奶子吃,两个嘲勾来去,就刮剌上了,非止一日。但来前边,归入后边就脸红。月娘察知其事,骂了一顿。家丑不可外扬,与了他一套衣裳,四根簪子,拣了个好日子,就与来兴儿完房,做了媳妇了。白日上灶看哥儿,后边扶持,到夜间往前边他屋里睡去。

一日,八月十五日,月娘生日。有吴大妗、二妗子,并三个姑子,都来与月娘做生日,在后边堂屋里吃酒。晚夕,都在孟玉楼住的厢房内听宣卷。到二更时分,中秋儿便在后边灶上看茶,由着月娘叫,都不应。月娘亲自走到上房里,只见玳安儿正按着小玉在炕上干得好。看见月娘推门进来,慌的凑手脚不迭。月娘便一声儿也没言语,只说得一声:“臭肉儿,不在后边看茶去,且在这里做甚么哩。”那小玉道:“我叫中秋儿灶上顿茶哩。”低着头,往后边去了。玳安便走出仪门,往前边来。

过了两日,大妗子、二妗子,三个女僧都家去了。这月娘把来兴儿房腾出收拾了,与玳安住。却教来兴儿搬到来昭屋里,看守大门去了。替玳安做了两床铺盖,一身装新衣服,盔了一顶新网新帽,做了双新靴袜;又替小玉编了一顶\textuni{4BFC}髻,与了他几件金银首饰,四根金头银脚簪,环坠戒指之类,两套段绢衣服,择日就配与玳安儿做了媳妇。白日里还进来在房中答应,只晚夕临关仪门时便出去和玳安歇去。这丫头拣好东好西,甚么不拿出来和玳安吃?这月娘当看见只推不看见。常言道:“溺爱者不明,贪得者无厌”,“羊酒不均,驷马奔镇”,“处家不正,奴婢抱怨”。

却说平安儿见月娘把小玉配与玳安,衣服穿戴胜似别人。他比玳安倒大两岁,今年二十二岁,倒不与他妻室。一日在假当铺,看见傅伙计当了人家一副金头面,一柄镀金钩子,当了三十两银子。那家只把银子使了一个月,加了利钱就来赎讨。傅伙计同玳安寻取来,放在铺子大橱柜里。不提防这平安儿见财起心,就连匣儿偷了,走去南瓦子里武长脚家——有两个私窠子,一个叫薛存儿,一个叫伴儿,在那里歇了两夜。忘八见他使钱儿猛大,匣子蹙着金头面,撅着银挺子打酒买东西。报与土番,就把他截在屋里,打了两个耳刮子就拿了。

也是合当有事,不想吴典恩新升巡简,骑着马,头里打着一对板子,正从街上过来,看见,问:“拴的甚么人?”土番跪下禀说:“如此这般,拐带出来瓦子里宿娼,拿金银头面行使。小的可疑,拿了。”吴典恩分付:“与我带来审问。”一面拿到巡简厅儿内。吴典恩坐下,两边弓皂排列。土番拴平安儿到根前,认的是吴典恩当初是他家伙计:“已定见了我就放的。”开口就说:“小的是西门庆家平安儿。”吴典恩说:“你既是他家人,拿这金东西在这坊子里做甚么?”平安道:“小的大娘借与亲戚家头面戴,使小的敢去,来晚了,城门闭了,小的投在坊子,权借宿一夜,不料被土番拿了。”吴典恩骂道:“你这奴才,胡说!你家这般头面多,金银广,教你这奴才把头面拿出来老婆家歇宿行使?想必是你偷盗出来的。趁早说来,免我动刑!”平安道:“委的亲戚家借去头面,家中大娘使我讨去来,并不敢说谎。”吴典恩大怒,骂道:“此奴才真贼,不打如何肯认?”喝令左右:“与我拿夹棍夹这奴才!”一面套上夹棍,夹的小厮犹如杀猪叫,叫道:“爷休夹小的,等小的实说了罢。”吴典恩道:“你只实说,我就不夹你。”平安儿道:“小的偷的假当铺当的人家一副金头面,一柄镀金银子。”吴典恩问道:“你因甚么偷出来?”平安道:“小的今年二十二岁,大娘许了替小的娶媳妇儿,不替小的娶。家中使的玳安儿小厮才二十岁,倒把房里丫头配与他,完了房。小的因此不愤,才偷出假当铺这头面走了。”吴典恩道:“想必是这玳安儿小厮与吴氏有奸,才先把丫头与他配了。你只实说,没你的事,我便饶了你。”平安儿道:“小的不知道。”吴典恩道:“你不实说,与我拶起来。”左右套上拶子,慌的平安儿没口子说道:“爷休拶小的,等小的说就是了。”吴典恩道:“可又来,你只说了,须没你的事。”一面放了拶子。那平安说:“委的俺大娘与玳安儿有奸。先要了小玉丫头,俺大娘看见了,就没言语,倒与了他许多衣服首饰东西,配与他完房。”这吴典恩一面令吏典上来,抄了他口词,取了供状,把平安监在巡简司,等着出牌,提吴氏、玳安、小玉来,审问这件事。

那日,却说解当铺橱柜里不见了头面,把傅伙计唬慌了。问玳安,玳安说:“我在生药铺子里吃饭,我不知道。”傅伙计道:“我把头面匣子放在橱里,如何不见了?”一地里寻平安儿寻不着,急的傅伙计插香赌誓。那家子讨头面,傅伙计只推还没寻出来哩。那人走了几遍,见没有头面,只顾在门前嚷闹,说:“我当了一个月,本利不少你的,你如何不与我?头面、钩子值七八十两银子。”傅伙计见平安儿一夜不来家,就知是他偷出去了。四下使人找寻不着,那讨头面主儿又在门首嚷乱。对月娘说,赔他五十两银子,那人还不肯,说:“我头面值六十两,钩子连宝石珠子镶嵌共值十两,该赔七十两银子。”傅伙计又添了他十两,还不肯,定要与傅伙计合口。正闹时,有人来报说:“你家平安儿偷了头面,在南瓦子养老婆,被吴巡简拿在监里,还不教人快认赃去!”这吴月娘听见吴典恩做巡简,“是咱家旧伙计。”一面请吴大舅来商议,连忙写了领状,第二日教傅伙计领赃去。有了原物在,省得两家领。

傅伙计拿状子到巡简司,实承望吴典恩看旧时分上,领得头面出来,不想反被吴典恩老狗奴才尽力骂了顿。叫皂隶拉倒要打,褪去衣裳,把屁脱脱了半日,饶放起来,说道:“你家小厮在这里供出吴氏与玳安许多奸情来,我这里申过府县,还要行牌提取吴氏来对证。你这老狗骨头,还敢来领赃!”倒吃他千奴才、万老狗,骂将出来,唬的往家中走不迭。来家不敢隐讳,如此这般,对月娘说了。月娘不听便罢了,听了,正是“分开八块顶梁骨,倾下半桶冰雪来”,慌的手脚麻木。又见那讨头面人,在门前大嚷大闹,说道:“你家不见了我头面,又不与我原物,又不赔我银子,只反哄着我两头来回走。今日哄我去领赃,明日等领头面,端的领的在那里?这等不合理。”那傅伙计赔下情,将好言央及安抚他:“略从容两日,就有头面来了。若无原物,加倍赔你。”那人说:“等我回声当家的去。”说毕去了。

这吴月娘忧上加忧,眉头不展。使小厮请吴大舅来商议,教他寻人情对吴典恩说,掩下这桩事罢。吴大舅说:“只怕他不受人情,要些贿赂打点他。”月娘道:“他当初这官,还是咱家照顾他的,还借咱家一百两银子,文书俺爹也没收他的,今日反恩将仇报起来。”吴大舅说:“姐姐,说不的那话了。从来忘恩背义,才一个儿也怎的?”吴月娘道:“累及哥哥,上紧寻个路儿,宁可送他几十两银子罢。领出头面来还了人家,省得合口费舌。”打发吴大舅吃了饭去了。

月娘送哥哥到大门首,也是合当事情凑巧,只见薛嫂儿提着花箱儿,领着一个小丫头过来。月娘叫住,便问:“老薛,你往那里去?怎的一向不来走走?”薛嫂道:“你老人家到且说的好,这两日好不忙哩。偏有许多头绪儿,咱家小奶奶那里,使牢子大官儿,叫了好几遍,还不得空儿去哩。”月娘道:“你看妈妈了撒风,他又做起俺小奶奶来了。”薛嫂道:、如今不做小奶奶,倒做了大奶奶了。”月娘道:“他怎的倒大奶奶?”薛嫂道:“你老人家还不知道,他好小造化儿!自从生了哥儿,大奶奶死了,守备老爷就把他扶了正房,做了封赠娘子。正经二奶奶孙氏不如他。手下买了两个奶子,四个丫头扶侍。又是两个房里得宠学唱的姐儿,都是老爷收用过的。要打时就打,老爷敢做主儿?自恁还恐怕气了他。那日不知因甚么,把雪娥娘子打了一顿,把头发都撏了,半夜叫我去领出来,卖了八两银子。今日我还睡哩,又使牢子叫了我两遍,教我快往宅里去,问我要两副大翠重云子钿儿,又要一副九凤钿儿。先与了我五两银子。银子不知使的那里去了,还没送与他生活去哩。这一见了我,还不知怎生骂我哩。”月娘道:“你到后边,等我瞧瞧怎样翠钿儿。”一面让薛嫂到后边坐下。薛嫂打开花箱,取出与吴月娘看。只见做的好样儿,金翠掩映,背面贴金。那个钿儿,每个凤口内衔着一挂宝珠牌儿,十分奇巧。薛嫂道:“只这副钿儿,做着本钱三两五钱银子;那副重云子的,只一两五钱银子,还没寻他的钱。”

正说着,只见玳安走来,对月娘说:“讨头面的又在前边嚷哩,说等不的领赃,领到几时?若明日没头面,要和傅二叔打了,到个去处理会哩。傅二叔心里不好,往家去了。那人嚷了回去了。”薛嫂问:“是甚么勾当?”月娘便长吁了一口气,如此这般,告诉薛嫂说:“平安儿奴才,偷去印子铺人家当的一副金头面,一副镀金钩子,走在城外坊子里养老婆,被吴巡简拿住,监在监里。人家来讨头面没有,在门前嚷闹。吴巡简又勒掯刁难,不容俺家领赃,又要打将伙计来要钱,白寻不出个头脑来。死了汉子,败落一齐来,就这等被人欺负,好苦也!”说着那眼中泪纷纷落将下来。

薛嫂道:“好奶奶,放着路儿不会寻。咱家小奶奶,你这里写个贴儿,等我对他说声,教老爷差人分付巡简司,莫说一副头面,就十副头面也讨去了。”月娘道:“周守备,他是武职官,怎管的着那巡简司?”薛嫂道:“奶奶,你还不知道,如今周爷,朝廷新与他的敕书,好不管的事情宽广。地方河道,军马钱粮,都在他手里打卯递手本。又河东水西,捉拿强盗贼情,正在他手里。”月娘听了,便道:“既然管着,老薛就累你,多上覆庞大姐说声。一客不烦二主,教他在周爷面前美言一句儿,问巡简司讨出头面来。我破五两银子谢你。”薛嫂道:“好奶奶,钱恁中使。我见你老人家刚才凄惶,我到下意不去。你教人写了帖儿,等我到府里和小奶奶说。成了,随你老人家;不成,我还来回你老人家话。”这吴月娘一面叫小玉摆茶与薛嫂吃。薛嫂儿道:“不吃罢,你只教大官儿写了贴儿来,你不知我一身的事哩。”月娘道:“你也出来这半日了,吃了点心儿去。”小玉即便放卓儿,摆上茶食来。月娘陪他吃茶。薛嫂儿递与丫头两个点心吃。月娘问丫头几岁了,薛嫂道:“今年十二岁了。”不一时,玳安前边写了说贴儿。薛嫂儿吃了茶,放在袖内,作辞月娘,提着花箱出门,径到守备府中。

春梅还在暖床上睡着没起来哩。只见大丫鬟月桂进来说:“老薛来了。”春梅便叫小丫头翠花,把里面窗寮开了。日色照的纱窗十分明亮。薛嫂进来说道:“奶奶,这咱还未起来?”放下花箱,便磕下头去。春梅道:“不当家化化的,磕甚么头?”说道:“我心里不自在,今日起来的迟些。”问道:“你做的翠云子和九凤钿儿拿了来不曾?”薛嫂道:“奶奶,这两副钿儿,好不费手!昨日晚夕我才打翠花铺里讨将来,今日要送来,不想奶奶又使了牢子去。”一面取出来,与春梅过目。春梅还嫌翠云子做的不十分现撇,还放在纸匣儿内,交与月桂收了。看茶与薛嫂儿吃。薛嫂便叫小丫鬟进来,“与奶奶磕头。”春梅问:“是那里的?”薛嫂儿道:“二奶奶和我说了好几遍,说荷花只做的饭,教我替他寻个小孩儿,学做些针指。我替他领了这个孩子来了。到是乡里人家女孩儿,今年才十二岁,正是养材儿。”春梅道:“你亦发替他寻个城里孩子,还伶便些。这乡里孩子,晓的甚么?”因问:“这丫头要多少银子?”薛嫂儿道:“要不多,只四两银子,他老子要投军使。”春梅叫海棠:“你领到二娘房里去,明日兑银子与他罢。”又叫月桂:“大壶内有金华酒,筛来与薛嫂儿烫寒。再有甚点心,拿一盒子与他吃。省得他又说,大清早辰拿寡酒灌他。”

薛嫂道:“桂姐,且不要筛上来,等我和奶奶说了话着,刚才也吃了些甚么来了。”春梅道:“你对我说,在谁家?吃甚来?”薛嫂道:“刚才大娘那头,留我吃了些甚么来了。如此这般,望着我好不哭哩。说平安儿小厮,偷了印子铺内人家当的金头面,还有一把镀金钩子,在外面养老婆,吃番子拿在巡简司拶打。这里人家又要头面嚷乱。那吴巡简旧日是咱那里伙计,有爹在日,照顾他的官。今日一旦反面无恩,夹打小厮,攀扯人,又不容这里领赃。要钱,才把傅伙计打骂将来。唬的伙计不好了,躲的往家去了。央我来,多多上覆你老人家。可怜见,举眼儿无亲的。教你替他对老爷说声,领出头面来,交付与人家去了,大娘亲来拜谢你老人家。”春梅问道:“有个贴儿没有?不打紧,你爷出巡去了,怕不的今晚来家,等我对你爷说。”薛嫂儿道:“他有说贴儿在此。”向袖中取出。春梅看了,顺手就放在窗户台上。

不一时,托盘内拿上四样嗄饭菜蔬,月桂拿大银钟,满满斟了一钟,流沿儿递与薛嫂。薛嫂道:“我的奶奶,我怎捱的这大行货子?”春梅笑道:“比你家老头子那大货差些儿。那个你倒捱了,这个你倒捱不的,好歹与我捱了。要不吃,月桂,你与我捏着鼻子灌他。”薛嫂道:“你且拿了点心,与我打个底儿着。”春梅道:“老妈子,单管说谎。你才说吃了来,这回又说没打底儿。”薛嫂道:“吃了他两个茶食,这咱还有哩?”月桂道:“薛妈妈,你且吃了这大钟酒,我拿点心与你吃。俺奶奶怪我没用,要打我哩。”这薛嫂没奈何,只得灌了一钟,觉心头小鹿儿劈劈跳起来。那春梅努个嘴儿,又叫海棠斟满一钟教他吃。薛嫂推过一边说:“我的那娘,我却一点儿也吃不的了。”海棠道:“你老人家捱一月桂姐一下子,不捱我一下子,奶奶要打我。”那薛嫂儿慌的直撅儿跪在地下。春梅道:“也罢,你拿过那饼与他吃了,教他好吃酒。”月桂道:“薛妈妈,谁似我恁疼你,留下恁好玫瑰馅饼儿与你吃。”就拿过一大盘子顶皮酥玫瑰饼儿来。那薛嫂儿只吃了一个,别的春梅都教他袖在袖子里:“到家稍与你家老王八吃。”薛嫂儿吃了酒,盖着脸儿,把一盘子火薰肉,腌腊鹅,都用草纸包裹,塞在袖内。海棠使气白赖,又灌了半钟酒。见他呕吐上来,才收过家伙,不要他吃了。春梅分付:“明日来讨话说,兑丫头银子与你。”临出门,春梅又分付:“妈妈,你休推聋装哑,那翠云子做的不好,明日另带两副好的我瞧。”薛嫂道:“我知道。奶奶叫个大姐送我送,看狗咬了我腿。”春梅笑道:“俺家狗都有眼,只咬到骨秃根前就住了。”一面使兰花送出角门来。

话休饶舌。周守备至日落时分,出巡来家,进入后厅,左右丫鬟接了冠服。进房见了春梅、小衙内,心中欢喜。坐下,月桂、海棠拿茶吃了,将出巡之事告诉一遍。不一时,放桌儿摆饭。饭罢,掌上烛,安排杯酌饮酒。因问:“前边没甚事?”春梅一面取过薛嫂拿的贴儿来,与守备看,说吴月娘那边,如此这般,“小厮平安儿偷了头面,被吴巡简拿住监禁,不容领赃。只拷打小厮,攀扯诬赖吴氏奸情,索要银两,呈详府县”等事。守备看了说:“此事正是我衙门里事,如何呈详府县?吴巡简那厮这等可恶!我明日出牌,连他都提来发落。”又说:“我闻得吴巡简是他门下伙计,只因往东京与蔡太题进礼,带挈他做了这个官,如何倒要诬害他家!”春梅道:“正是这等说。你替他明日处处罢。”一宿晚景题过。

次日,旋教吴月娘家补了一纸状,当厅出了大花栏批文,用一个封套装了。上批:“山东守御府为失盗事,仰巡简司官连人赃解缴。右差虞侯张胜、李安。准此。”当下二人领出公文来,先到吴月娘家。月娘管待了酒饭,每人与了一两银子鞋脚钱。傅伙计家中睡倒了,吴二舅跟随到巡简司。吴巡简见平安监了两日,不见西门庆家中人来打点,正教吏典做文书,申呈府县。只见守御府中两个公人到了,拿出批文来与他。见封套上朱红笔标着:“仰巡简司官连人解缴”,拆开,见里面吴氏状子,唬慌了。反赔下情,与李安、张胜每人二两银子。随即做文书解人上去。到于守备府前,伺候半日。待的守备升厅,两边军牢排下,然后带进入去。这吴巡简把文书呈递上去,守备看了一遍,说:“此是我衙门里事,如何不申解前来?只顾延捱监滞,显有情弊。”那吴巡简禀道:“小官才待做文书申呈老爷案下,不料老爷钧批到了。”守备喝道:“你这狗官可恶!多大官职?这等欺玩法度,抗违上司!我钦奉朝廷敕命,保障地方,巡捕盗贼,提督军务,兼管河道,职掌开载已明。你如何拿了这件,不行申解,妄用刑杖拷打犯人,诬攀无辜?显有情弊!”那吴巡简听了,摘去冠帽,在阶前只顾磕头。守备道:“本当参治你这狗官,且饶你这遭,下次再若有犯,定行参究。”一面把平安提到厅上,说道:“你这奴才,偷盗了财物,还肆言谤主。人家都是你恁般,也不敢使奴才了。”喝左右:“与我打三十大棍,放了。将赃物封贮,教本家人来领去。”一面唤进吴二舅来,递了领状。守备这里还差张胜拿贴儿同送到西门庆家,见了分上。吴月娘打发张胜酒饭,又与了一两银子。走来府里,回了守备、春梅话。

那吴巡简干拿了平安儿一场,倒折了好几两银子。月娘还了那人家头面、钩子儿。是他原物,一声儿没言语去了。傅伙计到家,伤寒病睡倒了,只七日光景,调治不好,呜呼哀哉死了。月娘见这等合气,把印子铺只是收本钱赎讨,再不解当出银子去了。止是教吴二舅同玳安,在门首生药铺子日逐转得来,家中盘缠。此事表过不题。

一日,吴月娘叫将薛嫂儿来,与了三两银子。薛嫂道:“不要罢,传的府里奶奶怪我。”月娘道:“天不使空人,多有累你,我见他不题出来就是了。”于是买下四盘下饭,宰了一口鲜猪,一坛南酒,一匹纻丝尺头,薛嫂押着来守备府中,致谢春梅。玳安穿着青绢褶儿,拿着礼贴儿,薛嫂领着径到后堂。春梅出来,戴着金梁冠儿,上穿绣袄,下着锦裙,左右丫鬟养娘侍奉。玳安扒到地下磕头。春梅分付:“放桌儿,摆茶食与玳安吃。”说道:“没甚事,你奶奶免了罢。如何又费心送这许多礼来,你周爷已定不肯受。”玳安道:“家奶奶说,前日平安儿这场事,多有累周爷、周奶奶费心,没甚么,些少微礼儿,与爷、奶奶赏人罢了。”春梅道:“如何好受的?”薛嫂道:“你老人家若不受,惹那头又怪我。”春梅一面又请进守备来计较了,止受了猪酒下饭,把尺头带回将来了。与了玳安一方手帕,三钱银子,抬盒人二钱。春梅因问:“你几时笼起头去,包了网巾?几时和小玉完房来?”玳安道:“是八月内来。”春梅道:“到家多顶上你奶奶,多谢了重礼。待要请你奶奶来坐坐,你周爷早晚又出巡去。我到过年正月里,哥儿生日,我往家里来走走。”玳安道:“你老人家若去,小的到家对俺奶奶说,到那日来接奶奶。”说毕,打发玳安出门。薛嫂便向玳安说:“大官儿,你先去罢,奶奶还要与我说话哩。”那玳安儿押盒担回家,见了月娘说:“如此这般,春梅姐让到后边,管待茶食吃。问了回哥儿好,家中长短。与了我一方手帕,三钱银子,抬盒人二钱银子。多顶上奶奶,多谢重礼,都不受来,被薛嫂儿和我再三说了,才受了下饭猪酒,抬回尺头。要不是请奶奶过去坐坐,一两日周爷出巡去。他只到过年正月孝哥生日,要来家里走走。”又告说:“他住着五间正房,穿着锦裙绣袄,戴着金梁冠儿,出落的越发胖大了。手下好少丫头、奶子侍奉!月娘问:“他其实说明年往咱家来?”玳安儿道:“委实对我说来。”月娘道:“到那日,咱这边使人接他去。”因问:“薛嫂怎的还不来?”玳安道:“我出门,他还坐着说话,教我先来了。”自此两家交往不绝。正是:世情看冷暖,人面逐高低。有诗为证:

\[
得失荣枯命里该,皆因年月日时栽。
胸中有志应须至,蠹里无财莫论才。
\]

\newpage
%# -*- coding:utf-8 -*-
%%%%%%%%%%%%%%%%%%%%%%%%%%%%%%%%%%%%%%%%%%%%%%%%%%%%%%%%%%%%%%%%%%%%%%%%%%%%%%%%%%%%%


\chapter{春梅姐游旧家池馆\KG 杨光彦作当面豺狼}


词曰:

\[
人生千古伤心事,还唱《后庭花》。旧时王谢,堂前燕子,飞向谁家?恍然一梦,仙肌胜雪,宫鬓堆雅。江州司马,青衫泪湿,想在天涯。\named{右调《青衫湿》}
\]

话说光阴迅速,日月如梭,又早到正月二十一日。春梅和周守备说了,备一张祭桌,四样羹果,一坛南酒,差家人周义送与吴月娘。一者是西门庆三周年,二者是孝哥儿生日。月娘收了礼物,打发来人帕一方,银三钱。这边连忙就使玳安儿穿青衣,具请书儿请去。上写着:

\[
重承厚礼,感感。即刻舍具菲酌,奉酬腆仪。仰希高轩俯临,不外,幸甚。西门吴氏端肃拜请大德周老夫人妆次
\]
春梅看了,到日中才来。戴着满头珠翠金凤头面钗梳,胡珠环子。身穿大红通袖、四兽朝麒麟袍儿,翠蓝十样锦百花裙,玉玎当禁步,束着金带。坐着四人大轿,青段销金轿衣。军牢执藤棍喝道,家人伴当跟随,抬着衣匣。后边两顶家人媳妇小轿儿,紧紧跟随。吴月娘这边请人吴大妗子相陪,又叫了四个唱的弹唱。听见春梅来到,月娘亦盛妆缟素打扮,头上五梁冠儿,戴着稀稀几件金翠首饰,上穿白绫袄,下边翠蓝段子裙,与大妗子迎接至前厅。春梅大轿子抬至仪门首,才落下轿来。两边家人围着,到于厅上叙礼,向月娘插烛也似拜下去。月娘连忙答礼相见,说道:“向日有累姐姐费心,粗尺头又不肯受。今又重承厚礼祭桌,感激不尽。”春梅道:“惶恐。家官府没甚么,这些薄礼,表意而已。一向要请奶奶过去,家官府不时出巡,所以不曾请得。”月娘道:“姐姐,你是几时好日子?我只到那日买礼看姐姐去罢。”春梅道:“奴贱日是四月廿五日。”月娘道:“奴到那日已定去。”

两个叙礼毕,春梅务要把月娘让起,受了两礼。然后吴大妗子相见,亦还下礼去。春梅道:“你看大妗子,又没正经。”一手扶起受礼。大妗子再三不肯,止受了半礼。一面让上坐,月娘和大妗子主位相陪。然后家人、媳妇、丫鬟、养娘,都来参见。春梅见了奶子如意儿抱着孝哥儿,吴月娘道:“小大哥还不来与姐姐磕个头儿,谢谢姐姐。今日来与你做生日。”那孝哥儿真个下如意儿身来,与春梅唱喏。月娘道:“好小厮,不与姐姐磕头,只唱喏。”那春梅连忙向袖中摸出一方锦手帕,一副金八吉祥儿,教替他塞帽儿上。月娘道:“又教姐姐费心。”又拜谢了。落后小玉、奶子来见磕头。春梅与了小玉一对头簪子,与了奶子两枝银簪儿。月娘道:“姐姐,你还不知,奶子与了来兴儿做媳妇儿了。来兴儿那媳妇害病没了。”春梅道:“他一心要在咱家,倒也好。”一面丫鬟拿茶上来,吃了茶,月娘道:“请娘娘后边明间内坐罢,这客位内冷。”

春梅来后边西门庆灵前,又早点起灯烛,摆下桌面祭礼。春梅烧了纸,落了几点眼泪。然后周围设放围屏,火炉内生起炭火,安放八大仙桌席,摆茶上来。无非是细巧蒸酥,希奇果品,绝品芽茶。月娘和大妗子陪着吃了茶,让春梅进上房里换衣裳。脱了上面袍儿,家人媳妇开衣匣,取出衣服,更换了一套绿遍地锦妆花袄儿,紫丁香色遍地金裙。在月娘房中坐着,说了一回,月娘因问道:“哥儿好么?今日怎不带他来这里走走?”春梅道:“不是也带他来与奶奶磕头,他爷说天气寒冷,怕风冒着他。他又不肯在房里,只要那当直的抱出来厅上外边走。这两日,不知怎的,只是哭。”月娘道:“他周爷也好大年纪,得你替他养下这点孩子也彀了,也是你裙带上的福。说他孙二娘还有位姐儿,几岁儿了?”春梅道:“他二娘养的叫玉姐,今年交生四岁。俺这个叫金哥。”月娘道:“说他周爷身边还有两位房里姐儿?”春梅道:“是两个学弹唱的丫头子,都有十六七岁,成日淘气在那里。”月娘道:“他爷也常往他身边去不去?”春梅道:“奶奶,他那里得工夫在家?多在外,少在里。如今四外好不盗贼生发,朝廷敕书上,又教他兼管许多事情:镇守地方,巡理河道,提拿盗贼,操练人马。常不时往外出巡几遭,好不辛苦哩。”说毕,小玉又拿茶来吃了。春梅向月娘说:“奶奶,你引我往俺娘那边花园山子下走走。”月娘道:“我的姐姐,还是那咱的山子花园哩!自从你爹下世,没人收拾他,如今丢搭的破零零的。石头也倒了,树木也死了,俺等闲也不去了。”春梅道:“不妨,奴就往俺娘那边看看去。”这月娘强不过,只得叫小玉拿花园门山子门钥匙,开了门,月娘、大妗子陪春梅,到里边游看了半日。但见:

\[
垣墙欹损,台榭歪斜。两边画壁长青笞,满地花砖生碧草。山前怪石遭塌毁,不显嵯峨;亭内凉床被渗漏,已无框档。石洞口蛛丝结网,鱼池内虾蟆成群。狐狸常睡卧云亭,黄鼠往来藏春阁。料想经年无人到,也知尽日有云来。
\]

春梅看了一回,先走到李瓶儿那边。见楼上丢着些折桌、坏凳、破椅子,下边房都空锁着,地下草长的荒荒的。方来到他娘这边,楼上还堆着些生药香料,下边他娘房里,止有两座厨柜,床也没了。因问小玉:“俺娘那张床往那去了?怎的不见?”小玉道:“俺三娘嫁人,赔了俺三娘去了。”月娘走到跟前说:“因你爹在日,将他带来那张八步床赔了大姐在陈家,落后他起身,却把你娘这张床赔了他,嫁人去了。”春梅道:“我听见大姐死了,说你老人家把床还抬的来家了。”月娘道:“那床没钱使,只卖了八两银子,打发县中皂隶,都使了。”春梅听言,点了点头儿。那星眼中由不的酸酸的,口中不言,心内暗道:“想着俺娘那咱,争强不伏弱的问爹要买了这张床。我实承望要回了这张床去,也做他老人家一念儿,不想又与了人去了。”由不的心下惨切。又问月娘:“俺六娘那张螺甸床怎的不见?”月娘道:“一言难尽。自从你爹下世,日逐只有出去的,没有进来的。常言家无营活计,不怕斗量金。也是家中没盘缠,抬出去交人卖了。”春梅问:“卖了多少银子?”月娘道:“止卖了三十五两银子。”春梅道:“可惜了,那张床,当初我听见爹说,值六十两多银子,只卖这些儿。早知你老人家打发,我到与你老人家三四十两银子要了也罢。”月娘道:“好姐姐,人那有早知道的?”一面叹息了半日。

只见家人周仁走来接,说:“爷请奶奶早些家来,哥儿寻奶奶哭哩。”这春梅就抽身往后边来。月娘叫小玉锁了花园门,同来到后边明间内。又早屏开孔雀,帘控鲛绡,摆下酒筵。两个妓女,银筝琵琶,在旁弹唱。吴月娘递酒安席,安春梅上座,春梅不肯,务必拉大妗子,同他一处坐的。月娘主位,筵前递了酒,汤饭点心,割切上席。春梅叫家人周仁,赏了厨子三钱银子。说不尽盘堆羿品,酒泛金波。当下传杯换盏,吃至晚色将落时分,只见宅内又差伴当,拿灯笼来接。月娘那里肯放,教两个妓女在跟前跪着弹唱劝酒。分付:“你把好曲儿孝顺你周奶奶一个儿。”一面叫小玉斟上大钟,放在跟前,说:“姐姐,你分付个心爱的曲儿,叫他两个唱与你下酒。”春梅道:“奶奶,奴吃不得了,怕孩儿家中寻我。”月娘道:“哥儿寻,左右有奶子看着,天色也还早哩,我晓得你好小量儿!”春梅因问那两个妓女:“你叫甚名字?是谁家的?”两个跪下说:“小的一个是韩金钏儿妹子韩玉钏儿,一个是郑爱香儿侄女郑娇儿。”春梅道:“你每会唱《懒画眉》不会?”玉钏儿道:“奶奶分付,小的两个都会。”月娘道:“你两个既会唱,斟上酒你周奶奶吃,你每慢唱。”小玉在旁连忙斟上酒,两个妓女,一个弹筝,一个琵琶,唱道:

\[
冤家为你几时休?捱到春来又到秋。谁人知道我心头。天,害的我伶仃瘦,听和音书两泪流。从前已往诉缘由,谁想你无情把我丢!
\]
那春梅吃过,月娘双令郑娇儿递上一杯酒与春梅。春梅道:“你老人家也陪我一杯。”两家于是都齐斟上,两个妓女又唱道:

\[
冤家为你减风流,鹊噪檐前不肯休,死声活气没来由。天,倒惹的情拖逗,助的凄凉两泪流。从他去后意无休,谁想你辜恩把我丢。
\]
春梅说:“奶奶,你也教大妗子吃杯儿。”月娘道:“大妗子吃不的,教他拿小钟儿陪你罢。”一面令小玉斟上大妗子一小钟儿酒。两个妓女又唱道:

\[
冤家为你惹场忧,坐想行思日夜愁,香肌憔瘦减温柔。天,要见你不能勾,闷的我伤心两泪流。从前与你共绸缪,谁想你今番把我丢。
\]
春梅见小玉在跟前,也斟了一大钟教小玉吃。月娘道:“姐姐,他吃不的。”春梅道:“奶奶,他也吃两三钟儿,我那咱在家里没和他吃?”于是斟上,教小玉也吃了一杯。妓女唱道:

\[
冤家为你惹闲愁,病枕着床无了休,满腹忧闷锁眉头。天,忘了还依旧,助的我腮边两泪流。从前与你两无休,谁想你经年把我丢。
\]

看官听说,当时春梅为甚教妓女唱此词?一向心中牵挂陈敬济,在外不得相会。情种心苗,故有所感,发于吟咏。又见他两个唱的口儿甜,乖觉,奶奶长、奶奶短奉承,心中欢喜。叫家人周仁近前来,拿出两包儿赏赐来,每人二钱银子。两个妓女放下乐器,磕头谢了。不一时,春梅起身,月娘款留不住。伴当打灯笼,拜辞出门,坐上大轿。家人媳妇,都坐上小轿。前后打着四个灯笼,军牢喝道而去。正是:时来顽铁有光辉,远去黄金无艳色。有诗为证:

\[
点绛唇红弄玉娇,凤凰飞下品鸾箫。
堂高闲把湘帘卷,燕子还来续旧巢。
\]

且说春梅自从来吴月娘家赴席之后,因思想陈敬济,不知流落在何处。归到府中,终日只是卧床不起,心下没好气。守备察知其意,说道:“只怕思念你兄弟,不得其所。”一面叫张胜、李安来,分付道:“我一向委你寻你奶奶兄弟,如何不用心找寻?”二人告道:“小的一向找寻来,一地里寻不着下落,已回了奶奶话了。”守备道:“限你二人五日,若找寻不着,讨分晓。”这张胜、李安领了钧语下来,都带了愁颜。沿街绕巷,各处留心,找问不题。

话分两头。单表陈敬济自从守备府中打了出来,欲投宴公庙。又听见人说师父任道士死了,就害怕不敢进庙来,又没脸儿见杏庵主老,白日里到处打油飞,夜晚间还钻入冷铺中存身。一日,也是合当有事,敬济正在街上站立,只见铁指甲杨大郎,头戴新罗帽儿,身穿白绫袄子,骑着一匹驴儿,拣银鞍辔,一个小厮跟随,正从街心走过来。敬济认得是杨光彦,便向前一把手,把嚼环拉住,说道:“杨大哥,一向不见。自从清江浦把我半船货物偷拐走了,我好意往你家问,反吃你兄弟杨二风拿瓦楔钻破头,赶着打上我家门来。今日弄的我一贫如洗,你是会摇摆受用。”那杨大郎见陈敬济已自讨吃,便佯佯而笑,说:“今日晦气,出门撞见瘟死鬼,量你这饿不死贼花子,那里讨半船货?我拐了你的,你不撒手?须吃我一顿马鞭子。”敬济便道:“我如今穷了,你有银子,与我些盘缠。不然,咱到个去处讲讲。”杨大郎见他不放,跳下驴来,向他身上抽了几鞭子。喝令小厮:“与我撏了这少死的花子去!”那小厮使力把敬济推了一交,杨大郎又向前踢了几脚,踢打的敬济怪叫。须臾,围了许多人。旁边闪过一个人来,青高装帽子,勒着手帕,倒披紫袄,白布裤子,精着两条腿,趿着蒲鞋,生的阿兜眼,扫帚眉,料绰口,三须胡子,面上紫肉横生,手腕横筋竞起。吃的楞楞睁睁,提着拳头,向杨大郎说道:“你此位哥好不近理,他年少这般贫寒,你只顾打他怎的?自古嗔拳不打笑面,他又不曾伤犯着你。你有钱,看平日相交,与他些;没钱罢了,如何只顾打他?自古路见不平,也有向灯向火。”杨大郎说:“你不知,他赖我拐了他半船货,量他恁穷样,那有半船货物?”那人道:“想必他当时也是有根基人家娃娃,天生就这般穷来?阁下就是这般有钱?老兄依我,你有银子与他些盘缠罢。”那杨大郎见那人说了,袖内汗巾儿上拴着四五钱一块银子,解下来递与敬济,与那人举一举手儿,上驴子扬长去了。

敬济地下扒起来,抬头看那人时,不是别人,却是旧时同在冷铺内,和他一铺睡的土作头儿飞天鬼侯林儿。近来领着五十名人,在城南水月寺晓月长老那里做工,起盖伽蓝殿。因一只手拉着敬济说道:“兄弟,刚才若不是我拿几句言语讥犯他,他肯拿出这五钱银子与你?那贼却知见范,他若不知范时,好不好吃我一顿好拳头。你跟着我,咱往酒店内吃酒去来。”到一个食荤小酒店,案头上坐下,叫量酒:“拿四卖嗄饭,两大壶酒来。”不一时,量酒摆下小菜嗄饭,四盘四碟,两大坐壶时兴橄榄酒。不用小杯,拿大磁瓯子,因问敬济:“兄弟,你吃面吃饭?”量酒道:“面是温淘,饭是白米饭。”敬济道:“我吃面。”须臾,掉上两三碗温面上来。侯林儿只吃一碗,敬济吃了两碗。然后吃酒。侯林儿向敬济说:“兄弟,你今日跟我往坊子里睡一夜,明日我领你城南水月寺晓月长老那里,修盖伽蓝殿,并两廊僧房。你哥率领着五十名做工。你到那里,不要你做重活,只抬几筐土儿就是了,也算你一工,讨四分银子。我外边赁着一间厦子,晚夕咱两个就在那里歇,做些饭打发咱的人吃。把门你一把锁锁了,家当都交与你,好不好?强如你在那冷铺中,替花子摇铃打梆,这个还官样些。”敬济道:“若是哥哥这般下顾兄弟,可知好哩。不知这工程做的长远不长远?”侯林儿道:“才做了一个月。这工程做到十月里,不知完不完。”两个说话之间,你一钟,我一盏,把两大壶酒都吃了。量酒算帐,该一钱三分半银子。敬济就要拿出银子来秤,侯林儿推过一边,说:“傻兄弟,莫不教你出钱?哥有银子在此。”一面扯出包儿来,秤了一钱五分银子与掌柜的。还找了一分半钱袖了,搭伏着敬济肩背,同到坊子里,两个在一处歇卧。二人都醉了。这侯林儿晚夕干敬济后庭花,足干了一夜。亲哥、亲达达、亲汉子、亲爷,口里无般不叫将出来。

到天明,同往城南水月寺。果然寺外侯林儿赁下半间厦子,里面烧着炕柴,早也买下许多碗盏家活。早辰上工,叫了名字。众人看见敬济,不上二十四五岁,白脸子,生的眉目清俊,就知是侯林儿兄弟,都乱调戏他。先问道:“那小伙子儿,你叫甚名字?”陈敬济道:“我叫陈敬济。”那人道:“陈敬济,可不由着你就挤了。”又一人说:“你恁年小小的,怎干的这营生?捱的这大扛头子?”侯林儿喝开众人,骂:“怪花子,你只顾奚落他怎的?”一面散了锹镢筐扛,派众人抬土的抬土,和泥的和泥,打杂的打杂。

原来晓月长老,教一个叶头陀做火头,造饭与各作匠人吃。这叶头陀年约五十岁,一个眼瞎,穿着皂直裰,精着脚,腰间束着烂绒绦,也不会看经,只会念佛,善会麻衣神相。众人都叫他做叶道。一日做了工下来,众人都吃毕饭,也有闲坐的,卧的,也有蹲着的。只见敬济走向前,问叶头陀讨茶吃。这叶头陀只顾上上下下看他。内有一人说:“叶道,这个小伙子儿是新来的,你相他一相。”又一人说:“你相他相,倒相个兄弟。”一个说:“倒相个二尾子。”叶头陀教他近前,端详了一回,说道:“色怕嫩兮又怕娇,声娇气嫩不相饶。老年色嫩招辛苦,少年色嫩不坚牢。只吃了你面皮嫩的亏,一生多得阴人宠爱。八岁十八二十八,做作百般人可爱,纵然弄假又成真。休怪我说,一生心伶机巧,常得阴人发迹。你今多大年纪?”敬济道:“我二十四岁。”叶道道:“亏你前年怎么过来,吃了你印堂太窄,子丧妻亡,悬壁昏暗,人亡家破;唇不盖齿,一生惹是招非;鼻若灶门,家私倾散。那一年遭官司口舌,倾家散业,见过不曾?”敬济道:“都见过了。”叶头陀道:“只一件,你这山根不宜断绝。麻衣祖师说得两句好:‘山根断兮早虚花,祖业飘零定破家。’早年父祖丢下家业,不拘多少,到你手里,都了当了。你上停短兮下停长,主多成多败,钱财使尽又还来。总然你久后营得家计,犹如烈日照冰霜。你如今往后,还有一步发迹,该有三妻之命。克过一个妻宫不曾?”敬济道:“已克过了。”叶头陀道:“后来还有三妻之会,但恐美中不美。三十上,小人有些不足,花柳中少要行走。”一个人说:“叶道,你相差了,他还与人家做老婆,那有三个妻来?”众人正笑做一团,只听得晓月长老打梆了,各人都拿锹镢筐扛,上工做活去了。如此者,敬济在水月寺,也做了约一月光景。

一日,三月中旬天气,敬济正与众人抬出土来,在山门墙下,倚着墙根,向日阳蹲踞着捉身上虱虮。只见一个人,头带万字头巾,身穿青窄衫,紫裹肚,腰系缠带,脚穿扁靴,骑着一匹黄马,手中提着一篮鲜花儿。见了敬济,猛然跳下马来,向前深深的唱了诺,便叫:“陈舅,小人那里没寻,你老人家原来在这里。”倒唬了敬济一跳。连忙还礼不迭,问:“哥哥,你是那里来的?”那人道:“小人是守备周爷府中亲随张胜,自从舅舅府中官事出来,奶奶不好直到如今,老爷使小人那里不找寻舅舅,不知在这里。今早不是俺奶奶使小人到外庄上,折取这几杂芍药花儿,打这里过,怎得看见你老人家在这里?一来也是你老人家际遇,二者小人有缘。不消犹豫,就骑上马,我跟你老人家往府中去。”那众做工的人看着,面面相觑,不敢做声。这陈敬济把钥匙递与侯林儿,骑上马,张胜紧紧跟随,径往守备府中来。正是:良人得意正年少,今夜月明何处楼?有诗为证:

\[
白玉隐于顽石里,黄金埋在污泥中。
今朝贵人提拔起,如立天梯上九重。
\]

\newpage
%# -*- coding:utf-8 -*-
%%%%%%%%%%%%%%%%%%%%%%%%%%%%%%%%%%%%%%%%%%%%%%%%%%%%%%%%%%%%%%%%%%%%%%%%%%%%%%%%%%%%%


\chapter{假弟妹暗续鸾胶\KG 真夫妇明谐花烛}


词曰:

\[
追悔当初辜深愿,经年价,两成幽怨。任越水吴山,似屏如障堪游玩,奈独自慵抬眼。赏烟花,听弦管,徒欢娱,转加肠断。总时转丹青,强拈书信频频看,又曾似亲眼见。
\]

话说陈敬济,到于守备府中,下了马,张胜先进去禀报春梅。春梅分付,教他在外边班直房内,用香汤沐浴了身体,后边使养娘包出一套新衣服靴帽来,与他更换了。然后禀了春梅。那时守备还未退厅,春梅请敬济到后堂,盛妆打扮,出来相见。这敬济进门就望春梅拜了四双八拜,让姐姐受礼。那春梅受了半礼,对面坐下。叙了寒温离别之情,彼此皆眼中垂泪。春梅恐怕守备退厅进来,见无人在根前,使眼色与敬济,悄悄说:“等住回他若问你,只说是姑表兄弟。我大你一岁,二十五岁了,四月廿五日午时生的。”敬济道:“我知道了。”不一时,丫鬟拿上茶来,两人吃了茶,春梅便问:“你一向怎么出了家做了道士?守备不知是我的亲,错打了你,悔的要不的。若不是那时就留下你,争奈有雪娥那贱人在这里,不好安插你的。所以放你去了。落后打发了那贱人,才使张胜到处寻你不着,谁知你在城外做工,流落至此地位。”敬济道:“不瞒姐姐说,一言难尽。自从与你相别,要娶六姐,我父亲死在东京,来迟了,不曾娶成,被武松杀了。闻得你好心,葬埋了他永福寺,我也到那里烧纸来。落后又把俺娘没了,刚打发丧事出去,被人坑陷了资本。来家又是大姐死了,被俺丈母那淫妇告了一状,床帐妆奁,都搬的去了。打了一场官司,将房儿卖了,弄的我一贫如洗。多亏了俺爹朋友王杏庵周济,把我才送到临清晏公庙那里出家。不料又被光棍打了,拴到咱府中。自从咱府中出去,投亲不理,投友不顾,因此在寺内佣工。多亏姐姐挂心,使张管家寻将我来,得见姐姐一面,犹如再世为人了。”说到伤心处,两个都哭了。

正说话中间,只见守备退厅,左右掀开帘子,守备进来。这陈敬济向前,倒身下拜。慌的守备答礼相还,说:“向日不知是贤弟,被下人隐瞒,误有冲撞,贤弟休怪。”敬济道:“不才有话,一向缺礼,有失亲近,望乞恕罪。”又磕下头去。守备一手扯起,让他上坐。敬济乖觉,那里肯,务要拉下椅儿旁边坐了。守备关席,春梅陪他对坐下。须臾,换茶上来。吃毕,守备便问:“贤弟贵庚?一向怎的不见?如何出家?”敬济使告说:“小弟虚度二十四岁。俺姐姐长我一岁,是四月二十五日午时生。向因父母双亡,家业凋丧,妻又没了,出家在晏公庙。不知家姐嫁在府中,有失探望。”守备道:“自从贤弟那日去后,你令姐昼夜忧心,常时啾啾唧唧,不安直到如今。一向使人找寻贤弟不着,不期今日相会,实乃三生有缘。”

看官听说,若论周守备与西门庆相交,也该认得陈敬济,原来守备为人老成正气,旧时虽然来往,并不留心管他家闲事。就是时常宴会,皆同的是荆都监、夏提刑一班官长,并未与敬济见面。况前日又做了道士一番,那里还想的到西门庆家女婿?所以被他二人瞒过,只认是春梅姑表兄弟。一面分付左右放桌儿,安排酒上来。须臾,摆设许多杯盘肴馔,汤饭点心,堆满桌上,银壶玉盏,酒泛金波。守备相陪叙话,吃至晚来,掌上灯烛方罢。守备分付家人周仁,打扫西书院干净,那里床帐都有。春梅拿出两床铺盖衾枕,与他安歇。又拨了一个小厮喜儿答应他。又包出两套绸绢衣服来,与他更换。每日饭食,春梅请进后边吃。正是:一朝时运至,半点不由人。光阴迅速,日月如梭,但见:

\[
行见梅花腊底,忽逢元旦新正。
不觉艳杏盈枝,又早新荷贴水。
\]
敬济在守备府里,住了个月有余。一日是四月二十五日,春梅的生日。吴月娘那边买了礼来,一盘寿桃,一盘寿面,两只汤鹅,四只鲜鸡,两盘果品,一坛南酒。玳安穿青衣拿贴儿送来。守备正在厅上坐的,门上人禀报,抬进礼来。玳安递上贴儿,扒在地下磕头。守备看了礼贴儿,说道:“多承你奶奶费心,又送礼来。”一面分付家人:“收进礼去,讨茶来与大官儿吃。把礼贴教小伴当送与你舅收了。封了一方手帕、三钱银子与大官儿,抬盒人钱一百文,拿回贴儿,多上覆。”说毕,守备穿了衣服,就起身拜人去了。玳安只顾在厅前伺候,讨回贴儿。只见一个年少的,戴着瓦楞帽儿,穿着青纱道袍,凉鞋净袜,从角门里走出来,手中拿着贴儿赏钱,递与小伴当,一直往后边去了。“可霎作怪,模样倒好相陈姐夫一般。他如何却在这里?”只见小伴当递与玳安手帕银钱,打发出门。

到于家中,回月娘话。见回贴上写着“周门庞氏敛衽拜”。月娘便问:“你没见你姐?”玳安道:“姐姐倒没见,倒见姐夫来。”月娘笑道:“怪囚,你家倒有恁大姐夫!守备好大年纪,你也叫他姐夫。”玳安道:“不是守备,是咱家的陈姐夫。我初进去,周爷正在厅上,我递上贴儿与他磕了头,他说:‘又生受你奶奶送重礼来。’分付伴当拿茶与我吃,‘把贴儿拿与你舅收了,讨一方手帕、三钱银子与大官儿,抬盒人是一百文钱。’说毕,周爷穿衣服出来,上马拜人去了。半日,只见他打角门里出来,递与伴当回贴赏赐,他就进后边去了,我就押着盒担出来。不是他却是谁?”月娘道:“怪小囚儿,休胡说白道的。那羔子知道流落在那里讨吃?不是冻死,就是饿死,他平白在那府里做甚么?守备认的他甚么毛片儿,肯招揽下他?”玳安道:“奶奶敢和我两个赌,我看得千真万真,就烧的成灰骨儿我也认的。”月娘道:“他穿着甚么?”玳安道:“他戴着新瓦楞帽儿,金簪子。身穿着青纱道袍,凉鞋净袜。吃的好了。”月娘道:“我不信,不信。”这里说话不题。

却说陈敬济进入后边,春梅还在房中镜台前搽脸,描画双蛾。敬济拿吴月娘礼贴儿与他看。因问:“他家如何送礼来与你?是那里缘故?”这春梅便把清明郊外,永福寺撞遇月娘相见的话,诉说一遍。后来怎生平安儿偷了解当铺头面,吴巡简怎生夹打平安儿,追问月娘奸情之事,薛嫂又怎生说人情,守备替他处断了事,落后他家买礼来相谢。正月里,我往他家与孝哥儿做生日,勾搭连环到如今。他许下我生日买礼来看我一节,说了一遍。敬济听了,把眼瞅了春梅一眼,说:“姐姐,你好没志气。想着这贼淫妇那咱,把咱姐儿们生生的拆散开了,又把六姐命丧了,永世千年,门里门外不相逢才好,反替他去说人情儿。那怕那吴典恩拷打玳安小厮,供出奸情来,随他那淫妇一条绳子拴去,出丑见官,管咱每大腿事?他没和玳安小厮有奸,怎的把丫头小玉配与他?有我早在这里,我断不教你替他说人情。他是你我仇人,又和他上门往来做甚么?六月连阴——想他好情儿!”几句话,说得春梅闭口无言。这春梅道:“过往勾当,也罢了,还是我心好,不念旧仇。”敬济道:“如今人好心不得这报哩。”春梅道:“他既送了礼,莫不白受他的?他还等着我这里人请他去哩。”敬济道:“今后不消理那淫妇了,又请他怎的?”春梅道:“不请他又不好意思的。丢个贴儿与他,来不来随他就是了。他若来时,你在那边书院内,休出来见他,往后咱不招惹他就是了。”敬济恼的一声儿不言语,走到前边,写了贴儿。春梅使家人周义去请吴月娘。月娘打扮出门,教奶子如意儿抱着孝哥儿,坐着一顶小轿,玳安跟随,来到府中。春梅、孙二娘都打扮出来,迎接至后厅相见,叙礼坐下。如意儿抱着孝哥儿,相见磕头毕。敬济躲在那边书院内,不走出来,由着春梅、孙二娘在后厅摆茶安席递酒。叫了两个妓女韩玉钏、郑娇儿弹唱,俱不必细说。

玳安在前边厢房内管待。只见一个小伴当,打后边拿着一盘汤饭点心下饭,往西角门书院中走。玳安便问他拿与谁吃,小伴当说:“是与舅吃的。”玳安道:“代舅姓甚么?”小伴当道:“姓陈。”这玳安贼,悄悄后边跟着他到西书院。小伴当便掀帘子进去,放卓儿吃。这玳安悄悄走出外来,依旧坐在厢房内。直待天晚,家中灯笼来接,吴月娘轿子起身。到家,一五一十告诉月娘说:“果然陈姐夫在他家居住。”自从春梅这边被敬济把拦,两家都不相往还。正是:

\[
谁知竖子多间阻,一念翻成怨恨媒。
\]

敬济在府中与春梅暗地勾搭,人都不知。或守备不在,春梅就和敬济在房中吃饭吃酒,闲时下棋调笑,无所不至。守备在家,便使丫头小厮拿饭往书院与他吃。或白日里,春梅也常往书院内,和他坐半日,方归后边来。彼此情热,俱不必细说。

一日,守备领人马出巡,正值五月端午佳节。春梅在西书院花亭上置了一卓酒席,和孙二娘、陈敬济吃雄黄酒,解粽欢娱。丫鬟侍妾都两边侍奉。春梅令海棠、月桂两个侍妾在席前弹唱。当下直吃到炎光西坠、微雨生凉的时分。春梅拿起大金荷花杯来相劝。酒过数巡,孙二娘不胜酒力,起身先往后边房中看去了。独落下春梅和敬济在花亭上吃酒,猜枚行令,你一杯,我一杯。不一时,丫鬟掌上纱灯来,养娘金匮、玉堂打发金哥儿睡去了。敬济输了,便走入书房内躲酒不出来。这春梅先使海棠来请,见敬济不去,又使月桂来,分付:“他不来,你好歹与我拉将来。拉不将来,回来把你这贱人打十个嘴巴。”这月桂走至西书房中,推开门,见敬济歪在床上,推打鼾睡,不动。月桂说:“奶奶叫我来请你老人家,请不去,要打我哩。”那敬济口里喃喃呐呐说:“打你不干我事。我醉了,吃不的了。”被月桂用手拉将起来,推着他:“我好歹拉你去,拉不将你去,也不算好汉。”推拉的敬济急了,黑影子里佯装着醉,作耍当真,搂了月桂在怀里就亲个嘴。那月桂亦发上头上脑说:“人好意叫你,你就大不正,倒做这个营生。”敬济道:“我的儿,你若肯了,那个好意做大不成?”又按着亲了个嘴,方走到花亭上。月桂道:“奶奶要打我,还是我把舅拉将来了。”春梅令海棠斟上大钟,两个下盘棋,赌酒为乐。当下你一盘,我一盘,熬的丫鬟都打睡去了。春梅又使月桂、海棠后边取茶去,两个在花亭上,解佩露相如之玉,朱唇点汉署之香。正是:得多少花阴曲槛灯斜照,旁有坠钗双凤翘。有诗为证:

\[
花亭欢洽鬓云斜,粉汗凝香沁绛纱。
深院日长人不到,试看黄鸟啄名花。
\]
两个正干得好,忽然丫鬟海棠送茶来:“请奶奶后边去,金哥睡醒了,哭着寻奶奶哩。”春梅陪敬济又吃了两钟酒,用茶嗽了口,然后抽身往后边来。丫鬟收拾了家活,喜儿扶敬济归书房寝歇,不在话下。

一日,朝廷敕旨下来,命守备领本部人马,会同济州府知府张叔夜,征剿梁山泊贼王宋江,早晚起身。守备对春梅说:“你在家看好哥儿,叫媒人替你兄弟寻上一门亲事。我带他个名字在军门,若早侥幸得功,朝廷恩典,升他一官半职,于你面上,也有光辉。”这春梅应诺了。迟了两三日,守备打点行装,整率人马,留下张胜、李安看家,止带家人周仁跟了去。不题。

一日,春梅叫将薛嫂儿来,如此这般和他说:“他爷临去分付,叫你替我兄弟寻门亲事,你须寻个门当户对好女儿,不拘十六七岁的也罢,只要好模样儿,联明伶俐些的。他性儿也有些厥劣。”薛嫂儿道:“我不知道他也怎的?不消你老人家分付。想着大姐那等的还嫌哩。”春梅道:“若是寻的不好,看我打你耳刮子不打?我要赶着他叫小妗子儿哩,休要当耍子儿。”说毕,春梅令丫鬟摆茶与他吃。只见陈敬济进来吃饭。薛嫂向他道了万福,说:“姑夫,你老人家一向不见,在那里来?且喜呀,刚刚奶奶分付,交我替你老人家寻个好娘子,你怎么谢我?”那陈敬济把脸儿迸着不言语。薛嫂道:“老花子怎的不言语?”春梅道:“你休要叫他姑夫,那个已是揭过去的帐了,你只叫他陈舅就是了。”薛嫂道:“真该打,我这片子狗嘴,只要叫错了,往后赶着你只叫舅爷罢。”那敬济忍不住,扑吃的笑了,说道:“这个才可到我心上。”那薛嫂撒风撒痴,赶着打了他一下,说道:“你看老花子说的好话儿,我又不是你影射的,怎么可在你心上?”连春梅也笑了。

不一时,月桂安排茶食与薛嫂吃了,说道:“我替你老人家用心踏着,有人家相应好女子儿,就来说。”春梅道:“财礼羹果,花红酒礼,头面衣服,不少他的,只要好人家好女孩儿,方可进入我门来。”薛嫂道:“我晓得,管情应的你老人家心便了。”良久,敬济吃了饭,往前边去了。薛嫂儿还坐着,问春梅:“他老人家几时来的?”春梅便把出家做道士一节说了:“我寻得他来,做我个亲人儿。”薛嫂道:“好好,你老人家有后眼。”又道:“前日你老人家好日子,说那头他大娘来做生日来?”春梅道:“他先送礼来,我才使人请他,坐了一日去了。”薛嫂道:“我那日在一个人家铺床,整乱了一日。心内要来,急的我要不的。”又问:“他陈舅,也见他那头大娘来?”春梅道:“他肯下气见他?为请他,好不和我乱成一块。嗔我替他家说人情,说我没志气。那怕吴典恩打着小厮,攀扯他出官才好,管你腿事?你替他寻分上,想着他昔日好情儿?”薛嫂道:“他老人家也说的是,及到其间,也不计旧仇罢了。”春梅道:“咱既受了他礼,不请他来坐坐儿,又使不的。宁可教他不仁,休要咱不义。”薛嫂道:“怪不的你老人家有恁大福,休的心忒好了!”当下薛嫂儿说了半日话,提着花箱儿,拜辞出门。

过了两日,先来说:“城里朱千户家小姐,今年十五岁,也好陪嫁,只是没了娘的儿了。”春梅嫌小不要。又说应伯爵第二个女儿,年二十二岁。春梅又嫌应伯爵死了,在大爷手内聘嫁,没甚陪送,也不成。都回出婚帖儿来。又迟了几日,薛嫂儿送花儿来,袖中取出个婚贴儿,大红段子上写着:“开段铺葛员外家大女儿,年二址岁,属鸡的,十一月十五日子时生,小字翠屏。”“生的上画儿般模样儿,五短身材,瓜子面皮,温柔典雅,联明伶俐,针指女工,自不必说。父母俱在,有万贯钱财。在大街上开段子铺,走苏杭、南京,无比好人家。陪嫁都是南京床帐箱笼。”春梅道:“既是好,成了这家的罢。”就交薛嫂儿先通信去。那薛嫂儿连忙说去了。正是:

\[
欲向绣房求艳质,须凭红叶是良媒。
\]
有诗为证:

\[
天仙机上系香罗,千里姻缘竟足多。
天上牛郎配织女,人间才子伴娇娥。
\]

这里薛嫂通了信来,葛员外家知是守备府里,情愿做亲,又使一个张媒人同说媒。春梅这里备了两抬茶叶、粮饼、羹果,教孙二娘坐轿子,往葛员外家插定女儿。回来对春梅说:“果然好个女子,生的一表人才,如花似朵,人家又相当。”春梅这里择定吉日,纳采行礼。十六盘羹果茶饼,两盘头面,二盘珠翠,四抬酒,两牵羊,一顶鬒髻,全副金银头面簪环之类。两件罗段袍儿,四季衣服。其余绵花布绢,二十两礼银,不必细说。阴阳生择在六月初八日,准娶过门。春梅先问薛嫂儿:“他家那里有陪床使女没有?”薛嫂儿道:“床帐妆奁都有,只没有使女陪床。”春梅道:“咱这里买一个十三四岁丫头子,与他房里使唤,掇桶子倒水方便些。”薛嫂道:“有,我明日带一个来。”

到次日,果然领了一个丫头,说:“是商人黄四家儿子房里使的丫头,今年才十三岁。黄四因用下官钱粮,和李三还有咱家出去的保官儿,都为钱粮捉拿在监里追赃,监了一年多,家产尽绝,房儿也卖了。李三先死,拿儿子李活监着。咱家保官儿那儿僧宝儿,如今流落在外,与人家跟马哩。”春梅道:“是来保?”薛嫂道:“他如今不叫来保,改了名字叫汤保了。”春梅道:“这丫头是黄四家丫头,要多少银子?”薛嫂道:“只要四两半银子。紧等着要交赃去。”春梅道:“甚么四两半,与他三两五钱银子留下罢。”一面就交了三两五钱雪花官银与他,写了文书。改了名字,唤做金钱儿。

话休饶舌,又早到六月初八。春梅打扮珠翠凤冠,穿通袖大红袍儿,束金镶碧玉带。坐四人大轿,鼓乐灯笼,娶葛家女子,奠雁过门。陈敬济骑大白马,拣银鞍辔,青衣军牢喝道。头戴儒巾,穿着青段圆领,脚下粉底皂靴,头上簪着两支金花。正是:久旱逢甘雨,他乡遇故知。洞房花烛夜,金榜挂名时。一番拆洗一番新。到守备府中,新人轿子落下。头盖大红销金盖袱,添妆含饭,抱着宝瓶进入大门。阴阳生引入画堂,先参拜了堂,然后归到洞房。春梅安他两口儿坐帐,然后出来。阴阳生撒帐毕,打发喜钱出门,鼓手都散了。敬济与这葛翠屏小姐坐了回帐,骑马打灯笼,往岳丈家谢亲。吃的大醉而归。晚夕女貌郎才,未免燕尔新婚,交媾云雨。正是:得多少——

\[
春点杏桃红绽蕊,风欺杨柳绿翻腰。
\]

当夜敬济与这葛翠屏小姐倒且是合得着。两个被底鸳鸯,帐中鸾凤,如鱼似水,合卺欢娱。三日完饭,春梅在府厅后堂张筵挂采,鼓乐笙歌,请亲眷吃会亲酒,俱不必细说。每日春梅吃饭,必请他两口儿同在房中一处吃。彼此以姑妗称之,同起同坐。丫头养娘、家人媳妇,谁敢道个不字?原来春梅收拾西厢房三间,与他做房,里面铺着床帐,糊的雪洞般齐整,垂着帘帏。外边西书院,是他书房。里面亦有床榻、几席、古书并守备往来书柬拜贴,并各处递来手本揭贴,都打他手里过。春梅不时出来书院中,和他闲坐说话,两个暗地交情。正是:

\[
朝陪金谷宴,暮伴绮楼娃。
休道欢娱处,流光逐落霞。
\]

\newpage
%# -*- coding:utf-8 -*-
%%%%%%%%%%%%%%%%%%%%%%%%%%%%%%%%%%%%%%%%%%%%%%%%%%%%%%%%%%%%%%%%%%%%%%%%%%%%%%%%%%%%%


\chapter{陈敬济临清逢旧识\KG 韩爱姐翠馆遇情郎}


诗曰:

\[
教坊脂粉洗铅华,一片闲心对落花。
旧曲听来犹有恨,故园归去已无家。
云鬟半挽临妆镜,两泪空流湿绛纱。
今日相逢白司马,樽前重与诉琵琶。
\]

话说一日,周守备与济南府知府张叔夜,领人马剿梁山泊贼王宋江三十六人,万余草寇,都受了招安。地方平复,表奏朝廷,大喜。加升张叔夜为都御史、山东安抚大使、升备周秀为济南兵马制置,管理分巡河道,提察盗贼。部下从征有功人员,各升一级。军门带得敬济名字,升为参谋之职,月给米二石,冠带荣身。守备至十月中旬,领了敕书,率领人马来家。先使人来报与春梅家中知道。春梅满心欢喜,使陈敬济与张胜、李安出城迎接。家中厅上排设酒筵,庆官贺喜。官员人等来拜贺送礼者不计其数。守备下马,进入后堂,春梅、孙二娘接着。参贺已毕,陈敬济就穿大红员领,头戴冠帽,脚穿皂靴,束着角带,和新妇葛氏两口儿拜见。守备见好个女子,赏了一套衣服、十两银子打头面,不在话下。

晚夕,春梅和守备在房中饮酒,未免叙些家常事务。春梅道:“为娶我兄弟媳妇,又费许多东西。”守备道:“阿呀,你止这个兄弟,投奔你来,无个妻室,不成个前程道理。就是费了几两银子,不曾为了别人。”春梅道:“你今又替他挣了这个前程,足以荣身勾了。”守备道:“朝廷旨意下来,不日我往济南府到任。你在家看家,打点些本钱,教他搭个主管,做些大小买卖。三五日教他下去,查算帐目一遭,转得些利钱来,也勾他搅计。”春梅道:“你说的也是。”两个晚夕,夫妻同欢,不可细述。在家中住了十个日子,到十一月初旬时分,守备收拾起身。带领张胜、李安,前去济南到任,留周仁、周义看家。陈敬济送到城南永福寺方回。

一日,春梅向敬济商议:“守备教你如此这般,河下寻些买卖,搭个主管,觅得些利息,也勾家中费用。”这敬济听言,满心欢喜。一日,正打街前走,寻觅主管伙计。也是合当有事,不料撞遇旧时朋友陆二哥陆秉义,作揖说:“哥怎的一向不见?”敬济道:“我因亡妻为事,又被杨光彦那厮拐了我半船货物,坑陷的我一贫如洗。我如今又好了,幸得我姐姐嫁在守备府中,又娶了亲事,升做参谋,冠带荣身。如今要寻个伙计作些买卖,一地里没寻处。”陆秉义道:“杨光彦那厮拐了你货物,如今搭了个姓谢的做伙计,在临清马头上开了一座大酒店,又放债与四方趁熟窠子娼门人使,好不获大利息。他每日穿好衣,吃好肉,骑着一匹驴儿,三五日下去走一遭,算帐收钱,把旧朋友都不理。他兄弟在家开赌场,斗鸡养狗,人不敢惹他。”敬济道:“我去年曾见他一遍,他反面无情,打我一顿,被一朋友救了。我恨他入于骨髓。”因拉陆二郎入路旁一酒店内吃酒。两人计议:“如何处置他,出我这口气?”陆秉义道:“常言说得好:恨小非君子,无毒不丈夫。咱如今将理和他说,不见棺材不下泪,他必然不肯。小弟有一计策,哥也不消做别的买卖,只写一张状子,把他告到那里,追出你货物银子来。就夺了这座酒店,再添上些本钱,等我在马头上和谢三哥掌柜发卖。哥哥你三五日下去走一遭,查算帐目,管情见一月,你稳拍拍的有四十两银子利息,强如做别的生意。”看官听说,当时只因这陆秉义说出这桩事,有分数,数个人死于非命。陈敬济一种死,死之太苦;一种亡,亡之太屈。正是:

\[
非干前定数,半点不由人。
\]
敬济听了,道:“贤弟,你说的是。我到家就对我姐夫和姐姐说。这买卖成了,就安贤弟同谢三郎做主管。”当下两个吃了回酒,各下楼来,还了酒钱。敬济分付陆二哥:“兄弟,千万谨言。”陆二郎道:“我知道。”各散回家。

这敬济就一五一十对春梅说:“争奈他爷不在,如何理会?”有老家人周忠在旁,便道:“不要紧,等舅写了一张状子,该拐了多少银子货物,拿爷个拜贴儿,都封在里面。等小的送与提刑所两位官府案下,把这姓杨的拿去衙门中,一顿夹打追问,不怕那厮不拿出银子来。”敬济大喜,一面写就一纸状子,拿守备拜贴,弥封停当,就使老家人周忠送到提刑院。两位官府正升厅问事,门上人禀道:“帅府周爷差人下书。”何千户与张二官府唤周忠进见,问周爷上任之事,说了一遍。拆开封套观看,见了拜贴、状子。自恁要做分上,即便批行,差委缉捕番捉,往河下拿杨光彦去。回了个拜贴,付与周忠:“到家多上覆你爷、奶奶,待我这里追出银两,伺候来领。”周忠拿回贴到府中,回覆了春梅说话:“即时准行拿人去了。待追出银子,使人领去。”敬济看见两个折贴上面写着:“侍生何永寿、张懋德顿首拜”。敬济心中大喜。

迟不上两日光景,提刑缉捕观察番捉,往河下把杨光彦并兄弟杨二风都拿到衙门中。两位官府,据着陈敬济状子审问。一顿夹打,监禁数日,追出三百五十两银子,一百桶生眼布。其余酒店中家活,共算了五十两,陈敬济状上告着九百两,还差三百五十两银子。把房儿卖了五十两,家产尽绝。这敬济就把谢家大酒楼夺过来,和谢胖子合伙。春梅又打点出五百两本钱,共凑了一千两之数。委付陆秉义做主管,重新把酒楼装修、油漆彩画,阑干灼耀,栋宇光新,桌案鲜明,酒肴齐整。真个是:

\[
启瓮三家醉,开樽十里香。
神仙留玉佩,卿相解金貂。
\]

从正月半头,陈敬济在临清马头上大酒楼开张,见一日也发卖三五十两银子。都是谢胖子和陆秉义眼同经手,在柜上掌柜。敬济三五日骑头口,伴当小姜儿跟随,往河下算帐一遭。若来,陆秉义和谢胖子两个伙计,在楼上收拾一间干净阁儿,铺陈床帐,安放卓椅,糊的雪洞般齐整。摆设酒席,交四个好出色粉头相陪。陈三儿那里往来做量酒。

一日,三月佳节,春光明媚,景物芬芳,翠依依槐柳盈堤,红馥馥杏桃灿锦。陈敬济在楼上,搭伏定绿阑干,看那楼下景致,好生热闹。有诗为证:

\[
风拂烟笼锦绣妆,太平时节日初长。
能添壮士英雄胆,善解佳人愁闷肠。
三尺晓垂杨柳岸,一竿斜插杏花旁。
男儿未遂平生志,且乐高歌入醉乡。
\]

一日,敬济在楼窗后瞧看,正临着河边,泊着两只剥船。船上载着许多箱笼,卓凳家活,四五个人,尽搬入楼下空屋里来。船上有两个妇人,一个中年妇人,长挑身材,紫膛色;一个年小妇人,搽脂抹粉,生的白净标致,约有二十多岁。尽走入屋里来。敬济问谢主管:“是甚么人?也不问一声,擅自搬入我屋里来。”谢主管道:“此两个是东京来的妇人,投亲不着,一时间无处寻房住,央此间邻居范老来说,暂住两三日便去。正欲报知官人,不想官人来问。”这敬济正欲发怒,只见那年小妇人敛衽向前,望敬济深深的道了个万福,告说:“官人息怒,非干主管之事,是奴家大胆,一时出于无奈,不及先来宅上禀报,望乞恕罪。容略住得三五日,拜纳房金,就便搬去。”这敬济见小妇人会说话儿,只顾上上下下把眼看他。那妇人一双星眼斜盼敬济,两情四目,不能定情。敬济口中不言,心内暗想:“倒相那里会过,这般眼熟。”那长挑身材中年妇人,也定睛看着敬济,说道:“官人,你莫非是西门老爷家陈姑爷么?”这敬济吃了一惊,便道:“你怎的认得我?”那妇人道:“不瞒姑爷说,奴是旧伙计韩道国浑家,这个就是我女孩儿爱姐。”敬济道:“你两口儿在东京,如何来在这里?你老公在那里?”那妇人道:“在船上看家活。”敬济急令量酒请来相见。

不一时,韩道国走来作揖,已是掺白须鬓,因说起:“韩中蔡太师、童太尉、李右相、朱太尉、高太尉、李太监六人,都被太学国子生陈东上本参劾,后被科道交章弹奏倒了。圣旨下来,拿送三法司问罪,发烟瘴地面,永远充军。太师儿子礼部尚书蔡攸处斩,家产抄没入官。我等三口儿各自逃生,投到清河县寻我兄弟第二的。不想第二的把房儿卖了,流落不知去向。三口儿雇船,从河道中来,不料撞遇姑夫在此,三生有幸。”因问:“姑夫今还在西门老爷家里?”敬济把头项摇了一摇,说:“我也不在他家了。我在姐夫守备周爷府中,做了参谋官,冠带荣身。近日合了两个伙计,在此马头上开这个酒店,胡乱过日子。你每三口儿既遇着我,也不消搬去,便在此间住也不妨,请自稳便。”妇人与韩道国一齐下礼。说罢,就搬运船上家活箱笼上来。敬济看得心痒,也使伴当小姜儿和陈三儿替他搬运了几件家活。王六儿道:“不劳姑夫费心用力。”彼此俱各欢喜。敬济道:“你我原是一家,何消计较?”敬济见天色将晚,有申牌时分,要回家。分付主管:“咱蚤送些茶盒与他。”上马,伴当跟随来家,一夜心心念念,只是放韩爱姐不下。

过了一日,到第三日早起身,打扮衣服齐整,伴当小姜跟随来河下大酒楼店中,看着做了回买卖。韩道国那边使的八老来请吃茶。敬济心下正要瞧去,恰好八老来请,便起身进去。只见韩爱姐见了,笑容可掬,接将出来,道了万福:“官人请里面坐。”敬济到阁子内会下,王六儿和韩道国都来陪坐。少顷茶罢,彼此叙此旧时的闲话,敬济不住把眼只睃那韩爱姐,爱姐一双一双涎澄澄秋波只看敬济,彼此都有意了。有诗为证:

\[
弓鞋窄窄剪春罗,香体酥胸玉一窝。
丽质不胜袅娜态,一腔幽恨蹙秋波。
\]

少顷,韩道国走出去了。爱姐因问:“官人青春多少?”敬济道:“虚度二十六岁。”敬济问:“姐姐青春几何?”爱姐笑道:“奴与官人一缘一会,也是二十六岁。旧日又是大老爹府上相会过面,如何又幸遇在一处,正是有缘千里来相会。”那王六儿见他两个说得入港,看见关目,推个故事,也走出去了。止有他两人对坐。爱姐把些风月话儿来勾敬济,敬济自幼干惯的道儿,怎不省得!便涎着脸儿,调戏答话。原来这韩爱姐从东京来,一路儿和他娘已做些道路。今见了敬济,也是夙世有缘,三生一笑,不由的情投意合,见无人处,就走向前,挨在他身边坐下,作娇作痴,说道:“官人,你将头上金簪子借我看一看。”敬济正欲拔时,早被爱姐一手按住敬济头髻,一手拔下簪子来。便笑吟吟起身,说:“我和你去楼上说句话儿。”一头说,一头走。敬济得不的这一声,连忙跟上楼来。正是:

\[
风来花自舞,春入鸟能言。
\]
敬济跟他上楼,便道:“姐姐有甚话说?”爱姐道:“奴与你是宿世姻缘,今朝相遇,愿偕枕席之欢,共效于飞之乐。”敬济道:“难得姐姐见怜,只怕此间有人知觉。”韩爱姐做出许多妖娆来,搂敬济在怀,将尖尖玉手扯下他裤子来。两个情兴如火,按纳不住,爱姐不免解衣仰卧,在床上交媾在一处。正是:

\[
色胆如天怕甚事,鸳帏云雨百年情。
\]
敬济问:“你叫几姐?”那韩爱姐道:“奴是端午所生,就叫五姐,又名爱姐。”霎时云收雨散,偎倚共坐。韩爱姐将金簪子原插在他头上,又告敬济说:“自从三口儿东京来,投亲不着,盘缠缺欠。你有银子,见借与我父亲五两,奴按利纳还,不可推阻。”敬济应允,说:“不打紧,姐姐开口,就兑五两来。”两个又坐了半日,恐怕人谈论,吃了一杯茶,爱姐留吃午饭,敬济道:“我那边有事,不吃饭了,少间就送盘缠来与你。”爱姐道:“午后奴略备一杯水酒,官人不要见却,好歹来坐坐。”

敬济在店内吃了午饭,又在街上闲散走了一回。撞见昔日晏公庙师兄金宗明作揖,把前事诉说了一遍。金宗明道:“不知贤弟在守备老爷府中认了亲,在大楼开店,有失拜望。明日就使徒弟送茶来,闲中请去庙中坐一坐。”说罢,宗明归去了。敬济走到店中,陆主管道:“里边住的老韩请官人吃酒,没处寻。”正说着,恰好八老又来请。就请二位主管相陪,再无他客。敬济就同二主管,走到里边房内,蚤已安排酒席齐整。敬济上坐,韩道国主位,陆秉义、谢胖子打横,王六儿与爱姐旁边佥坐,八老往来筛酒下菜。吃过数杯,两个主管会意,说道:“官人慢坐,小人柜上看去。”起身去了。敬济平昔酒量,不十分洪饮,又见主管去了,开怀与韩道国三口儿吃了数杯,便觉有些醉将上来。爱姐便问:“今日官人不回家去罢了?”敬济道:“这咱晚了,回去不得,明日起身去罢。”王六儿、韩道国吃了一回,下楼去了。敬济向袖中取出五两银子,递与爱姐。爱姐到下边交与王六儿,复上来。两个交杯换盏,倚翠偎红,吃至天晚。爱姐卸下浓妆,留敬济就在楼上阁儿里歇了。当下枕畔山盟,衾中海誓,莺声燕语,曲尽绸缪,不能悉记。爱姐在东京蔡太师府中,与翟管家做妾,曾扶持过老太太,也学会些弹唱,又能识字会写,种种可人。敬济欢喜不胜,就同六姐一般,正可在心上。以此与他盘桓一夜,停眠罢宿,免不的第二日起来得迟,约饭时才起来。王六儿安排些鸡子肉圆子,做了个头脑与他扶头。两个吃了几杯暖酒。少顷主管来,请敬济那边摆饭。敬济梳洗毕,吃了饭,又来辞爱姐,要回去。那爱姐不舍,只顾抛泪。敬济道:“我到家三、五日,就来看你,你休烦恼。”说毕,伴当跟随,骑马往城中去了。一路上分付小姜儿:“到家休要说出韩家之事。”小姜儿道:“小的知道,不必分付。”

敬济到府中,只推店中买卖忙,算了帐目不觉天晚,归来不得,歇了一夜。交割与春梅利息银两,见一遭儿也有三十两银子之数。回到家中,又被葛翠屏噪聒:“官人怎的外边歇了一夜?想必在柳陌花街行踏,把我丢在家中,独自空房,就不思想来家。”一连留住陈敬济七八日,不放他往河下来。店中只使小姜儿,来问主管讨算利息。主管一一封了银子去。

韩道国免不得又交老婆王六儿又招惹别的熟人儿,或是商客来屋里走动,吃茶吃酒。这韩道国先前尝着这个甜头,靠老婆衣饭肥家。况王六儿年纪虽老,风韵犹存,恰好又得他女儿来接代,也不断绝这样行业,如今索性大做了。当下见敬济不来,量酒陈三儿替他勾了一个湖州贩丝绵客人何官人来,请他女儿爱姐。那何官人年约五十余岁,手中有千两丝绵绸绢货物,要请爱姐。爱姐一心想着敬济,推心中不快,三回五次不肯下楼来,急的韩道国要不的。那何官人又见王六儿长挑身材,紫膛色,瓜子面皮,描的大大小鬓,涎邓邓一双星眼,眼光如醉,抹的鲜红嘴唇,料此妇人一定好风情,就留下一两银子,在屋里吃酒,和王六儿歇了一夜。韩道国便躲避在外边歇了,他女儿见做娘的留下客,只在楼上不下楼来,自此以后,那何官人被王六儿搬弄得快活,两个打得一似火炭般热,没三两日不来与他过夜。韩道国也禁过他许多钱使。

这韩爱姐见敬济一去十数日不来,心中思想,挨一日似三秋,盼一夜如半夏,未免害木边之目,田下之心。使八老往城中守备府中探听。看见小姜儿,悄悄问他:“官人如何不去?”小姜儿说:“官人这两日有些身子不快,不曾出门。”回来诉与爱姐。爱姐与王六儿商议,买了一副猪蹄,两只烧鸭,两尾鲜鱼,一盒酥饼,在楼上磨墨挥笔,写封柬帖,使八老送到城中与敬济去,叮咛嘱付:“你到城中,须索见陈官人亲收,讨回贴来。”八老怀内揣着柬帖,挑着礼物,一路无词。来到城内守备府前,坐在沿街石台基上。只见伴当小姜儿出来,看见八老:“你又来做甚么?”八老与他声喏,拉在僻净处说:“我特来见你官人,送礼来了。还有话说,我只有此等你。你可通报官人知道。”小姜随即转身进去。不多时,只见敬济摇将出来。那时约五月,天气暑热。敬济穿着纱衣服,头戴着瓦楞帽,凉鞋净袜。八老慌忙声喏,说道:“官人贵体好些?韩爱姐使我稍一柬帖,送礼来了。”敬济接了柬帖,说:“五姐好么?”八老道:“五姐见官人一向不去,心中也不快在那里。多上覆官人,几时下去走走?”敬济拆开柬帖观看上面写着甚言词:

\[
贱妾韩爱姐敛衽拜,谨启情郎陈大官人台下:自别尊颜,思慕之心未尝少怠。向蒙期约,妾倚门凝望,不见降临。昨遣八老探问起居,不遇而回。闻知贵恙欠安,令妾空怀账望,坐卧闷恹,不能顿生两翼而傍君之左右也。君在家,自有娇妻美爱,又岂肯动念于妾,犹吐去之果核也。兹具腥味、茶盒数事,少伸问安诚意,幸希笑纳。情照不宣。外具锦绣鸳鸯香囊一个,青丝一缕,少表寸心。仲夏念日贱妾爱姐再拜。
\]

敬济看了柬帖并香囊。香囊里面安放青丝一缕,香囊上扣着“寄与情郎陈君膝下”八字,依先折了,藏在袖中。府旁侧首有个酒店,令小姜儿:“领八老同店内吃钟酒,等我写回帖与你。”小姜不敢怠慢,把四盒礼物收进去了。敬济走到书院房内,悄悄写了回柬,又包了五两银子,到酒店内问八老:“吃了酒不曾?”八老道:“多谢官人好酒,吃不得了,起身去罢。”敬济将银子并回柬付与八老,说:“到家多多拜上五姐,这五两白金与他盘缠,过三两日,我自去看他。”八老收了银、柬,一直去了。敬济回家,走入房中,葛翠屏便问:“是谁家送的礼物?”敬济悉言:“店主人谢胖子,打听我不快,送礼物来问安。”翠屏亦信其实。两口儿计议,交丫鬟金钱儿拿盘子,拿了一只烧鸭,一尾鲜血,半副蹄子,送到后边与春梅吃,说是店主人家送的,也不查问。此事表过不题。

却说八老到河下,天已晚了,入门将银、柬都付与爱姐收了。拆开银、柬,灯下观看,上面写道:

\[
爱弟敬济顿首字覆爱卿韩五姐妆次:向蒙会问,又承厚款,亦且云情雨意,祚席钟爱,无时少怠。所云期望,正欲趋会,偶因贱躯不快,有失卿之盼望。又蒙遣人垂顾,兼惠可口佳肴,锦囊佳制,不胜感激!只在二三日间,容当面布。外具白金五两,绫帕一方,少伸远芹之敬,优乞心鉴,万万。敬济再拜。
\]
爱姐看了,见帕上写着四句诗曰:

\[
吴绫帕儿织回文,洒翰挥毫墨迹新。
寄与多情韩五姐,永谐鸾凤百年情。
\]
看毕,爱姐把银子付与王六儿。母子千欢万喜,等候敬济,不在话下。正是:得意友来情不厌,知心人至话相投。有诗为证:

\[
碧纱窗下启笺封,一纸云鸿香气浓。
知你挥毫经玉手,相思都付不言中。
\]

\newpage
%# -*- coding:utf-8 -*-
%%%%%%%%%%%%%%%%%%%%%%%%%%%%%%%%%%%%%%%%%%%%%%%%%%%%%%%%%%%%%%%%%%%%%%%%%%%%%%%%%%%%%


\chapter{刘二醉骂王六儿\KG 张胜窃听张敬济}


词曰:

\[
白云山,红叶树,阅尽兴亡,一似朝还暮。多少夕阳芳草渡,潮落潮生,还送人来去。阮公途,杨子路,九折羊肠,曾把车轮误。记得寒芫嘶马处,翠官银筝,夜夜歌楼曙。\named{右调《苏幕遮》}
\]

话说陈敬济,过了两日,到第三日,却是五月二十日他的生日,后厅整置酒肴,与他上寿,合家欢乐了一日。次日早辰,敬济说:“我一向不曾往河下去,今日没事,去走一遭,一者和主管算帐,二来就避炎暑,走走便回。”春梅分付:“你去坐一乘轿子,少要劳碌。”交两个军牢抬着轿子,小姜儿跟随,径往河下在酒楼店中来。

一路无词,午后时分到了,下轿进入里面。两个主管齐来参见,说:“官人贵体好些?”敬济道:“生受二位伙计挂心。”他一心只在韩爱姐身上,坐了一回便起身,分付主管:“查下帐目,等我来算。”就转身到后边。八老又早迎见,报与王六儿夫妇。韩爱姐正在楼上,凭栏盼望,挥毫作诗遣怀。忽报陈敬济来了,连忙轻移莲步,款蹙湘裙,走下楼来。母子面上堆下笑来迎接,说道:“官人,贵人难见面,那阵风儿吹你到俺这里?”敬济与他母子作了揖,同进阁儿内坐定。少顷,王六儿点茶上来。吃毕茶,爱姐道:“请官人到楼上奴房内坐。”敬济上的楼来,两个如鱼得水,似膝投胶,无非说些深情密意的话儿。爱姐砚台底下,露出一幅花笺,敬济取来观看。爱姐便说:“此是奴家盼你不来,作得一首诗,以消遣闷怀,恐污官人贵目。”敬济念了一遍,上写着:

\[
倦倚绣床愁懒动,闲垂锦帐鬓鬟低。
玉郎一去无消息,一日相思十二时。
\]
敬济看了,极口称羡不已。不一时,王六儿安排酒肴上楼,拨过镜架,就摆在梳妆卓上。两个并坐,爱姐筛酒一杯,双手递与敬济,深深道个万福,说:“官人一向不来,妾心无时不念。前八老来,又多谢盘缠,举家感之不尽。”敬济接酒在手,还了喏,说:“贱疾不安,有失期约,姐姐休怪。”酒尽,也筛一杯敬奉爱姐吃过,两个坐定,把酒来斟。王六儿、韩道国上来,也陪吃了几杯,各取方便下楼去了,教他二人自在吃几杯,叙些阔别话儿。良久,吃得酒浓时,情兴如火,免不得再把旧情一叙。交欢之际,无限恩情。穿衣起来,洗手更酌,又饮数杯。醉眼朦胧,余兴未尽。这小郎君,一向在家中不快,又心在爱姐,一向未与浑家行事。今日一旦见了情人,未肯一次即休。正是生死冤家,五百年前撞在一处,敬济魂灵都被他引乱。少顷,情窦复起,又干一度。自觉身体困倦,打熬不过,午饭也没吃,倒在床上就睡着了。

也是合当祸起,不想下边贩丝绵何官人来了,王六儿陪他在楼下吃酒。韩道国出去街上买菜蔬、肴品、果子来配酒。两个在下边行房。落后韩道国买将果菜来,三人又吃了几杯。约日西时分,只见洒家店坐地虎刘二,吃的酩酊大醉,軃开衣衫,露着一身紫肉,提着拳头走来酒楼下,大叫:“采出何蛮子来!”唬的两个主管见敬济在楼上睡,恐他听见,慌忙走出柜来,向前声诺,说道:“刘二哥,何官人并不曾来。”这刘二那里依听。大拔步撞入后边韩道国屋里,一手把门帘扯去半边,看见何官人正和王六儿并肩饮酒,心中大怒,便骂何官人:“贼狗男女,我肏你娘!那里没寻你,却在这里。你在我店中,占着两个粉头,几遭歇钱不与,又塌下我两个月房钱,却来这里养老婆!”那何官人忙出来道:“老二你休怪,我去罢。”那刘二骂道:“去你这狗入的!”不防飕的一拳来,正打在何官人面上,登时就青肿起来。那何官人也不顾,径夺门跑了。刘二将王六儿酒卓,一脚登翻,家活都打了。王六儿便骂道:“是那里少死的贼杀了!无事来老娘屋里放屁。娘不是耐惊耐怕儿的人!”被刘二向前一脚,跺了个仰八叉,骂道:“我入你淫妇娘!你是那里来的无名少姓私窠子?不来老爷手里报过,许你在这酒店内趁熟?还与我搬去!若搬迟,须吃我一顿好拳头。”那王六儿道:“你是那里来的光棍捣子?老娘就没了亲戚儿?许你便来欺负老娘,要老娘这命做甚么?”一头撞倒哭起来。刘二骂道:“我把淫妇肠子也踢断了,你还不知老爷是谁哩!”这里喧乱,两边邻舍并街上过往人,登时围看约有许多。有知道的旁边人说:“王六儿,你新来不知,他是守备老爷府中管事张虞候的小舅子,有名坐地虎刘二。在洒家店住,专一是打粉头的班头,降酒店的领袖。你让他些儿罢,休要不知利害。这地方人,谁敢惹他!”王六儿道:“还有大似他的,睬这杀才做甚么?”陆秉义见刘二打得凶,和谢胖子做好做歹,把他劝的去了。

陈敬济正睡在床上,听见楼下攘乱,便起来看,时天已日西时分,问:“那里攘乱?”那韩道国不知走的往那里去了,只见王六儿披发垢面上楼,如此这般告诉说:“那里走来一个杀才捣子,诨名唤坐地虎刘二,在洒家店住,说是咱府里管事张虞候小舅子。因寻酒店,无事把我踢打,骂了恁一顿去了。又把家活酒器都打得粉碎。”一面放声大哭起来。敬济就叫上两个主管去问。两个主管隐瞒不住,只得说:“是府中张虞候小舅子刘二,来这里寻何官人讨房钱,见他在屋里吃酒,不由分说,把帘子扯下半边来,打了何官人一拳,唬的何官人跑了。又和老韩娘子两个相骂,踢了一交,烘的满街人看。”敬济听了,便晓得是前番做道士,被他打的刘二了。欲要声张,又恐刘二泼皮行凶,一时斗他不过。又见天色晚了,因问:“刘二那厮如今在那里?”主管道:“被小人劝他回去了。”敬济安抚王六儿道:“你母子放心,有我哩,不妨事。你母子只情住着,我家去自有处置。”主管算了利钱银两递与他,打发起身上轿,伴当跟随。刚赶进城来,天已昏黑,心中甚恼。到家见了春梅,交了利息银两,归入房中。

一宿无话。到次日,心心念念要告春梅说,展转寻思:“且住,等我慢慢寻张胜那厮几件破绽,亦发教我姐姐对老爷说了,断送了他性命。叵耐这厮,几次在我身上欺心,敢说我是他寻得来,知我根本出身,量视我禁不得他。”正是:

\[
冤仇还报当如此,机会遭逢莫远图。
踏破铁鞋无觅处,得来全不费工夫。
\]

一日,敬济来到河下酒店内,见了爱姐母子,说:“外日吃惊。”又问陆主管道:“刘二那厮可曾走动?”陆主管道:“自从那日去了,再不曾来。”又问韩爱姐:“那何官人也没来行走?”爱姐道:“也没曾来。”这敬济吃了饭,算毕帐目,不免又到爱姐楼上。两个叙了回衷肠之话,干讫一度出来,因闲中叫过量酒陈三儿近前,如此这般,打听府中张胜和刘二几桩破绽。这陈三儿千不合,万不合,说出张胜包占着府中出来的雪娥,在洒家店做表子。刘二又怎的各处巢窝,加三讨利,举放私债,逞着老爷名坏事。这敬济听记在心,又与了爱姐二三两盘缠,和主管算了帐目,包了利息银两,作别骑头口来家。

闲话休题。一向怀意在心,一者也是冤家相凑,二来合当祸起。不料东京朝中徽宗天子,见大金人马犯边,抢至腹内地方,声息十分紧急。天子慌了,与大臣计议,差官往北国讲和,情愿每年输纳岁币,金银彩帛数百万。一面传位与太子登基,改宣和七年为靖康元年,宣帝号为钦宗。皇帝在位,徽宗自称太上道君皇帝,退居龙德宫。朝中升了李纲为兵部尚书,分部诸路人马。种师道为大将,总督内外军务。

一日,降了一道敕书来济南府,升周守备为山东都统制,提调人马一万,前往东昌府驻扎,会同巡抚都御史张叔夜,防守地方,阻挡金兵。守备领了敕书,不敢怠慢,一面叫过张胜、李安两个虞候近前分付,先押两车箱驮行李细软器物家去。原来在济南做了一年官,也撰得巨万金银。都装在行李驮箱内,委托二人押到家中:“交割明白,昼夜巡风仔细。我不日会同你巡抚张爷,调领四路兵马,打清河县起身。”二人当日领了钧旨,打点车辆,起身先行。一路无词。有日到了府中,交割明白,二人昼夜内外巡风,不在话下。

却说陈敬济见张胜押车辆来家,守备升了山东统制,不久将到,正欲把心腹中事要告诉春梅,等守备来家,发露张胜之事。不想一日因浑家葛翠屏往娘家回门住去了,他独自个在西书房寝歇,春梅蓦进房中看他。见丫鬟跟随,两个就解衣在房内云雨做一处。不防张胜摇着铃,巡风过来,到书院角门外,听见书房内仿佛有妇人笑语之声,就把铃声按住,慢慢走来窗下窃听。原来春梅在里面与敬济交媾。听得敬济告诉春梅说:“叵耐张胜那厮,好生欺压于我,说我当初亏他寻得来,几次在下人前败坏我。昨日见我在河下开酒店,一径使小舅子坐地虎刘二,来打我的酒店,把酒客都打散了。专一倚逞他在姐夫麾下,在那里开巢窝,放私债,又把雪娥隐占在外奸宿,只瞒了姐姐一人眼目。我几次含忍,不敢告姐姐说,趁姐夫来家,若不早说知,往后我定然不敢往河下做买卖去了。”春梅听了,说道:“这厮恁般无礼。雪娥那贱人,我卖了他,如何又留住在外?”敬济道:“他非是欺压我,就是欺压姐姐一般。”春梅道:“等他爷来家,交他定结果了这厮。”

常言道:“隔墙须有耳,窗外岂无人。”两个只管在内说,却不知张胜窗外听得明明白白,口中不言,心内暗道:“此时教他算计我,不如我先算计了他罢。”一面撇下铃,走到前边班房内,取了把解腕钢刀,说时迟,那时快,在石上磨了两磨,走入书院中来。不想天假其便,还是春梅不该死于他手。忽被后边小丫鬟兰花儿,慌慌走来叫春梅,报说:“小衙内金哥儿忽然风摇倒了,快请奶奶看去。”唬的春梅两步做一步走,奔了后房中看孩儿去了。刚进去了,那张胜提着刀子,径奔到书房内,不见春梅,只见敬济睡在被窝内。见他进来,叫道:“阿呀,你来做甚么?”张胜怒道:“我来杀你!你如何对淫妇说,倒要害我?我寻得你来不是了?反恩将仇报!常言“黑头虫儿不可救,救之就要吃人肉”,休走,吃我一刀子!明年今日是你死忌!”那敬济光赤条身子,没处躲,只搂着被,吃他拉过一边,向他身就扎了一刀子来。扎着软肋,鲜血就邈出来。这张胜见他挣扎,复又一刀去,攘着胸膛上,动弹不得了。一面采着头发,把头割下来,正是:

\[
三寸气在千般用,一日无常万事休。
\]

可怜敬济青春不上三九,死于非命。张胜提刀,绕屋里床背后,寻春梅不见,大拔步径望后厅走。走到仪门首,只见李安背着牌铃,在那里巡风。一见张胜凶神也似提着刀跑进来,便问:“那里去?”张胜不答,只顾走,被李安拦住。张胜就向李安戳一刀来。李安冷笑,说道:“我叔叔有名山东夜叉李贵,我的本事不用借。”早飞起右脚,只听忒楞的一声,把手中刀子踢落一边。张胜急了,两个就揪采在一处,被李安一个泼脚,跌番在地,解下腰间缠带登时绑了。嚷的后厅春梅知道,说:“张胜持刀入内,小的拿住了。”

那春梅方救得金哥苏醒,听言大惊失色。走到书院内,见敬济已被杀死在房中,一地鲜血横流,不觉放声大哭。一面使人报知浑家。葛翠屏慌奔家来,看见敬济杀死,哭倒在地,不省人事。被春梅扶救苏醒过来。拖过尸首,买棺材装殡。把张胜墩锁在监内,单等统制来家处治这件事。

那消数日,只见军情事务紧急,兵牌来催促。周统制调完各路兵马,张巡抚又早先往东昌府那里等候取齐。统制到家,春梅把杀死敬济一节说了。李安将凶器放在面前,跪禀前事。统制大怒,坐在厅上,提出张胜,也不问长短,喝令军牢,五棍一换,打一百棍,登时打死。随马上差旗牌快手,往河下捉拿坐地虎刘二,锁解前来。孙雪娥见拿了刘二,恐怕拿他,走到房中,自缢身死。旗牌拿刘二到府中,统制也分付打一百棍,当日打死。烘动了清河县,大闹了临清州。正是:

\[
平生作恶欺天,今日上苍报应。
\]
有诗为证:

\[
为人切莫用欺心,举头三尺有神明。
若还作恶无报应,天下凶徒人食人。
\]
当时统制打死二人,除了地方之害。分付李安将马头大酒店还归本主,把本钱收算来家。分付春梅在家,与敬济修斋做七,打发城外永福寺葬埋。留李安、周义看家,把周忠、周仁带去军门答应。春梅晚夕与孙二娘,置酒送饯,不觉簇地两行泪下,说:“相公此去,未知几时回还,出战之间,须要仔细。番兵猖獗,不可轻敌。”统制道:“你每自在家清心寡欲,好生看守孩儿,不必忧念。我既受朝廷爵禄,尽忠报国。至于吉凶存亡,付之天也。”嘱咐毕,过了一宿。次日,军马都在城外屯集,等候统制起程。一路无词。有日到了东昌府下,统制差一面令字蓝旗,打报进城。巡抚张叔夜,听见周统制人马来到,与东昌府知府达天道出衙迎接。至公厅叙礼坐下,商议军情,打听声息紧慢。驻马一夜,次日人马早行,往关上防守去了。不在话下。

却表韩爱姐母子,在谢家楼店中听见陈敬济已死,爱姐昼夜只是哭泣,茶饭都不吃,一心只要往城内统制府中,见敬济尸首一见,死也甘心。父母、旁人百般劝解不众。韩道国无法可处,使八老往统制府中打听,敬济灵柩已出了殡,埋在城外永福寺内。这八老走来,回了话。爱姐一心要到他坟上烧纸,哭一场,也是和他相交一场。做父母的只得依他。雇了一乘轿子,到永福寺中,问长老葬于何处。长老令沙弥引到寺后,新坟堆便是。这韩爱姐下了轿子,到坟前点着纸袋,道了万福,叫声:“亲郎我的哥哥!奴实指望和你同谐到老,谁想今日死了!”放声大哭,哭的昏晕倒了,头撞于地下,就死过去了。慌了韩道国和王六儿,向前扶救,叫姐姐,叫不应,越发慌了。

不想那日,正是葬的三日,春梅与浑家葛翠屏坐着两乘轿子,伴当跟随,抬三牲祭物,来与他暖墓烧纸。看见一个年小的妇人,穿着缟素,头戴孝髻,哭倒在地。一个男子汉和一中年妇人,搂抱他扶起来,又倒了,不省人事,吃了一惊。因问那男子汉是那里的,这韩道国夫妇向前施礼,把从前已往话,告诉了一遍:“这个是我的女孩儿韩爱姐。”春梅一闻爱姐之名,就想起昔日曾在西门庆家中会过,又认得王六儿。韩道国悉把东京蔡府中出来一节,说了一遍:“女孩儿曾与陈官人有一面之交,不料死了。他只要来坟前见他一见,烧纸钱,不想到这里,又哭倒了。”当下两个救了半日,这爱姐吐了口粘痰,方才苏醒,尚哽咽哭不出声来。痛哭了一场起来,与春梅、翠屏插烛也似磕了四个头,说道:“奴与他虽是露水夫妻,他与奴说山盟,言海誓,情深意厚,实指望和他同谐到老,谁知天不从人愿,一旦他先死了,撇得奴四脯着地。他在日曾与奴一方吴绫帕儿,上有四句情诗。知道宅中有姐姐,奴愿做小,倘不信——”向袖中取出吴绫帕儿来,上面写诗四句,春梅同葛翠屏看了。诗云:

\[
吴绫帕儿织回纹,洒翰挥毫墨迹新。
寄与多情韩五姐,永谐鸾凤百年情。
\]
爱姐道:“奴也有个小小鸳鸯锦囊,与他佩载在身边。两面都扣绣着并头莲,每朵莲花瓣儿一个字儿:寄与情郎陈君膝下。”春梅便问翠屏:“怎的不见这个香囊?”翠屏道:“在底裤子上拴着,奴替他装殓在棺椁内了。”

当下祭毕,让他母子到寺中摆茶饭,劝他吃了些。王六儿见天色将晚,催促他起身,他只顾不思动身。一面跪着春梅、葛翠屏哭说:“奴情愿不归父母,同姐姐守孝寡居。明日死,傍他魂灵,也是奴和他恩情一场,说是他妻小。”说着那泪如泉涌。翠屏只顾不言语。春梅便说:“我的姐姐,只怕年小青春,守不住,却不误了你好时光。”爱姐便道:“奶奶说那里话?奴既为他,虽刳目断鼻也当守节,誓不再配他人。”嘱付他父母:“你老公婆回去罢,我跟奶奶和姐姐府中去也。”那王六儿眼中垂泪,哭道:“我承望你养活俺两口儿到老,才从虎穴龙潭中夺得你来。今日倒闪赚了我。”那爱姐口里只说:“我不去了。你就留下我,到家也寻了无常。”那韩道国因见女儿坚意不去,和王六儿大哭一场,洒泪而别,回上临清店中去了。这韩爱姐同春梅、翠屏,坐轿子往府里来。那王六儿一路上悲悲切切,只是舍不的他女儿,哭了一场又一场。那韩道国又怕天色晚了,雇上两匹头口,望前赶路。正是:

\[
马迟心急路途穷,身似浮萍类转蓬。
只有都门楼上月,照人离恨各西东。
\]

\newpage
%# -*- coding:utf-8 -*-
%%%%%%%%%%%%%%%%%%%%%%%%%%%%%%%%%%%%%%%%%%%%%%%%%%%%%%%%%%%%%%%%%%%%%%%%%%%%%%%%%%%%%


\chapter{韩爱姐路遇二捣鬼\KG 普静师幻度孝哥儿}


诗曰:

\[
旧日豪华事已空,银屏金屋梦魂中。
黄芦晚日空残垒,碧草寒烟锁故宫。
隧道鱼灯油欲尽,妆台鸾镜匣长封。
凭谁话尽兴亡事,一衲闲云两袖风。
\]

话说韩道国与王六儿,归到谢家酒店内,无女儿,道不得个坐吃山崩,使陈三儿去,又把那何官人勾来续上。那何官人见地方中没了刘二,除了一害,依旧又来王六儿家行走,和韩道国商议:“你女儿爱姐,只是在府中守孝,不出来了,等我卖尽货物,讨了赊帐,你两口跟我往湖州家去罢,省得在此做这般道路。”韩道国说:“官人下顾,可知好哩。”一日卖尽了货物,讨上赊帐,雇了船,同王六儿跟往湖州去了,不题。

却表爱姐在府中,与葛翠屏两个持贞守节,姊妹称呼,甚是合当。白日里与春梅做伴儿在一处。那时金哥儿大了,年方六岁。孙二娘所生玉姐年长十岁,相伴两个孩儿,便没甚事做。

谁知自从陈敬济死后,守备又出征去了。这春梅每日珍馐百味,绫锦衣衫,头上黄的金,白的银,圆的珠,光照的无般不有。只是晚夕难禁独眠孤枕,欲火烧心。因见李安一条好汉,只因打杀张胜,巡风早晚十分小心。

一日,冬月天气,李安正在班房内上宿,忽听有人敲后门,忙问道:“是谁?”只闻叫道:“你开门则个。”李安连忙开了房门,却见一个人抢入来,闪身在灯光背后。李安看时,却认得是养娘金匮。李安道:“养娘,你这咱晚来有甚事?”金匮道:“不是我私来,里边奶奶差出我来的。”李安道:“奶奶叫你来怎么?”金匮笑道:“你好不理会得。看你睡了不曾,教我把一件物事来与你。”向背上取下一包衣服,“把与你,包内又有几件妇女衣服与你娘。前日多累你押解老爷行李车辆,又救得奶奶一命,不然也吃张胜那厮杀了。”说毕,留下衣服,出门走了两步,又回身道:“还有一件要紧的。”又取出一锭五十两大元宝来,撇与李安自去了。

当夜踌躇不决。次早起来,径拿衣服到家与他母亲。做娘的问道:“这东西是那里的?”李安把夜来事说了一遍。做母亲的听言叫苦:“当初张胜干坏事,一百棍打死,他今日把东西与你,却是甚么意思?我今六十已上年纪,自从没了你爹爹,满眼只看着你,若是做出事来,老身靠谁?明早便不要去了。”李安道:“我不去,他使人来叫,如何答应?”婆婆说:“我只说你感冒风寒病了。”李安道:“终不成不去,惹老爷不见怪么?”做娘的便说:“你且投到你叔叔,山东夜叉李贵那里住上几个月,再来看事故何如。”这李安终是个孝顺的男子,就依着娘的话,收拾行李,往青州府投他叔叔李贵去了。春梅以后见李安不来,三、四、五次使小伴当来叫。婆婆初时答应家中染病,次后见人来验看,才说往原籍家中,讨盘缠去了。这春梅终是恼恨在心不题。

时光迅速,日月如梭,又早腊尽阳回,正月初旬天气。统制领兵一万三千,在东昌府屯住已久,使家人周忠,捎书来家。教搬取春梅、孙二娘,并金哥、玉姐家小上车。止留下周忠:“东庄上请你二爷看守宅子。”原来统制还有个族弟周宣,在庄上住。周忠在府中,与周宣、葛翠屏、韩爱姐看守宅子。周仁与众军牢保定车辆,往东昌府来。此一去,不为身名离故土,争知此去少回程。有词一篇,单道周统制果然是一员好将材。当此之时,中原荡扫,志欲吞胡。但见:

\[
四方盗起如屯峰,狼烟烈焰薰天红。
将军一怒天下安,腥膻扫尽夷从风。
公事忘私愿已久,此身许国不知有。
金戈抑日酬战征,麒麟图画功为首。
雁门关外秋风烈,铁衣披张卧寒月。
汗马卒勤二十年,赢得斑斑鬓如雪。
天子明见万里余,几番劳勣来旌书。
肘悬金印大如斗,无负堂堂七尺躯。
\]

有日,周仁押家眷车辆到于东昌。统制见了春梅、孙二娘、金哥、玉姐,众丫鬟家小都到了,一路平安,心中大喜。就在统制府衙后厅居住。周仁悉把“东庄上请了二爷来宅内,同小的老子周忠看守宅舍”,说了一遍。周统制又问:“怎的李安不见?”春梅道:“又题甚李安?那厮我因他捉获了张胜,好意赏了他两件衣服,与他娘穿。他到晚夕巡风,进入后厅,把他二爷东庄上收的子粒银——一包五十两,放在明间卓上,偷的去了。几番使伴当叫他,只是推病不来。落后又使叫去,他躲的上青州原籍家去了。”统制便道:“这厮我倒看他,原来这等无恩!等我慢慢差人拿他去。”这春梅也不题起韩爱姐之事。

过了几日,春梅见统制日逐理论军情,干朝廷国务,焦心劳思,日中尚未暇食,至于房帏色欲之事,久不沾身。因见老家人周忠次子周义,年十九岁,生的眉清目秀,眉来眼去,两个暗地私通,就勾搭了。朝朝暮暮,两个在房中下棋饮酒,只瞒过统制一人不知。

一日,不想北国大金皇帝灭了辽国。又见东京钦宗皇帝登基,集大势番兵,分两路寇乱中原。大元帅粘没喝,领十万人马,出山西太原府井陉道,来抢东京;副帅斡离不由檀州来抢高阳关。边兵抵挡不住,慌了兵部尚书李纲、大将种师道,星夜火牌羽书,分调山东、山西、河南、河北、关东、陕西分六路统制人马,各依要地,防守截杀。那时陕西刘延庆领延绥之兵,关东王禀领汾绛之兵,河北王焕领魏搏之兵,河南辛兴宗领彰德之兵,山西杨惟忠领泽潞之兵,山东周秀领青兖之兵。

却说周统制,见大势番兵来抢边界,兵部羽书火牌星火来,连忙整率人马,全装披挂,兼道进兵。比及哨马到高阳关上,金国干离不的人马,已抢进关来,杀死人马无数。正值五月初旬,黄沙四起,大风迷目。统制提兵进赶,不防被干离不兜马反攻,没鞦一箭,正射中咽喉,随马而死。众番将就用钩索搭去,被这边将士向前仅抢尸首,马戴而远,所伤军兵无数。可怜周统制一旦阵亡,亡年四十七岁。正是:

\[
于家为国忠良将,不辩贤愚血染沙。
\]
古人意不尽,作诗一首,以叹之曰:

\[
胜败兵家不可期,安危端自命为之。
出师未捷身先丧,落日江流不胜悲。
\]
巡抚张叔夜,见统制没于阵上,连忙鸣金收军,查点折伤士卒,退守东昌。星夜奏朝廷,不在话下。部下士卒,载尸首还到东昌府。春梅合家大小,号哭动天,合棺木盛殓,交割了兵符印信。一日,春梅与家人周仁,发丧载灵柩归清河县不题。

话分两头。单表葛翠屏与韩爱姐,自从春梅去后,两个在家清茶淡饭,守节持贞,过其日月。正值春尽夏初天气,景物鲜明,日长针指困倦。姊妹二人闲中徐步,到西书院花亭上。见百花盛开,莺啼燕语,触景伤情。葛翠屏心还坦然,这韩爱姐,一心只想念陈敬济,凡事无情无绪,睹物伤悲,不觉潸然泪下。姊妹二人正在悲凄之际,只见二爷周宣,走来劝道:“你姊妹两个少要烦恼,须索解叹。我连日做得梦,有些不吉。梦见一张弓挂在旗竿上,旗竿折了,不知是凶是吉?”韩爱姐道:“倒只怕老爷边上,有些说话。”正在犹疑之间,忽见家人周仁,挂着一身孝,慌慌张张走来,报道:“祸事,老爷如此这般,五月初七日,在边关上阵亡了!大奶奶、二奶奶家眷,载着灵车都来了。”慌了二爷周宣,收拾打扫前厅干净,停放灵柩,摆下祭祀,合家大小,哀号起来。一面做斋累七,僧道念经。金哥、玉姐披麻带孝,吊客往来,择日出殡,安葬于祖茔。俱不必细说。

却说二爷周宣,引着六岁金哥儿,行文书申奏朝廷,讨祭葬,袭替祖职。朝廷明降,兵部覆题引奏:已故统制周秀,奋身报国,没于王事,忠勇可嘉。遣官谕祭一坛,墓顶追封都督之职。伊子照例优养,出幼袭替祖职。

这春梅在内颐养之余,淫情愈盛。常留周义在香阁中,镇日不出。朝来暮往,淫欲无度,生出骨蒸痨病症。逐日吃药,减了饮食,消了精神,体瘦如柴,而贪淫不已。一日,过了他生辰,到六月伏暑天气,早辰晏起,不料他搂着周义在床上,一泄之后,鼻口皆出凉气,淫津流下一洼口,就鸣呼哀哉,死在周义身上。亡年二十九岁。这周义见没了气儿,就慌了手脚,向箱内抵盗了些金银细软,带在身边,逃走出外。丫鬟养娘不敢隐匿,报与二爷周宣得知。把老家人周忠锁了,押着抓寻周义。可霎作怪,正走在城外他姑娘家投住,一条索子拴将来。已知其情,恐扬出丑去,金哥久后不可袭职,拿到前厅,不由分说,打了四十大棍,即时打死。把金哥与孙二娘看着。一面发丧于祖茔,与统制合葬毕。房中两个养娘并海棠、月桂,都打发各寻投向嫁人去了。止有葛翠屏与韩爱姐,再三劝他,不肯前去。

一日,不想大金人马抢了东京汴梁,太上皇帝与靖康皇帝,都被虏上北地去了。中原无主,四下荒乱。兵戈匝地,人民逃窜。黎庶有涂炭之哭,百姓有倒悬之苦。大势番兵已杀到山东地界,民间夫逃妻散,鬼哭神号,父子不相顾。葛翠屏已被他娘家领去,各逃生命。止丢下韩爱姐,无处依倚,不免收拾行装,穿着随身惨淡衣衫,出离了清河县,前往临清找寻他父母。到临清谢家店,店也关闭,主人也走了。不想撞见陈三儿,三儿说:“你父母去年就跟了何官人,往江南湖州去了。”

这韩爱姐一路上怀抱月琴,唱小词曲,往前抓寻父母。随路饥餐渴饮,夜住晓行,忙忙如丧家之犬,急急如漏网之鱼。弓鞋又小,千辛万苦。行了数日,来到徐州地方,天色晚了,投在孤村里面。一个婆婆,年纪七旬之上,正在灶上杵米造饭。这韩爱姐便向前道了万福,告道:“奴家是清河县人氏,因为荒乱,前往江南投亲,不期天晚,权借婆婆这里投宿一宵,明早就行,房金不少。”那婆婆看这女子,不是贫难人家婢女,生得举止典雅,容貌非俗。因说道:“既是投宿,娘子请炕上坐,等老身造饭,有几个挑河夫子来吃。”那老婆婆炕上柴灶,登时做出一大锅稗稻插豆子干饭,又切了两大盘生菜,撮上一包盐,只见几个汉子,都蓬头精腿,裈裤兜裆,脚上黄泥,进来放下锹镢,便问道:“老娘有饭也未?”婆婆道:“你每自去盛吃。”

当下各取饭菜,四散正吃。只见内一人,约四十四五年纪,紫面黄发,便问婆婆:“这炕上坐的是甚么人?”婆婆道:“此位娘子,是清河县人氏,前往江南寻父母去,天晚在此投宿。”那人便问:“娘子,你姓甚么?”爱姐道:“奴家姓韩,我父亲名韩道国。”那人向前扯住问道:“姐姐,你不是我侄女韩爱姐么?”那爱姐道:“你倒好似我叔叔韩二。”两个抱头相哭做一处。因问:“你爹娘在那里?你在东京,如何至此?”这韩爱姐一五一十,从头说了一遍,“因我嫁在守备府里,丈夫没了,我守寡到如今。我爹娘跟了何官人,往湖州去了。我要找寻去,荒乱中又没人带去,胡乱单身唱词,觅些衣食前去,不想在这里撞见叔叔。”那韩二道:“自从你爹娘上东京,我没营生过日,把房儿卖了,在这里挑河做夫子,每日觅碗饭吃。既然如此,我和你往湖州,寻你爹娘去。”爱姐道:“若是叔叔同去,可知好哩。”当下也盛了一碗饭,与爱姐吃。爱姐呷了一口,见粗饭,不能咽,只呷了半碗,就不吃了。一宿晚景题过。

到次日到明,众夫子都去了,韩二交纳了婆婆房钱,领爱姐作辞出门,望前途所进。那韩爱姐本来娇嫩,弓鞋又小,身边带着些细软钗梳,都在路上零碎盘缠。将到淮安上船,迤逶望江南湖州来,非止一日,抓寻到湖州何官人家,寻着父母,相见会了。不想何官人已死,家中又没妻小,止是王六儿一人,丢下六岁女儿,有几顷水稻田地。不上一年,韩道国也死了。王六儿原与韩二旧有揸儿,就配了小叔,种田过日。那湖州有富家子弟,见韩爱姐生的聪明标致,都来求亲。韩二再三教他嫁人,爱姐割发毁目,出家为尼,誓不再配他人。后来至三十一岁,无疾而终。正是:

\[
贞骨未归三尺土,怨魂先彻九重天。
\]
后韩二与王六儿成其夫妇,请受何官人家业田地,不在话下。

却说大金人马,抢过东昌府来,看看到清河县地界。只见官吏逃亡,城门昼诸,人民逃窜,父子流亡。但见:

\[
烟生四野,日蔽黄沙。封豕长蛇,互相吞噬。龙争虎斗,各自争强。皂帜红旗,布满郊野。男啼女哭,万户惊惶。番军虏将,一似蚁聚蜂屯;短剑长枪,好似森森密竹。一处处死尸朽骨,横三竖四;一攒攒折刀断剑,七断八截。个个携男抱女,家家闭门关户。十室九空,不显乡村城郭;獐奔鼠窜,那契礼乐衣冠。正是:得多少宫人红袖哭,王子白衣行。
\]

那时,吴月娘见番兵到了,家家都关锁门户,乱窜逃去,不免也打点了些金珠宝玩,带在身边。那时吴大舅已死,止同吴三舅、玳安、小玉,领着十五岁孝哥儿,把家中前后都倒锁了,要往济南府投奔云理守。一来避兵,二者与孝哥完就亲事。一路上只见人人荒乱,个个惊骇。可怜这吴月娘,穿着随身衣服,和吴二舅男女五口,杂在人队里挨出城门,到于郊外,往前奔行。到于空野十字路口,只见一个和尚,身披紫褐袈裟,手执九环锡杖,脚趿芒鞋,肩上背着条布袋,袋内裹着经典,大移步迎将来,与月娘打了个问讯,高声大叫道:“吴氏娘子,你到那里去?还与我徒弟来!”唬的月娘大惊失色,说道:“师父,你问我讨甚么徒弟?”那和尚又道:“娘子,你休推睡里梦里,你曾记的十年前,在岱岳东峰,被殷天锡赶到我山洞中投宿。我就是那雪洞老和尚,法号普静。你许下我徒弟,如何不与我?”吴二舅便道:“师父出家人,如何不近道?此等荒乱年程,乱窜逃生,他有此孩儿,久后还要接代香火,他肯舍与你出家去?”和尚道:“你真个不与我去?”吴二舅道:“师父,你休闲说,误了人的去路。后面只怕番兵来到,朝不保暮。”和尚道:“你既不与我徒弟,如今天色已晚,也走不出路去。番人就来,也不到此处,你且跟我到这寺中歇一夜,明早去罢。”吴月娘问:“师父,是那寺中?”那和尚用手只一指,道:“那路旁便是。”和尚引着来到永福寺。吴月娘认的是永福寺,曾走过一遭。

比及来到寺中,长老僧众都走去大半,止有几个禅和尚在后边打座。佛前点着一大盏硫璃海灯,烧看一炉香。已是日色衔山时分,当晚吴月娘与吴二舅、玳安、小玉、孝哥儿,男女五口儿,投宿在寺中方丈内。小和尚有认的,安排了些饭食,与月娘等吃了。那普静老师,跏趺在禅堂床上敲木鱼,口中念经。月娘与孝哥儿、小玉在床上睡,吴二舅和玳安做一处,着了荒乱辛苦底人,都睡着了。止有小玉不曾睡熟,起来在方丈内,打门缝内看那普静老师父念经。看看念至三更时,只见金风凄凄,斜月朦朦,人烟寂静,万籁无声。佛前海灯,半明不暗。这普静老师见天下荒乱,人民遭劫,阵亡横死者极多,发慈悲心,施广惠力,礼白佛言,荐拔幽魂,解释宿冤,绝去挂碍,各去超生。于是诵念了百十遍解冤经咒。少顷,阴风凄凄,冷气飕飕。有数十辈焦头烂额,蓬头泥面者,或断手折臂者,或有刳腹剜心者,或有无头跛足者,或有吊颈枷锁者,都来悟领禅师经咒,列于两旁。禅师便道:“你等众生,冤冤相报,不肯解脱,何日是了?汝当谛听吾言,随方托化去罢。偈曰:

\[
劝尔莫结冤,冤深难解结。一日结成冤,千日解不彻。
若将冤解冤,如汤去泼雪。我见结冤人,尽被冤磨折。
我今此忏悔,各把性悟彻。照见本来心,冤愆自然雪。
仗此经力深,荐拔诸恶业。汝当各托生,再勿将冤结。
\]

当下众魂都拜谢而去。小玉窃看,都不认得。少顷,又一大汉进来,身长七尺,形容魁伟,全装贯甲,胸前关着一矢箭,自称“统制周秀,因与番将对敌,折于阵上,今蒙师荐拔,今往东京,托生于沈镜为次子,名为沈守善去也。”言未已,又一人,素体荣身,口称是清河县富户西门庆,“不幸溺血而死,今蒙师荐拔,今往东京城内,托生富户沈通为次子沈越去也。”小玉认的是他爹,唬的不敢言语。已而又有一人,提着头,浑身皆血,自言是陈敬济,“因被张胜所杀,蒙师经功荐拔,今往东京城内,与王家为子去也。”已而又见一妇人,也提着头,胸前皆血。自言:“奴是武大妻、西门庆之妾潘氏是也。不幸被仇人武松所杀。蒙师荐拔,今往东京城内黎家为女托生去也。”已而又有一人,身躯矮小,面背青色,自言是武植,“因被王婆唆潘氏下药吃毒而死,蒙师荐拔,今往徐州乡民范家为男,托生去也。”已而又有一妇人,面色黄瘦,血水淋漓,自言:“妾身李氏,乃花子虚之妻,西门庆之妾,因害血山崩而死。蒙师荐拔,今往东京城内,袁指挥家托生为女去也。”已而又一男,自言花子虚,“不幸被妻气死,蒙师荐拔,今往东京郑千户家托生为男。”已而又见一女人,颈缠脚带,自言西门庆家人来旺妻宋氏,“自缢身死,蒙师荐拔,今往东京朱家为女去也。”已而又一妇人,面黄肌瘦,自言周统制妻庞氏春梅,“因色痨而死,蒙师荐拔,今往东京与孔家为女,托生去也。”已而又一男子,裸形披发,浑身杖痕,自言是打死的张胜,“蒙师荐拔,今往东京大兴卫贫人高家为男去也。”已而又有一女人,项上缠着索子,自言是西门庆妾孙雪娥,不幸自缢身死,“蒙师荐拔,今往东京城外贫民姚家为女去也。”已而又一女人,年小,项缠脚带,自言“西门庆之女,陈敬济之妻,西门大姐是也,不幸亦缢身死,蒙师荐拔,今往东京城外,与番役钟贵为女,托生去也。”已而又见一小男子,自言周义,“亦被打死,蒙师荐拔,今往东京城外高家为男,名高留住儿,托生去也。”言毕,各恍然不见。小玉唬的战栗不已。原来这和尚,只是和这些鬼说话。

正欲向床前告诉吴月娘,不料月娘睡得正熟,一灵真性,同吴二舅众男女,身带着一百颗胡珠,一柄宝石绦环,前往济南府,投奔亲家云理守。一路到于济南府,寻问到云参将寨门,通报进去。云参将听见月娘送亲来了,一见如故。叙毕礼数。原来新近没了娘子,央浼邻舍王婆来陪待月娘,在后堂酒饭,甚是丰盛。吴二舅、玳安另在一处管待。因说起避兵就亲之事,因把那百颗胡珠、宝石、绦环教与云理守,权为茶礼。云理守收了,并不言其就亲之事。到晚,又教王婆陪月娘一处歇卧。将言说念月娘,以挑探其意,说:“云理守虽武官,乃读书君子,从割衫襟之时,就留心娘子。不期夫人没了,鳏居至今。今据此山城,虽是任小,上马管军,下马管民,生杀在于掌握。娘子若不弃,愿成伉俪之欢,一双两好,令郎亦得谐秦晋之配。等待太平之日,再回家去不迟。”月娘听言,大惊失色,半晌无言。这王婆回报云理寺。

次日夕晚,置酒后堂,请月娘吃酒。月娘只知他与孝哥儿完亲,连忙来到席前叙坐。云理守乃道:“嫂嫂不知,下官在此虽是山城,管着许多人马,有的是财帛衣服,金银宝物,缺少一个主家娘子。下官一向思想娘子,如喝思浆,如热思凉。不想今日娘子到我这里与令郎完亲,天赐姻缘,一双两好,成其夫妇,在此快活一世,有何不可?”月娘听了,心中大怒,骂道:“云理守,谁知你人皮包着狗骨!我过世丈夫不曾把你轻待,如何一旦出此犬马之言?”云理守笑嘻嘻向前,把月娘搂住,求告说:“娘子,你自家中,如何走来我这里做甚?自古上门买卖好做,不知怎的,一见你,魂灵都被你摄在身上。没奈何,好歹完成了罢。”一面拿过酒来和月娘吃。月娘道:“你前边叫我兄弟来,等我与他说句话。”云理守笑道:“你兄弟和玳安儿小厮,已被我杀了。”即令左右:“取那件物事,与娘子看。”不一时,灯光下,血沥沥提了吴二舅、玳安两颗头来。唬的月娘面如土色,一面哭倒在地。被云理守向前抱起:“娘子不须烦恼,你兄弟已死,你就与我为妻。我一个总兵官,也不玷辱了你。”月娘自思道:“这贼汉将我兄弟家人害了命,我若不从,连我命也丧了。”乃回嗔作喜,说道:“你须依我,奴方与你做夫妻。”云理守道:“不拘甚事,我都依。”月娘道:“你先与我孩儿完了房,我却与你成婚。”云理守道:“不打紧。”一面叫出云小姐来,和孝哥儿推在一处,饮合卺杯,绾同心结,成其夫妇。然后扯月娘和他云雨。这月娘却拒阻不肯,被云理守忿然大怒,骂道:“贱妇!你哄的我与你儿子成了婚姻,敢笑我杀不得你的孩儿?”向床头提剑,随手而落,血溅数步之远。正是:

\[
三尺利刀着项上,满腔鲜血湿模糊。
\]

月娘见砍死孝哥儿,不觉大叫一声。不想撒手惊觉,却是南柯一梦。唬的浑身是汗,遍体生津。连道:“怪哉,怪哉。”小玉在旁,便问:“奶奶怎的哭?”月娘道:“适间做得一梦不详。”不免告诉小玉一遍。小玉道:“我倒刚才不曾睡着,悄悄打门缝见那和尚原来和鬼说了一夜话。刚才过世俺爹、五娘、六娘和陈姐夫、周守备、孙雪娥、来旺儿媳妇子、大姐都来说话,各四散去了。”月娘道:“这寺后见埋着他每,夜静时分,屈死淹魂如何不来!”

娘儿们说了回话,不觉五更,鸡叫天明。吴月娘梳洗面貌,走到禅堂中,礼佛烧香。只见普静老师在禅床上高叫:“那吴氏娘子,你如何可省悟得了么?”这月娘便跪下参拜:“上告尊师,弟子吴氏,肉眼凡胎,不知师父是一尊古佛。适间一梦中都已省悟了。”老师道:“既已省悟,也不消前去,你就去,也无过只是如此。倒没的丧了五口儿性命。你这儿子,有分有缘遇着我,都是你平日一点善根所种。不然,定然难免骨肉分离。当初,你去世夫主西门庆造恶非善,此子转身托化你家,本要荡散其财本,倾覆其产业,临死还当身首羿处。今我度脱了他去,做了徒弟,常言‘一子出家,九祖升天’,你那夫主冤愆解释,亦得超生去了。你不信,跟我来,与你看一看。”于是叉步来到方丈内,只见孝哥儿还睡在床上。老师将手中禅杖,向他头上只一点,教月娘众人看。忽然翻过身来,却是西门庆,项带沉枷,腰系铁索。复用禅杖只一点,依旧是孝哥儿睡在床上。月娘见了,不觉放声大哭,原来孝哥儿即是西门庆托生。

良久,孝哥儿醒了。月娘问他:“如何你跟了师父出家。”在佛前与他剃头,摩顶受记。可怜月娘扯住恸哭了一场,干生受养了他一场。到十五岁,指望承家嗣业,不想被这老师幻化去了。吴二舅、小玉、玳安亦悲不胜。当下这普静老师,领定孝哥儿,起了他一个法名,唤做明悟。作辞月娘而去。临行,分付月娘:“你们不消往前途去了。如今不久番兵退去,南北分为两朝,中原已有个皇帝,多不上十日,兵戈退散,地方宁静了,你每还回家去安心度日。”月娘便道:“师父,你度托了孩儿去了,甚年何日我母子再得见面?”不觉扯住,放声大哭起来。老师便道:“娘子休哭!那边又有一位老师来了。”哄的众人扭颈回头,当下化阵清风不见了。正是:

\[
三降尘寰人不识,倏然飞过岱东峰。
\]

不说普静老师幻化孝哥儿去了,且说吴月娘与吴二舅众人,在永福寺住了十日光景,果然大金国立了张邦昌在东京称帝,置文武百官。徽宗、钦宗两君北,康王泥马渡江,在建康即位,是为高宗皇帝。拜宗泽为大将,复取山东、河北。分为两朝,天下太平,人民复业。后月娘归家,开了门户,家产器物都不曾疏失。后就把玳安改名做西门庆,承受家业,人称呼为“西门小员外”。养活月娘到老,寿年七十岁,善终而亡。此皆平日好善看经之报。有诗为证:

\[
阀阅遗书思惘然,谁知天道有循环。
西门豪横难存嗣,敬济颠狂定被歼。
楼月善良终有寿,瓶梅淫佚早归泉。
可怪金莲遭恶报,遗臭千年作话传。
\]

\newpage


\end{document}
