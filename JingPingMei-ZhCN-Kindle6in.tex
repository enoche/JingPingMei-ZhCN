%!TEX program = xelatex
% 完整编译: xelatex -> biber/bibtex -> xelatex -> xelatex
\documentclass[lang=cn,12pt,twoside]{elegantpaper}  % uncomment for kindle

% 本书使用的中文字体
\setCJKmainfont[BoldFont={Source Han Serif SC Bold}, ItalicFont={LiSu}]{SimHei}
%\setCJKmainfont[BoldFont={Source Han Serif SC Bold}, ItalicFont={LiSu}]{Source Han Serif SC SemiBold}
\setCJKsansfont{Microsoft YaHei}
\setCJKmonofont{STFangsong}

%\setCJKfamilyfont{Askaisong}{AaKaisong2wanzijf}
%\newcommand{\kaisong}{\CJKfamily{Askaisong}}
\setCJKfamilyfont{Askaisong}{CloudLiBianGBK}
\newcommand{\kaisong}{\CJKfamily{Askaisong}}
\setCJKfamilyfont{AHeiti}{SimHei}
\newcommand{\simhei}{\CJKfamily{AHeiti}}

\title{金瓶梅【崇祯本】}
\author{{\kaisong 兰陵笑笑生} \\ 作者 \and {\kaisong 南洋一粟} \\  编辑\footnote{勘误、排版\LaTeX{}:https://enoche.github.io/}}
\date{}


\titleformat{\section}{\simhei}{\thesection}{1em}{}
\renewcommand{\thesection}{第\chinese{section}回} 



% 设置章节编号
%\renewcommand\thesection{{第\chinese{section}回、}}
%\renewcommand\thesubsection{\arabic{subsection}.}
%\renewcommand\thesubsubsection{\alph{subsubsection})}
% 设置章节编号与章节标题的距离
%\makeatletter
%\renewcommand\@seccntformat[1]{%
%	{\csname the#1\endcsname}\hspace{0.1em}
%}
%\makeatother


\usepackage{array}

\newcommand{\ccr}[1]{\makecell{{\color{#1}\rule{1cm}{1cm}}}}
\renewcommand{\abstractname}{序}

\newcommand{\KG}{\quad}%
\newcommand{\chapter}{\section}%

\newenvironment{cipaim}[1]{【#1】}
%\setlength{\footskip}{16pt}

\usepackage[
	open,
	openlevel=2,
	atend,
	numbered
]{bookmark}

\usepackage[
paperheight=140mm,   % uncomment for Kindle
paperwidth=110mm,
left=0.1in,
right=0.1in,
top=0.35in,
bottom=0.1in,
headheight=40pt,
headsep=10pt]{geometry}

\usepackage{lastpage}
\usepackage{fancyhdr}
\pagestyle{fancy}
\renewcommand{\sectionmark}[1]{\markright{\thesection ~ \ #1}}
\fancyhead{} % clear all fields
\fancyhead[LE,RO]{\footnotesize $\cdot$\enskip\thepage\enskip$\cdot$}
%\fancyhead[RE]{\yahei{\rightmark}}
%\fancyhead[LO]{\yahei 金瓶梅【崇祯本】}
\fancyhead[RE]{\footnotesize {\kaisong{\rightmark}}}
\fancyhead[LO]{\footnotesize {\kaisong 金瓶梅【崇祯本】}}
\fancyfoot{}
%\rhead{$\cdot$ \thepage $\cdot$}

\renewcommand{\headrulewidth}{1.0pt}

\begin{document}

%\vspace*{\textheight}
%\columnbreak

\maketitle
\thispagestyle{empty}

\newpage

\begin{abstract}
\thispagestyle{empty}
{\kaisong\footnotesize 《金瓶梅》,秽书也。袁石公亟称之,亦自寄其牢骚耳,非有取于《金瓶梅》也。然作者亦自有意,盖为世戒,非为世劝也。如诸妇多矣,而独以潘金莲、李瓶儿、春梅命名者,亦楚《梼杌》之意也。盖金莲以奸死,瓶儿以孽死,春梅以淫死,较诸妇为更惨耳。借西门庆以描画世之大净,应伯爵以描绘世之小丑,诸淫妇以描画世之丑婆、净婆,令人读之汗下。盖为世戒,非为世劝也。

余尝曰:“读《金瓶梅》而生怜悯心者,菩萨也;生畏惧心者,君子也;生欢喜心者,小人也;生效法心者,乃禽兽耳。”余友人褚孝秀偕一少年同赴歌舞之筵,衍至霸王夜宴,少年垂涎曰:“男儿何可不如此!”褚孝秀曰:“也只为这乌江设此一着耳。”同座闻之,叹为有道之言。若有人识得此意,方许他读《金瓶梅》也。不然,石公几为导淫宣欲之尤矣。奉劝世人,勿为西门之后车可也。

\bigskip\mbox{}\large\hfill 东吴弄珠客题}
\end{abstract}
\newpage
\pagestyle{plain}
\tableofcontents
\newpage
\pagestyle{fancy}

\setcounter{page}{1}
\input{chapters/JPM001}
\newpage
\input{chapters/JPM002}
\newpage
\input{chapters/JPM003}
\newpage
\input{chapters/JPM004}
\newpage
\input{chapters/JPM005}
\newpage
\input{chapters/JPM006}
\newpage
\input{chapters/JPM007}
\newpage
\input{chapters/JPM008}
\newpage
\input{chapters/JPM009}
\newpage
\input{chapters/JPM010}
\newpage
\input{chapters/JPM011}
\newpage
\input{chapters/JPM012}
\newpage
\input{chapters/JPM013}
\newpage
\input{chapters/JPM014}
\newpage
\input{chapters/JPM015}
\newpage
\input{chapters/JPM016}
\newpage
\input{chapters/JPM017}
\newpage
\input{chapters/JPM018}
\newpage
\input{chapters/JPM019}
\newpage
\input{chapters/JPM020}
\newpage
\input{chapters/JPM021}
\newpage
\input{chapters/JPM022}
\newpage
\input{chapters/JPM023}
\newpage
\input{chapters/JPM024}
\newpage
\input{chapters/JPM025}
\newpage
\input{chapters/JPM026}
\newpage
\input{chapters/JPM027}
\newpage
\input{chapters/JPM028}
\newpage
\input{chapters/JPM029}
\newpage
\input{chapters/JPM030}
\newpage
\input{chapters/JPM031}
\newpage
\input{chapters/JPM032}
\newpage
\input{chapters/JPM033}
\newpage
\input{chapters/JPM034}
\newpage
\input{chapters/JPM035}
\newpage
\input{chapters/JPM036}
\newpage
\input{chapters/JPM037}
\newpage
\input{chapters/JPM038}
\newpage
\input{chapters/JPM039}
\newpage
\input{chapters/JPM040}
\newpage
\input{chapters/JPM041}
\newpage
\input{chapters/JPM042}
\newpage
\input{chapters/JPM043}
\newpage
\input{chapters/JPM044}
\newpage
\input{chapters/JPM045}
\newpage
\input{chapters/JPM046}
\newpage
\input{chapters/JPM047}
\newpage
\input{chapters/JPM048}
\newpage
\input{chapters/JPM049}
\newpage
\input{chapters/JPM050}
\newpage
\input{chapters/JPM051}
\newpage
\input{chapters/JPM052}
\newpage
\input{chapters/JPM053}
\newpage
\input{chapters/JPM054}
\newpage
\input{chapters/JPM055}
\newpage
\input{chapters/JPM056}
\newpage
\input{chapters/JPM057}
\newpage
\input{chapters/JPM058}
\newpage
\input{chapters/JPM059}
\newpage
\input{chapters/JPM060}
\newpage
\input{chapters/JPM061}
\newpage
\input{chapters/JPM062}
\newpage
\input{chapters/JPM063}
\newpage
\input{chapters/JPM064}
\newpage
\input{chapters/JPM065}
\newpage
\input{chapters/JPM066}
\newpage
\input{chapters/JPM067}
\newpage
\input{chapters/JPM068}
\newpage
\input{chapters/JPM069}
\newpage
\input{chapters/JPM070}
\newpage
\input{chapters/JPM071}
\newpage
\input{chapters/JPM072}
\newpage
\input{chapters/JPM073}
\newpage
\input{chapters/JPM074}
\newpage
\input{chapters/JPM075}
\newpage
\input{chapters/JPM076}
\newpage
\input{chapters/JPM077}
\newpage
\input{chapters/JPM078}
\newpage
\input{chapters/JPM079}
\newpage
\input{chapters/JPM080}
\newpage
\input{chapters/JPM081}
\newpage
\input{chapters/JPM082}
\newpage
\input{chapters/JPM083}
\newpage
\input{chapters/JPM084}
\newpage
\input{chapters/JPM085}
\newpage
\input{chapters/JPM086}
\newpage
\input{chapters/JPM087}
\newpage
\input{chapters/JPM088}
\newpage
\input{chapters/JPM089}
\newpage
\input{chapters/JPM090}
\newpage
\input{chapters/JPM091}
\newpage
\input{chapters/JPM092}
\newpage
\input{chapters/JPM093}
\newpage
\input{chapters/JPM094}
\newpage
\input{chapters/JPM095}
\newpage
\input{chapters/JPM096}
\newpage
\input{chapters/JPM097}
\newpage
\input{chapters/JPM098}
\newpage
\input{chapters/JPM099}
\newpage
\input{chapters/JPM100}
\newpage


\end{document}
