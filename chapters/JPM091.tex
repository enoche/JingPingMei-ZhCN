%# -*- coding:utf-8 -*-
%%%%%%%%%%%%%%%%%%%%%%%%%%%%%%%%%%%%%%%%%%%%%%%%%%%%%%%%%%%%%%%%%%%%%%%%%%%%%%%%%%%%%


\chapter{孟玉楼爱嫁李衙内\KG 李衙内怒打玉簪儿}


诗曰:

\[
簟展湘纹浪欲生,幽怀自感梦难成。
倚床剩觉添风味,开户羞将待月明。
拟倩蜂媒传密意,难将萤火照离情。
遥怜织女佳期近,时看银河几曲横。
\]

话说一日,陈敬济听见薛嫂儿说知孙雪娥之事。这陈敬济乘着这个根由,就如此这般,使薛嫂儿往西门庆家对月娘说。薛嫂只得见月娘,说:“陈姑夫在外声言发话,说不要大姐,要写状子,巡抚、巡按处告示,说老爹在日,收着他父亲寄放的许多金银箱笼细软之物。”这月娘一来因孙雪娥被来旺儿盗财拐去,二者又是来安儿小厮走了,三者家人来兴媳妇惠秀又死了,刚打发出去,家中正七事八事,听见薛嫂儿来说此话,唬的慌了手脚,连忙雇轿子,打发大姐家去。但是大姐床奁箱厨陪嫁之物,交玳安雇人,都抬送到陈敬济家。敬济说:“这是他随身嫁我的床帐妆奁,还有我家寄放的细软金银箱笼,须索还我。”薛嫂道:“你大丈母说来,当初丈人在时,止收下这个床奁嫁妆,并没见你别的箱笼。”敬济又要使女元宵儿。薛嫂儿和玳安儿来对月娘说。月娘不肯把元宵与他,说:“这丫头是李娇儿房中使的,如今留着晚早看哥儿哩。”把中秋儿打发将来,说:“原是买了伏侍大姐的。”这敬济又不要中秋儿,两头来回只教薛嫂儿走。他娘张氏向玳安说:“哥哥,你到家拜上你大娘,你家姐儿们多,也不稀罕这个使女看守哥儿。既是与了大姐房里好一向,你姐夫已是收用过了他,你大娘只顾留怎的?”玳安一面到家,把此话对月娘说了。月娘无言可对,只得把元宵儿打发将来。敬济收下,满心欢喜,说道:“可怎的也打我这条道儿来?”正是:

\[
饶你奸似鬼,吃我洗脚水。
\]

按下一头。单说李知县儿子李衙内,自从清明郊外看见吴月娘、孟玉楼两人一般打扮,生的俱有姿色,知是西门庆妻小。衙内有心,爱孟玉楼生的长挑身材,瓜子面皮,模样儿风流俏丽。原来衙内丧偶,鳏居已久,一向着媒妇各处求亲,都不遂意。及见玉楼,便觉动心,但无门可入,未知嫁与不嫁,从违如何。不期雪娥缘事在官,已知是西门庆家出来的,周旋委曲,在伊父案前,将各犯用刑研审,追出赃物数目,望其来领。月娘害怕,又不使人见官。衙内失望,因此才将赃物入官,雪娥官卖。至是衙内谋之于廊吏何不韦,径使官媒婆陶妈妈来西门庆家访求亲事,许说成此门亲事,免县中打卯,还赏银五两。

这陶妈妈听了,喜欢的疾走如飞,一日到于西门庆门首。来昭正在门首立,只见陶妈妈向前道了万福,说道:“动问管家哥一声,此是西门老爹家?”来昭道:“你是那里来的?老爹已下世了,有甚话说?”陶妈妈道:“累及管家进去禀声,我是本县官媒人,名唤陶妈妈,奉衙内小老爹钧语,分付说咱宅内有位奶奶要嫁人,敬来说亲。”那来昭喝道:“你这婆子,好不近理!我家老爹没了一年有余,止有两位奶奶守寡,并不嫁人。常言疾风暴雨,不入寡妇之门。你这媒婆,有要没紧,走来胡撞甚亲事?还不走快着,惹的后边奶奶知道,一顿好打。”那陶妈妈笑道:“管家哥,常言官差吏差,来人不差。小老爹不使我,我敢来?嫁不嫁,起动进去禀声,我好回话去。”来昭道:“也罢,与人方便,自己方便,你少待片时,等我进去。两位奶奶,一位奶奶有哥儿,一位奶奶无哥儿,不知是那一位奶奶要嫁人?”陶妈妈道:“衙内小老爹说,清明那日郊外曾看见来,是面上有几点白麻子的那位奶奶。”

来昭听了,走到后边,如此这般告诉月娘说:“县中使了个官媒人在外面。”倒把月娘吃了一惊,说:“我家并没半个字儿迸出,外边人怎得晓的?”来昭道:“曾在郊外,清明那日见来,说脸上有几个白麻子儿的。”月娘便道:“莫不孟三姐也‘腊月里罗卜——动人心’?忽剌八要往前进嫁人?正是‘世间海水知深浅,惟有人心难忖量’”。一面走到玉楼房中坐下,便问:“孟三娘,奴有件事儿来问你,外面有个保山媒人,说是县中小衙内,清明那日曾见你一面,说你要往前进。端的有此话么?”看官听说,当时没巧不成话,自古姻缘着线牵。那日郊外,孟玉楼看见衙内生的一表人物,风流博浪,两家年甲多相仿佛,又会走马拈弓弄箭,彼此两情四目都有意,已在不言之表。但未知有妻子无妻子,口中不言,心内暗度:“男子汉已死,奴身边又无所出。虽故大娘有孩儿,到明日长大了,各肉儿各疼。闪的我树倒无阴,竹篮儿打水。”又见月娘自有了孝哥儿,心肠改变,不似往时,“我不如往前进一步,寻上个叶落归根之处,还只顾傻傻的守些甚么?到没的担阁了奴的青春年少。”正在思慕之间,不想月娘进来说此话,正是清明郊外看见的那个人,心中又是欢喜,又是羞愧,口里虽说:“大娘休听人胡说,奴并没此话。”不觉把脸来飞红了,正是:

\[
含羞对众休开口,理鬓无言只揾头。
\]
月娘说:“此是各人心里事,奴也管不的许多。”一面叫来昭:“你请那保山进来。”来昭门首唤陶妈妈,进到后边见月娘,行毕了礼数,坐下。小丫鬟倒茶吃了。月娘便问:“保山来,有甚事?”陶妈妈便道:“小媳妇无事不登三宝殿,奉本县正宅衙内分付,说贵宅上有一位奶奶要嫁人,讲说亲事。”月娘道:“俺家这位娘子嫁人,又没曾传出去,你家衙内怎得知道?”陶妈妈道:“俺家衙内说来,清明那日,在郊外亲见这位娘子,生的长挑身材,瓜子面皮,脸上有稀稀几个白麻子,便是这位奶奶。”月娘听了,不消说就是孟三姐了。于是领陶妈妈到玉楼房中明间内坐下。

等勾多时,玉楼梳洗打扮出来。陶妈妈道了万福,说道:“就是此位奶奶,果然话不虚传,人材出众,盖世无双,堪可与俺衙内老爹做个正头娘子。”玉楼笑道:“妈妈休得乱说。且说你衙内今年多大年纪?原娶过妻小没有?房中有人也无?姓甚名谁?有官身无官身?从实说来,休要捣谎。”陶妈妈道:“天么,天么!小媳妇是本县官媒,不比外边媒人快说谎。我有一句说一句,并无虚假。俺知县老爹年五十多岁,止生了衙内老爹一人,今年属马的,三十一岁,正月二十三日辰时建生。见做国子监上舍,不久就是举人、进士。有满腹文章,弓马熟闲,诸子百家,无不通晓。没有大娘子二年光景,房内止有一个从嫁使女答应,又不出众。要寻个娘子当家,敬来宅上说此亲事。若是咱府上做这门亲事,老爹说来,门面差摇,坟茔地土钱粮,一例尽行蠲免,有人欺负,指名说来,拿到县里,任意拶打。”玉楼道:“你衙内有儿女没有?原籍那里人氏?诚恐一时任满,千山万水带去,奴亲都在此处,莫不也要同他去?”陶妈妈道:“俺衙内身边,儿花女花没有,好不单径。原籍是咱北京真定府枣强县人氏,过了黄河不上六七百里。他家中田连阡陌,骡马成群,人丁无数,走马牌楼,都是抚按明文,圣旨在上,好不赫耀吓人。如今娶娘子到家,做了正房,过后他得了官,娘子便是五花官诰,坐七香车,为命妇夫人,有何不好?”这孟玉楼被陶妈妈一席话,说得千肯万肯,一面唤兰香放桌儿,看茶食点心与保山吃。因说:“保山,你休怪我叮咛盘问。你这媒人们说谎的极多,奴也吃人哄怕了。”陶妈妈道:“好奶奶,只要一个比一个。清自清,浑自浑,好的带累了歹的。小媳妇并不捣谎,只依本分做媒。奶奶若肯了,写个婚帖儿与我,好回小老爹话去。”玉楼取了一条大红段子,使玳安交铺子里傅伙计写了生时八字。吴月娘便说:“你当初原是薛嫂儿说的媒,如今还使小厮叫将薛嫂儿来,两个同拿了贴儿去,说此亲事,才是礼。”不多时,使玳安儿叫了薛嫂儿来,见陶妈妈道了万福。当行见当行,拿着贴儿出离西门庆家门,往县中回衙内话去。一个是这里冰人,一个是那头保山,两张口四十八个牙,这一去管取说得月里嫦娥寻配偶,巫山神女嫁襄王。

陶妈妈在路上问薛嫂儿:“你就是这位娘子的原媒?”薛嫂道:“便是。”陶妈妈问他:“原先嫁这里,根儿是何人家的女儿?嫁这里是女儿,是再婚?”这薛嫂儿便一五一十,把西门庆当初从杨家娶来的话告诉一遍。因见婚贴儿上写“女命三十七岁,十一月二十七日子时生”,说:“只怕衙内嫌年纪大些,怎了?他今才三十一岁,倒大六岁。”薛嫂道:“咱拿了这婚贴儿,交个过路的先生,算看年命妨碍不妨碍。若是不对,咱瞒他几岁儿,也不算说谎。”

二人走来,再不见路过响板的先生,只见路南远远的一个卦肆,青布帐幔,挂着两行大字:“子平推贵贱,铁笔判荣枯;有人来算命,直言不容情。”帐子底下安放一张桌子,里面坐着个能写快算灵先生。这两个媒人向前道了万福,先生便让坐下。薛嫂道:“有个女命累先生算一算。”向袖中拿出三分命金来,说:“不当轻视,先生权且收了,路过不曾多带钱来。”先生道:“请说八字。”陶妈妈递与他婚帖看,上面有八字生日年纪,先生道:“此是合婚。”一百捏指寻纹,把算子摇了一摇,开言说道:“这位女命今年三十七岁了,十一月廿七日子时生。甲子月,辛卯日,庚子时,理取印绶之格。女命逆行,见在丙申运中。丙合辛生,往后大有威权,执掌正堂夫人之命。四柱中虽夫星多,然是财命,益夫发福,受夫宠爱,这两年定见妨克,见过了不曾?”薛嫂道:“已克过两位夫主了。”先生道:“若见过,后来好了。”薛嫂儿道:“他往后有子没有?”先生道:“子早哩。直到四十一岁才有一子送老。一生好造化,富贵荣华无比。”取笔批下命词四句道:

\[
娇姿不失江梅态,三揭红罗两画眉。
会看马首升腾日,脱却寅皮任意移。
\]
薛嫂问道:“先生,如何是‘会看马首升腾日,脱却寅皮任意移’?这两句俺每不懂,起动先生讲说讲说。”先生道:“马首者,这位娘子如今嫁个属马的夫主,才是贵星,享受荣华。寅皮是克过的夫主,是属虎的,虽是宠爱,只是偏房。往后一路功名,直到六十八岁,有一子,寿终,夫妻偕老。”两个媒人说道:“如今嫁的倒果是个属马的,只怕大了好几岁,配不来。求先生改少两岁才好。”先生道:“既要改,就改做丁卯三十四岁罢。”薛嫂道:“三十四岁,与属马的也合的着么?”先生道:“丁火庚金,火逢金炼,定成大器,正合得着。”当下改做三十四岁。

两个拜辞了先生,出离卦肆,径到县中。门子报入,衙内便唤进陶、薛二媒人,旋磕了头。衙内便问:“那个妇人是那里的?”陶妈妈道:“是那边媒人。”因把亲事说成,告诉一遍,说:“娘子人才无比的好,只争年纪大些。小媳妇不敢擅便,随衙内老爹尊意,讨了个婚贴在此。”于是递上去。李衙内看了,上写着“三十四岁,十一月廿七日子时生”,说道:“就大三两岁,也罢。”薛嫂儿插口道:“老爹见的是,自古道,妻大两,黄金长;妻大三,黄金山。这位娘子人材出众,性格温柔,诸子百家,当家理纪,自不必说。”衙内道:“我已见过,不必再相。只择吉日良时,行茶礼过去就是了。”两个媒人禀说:“小媳妇几时来伺候?”衙内道:“事不迟稽迟,你两个明日来讨话,往他家说。”每个赏了一两银子,做脚步钱。两个媒人欢喜出门,不在话下。

这李衙内见亲事已成,喜不自胜,即唤廊吏何不韦来商议,对父亲李知县说了。令阴阳生择定四月初八日行礼,十五日准娶妇人过门。就兑出银子来,委托何不韦、小张闲买办茶红酒礼,不必细说。两个媒人次日讨了日期,往西门庆家回月娘、玉楼话。正是:

\[
姻缘本是前生定,曾向蓝田种玉来。
\]

四月初八日,县中备办十六盘羹果茶饼,一副金丝冠儿,一副金头面,一条玛瑙带,一副丁当七事,金镯银钏之类,两件大红宫锦袍儿,四套妆花衣服,三十两礼钱,其余布绢绵花,共约二十余抬。两个媒人跟随,廊吏何不韦押担,到西门庆家下了茶。

十五日,县中拨了许多快手闲汉来,搬抬孟玉楼床帐嫁妆箱笼。月娘看着,但是他房中之物,尽数都交他带去。原旧西门庆在日,把他一张八步彩漆床陪了大姐,月娘就把潘金莲房中那张螺钿床陪了他。玉楼交兰香跟他过去,留下小鸾与月娘看哥儿。月娘不肯,说:“你房中丫头,我怎好留下你的?左右哥儿有中秋儿、绣春和奶子,也勾了。”玉楼止留下一对银回回壶与哥儿耍子,做一念儿,其余都带过去了。到晚夕,一顶四人大轿,四对红纱灯笼,八个皂隶跟随来娶。玉楼戴着金梁冠儿,插着满头珠翠、胡珠子,身穿大红通袖袍儿,先辞拜西门庆灵位,然后拜月娘。月娘说道:“孟三姐,你好狠也!你去了,撇的奴孤另另独自一个,和谁做伴儿?”两个携手哭了一回。然后家中大小都送出大门。媒人替他带上红罗销金盖袱,抱着金宝瓶,月娘守寡出不的门,请大姨送亲,送到知县衙里来。满街上人看见说:“此是西门大官人第三娘子,嫁了知县相公儿子衙内,今日吉日良时娶过门。”也有说好的,也有说歹的。说好者,当初西门大官人怎的为人做人,今日死了,止是他大娘子守寡正大,有儿子,房中搅不过这许多人来,都交各人前进,甚有张主。有那说歹的,街谈巷议,指戳说道:“西门庆家小老婆,如今也嫁人了。当初这厮在日,专一违天害理,贪财好色,奸骗人家妻女。今日死了,老婆带的东西,嫁人的嫁人,拐带的拐带,养汉的养汉,做贼的做贼,都野鸡毛儿零撏了。常言三十年远报,而今眼下就报了。”旁人纷纷议论不题。

且说孟大姨送亲到县衙内,铺陈床帐停当,留坐酒席来家。李衙内赏薛嫂儿、陶妈妈每人五两银子,一段花红利市,打发出门。至晚,两个成亲,极尽鱼水之欢,于飞之乐。到次日,吴月娘送茶完饭。杨姑娘已死,孟大妗子、二妗子、孟大姨都送茶到县中。衙内这边下回书,请众亲戚女眷做三日,扎彩山,吃筵席。都是三院乐人妓女,动鼓乐扮演戏文。吴月娘那日亦满头珠翠,身穿大红通袖袍儿,百花裙,系蒙金带,坐大轿来衙中,进入后边院落,静俏俏无个人接应。想起当初,有西门庆在日,姊妹们那样闹热,往人家赴席来家,都来相见说话,一条板凳坐不了,如今并无一个儿了。一面扑着西门庆灵床儿,不觉一阵伤心,放声大哭。哭了一回,被丫鬟小玉劝止。正是:

\[
平生心事无人识,只有穿窗皓月知。
\]

这里月娘忧闷不题。却说李衙内和玉楼两个,女貌郎才,如鱼如水,正合着油瓶盖。每日燕尔新婚,在房中厮守,一步不离。端详玉楼容貌,越看越爱。又见带了两个从嫁丫鬟,一个兰香,年十八岁,会弹唱;一个小鸾,年十五岁,俱有颜色。心中欢喜没入脚处。有诗为证:

\[
堪夸女貌与郎才,天合姻缘礼所该。
十二巫山云雨会,两情愿保百年偕。
\]

原来衙内房中,先头娘子丢了一个大丫头,约三十年纪,名唤玉簪儿。专一搽胭抹粉,作怪成精。头上打着盘头揸髻,用手贴苫盖,周围勒销金箍儿,假充作\textuni{4BFC}髻,身上穿一套怪绿乔红的裙袄,脚上穿着双拨船样四个眼的剪绒鞋,约长尺二。在人根前,轻身浪颡,做势拿班。衙内未娶玉楼时,他便逐日顿羹顿饭,殷勤伏侍,不说强说,不笑强笑,何等精神。自从娶过玉楼来,见衙内和他如胶似漆,把他不去揪采,这丫头就使性儿起来。一日,衙内在书房中看书,这玉簪儿在厨下顿了一盏好果仁炮茶,双手用盘儿托来书房里,笑嘻嘻掀开帘儿,送与衙内。不想衙内看了一回书,搭伏定书桌就睡着了。这玉簪儿叫道:“爹,谁似奴疼你,顿了这盏好茶儿与你吃。你家那新娶的娘子,还在被窝里睡得好觉儿,怎不交他那小大姐送盏茶来与你吃?”因见衙内打盹,在眼前只顾叫不应,说道:“老花子,你黑夜做夜作使乏了也怎的?大白日里盹磕睡,起来吃茶!”叫衙内醒了,看见是他,喝道:“怪碜奴才!把茶放下,与我过一边去。”这玉簪儿满脸羞红,使性子把茶丢在桌上,出来说道:“好不识人敬重!奴好意用心,大清早辰送盏茶儿来你吃,倒吆喝我起来。常言:‘丑是家中宝,可喜惹烦恼’。我丑,你当初瞎了眼,谁交你要我来?”被衙内听见,赶上尺力踢了两靴脚。这玉簪儿登时把那付奴脸膀的有房梁高,也不搽脸了,也不顿茶了。赶着玉楼,也不叫娘,只你也我也,无人处,一屁股就在玉楼床上坐下。玉楼亦不去理他。他背地又压伏兰香、小鸾说:“你休赶着我叫姐,只叫姨娘。我与你娘系大小之分。”又说:“你只背地叫罢,休对着你爹叫。你每日跟随我行,用心做活,你若不听我说,老娘拿煤锹子请你。”后来几次见衙内不理他,他就撒懒起来,睡到日头半天还不起来,饭儿也不做,地儿也不扫。玉楼分付兰香、小鸾:“你休靠玉簪儿了,你二人自去厨下做饭,打发你爹吃罢。”这玉簪又气不愤,使性谤气,牵家打伙,在厨房内打小鸾,骂兰香:“贼小奴才,小淫妇儿!碓磨也有个先来后到,先有你娘来,先有我来?都是你娘儿们占了罢,不献这个勤儿也罢了!当原先俺死的那个娘也没曾失口叫我声玉簪儿,你进门几日,就题名道姓叫我。我是你手里使的人也怎的?你未来时,我和俺爹同床共枕,那一日不睡到斋时才起来。和我两个如糖拌蜜,如蜜搅酥油一般打热。房中事,那些儿不打我手里过。自从你来了,把我蜜罐儿也打碎了,把我姻缘也拆散开了,一撵撵到我明间,冷清清支板凳打官铺,再不得尝着俺爹那件东西儿如今甚么滋味了。我这气苦也没处声诉。你当初在西门庆家,也曾做第三个小老婆来,你小名儿叫玉楼,敢说老娘不知道?你来在俺家,你识我见,大家脓着些罢了。会那等乔张致,呼张唤李,谁是你买到的?属你管辖?”不知玉楼在房听见,气的发昏,又不好声言对衙内说。

一日热天,也是合当有事。晚夕衙内分付他厨下热水,拿浴盆来房中,要和玉楼洗澡。玉楼便说:“你交兰香热水罢,休要使他。”衙内不从,说道:“我偏使他,休要惯了这奴才。”玉簪儿见衙内要水,和妇人共浴兰汤,效鱼水之欢,心中正没好气,拿浴盆进房,往地下只一墩,用大锅浇上一锅滚水,只中喃喃呐呐说道:“也没见这娘淫妇,刁钻古怪,禁害老娘!无故也只是个浪精毴,没三日不拿水洗。像我与俺主子睡,成月也不见点水儿,也不见展污了甚么佛眼儿。偏这淫妇会,两番三次刁蹬老娘。”直骂出房门来。玉楼听见,也不言语。衙内听了此言,心中大怒,澡也洗不成,精脊梁趿着鞋,向床头取拐子,就要走出来。妇人拦阻住,说道:“随他骂罢,你好惹气。只怕热身子出去,风试着你,倒值了多的。”衙内那里按纳得住,说道:“你休管。这奴才无礼!”向前一把手采住他头发,拖踏在地下,轮起拐子,雨点打将下来。饶玉楼在旁劝着,也打了二三十下在身。打的这丫头急了,跪在地下告说:“爹,你休打我,我想爹也看不上我在家里了,情愿卖了我罢。”衙内听了,亦发恼怒起来,又狠了几下。玉楼劝道:“他既要出去,你不消打,倒没得气了你。”衙内随令伴当即时叫将陶妈妈来,把玉簪儿领出去,便卖银子来交,不在话下。正是:蚊虫遭扇打,只为嘴伤人。有诗为证:

\[
百禽啼后人皆喜,惟有鸦鸣事若何。
见者多言闻者唾,只为人前口嘴多。
\]
