%# -*- coding:utf-8 -*-
%%%%%%%%%%%%%%%%%%%%%%%%%%%%%%%%%%%%%%%%%%%%%%%%%%%%%%%%%%%%%%%%%%%%%%%%%%%%%%%%%%%%%


\chapter{定挨光王婆受贿\KG 设圈套浪子私挑}


诗曰:

\[
乍对不相识,徐思似有情。杯前交一面,花底恋双睛。
艖俹惊新态,含胡问旧名。影含今夜烛,心意几交横。
\]

话说西门庆央王婆,一心要会那雌儿一面,便道:“干娘,你端的与我说这件事成,我便送十两银子与你。”王婆道:“大官人,你听我说:但凡‘挨光’的两个字最难。怎的是‘挨光’?比如如今俗呼‘偷情’就是了。要五件事俱全,方才行的。第一要潘安的貌;第二要驴大行货;第三要邓通般有钱;第四要青春少小,就要绵里针一般软款忍耐;第五要闲工夫。此五件,唤做‘潘驴邓小闲’。都全了,此事便获得着。”西门庆道:“实不瞒你说,这这五件事我都有。第一件,我的貌虽比不得潘安,也充得过;第二件,我小时在三街两巷游串,也曾养得好大龟;第三,我家里也有几贯钱财,虽不及邓通,也颇得过日子;第四,我最忍耐;他便打我四百顿,休想我回他一拳;第五,我最有闲工夫,不然如何来得恁勤。干娘,你自作成,完备了时,我自重重谢你。”王婆道:“大官人,你说五件事都全,我知道还有一件事打搅,也多是成不得。”西门庆道:“且说,甚么一件事打搅?”王婆道:“大官人休怪老身直言,但凡挨光最难,十分,有使钱到九分九厘,也有难成处。我知你从来悭吝,不肯胡乱便使钱,只这件打搅。”西门庆道:“这个容易,我只听你言语便了。”王婆道:“若大官人肯使钱时,老身有一条妙计,须交大官人和这雌儿会一面。”西门庆道:“端的有甚妙计?”王婆笑道:“今日晚了,且回去,过半年三个月来商量。”西门庆央及道:“干娘,你休撒科!自作成我则个,恩有重报。”王婆笑哈哈道:“大官人却又慌了。老身这条计,虽然入不得武成王庙,端的强似孙武子教女兵,十捉八九着。今日实对你说了罢:这个雌儿来历,虽然微末出身,却倒百伶百俐,会一手好弹唱,针指女工,百家歌曲,双陆象棋,无所不知。小名叫做金莲,娘家姓潘,原是南门外潘裁的女儿,卖在张大户家学弹唱。后因大户年老,打发出来,不要武大一文钱,白白与了他为妻。这雌儿等闲不出来,老身无事常过去与他闲坐。他有事亦来请我理会,他也叫我做干娘。武大这两日出门早。大官人如干此事,便买一匹蓝绸、一匹白绸、一匹白绢,再用十两好绵,都把来与老身。老身却走过去问他借历日,央及他拣个好日期,叫个裁缝来做。他若见我这般说,拣了日期,不肯与我来做时,此事便休了;他若欢天喜地说:‘我替你做。’不要我叫裁缝,这光便有一分了。我便请得他来做,就替我缝,这光便二分了。他若来做时,午间我却安排些酒食点心请他吃。他若说不便当,定要将去家中做,此事便休了;他不言语吃了时,这光便有三分了。这一日你也莫来,直至第三日,晌午前后,你整整齐齐打扮了来,以咳嗽为号,你在门前叫道:‘怎的连日不见王干娘?我买盏茶吃。’我便出来请你入房里坐吃茶。他若见你便起身来,走了归去,难道我扯住他不成?此事便休了。他若见你入来,不动身时,这光便有四分了。坐下时,我便对雌儿说道:‘这个便是与我衣服施主的官人,亏杀他。’我便夸大官人许多好处,你便卖弄他针指。若是他不来兜揽答应时,此事便休了;他若口中答应与你说话时,这光便有五分了。我便道:‘却难为这位娘子与我作成出手做,亏杀你两施主,一个出钱,一个出力。不是老身路歧相央,难得这位娘子在这里,官人做个主人替娘子浇浇手。’你便取银子出来,央我买。若是他便走时,难道我扯住他?此事便休了。他若是不动身时,事务易成,这光便有六分了。我却拿银子,临出门时对他说:‘有劳娘子相待官人坐一坐。’他若起身走了家去,我终不成阻挡他?此事便休了。若是他不起身,又好了,这光便有七分了。待我买得东西提在桌子上,便说:‘娘子且收拾过生活去,且吃一杯儿酒,难得这官人坏钱。’他不肯和你同桌吃,去了,此事便休了。若是他不起身,此事又好了,这光便有八分了。待他吃得酒浓时,正说得入港,我便推道没了酒,再交你买,你便拿银子,又央我买酒去并果子来配酒。我把门拽上,关你两个在屋里。他若焦燥跑了归去时,此事便休了;他若由我拽上门,不焦躁时,这光便有九分,只欠一分了。只是这一分倒难。大官人你在房里,便着几句甜话儿说入去,却不可燥暴,便去动手动脚打搅了事,那时我不管你。你先把袖子向桌子上拂落一双箸下去,只推拾箸,将手去他脚上捏一捏。他若闹吵起来,我自来搭救。此事便休了,再也难成。若是他不做声时,此事十分光了。这十分光做完备,你怎的谢我?”西门庆听了大喜道:“虽然上不得凌烟阁,干娘你这条计,端的绝品好妙计!”王婆道:却不要忘了许我那十两银子。”西门庆道:“便得一片橘皮吃,切莫忘了洞庭湖。这条计,干娘几时可行?”婆道:“只今晚来有回报。我如今趁武大未归,过去问他借历日,细细说与他。你快使人送将绸绢绵子来,休要迟了。”西门庆道:“干娘,这是我的事,如何敢失信。”于是作别了王婆,离了茶肆,就去街上买了绸绢三匹并十两清水好绵。家里叫了玳安儿用毡包包了,一直送入王婆家来。王婆欢喜收下,打发小厮回去。正是:

\[
巫山云雨几时就,莫负襄王筑楚台。
\]

当下王婆收了绸绢绵子,开了后门,走过武大家来。那妇人接着,走去楼上坐的。王婆道:“娘子怎的这两日不过贫家吃茶?”那妇人道:“便是我这几日身子不快,懒走动的。”王婆道:“娘子家里有历日,借与老身看一看,要个裁衣的日子。”妇人道:“干娘裁甚衣服?”王婆道:“便是因老身十病九痛,怕一时有些山高水低,我儿子又不在家。”妇人道:“大哥怎的一向不见?”王婆道:“那厮跟了个客人在外边,不见个音信回来,老身日逐耽心不下。”妇人道:“大哥今年多少年纪?”王婆道:“那厮十七岁了。”妇人道:“怎的不与他寻个亲事,与干娘也替得手?”王婆道:“因是这等说,家中没人。待老身东楞西补的来,早晚要替他寻下个儿。等那厮来,却再理会。见如今老身白日黑夜只发喘咳嗽,身子打碎般,睡不倒的,只害疼,一时先要预备下送终衣服。难得一个财主官人,常在贫家吃茶,但凡他宅里看病,买使女,说亲,见老身这般本分,大小事儿无不管顾老身。又布施了老身一套送终衣料,绸绢表里俱全,又有若干好绵,放在家里一年有余,不能勾做得。今年觉得好生不济,不想又撞着闰月,趁着两日倒闲,要做又被那裁缝勒掯,只推生活忙,不肯来做。老身说不得这苦也!”那妇人听了笑道:“只怕奴家做得不中意。若是不嫌时,奴这几日倒闲,出手与干娘做如何?”那婆子听了,堆下笑来说道:“若得娘子贵手做时,老身便死也得好处去。久闻娘子好针指,只是不敢来相央。”那妇人道:“这个何妨!既是许了干娘,务要与干娘做了,将历日去交人拣了黄道好日,奴便动手。”王婆道:“娘子休推老身不知,你诗词百家曲儿内字样,你不知识了多少,如何交人看历日?”妇人微笑道:“奴家自幼失学。”婆子道:“好说,好说。”便取历日递与妇人。妇人接在手内,看了一回,道:“明日是破日,后日也不好,直到外后日方是裁衣日期。”王婆一把手取过历头来挂在墙上,便道:“若得娘子肯与老身做时,就是一点福星。何用选日!老身也曾央人看来,说明日是个破日,老身只道裁衣日不用破日,我不忌他。”那妇人道:“归寿衣服,正用破日便好。”王婆道:“既是娘子肯作成,老身胆大,只是明日起动娘子,到寒家则个。”妇人道:“何不将过来做?”王婆道:“便是老身也要看娘子做生活,又怕门首没人。”妇人道:“既是这等说,奴明日饭后过来。”那婆子千恩万谢下楼去了,当晚回覆了西门庆话,约定后日准来。当夜无话。

次日清晨,王婆收拾房内干净,预备下针线,安排了茶水,在家等候。且说武大吃了早饭,挑着担儿自出去了。那妇人把帘儿挂了,分付迎儿看家,从后门走过王婆家来。那婆子欢喜无限,接入房里坐下,便浓浓点一盏胡桃松子泡茶与妇人吃了。抹得桌子干净,便取出那绸绢三匹来。妇人量了长短,裁得完备,缝将起来。婆子看了,口里不住喝采道:“好手段,老身也活了六七十岁,眼里真个不曾见这般好针指!”那妇人缝到日中,王婆安排些酒食请他,又下了一箸面与那妇人吃。再缝一歇,将次晚来,便收拾了生活,自归家去。恰好武大挑担儿进门,妇人拽门下了帘子。武大入屋里,看见老婆面色微红,问道:“你那里来?”妇人应道:“便是间壁干娘央我做送终衣服,日中安排些酒食点心请我吃。”武大道:“你也不要吃他的才是,我们也有央及他处。他便央你做得衣裳,你便自归来吃些点心,不值得甚么,便搅挠他。你明日再去做时,带些钱在身边,也买些酒食与他回礼。常言道:远亲不如近邻,休要失了人情。他若不肯交你还礼时,你便拿了生活来家,做还与他便了。”正是:

\[
阿母牢笼设计深,大郎愚卤不知音。
带钱买酒酬奸诈,却把婆娘自送人。
\]
妇人听了武大言语,当晚无话。

次日饭后,武大挑担儿出去了,王婆便踅过来相请。妇人去到他家屋里,取出生活来,一面缝来。王婆忙点茶来与他吃了茶。看看缝到日中,那妇人向袖中取出三百文钱来,向王婆说道:“干娘,奴和你买盏酒吃。”王婆道:“啊呀,那里有这个道理。老身央及娘子在这里做生活,如何交娘子倒出钱,婆子的酒食,不到吃伤了哩!”那妇人道:“却是拙夫分付奴来,若是干娘见外时,只是将了家去,做还干娘便了。”那婆子听了道:“大郎直恁地晓事!既然娘子这般说时,老身且收下。”这婆子生怕打搅了事,自又添钱去买好酒好食来,殷勤相待。看官听说:但凡世上妇人,由你十分精细,被小意儿纵十个九个着了道儿。这婆子安排了酒食点心,和那妇人吃了。再缝了一歇,看看晚来,千恩万谢归去了。

话休絮烦。第三日早饭后,王婆只张武大出去了,便走过后後门首叫道:“娘子,老身大胆。”那妇人从楼上应道:“奴却待来也。”两个厮见了,来到王婆房里坐下,取过生活来缝。那婆子点茶来吃,自不必说。妇人看看缝到晌午前后。却说西门庆巴不到此日,打选衣帽齐齐整整,身边带着三五两银子,手里拿着洒金川扇儿,摇摇摆摆迳往紫石街来。到王婆门首,便咳嗽道:“王干娘,连日如何不见?”那婆子瞧科,便应道:“兀的谁叫老娘?”西门庆道:“是我。”那婆子赶出来看了,笑道:“我只道是谁,原来是大官人!你来得正好,且请入屋里去看一看。”把西门庆袖子只一拖,拖进房里来,对那妇人道:“这个便是与老身衣料施主官人。”西门庆睁眼看着那妇人:云鬟叠翠,粉面生春,上穿白布衫儿,桃红裙子,蓝比甲,正在房里做衣服。见西门庆过来,便把头低了。这西门庆连忙向前屈身唱喏。那妇人随即放下生活,还了万福。王婆便道:“难得官人与老身段匹绸绢,放在家一年有余,不曾得做,亏杀邻家这位娘子出手与老身做成全了。真个是布机也似好针线,缝的又好又密,真个难得!大官人,你过来且看一看。”西门庆拿起衣服来看了,一面喝采,口里道:“这位娘子,传得这等好针指,神仙一般的手段!”那妇人低头笑道:“官人休笑话。”西门庆故问王婆道:“干娘,不敢动问,这位娘子是谁家宅上的娘子?”王婆道:“你猜。”西门庆道:“小人如何猜得着。”王婆哈哈笑道:“大官人你请坐,我对你说了罢。”那西门庆与妇人对面坐下。那婆子道:“好交大官人得知罢,你那日屋檐下走,打得正好。”西门庆道:“就是那日在门首叉竿打了我的?倒不知是谁家宅上娘子?”妇人分外把头低了一低,笑道:“那日奴误冲撞,官人休怪!”西门庆连忙应道:“小人不敢。”王婆道:“就是这位,却是间壁武大娘子。”西门庆道:“原来如此,小人失瞻了。”王婆因望妇人说道:“娘子你认得这位官人么?”妇人道:“不识得。”婆子道:“这位官人,便是本县里一个财主,知县相公也和他来往,叫做西门大官人。家有万万贯钱财,在县门前开生药铺。家中钱过北斗,米烂成仓,黄的是金,白的是银,圆的是珠,放光的是宝,也有犀牛头上角,大象口中牙。他家大娘子,也是我说的媒,是吴千户家小姐,生得面伶百俐。”因问:“大官人,怎的不过贫家吃茶?”西门庆道:“便是家中连日小女有人家定了,不得闲来。”婆子道:“大姐有谁家定了?怎的不请老身去说媒?”西门庆道:“被东京八十万禁军杨提督亲家陈宅定了。他儿子陈敬济才十七岁,还上学堂。不是也请干娘说媒,他那边有了个文嫂儿来讨帖儿,俺这里又使常在家中走的卖翠花的薛嫂儿,同做保山,说此亲事。干娘若肯去,到明日下小茶,我使人来请你。”婆子哈哈笑道:“老身哄大官人耍子。俺这媒人们都是狗娘养下来的,他们说亲时又没我,做成的熟饭儿怎肯搭上老身一分?常言道:当行压当行。到明日娶过了门时,老身胡乱三朝五日,拿上些人情去走走,讨得一张半张桌面,到是正经。怎的好和人斗气!”两个一递一句说了一回。婆子只顾夸奖西门庆,口里假嘈,那妇人便低了头缝针线。

\[
水性从来是女流,背夫常与外人偷。
金莲心爱西门庆,淫荡春心不自由。
\]

西门庆见金莲有几分情意欢喜,恨不得就要成双。王婆便去点两盏茶来,递一盏西门庆,一盏与妇人,说道:“娘子相待官人吃些茶。”旋又看着西门庆,把手在脸上摸一摸,西门庆已知有五分光了。自古“风流茶说合,酒是色媒人”。王婆便道:“大官人不来,老身也不敢去宅上相请。一者缘法撞遇,二者来得正好。常言道:一客不烦二主。大官人便是出钱的,这位娘子便是出力的,亏杀你这两位施主。不是老身路歧相烦,难得这位娘子在这里,官人好与老身做个主人,拿出些银子买些酒食来,与娘子浇浇手,如何?”西门庆道:“小人也见不到这里,有银子在此。”便向茄袋里取出来,约有一两一块,递与王婆,交备办酒食。那妇人便道“不消生受。”口里说着恰不动身。王婆接了银子,临出门便道:“有劳娘子相陪大官人坐一坐,我去就来。”那妇人道:“干娘免了罢。”却亦不动身。王婆便出门去了,丢下西门庆和那妇人在屋里。

这西门庆一双眼不转睛,只看着那妇人。那婆娘也把眼来偷睃西门庆,又低着头做生活。不多时,王婆买了见成肥鹅烧鸭、熟肉鲜鲊、细巧果子,归来尽把盘碟盛了,摆在房里桌子上。看那妇人道:“娘子且收拾过生活,吃一杯儿酒。”那妇人道:“你自陪大官人吃,奴却不当。”那婆子道:“正是专与娘子浇手,如何却说这话!”一面将盘馔却摆在面前,三人坐下,把酒来斟。西门庆拿起酒盏来道:“干娘相待娘子满饮几杯。”妇人谢道:“奴家量浅,吃不得。”王婆道:“老身得知娘子洪饮,且请开怀吃两盏儿。”那妇人一面接酒在手,向二人各道了万福。西门庆拿起箸来说道:“干娘替我劝娘子些菜儿。”那婆子拣好的递将过来与妇人吃。一连斟了三巡酒,那婆子便去烫酒来。西门庆道:“小人不敢动问,娘子青春多少?”妇人低头应道:“二十五岁。”西门庆道:“娘子到与家下贱内同庚,也是庚辰属龙的。他是八月十五日子时。”妇人又回应道:“将天比地,折杀奴家。”王婆便插口道:“好个精细的娘子,百伶百俐,又不枉做得一手好针线。诸子百家,双陆象棋,折牌道字,皆通。一笔好写。”西门庆道:“却是那里去讨。”王婆道:“不是老身说是非,大官人宅上有许多,那里讨得一个似娘子的!”西门庆道:“便是这等,一言难尽。只是小人命薄,不曾招得一个好的在家里。”王婆道:“大官人先头娘子须也好。”西门庆道:“休说!我先妻若在时,却不恁的家无主,屋到竖。如今身边枉自有三五七口人吃饭,都不管事。”婆子嘈道:“连我也忘了,没有大娘子得几年了?”西门庆道:“说不得,小人先妻陈氏,虽是微末出身,却倒百伶百俐,是件都替的我。如今不幸他没了,已过三年来。今继娶这个贱累,又常有疾病,不管事,家里的勾当都七颠八倒。为何小人只是走了出来?在家里时,便要呕气。”婆子道:“大官人,休怪我直言,你先头娘子并如今娘子,也没这大娘子这手针线,这一表人物。”西门庆道:“便是房下们也没这大娘子一般儿风流。”那婆子笑道:“官人,你养的外宅东街上住的,如何不请老身去吃茶?”西门庆道:“便是唱慢曲儿的张惜春。我见他是路歧人,不喜欢。”婆子又道:“官人你和勾栏中李娇儿却长久。”西门庆道:“这个人见今已娶在家里。若得他会当家时,自册正了他。”王婆道:“与卓二姐却相交得好?”西门庆道:“卓丢儿别要说起,我也娶在家做了第三房。近来得了个细疾,却又没了。”婆子道:“耶乐,耶乐!若有似大娘子这般中官人意的,来宅上说,不妨事么?”西门庆道:“我的爹娘俱已没了,我自主张,谁敢说个不字?”王婆道:“我自说耍,急切便那里有这般中官人意的!”西门庆道:“做甚么便没?只恨我夫妻缘分上薄,自不撞着哩。”西门庆和婆子一递一句说了一回。王婆道:“正好吃酒,却又没了。官人休怪老身差拨,买一瓶儿酒来吃如何?”西门庆便向茄袋内,还有三四两散银子,都与王婆,说道:“干娘,你拿了去,要吃时只顾取来,多的干娘便就收了。”那婆子谢了起身。睃那粉头时,三钟酒下肚,哄动春心,又自两个言来语去,都有意了,只低了头不起身。正是:

\[
眼意眉情卒未休,姻缘相凑遇风流。
王婆贪贿无他技,一味花言巧舌头。
\]
