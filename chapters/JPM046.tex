%# -*- coding:utf-8 -*-
%%%%%%%%%%%%%%%%%%%%%%%%%%%%%%%%%%%%%%%%%%%%%%%%%%%%%%%%%%%%%%%%%%%%%%%%%%%%%%%%%%%%%


\chapter{元夜游行遇雪雨\KG 妻妾戏笑卜龟儿}


词曰:

\[
小市东门欲雪天,众中依约见神仙。蕊黄香细贴金蝉。饮散黄昏人草草,醉容无语立门前。马嘶尘哄一街烟。
\]

话说西门庆那日,打发吴月娘众人往吴大妗子家吃酒去了。李智、黄四约坐到黄昏时分,就告辞起身。伯爵赶送出去,如此这般告诉:“我已替二公说了,准在明日还找五百两银子。”那李智、黄四向伯爵打了恭又打恭,去了。伯爵复到厢房中,和谢希大陪西门庆饮酒,只见李铭掀帘子进来。伯爵看见,便道:“李日新来了。”李铭扒在地下磕头。西门庆问道:“吴惠怎的不来?”李铭道:“吴惠今日东平府官身也没去,在家里害眼。小的叫了王柱来了。”便叫王柱:“进来,与爹磕头。”那王柱掀帘进入房里,朝上磕了头,与李铭站立在旁。伯爵道:“你家桂姐刚才家去了,你不知道?”李铭道:“小的官身到家,洗了洗脸就来了,并不知道。”伯爵向西门庆说:“他两个怕不的还没吃饭哩,哥吩咐拿饭与他两个吃。”书童在旁说:“二爹,叫他等一等,亦发和吹打的一答里吃罢,敢也拿饭去了。”伯爵令书童取过一个托盘来,桌上掉了两碟下饭,一盘烧羊肉,递与李铭:“等拿了饭来,你每拿两碗在这明间吃罢。”说书童儿:“我那傻孩子,常言道:方以类聚,物以群分。你不知,他这行人故虽是当院出身,小优儿比乐工不同,一概看待也罢了,显的说你我不帮衬了。”被西门庆向伯爵头上打了一下,笑骂道:“怪不的你这狗才,行计中人只护行计中人,又知这当差的甘苦。”伯爵道:“傻孩儿,你知道甚么!你空做子弟一场,连‘惜玉怜香’四个字你还不晓的。粉头、小优儿如同鲜花一般,你惜怜他,越发有精神。你但折剉他,敢就《八声甘州》恹恹瘦损,难以存活。”西门庆笑道:“还是我的儿晓的道理。”

那李铭、王柱须臾吃了饭,应伯爵叫过来吩咐:“你两个会唱‘雪月风花共裁剪’不会?”李铭道:“此是黄钟,小的每记的。”于是,王柱弹琵琶,李铭\textShouLuan 筝,顿开喉音唱了一套。唱完了,看看晚来,正是:

\[
金乌渐渐落西山,玉兔看看上画阑;
佳人款款来传报,月透纱窗衾枕寒。
\]

西门庆命收了家火,使人请傅伙计、韩道国、云主管、贲四、陈敬济,大门首用一架围屏安放两张桌席,悬挂两盏羊角灯,摆设酒筵,堆集许多春檠果盒,各样肴馔。西门庆与伯爵、希大都一带上面坐了,伙计、主管两旁打横。大门首两边,一边十二盏金莲灯。还有一座小烟火,西门庆吩咐等堂客来家时放。先是六个乐工,抬铜锣铜鼓在大门首吹打。吹打了一回,又请吹细乐上来。李铭、王柱两个小优儿筝、琵琶上来,弹唱灯词。那街上来往围看的人,莫敢仰视。西门庆带忠靖冠,丝绒鹤氅,白绫袄子。玳安与平安两个,一递一桶放花儿。两名排军执揽杆拦挡闲人,不许向前拥挤。不一时,碧天云静,一轮皓月东升之时,街上游人十分热闹,但见:

\[
户户鸣锣击鼓,家家品竹弹丝。游人队队踏歌声,士女翩翩垂舞调。鳌山结彩,巍峨百尺矗晴云;凤禁褥香,缥缈千层笼绮队。闲庭内外,溶溶宝月光辉;画阁高低,灿灿花灯照耀。三市六街人闹热,凤城佳节赏元宵。
\]

且说春梅、迎春、玉箫、兰香、小玉众人,见月娘不在,听见大门首吹打铜鼓弹唱,又放烟火,都打扮着走来,在围屏后扒着望外瞧。书童儿和画童儿两个,在围屏后火盆上筛酒。原来玉箫和书童旧有私情,两个常时戏狎。两个因按在一处夺瓜子儿嗑,不防火盆上坐着一锡瓶酒,推倒了,那火烘烘望上腾起来,漰了一地灰起去。那王箫还只顾嘻笑,被西门庆听见,使下玳安儿来问:“是谁笑?怎的这等灰起?”那日春梅穿着新白绫袄子,大红遍地金比甲,正坐在一张椅儿上,看见他两个推倒了酒,就扬声骂玉箫道:“好个怪浪的淫妇!见了汉子,就邪的不知怎么样儿的了,只当两个把酒推倒了才罢了。都还嘻嘻哈哈,不知笑的是甚么!把火也漰死了,平白落人恁一头灰。”玉箫见他骂起来,唬的不敢言语,往后走了。慌的书童儿走上去,回说:“小的火盆上筛酒来,扒倒了锡瓶里酒了。”西门庆听了,便不问其长短,就罢了。

先是那日,贲四娘子打听月娘不在,平昔知道春梅、玉箫、迎春、兰香四个是西门庆贴身答应得宠的姐儿,大节下安排了许多菜蔬果品,使了他女孩儿长儿来,要请他四个去他家里坐坐。众人领了来见李娇儿。李娇儿说:“我灯草拐杖——做不得主。你还请问你爹去。”问雪娥,雪娥亦发不敢承揽。看看挨到掌灯以后,贲四娘子又使了长儿来邀四人。兰香推玉箫,玉箫推迎春,迎春推春梅,要会齐了转央李娇儿和西门庆说,放他去。那春梅坐着,纹丝儿也不动,反骂玉箫等:“都是那没见食面的行货子,从没见酒席,也闻些气儿来!我就去不成,也不到央及他家去。一个个鬼撺攥的也似,不知忙些甚么,教我半个眼儿看的上!”那迎春、玉箫、兰香都穿上衣裳,打扮的齐齐整整出来,又不敢去,这春梅又只顾坐着不动身。书童见贲四嫂又使了长儿来邀,说道:“我拚着爹骂两句也罢,等我上去替姐每禀禀去。”一直走到西门庆身边,附耳说道:“贲四嫂家大节间要请姐每坐坐,姐教我来禀问爹,去不去?”西门庆听了,吩咐:“教你姐每收拾去,早些来,家里没人。”这书童连忙走下来,说道:“还亏我到上头,一言就准了。教你姐快收拾去,早些来。”那春梅才慢慢往房里匀施脂粉去了。

不一时,四个都一答儿里出门。书童扯围屏掩过半边来,遮着过去。到了贲四家,贲四娘子见了,如同天上落下来的一般,迎接进屋里。顶槅上点着绣球纱灯,一张桌儿上整齐肴菜。赶着春梅叫大姑,迎春叫二姑,玉箫是三姑,兰香是四姑,都见过礼。又请过韩回子娘子来相陪。春梅、迎春上坐,玉箫、兰香对席,贲四嫂与韩回子娘子打横,长儿往来烫酒拿菜。按下这里不题。

西门庆因叫过乐工来吩咐:“你每吹一套‘东风料悄’《好事近》与我听。”正值后边拿上玫瑰元宵来,众人拿起来同吃,端的香甜美味,入口而化,甚应佳节。李铭、王柱席前拿乐器,接着弹唱此词,端的声韵悠扬,疾徐合节。这里弹唱饮酒不题。

且说玳安与陈敬济袖着许多花炮,又叫两个排军拿着两个灯笼,竟往吴大妗于家来接月娘。众人正在明间饮酒,见了陈敬济来:“教二舅和姐夫房里坐,你大舅今日不在家,卫里看着造册哩。”一面放桌儿,拿春盛点心酒菜上来,陪敬济。玳安走到上边,对月娘说:“爹使小的来接娘每来了,请娘早些家去,恐晚夕人乱,和姐夫一答儿来了。”月娘因头里恼他,就一声儿没言语答他。吴大妗子便叫来定儿:“拿些儿甚么与玳安儿吃。”来定儿道:“酒肉汤饭,都前头摆下了。”吴月娘道:“忙怎的?那里才来乍到就与他吃!教他前边站着,我每就起身。”吴大妗子道:“三姑娘慌怎的?上门儿怪人家?大节下,姊妹间,众位开怀大坐坐儿。左右家里有他二娘和他姐在家里,怕怎的?老早就要家去!是别人家又是一说。”因叫郁大姐:“你唱个好曲儿,伏侍他众位娘。”孟玉楼道:“他六娘好不恼他哩,说你不与他做生日。”郁大姐连忙下席来,与李瓶儿磕了四个头,说道:“自从与五娘做了生日,家去就不好起来。昨日妗奶奶这里接我,教我才收拾\textuni{499B}\textuni{49B7}了来。若好时,怎的不与你老人家磕头?”金莲道:“郁大姐,你六娘不自在哩,你唱个好的与他听,他就不恼你了。”那李瓶儿在旁只是笑,不做声。郁大姐道:“不打紧,拿琵琶过来,等我唱。”大妗子叫吴舜臣媳妇郑三姐:“你把你二位姑娘和众位娘的酒儿斟上。这一日还没上过钟酒儿。”那郁大姐接琵琶在手,用心用意唱了一个《一江风》。

正唱着,月娘便道:“怎的这一回子恁凉凄凄的起来?”来安儿在旁说道:“外边天寒下雪哩。”孟玉楼道:“姐姐,你身上穿的不单薄?我倒带了个绵披袄子来了。咱这一回,夜深不冷么?”月娘道:“既是下雪,叫个小厮家里取皮袄来咱每穿。”那来安连忙走下来,对玳安说:“娘吩咐,叫人家去取娘们皮袄哩。”那玳安便叫琴童儿:“你取去罢,等我在这里伺候。”那琴童也不问,一直家去了。少顷,月娘想起金莲没皮袄,因问来安儿:“谁取皮袄去了?”来安道:“琴童取去了。”月娘道:“也不问我,就去了。”玉楼道:“刚才短了一句话,不该教他拿俺每的,他五娘没皮袄,只取姐姐的来罢。”月娘道:“怎的没有?还有当的人家一件皮袄,取来与六姐穿就是了。”因问:“玳安那奴才怎的不去,却使这奴才去了?你叫他来!”一面把玳安叫到跟前,吃月娘尽力骂了几句道:“好奴才!使你怎的不动?又坐坛遣将儿,使了那个奴才去了。也不问我声儿,三不知就去了。怪不的你做大官儿,恐怕打动你展翅儿,就只遣他去!”玳安道:“娘错怪了小的。头里娘吩咐若是叫小的去,小的敢不去?来安下来,只说叫一个家里去。”月娘道:“那来安小奴才敢吩咐你?俺每恁大老婆,还不敢使你哩!如今惯的你这奴才们有些摺儿也怎的?一来主子烟薰的佛像——挂在墙上,有恁施主,有恁和尚。你说你恁行动两头戳舌,献勤出尖儿,外合里应,好懒食馋,背地瞒官作弊,干的那茧儿我不知道哩!头里你家主子没使你送李桂儿家去,你怎的送他?人拿着毡包,你还匹手夺过去了。留丫头不留丫头不在你,使你进来说,你怎的不进来?你便送他,图嘴吃去了,却使别人进来。须知我若骂只骂那个人了。你还说你不久惯牢成!”玳安道:“这个也没人,就是画童儿过的舌。爹见他抱着毡包,教我:‘你送送你桂姨去罢’,使了他进来的。娘说留丫头不留丫头不在于小的,小的管他怎的!”月娘大怒,骂道:“贼奴才,还要说嘴哩!我可不这里闲着和你犯牙儿哩。你这奴才,脱脖倒坳过颺了。我使着不动,耍嘴儿,我就不信到明日不对他说,把这欺心奴才打与你个烂羊头也不算。”吴大妗子道:“玳安儿,还不快替你娘每取皮袄去。”又道:“姐姐,你吩咐他拿那里皮袄与他五娘穿?”潘金莲接过来说道:“姐姐,不要取去,我不穿皮袄,教他家里捎了我的披袄子来罢。人家当的,好也歹也,黄狗皮也似的,穿在身上,教人笑话,也不长久,后还赎的去了。”月娘道:“这皮袄倒不是当的,是李智少十六两银子准折的。当的王招宣府里那件皮袄,与李娇儿穿了。”因吩咐玳安:“皮袄在大橱里,叫玉箫寻与你,就把大姐的皮袄也带了来。”

玳安把嘴谷都,走出来,陈敬济问道:“你到那去?”玳安道:“精是攮气的营生,一遍生活两遍做,这咱晚又往家里跑一遭。”迳走到家。西门庆还在大门首吃酒,傅伙计、云主管都去了,还有应伯爵、谢希大、韩道国、贲四众人吃酒未去,便问玳安:“你娘们来了?”玳安道:“没来,使小的取皮袄来了。”说毕,便往后走。先是琴童到家,上房里寻玉箫要皮袄。小玉坐在炕上正没好气,说道:“四个淫妇今日都在贲四老婆家吃酒哩。我不知道皮袄放在那里,往他家问他要去。”这琴童一直走到贲四家,且不叫,在窗外悄悄觑听。只见贲四嫂说道:“大姑和三姑,怎的这半日酒也不上,菜儿也不拣一箸儿?嫌俺小家儿人家,整治的不好吃也怎的?”春梅道:“四嫂,俺每酒够了。”贲四嫂道:“耶嚛!没的说。怎的这等上门儿怪人家!”又叫韩回子老婆:“你是我的切邻,就如副东一样,三姑、四姑跟前酒,你也替我劝劝儿,怎的单板着,象客一般?”又叫长姐:“筛酒来,斟与三姑吃,你四姑钟儿浅斟些儿罢。”兰香道:“我自来吃不的。”贲四嫂道:“你姐儿们今日受饿,没甚么可口的菜儿管待,休要笑话。今日要叫了先生来,唱与姑娘们下酒,又恐怕爹那里听着。浅房浅屋,说不的俺小家儿人家的苦。”说着,琴童儿敲了敲门,众人都不言语了。长儿问:“是谁?”琴童道:“是我,寻姐说话。”一面开了门,那琴童入来。玉箫便问:“娘来了?”那琴童看着待笑,半日不言语。玉箫道:“怪雌牙的,谁与你雌牙?问着不言语。”琴童道:“娘每还在妗子家吃酒哩,见天阴下雪,使我来家取皮袄来,都教包了去哩。”玉箫道:“皮袄在描金箱子里不是,叫小玉拿与你。”琴童道:“小玉说教我来问你要。”玉箫道:“你信那小淫妇儿,他不知道怎的!”春梅道:“你每有皮袄的,都打发与他。俺娘没皮袄,只我不动身。”兰香对琴童:“你三娘皮袄,问小鸾要。”迎春便向腰里拿钥匙与琴童儿:“教绣春开里间门拿与你。”

琴童儿走到后边,上房小玉和玉楼房中小鸾,都包了皮袄交与他。正拿着往外走,遇见玳安,问道:“你来家做甚么?”玳安道:“你还说哩!为你来了,平白教大娘骂了我一顿好的。又使我来取五娘的皮袄来。”琴童道:“我如今取六娘的皮袄去也。”玳安道:“你取了,还在这里等着我,一答儿里去。你先去了不打紧,又惹的大娘骂我。”说毕,玳安来到上房。小玉正在炕上笼着炉台烤火,口中嗑瓜子儿,见了玳安,问道:“你也来了?”玳安道:“你又说哩,受了一肚子气在这里。娘说我遣将儿。因为五娘没皮袄,又教我来,说大橱里有李三准折的一领皮袄,教拿去哩。”小玉道:“玉箫拿了里间门上钥匙,都在贲四家吃酒哩,教他来拿。”玳安道:“琴童往六娘房里去取皮袄,便来也,教他叫去,我且歇歇腿儿,烤烤火儿着。”那小玉便让炕头儿与他,并肩相挨着向火。小玉道:“壶里有酒,筛盏子你吃?”玳安道:“可知好哩,看你下顾。”小玉下来,把壶坐在火上,抽开抽屉,拿了一碟子腊鹅肉,筛酒与他。无人处两个就搂着咂舌亲嘴。

正吃着酒,只见琴童儿进来。玳安让他吃了一盏子,便使他:“叫玉箫姐来,拿皮袄与五娘穿。”那琴童抱毡包放下,走到贲四家叫玉箫。玉箫骂道:“贼囚根子,又来做甚么?”又不来。递与钥匙,教小玉开门。那小玉开了里间房门,取了一把钥匙,通了半日,白通不开。琴童儿又往贲四家问去。那玉箫道:“不是那个钥匙。娘橱里钥匙在床褥子座下哩。”小玉又骂道:“那淫妇丁子钉在人家不来,两头来回,只教使我。”及开了,橱里又没皮袄。琴童儿来回走的抱怨道:“就死也死三日三夜,又撞着恁瘟死鬼小奶奶儿们,把人魂也走出了。”向玳安道:“你说此回去,又惹的娘骂。不说屋里,只怪俺们。”走去又对玉箫说:“里间娘橱里寻,没有皮袄。”玉箫想了想,笑道:“我也忘记,在外间大橱里。”到后边,又被小玉骂道:“淫妇吃那野汉子捣昏了,皮袄在这里,却到处寻。”一面取出来,将皮袄包了,连大姐皮袄都交付与玳安、琴童。

两个拿到吴大妗子家,月娘又骂道:“贼奴才,你说同了都不来罢了。”那玳安不敢言语,琴童道:“娘的皮袄都有了,等着姐又寻这件青镶皮袄。”于是打开取出来。吴大妗子灯下观看,说道:“好一件皮袄。五娘,你怎的说他不好,说是黄狗皮。那里有恁黄狗皮,与我一件穿也罢了。”月娘道:“新新的皮袄儿,只是面前歇胸旧了些儿。到明日,从新换两个遍地金歇胸,就好了。孟玉楼拿过来,与金莲戏道:“我儿,你过来,你穿上这黄狗皮,娘与你试试看好不好。”金莲道:“有本事到明日问汉子要一件穿,也不枉的。平白拾人家旧皮袄披在身上做甚么!”玉楼戏道:“好个不认业的,人家有这一件皮袄,穿在身上念佛。”于是替他穿上。见宽宽大大,金莲才不言语。

当下月娘与玉楼、瓶儿俱是貂鼠皮袄,都穿在身上,拜辞吴大妗子、二妗子起身。月娘与了郁大姐一包二钱银子。吴银儿道:“我这里就辞了妗子、列位娘,磕了头罢。”当下吴大妗子与了一对银花儿,月娘与李瓶儿每人袖中拿出一两银子与他,磕头谢了。吴大妗子同二妗子、郑三姐都还要送月娘众人,因见天气落雪,月娘阻回去了。琴童道:“头里下的还是雪,这回沾在身上都是水珠儿,只怕湿了娘们的衣服,问妗子这里讨把伞打了家去。”吴二舅连忙取了伞来,琴童儿打着,头里两个排军打灯笼,引着一簇男女,走几条小巷,到大街上。陈敬济沿路放了许多花炮,因叫:“银姐,你家不远了,俺每送你到家。”月娘便问:“他家在那里?”敬济道:“这条胡同内一直进去,中间一座大门楼,就是他家。”吴银儿道:“我这里就辞了娘每家去。”月娘道:“地下湿,银姐家去罢,头里已是见过礼了。我还着小厮送你到家。”因叫过玳安:“你送送银家去。”敬济道:“娘,我与玳安两个去罢。”月娘道:“也罢,你与他两个同送他送。”那敬济得不的一声,同玳安一路送去了。

吴月娘众人便回家来。潘金莲路上说:“大姐姐,你原说咱每送他家去,怎的又不去了?”月娘笑道:“你也只是个小孩儿,哄你说耍子儿,你就信了。丽春院是那里,你我送去?”金莲道:“像人家汉子在院里嫖了来,家里老婆没曾往那里寻去?寻出没曾打成一锅粥?”月娘道:“你等他爹到明日往院里去,你寻他寻试试。倒没的教人家汉子当粉头拉了去,看你——”两个口里说着,看看走到东街上,将近乔大户门首。只见乔大户娘子和他外甥媳妇段大姐,在门首站立。远远见月娘一簇男女过来,就要拉请进去。月娘再三说道:“多谢亲家盛情,天晚了,不进去罢。”那乔大户娘子那里肯放,说道:“好亲家,怎的上门儿怪人家?”强把月娘众人拉进去了。客位内挂着灯,摆设酒果,有两个女儿弹唱饮酒,不题。

却说西门庆,在门首与伯爵众人饮酒将阑。伯爵与希大整吃了一日,顶颡吃不下去,见西门庆在椅子上打盹,赶眼错把果碟儿都倒在袖子里,和韩道国就走了。只落下贲四,陪西门庆打发了乐工赏钱。吩咐小厮收家火,熄灯烛,归后边去了。只见平安走来,贲四家叫道:“你们还不起身,爹进去了。”玉箫听见,和迎春、兰香慌的辞也不辞,都一溜烟跑了。只落下春梅,拜谢了贲四嫂,才慢慢走回来。看见兰香在后边脱了鞋赶不上,因骂道:“你们都抢棺材奔命哩!把鞋都跑脱了,穿不上,象甚腔儿!”到后边,打听西门庆在李娇儿房里,都来磕头。大师父见西门庆进入李娇儿房中,都躲到上房,和小玉在一处。玉箫进来,道了万福,那小玉就说玉箫:“娘那里使小厮来要皮袄,你就不来管管儿,只教我拿。我又不知那根钥匙开橱门,及自开了又没有,落后却在外边大橱拒里寻出来。你放在里头,怎昏抢了不知道?姐姐每都吃勾来了罢,几曾见长出块儿来!”玉箫吃的脸红红的,道:“怪小淫妇儿,如何狗挝了脸似的?人家不请你,怎的和俺们使性儿!”小玉道:“我稀罕那淫妇请!”大师父在旁劝道:“姐姐每义让一句儿罢,你爹在屋里听着。只怕你娘们来家,顿下些茶儿伺候。”正说着,只见琴童抱进毡包来。玉箫便问:“娘来了?”琴童道:“娘每来了,又被乔亲家娘在门首让进去吃酒哩,也将好起身。”两个才不言语了。

不一时,月娘等从乔大户娘子家出来。到家门首,贲四娘子走出来厮见。陈敬济和贲四一面取出一架小烟火来,在门首又看放了一回烟火,方才进来,与李娇儿、大师父道了万福。雪娥走来,向月娘磕了头,与玉楼等三人见了礼。月娘因问:“他爹在那里?”李娇儿道:“刚才在我那屋里,我打发他睡了。”月娘一声儿没言语。只见春梅、迎春、玉箫、兰香进来磕头。李娇儿便说:“今日前边贲四嫂请了四个去,坐了回儿就来了。”月娘听了,半日没言语。骂道:“恁成精狗肉们,平白去做甚么!谁教他去来?”李娇儿道:“问过他爹才去来。”月娘道:“问他?好有张主的货!你家初一十五开的庙门早了,放出些小鬼来了。”大师父道:“我的奶奶,恁四个上画儿的姐姐,还说是小鬼。”月娘道:“上画儿只画的半边儿,平白放出去做甚么?与人家喂眼!”孟玉楼见月娘说来的不好,就先走了。落后金莲见玉楼起身,和李瓶儿、大姐也走了。止落下大师父,和月娘同在一处睡了。那雪霰直下到四更方止。正是:

\[
香消烛冷楼台夜,挑菜烧灯扫雪天。
\]

一宿晚景题过。到次日,西门庆往衙门中去了。月娘约饭时前后,与孟玉楼、李瓶儿三个同送大师父家去。因在大门里首站立,见一个乡里卜龟儿卦儿的老婆子,穿着水合袄、蓝布裙子,勒黑包头,背着褡裢,正从街上走来。月娘使小厮叫进来,在二门里铺下卦帖,安下灵龟,说道:“你卜卜俺每。”那老婆扒在地下磕了四个头:“请问奶奶多大年纪?”月娘道:“你卜个属龙的女命。”那老婆道:“若是大龙,四十二岁,小龙儿三十岁。”月娘道:“是三十岁了,八月十五日子时生。”那老婆把灵龟一掷,转了一遭儿住了。揭起头一张卦帖儿。上面画着一个官人和一位娘子在上面坐,其余都是侍从人,也有坐的,也有立的,守着一库金银财宝。老婆道:“这位当家的奶奶是戊辰生,戊辰己巳大林木。为人一生有仁义,性格宽洪,心慈好善,看经布施,广行方便。一生操持,把家做活,替人顶缸受气,还不道是。喜怒有常,主下人不足。正是:喜乐起来笑嘻嘻,恼将起来闹哄哄。别人睡到日头半天还未起,你老早在堂前转了。梅香洗铫铛,虽是一时风火性,转眼却无心。和人说也有,笑也有,只是这疾厄宫上着刑星,常沾些啾唧。亏你这心好,济过来了,往后有七十岁活哩。”孟玉楼道:“你看这位奶奶命中有子没有?”婆子道:“休怪婆子说,儿女宫上有些不实,往后只好招个出家的儿子送老罢了。随你多少也存不的。”玉楼向李瓶儿笑道:“就是你家吴应元,见做道士家名哩。”月娘指着玉楼:“你也叫他卜卜。”玉楼道:“你卜个三十四岁的女命,十一月二十七日寅时生。”那婆子从新撇了卦帖,把灵龟一卜,转到命宫上住了。揭起第二张卦帖来,上面画着一个女人,配着三个男人:头一个小帽商旅打扮;第二个穿红官人;第三个是个秀才。也守着一库金银,左右侍从伏侍。婆子道:“这位奶奶是甲子年生。甲子乙丑海中金。命犯三刑六害,夫主克过方可。”玉楼道:“已克过了。”婆子道:“你为人温柔和气,好个性儿。你恼那个人也不知,喜欢那个人也不知,显不出来。一生上人见喜下钦敬,为夫主宠爱。只一件,你饶与人为了美,多不得人心。命中一生替人顶缸受气,小人驳杂,饶吃了还不道你是。你心地好了,虽有小人也拱不动你。”玉楼笑道:“刚才为小厮讨银子和他乱了,这回说是顶缸受气。”月娘道:“你看这位奶奶往后有子没有?”婆子道:“济得好,见个女儿罢了。子上不敢许,若说寿,倒尽有。”月娘道:“你卜卜这位奶奶。李大姐,你与他八字儿。”李瓶儿笑道:“我是属羊的。”婆子道:“若属小羊的,今年念七岁,辛未年生的。生几月?”李瓶儿道:“正月十五日午时。”那婆子卜转龟儿,到命宫上矻磴住了。揭起卦帖来,上面画着一个娘子,三个官人:头一个官人穿红,第二个官人穿绿,第三个穿青。怀着个孩儿,守着一库金银财宝,旁边立着个青脸獠牙红发的鬼。婆子道:“这位奶奶,庚午辛未路旁土。一生荣华富贵,吃也有,穿也有,所招的夫主都是贵人。为人心地有仁义,金银财帛不计较,人吃了转了他的,他喜欢;不吃他,不转他,到恼。只是吃了比肩不和的亏,凡事恩将仇报。正是:比肩刑害乱扰扰,转眼无情就放刁;宁逢虎摘三生路,休遇人前两面刀。奶奶,你休怪我说:你尽好匹红罗,只可惜尺头短了些。气恼上要忍耐些,就是子上也难为。”李瓶儿道:“今已是寄名做了道士。”婆子道:“既出了家,无妨了。又一件,你老人家今年计都星照命,主有血光之灾,仔细七八月不见哭声才好。”说毕,李瓶儿袖中掏出五分一块银子,月娘和玉楼每人与钱五十文。

刚打发卜龟卦婆子去了,只见潘金莲和大姐从后边出来,笑道:“我说后边不见,原来你每都往前头来了。”月娘道:“俺们刚才送大师父出来,卜了这回龟儿卦。你早来一步,也教他与你卜卜儿。”金莲摇头儿道:“我是不卜他。常言:算的着命,算不着行。想前日道士说我短命哩,怎的哩?说的人心里影影的。随他明日街死街埋,路死路埋,倒在洋沟里就是棺材。”说毕,和月娘同归后边去了。正是:

\[
万事不由人算计,一生都是命安排。
\]
