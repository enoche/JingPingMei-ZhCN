%# -*- coding:utf-8 -*-
%%%%%%%%%%%%%%%%%%%%%%%%%%%%%%%%%%%%%%%%%%%%%%%%%%%%%%%%%%%%%%%%%%%%%%%%%%%%%%%%%%%%%


\chapter{春梅姐游旧家池馆\KG 杨光彦作当面豺狼}


词曰:

\[
人生千古伤心事,还唱《后庭花》。旧时王谢,堂前燕子,飞向谁家?恍然一梦,仙肌胜雪,宫鬓堆雅。江州司马,青衫泪湿,想在天涯。\named{右调《青衫湿》}
\]

话说光阴迅速,日月如梭,又早到正月二十一日。春梅和周守备说了,备一张祭桌,四样羹果,一坛南酒,差家人周义送与吴月娘。一者是西门庆三周年,二者是孝哥儿生日。月娘收了礼物,打发来人帕一方,银三钱。这边连忙就使玳安儿穿青衣,具请书儿请去。上写着:

\[
重承厚礼,感感。即刻舍具菲酌,奉酬腆仪。仰希高轩俯临,不外,幸甚。西门吴氏端肃拜请大德周老夫人妆次
\]
春梅看了,到日中才来。戴着满头珠翠金凤头面钗梳,胡珠环子。身穿大红通袖、四兽朝麒麟袍儿,翠蓝十样锦百花裙,玉玎当禁步,束着金带。坐着四人大轿,青段销金轿衣。军牢执藤棍喝道,家人伴当跟随,抬着衣匣。后边两顶家人媳妇小轿儿,紧紧跟随。吴月娘这边请人吴大妗子相陪,又叫了四个唱的弹唱。听见春梅来到,月娘亦盛妆缟素打扮,头上五梁冠儿,戴着稀稀几件金翠首饰,上穿白绫袄,下边翠蓝段子裙,与大妗子迎接至前厅。春梅大轿子抬至仪门首,才落下轿来。两边家人围着,到于厅上叙礼,向月娘插烛也似拜下去。月娘连忙答礼相见,说道:“向日有累姐姐费心,粗尺头又不肯受。今又重承厚礼祭桌,感激不尽。”春梅道:“惶恐。家官府没甚么,这些薄礼,表意而已。一向要请奶奶过去,家官府不时出巡,所以不曾请得。”月娘道:“姐姐,你是几时好日子?我只到那日买礼看姐姐去罢。”春梅道:“奴贱日是四月廿五日。”月娘道:“奴到那日已定去。”

两个叙礼毕,春梅务要把月娘让起,受了两礼。然后吴大妗子相见,亦还下礼去。春梅道:“你看大妗子,又没正经。”一手扶起受礼。大妗子再三不肯,止受了半礼。一面让上坐,月娘和大妗子主位相陪。然后家人、媳妇、丫鬟、养娘,都来参见。春梅见了奶子如意儿抱着孝哥儿,吴月娘道:“小大哥还不来与姐姐磕个头儿,谢谢姐姐。今日来与你做生日。”那孝哥儿真个下如意儿身来,与春梅唱喏。月娘道:“好小厮,不与姐姐磕头,只唱喏。”那春梅连忙向袖中摸出一方锦手帕,一副金八吉祥儿,教替他塞帽儿上。月娘道:“又教姐姐费心。”又拜谢了。落后小玉、奶子来见磕头。春梅与了小玉一对头簪子,与了奶子两枝银簪儿。月娘道:“姐姐,你还不知,奶子与了来兴儿做媳妇儿了。来兴儿那媳妇害病没了。”春梅道:“他一心要在咱家,倒也好。”一面丫鬟拿茶上来,吃了茶,月娘道:“请娘娘后边明间内坐罢,这客位内冷。”

春梅来后边西门庆灵前,又早点起灯烛,摆下桌面祭礼。春梅烧了纸,落了几点眼泪。然后周围设放围屏,火炉内生起炭火,安放八大仙桌席,摆茶上来。无非是细巧蒸酥,希奇果品,绝品芽茶。月娘和大妗子陪着吃了茶,让春梅进上房里换衣裳。脱了上面袍儿,家人媳妇开衣匣,取出衣服,更换了一套绿遍地锦妆花袄儿,紫丁香色遍地金裙。在月娘房中坐着,说了一回,月娘因问道:“哥儿好么?今日怎不带他来这里走走?”春梅道:“不是也带他来与奶奶磕头,他爷说天气寒冷,怕风冒着他。他又不肯在房里,只要那当直的抱出来厅上外边走。这两日,不知怎的,只是哭。”月娘道:“他周爷也好大年纪,得你替他养下这点孩子也彀了,也是你裙带上的福。说他孙二娘还有位姐儿,几岁儿了?”春梅道:“他二娘养的叫玉姐,今年交生四岁。俺这个叫金哥。”月娘道:“说他周爷身边还有两位房里姐儿?”春梅道:“是两个学弹唱的丫头子,都有十六七岁,成日淘气在那里。”月娘道:“他爷也常往他身边去不去?”春梅道:“奶奶,他那里得工夫在家?多在外,少在里。如今四外好不盗贼生发,朝廷敕书上,又教他兼管许多事情:镇守地方,巡理河道,提拿盗贼,操练人马。常不时往外出巡几遭,好不辛苦哩。”说毕,小玉又拿茶来吃了。春梅向月娘说:“奶奶,你引我往俺娘那边花园山子下走走。”月娘道:“我的姐姐,还是那咱的山子花园哩!自从你爹下世,没人收拾他,如今丢搭的破零零的。石头也倒了,树木也死了,俺等闲也不去了。”春梅道:“不妨,奴就往俺娘那边看看去。”这月娘强不过,只得叫小玉拿花园门山子门钥匙,开了门,月娘、大妗子陪春梅,到里边游看了半日。但见:

\[
垣墙欹损,台榭歪斜。两边画壁长青笞,满地花砖生碧草。山前怪石遭塌毁,不显嵯峨;亭内凉床被渗漏,已无框档。石洞口蛛丝结网,鱼池内虾蟆成群。狐狸常睡卧云亭,黄鼠往来藏春阁。料想经年无人到,也知尽日有云来。
\]

春梅看了一回,先走到李瓶儿那边。见楼上丢着些折桌、坏凳、破椅子,下边房都空锁着,地下草长的荒荒的。方来到他娘这边,楼上还堆着些生药香料,下边他娘房里,止有两座厨柜,床也没了。因问小玉:“俺娘那张床往那去了?怎的不见?”小玉道:“俺三娘嫁人,赔了俺三娘去了。”月娘走到跟前说:“因你爹在日,将他带来那张八步床赔了大姐在陈家,落后他起身,却把你娘这张床赔了他,嫁人去了。”春梅道:“我听见大姐死了,说你老人家把床还抬的来家了。”月娘道:“那床没钱使,只卖了八两银子,打发县中皂隶,都使了。”春梅听言,点了点头儿。那星眼中由不的酸酸的,口中不言,心内暗道:“想着俺娘那咱,争强不伏弱的问爹要买了这张床。我实承望要回了这张床去,也做他老人家一念儿,不想又与了人去了。”由不的心下惨切。又问月娘:“俺六娘那张螺甸床怎的不见?”月娘道:“一言难尽。自从你爹下世,日逐只有出去的,没有进来的。常言家无营活计,不怕斗量金。也是家中没盘缠,抬出去交人卖了。”春梅问:“卖了多少银子?”月娘道:“止卖了三十五两银子。”春梅道:“可惜了,那张床,当初我听见爹说,值六十两多银子,只卖这些儿。早知你老人家打发,我到与你老人家三四十两银子要了也罢。”月娘道:“好姐姐,人那有早知道的?”一面叹息了半日。

只见家人周仁走来接,说:“爷请奶奶早些家来,哥儿寻奶奶哭哩。”这春梅就抽身往后边来。月娘叫小玉锁了花园门,同来到后边明间内。又早屏开孔雀,帘控鲛绡,摆下酒筵。两个妓女,银筝琵琶,在旁弹唱。吴月娘递酒安席,安春梅上座,春梅不肯,务必拉大妗子,同他一处坐的。月娘主位,筵前递了酒,汤饭点心,割切上席。春梅叫家人周仁,赏了厨子三钱银子。说不尽盘堆羿品,酒泛金波。当下传杯换盏,吃至晚色将落时分,只见宅内又差伴当,拿灯笼来接。月娘那里肯放,教两个妓女在跟前跪着弹唱劝酒。分付:“你把好曲儿孝顺你周奶奶一个儿。”一面叫小玉斟上大钟,放在跟前,说:“姐姐,你分付个心爱的曲儿,叫他两个唱与你下酒。”春梅道:“奶奶,奴吃不得了,怕孩儿家中寻我。”月娘道:“哥儿寻,左右有奶子看着,天色也还早哩,我晓得你好小量儿!”春梅因问那两个妓女:“你叫甚名字?是谁家的?”两个跪下说:“小的一个是韩金钏儿妹子韩玉钏儿,一个是郑爱香儿侄女郑娇儿。”春梅道:“你每会唱《懒画眉》不会?”玉钏儿道:“奶奶分付,小的两个都会。”月娘道:“你两个既会唱,斟上酒你周奶奶吃,你每慢唱。”小玉在旁连忙斟上酒,两个妓女,一个弹筝,一个琵琶,唱道:

\[
冤家为你几时休?捱到春来又到秋。谁人知道我心头。天,害的我伶仃瘦,听和音书两泪流。从前已往诉缘由,谁想你无情把我丢!
\]
那春梅吃过,月娘双令郑娇儿递上一杯酒与春梅。春梅道:“你老人家也陪我一杯。”两家于是都齐斟上,两个妓女又唱道:

\[
冤家为你减风流,鹊噪檐前不肯休,死声活气没来由。天,倒惹的情拖逗,助的凄凉两泪流。从他去后意无休,谁想你辜恩把我丢。
\]
春梅说:“奶奶,你也教大妗子吃杯儿。”月娘道:“大妗子吃不的,教他拿小钟儿陪你罢。”一面令小玉斟上大妗子一小钟儿酒。两个妓女又唱道:

\[
冤家为你惹场忧,坐想行思日夜愁,香肌憔瘦减温柔。天,要见你不能勾,闷的我伤心两泪流。从前与你共绸缪,谁想你今番把我丢。
\]
春梅见小玉在跟前,也斟了一大钟教小玉吃。月娘道:“姐姐,他吃不的。”春梅道:“奶奶,他也吃两三钟儿,我那咱在家里没和他吃?”于是斟上,教小玉也吃了一杯。妓女唱道:

\[
冤家为你惹闲愁,病枕着床无了休,满腹忧闷锁眉头。天,忘了还依旧,助的我腮边两泪流。从前与你两无休,谁想你经年把我丢。
\]

看官听说,当时春梅为甚教妓女唱此词?一向心中牵挂陈敬济,在外不得相会。情种心苗,故有所感,发于吟咏。又见他两个唱的口儿甜,乖觉,奶奶长、奶奶短奉承,心中欢喜。叫家人周仁近前来,拿出两包儿赏赐来,每人二钱银子。两个妓女放下乐器,磕头谢了。不一时,春梅起身,月娘款留不住。伴当打灯笼,拜辞出门,坐上大轿。家人媳妇,都坐上小轿。前后打着四个灯笼,军牢喝道而去。正是:时来顽铁有光辉,远去黄金无艳色。有诗为证:

\[
点绛唇红弄玉娇,凤凰飞下品鸾箫。
堂高闲把湘帘卷,燕子还来续旧巢。
\]

且说春梅自从来吴月娘家赴席之后,因思想陈敬济,不知流落在何处。归到府中,终日只是卧床不起,心下没好气。守备察知其意,说道:“只怕思念你兄弟,不得其所。”一面叫张胜、李安来,分付道:“我一向委你寻你奶奶兄弟,如何不用心找寻?”二人告道:“小的一向找寻来,一地里寻不着下落,已回了奶奶话了。”守备道:“限你二人五日,若找寻不着,讨分晓。”这张胜、李安领了钧语下来,都带了愁颜。沿街绕巷,各处留心,找问不题。

话分两头。单表陈敬济自从守备府中打了出来,欲投宴公庙。又听见人说师父任道士死了,就害怕不敢进庙来,又没脸儿见杏庵主老,白日里到处打油飞,夜晚间还钻入冷铺中存身。一日,也是合当有事,敬济正在街上站立,只见铁指甲杨大郎,头戴新罗帽儿,身穿白绫袄子,骑着一匹驴儿,拣银鞍辔,一个小厮跟随,正从街心走过来。敬济认得是杨光彦,便向前一把手,把嚼环拉住,说道:“杨大哥,一向不见。自从清江浦把我半船货物偷拐走了,我好意往你家问,反吃你兄弟杨二风拿瓦楔钻破头,赶着打上我家门来。今日弄的我一贫如洗,你是会摇摆受用。”那杨大郎见陈敬济已自讨吃,便佯佯而笑,说:“今日晦气,出门撞见瘟死鬼,量你这饿不死贼花子,那里讨半船货?我拐了你的,你不撒手?须吃我一顿马鞭子。”敬济便道:“我如今穷了,你有银子,与我些盘缠。不然,咱到个去处讲讲。”杨大郎见他不放,跳下驴来,向他身上抽了几鞭子。喝令小厮:“与我撏了这少死的花子去!”那小厮使力把敬济推了一交,杨大郎又向前踢了几脚,踢打的敬济怪叫。须臾,围了许多人。旁边闪过一个人来,青高装帽子,勒着手帕,倒披紫袄,白布裤子,精着两条腿,趿着蒲鞋,生的阿兜眼,扫帚眉,料绰口,三须胡子,面上紫肉横生,手腕横筋竞起。吃的楞楞睁睁,提着拳头,向杨大郎说道:“你此位哥好不近理,他年少这般贫寒,你只顾打他怎的?自古嗔拳不打笑面,他又不曾伤犯着你。你有钱,看平日相交,与他些;没钱罢了,如何只顾打他?自古路见不平,也有向灯向火。”杨大郎说:“你不知,他赖我拐了他半船货,量他恁穷样,那有半船货物?”那人道:“想必他当时也是有根基人家娃娃,天生就这般穷来?阁下就是这般有钱?老兄依我,你有银子与他些盘缠罢。”那杨大郎见那人说了,袖内汗巾儿上拴着四五钱一块银子,解下来递与敬济,与那人举一举手儿,上驴子扬长去了。

敬济地下扒起来,抬头看那人时,不是别人,却是旧时同在冷铺内,和他一铺睡的土作头儿飞天鬼侯林儿。近来领着五十名人,在城南水月寺晓月长老那里做工,起盖伽蓝殿。因一只手拉着敬济说道:“兄弟,刚才若不是我拿几句言语讥犯他,他肯拿出这五钱银子与你?那贼却知见范,他若不知范时,好不好吃我一顿好拳头。你跟着我,咱往酒店内吃酒去来。”到一个食荤小酒店,案头上坐下,叫量酒:“拿四卖嗄饭,两大壶酒来。”不一时,量酒摆下小菜嗄饭,四盘四碟,两大坐壶时兴橄榄酒。不用小杯,拿大磁瓯子,因问敬济:“兄弟,你吃面吃饭?”量酒道:“面是温淘,饭是白米饭。”敬济道:“我吃面。”须臾,掉上两三碗温面上来。侯林儿只吃一碗,敬济吃了两碗。然后吃酒。侯林儿向敬济说:“兄弟,你今日跟我往坊子里睡一夜,明日我领你城南水月寺晓月长老那里,修盖伽蓝殿,并两廊僧房。你哥率领着五十名做工。你到那里,不要你做重活,只抬几筐土儿就是了,也算你一工,讨四分银子。我外边赁着一间厦子,晚夕咱两个就在那里歇,做些饭打发咱的人吃。把门你一把锁锁了,家当都交与你,好不好?强如你在那冷铺中,替花子摇铃打梆,这个还官样些。”敬济道:“若是哥哥这般下顾兄弟,可知好哩。不知这工程做的长远不长远?”侯林儿道:“才做了一个月。这工程做到十月里,不知完不完。”两个说话之间,你一钟,我一盏,把两大壶酒都吃了。量酒算帐,该一钱三分半银子。敬济就要拿出银子来秤,侯林儿推过一边,说:“傻兄弟,莫不教你出钱?哥有银子在此。”一面扯出包儿来,秤了一钱五分银子与掌柜的。还找了一分半钱袖了,搭伏着敬济肩背,同到坊子里,两个在一处歇卧。二人都醉了。这侯林儿晚夕干敬济后庭花,足干了一夜。亲哥、亲达达、亲汉子、亲爷,口里无般不叫将出来。

到天明,同往城南水月寺。果然寺外侯林儿赁下半间厦子,里面烧着炕柴,早也买下许多碗盏家活。早辰上工,叫了名字。众人看见敬济,不上二十四五岁,白脸子,生的眉目清俊,就知是侯林儿兄弟,都乱调戏他。先问道:“那小伙子儿,你叫甚名字?”陈敬济道:“我叫陈敬济。”那人道:“陈敬济,可不由着你就挤了。”又一人说:“你恁年小小的,怎干的这营生?捱的这大扛头子?”侯林儿喝开众人,骂:“怪花子,你只顾奚落他怎的?”一面散了锹镢筐扛,派众人抬土的抬土,和泥的和泥,打杂的打杂。

原来晓月长老,教一个叶头陀做火头,造饭与各作匠人吃。这叶头陀年约五十岁,一个眼瞎,穿着皂直裰,精着脚,腰间束着烂绒绦,也不会看经,只会念佛,善会麻衣神相。众人都叫他做叶道。一日做了工下来,众人都吃毕饭,也有闲坐的,卧的,也有蹲着的。只见敬济走向前,问叶头陀讨茶吃。这叶头陀只顾上上下下看他。内有一人说:“叶道,这个小伙子儿是新来的,你相他一相。”又一人说:“你相他相,倒相个兄弟。”一个说:“倒相个二尾子。”叶头陀教他近前,端详了一回,说道:“色怕嫩兮又怕娇,声娇气嫩不相饶。老年色嫩招辛苦,少年色嫩不坚牢。只吃了你面皮嫩的亏,一生多得阴人宠爱。八岁十八二十八,做作百般人可爱,纵然弄假又成真。休怪我说,一生心伶机巧,常得阴人发迹。你今多大年纪?”敬济道:“我二十四岁。”叶道道:“亏你前年怎么过来,吃了你印堂太窄,子丧妻亡,悬壁昏暗,人亡家破;唇不盖齿,一生惹是招非;鼻若灶门,家私倾散。那一年遭官司口舌,倾家散业,见过不曾?”敬济道:“都见过了。”叶头陀道:“只一件,你这山根不宜断绝。麻衣祖师说得两句好:‘山根断兮早虚花,祖业飘零定破家。’早年父祖丢下家业,不拘多少,到你手里,都了当了。你上停短兮下停长,主多成多败,钱财使尽又还来。总然你久后营得家计,犹如烈日照冰霜。你如今往后,还有一步发迹,该有三妻之命。克过一个妻宫不曾?”敬济道:“已克过了。”叶头陀道:“后来还有三妻之会,但恐美中不美。三十上,小人有些不足,花柳中少要行走。”一个人说:“叶道,你相差了,他还与人家做老婆,那有三个妻来?”众人正笑做一团,只听得晓月长老打梆了,各人都拿锹镢筐扛,上工做活去了。如此者,敬济在水月寺,也做了约一月光景。

一日,三月中旬天气,敬济正与众人抬出土来,在山门墙下,倚着墙根,向日阳蹲踞着捉身上虱虮。只见一个人,头带万字头巾,身穿青窄衫,紫裹肚,腰系缠带,脚穿扁靴,骑着一匹黄马,手中提着一篮鲜花儿。见了敬济,猛然跳下马来,向前深深的唱了诺,便叫:“陈舅,小人那里没寻,你老人家原来在这里。”倒唬了敬济一跳。连忙还礼不迭,问:“哥哥,你是那里来的?”那人道:“小人是守备周爷府中亲随张胜,自从舅舅府中官事出来,奶奶不好直到如今,老爷使小人那里不找寻舅舅,不知在这里。今早不是俺奶奶使小人到外庄上,折取这几杂芍药花儿,打这里过,怎得看见你老人家在这里?一来也是你老人家际遇,二者小人有缘。不消犹豫,就骑上马,我跟你老人家往府中去。”那众做工的人看着,面面相觑,不敢做声。这陈敬济把钥匙递与侯林儿,骑上马,张胜紧紧跟随,径往守备府中来。正是:良人得意正年少,今夜月明何处楼?有诗为证:

\[
白玉隐于顽石里,黄金埋在污泥中。
今朝贵人提拔起,如立天梯上九重。
\]
