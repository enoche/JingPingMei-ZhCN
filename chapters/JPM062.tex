%# -*- coding:utf-8 -*-
%%%%%%%%%%%%%%%%%%%%%%%%%%%%%%%%%%%%%%%%%%%%%%%%%%%%%%%%%%%%%%%%%%%%%%%%%%%%%%%%%%%%%


\chapter{潘道士法遣黄巾士\KG 西门庆大哭李瓶儿}


诗曰:

\[
玉钗重合两无缘,鱼在深潭鹤在天。
得意紫鸾休舞镜,传言青鸟罢衔笺。
金盆已覆难收水,玉轸长笼不续弦。
若向蘼芜山下过,遥将红泪洒穷泉。
\]

话说西门庆见李瓶儿服药无效,求神问卜发课,皆有凶无吉,无法可处。初时,李瓶儿还\def\textMenZuo \textuni{499B}着梳头洗脸,下炕来坐净桶,次后渐渐饮食减少,形容消瘦,那消几时,把个花朵般人儿,瘦弱得黄叶相似,也不起炕了,只在床褥上铺垫草纸。恐怕人嫌秽恶,教丫头只烧着香。西门庆见他胳膊儿瘦得银条相似,只守着在房内哭泣,衙门中隔日去走一走。李瓶儿道:“我的哥,你还往衙门中去,只怕误了你公事。我不妨事,只吃下边流的亏,若得止住了,再把口里放开,吃些饮食儿,就好了。你男子汉,常绊在我房中做甚么!”西门庆哭道:“我的姐姐,我见你不好,心中舍不的你。”李瓶儿道:“好傻子,只不死,死将来你拦的住那些!”又道:“我有句话要对你说:我不知怎的,但没人在房里,心中只害怕,恰似影影绰绰有人在跟前一般。夜里要便梦见他,拿刀弄杖,和我厮嚷,孩子也在他怀里。我去夺,反被他推我一交,说他又买了房子,来缠了好几遍,只叫我去。只不好对你说。”西门庆听了说道:“人死如灯灭,这几年知道他往那里去了!此是你病的久,神虚气弱了,那里有甚么邪魔魍魉、家亲外祟!我如今往吴道官庙里,讨两道符来,贴在房门上,看有邪祟没有。”

说毕,走到前边,即差玳安骑头口往玉皇庙讨符去。走到路上,迎见应怕爵和谢希大,忙下头口。伯爵因问:“你往那里去?你爹在家里?”玳安道:“爹在家里,小的往玉皇庙讨符去。”伯爵与谢希大到西门庆家,因说道:“谢子纯听见嫂子不好,唬了一跳,敬来问安。”西门庆道:“这两日身上瘦的通不象模样了,丢的我上不上,下不下,却怎生样的?”伯爵道:“哥,你使玳安往庙里做甚么去?”西门庆悉把李瓶儿害怕之事告诉一遍:“只恐有邪祟,教小厮讨两道符来镇压镇压。”谢希大道:“哥,此是嫂子神气虚弱,那里有甚么邪祟!”伯爵道:“哥若遣邪也不难,门外五岳观潘道士,他受的是天心五雷法,极遣的好邪,有名唤着潘捉鬼,常将符水救人。哥,你差人请他来,看看嫂子房里有甚邪祟,他就知道。你就教他治病,他也治得。”西门庆道:“等讨了吴道官符来看,在那里住?没奈何,你就领小厮骑了头口,请了他来。”伯爵道:“不打紧,等我去。天可怜见嫂子好了,我就头着地也走。”说了一回话,伯爵和希大起身去了。

玳安儿讨了符来,贴在房中。晚间李瓶儿还害怕,对西门庆说:“死了的,他刚才和两个人来拿我,见你进来,躲出去了。”西门庆道:“你休信邪,不妨事。昨日应二哥说,此是你虚极了。他说门外五岳观有个潘道士,好符水治病,又遣的好邪,我明日早教应伯爵去请他来看你,有甚邪祟,教他遣遣。”李瓶儿道:“我的哥哥,你请他早早来,那厮他刚才发恨而去,明日还来拿我哩!你快些使人请去。”西门庆道:“你若害怕,我使小厮拿轿子接了吴银儿,和你做两日伴儿。”李瓶儿摇头儿说:“你不要叫他,只怕误了他家里勾当。”西门庆道:“叫老冯来伏侍你两日儿如何?”李瓶儿点头儿。这西门庆一面使来安,往那边房子里叫冯妈妈,又不在,锁了门出去了。对一丈青说下:“等他来,好歹教他快来宅内,六娘叫他哩。”西门庆一面又差下玳安:“明日早起,你和应二爹往门外五岳观请潘道士去。”俱不在话下。

次日,只见王姑子挎着一盒儿粳米、二十块大乳饼、一小盒儿十香瓜茄来看。李瓶儿见他来,连忙教迎春搊扶起来坐的。王姑子道了问讯,李瓶儿请他坐下,道:“王师父,你自印经时去了,影边儿通不见你。我恁不好,你就不来看我看儿?”王姑子道:“我的奶奶,我通不知你不好,昨日大娘使了大官儿到庵里,我才晓得。又说印经哩,你不知道,我和薛姑子老淫妇合了一场好气。与你老人家印了一场经,只替他赶了网儿。背地里和印经的打了五两银子夹帐,我通没见一个钱儿。你老人家作福,这老淫妇到明日堕阿鼻地狱!为他气的我不好了,把大娘的寿日都误了,没曾来。”李瓶儿道:“他各人作业,随他罢,你休与他争执了。”王姑子道:“谁和他争执甚么。”李瓶儿道:“大娘好不恼你哩,说你把他受生经都误了。”王姑子道:“我的菩萨,我虽不好,敢误了他的经?——在家整诵了一个月,昨日圆满了,今日才来。先到后边见了他,把我这些屈气告诉了他一遍。我说,不知他六娘不好,没甚么,这盒粳米和些十香爪、几块乳饼,与你老人家吃粥儿。大娘才叫小玉姐领我来看你老人家。”小玉打开盒儿,李瓶儿看了说道:“多谢你费心。”王姑子道:“迎春姐,你把这乳饼就蒸两块儿来,我亲看你娘吃些粥儿。”迎春一面收下去了。李瓶儿吩咐迎春:“摆茶来与王师父吃。”王姑子道:“我刚才后边大娘屋里吃了茶,煎些粥来,我看着你吃些。”

不一时,迎春安放桌儿,摆了四样茶食,打发王姑子吃了,然后拿上李瓶儿粥来,一碟十香甜酱瓜茄、一碟蒸的黄霜霜乳饼、两盏粳米粥,一双小牙筷。迎春拿着,奶子如意儿在旁拿着瓯儿,喂了半日,只呷了两三口粥儿,咬了一些乳饼儿,就摇头儿不吃了,教:“拿过去罢。”王姑子道:“人以水食为命,恁煎的好粥儿,你再吃些儿不是?”李瓶儿道:“也得我吃得下去是!”迎春便把吃茶的桌儿掇过去。王姑子揭开被,看李瓶儿身上,肌体都瘦的没了,唬了一跳,说道:“我的奶奶,我去时你好些了,如何又不好了,就瘦的恁样的了?”如意儿道:“可知好了哩!娘原是气恼上起的病,爹请了太医来看,每日服药,已是好到七八分了。只因八月内,哥儿着了惊唬不好,娘昼夜忧戚,那样劳碌,连睡也不得睡,实指望哥儿好了,不想没了。成日哭泣,又着了那暗气,暗恼在心里,就是铁石人也禁不的,怎的不把病又发了!是人家有些气恼儿,对人前分解分解也还好,娘又不出语,着紧问还不说哩。”王姑子道:“那讨气来?你爹又疼他,你大娘又敬他,左右是五六位娘,端的谁气着他?”奶子道:“王爷,你不知道——”因使绣春外边瞧瞧,看关着门不曾:“——俺娘都因为着了那边五娘一口气。——他那边猫挝了哥儿手,生生的唬出风来。爹来家,那等问着,娘只是不说。落后大娘说了,才把那猫来摔杀了。他还不承认,拿我每煞气。八月里,哥儿死了,他每日那边指桑树骂槐树,百般称快。俺娘这屋里分明听见,有个不恼的!左右背地里气,只是出眼泪。因此这样暗气暗恼,才致了这一场病。——天知道罢了!娘可是好性儿,好也在心里,歹也在心里,姊妹之间,自来没有个面红面赤。有件称心的衣裳,不等的别人有了,他还不穿出来。这一家子,那个不叨贴娘些儿?可是说的,饶叨贴了娘的,还背地不道是。”王姑子道:“怎的不道是?”如意儿道:“象五娘那边潘姥姥,来一遭,遇着爹在那边歇,就过来这屋里和娘做伴儿。临去,娘与他鞋面、衣服、银子,甚么不与他?五娘还不道是。”李瓶儿听见,便嗔如意儿:“你这老婆,平白只顾说他怎的?我已是死去的人了,随他罢了。天不言而自高,地不言而自厚。”王姑子道:“我的佛爷,谁如你老人家这等好心!天也有眼,望下看着哩。你老人家往后来还有好处。”李瓶儿道:“王师父,还有甚么好处!一个孩儿也存不住,去了。我如今又不得命,身底下弄这等疾,就是做鬼,走一步也不得个伶俐。我心里还要与王师父些银子儿,望你到明日我死了,你替我在家请几位师父,多诵些《血盆经》,忏忏我这罪业。”王姑子道:“我的菩萨,你老人家忒多虑了。你好心人,龙天自然加护。”正说着,只见琴童儿进来对迎春说:“爹吩咐把房内收拾收拾,花大舅便进来看娘,在前边坐着哩。”王姑子便起身说道:“我且往后边去走走。”李瓶儿道:“王师父,你休要去了,与我做两日伴儿,我还和你说话哩。”王姑子道:“我的奶奶,我不去。”

不一时,西门庆陪花大舅进来看问,见李瓶儿睡在炕上不言语,花子由道:“我不知道,昨日听见这边大官儿去说,才晓的。明日你嫂子来看你。”那李瓶儿只说了一声:“多有起动。”就把面朝里去了。花子由坐了一回,起身到前边,向西门庆说道:“俺过世老公公在广南镇守,带的那三七药,曾吃了不曾?不拘妇女甚崩漏之疾,用酒调五分末儿,吃下去即止。大姐他手里曾收下此药,何不服之?”西门庆道:“这药也吃过了。昨日本县胡大尹来拜,我因说起此疾,他也说了个方儿:棕炭与白鸡冠花煎酒服之。只止了一日,到第二日,流的比常更多了。”花子由道:“这个就难为了。姐夫,你早替他看下副板儿,预备他罢。明日教他嫂子来看他。”说毕,起身去了。

奶子与迎春正与李瓶儿垫草纸在身底下,只见冯妈妈来到,向前道了万福。如意儿道:“冯妈妈贵人,怎的不来看看娘?昨日爹使来安儿叫你去,说你锁着门,往那里去来?”冯婆子道:“说不得我这苦。成日往庙里修法,早晨出去了,是也直到黑,不是也直到黑来家,偏有那些张和尚、李和尚、王和尚。”如意儿道:“你老人家怎的有这些和尚?早时没王师父在这里?”那李瓶儿听了,微笑了一笑儿,说道:“这妈妈子,单管只撒风。”如意儿道:“冯妈妈,叫着你还不来!娘这几日,粥儿也不吃,只是心内不耐烦,你刚才来到,就引的娘笑了一笑儿。你老人家伏侍娘两日,管情娘这病就好了。”冯妈妈道:“我是你娘退灾的博士!”又笑了一回。因向被窝里摸了摸他身上,说道:“我的娘,你好些儿也罢了!”又问:“坐杩子还下的来?”迎春道:“下的来倒好!前两遭,娘还\def\textMenZuo \textuni{499B},俺每搊扶着下来。这两日通只在炕上铺垫草纸,一日两三遍。”

正说着,只见西门庆进来,看见冯妈妈,说道:“老冯,你也常来这边走走,怎的去了就不来?”婆子道:“我的爷,我怎不来?这两日腌菜的时候,挣两个钱儿,腌些菜在屋里,遇着人家领来的业障,好与他吃。不然,我那讨闲钱买菜来与他吃?”西门庆道:“你不对我说,昨日俺庄子上起菜,拨两三畦与你也够了。”婆子道:“又敢缠你老人家。”说毕,过那边屋里去了。

西门庆便坐在炕沿上,迎春在旁熏爇芸香。西门庆便问:“你今日心里觉怎样?”又问迎春:“你娘早晨吃些粥儿不曾?”迎春道:“吃的倒好!王师父送了乳饼,蒸来,娘只咬了一些儿,呷了不上两口粥汤,就丢下了。”西门庆道:“应二哥刚才和小厮门外请那潘道士,又不在了。明日我教来保再请去。”李瓶儿道:“你上紧着人请去,那厮,但合上眼,只在我跟前缠。”西门庆道:“此是你神弱了,只把心放正着,休要疑影他。请他来替你把这邪崇遣遣,再服他些药,管情你就好了。”李瓶儿道:“我的哥哥,奴已是得了这个拙病,那里好甚么!奴指望在你身边团圆几年,也是做夫妻一场,谁知到今二十七岁,先把冤家死了,奴又没造化,这般不得命,抛闪了你去。若得再和你相逢,只除非在鬼门关上罢了。”说着,一把拉着西门庆手,两眼落泪,哽哽咽咽,再哭不出声来。那西门庆又悲恸不胜,哭道:“我的姐姐,你有甚话,只顾说。”两个正在屋里哭,忽见琴童儿进来,说:“答应的禀爹,明日十五,衙门里拜牌,画公座,大发放,爹去不去?班头好伺候。”西门庆道:“我明日不得去,拿帖儿回了夏老爹,自己拜了牌罢。”琴童应诺去了。李瓶儿道:“我的哥哥,你依我还往衙门去,休要误了公事。我知道几时死,还早哩!”西门庆道:“我在家守你两日儿,其心安忍!你把心来放开,不要只管多虑了。刚才花大舅和我说,教我早与你看下副寿木,冲你冲,管情你就好了。”李瓶儿点头儿,便道:“也罢,你休要信着人使那憨钱,将就使十来两银子,买副熟料材儿,把我埋在先头大娘坟旁,只休把我烧化了,就是夫妻之情。早晚我就抢些浆水,也方便些。你偌多人口,往后还要过日子哩!”西门庆不听便罢,听了如刀剜肝胆、剑锉身心相似。哭道:“我的姐姐,你说的是那里话!我西门庆就穷死了,也不肯亏负了你!”

正说着,只见月娘亲自拿着一小盒儿鲜苹菠进来,说道:“李大姐,他大妗子那里送苹菠儿来你吃。”因令迎春:“你洗净了,拿刀儿切块来你娘吃。”李瓶儿道:“又多谢他大妗子挂心。”不一时,迎春旋去皮儿,切了,用瓯儿盛贮,拈了一块,与他放在口内,只嚼了些味儿,还吐出来了。月娘恐怕劳碌他,安顿他面朝里就睡了。

西门庆与月娘都出外边商议。月娘道:“李大姐,我看他有些沉重,你须早早与他看一副材板儿,省得到临时马捉老鼠,又乱不出好板来。”西门庆道:“今日花大哥也是这般说。适才我略与他题了题儿,他吩咐:‘休要使多了钱,将就抬副熟板儿罢。你偌多人口,往后还要过日子。’倒把我伤心了这一会。我说亦发等请潘道士来看了,看板去罢。”月娘道:“你看没分晓,一个人形也脱了,关口都锁住,勺水也不进,还指望好!咱一壁打鼓,一壁磨旗。幸的他好了,把棺材就舍与人,也不值甚么。”西门庆道:“既是恁说……”就出到厅上,叫将贲四来,问他:“谁家有好材板,你和姐夫两个拿银子看一副来。”贲四道:“大街上陈千户家,新到了几副好板。”西门庆道:“既有好板,”即令陈敬济:“你后边问你娘要五锭大银子来,你两个看去。”那陈敬济忙进去取了五锭元宝出来,同贲四去了。直到后晌才来回话,说:“到陈千户家看了几副板,都中等,又价钱不合。回来路上,撞见乔亲家爹,说尚举人家有一副好板——原是尚举人父亲在四川成都府做推官时,带来预备他老夫人的两副桃花洞,他使了一副,只剩下这一副——墙磕、底盖、堵头俱全,共大小五块,定要三百七十两银子。乔亲家爹同俺每过去看了,板是无比的好板。乔亲家与做举人的讲了半日,只退了五十两银子。不是明年上京会试用这几两银子,他也还舍不得卖哩。”西门庆道:“既是你乔亲家爹主张,兑三百二十两抬了来罢,休要只顾摇铃打鼓的。”陈敬济道:“他那里收了咱二百五十两,还找与他七十两银子就是了。”一面问月娘又要出七十两银子,二人去了。

比及黄昏时分,只见几个闲汉,用大红毡条裹着,抬板进门,放在前厅天井内。打开,西门庆观看,果然好板。随即叫匠人来锯开,里面喷香。每块五寸厚,二尺五寸宽,七尺五寸长。看了满心欢喜。又旋寻了伯爵到来看,因说:“这板也看得过了。”伯爵喝采不已,说道,“原说是姻缘板,大抵一物必有一主。嫂子嫁哥一场,今日情受这副材板够了。”吩咐匠人:“你用心只要做的好,你老爹赏你五两银子。”匠人道:“小人知道。”一面在前厅七手八脚,连夜攒造。伯爵嘱来保:“明日早五更去请潘道士,他若来,就同他一答儿来,不可迟滞。”说毕,陪西门庆在前厅看着做材,到一更时分才家去。西门庆道:“明日早些来,只怕潘道士来的早。”伯爵道:“我知道。”作辞出门去了。

却说老冯与王姑子,晚夕都在李瓶儿屋里相伴。只见西门庆前边散了,进来看视,要在屋里睡。李瓶儿不肯,说道:“没的这屋里龌龌龊龊的,他每都在这里,不方便,你往别处睡去罢。”西门庆又见王姑子都在这里,遂过那边金莲房里去了。

李瓶儿教迎春把角门关了,上了拴,教迎春点着灯,打开箱子,取出几件衣服、银首饰来,放在旁边。先叫过王姑子来,与了他五两一锭银子、一匹绸子:“等我死后,你好歹请几位师父,与我诵《血盆经忏》。”王姑子道:“我的奶奶,你忒多虑了。天可怜见,你只怕好了。”李瓶儿道:“你只收着,不要对大娘说我与你银子,只说我与了你这匹绸子做经钱。”王姑子道,“我知道。”于是把银子和绸子收了。又唤过冯妈妈来,向枕头边也拿过四两银子、一件白绫袄、黄绫裙、一根银掠儿,递与他,说道:“老冯,你是个旧人,我从小儿,你跟我到如今。我如今死了去,也没甚么,这一套衣服并这件首饰儿,与你做一念儿。这银子你收着,到明日做个棺材本儿。你放心,那边房子,等我对你爹说,你只顾住着,只当替他看房儿,他莫不就撵你不成!”冯妈妈一手接了银子和衣服,倒身下拜,哭着说道:“老身没造化了。有你老人家在一日,与老身做一日主儿。你老人家若有些好歹,那里归着?”李瓶儿又叫过奶子如意儿,与了他一袭紫绸子袄儿、蓝绸裙、一件旧绫披袄儿、两根金头簪子、一件银满冠儿,说道:“也是你奶哥儿一场。哥儿死了,我原说的,教你休撅上奶去,实指望我在一日,占用你一日,不想我又死去了。我还对你爹和你大娘说,到明日我死了,你大娘生了哥儿,就教接你的奶儿罢。这些衣服,与你做一念儿,你休要抱怨。”那奶子跪在地下,磕着头哭道:“小媳妇实指望伏侍娘到头,娘自来没曾大气儿呵着小媳妇。还是小媳妇没造化,哥儿死了,娘又病的这般不得命。好歹对大娘说,小媳妇男子汉又没了,死活只在爹娘这里答应了,出去投奔那里?”说毕,接了衣服首饰,磕了头起来,立在旁边,只顾揩眼泪。李瓶儿一面叫过迎春、绣春来跪下,嘱咐道:“你两个,也是你从小儿在我手里答应一场,我今死去,也顾不得你每了。你每衣服都是有的,不消与你了。我每人与你这两对金裹头簪儿、两枝金花儿做一念儿。大丫头迎春,已是他爹收用过的,出不去了,我教与你大娘房里拘管。这小丫头绣春,我教你大娘寻家儿人家,你出身去罢。省的观眉说眼,在这屋里教人骂没主子的奴才。我死了,就见出样儿来了。你伏侍别人,还象在我手里那等撤娇撒痴,好也罢,歹也罢了,谁人容的你?”那绣春跪在地下哭道:“我娘,我就死也不出这个门。”李瓶儿道:“你看傻丫头,我死了,你在这屋里伏侍谁?”绣春道:“我守着娘的灵。”李瓶儿道:“就是我的灵,供养不久,也有个烧的日子,你少不的也还出去。”绣春道:“我和迎春都答应大娘。”李瓶儿道:“这个也罢了。”这绣春还不知甚么,那迎春听见李瓶儿嘱咐他,接了首饰,一面哭的言语都说不出来。正是:

\[
流泪眼观流泪眼,断肠人送断肠人。
\]

当夜,李瓶儿都把各人嘱咐了。到天明,西门庆走进房来。李瓶儿问:“买了我的棺材来了没有?”西门庆道:“昨日就抬了板来,在前边做哩。——且冲冲你,你若好了,情愿舍与人罢。”李瓶儿因问:“是多少银子买的?休要使那枉钱。”西门庆道:“没多,只百十两来银子。”李瓶儿道:“也还多了。预备下,与我放着。”西门庆说了回出来,前边看着做材去了。吴月娘和李娇儿先进房来,看见他十分沉重,便问道:“李大姐,你心里却怎样的?”李瓶儿攥着月娘手哭道:“大娘,我好不成了。”月娘亦哭道:“李大姐,你有甚么话儿,二娘也在这里,你和俺两个说。”李瓶儿道:“奴有甚话儿——奴与娘做姊妹这几年,又没曾亏了我,实承望和娘相守到白头,不想我的命苦,先把个冤家没了,如今不幸,我又得了这个拙病死去了。我死之后,房里这两个丫头无人收拘。那大丫头已是他爹收用过的,教他往娘房里伏侍娘。小丫头,娘若要使唤,留下;不然,寻个单夫独妻,与小人家做媳妇儿去罢,省得教人骂没主子的奴才。也是他伏侍奴一场,奴就死,口眼也闭。奶子如意儿,再三不肯出去,大娘也看奴分上,也是他奶孩儿一场,明日娘生下哥儿,就教接他奶儿罢。”月娘说道:“李大姐,你放宽心,都在俺两个身上。说凶得吉,若有些山高水低,迎春教他伏侍我,绣春教他伏侍二娘罢。如今二娘房里丫头不老实做活,早晚要打发出去,教绣春伏侍他罢。奶子如意儿,既是你说他没投奔,咱家那里占用不下他来?就是我有孩子没孩子,到明日配上个小厮,与他做房家人媳妇也罢了。”李娇儿在旁便道:“李大姐,你休只要顾虑,一切事都在俺两个身上。绣春到明日过了你的事,我收拾房内伏侍我,等我抬举他就是了。”李瓶儿一面叫奶子和两个丫头过来,与二人磕头。那月娘由不得眼泪出。

不一时,盂玉楼、潘金莲、孙雪娥都进来看他,李瓶儿都留了几句姊妹仁义之言。落后待的李娇儿、玉楼、金莲众人都出去了,独月娘在屋里守着他,李瓶儿悄悄向月娘哭泣道:“娘到明日好生看养着,与他爹做个根蒂儿,休要似奴粗心,吃人暗算了。”月娘道:“姐姐,我知道。”看官听说:只这一句话,就感触目娘的心来。后次西门庆死了,金莲就在家中住不牢者,就是想着李瓶儿临终这句话。正是:

\[
惟有感恩并积恨,千年万载不生尘。
\]

正说话间,只见琴童吩咐房中收拾焚下香,五岳观请了潘法官来了。月娘一面看着,教丫头收拾房中干净,伺候净茶净水,焚下百合真香。月娘与众妇女都藏在那边床屋里听观。不一时,只见西门庆领了那潘道士进来。怎生形相?但见:

\[
头戴云霞五岳冠,身穿皂布短褐袍,腰系杂色彩丝绦,背插横纹古铜剑。两只脚穿双耳麻鞋,手执五明降鬼扇。八字眉,两个杏子眼;四方口,一道落腮胡。威仪凛凛,相貌堂堂。若非霞外云游客,定是蓬莱玉府人。
\]
潘道士进入角门,刚转过影壁,将走到李瓶儿房穿廊台基下,那道士往后退讫两步,似有呵叱之状,尔语数四,方才左右揭帘进入房中,向病榻而至。运双晴,拿力以慧通神目一视,仗剑手内,掐指步罡,念念有辞,早知其意。走出明间,朝外设下香案。西门庆焚了香,这潘道士焚符,喝道:“值日神将,不来等甚?”噀了一口法水去,忽阶下卷起一阵狂风,仿佛似有神将现于面前一般。潘道士便道:“西门氏门中,有李氏阴人不安,投告于我案下。汝即与我拘当坊土地、本家六神查考,有何邪祟,即与我擒来,毋得迟滞!”良久,只见潘道士瞑目变神,端坐于位上,据案击令牌,恰似问事之状,良久乃止。出来,西门庆让至前边卷棚内,问其所以,潘道士便说:“此位娘子,惜乎为宿世冤愆诉于阴曹,非邪祟也,不可擒之。”西门庆道:“法官可解禳得么?”潘道士道:“冤家债主,须得本人,虽阴官亦不能强。”因见西门庆礼貌虔切,便问:“娘于年命若干?”西门庆道:“属羊的,二十七岁。”潘道士道:“也罢,等我与他祭祭本命星坛,看他命灯如何。”西门庆问:“几时祭?用何香纸祭物?”潘道士道:“就是今晚三更正子时,用白灰界画,建立灯坛,以黄绢围之,镇以生辰坛斗,祭以五谷枣汤,不用酒脯,只用本命灯二十七盏,上浮以华盖之仪,余无他物,官人可斋戒青衣,坛内俯伏行礼,贫道祭之,鸡犬皆关去,不可入来打搅。”西门庆听了,忙吩咐一一备办停当。就不敢进去,只在书房中沐浴斋戒,换了净衣。留应伯爵也不家去了,陪潘道士吃斋馔。

到三更天气,建立灯坛完备,潘道士高坐在上。下面就是灯坛,按青龙、白虎、朱雀、玄武,上建三台华盖;周列十二宫辰,下首才是本命灯,共合二十七盏。先宣念了投词。西门庆穿青衣俯伏阶下,左右尽皆屏去,不许一人在左右。灯烛荧煌,一齐点将起来。那潘道士在法座上披下发来,仗剑,口中念念有词。望天罡,取真气,布步玦,蹑瑶坛。正是:三信焚香三界合,一声令下一声雷。但见晴天月明星灿,忽然地黑天昏,起一阵怪风。正是:

\[
非干虎啸,岂是龙吟?仿佛入户穿帘,定是催花落叶。推云出岫,送雨归川。雁迷失伴作哀鸣,鸥鹭惊群寻树杪。姮娥急把蟾宫闭,列子空中叫救人。
\]
大风所过三次,忽一阵冷气来,把李瓶儿二十七盏本命灯尽皆刮灭。潘道士明明在法座上见一个白衣人领着两个青衣人,从外进来,手里持着一纸文书,呈在法案下。潘道士观看,却是地府勾批,上面有三颗印信,唬的慌忙下法座来,向前唤起西门庆来,如此这般,说道:“官人请起来罢!娘子已是获罪于天,无所祷也!本命灯已灭,岂可复救乎?只在旦夕之间而已。”那西门庆听了,低首无语,满眼落泪,哀告道:“万望法师搭救则个!”潘道士道:“定数难逃,不能搭救了。”就要告辞。西门庆再三款留:“等天明早行罢!”潘道士道:“出家人草行露宿,山栖庙止,自然之道。”西门庆不复强之。因令左右取出布一匹、白金三两作经衬钱。潘道士道:“贫道奉行皇天至道,对天盟誓,不敢贪受世财,取罪不便。”推让再四,只令小童收了布匹,作道袍穿,就作辞而行。嘱咐西门庆:“今晚,官人切忌不可往病人房里去,恐祸及汝身。慎之!慎之!”言毕,送出大门,拂袖而去。

西门庆归到卷棚内,看着收拾灯坛。见没救星,心中甚恸,向伯爵,不觉眼泪出。伯爵道:“此乃各人禀的寿数,到此地位,强求不得。哥也少要烦恼。”因打四更时分,说道:“哥,你也辛苦了,安歇安歇罢。我且家去,明日再来。”西门庆道:“教小厮拿灯笼送你去。”即令来安取了灯送伯爵出去,关上门进来。

那西门庆独自一个坐在书房内,掌着一枝蜡烛,心中哀恸,口里只长吁气,寻思道:“法官教我休往房里去,我怎生忍得!宁可我死了也罢。须厮守着和他说句话儿。”于是进入房中。见李瓶儿面朝里睡,听见西门庆进来,翻过身来便道:“我的哥哥,你怎的就不进来了?”因问:“那道士点得灯怎么说?”西门庆道:“你放心,灯上不妨事。”李瓶儿道:“我的哥哥,你还哄我哩,刚才那厮领着两个人又来,在我跟前闹了一回,说道:‘你请法师来遣我,我已告准在阴司,决不容你!’发恨而去,明日便来拿我也。”西门庆听了,两泪交流,放声大哭道:“我的姐姐,你把心来放正着,休要理他。我实指望和你相伴几日,谁知你又抛闪了我去了。宁教我西门庆口眼闭了,倒也没这等割肚牵肠。”那李瓶儿双手搂抱着西门庆脖子,呜呜咽咽悲哭,半日哭不出声。说道:“我的哥哥,奴承望和你白头相守,谁知奴今日死去也。趁奴不闭眼,我和你说几句话儿:你家事大,孤身无靠,又没帮手,凡事斟酌,休要一冲性儿。大娘等,你也少要亏了他。他身上不方便,早晚替你生下个根绊儿,庶不散了你家事。你又居着个官,今后也少要往那里去吃酒,早些儿来家,你家事要紧。比不的有奴在,还早晚劝你。奴若死了,谁肯苦口说你?”西门庆听了,如刀剜心肝相似,哭道:“我的姐姐,你所言我知道,你休挂虑我了。我西门庆那世里绝缘短幸,今世里与你做夫妻不到头。疼杀我也!天杀我也!”李瓶儿又吩咐迎春、绣春之事:“奴已和他大娘说来,到明日我死,把迎春伏侍他大娘;那小丫头,他二娘已承揽。——他房内无人,便教伏侍二娘罢。”西门庆道:“我的姐姐,你没的说,你死了,谁人敢分散你丫头!奶子也不打发他出去,都教他守你的灵。”李瓶儿道:“甚么灵!回个神主子,过五七烧了罢了。”西门庆道:“我的姐姐,你不要管他,有我西门庆在一日,供养你一日。”两个说话之间,李瓶儿催促道:“你睡去罢,这咱晚了。”西门庆道:“我不睡了,在这屋里守你守儿。”李瓶儿道:“我死还早哩,这屋里秽污,熏的你慌,他每伏侍我不方便。”

西门庆不得已,吩咐丫头:“仔细看守你娘。”往后边上房里,对月娘悉把祭灯不济之事告诉一遍:“刚才我到他房中,我观他说话儿还伶俐。天可怜,只怕还熬出来也不见得。”月娘道:“眼眶儿也塌了,嘴唇儿也干了,耳轮儿也焦了,还好甚么!也只在早晚间了。他这个病是恁伶俐,临断气还说话儿。”西门庆道:“他来了咱家这几年,大大小小,没曾惹了一个人,且是又好个性格儿,又不出语,你教我舍的他那些儿!”题起来又哭了。月娘亦止不住落泪。

不说西门庆与月娘说话,且说李瓶儿唤迎春、奶子:“你扶我面朝里略倒倒儿。”因问道:“有多咱时分了?”奶子道:“鸡还未叫,有四更天了。”叫迎春替他铺垫了身底下草纸,搊他朝里,盖被停当,睡了。众人都熬了一夜没曾睡,老冯与王姑子都已先睡了。迎春与绣春在面前地坪上搭着铺,刚睡倒没半个时辰,正在睡思昏沉之际,梦见李瓶儿下炕来,推了迎春一推,嘱咐:“你每看家,我去也。”忽然惊醒,见桌上灯尚未灭。忙向床上视之,还面朝里,摸了摸,口内已无气矣。不知多咱时分呜呼哀哉,断气身亡。可怜一个美色佳人,都化作一场春梦。正是:

\[
阎王教你三更死,怎敢留人到五更!
\]

迎春慌忙推醒众人,点灯来照,果然没了气儿,身底下流血一洼,慌了手脚,忙走去后边,报知西门庆。西门庆听见李瓶儿死了,和吴月娘两步做一步奔到前边,揭起被,但见面容不改,体尚微温,悠然而逝,身上止着一件红绫抹胸儿。西门庆也不顾甚么身底下血渍,两只手捧着他香腮亲着,口口声声只叫:“我的没救的姐姐,有仁义好性儿的姐姐!你怎的闪了我去了?宁可教我西门庆死了罢。我也不久活于世了,平白活着做甚么!”在房里离地跳的有三尺高,大放声号哭。吴月娘亦揾泪哭涕不止。落后,李娇儿、孟玉楼、潘金莲、孙雪娥、合家大小丫头养娘都哭起来,哀声动地。月娘向众人道:“不知多咱死的,恰好衣服儿也不曾穿一件在身上。”玉楼道:“我摸他身上还温温儿的,也才去了不多回儿。咱趁热脚儿不替他穿上衣裳,还等甚么?”月娘见西门庆磕伏在他身上,挝脸儿那等哭,只叫:“天杀了我西门庆了!姐姐你在我家三年光景,一日好日子没过,都是我坑陷了你了!”月娘听了,心中就有些不耐烦了,说道:“你看韶刀!哭两声儿,丢开手罢了。一个死人身上,也没个忌讳,就脸挝着脸儿哭,倘或口里恶气扑着你是的!他没过好日子,谁过好日子来?各人寿数到了,谁留的住他!那个不打这条路儿来?”因令李娇儿、孟玉楼:“你两个拿钥匙,那边屋里寻他几件衣服出来,咱每眼看着与他穿上。”又叫:“六姐,咱两个把这头来替他整理整理。”西门庆又向月娘说:“多寻出两套他心爱的好衣服,与他穿了去。”月娘吩咐李娇儿、玉楼:“你寻他新裁的大红缎遍地锦袄儿、柳黄遍地锦裙,并他今年乔亲家去那套丁香色云绸妆花衫、翠蓝宽拖子裙,并新做的白绫袄、黄绸子裙出来罢。”

当下迎春拿着灯,孟玉楼拿钥匙,走到那边屋里,开了箱子,寻了半日,寻出三套衣裳来,又寻出一件衬身紫绫小袄儿、一件白绸子裙、一件大红小衣儿并白绫女袜儿、妆花膝裤腿儿。李娇儿抱过这边屋里与月娘瞧。月娘正与金莲灯下替他整理头髻,用四根金簪儿绾一方大鸦青手帕,旋勒停当。李娇儿因问:“寻双甚么颜色鞋,与他穿了去?”潘金莲道:“姐姐,他心爱穿那双大红遍地金高底鞋儿,只穿了没多两遭儿,倒寻出来与他穿去罢。”吴月娘道:“不好,倒没的穿到阴司里,教他跳火坑。你把前日往他嫂子家去穿的那双紫罗遍地金高底鞋,与他装绑了去罢。”李娇儿听了,忙叫迎春寻出来。众人七手八脚,都装绑停当。

西门庆率领众小厮,在大厅上收卷书画,围上帏屏,把李瓶儿用板门抬出,停于正寝。下铺锦褥,上覆纸被,安放几筵香案,点起一盏随身灯来。专委两个小厮在旁侍奉:一个打磐,一个炷纸,一面使玳安:“快请阴阳徐先生来看时批书。”月娘打点出装绑衣服来,就把李瓶儿床房门锁了,只留炕屋里,交付与丫头养娘。冯妈妈见没了主儿,哭的三个鼻头两行眼泪,王姑子且口里喃喃呐呐,替李瓶儿念《密多心经》、《药师经》、《解冤经》、《楞严经》并《大悲中道神咒》,请引路王菩萨与他接引冥途。西门庆在前厅,手拍着胸膛,抚尸大恸,哭了又哭,把声都哭哑了。口口声声只叫:“我的好性儿有仁义的姐姐。”

比及乱着,鸡就叫了。玳安请了徐先生来,向西门庆施礼,说道:“老爹烦恼,奶奶没了在于甚时候?”西门庆道:“因此时候不真:睡下之时,已可四更,房中人都困倦睡熟了,不知多咱时候没了。”徐先生道:“不打紧。”因令左右掌起灯来,揭开纸被观看,手掐丑更,说道:“正当五更二点辙,还属丑时断气。”西门庆即令取笔砚,请徐先生批书。徐先生向灯下问了姓氏并生辰八字,批将下来:“一故锦衣西门夫人李氏之丧。生于元祐辛未正月十五日午时,卒于政和丁酉九月十六日丑时。今日丙子,月令戊戌,犯天地往亡,煞高一丈,本家忌哭声,成服后无妨。入殓之时,忌龙、虎、鸡、蛇四生人,亲人不避。”吴月娘使出玳安来:“叫徐先生看看黑书上,往那方去了。”徐先生一面打开阴阳秘书观看,说道:“今乃丙子日,已丑时,死者上应宝瓶宫,下临齐地。前生曾在滨州王家作男子,打死怀胎母羊,今世为女人,属羊。虽招贵夫,常有疾病,比肩不和,生子夭亡,主生气疾而死。前九日魂去,托生河南汴梁开封府袁家为女,艰难不能度日。后耽阁至二十岁嫁一富家,老少不对,终年享福,寿至四十二岁,得气而终。”看毕黑书,众妇女听了,皆各叹息。西门庆就叫徐先生看破土安葬日期。徐先生请问:“老爹,停放几时?”西门庆哭道:“热突突怎么就打发出去的,须放过五七才好。”徐先生道:“五七内没有安葬日期,倒是四七内,宜择十月初八日丁酉午时破土,十二日辛丑未时安葬,合家六位本命都不犯。”西门庆道:“也罢,到十月十二日发引,再没那移了。”徐先生写了殃榜,盖伏死者身上,向西门庆道:“十九日辰时大殓,一应之物,老爹这里备下。”

刚打发徐先生出了门,天已发晓。西门庆使琴童儿骑头口,往门外请花大舅,然后分班差人各亲眷处报丧。又使人往衙门中给假,又使玳安往狮子街取了二十桶瀼纱漂白、三十桶生眼布来,叫赵裁雇了许多裁缝,在西厢房先造帷幕、帐子、桌围,并入殓衣衾缠带、各房里女人衫裙,外边小厮伴当,每人都是白唐巾,一件白直裰。又兑了一百两银子,教贲四往门外店里买了三十桶魁光麻布、二百匹黄丝孝绢,一面又教搭彩匠,在天井内搭五间大棚。西门庆因思想李瓶儿动止行藏模样,忽然想起忘了与他传神,叫过来保来问:“那里有好画师?寻一个来传神。我就把这件事忘了。”来保道:“旧时与咱家画围屏的韩先儿,他原是宣和殿上的画士,革退来家,他传的好神。”西门庆道:“他在那里住?快与我请来。”来保应诺去了。

西门庆熬了一夜没睡的人,前后又乱了一五更,心中又着了悲恸,神思恍乱,只是没好气,骂丫头、踢小厮,守着李瓶儿尸首,由不的放声哭叫。那玳安在旁,亦哭的言不的语不的。吴月娘正和李娇儿、孟玉楼、潘金莲在帐子后,打伙儿分孝与各房里丫头并家人媳妇,看见西门庆哑着喉咙只顾哭,问他,茶也不吃,只顾没好气。月娘便道:“你看恁劳叨!死也死了,你没的哭的他活?只顾扯长绊儿哭起来了。三两夜没睡,头也没梳,脸也没洗,乱了恁五更,黄汤辣水还没尝着,就是铁人也禁不的。把头梳了,出来吃些甚么,还有个主张。好小身子,一时摔倒了,却怎样儿的!”玉楼道:“原来他还没梳头洗脸哩?”月娘道:“洗了脸倒好!我头里使小厮请他后边洗脸,他把小厮踢进来,谁再问他来!”金莲道:“你还没见,头里我倒好意说,他已死了,你恁般起来,把骨秃肉儿也没了。你在屋里吃些甚么儿,出去再乱也不迟。他倒把眼睁红了的,骂我:‘狗攮的淫妇,管你甚么事!’我如今整日不教狗攮,却教谁攮哩!——恁不合理的行货子。只说人和他合气。”月娘道:“热突突死了,怎么不疼?你就疼,也还放在心里,那里就这般显出来?人也死了,不管那有恶气没恶气,就口挝着口那等叫唤,不知甚么张致。他可可儿来三年没过一日好日子,镇日教他挑水挨磨来?”孟玉楼道:“李大姐倒也罢了,倒吃他爹恁三等九格的。”

正说着,只见陈敬济手里拿着九匹水光绢,说:“爹教娘每剪各房里手帕,剩下的与娘每做裙子。”月娘收了绢,便道:“姐夫,你去请你爹进来扒口子饭。这咱七八晌午,他茶水还没尝着哩。”敬济道:“我是不敢请他。头里小厮请他吃饭,差些没一脚踢杀了,我又惹他做甚么?”月娘道:“你不请他,等我另使人请他来吃饭。”良久,叫过玳安来说道:“你爹还没吃饭,哭这一日了。你拿上饭去,趁温先生在这里,陪他吃些儿。”玳安道:“请应二爹和谢爹去了。等他来时,娘这里使人拿饭上去,消不的他几句言语,管情爹就吃了。”吴月娘说道:“硶嘴的囚根子,你是你爹肚里蛔虫?俺每这几个老婆倒不如你了。你怎的知道他两个来才吃饭?”玳安道:“娘每不知,爹的好朋友,大小酒席儿,那遭少了他两个?爹三钱,他也是三钱;爹二星,他也是二星。爹随问怎的着了恼,只他到,略说两句话儿,爹就眉花眼笑的。”

说了一回,棋童儿请了应伯爵、谢希大二人来到。进门扑倒灵前地下,哭了半日,只哭“我那有仁义的嫂子”,被金莲和玉楼骂道:“贼油嘴的囚根子,俺每都是没仁义的?”二人哭毕,爬起来,西门庆与他回礼,两个又哭了,说道:“哥烦恼,烦恼。”一面让至厢房内,与温秀才叙礼坐下。先是伯爵问道:“嫂子是甚时候殁了?”西门庆道:“正丑时断气。”伯爵道:“我到家已是四更多了,房下问我,我说看阴骘,嫂子这病已在七八了。不想刚睡下就做了一梦,梦见哥使大官儿来请我,说家里吃庆官酒,教我急急来到。见哥穿着一身大红衣服,向袖中取出两根玉簪儿与我瞧,说一根折了。我瞧了半日,对哥说:‘可惜了,这折了是玉的,完全的倒是硝子石。’哥说两根都是玉的。我醒了,就知道此梦做的不好。房下见我只顾咂嘴,便问:‘你和谁说话?’我道:‘你不知,等我到天晓告诉你。’等到天明,只见大官儿到了,戴着白,教我只顾跌脚。果然哥有孝服。”西门庆道:“我昨夜也做了恁个梦,和你这个一样儿。梦见东京翟亲家那里寄送了六根簪儿,内有一根\textShiFou 折了。我说,可惜了。醒来正告诉房下,不想前边断了气。好不睁眼的天,撇的我真好苦!宁可教我西门庆死了,眼不见就罢了。到明日,一时半刻想起来,你教我怎不心疼!平时,我又没曾亏欠了人,天何今日夺吾所爱之甚也!——先是一个孩儿没了,今日他又长伸脚去了。我还活在世上做甚么?虽有钱过北斗,成何大用?”伯爵道:“哥,你这话就不是了。我这嫂子与你是那样夫妻,热突突死了,怎的不心疼?争奈你偌大家事,又居着前程,这一家大小,泰山也似靠着你。你若有好歹,怎么了得!就是这些嫂子,都没主儿。常言:一在三在,一亡三亡。哥,你聪明怜俐人,何消兄弟每说?就是嫂子他青春年少,你疼不过,越不过他的情,成了服,令僧道念几卷经,大发送,葬埋在坟里,哥的心也尽了,也是嫂子一场的事,再还要怎样的?哥,你且把心放开。”当时,被伯爵一席话,说的西门庆心地透彻,茅塞顿开,也不哭了。须臾,拿上茶来吃了,便唤玳安:“后边说去,看饭来,我和你应二爹、温师父、谢爹吃。”伯爵道:“哥原来还未吃饭哩?”西门庆道:“自你去了,乱了一夜,到如今谁尝甚么儿来。”伯爵道:“哥,你还不吃饭,这个就胡突了,常言道:‘宁可折本,休要饥损。’《孝经》上不说的:‘教民无以死伤生,毁不灭性。’死的自死了,存者还要过日子。哥要做个张主。”正是:

\[
数语拨开君子路,片言题醒梦中人。
\]

