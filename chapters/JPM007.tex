%# -*- coding:utf-8 -*-
%%%%%%%%%%%%%%%%%%%%%%%%%%%%%%%%%%%%%%%%%%%%%%%%%%%%%%%%%%%%%%%%%%%%%%%%%%%%%%%%%%%%%


\chapter{薛媒婆说娶孟三儿\KG 杨姑娘气骂张四舅}


诗曰:

\[
我做媒人实自能,全凭两腿走殷勤。
唇枪惯把鳏男配,舌剑能调烈女心。
利市花常头上带,喜筵饼锭袖中撑。
只有一件不堪处,半是成人半败人。
\]

话说西门庆家中一个卖翠花的薛嫂儿,提着花厢儿,一地里寻西门庆不着。因见西门庆贴身使的小厮玳安儿,便问道:“大官人在那里?”玳安道:“俺爹在铺子里和傅二叔算帐。”原来西门庆家开生药铺,主管姓傅名铭,字自新,排行第二,因此呼他做傅二叔。这薛嫂听了,一直走到铺子门首,掀开帘子,见西门庆正与主管算帐,便点点头儿,唤他出来。西门庆见是薛嫂儿,连忙撇了主管出来,两人走在僻静处说话。西门庆问道:“有甚话说?”薛嫂道:“我有一件亲事,来对大官人说,管情中你老人家意,就顶死了的三娘的窝儿,何如?”西门庆道:“你且说这件亲事是那家的?”薛嫂道:“这位娘子,说起来你老人家也知道,就是南门外贩布杨家的正头娘子。手里有一分好钱。南京拔步床也有两张。四季衣服,插不下手去,也有四五只箱子。金镯银钏不消说,手里现银子也有上千两,好三梭布也有三二百筒。不料他男子汉去贩布,死在外边。他守寡了一年多,身边又没子女,止有一个小叔儿,才十岁。青春年少,守他什么!有他家一个嫡亲姑娘,要主张着他嫁人。这娘子今年不上二十五六岁,生的长挑身材,一表人物,打扮起来就是个灯人儿。风流俊俏,百伶百俐,当家立纪、针指女工、双陆棋子不消说。不瞒大官人说,他娘家姓孟,排行三姐,就住在臭水巷。又会弹一手好月琴,大官人若见了,管情一箭就上垛。”西门庆听见妇人会弹月琴,便可在他心上,就问薛嫂儿:“既是这等,几时相会看去?”薛嫂道:“相看到不打紧。我且和你老人家计议:如今他家一家子,只是姑娘大。虽是他娘舅张四,山核桃——差着一槅哩。这婆子原嫁与北边半边街徐公公房子里住的孙歪头。歪头死了,这婆子守寡了三四十年,男花女花都无,只靠侄男侄女养活。大官人只倒在他身上求他。这婆子爱的是钱财,明知侄儿媳妇有东西,随问什么人家他也不管,只指望要几两银子。大官人家里有的是那嚣段子,拿一段,买上一担礼物,明日亲去见他,再许他几两银子,一拳打倒他。随问旁边有人说话,这婆子一力张主,谁敢怎的!”这薛嫂儿一席话,说的西门庆欢从额角眉尖出,喜向腮边笑脸生。正是:

\[
媒妁殷勤说始终,孟姬爱嫁富家翁。
有缘千里能相会,无缘对面不相逢。
\]

西门庆当日与薛嫂相约下了,明日是好日期,就买礼往他姑娘家去。薛嫂说毕话,提着花厢儿去了。西门庆进来和傅伙计算帐。一宿晚景不题。

到次日,西门庆早起,打选衣帽整齐,拿了一段尺头,买了四盘羹果,装做一盒担,叫人抬了。薛嫂领着,西门庆骑着头口,小厮跟随,迳来杨姑娘家门首。薛嫂先入去通报姑娘,说道:“近边一个财主,要和大娘子说亲。我说一家只姑奶奶是大,先来觌面,亲见过你老人家,讲了话,然后才敢去门外相看。今日小媳妇领来,见在门首伺候。”婆子听见,便道:“阿呀,保山,你如何不先来说声!”一面吩咐丫鬟顿下好茶,一面道:“有请。”这薛嫂一力撺掇,先把盒担抬进去摆下,打发空盒担出去,就请西门庆进来相见。这西门庆头戴缠综大帽,一口一声只叫:“姑娘请受礼。”让了半日,婆子受了半礼。分宾主坐下,薛嫂在旁边打横。婆子便道:“大官人贵姓?”薛嫂道:“便是咱清河县数一数二的财主,西门大官人。在县前开个大生药铺,家中钱过北斗,米烂陈仓,没个当家立纪的娘子。闻得咱家门外大娘子要嫁,特来见姑奶奶讲说亲事。”婆子道:“官人傥然要说俺侄儿媳妇,自恁来闲讲罢了,何必费烦又买礼来,使老身却之不恭,受之有愧。”西门庆道:“姑娘在上,没的礼物,惶恐。”那婆子一面拜了两拜谢了,收过礼物去,拿茶上来。吃毕,婆子开口道:“老身当言不言谓之懦。我侄儿在时,挣了一分钱财,不幸先死了,如今都落在他手里,说少也有上千两银子东西。官人做小做大我不管你,只要与我侄儿念上个好经。老身便是他亲姑娘,又不隔从,就与上我一个棺材本,也不曾要了你家的。我破着老脸,和张四那老狗做臭毛鼠,替你两个硬张主。娶过门时,遇生辰时节,官人放他来走走,就认俺这门穷亲戚,也不过上你穷。”西门庆笑道:“你老人家放心,所说的话,我小人都知道了。只要你老人家主张得定,休说一个棺材本,就是十个,小人也来得起。”说着,便叫小厮拿过拜匣来,取出六锭三十两雪花官银,放在面前,说道:“这个不当甚么,先与你老人家买盏茶吃,到明日娶过门时,还你七十两银子、两匹缎子,与你老人家为送终之资。其四时八节,只管上门行走。”这老虔婆黑眼珠见了二三十两白晃晃的官银,满面堆下笑来,说道:“官人在上,不是老身意小,自古先断后不乱。”薛嫂在旁插口说:“你老人家忒多心,那里这等计较!我这大官人不是这等人,只恁还要掇着盒儿认亲。你老人家不知,如今知县知府相公也都来往,好不四海。你老人家能吃他多少?”一席话说的婆子屁滚尿流。吃了两道茶,西门庆便要起身,婆子挽留不住。薛嫂道:“今日既见了姑奶奶,明日便好往门外相看。”婆子道:“我家侄儿媳妇不用大官人相,保山,你就说我说,不嫁这样人家,再嫁甚样人家!”西门庆作辞起身。婆子道:“老身不知大官人下降,匆忙不曾预备,空了官人,休怪。”拄拐送出。送了两步,西门庆让回去了。薛嫂打发西门庆上马,因说道:“我主张的有理么?你老人家先回去罢,我还在这里和他说句话。明日须早些往门外去。”西门庆便拿出一两银子来,与薛嫂做驴子钱。薛嫂接了,西门庆便上马来家。他还在杨姑娘家说话饮酒,到日暮才归家去。

话休饶舌。到次日,西门庆打选衣帽齐整,袖着插戴,骑着匹白马,玳安、平安两个小厮跟随,薛嫂儿骑着驴子,出的南门外来。不多时,到了杨家门首。却是坐南朝北一间门楼,粉青照壁。薛嫂请西门庆下了马,同进去。里面仪门照墙,竹抢篱影壁,院内摆设榴树盆景,台基上靛缸一溜,打布凳两条。薛嫂推开朱红槅扇,三间倒坐客位,上下椅桌光鲜,帘栊潇洒。薛嫂请西门庆坐了,一面走入里边。片晌出来,向西门庆耳边说:“大娘子梳妆未了,你老人家请坐一坐。”只见一个小厮儿拿出一盏福仁泡茶来,西门庆吃了。这薛嫂一面指手画脚与西门庆说:“这家中除了那头姑娘,只这位娘子是大。虽有他小叔,还小哩,不晓得什么。当初有过世的官人在铺子里,一日不算银子,铜钱也卖两大箥箩。毛青鞋面布,俺每问他买,定要三分一尺。一日常有二三十染的吃饭,都是这位娘子主张整理。手下使着两个丫头,一个小厮。大丫头十五岁,吊起头去了,名唤兰香。小丫头名唤小鸾,才十二岁。到明日过门时,都跟他来。我替你老人家说成这亲事,指望典两间房儿住哩。”西门庆道:“这不打紧。”薛嫂道:“你老人家去年买春梅,许我几匹大布,还没与我。到明日不管一总谢罢了。”

正说着,只见使了个丫头来叫薛嫂。不多时,只闻环佩叮咚,兰麝馥郁,薛嫂忙掀开帘子,妇人出来。西门庆睁眼观那妇人,但见:

\[
月画烟描,粉妆玉琢。俊庞儿不肥不瘦,俏身材难减难增。素额逗几点微麻,天然美丽;缃裙露一双小脚,周正堪怜。行过处花香细生,坐下时淹然百媚。
\]
西门庆一见满心欢喜。妇人走到堂下,望上不端不正道了个万福,就在对面椅子上坐下。西门庆眼不转睛看了一回,妇人把头低了。西门庆开言说:“小人妻亡已久,欲娶娘子管理家事,未知尊意如何?”那妇人偷眼看西门庆,见他人物风流,心下已十分中意,遂转过脸来,问薛婆道:“官人贵庚?没了娘子多少时了?”西门庆道:“小人虚度二十八岁,不幸先妻没了一年有余。不敢请问,娘子青春多少?”妇人道:“奴家是三十岁。”西门庆道:“原来长我二岁。”薛嫂在旁插口道:“妻大两,黄金日日长。妻大三,黄金积如山。”说着,只见小丫鬟拿出三盏蜜饯金橙子泡茶来。妇人起身,先取头一盏,用纤手抹去盏边水渍,递与西门庆,道个万福。薛嫂见妇人立起身,就趁空儿轻轻用手掀起妇人裙子来,正露出一对刚三寸、恰半叉、尖尖趫趫金莲脚来,穿着双大红遍地金云头白绫高低鞋儿。西门庆看了,满心欢喜。妇人取第二盏茶来递与薛嫂。他自取一盏陪坐。吃了茶,西门庆便叫玳安用方盒呈上锦帕二方、宝钗一对、金戒指六个,放在托盘内送过去。薛嫂一面叫妇人拜谢了。因问官人行礼日期:“奴这里好做预备。”西门庆道:“既蒙娘子见允,今月二十四日,有些微礼过门来。六月初二准娶。”妇人道:“既然如此,奴明日就使人对姑娘说去。”薛嫂道:“大官人昨日已到姑奶奶府上讲过话了。”妇人道:“姑娘说甚来?”薛嫂道:“姑奶奶听见大官人说此椿事,好不喜欢!说道,不嫁这等人家,再嫁那样人家!我就做硬主媒,保这门亲事。”妇人道:“既是姑娘恁般说,又好了。”薛嫂道:“好大娘子,莫不俺做媒敢这等捣谎。”说毕,西门庆作辞起身。

薛嫂送出巷口,向西门庆说道:“看了这娘子,你老人家心下如何?”西门庆道:“薛嫂,其实累了你。”薛嫂道:“你老人家先行一步,我和大娘子说句话就来。”西门庆骑马进城去了。薛嫂转来向妇人说道:“娘子,你嫁得这位官人也罢了。”妇人道:“但不知房里有人没有人?见作何生理?”薛嫂道:“好奶奶,就有房里人,那个是成头脑的?我说是谎,你过去就看出来。他老人家名目,谁不知道,清河县数一数二的财主,有名卖生药放官吏债西门庆大官人。知县知府都和他来往。近日又与东京杨提督结亲,都是四门亲家,谁人敢惹他!”妇人安排酒饭,与薛嫂儿正吃着,只见他姑娘家使个小厮安童,盒子里盛着四块黄米面枣儿糕、两块糖、几十个艾窝窝,就来问:“曾受了那人家插定不曾?奶奶说来:这人家不嫁,待嫁甚人家。”妇人道:“多谢你奶奶挂心。今已留下插定了。”薛嫂道:“天么,天么!早是俺媒人不说谎,姑奶奶早说将来了。”妇人收了糕,取出盒子,装了满满一盒子点心腊肉,又与了安童五六十文钱,说:“到家多拜上奶奶。那家日子定在二十四日行礼,出月初二日准娶。”小厮去了。薛嫂道:“姑奶奶家送来什么?与我些,包了家去孩子吃。”妇人与了他一块糖、十个艾窝窝,方才出门,不在话下。

且说他母舅张四,倚着他小外甥杨宗保,要图留妇人东西,一心举保大街坊尚推官儿子尚举人为继室。若小可人家,还有话说,不想闻得是西门庆定了,知他是把持官府的人,遂动不得了。寻思千方百计,不如破为上计。即走来对妇人说:“娘子不该接西门庆插定,还依我嫁尚举人的是。他是诗礼人家,又有庄田地土,颇过得日子,强如嫁西门庆。那厮积年把持官府,刁徒泼皮。他家见有正头娘子,乃是吴千户家女儿,你过去做大是,做小是?况他房里又有三四个老婆,除没上头的丫头不算。你到他家,人多口多,还有的惹气哩!”妇人听见话头,明知张四是破亲之意,便佯说道:“自古船多不碍路。若他家有大娘子,我情愿让他做姐姐。虽然房里人多,只要丈夫作主,若是丈夫喜欢,多亦何妨。丈夫若不喜欢,便只奴一个也难过日子。况且富贵人家,那家没有四五个?你老人家不消多虑,奴过去自有道理,料不妨事。”张四道:“不独这一件。他最惯打妇煞妻,又管挑贩人口,稍不中意,就令媒婆卖了。你受得他这气么?”妇人道:“四舅,你老人家差矣。男子汉虽利害,不打那勤谨省事之妻。我到他家,把得家定,里言不出,外言不入,他敢怎的奴?”张四道:“不是我打听的,他家还有一个十四岁未出嫁的闺女,诚恐去到他家,三窝两块惹气怎了?”妇人道:“四舅说那里话,奴到他家,大是大,小是小,待得孩儿们好,不怕男子汉不欢喜,不怕女儿们不孝顺。休说一个,便是十个也不妨事。”张四道:“还有一件最要紧的事,此人行止欠端,专一在外眠花卧柳。又里虚外实,少人家债负。只怕坑陷了你。”妇人道:“四舅,你老人家又差矣。他少年人,就外边做些风流勾当,也是常事。奴妇人家,那里管得许多?惹说虚实,常言道:世上钱财傥来物,那是长贫久富家?况姻缘事皆前生分定,你老人家到不消这样费心。”张四见说不动妇人,到吃他抢白了几句,好无颜色,吃了两盏清茶,起身去了。有诗为证:

\[
张四无端散楚言,姻缘谁想是前缘。
佳人心爱西门庆,说破咽喉总是闲。
\]
张四羞惭归家,与婆子商议,单等妇人起身,指着外甥杨宗保,要拦夺妇人箱笼。

话休饶舌。到二十四日,西门庆行了礼。到二十六日,请十二位素僧念经烧灵,都是他姑娘一力张主。张四到妇人将起身头一日,请了几位街坊众邻,来和妇人说话。此时薛嫂正引着西门庆家小厮伴当,并守备府里讨的一二十名军牢,正进来搬抬妇人床帐、嫁妆箱笼。被张四拦住说道:“保山且休抬!有话讲。”一面同了街坊邻舍进来见妇人。坐下,张四先开言说:“列位高邻听着:大娘子在这里,不该我张龙说,你家男子汉杨宗锡与你这小叔杨宗保,都是我甥。今日不幸大外甥死了,空挣一场钱。有人主张着你,这也罢了。争奈第二个外甥杨宗保年幼,一个业障都在我身上。他是你男子汉一母同胞所生,莫不家当没他的份儿?今日对着列位高邻在这里,只把你箱笼打开,眼同众人看一看,有东西没东西,大家见个明白。”妇人听言,一面哭起来,说道:“众位听着,你老人家差矣!奴不是歹意谋死了男子汉,今日添羞脸又嫁人。他手里有钱没钱,人所共知,就是积攒了几两银子,都使在这房子上。房子我没带去,都留与小叔。家活等件,分毫不动。就是外边有三四百两银子欠帐,文书合同已都交与你老人家,陆续讨来家中盘缠。再有甚么银两来?”张四道:“你没银两也罢。如今只对着众位打开箱笼看一看。就有,你还拿了去,我又不要你的。”妇人道:“莫不奴的鞋脚也要瞧不成?”正乱着,只姑娘拄拐自后而出。众人便道:“姑娘出来。”都齐声唱喏。姑娘还了万福,陪众人坐下。姑娘开口道:“列位高邻在上,我是他是亲姑娘,又不隔从,莫不没我说处?死了的也是侄儿,活着的也是侄儿,十个指头咬着都疼。如今休说他男子汉手里没钱,他就有十万两银子,你只好看他一眼罢了。他身边又无出,少女嫩妇的,你拦着不教他嫁人做什么?”众街邻高声道:“姑娘见得有理!”婆子道:“难道他娘家陪的东西,也留下他的不成?他背地又不曾自与我什么,说我护他,也要公道。不瞒列位说,我这侄儿媳妇平日有仁义,老身舍不得他,好温克性儿。不然,老身管着他。”那张四在旁,把婆子瞅了一眼,说道:“你好公平心儿!凤凰无宝处不落。”只这一句话道着婆子真病,登时怒起,紫涨了面皮,指定张四大骂道:“张四,你休胡言乱语!我虽不能是杨家正头香主,你这老油嘴,是杨家那膫子\textuni{34B2}的?”张四道:“我虽是异姓,两个外甥是我姐姐养的,你这老咬虫,女生外向,怎一头放火,又一头放水?”姑娘道:“贱没廉耻老狗骨头!他少女嫩妇的,你留他在屋里,有何算计?既不是图色欲,便欲起谋心,将钱肥己。”张四道:“我不是图钱,只恐杨宗保后来大了,过不得日子。不似你这老杀才,搬着大引着小,黄猫儿黑尾。”姑娘道:“张四,你这老花根,老奴才,老粉嘴,你恁骗口张舌的好淡扯,到明日死了时,不使了绳子扛子。”张四道:“你这嚼舌头老淫妇,挣将钱来焦尾靶,怪不得你无儿无女。”姑娘急了,骂道:“张四,贼老苍根,老猪狗,我无儿无女,强似你家妈妈子穿寺院,养和尚,\textuni{34B2}道士,你还在睡梦里。”当下两个差些儿不曾打起来,多亏众邻舍劝住,说道:“老舅,你让姑娘一句儿罢。”薛嫂儿见他二人嚷做一团,领西门庆家小厮伴当,并发来众军牢,赶人闹里,七手八脚将妇人床帐、妆奁、箱笼,扛的扛,抬的抬,一阵风都搬去了。那张四气的眼大睁着,半晌说不出话来。众邻舍见不是事,安抚了一回,各人都散了。

到六月初二日,西门庆一顶大轿,四对红纱灯笼,他小叔杨宗保头上扎着髻儿,穿着青纱衣,撒骑在马上,送他嫂子成亲。西门庆答贺了他一匹锦缎、一柄玉绦儿。兰香、小鸾两个丫头,都跟了来铺床叠被。小厮琴童方年十五岁,亦带过来伏侍。到三日,杨姑娘家并妇人两个嫂子孟大嫂、二嫂都来做生日。西门庆与他杨姑娘七十两银子、两匹尺头。自此亲戚来往不绝。西门庆就把西厢房里收拾三间,与他做房。排行第三,号玉楼,令家中大小都随着叫三姨。到晚一连在他房中歇了三夜。正是:销金帐里,依然两个新人;红锦被中,现出两般旧物。有诗为证:

\[
怎睹多情风月标,教人无福也难消。
风吹列子归何处,夜夜婵娟在柳梢。
\]
