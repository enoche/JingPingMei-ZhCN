%# -*- coding:utf-8 -*-
%%%%%%%%%%%%%%%%%%%%%%%%%%%%%%%%%%%%%%%%%%%%%%%%%%%%%%%%%%%%%%%%%%%%%%%%%%%%%%%%%%%%%


\chapter{李瓶姐墙头密约\KG 迎春儿隙底私窥}


词曰:

\[
绣面芙蓉一笑开,斜飞宝鸭衬香腮。眼波才动被人猜。一面风情深有韵,半笺娇恨寄幽怀。月移花影约重来。
\]

话说一日西门庆往前边走来,到月娘房中。月娘告说:“今日花家使小厮拿帖来,请你吃酒。”西门庆观看帖子,写着:“即午院中吴银家一叙,希即过我同往,万万!”少顷,打选衣帽,叫了两个跟随,骑匹骏马,先迳到花家。不想花子虚不在家了。他浑家李瓶儿,夏月间戴着银丝鬒髻,金镶紫瑛坠子,藕丝对衿衫,白纱挑线镶边裙,裙边露一对红鸳凤嘴尖尖翘翘小脚,立在二门里台基上。那西门庆三不知走进门,两下撞了个满怀。这西门庆留心已久,虽故庄上见了一面,不曾细玩。今日对面见了,见他生的甚是白净,五短身才,瓜子面儿,细湾湾两道眉儿,不觉魂飞天外,忙向前深深作揖。妇人还了万福,转身入后边去了。使出一个头发齐眉的丫鬟来,名唤绣春,请西门庆客位内坐。他便立在角门首,半露娇容说:“大官人少坐一时。他适才有些小事出去了,便来也。”丫鬟拿出一盏茶来,西门庆吃了。妇人隔门说道:“今日他请大官人往那边吃酒去,好歹看奴之面,劝他早些回家。两个小厮又都跟去了,止是这两个丫鬟和奴,家中无人。”西门庆便道:“嫂子见得有理,哥家事要紧。嫂子既然分付在下,在下一定伴哥同去同来。”

正说着,只见花子虚来家,妇人便回房去了。花子虚见西门庆叙礼说道:“蒙哥下降,小弟适有些不得已小事出去,失迎,恕罪!”于是分宾主坐下,便叫小厮看茶。须臾,茶罢。又分付小厮:“对你娘说,看菜儿来,我和西门爹吃三杯起身。今日六月二十四,是院内吴银姐生日,请哥同往一乐。”西门庆道:“二哥何不早说?”即令玳安:“快家去,讨五钱银子封了来。”花子虚道:“哥何故又费心?小弟到不是了。”西门庆见左右放桌儿,说道:“不消坐了,咱往里边吃去罢。”花子虚道:“不敢久留,哥略坐一回。”少倾,就是齐整肴馔拿将上来,银高脚葵花锺,每人三锺,又是四个卷饼,吃毕收下来与马上人吃。

少倾,玳安取了分资来,一同起身上马,迳往吴四妈家与吴银儿做生日。到那里,花攒锦簇,歌舞吹弹,饮酒至一更时分方散。西门庆留心,把子虚灌得酩酊大醉。又因李瓶儿央浼之言,相伴他一同来家。小厮叫开大门,扶到他客位坐下。李瓶儿同丫鬟掌着灯烛出来,把子虚搀扶进去。

西门庆交付明白,就要告回。妇人旋走出来,拜谢西门庆,说道:“拙夫不才贪酒,多累看奴薄面,姑待来家,官人休要笑话。”那西门庆忙屈身还喏,说道:“不敢。嫂子这里分付,在下敢不铭心刻骨,同哥一搭里来家!非独嫂子耽心,显的在下干事不的了。方才哥在他家,被那些人缠住了,我强着催哥起身。走到乐星堂儿门首粉头郑爱香儿家,——小名叫做郑观音,生的一表人物,哥就要往他家去,被我再三拦住,劝他说道:‘恐怕家中嫂子放心不下。’方才一直来家。若到郑家,便有一夜不来。嫂子在上,不该我说,哥也糊涂,嫂子又青年,偌大家室,如何就丢了,成夜不在家?是何道理!”妇人道:“正是如此,奴为他这等在外胡行,不听人说,奴也气了一身病痛在这里。往后大官人但遇他在院中,好歹看奴薄面,劝他早早回家。奴恩有重报,不敢有忘。”这西门庆是头上打一下脚底板响的人,积年风月中走,甚么事儿不知道?今日妇人到明明开了一条大路,教他入港,岂不省腔!于是满面堆笑道:“嫂子说那里话!相交朋友做甚么?我一定苦心谏哥,嫂子放心。”妇人又道了万福,又叫小丫鬟拿了一盏果仁泡茶来。西门庆吃毕茶,说道:“我回去罢,嫂子仔细门户。”遂告辞归家。

自此西门庆就安心设计,图谋这妇人,屡屡安下应伯爵、谢希大这伙人,把子虚挂住在院里饮酒过夜。他便脱身来家,一径在门首站立。这妇人亦常领着两个丫鬟在门首。西门庆看见了,便扬声咳嗽,一回走过东来,又往西去,或在对门站立,把眼不住望门里睃盼。妇人影身在门里,见他来便闪进里面,见他过去了,又探头去瞧。两个眼意心期,已在不言之表。一日,西门庆正站在门首,忽见小丫鬟绣春来请。西门庆故意问道:“姐姐请我做甚么?你爹在家里不在?”绣春道:“俺爹不在家,娘请西门庆爹问句话儿。”这西门庆得不的一声,连忙走过来,到客位内坐下。良久,妇人出来,道了万福,便道:“前日多承官人厚意,奴铭刻于心,知感不尽。他从昨日出去,一连两日不来家了,不知官人曾会见他来不曾?”西门庆道:“他昨日同三四个在郑家吃酒,我偶然有些小事就来了。今日我不曾得进去,不知他还在那里没在。若是我在那里,恐怕嫂子忧心,有个不催促哥早早来家的?”妇人道:“正是这般说。奴吃煞他不听人说、在外边眠花卧柳不顾家事的亏。”西门庆道:“论起哥来,仁义上也好,只是有这一件儿。”说着,小丫鬟拿茶来吃了。西门庆恐子虚来家,不敢久恋,就要告归。妇人又千叮万嘱,央西门庆:“不拘到那里,好歹劝他早来家,奴一定恩有重报,决不敢忘官人!”西门庆道:“嫂子没的说,我与哥是那样相交!”说毕,西门庆家去了。

到次日,花子虚自院中回家,妇人再三埋怨说道:“你在外边贪酒恋色,多亏隔壁西门大官人,两次三番顾睦你来家。你买分礼儿谢谢他,方不失了人情。”那花子虚连忙买了四盒礼物,一坛酒,使小厮天福儿送到西门庆家。西门庆收下,厚赏来人去了。吴月娘便问说:“花家如何送你这礼?”西门庆道:“花二哥前日请我们在院中与吴银儿做生日,醉了,被我搀扶了他来家;又见常时院中劝他休过夜,早早来家。他娘子儿因此感我的情,想对花二哥说,故买此礼来谢我。”吴月娘听了,与他打个问讯,说道:“我的哥哥,你自顾了你罢,又泥佛劝土佛!你也成日不着个家,在外养女调妇,反劝人家汉子!”又道:“你莫不白受他这礼?”因问:“他帖上儿写着谁的名字?若是他娘子的名字,今日写我的帖儿,请他娘子过来坐坐,他也只恁要来咱家走走哩。若是他男子汉名字,随你请不请,我不管你。”西门庆道:“是花二哥名字,我明日请他便了。”次日,西门庆果然治酒,请过花子虚来,吃了一日酒。归家,李瓶儿说:“你不要差了礼数。咱送了他一分礼,他到请你过去吃了一席酒,你改日还该治一席酒请他,只当回席。”

光阴迅速,又早九月重阳。花子虚假着节下,叫了两个妓者,具柬请西门庆过来赏菊。又邀应伯爵、谢希大、祝实念、孙天化四人相陪。传花击鼓,欢乐饮酒。有诗为证:

\[
乌兔循环似箭忙,人间佳节又重阳。
千枝红树妆秋色,三径黄花吐异香。
不见登高乌帽客,还思捧酒绮罗娘。
秀帘琐闼私相觑,从此恩情两不忘。
\]

当日,众人饮酒到掌灯之后,西门庆忽下席来外边解手。不防李瓶儿正在遮槅子边站立偷觑,两个撞了个满怀,西门庆回避不及。妇人走到西角门首,暗暗使绣春黑影里走到西门庆跟前,低声说道:“俺娘使我对西门爹说,少吃酒,早早回家。晚夕,娘如此这般要和西门爹说话哩。”西门庆听了,欢喜不尽。小解回来,到席上连酒也不吃,唱的左右弹唱递酒,只是装醉不吃。看看到一更时分,那李瓶儿不住走来廉外,见西门庆坐在上面,只推做打盹。那应伯爵、谢希大,如同钉在椅子上,白不起身。熬的祝实念、孙寡嘴也去了,他两个还不动。把个李瓶儿急的要不的。西门庆已是走出来,被花子虚再不放,说道:“今日小弟没敬心,哥怎的白不肯坐?”西门庆道:“我本醉了,吃不去。”于是故意东倒西歪,教两个扶归家去了。应伯爵道:“他今日不知怎的,白不肯吃酒,吃了不多酒就醉了。既是东家费心,难为两个姐儿在此,拿大锺来,咱每再周四五十轮,散了罢。”李瓶儿在帘外听见,骂“涎脸的囚根子”不绝。暗暗使小厮天喜儿请下花子虚来,分付说:“你既要与这伙人吃,趁早与我院里吃去。休要在家里聒噪。我半夜三更,熬油费火,我那里耐烦!”花子虚道:“这咱晚我就和他们院里去,也是来家不成,你休再麻犯我。”妇人道:“你去,我不麻犯便了。”这花子虚得不的这一声,走来对众人说:“我们往院里去。”应伯爵道:“真个?休哄我。你去问声嫂子来,咱好起身。”子虚道:“房下刚才已是说了,教我明日来家。”谢希大道:“可是来,自吃应花子这等唠叨。哥刚才已是讨了老脚来,咱去的也放心。”于是连两个唱的,都一齐起身进院。此时已是二更天气,天福儿、天喜儿跟花子虚等三人,从新又到后巷吴银儿家去吃酒不题。

单表西门庆推醉到家,走到金莲房里,刚脱了衣裳,就往前边花园里去坐,单等李瓶儿那边请他。良久,只听得那边赶狗关门。少倾,只见丫鬟迎春黑影影里扒着墙,推叫猫,看见西门庆坐在亭子上,递了话。这西门庆就掇过一张桌凳来踏着,暗暗扒过墙来,这边已安下梯子。李瓶儿打发子虚去了,已是摘了冠儿,乱挽乌云,素体浓妆,立在穿廊下。看见西门庆过来,欢喜无尽,忙迎接进房中。灯烛下,早已安排一桌齐整酒肴果菜,壶内满贮香醪。妇人双手高擎玉斝,亲递与西门庆,深深道个万福:“奴一向感谢官人,蒙官人又费心酬答,使奴家心下不安。今日奴自治了这杯淡酒,请官人过来,聊尽奴一点薄情。又撞着两个天杀的涎脸,只顾坐住了,急的奴要不的。刚才吃我都打发到院里去了。”西门庆道:“只怕二哥还来家么?”妇人道:“奴已分付过夜不来了。两个小厮都跟去了。家里再无一人,只是这两个丫头,一个冯妈妈看门首,他是奴从小儿养娘心腹人。前后门都已关闭了。”西门庆听了,心中甚喜。两个于是并肩叠股,交杯换盏,饮酒做一处。迎春旁边斟酒,绣春往来拿菜儿。吃得酒浓时,锦帐中香熏鸳被,设放珊瑚,两个丫鬟撤开酒桌,拽上门去了。两人上床交欢。

原来大人家有两层窗寮,外面为窗,里面为寮。妇人打发丫鬟出去,关上里面两扇窗寮,房中掌着灯烛,外边通看不见。这迎春丫头,今年已十七岁,颇知事体,见他两个今夜偷期,悄悄向窗下,用头上簪子挺签破窗寮上纸,往里窥觑。端的二人怎样交接?但见:

\[
灯光影里,鲛绡帐中,一个玉臂忙摇,一个金莲高举。一个莺声呖呖,一个燕语喃喃。好似君瑞遇莺娘,犹若宋玉偷神女。山盟海誓,依稀耳中;蝶恋蜂恣,未能即罢。正是:被翻红浪,灵犀一点透酥胸;帐挽银钩,眉黛两弯垂玉脸。
\]

房中二人云雨,不料迎春在窗外,听看得明明白白。听见西门庆问妇人多少青春。李瓶儿道:“奴今年二十三岁。”因问:“他大娘贵庚?”西门庆道:“房下二十六岁了。”妇人道:“原来长奴三岁,到明日买分礼儿过去,看看大娘,只怕不好亲近。”西门庆道:“房下自来好性儿。”妇人又问:“你头里过这边来,他大娘知道不知?倘或问你时,你怎生回答?”西门庆道:“俺房下都在后边第四层房子里,惟有我第五个小妾潘氏,在这前边花园内,独自一所楼房居住,他不敢管我。”妇人道:“他五娘贵庚多少?”西门庆道:“他与大房下同年。”妇人道:“又好了,若不嫌奴有玷,奴就拜他五娘做个姐姐罢。到明日,讨他大娘和五娘的脚样儿来,奴亲自做两双鞋儿过去,以表奴情。”说着,又将头上关顶的金簪儿拨下两根来,替西门庆带在头上,说道:“若在院里,休要叫花子虚看见。”西门庆道:“这理会得。”当下二人如胶似漆,盘桓到五更时分。窗外鸡叫,东方渐白,西门庆恐怕子虚来家,整衣而起,照前越墙而过。两个约定暗号儿,但子虚不在家,这边就使丫鬟在墙头上暗暗以咳嗽为号,或先丢块瓦儿,见这边无人,方才上墙,这边西门庆便用梯凳扒过墙来。两个隔墙酬和,窃玉偷香,不由大门行走,街房邻舍怎的晓得?有诗为证:

\[
月落花阴夜漏长,相逢疑是梦高唐。
夜深偷把银缸照,犹恐憨奴瞰隙光。
\]

却说西门庆扒过墙来,走到潘金莲房里。金莲还睡未起,因问:“你昨日也不知又往那里去了这一夜?也不对奴说一声儿。”西门庆道:“花二哥又使小厮邀我往院里去,吃了半夜酒,才脱身走来家。”金莲虽故信了,还有几分疑影在心。一日,同孟玉楼饭后在花园亭子上做针指,猛可见一块瓦儿打在面前。那孟玉楼低着头纳鞋,没看见。这潘金莲单单把眼四下观看,影影绰绰只见隔壁墙头上一个白面探了一探,就下去了。金莲忙推玉楼,指与他瞧,说道:“三姐姐,你看这个,是隔壁花家那大丫头,想是上墙瞧花儿,看见俺们在这里,他就下去了。”说毕,也就罢了。到晚夕,西门庆自外赴席来家,进金莲房中。金莲与他接了衣裳,问他。饭不吃,茶也不吃,趔趄着脚儿,只往前边花园里走。这潘金莲贼留心,暗暗看着他。坐了好一回,只见先头那丫头在墙头上打了个照面,这西门庆就踏着梯凳过墙去了。那边李瓶儿接入房中,两个厮会不题。

这潘金莲归到房中,翻来复去,通一夜不曾睡。将到天明,只见西门庆过来,推开房门,妇人睡在床上,不理他。那西门庆先带几分愧色,挨近他床上坐下。妇人见他来,跳起来坐着,一手撮着他耳朵,骂道:“好负心的贼!你昨日端的那里去来?把老娘气了一夜!你原来干的那茧儿,我已是晓得不耐烦了!趁早实说,从前已往,与隔壁花家那淫妇偷了几遭?一一说出来,我便罢休。但瞒着一字儿,到明日你前脚儿过去,后脚我就吆喝起来,教你负心的囚根子死无葬身之地!你安下人标住他汉子在院里过夜,却这里要他老婆。我教你吃不了包着走!嗔道昨日大白日里,我和孟三姐在花园里做生活,只见他家那大丫头在墙那边探头舒脑的,原来是那淫妇使的勾使鬼来勾你来了。你还哄我老娘!前日他家那忘八,半夜叫了你往院里去,原来他家就是院里!”西门庆听了,慌的装矮子,只跌脚跪在地下,笑嘻嘻央及说道:“怪小油嘴儿,禁声些!实不瞒你,他如此这般问了你两个的年纪,到明日讨了鞋样去,每人替你做双鞋儿,要拜认你两个做姐姐,他情愿做妹子。”金莲道:“我是不要那淫妇认甚哥哥姐姐的。他要了人家汉子,又来献小殷勤儿,我老娘眼里是放不下砂子的人,肯叫你在我跟前弄了鬼儿去!”说着一只手把他裤子扯开,只见那话软仃当,银托子还带在上面,问道:“你实说,与淫妇弄了几遭?”西门庆道:“弄到有数儿的,只一遭。”妇人道:“你赌个誓,一遭就弄的他恁软如鼻涕浓如酱,却如风瘫了一般的!有些硬朗气儿也是人心。”说着把托子一揪,挂下来,骂道:“没羞的强盗,嗔道教我那里没寻,原来把这行货子悄地带出,和那淫妇肏捣去了。”西门庆满脸儿陪笑说道:“怪小淫妇儿,麻犯人死了,他再三教我捎了上覆来,他到明日过来与你磕头,还要替你做鞋。昨日使丫头替了吴家的样子去了。今日教我捎了这一对寿字簪儿送你。”于是除了帽子,向头上拔将下来,递与金莲。金莲接在手内观看,却是两根番石青填地、金玲珑寿字簪儿,乃御前所制,宫里出来的,甚是奇巧。金莲满心欢喜,说道:“既是如此,我不言语便了。等你过那边去,我这里与你两个观风,教你两个自在肏捣。你心下如何?”那西门庆欢喜的双手搂抱着说道:“我的乖乖的儿,正是如此。不枉的养儿不在屙金溺银,只要见景生情。我到明日梯己买一套妆花衣服谢你。”妇人道:“我不信那蜜嘴糖舌,既要老娘替你二人周旋,要依我三件事。”西门庆道:“不拘几件,我都依。”妇人道:“头一件不许你往院里去;第二件要依我说话;第三件你过去和他睡了,来家就要告我说,一字不许你瞒我。”西门庆道:“这个不打紧,都依你便了。”

自此为始,西门庆过去睡了来,就告妇人说:“李瓶儿怎的生得白净,身软如绵花,好风月,又善饮。俺两个帐子里放着果盒,看牌饮酒,常玩耍半夜不睡。”又向袖中取出一个物件儿来,递与金莲瞧,道:“此是他老公公内府画出来的,俺两个点着灯,看着上面行事。”金莲接在手中,展开观看。有词为证:

\[
内府衢花绫裱,牙签锦带妆成。大青小绿细描金,镶嵌斗方干净。女赛巫山神女,男如宋玉郎君,双双帐内惯交锋。解名二十四,春意动关情。
\]
金莲从前至尾看了一遍,不肯放手,就交与春梅道:“好生收在我箱子内,早晚看着耍子。”西门庆道:“你看两日,还交与我。此是人的爱物儿,我借了他来家瞧瞧,还与他。”金莲道:“他的东西,如何到我家?我又不曾从他手里要将来。就是打也打不出去。”西门庆道:“怪小奴才儿,休要耍问。”赶着夺那手卷。金莲道:“你若夺一夺儿,赌个手段,我就把他扯得稀烂,大家看不成。”西门庆笑道:“我也没法了,随你看完了与他罢么。你还了他这个去,他还有个稀奇物件儿哩,到明日我要了来与你。”金莲道:“我儿,谁养得你恁乖?你拿了来,我方与你这手卷去。”两个絮聒了一回。晚夕,金莲在房中香薰鸳被,款设银灯,艳妆澡牝,与西门庆展开手卷,在锦帐之中效“于飞”之乐。看观听说:巫蛊魇昧之物,自古有之。金莲自从叫刘瞎子回背之后,不上几时,使西门庆变嗔怒而为宠爱,化忧辱而为欢娱,再不敢制他。正是:

\[
饶你奸似鬼,也吃洗脚水。
\]
有词为证:
\[
记得书斋乍会时,云踪雨迹少人知。晓来鸾凤栖双枕,剔尽银灯半吐辉。思往事,梦魂迷,今宵喜得效于飞。颠鸾倒凤无穷乐,从此双双永不离。
\]
