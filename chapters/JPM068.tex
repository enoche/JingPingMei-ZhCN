%# -*- coding:utf-8 -*-
%%%%%%%%%%%%%%%%%%%%%%%%%%%%%%%%%%%%%%%%%%%%%%%%%%%%%%%%%%%%%%%%%%%%%%%%%%%%%%%%%%%%%


\chapter{应伯爵戏衔玉臂\KG 玳安儿密访蜂媒}


词曰:

\[
钟情太甚,到老也无休歇。月露烟云都是态,况与玉人明说。软语叮咛,柔情婉恋,熔尽肝肠铁。岐亭把盏,水流花谢时节。
\]

话说西门庆与李瓶儿烧纸毕,归潘金莲房中歇了一夜。到次日,先是应伯爵家送喜面来。落后黄四领他小舅子孙文相,宰了一口猪、一坛酒、两只烧鹅、四只烧鸡、两盒果子来与西门庆磕头。西门庆再三不受,黄四打旋磨儿跪着说:“蒙老爹活命之恩,举家感激不浅。无甚孝顺,些微薄礼,与老爹赏人,如何不受!”推阻了半日,西门庆止受猪酒:“留下送你钱老爹罢。”黄四道:“既是如此,难为小人一点穷心,无处所尽。”只得把羹果抬回去。又请问:“老爹几时闲暇?小人问了应二叔,里边请老爹坐坐。”西门庆道:“你休听他哄你哩!又费烦你,不如不央我了。”那黄四和他小舅子千恩万谢出门去了。

到十一月初一日,西门庆往衙门中回来,又往李知县衙内吃酒去,月娘独自一人,素妆打扮,坐轿子往乔大户家与长姐做生日,都不在家。到后晌,有庵里薛姑子,听见月娘许下他初五日念经拜《血盆忏》,于是悄悄瞒着王姑子,买了两盒礼物来见月娘。月娘不在家,李娇儿、孟玉楼留他吃茶,说:“大姐姐往乔亲家做生日去了。你须等他来,他还和你说话哩。”那薛姑子就坐住了。潘金莲思想着玉箫告他说,月娘吃了他的符水药才坐了胎气,又见西门庆把奶子要了,恐怕一时奶子养出孩子来,搀夺了他宠爱。于是把薛姑子让到前边他房里,悄悄央薛姑子,与他一两银子,替他配坐胎气符药,不在话下。

到晚夕,等的月娘回家,留他住了一夜。次日,问西门庆讨了五两银子经钱写法与他。这薛姑子就瞒着王姑子、大师父,到初五日早请了八众女僧,在花园卷棚内建立道场,讽诵《华严》、《金刚》经咒,礼拜《血盆》宝忏。晚夕设放焰口施食。那日请了吴大妗子、花大嫂并官客吴大舅、应伯爵、温秀才吃斋。尼僧也不动响器,只敲木鱼,击手馨,念经而已。

那日伯爵领了黄四家人,具帖初七日在院中郑爱月儿家置酒请西门庆。西门庆看了帖儿,笑道:“我初七日不得闲,张西村家吃生日酒。倒是明日空闲。”问还有谁,伯爵道:“再没人。只请了我与李三相陪哥,又叫了四个女儿唱《西厢记》。”西门庆吩咐与黄四家人斋吃了,打发回去,改了初六。伯爵便问:“黄四那日买了分甚么礼来谢你?”西门庆如此这般:“我不受他的,再三磕头礼拜,我只受了猪酒。添了两匹白鹇紵丝、两匹京缎、五十两银子,谢了龙野钱公了。”伯爵道:“哥,你不接钱尽够了,这个是他落得的。少说四匹尺头值三十两银子,那二十两,那里寻这分上去?便益了他,救了他父子二人性命!”当日坐至晚夕方散。西门庆向伯爵说:“你明日还到这边。”伯爵说:“我知道。”作别去了。八众尼僧直乱到一更多,方才道场圆满,焚烧箱库散了。

至次日,西门庆早往衙门中去了。且说王姑子打听得知,大清早晨走来,说薛姑子揽了经去,要经钱。月娘怪他道:“你怎的昨日不来?他说你往王皇亲家做生日去了。”王姑子道:“这个就是薛家老淫妇的鬼。他对着我说咱家挪了日子,到初六念经。难道经钱他都拿的去了,一些儿不留下?”月娘道:“还等到这咱哩?未曾念经,经钱写法就都找与他了。早是我还与你留下一匹衬钱布在此。”教小玉连忙摆了些昨日剩下的斋食与他吃了,把与他一匹蓝布。这王姑子口里喃喃呐呐骂道:“这老淫妇,他印造经,赚了六娘许多银子。原说这个经儿,咱两个使,你又独自掉揽的去了。”月娘道:“老薛说你接了六娘《血盆经》五两银子,你怎的不替他念?”王姑子道:“他老人家五七时,我在家请了四位师父,念了半个月哩。”月娘道:“你念了,怎的挂口儿不对我题?你就对我说,我还送些衬施儿与你。”那王姑子便一声儿不言语,讪讪的坐了一回,往薛姑子家嚷去了。正是:

\[
佛会僧尼是一家,法轮常转度龙华。
此物只好图生育,枉使金刀剪落花。
\]

却说西门庆从衙门中回来,吃了饭,应伯爵又早到了。盔的新缎帽,沉香色\textuni{2773D}褶,粉底皂靴,向西门庆声喏,说:“这天也有晌午,好去了。他那里使人邀了好几遍了。”西门庆道:“咱今邀葵轩同走走去。”使王经:“往对过请你温师父来。”王经去不多时,回说:“温师父不在家,望朋友去了。”伯爵便说:“咱等不的他。秀才家有要没紧望朋友,知多咱来?倒没的误了勾当。”西门庆吩咐琴童:“备黄马与应二爹骑。”伯爵道:“我不骑。你依我:省的摇铃打鼓,我先走一步儿,你坐轿子慢慢来就是了。”西门庆道:“你说的是,你先行罢。”那伯爵举手先走了。

西门庆吩咐玳安、琴童、四个排军,收拾下暖轿跟随。才待出门,忽平安儿慌慌张张从外拿着双帖儿来报,说:“工部安老爹来拜。先差了个吏送帖儿,后边轿子便来也。”慌的西门庆吩咐家中厨下备饭,使来兴儿买攒盘点心伺候。良久,安郎中来到,西门庆冠冕出迎。安郎中穿着妆花云鹭补子员领,起花萌金带,进门拜毕,分宾主坐定,左右拿茶上来。茶罢,叙其间阔之情。西门庆道:“老先生荣擢,失贺,心甚缺然。前日蒙赐华扎厚仪,生正值丧事,匆匆未及奉候起居为歉。”安郎中道:“学生有失吊问,罪罪!生到京也曾道达云峰,未知可有礼到否?”西门庆道:“正是,又承翟亲家远劳致赙。”安郎中道:“四泉一定今岁恭喜。”西门庆道,“在下才微任小,岂敢非望。”又说:“老先生荣擢美差,足展雄才。治河之功,天下所仰。”安郎中道:“蒙四泉过誉。一介寒儒,辱蔡老先生抬举,谬典水利,修理河道,当此民穷财尽之时。前者皇船载运花石,毁闸折坝,所过倒悬,公私困弊之极。又兼贼盗梗阻,虽有神输鬼役之才,亦无如之何矣。”西门庆道:“老先生大才展布,不日就绪,必大升擢矣。”因问:“老先生敕书上有期限否?”安郎中道:“三年钦限。河工完毕,圣上还要差官来祭谢河神。”说话中间,西门庆令放桌儿,安郎中道:“学生实说,还要往黄泰宇那里拜拜去。”西门庆道:“既如此,少坐片时,教从者吃些点心。”不一时,就是春盛案酒,一色十六碗下饭,金钟暖酒斟来,下人俱有攒盘点心酒肉。安郎中席间只吃了三钟,就告辞起身,说:“学生容日再来请教。”西门庆款留不住,送至大门首,上轿而去。回到厅上,解去冠带,换了巾帻,止穿紫绒狮补直身。使人问:“温师父来了不曾?”玳安回说:“温师父尚未回哩。有郑春和黄四叔家来定儿来邀,在这里半日了。”

西门庆即出门上轿,左右跟随,迳往郑爱月儿家来。比及进院门,架儿们都躲过一边,只该日俳长两边站立,不敢跪接。郑春与来定儿先通报去了。应伯爵正和李三打双陆,听见西门庆来,连忙收拾不及。郑爱月儿、爱香儿戴着海獭卧兔儿,一窝丝杭州攒,打扮的花仙也似,都出来门首迎接。西门庆下了轿,进入客位内。西门庆吩咐不消吹打,止住鼓乐。先是李三、黄四见毕礼数,然后郑家鸨子出来拜见了。才是爱月儿姊妹两个磕头。正面安放两张交椅,西门庆与应伯爵坐下,李智、黄四与郑家姊妹打横。玳安在旁禀问:“轿子在这里,回了家去?”西门庆令排军和轿子都回去,又吩咐琴童:“到家看你温师父来了,拿黄马接了来。”琴童应喏去了。伯爵因问:“哥怎的这半日才来?”西门庆悉把安郎中来拜留饭之事说了一遍。

须臾,郑春拿上茶来,爱香儿拿了一盏递与伯爵。爱月儿便递西门庆,那伯爵连忙用手去接,说:“我错接,只说你递与我来。”爱月儿道:“我递与你?——没修这样福来!”伯爵道:“你看这小淫妇儿,原来只认的他家汉子,倒把客人不着在意里。”爱月儿笑道:“今日轮不着你做客人哩!”吃毕茶,须臾四个唱《西厢》妓女都出来与西门庆磕头,一一问了姓名。西门庆对黄四说:“等住回上来唱,只打鼓儿,不吹打罢。”黄四道:“小人知道。”鸨子怕西门庆冷,又教郑春放下暖帘来,火盆内添上许多兽炭。只见几个青衣圆社听见西门庆在郑家吃酒,走来门首伺候,探头舒脑,不敢进去。有认得玳安的,向玳安打恭,央及作成作成。玳安悄俏进来替他禀问,被西门庆喝了一声,唬的众人一溜烟走了。不一时,收拾果品案酒上来,正面放两张桌席:西门庆独自一席,伯爵与温秀才一席——留下温秀才座位在左首。旁边一席李三和黄四,右边是他姊妹二人。端的肴堆异品,花插金瓶。郑奉、郑春在旁弹唱。

才递酒安席坐下,只见温秀才到了。头戴过桥巾,身穿绿云袄,进门作揖。伯爵道:“老先生何来迟也?留席久矣。”温秀才道:“学生有罪,不知老先生呼唤,适往敝同窗处会书,来迟了一步。”慌的黄四一面安放钟箸,与伯爵一处坐下。不一时,汤饭上来,两个小优儿弹唱一回下去。四个妓女才上来唱了一折“游艺中原”,只见玳安来说:“后边银姨那里使了吴惠和蜡梅送茶来了。”原来吴银儿就在郑家后边住,止隔一条巷。听见西门庆在这里吃酒,故使送茶。西门庆唤入里面,吴惠、蜡梅磕了头,说:“银姐使我送茶来爹吃。”揭开盒儿,斟茶上去,每人一盏瓜仁香茶。西门庆道:“银姐在家做甚么哩?”蜡梅道:“姐儿今日在家没出门。”西门庆吃了茶,赏了他两个三钱银子,即令玳安同吴惠:“你快请银姨去。”郑爱月儿急俐,便就教郑春:“你也跟了去,好歹缠了银姨来。他若不来,你就说我到明日就不和他做伙计了。”应伯爵道:“我倒好笑,你两个原来是贩\textuni{23B48}的伙计。”温秀才道:“南老好不近人情。自古同声相应,同气相求。本乎天者亲上,本乎地者亲下。同他做伙计亦是理之当然。”爱月儿道:“应花子,你与郑春他们都是伙计,当差供唱都在一处。”伯爵道:“傻孩子,我是老王八!那咱和你妈相交,你还在肚子里!”说笑中间,妓女又上来唱了一套“半万贼兵”。西门庆叫上唱莺莺的韩家女儿近前,问:“你是韩家谁的女儿?”爱香儿说:“爹,你不认的?他是韩金钏侄女儿,小名消愁儿,今年才十三岁。”西门庆道:“这孩子到明日成个好妇人儿。举止伶俐,又唱的好。”因令他上席递酒。黄四下汤下饭,极尽殷勤。

不一时,吴银儿来到。头上戴着白绉纱\textuni{4BFC}髻、珠子箍儿、翠云钿儿,周围撇一溜小簪儿。上穿白绫对衿袄儿,妆花眉子,下着纱绿潞绸裙,羊皮金滚边。脚上墨青素缎鞋儿。笑嘻嘻进门,向西门庆磕了头,后与温秀才等各位都道了万福。伯爵道:“我倒好笑,来到就教我惹气。俺每是后娘养的?只认的你爹,与他磕头,望着俺每只一拜。原来你这丽春院小娘儿这等欺客!我若有五棍儿衙门,定不饶你。”爱月儿叫:“应花子,好没羞的孩儿。你行头不怎么,光一味好撇。”一面安座儿,让银姐就在西门庆桌边坐下。西门庆见他戴着白\textuni{4BFC}髻,问:“你戴的谁人孝?”吴银儿道:“爹故意又问个儿,与娘戴孝一向了。”西门庆一闻与李瓶儿戴孝,不觉满心欢喜,与他侧席而坐,两个说话。

须臾汤饭上来,爱月儿下来与他递酒。吴银儿下席说:“我还没见郑妈哩。”一面走到鸨子房内见了礼,出来,鸨子叫:“月姐,让银姐坐。只怕冷,教丫头烧个火笼来,与银姐烤手儿。”随即添换热菜上来,吴银儿在旁只吃了半个点心,喝了两口汤。放下箸儿,和西门庆攀话道:“娘前日断七念经来?”西门庆道:“五七多谢你每茶。”吴银儿道:“那日俺每送了些粗茶,倒教爹把人情回了,又多谢重礼,教妈惶恐的要不的。昨日娘断七,我会下月姐和桂姐,也要送茶来,又不知宅内念经不念。”西门庆道:“断七那日,胡乱请了几位女僧,在家拜了拜忏。亲眷一个都没请,恐怕费烦。”饮酒说话之间,吴银儿又问:“家中大娘众娘每都好?”西门庆道:“都好。”吴银儿道:“爹乍没了娘,到房里孤孤儿的,心中也想么?”西门庆道:“想是不消说。前日在书房中,白日梦见他,哭的我要不的。”吴银儿道:“热突突没了,可知想哩!”伯爵道:“你每说的知情话,把俺每只顾旱着,不说来递钟酒,也唱个儿与俺听。俺每起身去罢!”慌的李三、黄四连忙撺掇他姐儿两个上来递酒。安下乐器,吴银儿也上来。三个粉头一般儿坐在席上,躧着火盆,合着声儿唱了套《中吕·粉蝶儿》“三弄梅花”,端的有裂石流云之响。

唱毕,西门庆向伯爵说:“你索落他姐儿三个唱,你也下来酬他一杯儿。”伯爵道:“不打紧,死不了人。等我打发他:仰靠着,直舒着,侧卧着,金鸡独立,随我受用;又一件,野马踩场,野狐抽丝,猿猴献果,黄狗溺尿,仙人指路,——哥,随他拣着要。”爱香道:“我不好骂出来的,汗邪了你这贼花子,胡说乱道的。”应伯爵用酒碟安三个钟儿,说:“我儿,你每在我手里吃两钟。不吃,望身上只一泼。”爱香道:“我今日忌酒。”爱月儿道:“你跪着月姨,教我打个嘴巴儿,我才吃。”伯爵道:“银姐,你怎的说?”吴银儿道:“二爹,我今日心里不自在,吃半盏儿罢。”爱月儿道:“花子,你不跪,我一百年也不吃。”黄四道:“二叔,你不跪,显的不是趣人。也罢,跪着不打罢。”爱月儿道:“跪了也不打多,只教我打两个嘴巴儿罢。”伯爵道:“温老先儿,你看着,怪小淫妇儿只顾赶尽杀绝。”于是奈何不过,真个直撅儿跪在地下。那爱月儿轻揎彩袖,款露春纤,骂道:“贼花子,再可敢无礼伤犯月姨了?——高声儿答应。你不答应,我也不吃。”伯爵无法可处,只得应声道:“再不敢伤犯月姨了。”这爱月儿方连打了两个嘴巴,方才吃那钟酒。伯爵起来道:“好个没仁义的小淫妇儿,你也剩一口儿我吃。把一钟酒都吃的净净儿的。”爱月儿道:“你跪下,等我赏你一钟吃。”于是满满斟上一杯,笑望伯爵口里只一灌。伯爵道,“怪小淫妇儿,使促狭灌撒了我一身。我老实说,只这件衣服,新穿了才头一日儿,就污浊了我的。我问你家汉子要。”笑了一回,各归席上坐定。

看看天晚,掌烛上来。西门庆吩咐取个骰盆来。先让温秀才,秀才道:“岂有此理!还从老先生来。”于是西门庆与银儿用十二个骰儿抢红,下边四个妓女拿着乐器弹唱。饮过一巡,吴银儿却转过来与温秀才、伯爵抢红,爱香儿却来西门庆席上递酒猜枚。须臾过去,爱月儿近前与西门庆抢红,吴银儿却往下席递李三、黄四酒。原来爱月几旋往房中新妆打扮出来,上着烟里火回纹锦对衿袄儿、鹅黄杭绢点翠缕金裙、妆花膝裤、大红凤嘴鞋儿,灯下海獭卧兔儿,越显的粉浓浓雪白的脸儿。真是:

\[
芳姿丽质更妖烧,秋水精神瑞雪标。
白玉生香花解语,千金良夜实难消。
\]
西门庆见了,如何不爱。吃了几钟酒,半酣上来,因想着李瓶儿梦中之言:少贪在外夜饮。一面起身后边净手。慌的鸨子连忙叫丫鬟点灯,引到后边。解手出来,爱月随即跟来伺候。盆中净手毕,拉着他手儿同到房中。

房中又早月窗半启,银烛高烧,气暖如春,兰麝馥郁,于是脱了上盖,止穿白绫道袍,两个在床上腿压腿儿做一处。先是爱月儿问:“爹今日不家去罢了。”西门庆道:“我还去。今日一者银儿在这里,不好意思;二者我居着官,今年考察在迩,恐惹是非,只是白日来和你坐坐罢了。”又说:“前日多谢你泡螺儿。你送了去,倒惹的我心酸了半日。当初止有过世六娘他会拣。他死了,家中再有谁会拣他!”爱月道:“拣他不难,只是要拿的着禁节儿便好。那瓜仁都是我口里一个个儿嗑的,说应花子倒挝了好些吃了。”西门庆道:“你问那讪脸花子,两把挝去喃了好些。只剩下没多,我吃了。”爱月儿道:“倒便益了贼花子,恰好只孝顺了他。”又说:“多谢爹的衣梅。妈看见吃了一个儿,欢喜的要不的。他要便痰火发了,晚夕咳嗽半夜,把人聒死了。常时口干,得恁一个在口里噙着他,倒生好些津液。我和俺姐姐吃了没多几个儿,连罐儿他老人家都收在房内早晚吃,谁敢动他!”西门庆道:“不打紧,我明日使小厮再送一罐来你吃。”爱月又问:“爹连日会桂姐没有?”西门庆道:“自从孝堂内到如今,谁见他来?”爱月儿道:“六娘五七,他也送茶去来?”西门庆道:“他家使李铭送去来。”爱月道:“我有句话儿,只放在爹心里。”西门庆问:“甚么话?”那爱月又想了想说:“我不说罢。若说了,显的姐妹每恰似我背地说他一般,不好意思的。”西门庆一面搂着他脖子说道:“怪小油嘴儿,甚么话?说与我,不显出你来就是了。”

两个正说得入港,猛然应伯爵入来大叫一声:“你两个好人儿,撇了俺每走在这里说梯己话儿!”爱月儿道:“哕,好个不得人意怪讪脸花子!猛可走来,唬了人恁一跳!”西门庆骂:“怪狗才,前边去罢。丢的葵轩和银姐在那里,都往后头来了。”这伯爵一屁股坐在床上,说:“你拿胳膊来,我且咬口儿,我才去。你两个在这里尽着\textuni{34B2}捣!”于是不由分说,向爱月儿袖口边勒出那赛鹅脂雪白的手腕儿来,夸道:“我儿,你这两只手儿,天生下就是发\textuni{23B20}\textuni{23B36}的行货子。”爱月儿道:“怪攮刀子的,我不好骂出来!”被伯爵拉过来,咬了一口走了。咬得老婆怪叫,骂:“怪花子,平白进来鬼混人死了!”便叫桃花儿:“你看他出去了,把弄道子门关上。”爱月便把李桂姐如今又和王三官儿好一节说与西门庆:“怎的有孙寡嘴、祝麻子、小张闲,架儿于宽、聂钺儿,踢行头白回子、向三,日逐标着在他家行走。如今丢开齐香儿,又和秦家玉芝儿打热,两下里使钱。使没了,将皮袄当了三十两银子,拿着他娘子儿一副金镯子放在李桂姐家,算了一个月歇钱。”西门庆听了,口中骂道:“这小淫妇儿,我恁吩咐休和这小厮缠,他不听,还对着我赌身发咒,恰好只哄着我。”爱月儿道:“爹也没要恼。我说与爹个门路儿,管情教王三官打了嘴,替爹出气。”西门庆把他搂在怀里说道:“我的儿,有甚门路儿,说与我知道。”爱月儿道:“我说与爹,休教一人知道。就是应花子也休对他题,只怕走了风。”西门庆道:“你告我说,我傻了,肯教人知道!”郑爱月道:“王三官娘林太太,今年不上四十岁,生的好不乔样!描眉画眼,打扮的狐狸也似。他儿子镇日在院里,他专在家,只寻外遇。假托在姑姑庵里打斋,但去,就在说媒的文嫂儿家落脚。文嫂儿单管与他做牵头,只说好风月。我说与爹,到明日遇他遇儿也不难。又一个巧宗儿:王三官娘子儿今才十九岁,是东京六黄太尉侄女儿,上画般标致,双陆、棋子都会。三官常不在家,他如同守寡一般,好不气生气死。为他也上了两三遭吊,救下来了。爹难得先刮剌上了他娘,不愁媳妇儿不是你的。”当下,被他一席话儿说的西门庆心邪意乱,搂着粉头说:“我的亲亲,你怎的晓的就里?”爱月儿就不说常在他家唱,只说:“我一个熟人儿,如此这般和他娘在某处会过一面,也是文嫂儿说合。”西门庆问:“那人是谁?莫不是大街坊张大户侄儿张二官儿?”爱月儿道:“那张懋德儿,好\textuni{34B2}的货,麻着个脸蛋子,密缝两个眼,可不砢硶杀我罢了!只好蒋家百家奴儿接他。”西门庆道:“我猜不着,端的是谁?”爱月儿道:“教爹得知了罢:原是梳笼我的一个南人。他一年来此做买卖两遭,正经他在里边歇不的一两夜,倒只在外边常和人家偷猫递狗,干此勾当。”西门庆听了,见粉头所事,合着他的板眼,亦发欢喜,说:“我儿,你既贴恋我心,我每月送三十两银子与你妈盘缠,也不消接人了。我遇闲就来。”爱月儿道:“爹,你若有我心时,甚么三十两二十两,随着掠几两银子与妈,我自恁懒待留人,只是伺候爹罢了。”西门庆道:“甚么话!我决然送三十两银子来。”说毕,两个上床交欢。床上铺的被褥约一尺高,爱月道:“爹脱衣裳不脱?”西门庆道:“咱连衣耍耍罢,只怕他们前边等咱。“一面扯过枕头来,粉头解去下衣,仰卧枕畔,西门庆把他两只小小金莲扛在肩上,解开蓝绫裤子,那话使上托子。但见花心轻折,柳腰款摆。正是:

\[
花嫩不禁柔,春风卒未休。
花心犹未足,脉脉情无极。
低低唤粉郎,春宵乐未央。
\]

两个交欢良久,至精欲泄之际,西门庆干的气喘吁吁,粉头娇声不绝,鬓云拖枕,满口只教:“亲达达,慢着些儿!”少顷,乐极情浓,一泄如注。云收雨散,各整衣理容,净了手,同携手来到席上。

吴银儿和爱香儿正与葵轩、伯爵掷色猜枚,觥筹交错,耍在热闹处。众人见西门庆进入,俱立起身来让坐。伯爵道:“你也下般的,把俺每丢在这里,你才出来,拿酒儿且扶扶头着。”西门庆道:“俺每说句话儿,有甚闲勾当!”伯爵道:“好话,你两个原来说梯己话儿。”当下伯爵拿大钟斟上暖酒,众人陪西门庆吃。四个妓女拿乐器弹唱。玳安在旁说道:“轿子来了。”西门庆呶了个嘴儿与他,那玳安连忙吩咐排军打起灯笼,外边伺候。西门庆也不坐,陪众人执杯立饮。吩咐四个妓女:“你再唱个‘一见娇羞’我听。”那韩消愁儿拿起琵琶来,款放娇声,拿腔唱道:

\[
一见娇羞,雨意云情两意投。我见他千娇百媚,万种妖娆,一捻温柔。通书先把话儿勾,传情暗里秋波溜。记在心头。心头,未审何时成就。
\]
唱了一个,吴银儿递西门庆酒,郑香儿便递伯爵,爱月儿奉温秀才,李智、黄四都斟上。四妓女又唱了一个。吃毕,众人又彼此交换递了两转,妓女又唱了两个。

唱毕,都饮过,西门庆就起身。一面令玳安向书袋内取出大小十一包赏赐来:四个妓女每人三钱,厨役赏了五钱,吴惠、郑春、郑奉每人三钱,撺掇打茶的每人二钱,丫头桃花儿也与了他三钱。俱磕头谢了。黄四再三不肯放,道:“应二叔,你老人家说声,天还早哩。老爹大坐坐,也尽小人之情,如何就要起身?我的月姨,你也留留儿。”爱月儿道:“我留他,他白不肯坐。”西门庆道:“你每不知,我明日还有事。”一面向黄四作揖道:“生受打搅!”黄四道:“惶恐!没的请老爹来受饿,又不肯久坐,还是小人没敬心。”说着,三个唱的都磕头说道:“爹到家多顶上大娘和众娘们,俺每闲了,会了银姐往宅内看看大娘去。”西门庆道:“你每闲了去坐上一日来。”一面掌起灯笼,西门庆下台矶,郑家鸨子迎着道万福,说道:“老爹大坐回儿,慌的就起身,嫌俺家东西不美口?还有一道米饭儿未曾上哩!”西门庆道:“够了。我明日还要起早,衙门中有勾当。应二哥他没事,教他大坐回儿罢。”那伯爵就要跟着起来,被黄四使力拦住,说道:“我的二爷,你若去了,就没趣死了。”伯爵道:“不是,你休拦我。你把温老先生有本事留下,我就算你好汉。”那温秀才夺门就走,被黄家小厮来定儿拦腰抱住。西门庆到了大门首,因问琴童儿:“温师父有头口在这里没有?”琴童道:“备了驴子在此,画童儿看着哩。”西门庆向温秀才道:“既有头口,也罢,老先儿你再陪应二哥坐坐,我先去罢。”于是,都送出门来。那郑月儿拉着西门庆手儿悄悄捏了一把,说道:“我说的话,爹你在心些,法不传六耳。”西门庆道:“知道了。”爱月又叫郑春:“你送老爹到家。”西门庆才上轿去了。吴银儿就在门首作辞了众人并郑家姐儿两个,吴惠打着灯回家去了。郑月儿便叫:“银姐,见了那个流人儿,好歹休要说。”吴银儿道:“我知道。”众人回至席上,重添兽炭,再泛流霞,歌舞吹弹,欢娱乐饮,直耍了三更方散。黄四摆了这席酒,也与了他十两银子,不在话下。当日西门庆坐轿子,两个排军打着灯,迳出院门,打发郑春回家。

一宿晚景题过。到次日,夏提刑差答应的来请西门庆早往衙门中审问贼情等事,直问到晌午来家。吃了饭,早是沈姨夫差大官沈定,拿帖儿送了个后生来,在缎子铺煮饭做火头,名唤刘包。西门庆留下了,正在书房中,拿帖儿与沈定回家去了。只见玳安在旁边站立,西门庆便问道:“温师父昨日多咱来的?”玳安道:“小的铺子里睡了好一回,只听见画童儿打对过门,那咱有三更时分才来了。今早问,温师父倒没酒;应二爹醉了,唾了一地,月姨恐怕夜深了,使郑春送了他家去了。”西门庆听了,哈哈笑了,因叫过玳安近前,说道:“旧时与你姐夫说媒的文嫂儿在那里住?你寻了他来,对门房子里见我。我和他说话。”玳安道:“小的不认的文嫂儿家,等我问了姐夫去。”西门庆道:“你问了他快去。”

玳安走到铺子里问陈敬济,敬济道:“问他做甚么?”玳安道:“谁知他做甚么,猛可教我抓寻他去。”敬济道:“出了东大街一直往南去,过了同仁桥牌坊转过往东,打王家巷进去,半中腰里有个发放巡捕的厅儿,对门有个石桥儿,转过石桥儿,紧靠着个姑姑庵儿,旁边有个小胡同儿,进小胡同往西走,第三家豆腐铺隔壁上坡儿,有双扇红对门儿的就是他家。你只叫文妈,他就出来答应你。”玳安听了说道:“再没有?小炉匠跟着行香的走——琐碎一浪荡。你再说一遍我听,只怕我忘了。”那陈敬济又说了一遍,玳安道:“好近路儿!等我骑了马去。”一面牵出大白马来骑上,打了一鞭,那马跑踍跳跃,一直去了。出了东大街迳往南,过同仁桥牌坊,由王家巷进去,果然中间有个巡捕厅儿,对门亦是座破石桥儿,里首半截红墙是大悲庵儿,往西小胡同上坡,挑着个豆腐牌儿,门首只见一个妈妈晒马粪。玳安在马上就问:“老妈妈,这里有个说媒的文嫂儿?”那妈妈道:“这隔壁对门儿就是。”

玳安到他门首,果然是两扇红对门儿,连忙跳下马来,拿鞭儿敲着门叫道:“文嫂在家不在?”只见他儿子文\textSiTang 开了门,问道:“是那里来的?”玳安道:“我是县门前提刑西门老爹家,来请,教文妈快去哩。”文\textSiTang 听见是提刑西门大官府里来的,便让家里坐。那玳安把马拴住,进入里面。见上面供养着利市纸,有几个人在那里算进香帐哩。半日拿了钟茶出来,说道:“俺妈不在了。来家说了,明日早去罢。”玳安道:“驴子见在家里,如何推不在?”侧身迳往后走。不料文嫂和他媳妇儿,陪着几个道妈妈子正吃茶,躲不及,被他看见了,说道:“这个不是文妈?就回我不在家!”文嫂笑哈哈与玳安道了个万福,说道:“累哥哥到家回声,我今日家里会茶。不知老爹呼唤我做甚么,我明日早去罢。”玳安道:“只分忖我来寻你,谁知他做甚么。原来你在这咭溜搭剌儿里住,教我抓寻了个小发昏。”文嫂儿道:“他老人家这几年买使女,说媒,用花儿,自有老冯和薛嫂儿、王妈妈子走跳,稀罕俺每!今日忽剌八又冷锅中豆儿爆,我猜着你六娘没了,一定教我去替他打听亲事,要补你六娘的窝儿。”玳安道:“我不知道。你到那里,俺爹自有话和你说。”文嫂儿道:“既如此,哥哥你略坐坐儿,等我打发会茶人去了,同你去罢。”玳安道:“俺爹在家紧等的火里火发,吩咐了又吩咐,教你快去哩。和你说了话,还要往府里罗同知老爹家吃酒去哩。”文嫂道:“也罢,等我拿点心你吃了,同你去。”玳安道:“不吃罢。”文嫂因问:“你大娘生了孩儿没有?”玳安道:“还不曾见哩。”文嫂一面打发玳安吃了点心,穿上衣裳,说道:“你骑马先行一步儿,我慢慢走。”玳安道:“你老人家放着驴子,怎不备上骑?”文嫂儿道:“我那讨个驴子来?那驴子是隔壁豆腐铺里的,借俺院儿里喂喂儿,你就当我的。”玳安道:“记的你老人家骑着匹驴儿来,往那去了?”文嫂儿道:“这咱哩!那一年吊死人家丫头,打官司把旧房儿也卖了,且说驴子哩!”玳安道:“房子到不打紧,且留着那驴子和你早晚做伴儿也罢了。别的罢了,我见他常时落下来好个大鞭子。”文嫂哈哈笑道:“怪猴子,短寿命,老娘还只当好话儿,侧着耳朵听。几年不见,你也学的恁油嘴滑舌的。到明日,还教我寻亲事哩!”玳安道:“我的马走的快,你步行,赤道挨磨到多咱晚,不惹的爹说?你也上马,咱两个叠骑着罢。”文嫂儿道:“怪小短命儿,我又不是你影射的!街上人看着,怪剌剌的。”玳安道:“再不,你备豆腐铺里驴子骑了去,到那里等我打发他钱就是了。”文嫂儿道:“这还是话。”一面教文\textSiTang 将驴子备了,带上眼纱,骑上,玳安与他同行,迳往西门庆宅中来。正是:

\[
欲向深闺求艳质,全凭红叶是良媒。
\]

